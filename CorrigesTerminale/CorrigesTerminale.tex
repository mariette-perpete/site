\documentclass[10pt]{article}
\usepackage[T1]{fontenc}
\usepackage[utf8]{inputenc}
\usepackage{fourier}
\usepackage[scaled=0.875]{helvet}
\renewcommand{\ttdefault}{lmtt}
\usepackage{amsmath,amssymb,makeidx}
\usepackage[normalem]{ulem}
\usepackage{fancybox}
\usepackage{cancel}
\usepackage{stmaryrd}
\usepackage{ulem}
\usepackage{tabularx}
\usepackage{geometry}
\usepackage{enumerate}
\geometry{hmargin=1.5cm,vmargin=1.5cm}
\usepackage{dcolumn}
\usepackage{textcomp}
\usepackage{lscape}
\usepackage{eurosym}
%\newcommand{\euro}{\eurologo{}}
\usepackage[dvips]{color}
\usepackage[all]{xy}

\usepackage{tikz,tkz-tab}

\usepackage{systeme}
\usepackage{ upgreek }


\usepackage{pstricks,pst-plot,pst-text,pst-tree,pstricks-add}
\usepackage{colortbl}
\usepackage{diagbox}
\usepackage{fontawesome5}
\usepackage{pifont}
\usepackage{wasysym}


\usepackage{theorem}
\theorembodyfont{\upshape}
\newtheorem{exo}{Exercice}
%\newtheorem{exo}{Exercice}%[section]
\usepackage{hyperref}
\hypersetup{
    colorlinks=true,       % false: liens encadrés; true: liens colorés
    linkcolor=blue,          % couleur des liens (ou bordures) internes
}

%\setlength{\voffset}{-1,5cm}
\usepackage{fancyhdr} 
\usepackage{graphicx}
\usepackage[frenchb]{babel}
\usepackage[np]{numprint}
\usepackage{multicol}
\usepackage{xlop}
\usepackage{soul}
\usepackage{etoolbox}
\usepackage{multirow}
\usepackage{diagbox}

\usepackage{tcolorbox}

\usepackage{xcolor}
\usepackage{stackengine}
    \setstackEOL{\\}
    
    \usepackage{listings}
\lstset{numbers=left, stepnumber=1}

\makeatletter
\patchcmd{\ttlh@hang}{\parindent\z@}{\parindent\z@\leavevmode}{}{}
\patchcmd{\ttlh@hang}{\noindent}{}{}{}
\makeatother
\setlength{\columnseprule}{0.0pt}

\usepackage{listings}
\lstset{%
  language=Python,
  basicstyle   = \ttfamily,
  keywordstyle =    \color{magenta},
  keywordstyle = [2]\color{orange},
  commentstyle =    \color{gray}\itshape,
  stringstyle  =    \color{cyan},
  numbers      = none,
  frame        = single,
  framesep     = 2pt,
  aboveskip    = 1ex
}

\newcommand\setItemnumber[1]{\setcounter{enumi}{\numexpr#1-1\relax}}


\title{Mathématiques -- Terminale spécialité}

\date{Corrigés des exercices}
\begin{document}
\setlength\parindent{0mm}
\renewcommand \footrulewidth{.2pt}

\maketitle

\tableofcontents


\newpage

\section{Compléments sur la dérivation}


\begin{exo}

La fonction $f$ est définie sur l'intervalle $\left[-2;6\right]$ par

\[ f(x) = 0,5x^2-2x-4.\]

\medskip

Pour tout $x\in\mathbb{R}~:$
\[f'(x)=0,5\times 2x-2\times 1-0=x-2.\]

La dérivée est du premier degré, donc pour obtenir le tableau de signe, il faut résoudre une équation, puis regarder le signe de $a~:$
\begin{align*}x-2&=0\\
 x-\cancel{2}+\cancel{2}&=0+2\\
 x&=2.
 \end{align*}

$a=1$ (puisque $x-2$ signifie $\textcolor{red}{1}x-2$), $a$ est $\oplus$ donc le signe est de la forme \fbox{$-~\upphi~+$}

\medskip


On en déduit le tableau de signe de $f'$ et le tableau de variations de $f~:$


\medskip

\setlength{\columnseprule}{1pt}

\begin{multicols}{2}

\begin{center}
\begin{tikzpicture}[scale=0.8]
\tkzTabInit{$x$/1,$f'(x)$/1,$f(x)$/2}{$-2$,$2$,$6$}
\tkzTabLine{,-,z,+,}
\tkzTabVar{+/$2$,-/$-6$,+/$2$}
\end{tikzpicture}
\end{center}

\columnbreak

Pour compléter l'extrémité des flèches, on calcule~:

\begin{itemize}
\item[\textbullet] $f(-2)=0,5\times (-2)^2-2\times (-2)-4=2$
\item[\textbullet] $f(2)=0,5\times 2^2-2\times 2-4=-6$
\item[\textbullet] $f(6)=0,5\times 6^2-2\times 6-4=2$
\end{itemize}

\medskip

On peut aussi faire un tableau de valeurs à la calculatrice.


\end{multicols}

\medskip

\textbf{Remarque~:} La courbe représentative est une parabole, dont le sommet $S$ a pour coordonnées $(2;-6).$


\begin{center}
\psset{xunit=1.0cm,yunit=0.5cm,algebraic=true,dimen=middle,dotstyle=o,dotsize=5pt 0,linewidth=2.pt,arrowsize=3pt 2,arrowinset=0.25}
\begin{pspicture*}(-2.36,-6.24)(6.46,2.42)
\multips(0,-6)(0,1.0){9}{\psline[linestyle=dashed,linecap=1,dash=1.5pt 1.5pt,linewidth=0.4pt,linecolor=lightgray]{c-c}(-2.36,0)(6.46,0)}
\multips(-2,0)(1.0,0){9}{\psline[linestyle=dashed,linecap=1,dash=1.5pt 1.5pt,linewidth=0.4pt,linecolor=lightgray]{c-c}(0,-6.24)(0,2.42)}
\psaxes[labelFontSize=\scriptstyle,xAxis=true,yAxis=true,Dx=1.,Dy=1.,ticksize=-2pt 0,subticks=2]{->}(0,0)(-2.36,-6.24)(6.46,2.42)
\rput{0.}(2.,-6.){\psplot[linewidth=2.pt,linecolor=blue]{-4.}{4.}{x^2/2/1.}}
\psdots[dotstyle=*,linecolor=red](2.,-6.)
\rput[bl](2.08,-5.8){\red{$S$}}
\end{pspicture*}
\end{center}


\end{exo}

\begin{exo}

On considère un segment $\left[AB\right]$ de longueur 4 et un point mobile $M$ pouvant se déplacer librement sur ce segment.

\begin{center}
\psset{xunit=1cm,yunit=1cm,algebraic=true,dimen=middle,dotstyle=o,dotsize=3pt 0,linewidth=0.8pt,arrowsize=3pt 2,arrowinset=0.25}
\begin{pspicture*}(0.6,0.27)(5.41,1.85)
\psline[linewidth=1.2pt](1,1)(5,1)
\rput[tl](2.92,1.5){$4$}
\rput[tl](1.5,0.7){$x$}
\psline{->}(3.2,1.4)(5,1.4)
\psline{->}(2.8,1.4)(1,1.4)
\psline{->}(1.4,0.6)(1,0.6)
\psline{->}(1.8,0.6)(2.2,0.6)
\psdots[dotstyle=*](1,1)
\rput[bl](0.85,1.08){$A$}
\psdots[dotstyle=*](5,1)
\rput[bl](5.06,1.08){$B$}
\psdots[dotstyle=*](2.2,1)
\rput[bl](2.26,1.08){$M$}
\end{pspicture*}
\end{center}
 

On note  $x$ la longueur du segment $\left[AM\right]$  et $f(x)$  le produit des longueurs $AM\times BM.$

\begin{enumerate}
\item $BM=AB-AM=4-x,$ donc
\begin{align*}
f(x)&=AM\times BM\\
&=x\times (4-x)\\
&=x\times 4+x\times (-x)\\
&=4x-x^2.
\end{align*}
\item Le produit des longueurs  $AM\times BM$ est donné par $f(x),$ donc maximiser ce produit revient à maximiser la fonction $f.$ On étudie donc les variations~ : pour tout $x\in\left[0;4\right],$
\[f'(x)=4\times 1-2x=-2x+4.\]

On résout~:

\begin{align*}-2x+4&=0\\
 -2x+\cancel{4}-\cancel{4}&=0-4\\
 \frac{\cancel{-2}x}{\cancel{-2}}&=\frac{-4}{-2}\\
 x&=2.
 \end{align*}

$a=-2,$ $a$ est $\ominus$ donc le signe est de la forme \fbox{$+~\upphi~-$}

\medskip


On obtient le tableau de signe de $f'$ et le tableau de variations de $f~:$


\medskip

\setlength{\columnseprule}{1pt}

\begin{multicols}{2}

\begin{center}
\begin{tikzpicture}[scale=0.8]
\tkzTabInit{$x$/1,$f'(x)$/1,$f(x)$/2}{$0$,$2$,$4$}
\tkzTabLine{,+,z,-,}
\tkzTabVar{-/,+/,-/}
\end{tikzpicture}
\end{center}

\columnbreak

Il n'est pas utile ici de compléter l'extrémité des flèches~: tout ce qui nous intéresse, c'est la valeur de $x$ pour laquelle $f$ atteint son maximum.
\end{multicols}

Conclusion~: $f$ atteint son maximum lorsque $x=2,$ donc le produit $AM\times BM$ est maximal lorsque $x=2~;$ c'est-à-dire quand $M$ est le milieu de $\left[AB\right].$
\end{enumerate}

\medskip

\textbf{Remarque~:} Cet exemple est celui qu'a choisi Fermat vers 1637 pour exposer sa méthode de l'adégalité -- ancêtre de la dérivation -- pour déterminer le maximum et le minimum d'une fonction.
\end{exo}

\begin{exo}

La fonction $g$ est définie sur $\mathbb{R}$ par 

\[g(x)=0,5x^3+0,75x^2-3x-1.\]

\medskip

Pour tout $x\in\mathbb{R}~:$

\[g'(x)=0,5\times 3x^2+0,75\times 2x-3\times 1-0=1,5x^2+1,5x-3.\]

La dérivée est du second degré, donc on utilise la méthode de la classe de première~:

\begin{itemize}
\item[\textbullet] $a=1,5,$ $b=1,5,$ $c=-3.$
\item[\textbullet] le discriminant est $\Delta=b^2-4ac=1,5^2-4\times 1,5\times (-3)=20,25.$
\item[\textbullet] $\Delta>0,$ donc il y a deux racines~:

\begin{align*}x_1&=\frac{-b-\sqrt{\Delta}}{2a}=\frac{-1,5-\sqrt{20,25}}{2\times 1,5}=\frac{-1,5-4,5}{3}=\frac{-6}{3}=-2,\\
x_2&=\frac{-b+\sqrt{\Delta}}{2a}=\frac{-1,5+\sqrt{20,25}}{2\times 1,5}=\frac{-1,5+4,5}{3}=\frac{3}{3}=1.
\end{align*}
\end{itemize}

\medskip

$a=1,5$  $a$ est $\oplus$ donc le signe est de la forme \fbox{$+~\upphi~-~\upphi~+$}

\medskip

\setlength{\columnseprule}{1pt}

\begin{multicols}{2}
\begin{center}
\begin{tikzpicture}[scale=0.7]
\tkzTabInit{$x$/1,$g'(x)$/1,$g(x)$/2}{$-\infty$,$-2$,$1$,$+\infty$}
\tkzTabLine{,+,z,-,z,+}
\tkzTabVar{-/,+/$4$,-/$-2.75$,+/}
\end{tikzpicture}
\end{center}

\columnbreak

\begin{itemize}
\item[\textbullet] $g(-2)=0,5\times(-2)^3+0,75\times (-2)^2-3\times (-2)-1=4$
\item[\textbullet] $g(1)=0,5\times 1^3+0,75\times 1^2-3\times 1-1=-2,75$
\end{itemize}


\end{multicols}

 \medskip
 
 \textbf{Remarque~:} Voici à quoi ressemble la courbe représentative~:
 

\begin{center}
\psset{xunit=1.0cm,yunit=0.5cm,algebraic=true,dimen=middle,dotstyle=o,dotsize=5pt 0,linewidth=2.pt,arrowsize=3pt 2,arrowinset=0.25}
\begin{pspicture*}(-4.08,-4.14)(4.,4.44)
\multips(0,-4)(0,1.0){9}{\psline[linestyle=dashed,linecap=1,dash=1.5pt 1.5pt,linewidth=0.4pt,linecolor=lightgray]{c-c}(-4.08,0)(4.,0)}
\multips(-4,0)(1.0,0){9}{\psline[linestyle=dashed,linecap=1,dash=1.5pt 1.5pt,linewidth=0.4pt,linecolor=lightgray]{c-c}(0,-4.14)(0,4.44)}
\psaxes[labelFontSize=\scriptstyle,xAxis=true,yAxis=true,Dx=1.,Dy=1.,ticksize=-2pt 0,subticks=2]{->}(0,0)(-4.08,-4.14)(4.,4.44)
\psplot[linewidth=2.pt,linecolor=blue,plotpoints=200]{-4.08}{4.0}{0.5*x^(3.0)+0.75*x^(2.0)-3.0*x-1.0}
\end{pspicture*}
\end{center}


\end{exo}

\begin{exo}

La fonction $h$ est définie sur $\left[1;+\infty\right[$ par 

\[h(x)=(x-6)\sqrt{x}.\]
 
 
 On utilise la formule pour la dérivée d'un produit avec
\begin{align*}
&u(x)=x-6&&,&& v(x)=\sqrt{x}, \\
& u'(x)=1&&, &&v'(x)=\frac{1}{2\sqrt{x}}.\\
\end{align*}

On obtient, pour tout $x\in \left[1;+\infty\right[~:$
\begin{align*}h'(x)&=u'(x)\times v(x)+u(x)\times v'(x)\\&=1\times\sqrt{x}+(x-6)\times\frac{1}{2\sqrt{x}}\\&=\frac{\sqrt{x}\times 2\sqrt{x}}{2\sqrt{x}}+\frac{x-6}{2\sqrt{x}}\\&=\frac{2x}{2\sqrt{x}}+\frac{x-6}{2\sqrt{x}}\qquad\qquad\qquad\left(\text{rappel~:}~\sqrt{x}\times\sqrt{x}=\sqrt{x}^2=x\right)\\&=\frac{3x-6}{2\sqrt{x}}.\end{align*}

\medskip

\begin{itemize}
\item[\textbullet] On résout rapidement~:
\[3x-6=0\iff 3x=6\iff x=\frac{6}{3}=2.\]
\item[\textbullet] Dans $3x-6,$ $a=3$ $\oplus$ , donc \fbox{$-~\upphi~+$}
\item[\textbullet] $2\sqrt{x}$ est strictement positif pour tout $x\in \left[1;+\infty\right[.$
\end{itemize}

\medskip

On a donc le tableau~:

\medskip

\setlength{\columnseprule}{1pt}

\begin{multicols}{2}
\begin{center}
\begin{tikzpicture}[scale=0.8]
\tkzTabInit{$x$/1,$3x-6$/1,$2\sqrt{x}$/1,$h'\left(x\right)$/1,$h\left(x\right)$/2}{$1$,$2$,$+\infty$}
\tkzTabLine{,-,z,+,}
\tkzTabLine{,+,,+,}
\tkzTabLine{,-,z,+,}
\tkzTabVar{+/$-5$,-/$-4\sqrt{2} $,+/}
\end{tikzpicture}
\end{center}

\columnbreak

\begin{itemize}
\item[\textbullet] $h(1)=(1-6)\times\sqrt{1}=-5\times 1=-5~;$
\item[\textbullet] $h(2)=(2-6)\times\sqrt{2}=-4\sqrt{2}.$
\end{itemize}

\end{multicols}

\end{exo}




\begin{exo}

La fonction $f$ est définie sur $\left[1;4\right]$ par $f(x)=x+\dfrac{4}{x}-3.$ On note $\mathcal{C}$ sa courbe représentative, $A,$ $B,$ $C$ les points de $\mathcal{C}$ d'abscisses respectives 1, 2, 4~; et $T_A,$ $T_B,$ $T_C$ les tangentes à $\mathcal{C}$ en ces points.
 
\begin{enumerate}
\item Pour dériver, le plus simple est de réécrire $f(x)$ sous la forme \[f(x)=x+4\times\dfrac{1}{x}-3.\] On obtient alors, pour tout $x\in \left[1;4\right]~:$
\begin{align*}
f'(x)&=1+4\times\left(-\frac{1}{x^2}\right)-0\\
&=1-\frac{4}{x^2}\\
&=\frac{x^2}{x^2}-\frac{4}{x^2}\\
&=\dfrac{x^2-4}{x^2}
\end{align*} 
\item \begin{itemize}
\item[\textbullet] Les racines de $x^2-4$ sont évidentes~: ce sont $x_1=-2$ et $x_2=2.$ Seule la deuxième est dans l'intervalle $\left[1;4\right].$
\item[\textbullet] $x^2$ est strictement positif pour tout $x\in \left[1;4\right].$
\end{itemize}

On obtient donc le tableau~:

\medskip

\setlength{\columnseprule}{1pt}

\begin{multicols}{2}
\begin{center}
\begin{tikzpicture}[scale=0.8]
\tkzTabInit{$x$/1,$x^2-4$/1,$x^2$/1,$f'\left(x\right)$/1,$f\left(x\right)$/2}{$1$,$2$,$4$}
\tkzTabLine{,-,z,+,}
\tkzTabLine{,+,,+,}
\tkzTabLine{,-,z,+,}
\tkzTabVar{+/$2$,-/$1$,+/$2$}
\end{tikzpicture}
\end{center}

\columnbreak

Le signe de $x^2-4$ sur $\left]-\infty;+\infty\right[$ est de la forme \fbox{$+~\upphi~-~\upphi~+$} Mais comme on travaille sur l'intervalle $\left[1;4\right],$ il ne reste plus que la partie droite \fbox{$-~\upphi~+$}

\medskip

On calcule les valeurs aux extrémités des flèches~:

\begin{itemize}
\item[\textbullet] $f(1)=1+\frac{4}{1}-3=2~;$
\item[\textbullet] $f(2)=2+\frac{4}{2}-3=1~;$
\item[\textbullet] $f(4)=4+\frac{4}{4}-3=2.$
\end{itemize}

\end{multicols}
\item On rappelle que la tangente à la courbe en un point d'abscisse $a$ a pour équation
\[y=f'(a)(x-a)+f(a).\]

Appliquons cette formule avec $a=1$ -- puisque le point $A$ a pour abscisse $1~:$

\medskip

$f(1)=2$ (déjà calculé) et $f'(1)=\frac{1^2-4}{1^2}=\frac{-3}{1}=-3,$ donc l'équation de $T_A$ est
\begin{align*}
y&=f'(1)(x-1)+f(1)\\
y&=-3(x-1)+2\\
y&=-3x+3+2\\
y&=-3x+5.
\end{align*}

Le point $A$ a pour coordonnées $(1;2),$ puisque $f(1)=2~;$ la tangente $T_A$ passe donc par ce point. Pour la tracer, il faut placer un deuxième point (c'est une droite)~; ce que l'on peut faire de trois façons différentes~:

\begin{enumerate}[(a)]
\item L'ordonnée à l'origine est $\textcolor{red}{5}$ (puisque $T_A:y=-3x\textcolor{red}{+5}$), donc $T_A$ passe par le point de coordonnées $(0;5).$
\item Le coefficient directeur de $T_A$ est $\textcolor{blue}{-3}$ (puisque $T_A:y=\textcolor{blue}{-3}x+5$), donc en partant de $A,$ il suffit d'avancer de $1$ carreau en abscisse et de descendre de $3$ carreaux en ordonnée -- $T_A$ passe donc par le point de coordonnées $(2;-1).$
\item On calcule un deuxième point avec la formule~: par exemple, si $x=2,$ $y=-3\times 2+5=-1.$ On obtient le point de coordonnées $(2;-1)$ (le même qu'avec la méthode (b)) et on trace la tangente.
\end{enumerate}

\item \begin{itemize}
\item[\textbullet] $f(2)=1$ et $f'(2)=\frac{2^2-4}{2^2}=\frac{0}{4}=0,$ donc l'équation de $T_B$ est
\begin{align*}
y&=f'(2)(x-2)+f(2)\\
y&=0(x-1)+1\\
y&=1.
\end{align*}

Le coefficient directeur étant égal à 0, la tangente $T_B$ est horizontale.
\item[\textbullet]  $f(4)=2$ et $f'(4)=\frac{4^2-4}{4^2}=\frac{12}{16}=0,75,$ donc l'équation de $T_C$ est
\begin{align*}
y&=f'(4)(x-4)+f(4)\\
y&=0,75(x-4)+2\\
y&=0,75x-3+2\\
y&=0,75x-1.\end{align*}

On trace la tangente $T_C$ par la même méthode que $T_A$ (le plus simple et le plus précis est d'utiliser l'ordonnée à l'origine).
\end{itemize}

\item On place les points $A,$ $B,$ $C,$ on trace les trois tangentes et on construit la courbe de la fonction $f$ (en bleu) en s'appuyant sur ces tangentes.


\begin{center}
\newrgbcolor{ffxfqq}{1. 0.4980392156862745 0.}
\psset{xunit=1.0cm,yunit=1.0cm,algebraic=true,dimen=middle,dotstyle=o,dotsize=5pt 0,linewidth=2.pt,arrowsize=3pt 2,arrowinset=0.25}
\begin{pspicture*}(-1.32,-1.88)(6.02,5.6)
\multips(0,-1)(0,1.0){8}{\psline[linestyle=dashed,linecap=1,dash=1.5pt 1.5pt,linewidth=0.4pt,linecolor=lightgray]{c-c}(-1.32,0)(6.02,0)}
\multips(-1,0)(1.0,0){8}{\psline[linestyle=dashed,linecap=1,dash=1.5pt 1.5pt,linewidth=0.4pt,linecolor=lightgray]{c-c}(0,-1.88)(0,5.6)}
\psaxes[labelFontSize=\scriptstyle,xAxis=true,yAxis=true,Dx=1.,Dy=1.,ticksize=-2pt 0,subticks=2]{->}(0,0)(-1.32,-1.88)(6.02,5.6)
\psline[linewidth=2.pt,linecolor=ffxfqq](0.,5.)(2.,-1.)
\rput[tl](2.1,-0.64){\ffxfqq{$T_A$}}
\psline[linewidth=2.pt,linecolor=green](0.,1.)(4.,1.)
\rput[tl](3.26,0.72){\green{$T_B$}}
\psplot[linewidth=2.pt,linecolor=magenta]{0.}{6.02}{(-4.--3.*x)/4.}
\rput[tl](0.48,-0.84){\magenta{$T_C$}}
\psplot[linewidth=2.pt,linecolor=blue,plotpoints=200]{1}{4}{x+4.0/x-3.0}
\rput[bl](1.08,2.2){\ffxfqq{$A$}}
\rput[bl](2.08,1.2){\green{$B$}}
\rput[bl](3.74,2.28){\magenta{$C$}}
\psdots[dotstyle=*,linecolor=ffxfqq](1.,2.)
\psdots[dotstyle=*,linecolor=green](2.,1.)
\psdots[dotstyle=*,linecolor=magenta](4.,2.)
\end{pspicture*}
\end{center}

\end{enumerate}

\end{exo}

\begin{exo}

La fonction $i$ est définie sur $\mathbb{R}$ par 

\[i(x)=\frac{2x}{x^2+1}.\]
 
\begin{enumerate}
\item On utilise la formule pour la dérivée d'un quotient avec
\begin{align*}
&u(x)=2x&&,&& v(x)=x^2+1, \\
& u'(x)=2&&, &&v'(x)=2x.\\
\end{align*}

On obtient, pour tout $x\in \mathbb{R}~:$
\begin{align*}i'(x)&=\frac{u'(x)\times v(x)-u(x)\times v'(x)}{(v(x))^2}
\\&=\frac{2\times\left(x^2+1\right)-2x\times 2x }{\left(x^2+1\right)^2}
\\&=\frac{2x^2+2-4x^2}{\left(x^2+1\right)^2}
\\&=\frac{-2x^2+2}{\left(x^2+1\right)^2}
.
\end{align*}


\item \begin{itemize}
\item[\textbullet] Les racines de $-2x^2+2$ sont assez évidentes~: 
\[-2x^2+2=0\iff 2=2x^2\iff 1=x^2\iff \left(x=1~\text{ou}~x=-1\right).\] 
\item[\textbullet] $\left(x^2+1\right)^2$ est strictement positif pour tout réel $x.$
\end{itemize}

On obtient donc le tableau~:

\medskip

\setlength{\columnseprule}{1pt}

\begin{multicols}{2}
\begin{center}
\begin{tikzpicture}[scale=0.7]
\tkzTabInit{$x$/1,$-2x^2+2$/1,$\left(x^2+1\right)^2$/1,$i'\left(x\right)$/1,$i\left(x\right)$/2}{$-\infty$,$-1$,$1$,$+\infty$}
\tkzTabLine{,-,z,+,z,-}
\tkzTabLine{,+,,+,,+}
\tkzTabLine{,-,z,+,z,-}
\tkzTabVar{+/,-/$-1$,+/$1$,-/}
\end{tikzpicture}
\end{center}

\columnbreak




\begin{itemize}
\item[\textbullet] $i(-1)=\frac{2\times(-1)}{(-1)^2+1}=\frac{-2}{2}=-1~;$
\item[\textbullet] $i(1)=\frac{2\times 1)}{1^2+1}=\frac{2}{2}=1.$
\end{itemize}

\end{multicols}


\item
\begin{enumerate}
\item $i(0)=\frac{2\times 0}{0^2+1}=\frac{0}{1}=0$ et $i'(0)=\frac{-2\times 0^2+2}{\left(0^2+1\right)^2}=\frac{2}{1}=2,$ donc l'équation de $(T)$ est
\begin{align*}
y&=f'(0)(x-0)+f(0)\\
y&=2x+0\\
y&=2x.\end{align*}

\item Pour étudier les positions relatives de $(C):y=\frac{2x}{x^2+1}$ et $(T):y=2x,$ on étudie \textbf{le signe de la différence}~:
\[\frac{2x}{x^2+1}-2x.\]
\begin{itemize}
\item[\textbullet] Pour les valeurs de $x$ pour lesquelles cette différence vaut 0, les deux courbes se coupent~;
\item[\textbullet] pour les valeurs de $x$ pour lesquelles cette différence est strictement positive, $(C)$ est au-dessus de $(T)~;$
\item[\textbullet] pour les valeurs de $x$ pour lesquelles cette différence est strictement négative, $(C)$ est en-dessous de $(T).$
\end{itemize}


\medskip

On commence par calculer la différence~:

\begin{align*}
\frac{2x}{x^2+1}-2x
&= \frac{2x}{x^2+1}-\frac{2x\left(x^2+1\right)}{x^2+1}
\\&=\frac{2x}{x^2+1}-\frac{2x^3+2x}{x^2+1}
\\&=\frac{\cancel{2x}-2x^3-\cancel{2x}}{x^2+1}
\\&=\frac{-2x^3}{x^2+1}.
\end{align*}

\medskip

\footnotesize
\setlength{\columnseprule}{1pt}

\begin{multicols}{2}

\begin{center}
\hspace*{-1cm}
\begin{tikzpicture}[scale=1]
\tkzTabInit{$x$/1,$-2x^3$/1,$\left(x^2+1\right)^2$/1,$\frac{-2x^3}{x^2+1}$/1,Positions relatives des courbes/3}{$-\infty$,$0$,$+\infty$}
\tkzTabLine{,+,z,-,}
\tkzTabLine{,+,,+,}
\tkzTabLine{,+,z,-,}
\tkzTabLine{,(C)\text{ au-dessus de }(T),\scriptsize{\Longstack{S\\e\\ \\c\\o\\u\\p\\e\\n\\t}},(C)\text{ en-dessous de }(T),}
\end{tikzpicture}
\end{center}
\normalsize

\columnbreak

Pour compléter le tableau de signe~:

\begin{itemize}
\item[\textbullet] $-2x^3=0$ lorsque $x=0~;$
\item[\textbullet] $-2x^3$ est $\ominus$ lorsque $x$ est strictement positif~;
\item[\textbullet] $-2x^3$ est $\oplus$ lorsque $x$ est strictement négatif~;
\item[\textbullet] $\left(x^2+1\right)^2$ est strictement positif pour tout réel $x.$
\end{itemize}

\end{multicols}
\end{enumerate}
\item ~{}


\begin{center}
\psset{xunit=1.0cm,yunit=1.0cm,algebraic=true,dimen=middle,dotstyle=o,dotsize=5pt 0,linewidth=2.pt,arrowsize=3pt 2,arrowinset=0.25}
\begin{pspicture*}(-5.92,-2.46)(5.98,2.64)
\multips(0,-2)(0,1.0){6}{\psline[linestyle=dashed,linecap=1,dash=1.5pt 1.5pt,linewidth=0.4pt,linecolor=lightgray]{c-c}(-5.92,0)(5.98,0)}
\multips(-5,0)(1.0,0){12}{\psline[linestyle=dashed,linecap=1,dash=1.5pt 1.5pt,linewidth=0.4pt,linecolor=lightgray]{c-c}(0,-2.46)(0,2.64)}
\psaxes[labelFontSize=\scriptstyle,xAxis=true,yAxis=true,Dx=1.,Dy=1.,ticksize=-2pt 0,subticks=2]{->}(0,0)(-5.92,-2.46)(5.98,2.64)
\psplot[linewidth=2.pt,linecolor=blue,plotpoints=200]{-5.919999999999999}{5.979999999999995}{2.0*x/(x^(2.0)+1.0)}
\psplot[linewidth=2.pt,linecolor=red]{-5.92}{5.98}{(-0.--2.*x)/1.}
\rput[tl](1.22,1.9){\red{$(T)$}}
\rput[tl](3.44,1.04){\blue{$(C)$}}
\end{pspicture*}
\end{center}
\end{enumerate}

\end{exo}

\begin{exo}

La distance (en m) parcourue au temps $t$ (en s) par une pierre en chute libre est $d(t)=5t^2.$

On lance cette pierre d'une hauteur de 20~m.



\begin{enumerate}
\item La pierre arrive au sol quand elle a parcouru 20~m. Il faut donc résoudre l'équation $5t^2=20~:$
\[5t^2=20\iff t^2=\frac{20}{5}\iff t^2=4\iff \left(t=2\quad\text{ou}\quad\underbrace{t=-2}_{\text{impossible}}\right).\]

Conclusion~: la pierre arrive au sol après 2~s.
\item On construit la courbe à partir d'un tableau de valeurs (avec un pas de 0,4 par exemple).

\begin{center}
\begin{tabular}{|l|c|c|c|c|c|c|}
\hline
   $t$ &$0$ &$0,4$ &$0,8$ &$1,2$ &$1,6$&$2$ \\
	\hline
	$d(t)$ &$0$ &$0,8$ &$3,2$ &$7,2$ &$12,8$&$20$ \\
	\hline
\end{tabular}
\end{center}

Pour obtenir ce tableau, on utilise la calculatrice (bien sûr, on met des $x$ à la place des $t$)~:

\medskip

\small

\setlength{\columnseprule}{1pt}
\begin{multicols}{4}

\begin{center}\textbf{Calculatrices collège}\end{center}

\medskip


\begin{itemize}
\item[\textbullet] \fbox{MODE}
\item[\textbullet] 4 : TABLE ou 4 : Tableau
\item[\textbullet] f(X)=$5\text{X}^2$ \fbox{EXE}

(si on demande g(X)=, ne rien rentrer)
\item[\textbullet] Début? $0$ \fbox{EXE}
\item[\textbullet] Fin? $2$ \fbox{EXE}
\item[\textbullet] Pas? $0,4$ \fbox{EXE}
\end{itemize}

\columnbreak

\begin{center}\textbf{NUMWORKS}\end{center}

\medskip

x s'obtient avec les touches  \fbox{alpha}  \fbox{x}

\begin{itemize}
\item[\textbullet] \fbox{\textcolor{yellow}{\faHome}}
\item[\textbullet] Fonctions \fbox{EXE} puis choisir Fonctions \fbox{EXE}
\item[\textbullet] f(x)=$5\text{x}^2$ \fbox{EXE}
\item[\textbullet] choisir Tableau \fbox{EXE} puis Régler l'intervalle \fbox{EXE}

\item[\textbullet] X début\qquad$0$ \fbox{EXE}
\item[\textbullet] X fin\qquad $2$ \fbox{EXE}
\item[\textbullet] Pas\qquad $0.4$ \fbox{EXE}
\item[\textbullet] choisir Valider
\end{itemize}

\columnbreak

\begin{center}\textbf{TI graphiques}\end{center}

\medskip


X s'obtient avec la touche \fbox{$x,t,\theta,n$}
\begin{itemize}
\item[\textbullet] \fbox{$f(x)$}
\item[\textbullet] $\text{Y}_1=5\text{X}^2$ \fbox{EXE}
\item[\textbullet] \fbox{2nde} \fbox{déf table}
\item[\textbullet] DébTable=$0$ \fbox{EXE}
\item[\textbullet] PasTable=$0.4$ \fbox{EXE}

ou

\tiny{$\Delta$}\normalsize Tbl=$0.4$ \fbox{EXE}
\item[\textbullet] \fbox{2nde} \fbox{table}
\end{itemize}

\columnbreak

\begin{center}\textbf{CASIO graphiques}\end{center}

\medskip


X s'obtient avec la touche \fbox{$\text{X},\theta,\text{T}$}
\begin{itemize}
\item[\textbullet] \fbox{MENU} puis choisir TABLE \fbox{EXE}
\item[\textbullet] $\text{Y}_1:5\text{X}^2$ \fbox{EXE}
\item[\textbullet] \fbox{F5} (on choisit donc SET)
\item[\textbullet] Start:$0$ \fbox{EXE}
\item[\textbullet] End:$2$ \fbox{EXE}
\item[\textbullet] Step:$0.4$ \fbox{EXE}
\item[\textbullet]\fbox{EXIT}
\item[\textbullet] \fbox{F6} (on choisit donc TABLE)
\end{itemize}
\end{multicols}

\normalsize


\begin{center}
\psset{xunit=2.5cm,yunit=0.25cm,algebraic=true,dimen=middle,dotstyle=o,dotsize=5pt 0,linewidth=2.pt,arrowsize=3pt 2,arrowinset=0.25}
\begin{pspicture*}(-0.28472279556053043,-2.064821322818694)(2.336654362677241,21.51292853619145)
\multips(0,0)(0,4.0){6}{\psline[linestyle=dashed,linecap=1,dash=1.5pt 1.5pt,linewidth=0.4pt,linecolor=lightgray]{c-c}(0,0)(2.336654362677241,0)}
\multips(0,0)(0.4,0){7}{\psline[linestyle=dashed,linecap=1,dash=1.5pt 1.5pt,linewidth=0.4pt,linecolor=lightgray]{c-c}(0,0)(0,21.51292853619145)}
\psaxes[labelFontSize=\scriptstyle,xAxis=true,yAxis=true,Dx=0.4,Dy=4.,ticksize=-2pt 0,subticks=2]{->}(0,0)(0.,0.)(2.336654362677241,21.51292853619145)
\rput[lt](1.8,2.9){\parbox{1.2343130420771191 cm}{temps \\  (en s)}}
\rput[lt](0.06674676755514843,19.46268941801665){\parbox{1.3124173894361588 cm}{distance \\  (en m)}}
\rput{0.}(0.,0.){\psplot[linewidth=2.pt,linecolor=blue]{0.}{2.}{x^2/2/0.1}}
\psplot[linewidth=2.pt,linecolor=red]{-0.28472279556053043}{2.336654362677241}{(-10.--10.*x)/0.5}
\psdots[dotsize=4pt 0,dotstyle=*,linecolor=blue](0.4,0.8)
\psdots[dotsize=4pt 0,dotstyle=*,linecolor=blue](0.8,3.2)
\psdots[dotsize=4pt 0,dotstyle=*,linecolor=blue](1.2,7.2)
\psdots[dotsize=4pt 0,dotstyle=*,linecolor=blue](1.6,12.8)
\psdots[dotsize=4pt 0,dotstyle=*,linecolor=red](2.,20.)
\end{pspicture*}
\end{center}


\item La vitesse de la pierre au moment de l'impact au sol est $d'(2).$

Or $d'(t)=5\times 2t=10t,$ donc $d'(2)=10\times 2=20.$ Ainsi la vitesse au moment de l'impact est de 20~m/s.

\medskip

\textbf{Remarques~:}

\begin{itemize}
\item[\textbullet] cette vitesse instantanée est le coefficient directeur de la tangente au point $A$ d'abscisse 2 (en rouge).
\item[\textbullet] la \og vraie formule \fg~{}(valable en l'absence de frottements) est $d(t)=4,9t^2.$ Dans l'exercice, on a pris 5 au lieu de 4,9 pour simplifier les calculs.
\end{itemize}
\end{enumerate}

\end{exo}


\begin{exo}



Dans cet exercice, on utilise deux propriétés du cours~:

\begin{itemize}
\item[\textbullet] la dérivée de $x\mapsto \text{e}^{ax+b}$ est $x\mapsto a\text{e}^{ax+b}~;$
\item[\textbullet] une exponentielle est strictement positive.
\end{itemize}

\medskip



\medskip

\small

\setlength{\columnseprule}{1pt}
\begin{multicols}{4}

Pour tout $x\in \mathbb{R}~:$

\begin{align*}
f(x)&=\text{e}^{0,5x+1}\\
f'(x)&=\underbrace{0,5}_{\oplus}\underbrace{\text{e}^{0,5x+1}}_{\oplus}
\end{align*}
\medskip

$f'>0$ donc $f$ strictement croissante sur $\mathbb{R}.$


\begin{center}
\psset{xunit=0.75cm,yunit=0.75cm,algebraic=true,dimen=middle,dotstyle=o,dotsize=5pt 0,linewidth=2.pt,arrowsize=3pt 2,arrowinset=0.25}
\begin{pspicture*}(-2.94,-0.5)(2.9,5.32)
\multips(0,0)(0,1.0){6}{\psline[linestyle=dashed,linecap=1,dash=1.5pt 1.5pt,linewidth=0.4pt,linecolor=lightgray]{c-c}(-2.94,0)(2.9,0)}
\multips(-2,0)(1.0,0){6}{\psline[linestyle=dashed,linecap=1,dash=1.5pt 1.5pt,linewidth=0.4pt,linecolor=lightgray]{c-c}(0,-0.5)(0,5.32)}
\psaxes[labelFontSize=\scriptstyle,xAxis=true,yAxis=true,Dx=1.,Dy=1.,ticksize=-2pt 0,subticks=2]{->}(0,0)(-2.94,-0.5)(2.9,5.32)
\psplot[linewidth=2.pt,linecolor=red,plotpoints=200]{-2.94}{2.900000000000002}{EXP(0.5*x+1.0)}
\end{pspicture*}
\end{center}
\columnbreak

Pour tout $x\in \mathbb{R}~:$

\begin{align*}
g(x)&=\text{e}^{-1,5x}\\
g'(x)&=\underbrace{-1,5}_{\ominus}\underbrace{\text{e}^{-1,5x}}_{\oplus}
\end{align*}
\medskip

$g'<0$ donc $g$ strictement décroissante sur $\mathbb{R}.$


\begin{center}
\psset{xunit=0.75cm,yunit=0.75cm,algebraic=true,dimen=middle,dotstyle=o,dotsize=5pt 0,linewidth=2.pt,arrowsize=3pt 2,arrowinset=0.25}
\begin{pspicture*}(-2.94,-0.5)(2.9,5.32)
\multips(0,0)(0,1.0){6}{\psline[linestyle=dashed,linecap=1,dash=1.5pt 1.5pt,linewidth=0.4pt,linecolor=lightgray]{c-c}(-2.94,0)(2.9,0)}
\multips(-2,0)(1.0,0){6}{\psline[linestyle=dashed,linecap=1,dash=1.5pt 1.5pt,linewidth=0.4pt,linecolor=lightgray]{c-c}(0,-0.5)(0,5.32)}
\psaxes[labelFontSize=\scriptstyle,xAxis=true,yAxis=true,Dx=1.,Dy=1.,ticksize=-2pt 0,subticks=2]{->}(0,0)(-2.94,-0.5)(2.9,5.32)
\psplot[linewidth=2.pt,linecolor=green,plotpoints=200]{-2.94}{2.900000000000002}{EXP(-1.5*x+0.0)}
\end{pspicture*}
\end{center}
\columnbreak
Pour tout $x\in \mathbb{R}~:$

\begin{align*}
h(x)&=\text{e}^{2x-2}\\
h'(x)&=\underbrace{2}_{\oplus}\underbrace{\text{e}^{2x-2}}_{\oplus}
\end{align*}
\medskip

$h'>0$ donc $h$ strictement croissante sur $\mathbb{R}.$


\begin{center}
\psset{xunit=0.75cm,yunit=0.75cm,algebraic=true,dimen=middle,dotstyle=o,dotsize=5pt 0,linewidth=2.pt,arrowsize=3pt 2,arrowinset=0.25}
\begin{pspicture*}(-2.94,-0.5)(2.9,5.32)
\multips(0,0)(0,1.0){6}{\psline[linestyle=dashed,linecap=1,dash=1.5pt 1.5pt,linewidth=0.4pt,linecolor=lightgray]{c-c}(-2.94,0)(2.9,0)}
\multips(-2,0)(1.0,0){6}{\psline[linestyle=dashed,linecap=1,dash=1.5pt 1.5pt,linewidth=0.4pt,linecolor=lightgray]{c-c}(0,-0.5)(0,5.32)}
\psaxes[labelFontSize=\scriptstyle,xAxis=true,yAxis=true,Dx=1.,Dy=1.,ticksize=-2pt 0,subticks=2]{->}(0,0)(-2.94,-0.5)(2.9,5.32)
\psplot[linewidth=2.pt,linecolor=orange,plotpoints=200]{-2.94}{2.900000000000002}{EXP(2*x-2.0)}
\end{pspicture*}
\end{center}
\columnbreak

Pour tout $x\in \mathbb{R}~:$

\begin{align*}
i(x)&=\text{e}^{-1x+1}\\
i'(x)&=\underbrace{-1}_{\ominus}\underbrace{\text{e}^{-1x+1}}_{\oplus}
\end{align*}
\medskip

$i'<0$ donc $i$ strictement décroissante sur $\mathbb{R}.$


\begin{center}
\psset{xunit=0.75cm,yunit=0.75cm,algebraic=true,dimen=middle,dotstyle=o,dotsize=5pt 0,linewidth=2.pt,arrowsize=3pt 2,arrowinset=0.25}
\begin{pspicture*}(-2.94,-0.5)(2.9,5.32)
\multips(0,0)(0,1.0){6}{\psline[linestyle=dashed,linecap=1,dash=1.5pt 1.5pt,linewidth=0.4pt,linecolor=lightgray]{c-c}(-2.94,0)(2.9,0)}
\multips(-2,0)(1.0,0){6}{\psline[linestyle=dashed,linecap=1,dash=1.5pt 1.5pt,linewidth=0.4pt,linecolor=lightgray]{c-c}(0,-0.5)(0,5.32)}
\psaxes[labelFontSize=\scriptstyle,xAxis=true,yAxis=true,Dx=1.,Dy=1.,ticksize=-2pt 0,subticks=2]{->}(0,0)(-2.94,-0.5)(2.9,5.32)
\psplot[linewidth=2.pt,linecolor=blue,plotpoints=200]{-2.94}{2.900000000000002}{EXP(-1*x+1.0)}
\end{pspicture*}
\end{center}



\end{multicols}

\medskip

\`A titre d'illustration, on a tracé les courbes des quatre fonctions. Elles ont toutes une allure très similaire, à deux différences près~:

\begin{itemize}
\item[\textbullet] elles montent lorsque $a>0,$ elles descendent lorsque $a<0~;$
\item[\textbullet] plus $|a|$ est grand, plus la pente de la partie inclinée est forte.
\end{itemize}


\end{exo}

\begin{exo}

La fonction $f$ est définie sur l'intervalle $\left[0;4\right]$ par

\[ f(x) = (-2x+1)\text{e}^{-x}.\]

\begin{enumerate}
\item  On utilise la formule pour la dérivée d'un produit avec

\begin{align*}
&u(x)=-2x+1&&,&& v(x)=\text{e}^{-x}, \\
& u'(x)=-2&&, &&v'(x)=-\text{e}^{-x}.\\
\end{align*}

On obtient, pour tout $x\in\left[0;4\right]~:$


\begin{align*}f'(x)&=u'(x)\times v(x)+u(x)\times v'(x)\\
&=-2\times\text{e}^{-x}+\left(-2x+1\right)\times \left(-\text{e}^{-x}\right)\\
&=-2\times\text{e}^{-x}+(-2x)\times\left(-\text{e}^{-x}\right)+1\times\left(-\text{e}^{-x}\right)\\
&=-2\times\text{e}^{-x}+2x\times\text{e}^{-x}-1\times\text{e}^{-x}\\
&=\left(-2+2x-1\right)\text{e}^{-x}\\
&=\left(2x-3\right)\text{e}^{-x}.
\end{align*}

\item On étudie le signe de $f'$ et on en déduit les variations de $f~:$

\begin{itemize}
\item[\textbullet] $2x-3=0\iff 2x=3\iff x=\frac{3}{2}\iff x=1,5~;$
\item[\textbullet] $\text{e}^{-x}$ est $\oplus$ pour tout réel $x.$
\end{itemize}

\medskip

\setlength{\columnseprule}{1pt}

\begin{multicols}{2}
\begin{center}
\begin{tikzpicture}[scale=0.8]
\tkzTabInit{$x$/1,$2x-3$/1,$\text{e}^{-x}$/1,$f'(x)$/1,$f(x)$/2}{$0$,$1.5$,$4$}
\tkzTabLine{,-,z,+,}
\tkzTabLine{,+,,+,}
\tkzTabLine{,-,z,+,}
\tkzTabVar{+/$1$,-/$-2\text{e}^{-1,5}$,+/$-7\text{e}^{-4}$}
\end{tikzpicture}
\end{center}

\columnbreak

\begin{itemize}
\item[\textbullet] $f(0)=(-2\times 0 +1)\times\underbrace{\text{e}^{-0}}_{=1}=1\times 1=1$
\item[\textbullet] $f(1,5)=(-2\times 1,5 +1)\times\text{e}^{-1,5}=-2\text{e}^{-1,5}\approx -0,45$
\item[\textbullet] $f(4)=(-2\times 4 +1)\times\text{e}^{-4}=-7\text{e}^{-4}\approx -0,13$
\end{itemize}
\end{multicols}

\end{enumerate}


\end{exo}

\begin{exo}

La fonction $g$ est définie sur $\mathbb{R}$ par $g(x)=\text{e}^x-x-1.$

\medskip

Pour tout $x\in\mathbb{R}~:$
\[g'(x)=\text{e}^x-1-0=\text{e}^x-1.\]

\medskip

\setlength{\columnseprule}{1pt}
\begin{multicols}{2}

On résout l'équation~:
\[\text{e}^x-1=0\iff \text{e}^x=1 \iff x=0.\]

\danger On a utilisé la propriété~: le seul nombre dont l'exponentielle est égale à 1 est 0.

\medskip

Pour avoir les signes dans chaque case du tableau, on remplace par des valeurs de $x~:$

\begin{itemize}
\item[\textbullet] pour l'intervalle $\left]-\infty;0\right[,$ on prend (par exemple) $x=-1$ et on calcule avec la calculatrice~:
\[g'(-1)=\text{e}^{-1}-1\approx -0,63\qquad \ominus ~;\]
\item[\textbullet] pour l'intervalle $\left]0;+\infty\right[,$ on prend (par exemple) $x=1$ et on calcule avec la calculatrice~:
\[g'(1)=\text{e}^{1}-1\approx 3,72\qquad \oplus .\]
\end{itemize}

\columnbreak

\begin{center}
\begin{tikzpicture}[scale=1]
\tkzTabInit{$x$/1,$g'(x)=\text{e}^x-1$/1,$g(x)$/2}{$-\infty$,$0$,$+\infty$}
\tkzTabLine{,-,z,+,}
\tkzTabVar{+/,-/$0$,+/}
\end{tikzpicture}

\medskip

\[g(0)=\text{e}^0-0-1=1-1=0.\]
\end{center}



\end{multicols}

\medskip

\textbf{Remarque~:} Le minimum de $g$ est $0,$ donc $g(x)\geq 0$ pour tout réel $x~;$ autrement dit $\text{e}^x-x-1\geq 0.$ Cette inégalité se réécrit
\[\text{e}^x\geq x+1.\]
On obtiendra ce résultat par une autre méthode dans l'exercice 18 (utilisation de la convexité). Cette inégalité sera utilisée plus tard dans l'année, pour démontrer des résultats sur les limites.


\end{exo}

\begin{exo}


\begin{align*}
\dfrac{\text{e}^8}{\text{e}^2\times \text{e}^1\times \text{e}^3}&=\dfrac{\text{e}^8}{\text{e}^{2+1+3}}=\dfrac{\text{e}^8}{\text{e}^{6}}=\text{e}^{8-6}=\text{e}^{2}\\
\dfrac{\text{e}\times\text{e}^2}{\left(\text{e}^2\right)^2}
&=\dfrac{\text{e}^1\times\text{e}^2}{\text{e}^{2\times 2}}
=\dfrac{\text{e}^{1+2}}{\text{e}^{4}}=\text{e}^{3-4}=\text{e}^{-1}\\
\left(\text{e}^2\right)^3\times\text{e}^{-5}&=\text{e}^{2\times 3}\times\text{e}^{-5}=\text{e}^{6-5}=\text{e}^{1}
\end{align*}

\end{exo}

\begin{exo}

Dans chaque cas, on note $\mathcal{S}$ l'ensemble des solutions.

\begin{enumerate}

\item \begin{align*}
&\text{e}^{x}=-3\\
&\text{Impossible, car une exponentielle est strictement positive}\\
&\mathcal{S}=\emptyset
\end{align*}

\item \begin{align*}
&\text{e}^{2x-1}=1\\
&2x-1=0&&\text{(le seul nombre dont l'exponentielle vaut 1 est 0)}\\
&x=\frac{1}{2}\\
&\mathcal{S}=\left\{\frac{1}{2}\right\}.
\end{align*}

\item L'équation $\text{e}^{2x}+2\text{e}^{x}=3$ se réécrit
\[\left(\text{e}^x\right)^2+2\text{e}^{x}-3=0.\]

Pour résoudre, il est astucieux de noter $X=\text{e}^x~;$ l'équation se réécrit alors sous la forme
\[X^2+2X-3=0.\]
On résout avec la méthode de la classe de première~:

\begin{itemize}
\item[\textbullet] $a=1,$ $b=2,$ $c=-3.$
\item[\textbullet] le discriminant est $\Delta=b^2-4ac=2^2-4\times 1\times (-3)=16.$
\item[\textbullet] $\Delta>0,$ donc il y a deux racines~:

\begin{align*}X_1&=\frac{-b-\sqrt{\Delta}}{2a}=\frac{-2-\sqrt{16}}{2\times 1}=\frac{-2-4}{2}=\frac{-6}{2}=-3,\\
X_2&=\frac{-b+\sqrt{\Delta}}{2a}=\frac{-2+\sqrt{16}}{2\times 1}=\frac{-2+4}{2}=\frac{2}{2}=1.
\end{align*}

\medskip

On a posé $X=\text{e}^x,$ donc il y a deux possibilités~:
\[\text{e}^x=-3\qquad\text{ou}\qquad \text{e}^x=1.\] La première équation n'a pas de solution, car une exponentielle est strictement positive~; la deuxième équation a une seule solution~: $x=0.$

\medskip

Conclusion~: L'unique solution de l'équation $\text{e}^{2x}+2\text{e}^{x}=3$ est $x=0~:$
\[\mathcal{S}=\left\{0\right\}.\]


\end{itemize}



\end{enumerate}
\end{exo}

\begin{exo}

On utilisera la propriété~: pour tout nombre réel $x,$\[\text{e}^x\times \text{e}^{-x}=1.\]


\begin{enumerate}
\item D'après l'identité remarquable $(a+b)^2=a^2+2ab+b^2~:$



\[\left(\text{e}^{x}+ \text{e}^{-x}\right)^2=\left(\text{e}^{x}\right)^2+2\times \underbrace{\text{e}^{x}\times \text{e}^{-x}}_{=1}+\left( \text{e}^{-x}\right)^2=\text{e}^{2x}+2+\text{e}^{-2x}.\]
\item On multiplie le numérateur et le dénominateur par $\text{e}^x~:$
\begin{align*}
\dfrac{\text{e}^{x}-\text{e}^{-x}}{\text{e}^x+\text{e}^{-x}}
&=\dfrac{\left(\text{e}^{x}-\text{e}^{-x}\right)\times\text{e}^x}{\left(\text{e}^{x}+\text{e}^{-x}\right)\times\text{e}^x}\\
&=\dfrac{\text{e}^{x}\times \text{e}^{x}-\text{e}^{-x}\times \text{e}^{x}}{\text{e}^{x}\times \text{e}^{x}-\text{e}^{-x}\times \text{e}^{x}}\\
&=\dfrac{\text{e}^{x+x}-\text{e}^{-x+x}}{\text{e}^{x+x}+\text{e}^{-x+x}}\\
&=\dfrac{\text{e}^{2x}-\text{e}^{0}}{\text{e}^{2x}+\text{e}^{0}}\\
&=\dfrac{\text{e}^{2x}-1}{\text{e}^{2x}+1}.
\end{align*}
\end{enumerate}


\end{exo}



\begin{exo}


\begin{enumerate}
\item La fonction $f$ est de la forme $f(x)=\text{e}^{u(x)},$ avec \[u(x)=-x^2,\qquad u'(x)=-2x.\]
 On a donc, pour tout $x\in\mathbb{R}~:$ \[f'(x)=u'(x)\times\text{e}^{u(x)}=-2x\text{e}^{-x^2}.\]
\item La fonction $h$ est de la forme $h(x)=\left(u(x)\right)^n,$ avec \[u(x)=-4x+1,\qquad u'(x)=-4,\qquad n=3.\]
 On a donc, pour tout $x\in\mathbb{R}~:$ \[h'(x)=n\times u'(x)\times \left(u(x)\right)^{n-1}=3\times (-4)\times (-4x+1)^{3-1}=-12\left(-4x+1\right)^2.\]
\item La fonction $i$ est de la forme $i(x)=\text{e}^{u(x)},$ avec \[u(x)=5x-9,\qquad u'(x)=5.\]
 On a donc, pour tout $x\in\mathbb{R}~:$ \[i'(x)=u'(x)\times\text{e}^{u(x)}=5\text{e}^{5x-9}.\]
\item La fonction $j$ est de la forme $j(x)=\left(u(x)\right)^n,$ avec \[u(x)=x^2-3x,\qquad u'(x)=2x-3,\qquad n=5.\]
 On a donc, pour tout $x\in\mathbb{R}~:$ \[j'(x)=n\times u'(x)\times \left(u(x)\right)^{n-1}=5\times \left(2x-3\right)\times \left(x^2-3x\right)^{5-1}=\left(10x-15\right)\times \left(x^2-3x\right)^4.\]
\item L'énoncé nous donne \[k(x)=\sqrt{x^2-x+2}.\] Il faut se méfier~: on ne peut calculer la racine carrée d'un nombre que si celui-ci est positif~; et on ne peut dériver une fonction de la forme $\sqrt{u}$ que lorsqu'elle est strictement positive. Intéressons-nous donc au signe de $x^2-x+2~:$

\medskip

Le discriminant est $\Delta=b^2-4ac=(-1)^2-4\times 1\times 2=-7.$ Il s'ensuit qu'il n'y a pas de racine, et que  $x^2-x+2$ est strictement positif sur $\mathbb{R}.$ La fonction $k$ est donc bien définie sur $\mathbb{R},$ mais aussi dérivable.

\medskip

Elle est de la forme $k(x)=\sqrt{u(x)},$ avec  \[u(x)=x^2-x+2,\qquad u'(x)=2x-1.\]
 On a donc, pour tout $x\in\mathbb{R}~:$ \[k'(x)=\dfrac{u'(x)}{2\sqrt{u(x)}}=\dfrac{2x-1}{2\sqrt{x^2-x+2}}.\]
\end{enumerate}


\medskip

\textbf{Remarque informelle~:} On a déjà vu les dérivées suivantes dans le cours de première~:

\begin{align*}
\left(x^n\right)'&=nx^{n-1}\\
\left(\text{e}^x\right)'&=\text{e}^x\\
\left(\sqrt{x}\right)'&=\dfrac{1}{2\sqrt{x}}
\end{align*}

\medskip

Les trois nouvelles formules du cours de terminale peuvent se réécrire

\begin{align*}
\left(u^n\right)'&=nu^{n-1}\times u'\\
\left(\text{e}^u\right)'&=\text{e}^u\times u'\\
\left(\sqrt{u}\right)'&=\dfrac{1}{2\sqrt{u}}\times u'
\end{align*}

\medskip

On voit qu'il suffit de remplacer $x$ par $u,$ et de multiplier par $u'.$

\end{exo}




\begin{exo}



\begin{enumerate}

\item Pour tout $x\in\mathbb{R}~:$
\begin{align*}
f(x)&=x^2\\
f'(x)&=2x\\
f''(x)&=2.
\end{align*}

\medskip

Conclusion~: $f''$ est strictement positive, donc $f$ est convexe sur $\mathbb{R}.$

\medskip

On peut aussi présenter les choses  avec un tableau de signe~:

\begin{center}
\hspace*{-1cm}
\begin{tikzpicture}[scale=1]
\tkzTabInit{$x$/1,$f''(x)=2$/1,Convexité/3}{$-\infty$,$+\infty$}
\tkzTabLine{,+,}

\tkzTabLine{,f\text{ convexe },}
\end{tikzpicture}
\end{center}

\medskip


\begin{center}
\psset{xunit=1.0cm,yunit=1.0cm,algebraic=true,dimen=middle,dotstyle=o,dotsize=5pt 0,linewidth=2.pt,arrowsize=3pt 2,arrowinset=0.25}
\begin{pspicture*}(-2.46,-0.74)(2.68,3.4)
\multips(0,0)(0,1.0){5}{\psline[linestyle=dashed,linecap=1,dash=1.5pt 1.5pt,linewidth=0.4pt,linecolor=lightgray]{c-c}(-2.46,0)(2.68,0)}
\multips(-2,0)(1.0,0){6}{\psline[linestyle=dashed,linecap=1,dash=1.5pt 1.5pt,linewidth=0.4pt,linecolor=lightgray]{c-c}(0,-0.74)(0,3.4)}
\psaxes[labelFontSize=\scriptstyle,xAxis=true,yAxis=true,Dx=1.,Dy=1.,ticksize=-2pt 0,subticks=2]{->}(0,0)(-2.46,-0.74)(2.68,3.4)
\rput{0.}(0.,0.){\psplot[linewidth=2.pt,linecolor=blue]{-3.}{3.}{x^2/2/0.5}}
\rput[tl](0.26,2.86){\blue{convexe}}
\end{pspicture*}
\end{center}



\item Pour tout $x\in\mathbb{R}~:$
\begin{align*}
g(x)&=x^3\\
g'(x)&=3x^2\\
g''(x)&=6x.
\end{align*}

\medskip

Cette fois, le tableau de signe est fortement recommandé~:


\begin{center}
\hspace*{-1cm}
\begin{tikzpicture}[scale=1.2]
\tkzTabInit{$x$/1,$g''(x)=6x$/1,Convexité/3}{$-\infty$,$0$,$+\infty$}
\tkzTabLine{,-,z,+,}

\tkzTabLine{,g\text{ concave},\scriptsize{\Longstack{P\\t\\ \\i\\n\\f\\l\\e\\x\\i\\o\\n}},g\text{ convexe},}
\end{tikzpicture}
\end{center}

Conclusion~:

\begin{itemize}
\item[\textbullet] $g$ est concave sur $\left]-\infty;0\right]~;$
\item[\textbullet] $g$ est convexe sur $\left[0;+\infty\right[~;$
\item[\textbullet] le point de coordonnées $(0;0)$ est un point d'inflexion.
\end{itemize}

\medskip


\begin{center}
\newrgbcolor{ffxfqq}{1. 0.4980392156862745 0.}
\psset{xunit=1.0cm,yunit=1.0cm,algebraic=true,dimen=middle,dotstyle=o,dotsize=5pt 0,linewidth=2.pt,arrowsize=3pt 2,arrowinset=0.25}
\begin{pspicture*}(-2.46,-1.76)(3.44,1.88)
\multips(0,-1)(0,1.0){4}{\psline[linestyle=dashed,linecap=1,dash=1.5pt 1.5pt,linewidth=0.4pt,linecolor=lightgray]{c-c}(-2.46,0)(3.44,0)}
\multips(-2,0)(1.0,0){6}{\psline[linestyle=dashed,linecap=1,dash=1.5pt 1.5pt,linewidth=0.4pt,linecolor=lightgray]{c-c}(0,-1.76)(0,1.88)}
\psaxes[labelFontSize=\scriptstyle,xAxis=true,yAxis=true,Dx=1.,Dy=1.,ticksize=-2pt 0,subticks=2]{->}(0,0)(-2.46,-1.76)(3.44,1.88)
\rput[tl](1.06,0.86){\blue{convexe}}
\psplot[linewidth=2.pt,linecolor=blue,plotpoints=200]{0}{3.4400000000000035}{x^(3.0)}
\psplot[linewidth=2.pt,linecolor=red,plotpoints=200]{-2.460000000000003}{0}{x^(3.0)}
\rput[tl](-2.34,-0.66){\red{concave}}
\rput[tl](0.58,-1.22){\green{point d'inflexion}}
\psline[linewidth=2.pt,linecolor=green]{->}(1.,-1.)(0.,0.)
\psdots[dotstyle=*,linecolor=green](0.,0.)
\end{pspicture*}
\end{center}
\item Pour tout $x\in\mathbb{R}~:$
\begin{align*}
h(x)&=\text{e}^{x}\\
h'(x)&=\text{e}^{x}\\
h''(x)&=\text{e}^{x}.
\end{align*}

\medskip

Conclusion~: $h''$ est strictement positive, donc $h$ est convexe sur $\mathbb{R}$ (cette fois, on se passe du tableau de signes).


\begin{center}
\psset{xunit=1.0cm,yunit=1.0cm,algebraic=true,dimen=middle,dotstyle=o,dotsize=5pt 0,linewidth=2.pt,arrowsize=3pt 2,arrowinset=0.25}
\begin{pspicture*}(-3.74,-0.86)(2.58,3.22)
\multips(0,0)(0,1.0){5}{\psline[linestyle=dashed,linecap=1,dash=1.5pt 1.5pt,linewidth=0.4pt,linecolor=lightgray]{c-c}(-3.74,0)(2.58,0)}
\multips(-3,0)(1.0,0){7}{\psline[linestyle=dashed,linecap=1,dash=1.5pt 1.5pt,linewidth=0.4pt,linecolor=lightgray]{c-c}(0,-0.86)(0,3.22)}
\psaxes[labelFontSize=\scriptstyle,xAxis=true,yAxis=true,Dx=1.,Dy=1.,ticksize=-2pt 0,subticks=2]{->}(0,0)(-3.74,-0.86)(2.58,3.22)
\rput[tl](-2.3,0.9){\blue{convexe}}
\psplot[linewidth=2.pt,linecolor=blue,plotpoints=200]{-3.7400000000000038}{2.5800000000000027}{EXP(x)}
\end{pspicture*}
\end{center}
\end{enumerate}

\end{exo}




\begin{exo}

La fonction $g$ est définie sur l'intervalle $\left[-1;3\right]$ par

\[ g(x) = -0,5 x^3+2x^2-2x.\]



\begin{enumerate}
\item \medskip

Pour tout $x\in\left[-1;3\right]~:$

\[g'(x)=-0,5\times 3x^2+2\times 2x-2\times 1=-1,5x^2+4x-2.\]

La dérivée est du second degré, donc on utilise la méthode de la classe de première~:

\begin{itemize}
\item[\textbullet] $a=-1,5,$ $b=4,$ $c=-2.$
\item[\textbullet] le discriminant est $\Delta=b^2-4ac=4^2-4\times (-1,5)\times (-2)=4.$
\item[\textbullet] $\Delta>0,$ donc il y a deux racines~:

\begin{align*}x_1&=\frac{-b-\sqrt{\Delta}}{2a}=\frac{-4-\sqrt{4}}{2\times (-1,5)}=\frac{-4-2}{-3}=\frac{-6}{-3}=2,\\
x_2&=\frac{-b+\sqrt{\Delta}}{2a}=\frac{-4+\sqrt{4}}{2\times (-1,5)}=\frac{-4+2}{-3}=\frac{-2}{-3}=\frac{2}{3}.
\end{align*}
\end{itemize}

\medskip

$a=-1,5$  $a$ est $\ominus$ donc le signe est de la forme \fbox{$-~\upphi~+~\upphi~-$}

\medskip

\setlength{\columnseprule}{1pt}

\begin{multicols}{2}
\begin{center}
\begin{tikzpicture}[scale=0.7]
\tkzTabInit{$x$/1,$g'(x)$/1,$g(x)$/2}{$-1$,$\frac{2}{3}$,$2$,$3$}
\tkzTabLine{,-,z,+,z,-}
\tkzTabVar{+/$3.5$,-/$-\frac{16}{27}$,+/$0$,-/$-1.5$}
\end{tikzpicture}
\end{center}

\columnbreak

\begin{itemize}
\item[\textbullet] $g(-1)=-0,5\times (-1)^3+2\times (-1)^2-2\times (-1)=3,5$
\item[\textbullet] $g\left(\frac{2}{3}\right)=-0,5\times \left(\frac{2}{3}\right)^3+2\times \left(\frac{2}{3}\right)^2-2\times \left(\frac{2}{3}\right)=-\frac{16}{27}$
\item[\textbullet] $g(2)=-0,5\times 2^3+2\times 2^2-2\times 2=0$
\item[\textbullet] $g(3)=-0,5\times 3^3+2\times 3^2-2\times 3=-1,5$

\end{itemize}


\end{multicols}

\item Pour tout $x\in\left[-1;3\right]~:$
\[g''(x)=-1,5\times 2x+4\times 1-0=-3x+4.\]


On étudie le signe de $g''~:$
\[-3x+4=0\iff -3x=-4\iff x=\frac{-4}{-3}=\frac{4}{3}.\]

\begin{center}
\begin{tikzpicture}[scale=1.2]
\tkzTabInit{$x$/1,$-3x+4$/1,Convexité/3}{$-1$,$\frac{4}{3}$,$3$}
\tkzTabLine{,+,z,-,}
\tkzTabLine{,g\text{ convexe},\scriptsize{\Longstack{P\\t\\ \\i\\n\\f\\l\\e\\x\\i\\o\\n}},g\text{ concave},}
\end{tikzpicture}
\end{center}

$g\left(\frac{4}{3}\right)=\left[\cdots\right]=-\frac{8}{27},$ donc le point de coordonnées $\left(\frac{4}{3};-\frac{8}{27}\right)$ est un point d'inflexion (noté $I$ sur la figure ci-dessous).
\item ~{}

\begin{center}
\psset{xunit=1.0cm,yunit=1.0cm,algebraic=true,dimen=middle,dotstyle=o,dotsize=5pt 0,linewidth=2.pt,arrowsize=3pt 2,arrowinset=0.25}
\begin{pspicture*}(-1.44,-1.64)(4.44,3.24)
\multips(0,-1)(0,1.0){5}{\psline[linestyle=dashed,linecap=1,dash=1.5pt 1.5pt,linewidth=0.4pt,linecolor=gray]{c-c}(-1.44,0)(4.44,0)}
\multips(-1,0)(1.0,0){6}{\psline[linestyle=dashed,linecap=1,dash=1.5pt 1.5pt,linewidth=0.4pt,linecolor=gray]{c-c}(0,-1.64)(0,3.24)}
\psaxes[labelFontSize=\scriptstyle,xAxis=true,yAxis=true,Dx=1.,Dy=1.,ticksize=-2pt 0,subticks=2]{->}(0,0)(-1.44,-1.64)(4.44,3.24)
\psplot[linewidth=2.pt,plotpoints=200,linecolor=blue]{-1.0}{1.33333}{-0.5*x^(3.0)+2.0*x^(2.0)-2.0*x+0}
\psplot[linewidth=2.pt,plotpoints=200,linecolor=red]{1.33333}{3}{-0.5*x^(3.0)+2.0*x^(2.0)-2.0*x+0.0}
\rput[tl](0.14,0.88){\blue{convexe}}
\rput[tl](3.16,-0.72){\red{concave}}
\psdots[dotstyle=*,linecolor=green](1.3333333333333333,-0.2777777777777777)
\rput[bl](1.42,-0.72){\green{$I$}}
\end{pspicture*}
\end{center}

\end{enumerate}

\end{exo}




\begin{exo}

La fonction $h$ est définie sur l'intervalle $\left[-1;4\right]$ par

\[ h(x) = (2x+3)\text{e}^{-x}.\]

On calcule les dérivées première et seconde~:

\medskip

\begin{enumerate}
\item \textbf{Dérivée première.} On utilise la formule pour la dérivée d'un produit avec

\begin{align*}
&u(x)=2x+3&&,&& v(x)=\text{e}^{-x}, \\
& u'(x)=2&&, &&v'(x)=-\text{e}^{-x}.\\
\end{align*}

On obtient, pour tout $x\in\left[-1;4\right]~:$


\begin{align*}h'(x)&=u'(x)\times v(x)+u(x)\times v'(x)\\
&=2\times\text{e}^{-x}+\left(2x+3\right)\times \left(-\text{e}^{-x}\right)\\
&=2\times\text{e}^{-x}+2x\times\left(-\text{e}^{-x}\right)+3\times\left(-\text{e}^{-x}\right)\\
&=2\times\text{e}^{-x}-2x\times\text{e}^{-x}-3\times\text{e}^{-x}\\
&=\left(2-2x-3\right)\text{e}^{-x}\\
&=\left(-2x-1\right)\text{e}^{-x}.
\end{align*}
\item \textbf{Dérivée seconde.}  On utilise la formule pour la dérivée d'un produit avec

\begin{align*}
&u(x)=-2x-1&&,&& v(x)=\text{e}^{-x}, \\
& u'(x)=-2&&, &&v'(x)=-\text{e}^{-x}.\\
\end{align*}

On obtient, pour tout $x\in\left[-1;4\right]~:$


\begin{align*}h''(x)&=u'(x)\times v(x)+u(x)\times v'(x)\\
&=-2\times\text{e}^{-x}+\left(-2x-1\right)\times \left(-\text{e}^{-x}\right)\\
&=-2\times\text{e}^{-x}+(-2x)\times\left(-\text{e}^{-x}\right)+(-1)\times\left(-\text{e}^{-x}\right)\\
&=-2\times\text{e}^{-x}+2x\times\text{e}^{-x}+1\times\text{e}^{-x}\\
&=\left(-2+2x+1\right)\text{e}^{-x}\\
&=\left(2x-1\right)\text{e}^{-x}.
\end{align*}
\end{enumerate}

\medskip

On étudie le signe de la dérivée seconde~:
\[h''(x)=\left(2x-1\right)\text{e}^{-x}.\]

\begin{itemize}
\item[\textbullet] $2x-1=0\iff 2x=1\iff x=\frac{1}{2}.$
\item[\textbullet] $\text{e}^{-x}$ est $\oplus$ pour tout $x\in\left[-1;4\right].$
\end{itemize}

\medskip

On a donc le tableau~:

\begin{center}
\begin{tikzpicture}[scale=1.2]
\tkzTabInit{$x$/1,$2x-1$/1,$\text{e}^{-x}$/1,$h''(x)$/1,Convexité/3}{$-1$,$\frac{1}{2}$,$4$}
\tkzTabLine{,-,z,+,}
\tkzTabLine{,+,,+,}
\tkzTabLine{,-,z,+,}
\tkzTabLine{,h\text{ concave},\scriptsize{\Longstack{P\\t\\ \\i\\n\\f\\l\\e\\x\\i\\o\\n}},h\text{ convexe},}
\end{tikzpicture}
\end{center}



\begin{center}
\psset{xunit=1.0cm,yunit=0.8cm,algebraic=true,dimen=middle,dotstyle=o,dotsize=5pt 0,linewidth=1.6pt,arrowsize=3pt 2,arrowinset=0.25}
\begin{pspicture*}(-2.26,-0.5)(5.3,3.8)
\multips(0,0)(0,1.0){6}{\psline[linestyle=dashed,linecap=1,dash=1.5pt 1.5pt,linewidth=0.4pt,linecolor=gray]{c-c}(-2.26,0)(6.18,0)}
\multips(-2,0)(1.0,0){9}{\psline[linestyle=dashed,linecap=1,dash=1.5pt 1.5pt,linewidth=0.4pt,linecolor=gray]{c-c}(0,0)(0,3.8)}
\psaxes[labelFontSize=\scriptstyle,xAxis=true,yAxis=true,Dx=1.,Dy=1.,ticksize=-2pt 0,subticks=2]{->}(0,0)(-2.26,-1.98)(5.3,3.8)
\psplot[linewidth=2.pt,plotpoints=200,linecolor=red]{-1}{0.5}{(2.0*x+3.0)*2.718281828459045^(-x)}
\psplot[linewidth=2.pt,plotpoints=200,linecolor=blue]{0.5}{4}{(2.0*x+3.0)*2.718281828459045^(-x)}
\psline[linewidth=2.pt,linecolor=green]{->}(1.1,3.3)(0.5,2.4261226388505337)
\rput[tl](0.3,3.8){\green{Point d'inflexion}}
\rput[tl](-1.68,1.46){\red{concave}}
\rput[tl](2.58,1.34){\blue{convexe}}
\psdots[dotsize=4pt 0,dotstyle=*,linecolor=green](0.5,2.4261226388505337)
\end{pspicture*}
\end{center}


\end{exo}

\begin{exo}

On note $\mathcal{C}$ la courbe de la fonction exponentielle et $T$ sa tangente au point $A(0;1).$

\begin{enumerate}

\item On pose $f(x)=\text{e}^x$ pour tout $x\in\mathbb{R}.$ On sait que $f'(x)=\text{e}^x$ pour tout $x\in\mathbb{R},$ donc\[f(0)=f'(0)=\text{e}^0=1.\] L'équation de la tangente $T$ est donc
\begin{align*}
y&=f'(0)(x-0)+f(0)\\
y&=1(x-0)+1\\
y&=x+1
\end{align*}
\item On a déjà vu dans un exercice précédent que la fonction exponentielle était convexe sur $\mathbb{R}.$ D'après le théorème 8 du cours, la courbe $\mathcal{C}$ est au-dessus de toutes ses tangentes~; elle est donc en particulier au-dessus de $T.$ Il s'ensuit que
\[\text{e}^x\geq x+1\] pour tout $x\in\mathbb{R}.$


\begin{center}
\psset{xunit=1.0cm,yunit=1.0cm,algebraic=true,dimen=middle,dotstyle=o,dotsize=5pt 0,linewidth=2.pt,arrowsize=3pt 2,arrowinset=0.25}
\begin{pspicture*}(-4.44,-1.04)(4.8,4.42)
\multips(0,-1)(0,1.0){6}{\psline[linestyle=dashed,linecap=1,dash=1.5pt 1.5pt,linewidth=0.4pt,linecolor=lightgray]{c-c}(-4.44,0)(4.8,0)}
\multips(-4,0)(1.0,0){10}{\psline[linestyle=dashed,linecap=1,dash=1.5pt 1.5pt,linewidth=0.4pt,linecolor=lightgray]{c-c}(0,-1.04)(0,4.42)}
\psaxes[labelFontSize=\scriptstyle,xAxis=true,yAxis=true,Dx=1.,Dy=1.,ticksize=-2pt 0,subticks=2]{->}(0,0)(-4.44,-1.04)(4.8,4.42)
\psplot[linewidth=2.pt,linecolor=red,plotpoints=200]{-4.439999999999999}{4.799999999999998}{EXP(x)}
\psplot[linewidth=2.pt,linecolor=blue]{-4.44}{4.8}{(--1.--1.*x)/1.}
\rput[tl](0.74,3.56){\red{$\mathcal{C}$}}
\rput[tl](1.78,2.6){\blue{$T$}}
\psdots[dotsize=5pt 0,dotstyle=*,linecolor=blue](0.,1.)
\rput[bl](0.12,0.68){\blue{$A$}}
\end{pspicture*}
\end{center}

\medskip

\textbf{Remarque~:} On a déjà démontré ce résultat par une étude de fonction, dans l'exercice 10.
\end{enumerate}

\end{exo}


\begin{exo}



\begin{enumerate}
\item Si $u(x)=x^2$ et $ v(x)=4x+1,$ alors
\[v\circ u(x)=v(u(x))=v\left(x^2\right)=4x^2+1.\]
\item Si $u(x)=x+2$ et $v(x)=x^3-3x,$ alors
\[v\circ u(x)=v(u(x))=v\left(x+2\right)=(x+2)^3-3(x+2).\]
\item  Si $u(x)=x-4$ et $ v(x)=\sqrt{x},$ alors
\[v\circ u(x)=v(u(x))=v\left(x-4\right)=\sqrt{x-4}.\]
\item Si $u(x)=2x+3$ et $v(x)=\text{e}^x,$ alors
\[v\circ u(x)=v(u(x))=v\left(2x+3\right)=\text{e}^{2x+3}.\]
\end{enumerate}

\end{exo}

\begin{exo}

\begin{enumerate}
\item Sachant que $v\circ u(x)=\sqrt{x^2+1},$ on peut prendre
\[u(x)=x^2+1\qquad,\qquad v(x)=\sqrt{x}.\]
\item Sachant que $v\circ u(x)=(x-3)^2+5(x-3)+1,$ on peut prendre
\[u(x)=x-3\qquad,\qquad v(x)=x^2+5x+1.\]
\item  Sachant que $v\circ u(x)=\text{e}^{3x-1},$ on peut prendre
\[u(x)=3x-1\qquad,\qquad v(x)=\text{e}^x.\]
\end{enumerate}

\medskip

\textbf{Remarque~:} Il y a une infinité de choix possibles. Par exemple, pour le deuxième, on pourrait prendre
\[u(x)=(x-3)^2+5(x-3)\qquad,\qquad v(x)=x+1~;\]

ou encore 
\[u(x)=(x-3)^2+5(x-3)+1\qquad,\qquad v(x)=x~;\]
etc.


\end{exo}

\begin{exo}

On considère dans un repère orthonormé la parabole $P:y=x^2$ et le point $A(3;0).$


\begin{enumerate}
\item Soit $m$ un réel et soit $M$ le point de $P$ d'abscisse $m.$ L'ordonnée de $M$ est $m^2,$ donc 
\begin{align*}AM&=\sqrt{\left(x_M-x_A\right)^2+\left(y_M-y_A\right)^2}
\\&=\sqrt{\left(m-3\right)^2+\left(m^2-0\right)^2}
\\&=\sqrt{m^2-2\times m\times 3+3^2+m^4}
\\&=\sqrt{m^4+m^2-6m+9}.
\end{align*}

\medskip

On remarque que $AM=f(m),$ où $f$ est la fonction définie dans la question suivante. De ce fait, trouver le point $M$ pour lequel la longueur $AM$ est minimale revient à trouver la valeur de $x$ pour laquelle $f$ atteint son minimum. Nous y reviendrons dans la question 3.
\item On pose $f(x)=\sqrt{x^4+x^2-6x+9}$ pour tout $x\in\mathbb{R}.$


La fonction $f$ est de la forme $f(x)=\sqrt{u(x)},$ avec  \[u(x)=x^4+x^2-6x+9,\qquad u'(x)=4x^3+2x-6.\]
 On a donc, pour tout $x\in\mathbb{R}~:$ \[f'(x)=\dfrac{u'(x)}{2\sqrt{u(x)}}=\dfrac{4x^3+2x-6}{2\sqrt{x^4+x^2-6x+9}}=\dfrac{\cancel{2}\left(2x^3+x-3\right)}{\cancel{2}\sqrt{x^4+x^2-6x+9}}=\dfrac{2x^3+x-3}{\sqrt{x^4+x^2-6x+9}}.\]


\medskip

Pour démontrer la formule de l'énoncé, on développe~:
\[(x-1)\left(2x^2+2x+3\right)=x\times 2x^2+x\times 2x+x\times 3-1\times 2x^2-1\times 2x-1\times 3=2x^3+2x^2+3x-2x^2-2x-3=2x^3+x-3.\]
On retombe sur le numérateur obtenu précédemment~; on a donc bien
\[f'(x)=\dfrac{(x-1)\left(2x^2+2x+3\right)}{\sqrt{x^4+x^2-6x+9}}.\]

Pour construire le tableau de variations de la fonction $f,$ il faut étudier le signe de $2x^2+2x+3.$ Son discriminant est $\Delta=2^2-4\times 2\times 3=-20,$ donc il n'y a pas de racine et $2x^2+2x+3$ est strictement positif pour tout réel $x.$ On peut donc compléter le tableau~:

\medskip

\begin{center}
\begin{tikzpicture}[scale=1.3]
\tkzTabInit{$x$/1,$x-1$/1,$2x^2+2x+3$/1,\footnotesize$\sqrt{x^4+x^2-6x+9}$/1,\normalsize$f'(x)$/1,$f(x)$/2}{$-\infty$,$1$,$+\infty$}
\tkzTabLine{,-,z,+,}
\tkzTabLine{,+,,+,}
\tkzTabLine{,+,,+,}
\tkzTabLine{,-,z,+,}
\tkzTabVar{+/,-/,+/}
\end{tikzpicture}
\end{center}

\item

\begin{itemize}
\item[\textbullet] La fonction $f$ atteint son minimum pour $x=1,$ donc la longueur $AM$ est minimale lorsque $m=1.$ Autrement dit, le point de $P$ le plus proche de $A$ est le point $M(1;1).$

\item[\textbullet] La tangente $(T)$ à la parabole $P$ au point $M$ a pour équation
\[y=g'(1)(x-1)+g(1),\]
avec $g(x)=x^2$ -- donc $g'(x)=2x,$ et $g'(1)=2\times 1=2.$ On a ainsi

\begin{align*}
(T):y&=g'(1)(x-1)+g(1)\\
y&=2(x-1)+1\\
y&=2x-1.
\end{align*}
\item[\textbullet] Pour prouver que $(AM)$ est perpendiculaire à $(T),$ on utilise le produit scalaire~:


$(T)$ passe par $M(1;1)$ et par $N(2;3)$ (puisque $2\times 2-1=3$), donc elle est dirigée par le vecteur $\overrightarrow{MN}\begin{pmatrix}1\\2\end{pmatrix}.$ Par ailleurs $\overrightarrow{AM}\begin{pmatrix}-2\\1\end{pmatrix},$ donc
\[\overrightarrow{MN}\cdot \overrightarrow{AM}=1\times (-2)+2\times 1=0.\] Les droites $(T)$ et $(AM)$ sont donc bien perpendiculaires.
\end{itemize}


\begin{center}
\newrgbcolor{ududff}{0.30196078431372547 0.30196078431372547 1.}
\psset{xunit=1.0cm,yunit=1.0cm,algebraic=true,dimen=middle,dotstyle=o,dotsize=5pt 0,linewidth=2.pt,arrowsize=3pt 2,arrowinset=0.25}
\begin{pspicture*}(-2.34,-0.92)(4.92,5.4)
\multips(0,0)(0,1.0){7}{\psline[linestyle=dashed,linecap=1,dash=1.5pt 1.5pt,linewidth=0.4pt,linecolor=lightgray]{c-c}(-2.34,0)(4.92,0)}
\multips(-2,0)(1.0,0){8}{\psline[linestyle=dashed,linecap=1,dash=1.5pt 1.5pt,linewidth=0.4pt,linecolor=lightgray]{c-c}(0,-0.92)(0,5.4)}
\psaxes[labelFontSize=\scriptstyle,xAxis=true,yAxis=true,Dx=1.,Dy=1.,ticksize=-2pt 0,subticks=2]{->}(0,0)(-2.34,-0.92)(4.92,5.4)
\pspolygon[linewidth=2.pt,linecolor=red,fillcolor=red!30!white,fillstyle=solid,opacity=0.1](1.3794733192202056,0.8102633403898972)(1.5692099788303084,1.1897366596101029)(1.1897366596101029,1.3794733192202056)(1.,1.)
\rput{0.}(0.,0.){\psplot[linewidth=2.pt,linecolor=blue]{-4.}{4.}{x^2/2/0.5}}
\psplot[linewidth=2.pt,linecolor=red]{-2.34}{4.92}{(-0.5--1.*x)/0.5}
\psline[linewidth=2.pt,linestyle=dashed,dash=2pt 2pt,linecolor=red](3.,0.)(1.,1.)
\rput[tl](2.28,3.36){\red{$(T)$}}
\rput[tl](-1.42,2.64){\blue{$P$}}
\psdots[dotstyle=*,linecolor=ududff](1.,1.)
\rput[bl](0.74,1.36){\ududff{$M$}}
\psdots[dotstyle=*,linecolor=red](3.,0.)
\rput[bl](3.08,0.2){\red{$A$}}
\end{pspicture*}
\end{center}
\end{enumerate}

\end{exo}

\section{Suites et récurrence}


\begin{exo}

On calcule trois ou quatre termes, suivant le cas -- suffisamment pour \og avoir compris le principe \fg.

\begin{enumerate}
\item Pour tout $n\in\mathbb{N}~:$ $u_n=\dfrac{n^2-1}{n+2}.$

\begin{align*}
u_0&=\dfrac{0^2-1}{0+2}=-\dfrac{1}{2}\\
u_1&=\dfrac{1^2-1}{1+2}=\dfrac{0}{3}=0\\
u_2&=\dfrac{2^2-1}{2+2}=\dfrac{3}{4}
\end{align*}

\item Pour tout $n\in\mathbb{N}^*~:$ $v_n=\dfrac{(-1)^n}{n}.$

\danger On \og démarre \fg~{} à $n=1,$ puisqu'on ne peut pas diviser par $0.$

\begin{align*}
v_1&=\dfrac{(-1)^1}{1}=\dfrac{-1}{1}=-1\\
v_2&=\dfrac{(-1)^2}{2}=\dfrac{1}{2}\\
v_3&=\dfrac{(-1)^3}{3}=\dfrac{-1}{3}=-\dfrac{1}{3}\\
v_4&=\dfrac{(-1)^4}{4}=\dfrac{1}{4}
\end{align*}

Les termes sont alternativement positifs et négatifs. On dit que la suite est alternée.
\item $u_0=3$ et pour tout $n\in\mathbb{N}~:$
\[u_{n+1}=2u_n-1.\]


\medskip

\setlength{\columnseprule}{1pt}

\begin{multicols}{4}

\begin{align*}
&\text{On prend}~n=0~:\\
u_{0+1}&=2u_0-1\\
u_{1}&=2\times 3-1\\
u_1&=5
\end{align*}

\begin{align*} 
&\text{On prend}~n=1~:\\
u_{1+1}&=2u_1-1\\
u_{2}&=2\times 5-1\\
u_2&=9
\end{align*}

\begin{align*}
&\text{On prend}~n=2~:\\
u_{2+1}&=2u_2-1\\
u_{3}&=2\times 9-1\\
u_3&=17
\end{align*}

\begin{align*}
&\text{On prend}~n=3~:\\
u_{3+1}&=2u_3-1\\
u_{4}&=2\times 17-1\\
u_4&=33
\end{align*}

\end{multicols}

\item $v_0=-1$ et $v_{n+1}=v_n+n$ \qquad pour tout $n\in\mathbb{N}.$

\medskip

\setlength{\columnseprule}{1pt}

\begin{multicols}{4}

\begin{align*}
&\text{On prend}~n=0~:\\
v_{0+1}&=v_0+0\\
v_{1}&=-1+0\\
v_1&=-1
\end{align*}

\begin{align*} 
&\text{On prend}~n=1~:\\
v_{1+1}&=v_1+1\\
v_{2}&=-1+1\\
v_2&=0
\end{align*}

\begin{align*}
&\text{On prend}~n=2~:\\
v_{2+1}&=v_2+2\\
v_{3}&=0+2\\
v_3&=2
\end{align*}

\begin{align*}
&\text{On prend}~n=3~:\\
v_{3+1}&=v_3+3\\
v_{4}&=2+3\\
v_4&=5
\end{align*}

\end{multicols}
\end{enumerate}

\end{exo}

\begin{exo}



\begin{enumerate}

\item Pour diminuer un nombre de 8~\%, il faut le multiplier par 0,92, car $100~\%-8~\%=92~\%=0,92$. On peut donc compléter le schéma~:

    \medskip


\begin{center}
    $\xymatrix@R=0.5pc@C=3pc{
    *+[F]+{10} \ar@/^0.5cm/[r]|{\red{\times 0,92}} & 
    *+[F]+{9,2} \ar@/^0.5cm/[r]|{\red{\times 0,92}} & *+[F]+{8,464} \\
    \txt{\blue{$v_0$}}&
    \txt{\blue{$v_1$}}&\txt{\blue{$v_2$}}\\
    \txt{\blue{Au départ}}&
    \txt{\blue{Après 1 jour}}&\txt{\blue{Après 2 jours}}
    }$
    \end{center}
   
   Conclusion~: 
    \[v_0=10~;~v_1=9,2~;~v_2=8,464.\]
    
    \medskip

La suite $(v_n)_{n\in\mathbb{N}}$ est géométrique de raison $q=0,92.$
\item La masse d'iode 131 après 10 jours est
\[v_{10}=v_0\times q^{10}=10\times 0,92^{10}\approx 4,3~\mu\text{g}.\]
\item On part de 10 $\mu$g d'iode 131, donc il s'agit de déterminer à partir de quand il en restera moins de 5 $\mu$g. Pour cela, on fait un tableau de valeurs avec la calculatrice, en rentrant la formule 
\[Y=10* 0.92^X\]
(on peut aussi utiliser le mode suite ou le mode tableur, suivant les modèles).

\medskip

Après quelques essais\footnote{On ne peut pas savoir en démarrant jusqu'à quelle valeur de $n$ il faut aller~; il faut donc faire des essais. Lorsque nous connaîtrons le logarithme népérien, nous pourrons donner une méthode plus efficace~; et nous pourrons même donner une formule~: la demi-vie est $-\frac{\ln 2}{\ln 0,92}.$}, on obtient~:

\begin{center}
\begin{tabular}{|c|c|c|}
\hline
	$n$&8&9\\\hline
$v_n$&$5,13$&$4,72$\\\hline
\end{tabular}
\end{center}

\medskip

Conclusion~: la demi-vie de l'iode 131 est de 8 jours et quelques.
\end{enumerate}

\end{exo}

\begin{exo}



\begin{enumerate}
\item $100~\%-15~\%=85~\%=0,85,$ donc pour diminuer un nombre de 15~\%, il faut le multiplier par 0,85. 

Ainsi, dans le schéma ci-dessous, l'intensité lumineuse est-elle multipliée par 0,85 à chaque nouvelle plaque~:

\begin{center}
\newrgbcolor{ududff}{0.30196078431372547 0.30196078431372547 1.}
\newrgbcolor{zzttqq}{0.6 0.2 0.}
\psset{xunit=0.75cm,yunit=0.75cm,algebraic=true,dimen=middle,dotstyle=o,dotsize=5pt 0,linewidth=2.pt,arrowsize=3pt 2,arrowinset=0.25}
\begin{pspicture*}(-1.2095636134454244,-0.8024760882800114)(15,7)
\pspolygon[linewidth=2.pt,linecolor=zzttqq,fillcolor=zzttqq!20!white,fillstyle=solid,opacity=0.1](0.,4.)(5.,4.)(7.,5.)(2.,5.)
\pspolygon[linewidth=2.pt,linecolor=zzttqq,fillcolor=zzttqq!20!white,fillstyle=solid,opacity=0.1](0.,2.)(5.,2.)(7.,3.)(2.,3.)
\pspolygon[linewidth=2.pt,linecolor=zzttqq,fillcolor=zzttqq!20!white,fillstyle=solid,opacity=0.1](0.,0.)(5.,0.)(7.,1.)(2.,1.)
\psline[linewidth=2.pt]{->}(9.006307576513136,3.604370307388384)(6.96981037851486,3.604370307388384)
\psline[linewidth=2.pt]{->}(8.989614976529543,5.607482305419475)(6.986502978498453,5.607482305419475)
\psline[linewidth=2.pt]{->}(8.989614976529543,1.6012583093572934)(6.986502978498453,1.584565709373701)
\psline[linewidth=2.pt]{->}(9.006307576513136,-0.3851610886902048)(6.986502978498453,-0.4018536886737972)
\rput[tl](-0.95,5.15){\textcolor{zzttqq}{1\up{re} plaque}}
\rput[tl](-0.95,3.15){\textcolor{zzttqq}{2\up{e} plaque}}
\rput[tl](-0.95,1.15){\textcolor{zzttqq}{3\up{e} plaque}}
\rput[tl](9.256696576267021,5.8){$v_0=12~\text{lm}$}
\rput[tl](9.25,3.8){$v_1=12\times 0,85=10,2~\text{lm}$}
\rput[tl](9.25,1.8){$v_2=10,2\times 0,85=8,67~\text{lm}$}
\rput[tl](9.25,-0.2){$v_3=8,67\times 0,85=7,3695~\text{lm}$}
\psline[linewidth=2.pt,linecolor=yellow](4.5,7.)(4.5,5.5)
\psline[linewidth=2.pt,linecolor=yellow](4.,7.)(3.5,5.5)
\psline[linewidth=2.pt,linecolor=yellow](5.,7.)(5.5,5.5)
\rput[tl](1.5,6.328984750608794){\yellow{lumière}}
\end{pspicture*}
\end{center}

\medskip

\textbf{Remarque~:} Le lumen est une unité de mesure du flux lumineux, utilisée notamment pour indiquer la capacité d'éclairement des ampoules électriques.
\item La suite $(v_n)_{n\in\mathbb{N}}$ est géométrique de raison $q=0,85,$ donc pour tout $n\in\mathbb{N}~:$
\[v_n=v_0\times q^n=12\times 0,85^n.\]
\item Comme on part de 12~lm, il s'agit de savoir le nombre de plaques nécessaires pour que l'intensité lumineuse soit inférieure à 0,12~lm (puisque $12\div 100=0,12$).

\medskip

Comme dans l'exercice précédent, on rentre la formule \[Y=12*0.85^X\] dans le mode fonction  de la calculatrice, puis on fait des essais. On obtient~:

\begin{center}
\begin{tabular}{|c|c|c|}
\hline
	$n$&28&29\\\hline
$v_n$&$0,13$&$0,11$\\\hline
\end{tabular}
\end{center}

\medskip

Conclusion~: il faut superposer au moins 29 plaques pour que l'intensité lumineuse soit divisée par 100.

\end{enumerate}

\end{exo}

\begin{exo}

Une suite $v$ est définie par $v_0=4$ et la relation de récurrence \[v_{n+1}=2v_n+2\] pour tout entier naturel $n.$

\begin{enumerate}
\item \begin{align*}
v_0&=4\\
v_1&=2\times 4+2=10\\
v_2&=2\times 10+2=22.\end{align*}

\newpage
\item Avec un schéma~:

\setlength{\columnseprule}{1pt}

\begin{multicols}{2}
~{}\begin{center}
    $\xymatrix@R=0.5pc@C=3pc{
    *+[F]+{4} \ar@/^0.5cm/[r]|{\red{+6}} \ar@/_0.5cm/[r]|{\green{\times 2,5}} & 
    *+[F]+{10} \ar@/^0.5cm/[r]|{\red{+12}} \ar@/_0.5cm/[r]|{\green{\times 2,2}} & *+[F]+{22} \\
    \txt{\blue{$v_0$}}&
    \txt{\blue{$v_1$}}&\txt{\blue{$v_2$}}    
    }$
    \end{center}
    
    
    \medskip
    
    Les résultats en rouge (\textcolor{red}{6} et \textcolor{red}{12}) sont différents, donc $u$ \textbf{n'est pas arithmétique.}
    
    \medskip
    
    Les résultats en vert (\textcolor{green}{2,5} et \textcolor{green}{2,2}) sont différents, donc $u$ \textbf{n'est pas géométrique.}
    
  \columnbreak
  
  Calculs utiles~:
  
  \medskip
  
  \begin{align*}
  10-4&=\textcolor{red}{6},\\
  22-10&= \textcolor{red}{12}.
  \end{align*}
  
  \medskip
  
  \begin{align*}
  10\div 4&=\textcolor{green}{2,5},\\
  22\div 10&= \textcolor{green}{2,2}.
  \end{align*}
  
  \end{multicols}


\end{enumerate}
\end{exo}

\begin{exo}

La suite $(u_n)_{n\in\mathbb{N}}$ est définie par $u_0=2$ et la relation de récurrence \[u_{n+1}=3 u_n-1\] pour tout $n\in\mathbb{N}.$


\begin{enumerate}
\item \begin{align*}
u_0&=2\\
u_1&=3\times 2-1=5\\
u_2&=3\times 5-1=14
\end{align*}

\item On pose $v_n=u_n-0,5$ pour tout entier naturel $n.$

\begin{align*}
v_0&=u_0-0,5=2-0,5=1,5\\
v_1&=u_1-0,5=5-0,5=4,5\\
v_2&=u_2-0,5=14-0,5=13,5
\end{align*}
\item Pour tout $n\in\mathbb{N}~:$



\begin{alignat*}{3}
&v_{n+1}&& =u_{n+1}-0,5 && \text{  (déf. de } (v_n)_{n\in\mathbb{N}})\\
& && =(3u_n-1)-0,5 && \text{  (rel. réc. pour } (u_n)_{n\in\mathbb{N}})\\
& && =3u_n-1,5 && \text{  (calcul)}\\
& && =3\left(u_n-\frac{1,5}{3}\right) && \text{  (factorisation)}\\
& && =3(u_n-0,5) && \text{  (calcul)}\\
& && =3v_n&& \text{  (déf. de } (v_n)_{n\in\mathbb{N}})\\
\end{alignat*}


Conclusion~: pour tout $n\in\mathbb{N},$ $v_{n+1}=3v_n,$
donc $(v_n)_{n\in\mathbb{N}}$ est géométrique de raison $q=3.$

\medskip

\textbf{Remarque~:} L'étude d'une suite arithmético-géométrique ($u_{n+1}=au_n+b,$ avec $a\not=1$) se ramène toujours à celle d'une suite géométrique $(v_n)_{n\in\mathbb{N}}.$ Pour prouver que $(v_n)_{n\in\mathbb{N}}$ est géométrique, la méthode est toujours celle que nous venons de donner. \`A la quatrième ligne de calcul, c'est $a$ qu'il faut mettre  en facteur (ici, on a mis 3 en facteur).

\item La suite $(v_n)_{n\in\mathbb{N}}$ est géométrique de raison $q=3,$ et  $v_0=u_0-0,5=2-0,5=1,5,$ donc pour tout $n\in\mathbb{N}~:$
\[v_n=v_0\times q^n=1,5\times 3^n.\]
\item Enfin $v_n=u_n-1,5$ donc
\[u_n=v_n+1,5=1,5\times 3^n+0,5.\]
\end{enumerate}

\end{exo}



\begin{exo}




\begin{enumerate}
\item On complète le schéma ci-dessous pour calculer les termes $u_1$ et $u_2.$ Les sommes écrites dans chaque case sont les sommes restant à rembourser aux dates indiquées.

\medskip

\begin{center}
    $\xymatrix@R=0.5pc@C=3pc{
    *+[F]+{\np{10000}} \ar@/^0.5cm/[r]|{\red{\times 1,02}} & 
    *+[F]+{\np{10200}} \ar@/^0.5cm/[r]|{\red{-~300}} & *+[F]+{\np{9900}}\ar@/^0.5cm/[r]|{\red{\times 1,02}} & *+[F]+{\np{10098}}\ar@/^0.5cm/[r]|{\red{-~300}} & *+[F]+{\np{9798}} \\
    \txt{\blue{$u_0$}}&\txt{\blue{$~$}}&
    \txt{\blue{$u_1$}}&\txt{\blue{$~$}}&\txt{\blue{$u_2$}}\\
    \txt{\blue{01/01/20}}&
    \txt{\blue{31/01/20}}&\txt{\blue{01/02/20}}&
    \txt{\blue{28/02/20}}&\txt{\blue{01/03/20}}
    }$
    \end{center}

Pour passer d'un terme de la suite au terme suivant, on multiplie par $1,02$ (ajout des intérêts) puis on retranche 300 (remboursement mensuel). On peut donc continuer plus rapidement~:

\begin{align*}
u_3&=\np{9798}\times 1,02-300=\np{9693,96}&&\text{(somme à rembourser le 01/04/20)},\\
u_4&=\np{9693,96}\times 1,02-300=\np{9587,84}&&\text{(somme à rembourser le 01/05/20)}
.\end{align*}

\item Pour tout $n\in\mathbb{N}~:$ \[u_{n+1}=1,02u_n-300.\]
\item Pour tout $n\in\mathbb{N}~:$



\begin{alignat*}{3}
&v_{n+1}&& =u_{n+1}-\np{15000} && \text{  (déf. de } (v_n)_{n\in\mathbb{N}})\\
& && =(1,02u_n-300)-\np{15000} && \text{  (rel. réc. pour } (u_n)_{n\in\mathbb{N}})\\
& && =1,02 u_n-\np{15300} && \text{  (calcul)}\\
& && =1,02\left(u_n-\frac{\np{15300}}{1,02}\right) && \text{  (factorisation)}\\
& && =1,02(u_n-\np{15000}) && \text{  (calcul)}\\
& && =1,02v_n&& \text{  (déf. de } (v_n)_{n\in\mathbb{N}})\\
\end{alignat*}


Conclusion~: pour tout $n\in\mathbb{N},$ $v_{n+1}=1,02v_n,$
donc $(v_n)_{n\in\mathbb{N}}$ est géométrique de raison $q=1,02.$


\item La suite $(v_n)_{n\in\mathbb{N}}$ est géométrique de raison $q=1,02,$ et  $v_0=u_0-\np{15000}=\np{10000}-\np{15000}=-\np{5000},$ donc pour tout $n\in\mathbb{N}~:$
\[v_n=v_0\times q^n=-\np{5000}\times 1,02^n.\]

Enfin $v_n=u_n-\np{15000}$ donc
\[u_n=v_n+\np{15000}=-\np{5000}\times 1,02^n+\np{15000}.\]
\item Déterminer la durée du crédit revient à savoir quand la somme restant à rembourser est nulle. En réalité, au bout d'un moment, elle est négative, comme on le voit avec un tableau de valeurs~:

\medskip

\begin{center}
\begin{tabular}{|c|c|c|}\hline
	$n$&$55$&$56$	\\ \hline   
$u_n$&$141,34$&$-155,83$ \\ \hline    
\end{tabular}
\end{center}

\medskip

\`A la fin du 55\up{e} fois, il reste 141,35~\euro~{} à rembourser~; et si on rembourse 300~\euro~{} au début du 56\up{e} mois, la banque nous devra 155,83~\euro.


Conclusion~:

\begin{itemize}
\item[\textbullet] le crédit dure 56 mois~;
\item[\textbullet] on rembourse 56 fois 300~\euro, mais à la fin on a dépassé de 155,83~\euro~{} ce que l'on devait à la banque~;
\item[\textbullet] la somme totale remboursée est donc
\[56\times 300-155,83=\np{16664,17}~\text{\euro}~;\]
\item[\textbullet] le \og coût du crédit \fg~{} est la différence entre ce que l'on a remboursé et ce que la banque nous a prêté~:
\[\text{Coût du crédit}=\text{Somme remboursée}-\text{Somme empruntée}=
\np{16664,17}-\np{10000}=\np{6664,17}~\text{\euro}.\]
\end{itemize}

\end{enumerate}


\end{exo}




\begin{exo}


La suite $(u_n)_{n\in\mathbb{N}}$ est définie par $u_0=0$ et pour tout $n\in\mathbb{N}~:$ \[u_{n+1}=2 u_n+1.\]

Pour tout $n\in\mathbb{N},$ on note $\mathcal{P}_n$ la propriété \[u_n=2^n-1.\]





\begin{itemize}
\item[{\textbullet}] \textbf{Initialisation.} On prouve que $\mathcal{P}_0$ est vraie.

\[
\left.
    \begin{array}{ll}
        u_0&=0 \\
        2^0-1&= 1-1=0
    \end{array}
\right \}\implies \mathcal{P}_0~\text{est vraie}.
\]



\item[{\textbullet}] \textbf{Hérédité.} Soit $k\in\mathbb{N}$ tel que $\mathcal{P}_k$ soit vraie. On a donc
\[u_k=2^k-1.\]

%\medskip

\newtcolorbox{mybox}[1]{colback=green!10!white,colframe=green!80!white,fonttitle=\bfseries,title=#1}
\begin{mybox}{Objectif}{Prouver que $\mathcal{P}_{k+1}$ est vraie, c'est-à-dire que \[u_{k+1}=2^{k+1}-1.\]
}\end{mybox}



%\medskip

On part de 
\[u_k=2^k-1.\]


On a alors~:

\begin{alignat*}{3}
&u_{k+1}&& =2{u_k}+1 && \text{  (rel. réc. pour } (u_n)_{n\in\mathbb{N}})\\
& && =2\left({2^k-1}\right)+1 && \text{ {(H.R.)}}\\
& && =2\times 2^k-2+1 && \text{  (on développe)}\\
& && =2^{k+1}-1 && \text{  (calcul)}.\\
\end{alignat*}



La propriété $\mathcal{P}_{k+1}$ est donc vraie.
\item[{\textbullet}] \textbf{Conclusion.} $\mathcal{P}_0$ est vraie et $\mathcal{P}_n$ est héréditaire, donc elle est vraie pour tout $n\in\mathbb{N}.$
\end{itemize}


\end{exo}

\begin{exo}


La suite $(u_n)_{n\in\mathbb{N}}$ est définie par $u_0=1$ et pour tout $n\in\mathbb{N}~:$ \[u_{n+1}=u_n+2n+3.\]

\begin{enumerate}
\item \begin{align*}
(n=0)\qquad u_{1}&=u_0+2\times 0+3=1+0+3=4\\
(n=1)\qquad u_{2}&=u_1+2\times 1+3=4+2+3=9\\
(n=2)\qquad u_{3}&=u_2+2\times 2+3=9+4+3=16
\end{align*}

\medskip

\textbf{Remarque~:} Pour passer de $u_n$ à $u_{n+1},$ on ajoute $2n+3,$ donc à partir de $u_0=1$ (rond rose ci-dessous)~:

\begin{itemize}
\item[\textbullet] on obtient $u_1$ en ajoutant $2\times 0+3=3$ ronds bleus~;
\item[\textbullet] on obtient $u_2$ en ajoutant $2\times 1+3=5$ ronds oranges~;
\item[\textbullet] on obtient $u_3$ en ajoutant $2\times 2+3=7$ ronds verts~;
\item[\textbullet] etc.
\end{itemize}

\medskip

\begin{center}
\newrgbcolor{ffxfqq}{1. 0.4980392156862745 0.}
\psset{xunit=0.75cm,yunit=0.75cm,algebraic=true,dimen=middle,dotstyle=o,dotsize=5pt 0,linewidth=2.pt,arrowsize=3pt 2,arrowinset=0.25}
\begin{pspicture*}(-4.64,0.)(8.64,5.48)
\pscircle[linewidth=2.pt,linecolor=magenta,fillcolor=magenta,fillstyle=solid,opacity=1](-1.,5.){0.3}
\pscircle[linewidth=2.pt,linecolor=blue,fillcolor=blue,fillstyle=solid,opacity=1](-2.,4.){0.3}
\pscircle[linewidth=2.pt,linecolor=blue,fillcolor=blue,fillstyle=solid,opacity=1](-1.,4.){0.3}
\pscircle[linewidth=2.pt,linecolor=blue,fillcolor=blue,fillstyle=solid,opacity=1](0.,4.){0.3}
\pscircle[linewidth=2.pt,linecolor=ffxfqq,fillcolor=ffxfqq,fillstyle=solid,opacity=1](-3.,3.){0.3}
\pscircle[linewidth=2.pt,linecolor=ffxfqq,fillcolor=ffxfqq,fillstyle=solid,opacity=1](-2.,3.){0.3}
\pscircle[linewidth=2.pt,linecolor=ffxfqq,fillcolor=ffxfqq,fillstyle=solid,opacity=1](-1.,3.){0.3}
\pscircle[linewidth=2.pt,linecolor=ffxfqq,fillcolor=ffxfqq,fillstyle=solid,opacity=1](0.,3.){0.3}
\pscircle[linewidth=2.pt,linecolor=ffxfqq,fillcolor=ffxfqq,fillstyle=solid,opacity=1](1.,3.){0.3}
\pscircle[linewidth=2.pt,linecolor=green,fillcolor=green,fillstyle=solid,opacity=1](-4.,2.){0.3}
\pscircle[linewidth=2.pt,linecolor=green,fillcolor=green,fillstyle=solid,opacity=1](-3.,2.){0.3}
\pscircle[linewidth=2.pt,linecolor=green,fillcolor=green,fillstyle=solid,opacity=1](-2.,2.){0.3}
\pscircle[linewidth=2.pt,linecolor=green,fillcolor=green,fillstyle=solid,opacity=1](-1.,2.){0.3}
\pscircle[linewidth=2.pt,linecolor=green,fillcolor=green,fillstyle=solid,opacity=1](0.,2.){0.3}
\pscircle[linewidth=2.pt,linecolor=green,fillcolor=green,fillstyle=solid,opacity=1](1.,2.){0.3}
\pscircle[linewidth=2.pt,linecolor=green,fillcolor=green,fillstyle=solid,opacity=1](2.,2.){0.3}
\pscircle[linewidth=2.pt,linecolor=magenta,fillcolor=magenta,fillstyle=solid,opacity=1](5.,2.){0.3}
\pscircle[linewidth=2.pt,linecolor=blue,fillcolor=blue,fillstyle=solid,opacity=1](5.,3.){0.3}
\pscircle[linewidth=2.pt,linecolor=blue,fillcolor=blue,fillstyle=solid,opacity=1](6.,3.){0.3}
\pscircle[linewidth=2.pt,linecolor=blue,fillcolor=blue,fillstyle=solid,opacity=1](6.,2.){0.3}
\pscircle[linewidth=2.pt,linecolor=ffxfqq,fillcolor=ffxfqq,fillstyle=solid,opacity=1](5.,4.){0.3}
\pscircle[linewidth=2.pt,linecolor=ffxfqq,fillcolor=ffxfqq,fillstyle=solid,opacity=1](6.,4.){0.3}
\pscircle[linewidth=2.pt,linecolor=ffxfqq,fillcolor=ffxfqq,fillstyle=solid,opacity=1](7.,4.){0.3}
\pscircle[linewidth=2.pt,linecolor=ffxfqq,fillcolor=ffxfqq,fillstyle=solid,opacity=1](7.,3.){0.3}
\pscircle[linewidth=2.pt,linecolor=ffxfqq,fillcolor=ffxfqq,fillstyle=solid,opacity=1](7.,2.){0.3}
\pscircle[linewidth=2.pt,linecolor=green,fillcolor=green,fillstyle=solid,opacity=1](5.,5.){0.3}
\pscircle[linewidth=2.pt,linecolor=green,fillcolor=green,fillstyle=solid,opacity=1](6.,5.){0.3}
\pscircle[linewidth=2.pt,linecolor=green,fillcolor=green,fillstyle=solid,opacity=1](7.,5.){0.3}
\pscircle[linewidth=2.pt,linecolor=green,fillcolor=green,fillstyle=solid,opacity=1](8.,5.){0.3}
\pscircle[linewidth=2.pt,linecolor=green,fillcolor=green,fillstyle=solid,opacity=1](8.,4.){0.3}
\pscircle[linewidth=2.pt,linecolor=green,fillcolor=green,fillstyle=solid,opacity=1](8.,3.){0.3}
\pscircle[linewidth=2.pt,linecolor=green,fillcolor=green,fillstyle=solid,opacity=1](8.,2.){0.3}
\Huge
\rput[tl](-3.44,0.93){$1+3+5+7$}
\rput[tl](4.4,0.93){$=4^2=16$}
\end{pspicture*}
\end{center}

On devine que $u_n$ sera toujours un carré~:

\begin{align*}
u_0&=1=1^2\\
u_1&=4=2^2\\
u_2&=9=3^2\\
u_3&=16=4^2
\end{align*}

Plus généralement, $u_n=(n+1)^2$ pour tout $n\in\mathbb{N}$ -- ce que l'on démontre rigoureusement dans la question suivante.

\item Pour tout $n\in\mathbb{N},$ on note $\mathcal{P}_n$ la propriété \[u_n=(n+1)^2.\]



\begin{itemize}
\item[{\textbullet}] \textbf{Initialisation.} On prouve que $\mathcal{P}_0$ est vraie.

\[
\left.
    \begin{array}{ll}
        u_0&=1 \\
        (0+1)^2&= 1
    \end{array}
\right \}\implies \mathcal{P}_0~\text{est vraie}.
\]



\item[{\textbullet}] \textbf{Hérédité.} Soit $k\in\mathbb{N}$ tel que $\mathcal{P}_k$ soit vraie. On a donc
\[u_k=(k+1)^2.\]

%\medskip

\newtcolorbox{mybox}[1]{colback=green!10!white,colframe=green!80!white,fonttitle=\bfseries,title=#1}
\begin{mybox}{Objectif}{Prouver que $\mathcal{P}_{k+1}$ est vraie, c'est-à-dire que \[u_{k+1}=((k+1)+1)^2,\] ou encore
\[u_{k+1}=(k+2)^2.\]
}\end{mybox}



%\medskip

On part de 
\[u_k=(k+1)^2.\]

On a alors~:


\begin{alignat*}{3}
&u_{k+1}&& =u_k+2k+3 && \text{  (rel. réc. pour } (u_n)_{n\in\mathbb{N}})\\
& && =(k+1)^2+2k+3 && \text{ {(H.R.)}}\\
& && =k^2+2k+1+2k+3 && \text{  (on développe avec l'IR)}\\
& && =k^2+4k+4 && \text{  (on réduit)}\\
& && =(k+2)^2 && \text{  (on factorise avec l'IR)}.\\
\end{alignat*}


La propriété $\mathcal{P}_{k+1}$ est donc vraie.
\item[{\textbullet}] \textbf{Conclusion.} $\mathcal{P}_0$ est vraie et $\mathcal{P}_n$ est héréditaire, donc elle est vraie pour tout $n\in\mathbb{N}.$
\end{itemize}

\end{enumerate}

\end{exo}



\begin{exo}

La suite $(u_n)_{n\in\mathbb{N}}$ est définie par $u_0=1$ et la relation de récurrence

\[u_{n+1}=0,5u_n+3\] pour tout $n\in\mathbb{N}.$

\begin{enumerate}
\item \begin{align*}
u_0&=1\\
u_1&=0,5\times 1+3=3,5\\
u_2&=0,5\times 3,5+3=4,75\\
u_3&=0,5\times 4,75+3=5,375
\end{align*}
\item Pour tout $n\in\mathbb{N},$ on note $\mathcal{P}_n$ la propriété \[u_n\leq 6.\]





\begin{itemize}
\item[{\textbullet}] \textbf{Initialisation.} On prouve que $\mathcal{P}_0$ est vraie.

\[
\left.
    \begin{array}{ll}
        u_0&=1 \\
        1&\leq 6
    \end{array}
\right \}\implies \mathcal{P}_0~\text{est vraie}.
\]



\item[{\textbullet}] \textbf{Hérédité.} Soit $k\in\mathbb{N}$ tel que $\mathcal{P}_k$ soit vraie. On a donc
\[u_k\leq 6.\]

%\medskip

\newtcolorbox{mybox}[1]{colback=green!10!white,colframe=green!80!white,fonttitle=\bfseries,title=#1}
\begin{mybox}{Objectif}{Prouver que $\mathcal{P}_{k+1}$ est vraie, c'est-à-dire que \[u_{k+1}\leq 6.\]
}\end{mybox}



%\medskip

On part de 
\[u_k\leq 6.\]


On multiplie par $\textcolor{red}{0,5}~:$

\begin{align*}u_k\textcolor{red}{\times 0,5}&\leq 6\textcolor{red}{\times 0,5}\\
0,5u_k&\leq 3
\end{align*}

Puis on ajoute  $\textcolor{blue}{3}~:$

\begin{align*}
0,5u_k\textcolor{blue}{+3}&\leq 3\textcolor{blue}{+3}\\
u_{k+1}&\leq 6.
\end{align*}

La propriété $\mathcal{P}_{k+1}$ est donc vraie.
\item[{\textbullet}] \textbf{Conclusion.} $\mathcal{P}_0$ est vraie et $\mathcal{P}_n$ est héréditaire, donc elle est vraie pour tout $n\in\mathbb{N}.$
\end{itemize}

\end{enumerate}

\end{exo}





\begin{exo}

La suite $(u_n)_{n\in\mathbb{N}}$ est définie par $u_0=1$ et la relation de récurrence

\[u_{n+1}=\dfrac{u_n}{u_n+1}\] pour tout $n\in\mathbb{N}.$

\begin{enumerate}
\item \begin{align*}
u_0&=1\\
u_1&=\dfrac{u_0}{u_0+1}=\dfrac{1}{1+1}=\dfrac{1}{2}\\
u_2&=\dfrac{u_1}{u_1+1}=\dfrac{\frac{1}{2}}{\frac{1}{2}+1}=\dfrac{\frac{1}{2}}{\frac{1}{2}+\frac{2}{2}}=\dfrac{\frac{1}{2}}{\frac{3}{2}}=\dfrac{1}{2}\times \dfrac{2}{3}=\dfrac{1}{3}\\
u_3&=\dfrac{u_2}{u_2+1}=\dfrac{\frac{1}{3}}{\frac{1}{3}+1}=\dfrac{\frac{1}{3}}{\frac{1}{3}+\frac{3}{3}}=\dfrac{\frac{1}{3}}{\frac{4}{3}}=\dfrac{1}{3}\times \dfrac{3}{4}=\dfrac{1}{4}
\end{align*}
\item Pour tout $n\in\mathbb{N},$ on note $\mathcal{P}_n$ la propriété \[u_n=\frac{1}{n+1}.\]






\begin{itemize}
\item[{\textbullet}] \textbf{Initialisation.} On prouve que $\mathcal{P}_0$ est vraie.

\[
\left.
    \begin{array}{ll}
        u_0&=1 \\
        \frac{1}{0+1}&=1
    \end{array}
\right \}\implies \mathcal{P}_0~\text{est vraie}.
\]



\item[{\textbullet}] \textbf{Hérédité.} Soit $k\in\mathbb{N}$ tel que $\mathcal{P}_k$ soit vraie. On a donc
\[u_k=\frac{1}{k+1}.\]


%\medskip

\newtcolorbox{mybox}[1]{colback=green!10!white,colframe=green!80!white,fonttitle=\bfseries,title=#1}
\begin{mybox}{Objectif}{Prouver que $\mathcal{P}_{k+1}$ est vraie, c'est-à-dire que \[u_{k+1}=\frac{1}{k+2}.\]
}\end{mybox}



%\medskip

On part de 
\[u_k=\frac{1}{k+1}.\]



On utilise la formule de récurrence et on remplace~:

\[u_{k+1}=\dfrac{u_k}{u_k+1}\overset{\text{H.R.}}{=}\dfrac{\frac{1}{k+1}}{\frac{1}{k+1}+1}=\dfrac{\frac{1}{k+1}}{\frac{1}{k+1}+\frac{k+1}{k+1}}=\dfrac{\frac{1}{k+1}}{\frac{k+2}{k+1}}=\dfrac{1}{\cancel{k+1}}\times \dfrac{\cancel{k+1}}{k+2}=\dfrac{1}{k+2}.\]



La propriété $\mathcal{P}_{k+1}$ est donc vraie.
\item[{\textbullet}] \textbf{Conclusion.} $\mathcal{P}_0$ est vraie et $\mathcal{P}_n$ est héréditaire, donc elle est vraie pour tout $n\in\mathbb{N}.$
\end{itemize}
\end{enumerate}

\end{exo}

\begin{exo}


Soit $q$ un réel différent de 1. Pour tout $n\in\mathbb{N},$ on note $\mathcal{P}_n$ la propriété 

\[1+q+q^2+\cdots+q^n=\frac{q^{n+1}-1}{q-1}.\]







\begin{itemize}
\item[{\textbullet}] \textbf{Initialisation.} On prouve que $\mathcal{P}_0$ est vraie.

La somme dans le membre de gauche va de 1 à $q^n,$ qui, dans le cas où $n=0,$ vaut 1. Autrement dit, la somme dans le membre de gauche est une somme d'un seul terme~: 1.

\medskip

D'un autre côté, $\frac{q^{0+1}-1}{q-1}=\frac{q-1}{q-1}=1.$ $\mathcal{P}_0$ est donc vraie.



\item[{\textbullet}] \textbf{Hérédité.} Soit $k\in\mathbb{N}$ tel que $\mathcal{P}_k$ soit vraie. On a donc
\[1+q+q^2+\cdots+q^k=\frac{q^{k+1}-1}{q-1}.\]

%\medskip

\newtcolorbox{mybox}[1]{colback=green!10!white,colframe=green!80!white,fonttitle=\bfseries,title=#1}
\begin{mybox}{Objectif}{Prouver que $\mathcal{P}_{k+1}$ est vraie, c'est-à-dire que 
\[1+q+q^2+\cdots+q^{k+1}=\frac{q^{k+2}-1}{q-1}.\]

}\end{mybox}



%\medskip

On part de 
\[1+q+q^2+\cdots+q^k=\frac{q^{k+1}-1}{q-1}.\]


On ajoute $\textcolor{blue}{q^{k+1}},$ puis on réduit au même dénominateur~:

\begin{align*}
1+q+q^2+\cdots+q^k\textcolor{blue}{+q^{k+1}}&=\frac{q^{k+1}-1}{q-1}\textcolor{blue}{+q^{k+1}}\\
&=\frac{q^{k+1}-1}{q-1}+\frac{q^{k+1}\textcolor{red}{\times (q-1)}}{\textcolor{red}{q-1}}
\\&=\frac{q^{k+1}-1}{q-1}+\frac{q^{k+1}\times q-q^{k+1}\times 1}{q-1}
\\&=\frac{q^{k+1}-1}{q-1}+\frac{q^{k+2}-q^{k+1}}{q-1}
\\&=\frac{\cancel{q^{k+1}}-1+q^{k+2}-\cancel{q^{k+1}}}{q-1}
\\&=\frac{q^{k+2}-1}{q-1}.
\end{align*}

\medskip

Conclusion~:
\[1+q+q^2+\cdots+q^{k+1}=\frac{q^{k+2}-1}{q-1},\] et la propriété $\mathcal{P}_{k+1}$ est donc vraie.
\item[{\textbullet}] \textbf{Conclusion.} $\mathcal{P}_0$ est vraie et $\mathcal{P}_n$ est héréditaire, donc elle est vraie pour tout $n\in\mathbb{N}.$
\end{itemize}

\medskip

\textbf{Remarque~:} La somme $1+q+q^2+\cdots+q^{n}$ se réécrit sous la forme condensée
\[\sum\limits_{i=0}^nq^i.\]

\end{exo}


\begin{exo}

Soit $x$ un réel positif.

Pour tout $n\in\mathbb{N}^*,$ on note $\mathcal{P}_n$ la propriété 

\[(1+x)^n\geq 1+nx.\]


\begin{itemize}
\item[{\textbullet}] \textbf{Initialisation.} On prouve que $\mathcal{P}_1$ est vraie (\danger Ça démarre à $n=1$ et non pas $n=0.$)


\[
\left.
    \begin{array}{ll}
        (1+x)^1&=1+x \\
        1+1 x&=1+x
    \end{array}
\right \}\implies (1+x)^1\geq 1+1x \implies \mathcal{P}_1~\text{est vraie}.
\]



\item[{\textbullet}] \textbf{Hérédité.} Soit $k\in\mathbb{N}^*$ tel que $\mathcal{P}_k$ soit vraie. On a donc
\[(1+x)^k\geq 1+kx.\]


\newtcolorbox{mybox}[1]{colback=green!10!white,colframe=green!80!white,fonttitle=\bfseries,title=#1}
\begin{mybox}{Objectif}{Prouver que $\mathcal{P}_{k+1}$ est vraie, c'est-à-dire que 
\[(1+x)^{k+1}\geq 1+(k+1)x.\]

}\end{mybox}



%\medskip

On part de 
\[(1+x)^k\geq 1+kx.\]


On multiplie par $\textcolor{red}{1+x}$ et on développe~:

\begin{align*}
(1+x)^k\textcolor{red}{\times (1+x)}&\geq (1+kx)\textcolor{red}{\times (1+x)}\\
(1+x)^{k+1}&\geq 1\times 1+1\times x+kx\times 1+kx\times x\\
(1+x)^{k+1}&\geq 1+ \textcolor{green}{x+kx}+kx^2\\
(1+x)^{k+1}&\geq 1+ \textcolor{green}{(k+1)x}+kx^2\\
\end{align*}

\medskip

Or $kx^2\geq 0,$ car $k$ et $x^2$ sont positifs, donc $1+ (k+1)x+kx^2\geq 1+(k+1)x~;$ et par conséquent

\[(1+x)^{k+1}\geq 1+(k+1)x.\]

La propriété $\mathcal{P}_{k+1}$ est donc vraie.
\item[{\textbullet}] \textbf{Conclusion.} $\mathcal{P}_1$ est vraie et $\mathcal{P}_n$ est héréditaire, donc elle est vraie pour tout $n\in\mathbb{N}^*.$
\end{itemize}


\end{exo}





\begin{exo}

La suite $(u_n)_{n\in\mathbb{N}}$ est définie par $u_0=1$ et la relation de récurrence \[u_{n+1}=\frac{4u_n}{u_n+4}\] pour tout $n\in\mathbb{N}.$ On pose également $v_n=\frac{4}{u_n}$ pour tout $n\in\mathbb{N}.$


\begin{enumerate}
\item Pour démontrer qu'une suite est \textcolor{red}{géométrique}, on part de $v_{n+1}=\cdots$ et on essaye d'aboutir à $\cdots =v_n\textcolor{red}{\times q}.$ Pour une suite \textcolor{blue}{arithmétique}, c'est le même principe~: on part de $v_{n+1}=\cdots$ et on essaye d'aboutir à $\cdots =v_n \textcolor{blue}{+r}.$

\medskip

Pour tout $n\in\mathbb{N}~:$

\[
v_{n+1}=\dfrac{4}{u_{n+1}}=\dfrac{4}{\frac{4u_n}{u_n+4}}=4\times \dfrac{u_n+4}{4u_n}=\dfrac{\cancel{4}\left(u_n+4\right)}{\cancel{4}u_n}=\dfrac{u_n}{u_n}+\dfrac{4}{u_n}=1+v_n.
\]


Conclusion~: pour tout $n\in\mathbb{N},$ $v_{n+1}=v_n+1,$
donc $(v_n)_{n\in\mathbb{N}}$ est arithmétique de raison $r=1.$



\item La suite $(v_n)_{n\in\mathbb{N}}$ est arithmétique de raison $r=1,$ et  $v_0=\frac{4}{u_0}=\frac{4}{1}=4,$ donc pour tout $n\in\mathbb{N}~:$
\[v_n=v_0+n\times r=4+n\times 1=n+4.\]


Enfin $v_n=\dfrac{4}{u_{n}}$ donc
\[u_n=\dfrac{4}{v_{n}}=\dfrac{4}{n+4}.\]

\medskip

\textbf{Remarque~:} On a utilisé~: si $a=\dfrac{b}{c},$ alors $c=\dfrac{b}{a}.$
\end{enumerate}

\end{exo}





\begin{exo}

\begin{enumerate}
\item \begin{enumerate}
\item Chaque année, 80~\% des abonnés se réabonnent (multiplication par $0,8$), puis 40  nouvelles personnes s'inscrivent, donc

\begin{align*}
u_0&=500\\
u_1&=500\times 0,8+40=440\\
u_2&=440\times 0,8+40=392
\end{align*}



Conclusion~: $u_1=440$ et  $u_2=392.$
\item La formule de récurrence est $u_{n+1}=u_n\times 0,8+40,$ ou encore
\[u_{n+1}=0,8u_n+40.\]
\item Avec un schéma~:


~{}\begin{center}
    $\xymatrix@R=0.5pc@C=3pc{
    *+[F]+{500} \ar@/^0.5cm/[r]|{\red{-60}} \ar@/_0.5cm/[r]|{\green{\times 0,88}} & 
    *+[F]+{440} \ar@/^0.5cm/[r]|{\red{-48}} \ar@/_0.5cm/[r]|{\green{\times \approx 0,89}} & *+[F]+{392} \\
    \txt{\blue{$u_0$}}&
    \txt{\blue{$u_1$}}&\txt{\blue{$u_2$}}    
    }$
    \end{center}
    
    
    \medskip
    
    Les résultats en rouge (\textcolor{red}{$-6$} et \textcolor{red}{$-48$}) sont différents, donc $u$ n'est pas arithmétique.  Les résultats en vert (\textcolor{green}{$0,88$} et \textcolor{green}{$0,89$}) sont différents, donc $u$ n'est pas géométrique.
    
\end{enumerate}
\item 
On pose $v_n=u_n -200$ pour tout $n\in\mathbb{N}.$
\begin{enumerate}
\item Pour tout $n\in\mathbb{N}~:$



\begin{alignat*}{3}
&v_{n+1}&& =u_{n+1}-200 && \text{  (déf. de } (v_n)_{n\in\mathbb{N}})\\
& && =(0,8u_n+40)-200 && \text{  (rel. réc. pour } (u_n)_{n\in\mathbb{N}})\\
& && =0,8u_n-160 && \text{  (calcul)}\\
& && =0,8\left(u_n-\frac{160}{0,8}\right) && \text{  (factorisation)}\\
& && =0,8(u_n-200) && \text{  (calcul)}\\
& && =0,8v_n&& \text{  (déf. de } (v_n)_{n\in\mathbb{N}})\\
\end{alignat*}

\medskip

Conclusion~: pour tout $n\in\mathbb{N},$ $v_{n+1}=0,8v_n,$
donc $(v_n)_{n\in\mathbb{N}}$ est géométrique de raison $q=0,8.$



\item La suite $(v_n)_{n\in\mathbb{N}}$ est géométrique de raison $q=0,8,$ et  $v_0=u_0-200=500-200=300,$ donc pour tout $n\in\mathbb{N}~:$
\[v_n=v_0\times q^n=300\times 0,8^n.\]

Enfin $v_n=u_n-200$ donc
\[u_n=v_n+200=300\times 0,8^n+200.\]
\item Suivant ce modèle, en 2030 (donc après 10 ans), il devrait y avoir
\[u_{10}=300\times 0,8^{10}+200\approx 232~\text{abonnés}.\]
\end{enumerate}
\item Pour tout $n\in\mathbb{N},$ on note $\mathcal{P}_n$ la propriété
\[u_n=300\times 0,8^n+200.\]






\begin{itemize}
\item[{\textbullet}] \textbf{Initialisation.} On prouve que $\mathcal{P}_0$ est vraie.

\[
\left.
    \begin{array}{ll}
        u_0&=500 \\
        300\times 0,8^0+200&= 300\times 1+200=500
    \end{array}
\right \}\implies \mathcal{P}_0~\text{est vraie}.
\]



\item[{\textbullet}] \textbf{Hérédité.} Soit $k\in\mathbb{N}$ tel que $\mathcal{P}_k$ soit vraie. On a donc
\[u_k=300\times 0,8^k+200.\]

%\medskip

\newtcolorbox{mybox}[1]{colback=green!10!white,colframe=green!80!white,fonttitle=\bfseries,title=#1}
\begin{mybox}{Objectif}{Prouver que $\mathcal{P}_{k+1}$ est vraie, c'est-à-dire que \[u_{k+1}=300\times 0,8^{k+1}+200.\]

}\end{mybox}



%\medskip

On part de 
\[u_k=300\times 0,8^k+200.\]


On a alors~:


\begin{alignat*}{3}
&u_{k+1}&& =0,8u_k+40 && \text{  (rel. réc. pour } (u_n)_{n\in\mathbb{N}})\\
& && =0,8\left(300\times 0,8^k+200\right)+40 && \text{ {(H.R.)}}\\
& && =300\times 0,8\times 0,8^k+0,8\times 200+40 && \text{  (on développe)}\\
& && =300\times 0,8^{k+1}+200 && \text{  (calcul)}.\\
\end{alignat*}



La propriété $\mathcal{P}_{k+1}$ est donc vraie.
\item[{\textbullet}] \textbf{Conclusion.} $\mathcal{P}_0$ est vraie et $\mathcal{P}_n$ est héréditaire, donc elle est vraie pour tout $n\in\mathbb{N}.$
\end{itemize}
\end{enumerate}

\end{exo}



\begin{exo}

On définit deux suites $\left(u_n\right)_{n\in\mathbb{N}}$ et $\left(v_n\right)_{n\in\mathbb{N}}$ par $u_0=0,$ $v_0=8$ et les relations de récurrence~:

\[\begin{cases}
u_{n+1}&=\frac{3}{4}u_n+\frac{1}{4}v_n\\
v_{n+1}&=\frac{1}{4}u_n+\frac{3}{4}v_n\end{cases}\] pour tout $n\in\mathbb{N}.$

\begin{enumerate}
\item \setlength{\columnseprule}{1pt}

\begin{multicols}{2}
\begin{align*}
u_0&=0\\
u_1&=\frac{3}{4}\times u_0+\frac{1}{4}\times v_0=\frac{3}{4}\times 0+\frac{1}{4}\times 8=2\\
u_2&=\frac{3}{4}\times u_1+\frac{1}{4}\times v_1=\frac{3}{4}\times 2+\frac{1}{4}\times 6=3
\end{align*}

\begin{align*}
v_0&=8\\
v_1&=\frac{1}{4}\times u_0+\frac{3}{4}\times v_0=\frac{1}{4}\times 0+\frac{3}{4}\times 8=6\\
v_2&=\frac{1}{4}\times u_1+\frac{3}{4}\times v_1=\frac{1}{4}\times 2+\frac{3}{4}\times 6=5
\end{align*}
\end{multicols}

\medskip


\begin{center}
\newrgbcolor{ududff}{0.30196078431372547 0.30196078431372547 1.}
\psset{xunit=1.0cm,yunit=1.0cm,algebraic=true,dimen=middle,dotstyle=o,dotsize=5pt 0,linewidth=2.pt,arrowsize=3pt 2,arrowinset=0.25}
\begin{pspicture*}(-0.74,-0.82)(8.84,0.88)
\psaxes[labelFontSize=\scriptstyle,xAxis=true,yAxis=false,Dx=1.,Dy=1.,ticksize=-2pt 0,subticks=2]{->}(0,0)(-0.74,-0.82)(8.84,0.88)
\rput[tl](-0.3,0.64){\ududff{$u_0$}}
\rput[tl](1.7,0.64){\ududff{$u_1$}}
\rput[tl](2.7,0.64){\ududff{$u_2$}}
\rput[tl](4.7,0.64){\red{$v_2$}}
\rput[tl](5.7,0.64){\red{$v_1$}}
\rput[tl](7.7,0.64){\red{$v_0$}}
\psdots[dotstyle=*,linecolor=ududff](0.,0.)
\psdots[dotstyle=*,linecolor=ududff](2.,0.)
\psdots[dotstyle=*,linecolor=ududff](3.,0.)
\psdots[dotstyle=*,linecolor=red](5.,0.)
\psdots[dotstyle=*,linecolor=red](6.,0.)
\psdots[dotstyle=*,linecolor=red](8.,0.)
\end{pspicture*}
\end{center}

\medskip

\textbf{Remarque~:} Pour placer $u_{n+1}$ et $v_{n+1},$ on coupe le segment $\left[u_n;v_n\right]$ en $4~;$ et on place $u_{n+1}$ au quart du segment, $v_{n+1}$ aux trois-quarts du segment.



\item On pose $s_n=v_n+u_n$ et $d_n=v_n-u_n$ pour tout $n\in\mathbb{N}.$

\begin{itemize}
\item[\textbullet] Pour tout $n\in\mathbb{N}~:$
\begin{align*}s_{n+1}&=v_{n+1}+u_{n+1}\\
&=\left(\frac{1}{4}u_n+\frac{3}{4}v_n\right)+\left(\frac{3}{4}u_n+\frac{1}{4}v_n\right)\\
&=\frac{4}{4}v_n+\frac{4}{4}u_n\\
&=v_n+u_n\\
&=s_n.\end{align*}

\medskip

Conclusion~: pour tout $n\in\mathbb{N},$ $s_{n+1}=s_n$ donc $\left(s_n\right)_{n\in\mathbb{N}}$ est constante. Et comme $s_0=v_0+u_0=8+0=8,$ $\left(s_n\right)_{n\in\mathbb{N}}$ est constante égale à 8~:
\[\text{pour tout }n\in\mathbb{N},~s_n=8.\]
\item[\textbullet] Pour tout $n\in\mathbb{N}~:$
\begin{align*}d_{n+1}&=v_{n+1}-u_{n+1}\\
&=\left(\frac{1}{4}u_n+\frac{3}{4}v_n\right)-\left(\frac{3}{4}u_n+\frac{1}{4}v_n\right)\\
&=\frac{2}{4}v_n-\frac{2}{4}u_n\\
&=\frac{1}{2}\left(v_n-u_n\right)\\
&=\frac{1}{2}d_n
.\end{align*}

\medskip

Conclusion~: pour tout $n\in\mathbb{N},$ $d_{n+1}=\frac{1}{2}d_n$ donc $\left(d_n\right)_{n\in\mathbb{N}}$ est géométrique de raison $q=\frac{1}{2}.$
\end{itemize}
\item  La suite $(d_n)_{n\in\mathbb{N}}$ est géométrique de raison $q=\frac{1}{2},$ et  $d_0=v_0-u_0=8-0=8,$ donc pour tout $n\in\mathbb{N}~:$
\[d_n=d_0\times q^n=8\times \left(\frac{1}{2}\right)^n.\] On sait par ailleurs que $s_n=8$ pour tout $n\in\mathbb{N}.$

\medskip
Les relations

\[\begin{cases}
s_n&=v_n+u_n\\
d_n&=v_n-u_n\end{cases}\]

se réécrivent donc

\[\begin{cases}
8&=v_n+u_n\\
8\times \left(\frac{1}{2}\right)^n&=v_n-u_n\end{cases}\]

On ajoute membre à membre~:

\begin{align*}
8+8\times \left(\frac{1}{2}\right)^n&=v_n+\cancel{u_n}+v_n-\cancel{u_n}\\
8+8\times \left(\frac{1}{2}\right)^n&=2v_n\\
\frac{8+8\times \left(\frac{1}{2}\right)^n}{2}&=v_n\\
4+4\times \left(\frac{1}{2}\right)^n&=v_n
\end{align*}

\medskip

Enfin, comme $s_n=v_n+u_n~:$
\[u_n=s_n-v_n=8-\left(4+4\times \left(\frac{1}{2}\right)^n\right)=8-4-4\times \left(\frac{1}{2}\right)^n=4-4\times \left(\frac{1}{2}\right)^n.\]

\medskip

Conclusion~: pour tout $n\in\mathbb{N},$

\[\boxed{v_n=4+4\times \left(\frac{1}{2}\right)^n}\qquad\qquad \boxed{u_n=4-4\times \left(\frac{1}{2}\right)^n}\]




\end{enumerate}


\end{exo}

\begin{exo}

\item \setlength{\columnseprule}{1pt}

\begin{multicols}{3}

\begin{center}
\textbf{Programme}
\end{center}

\vspace*{0.2cm}


\begin{lstlisting}
for i in range(1,6):
	print(i**2)
\end{lstlisting}



\vspace*{2cm}


\columnbreak


\begin{center}
\textbf{Traduction en français}
\end{center}


\newtcolorbox{mybox}[1]{colback=white,colframe=black!80!white,fonttitle=\bfseries,title=#1}\begin{mybox}{}

$\text{Pour i allant de 1 à 5:}$

$\qquad\text{afficher }\text{i}^2$
\end{mybox}

\vspace*{4cm}


\columnbreak

\begin{center}
\textbf{Commentaires}
\end{center}

\begin{itemize}
\item[\textbullet]  \danger La commande \[\text{for i in range(1,6)}\] signifie que i va de 1 à 5 -- il y a un décalage à la fin.
\item[\textbullet] On n’oublie pas les \og  : \fg~{} à la fin de la première ligne. L'incrément qui suit (équivalent à une tabulation sur Thonny) est alors automatiquement inséré lorsqu'on passe à la ligne.
\end{itemize}

\end{multicols}



\end{exo}

\begin{exo}
~{}

\setlength{\columnseprule}{1pt}

\begin{multicols}{2}

\begin{center}
\textbf{Programme}
\end{center}

\begin{lstlisting}
for i in range(1,11):
	print(8*i)
\end{lstlisting}








\columnbreak

\begin{center}
\textbf{Commentaires}
\end{center}

On affiche les résultats les uns en-dessous des autres~:

\[8\times 1=8,~8\times 2=16,~8\times 3=24,~\dots,~8\times 10=80.\]

\end{multicols}

\end{exo}

%\newpage

\begin{exo}

On explique le fonctionnement du programme en remplissant un tableau.


\setlength{\columnseprule}{1pt}

\begin{multicols}{3}

\begin{center}
\textbf{Programme}
\end{center}

\begin{lstlisting}
s=0
for i in range(1,101):
	s=s+i
print(s)
\end{lstlisting}



\vspace*{6cm}


\columnbreak

\begin{center}
\textbf{Tableau}
\end{center}


On a une boucle \textbf{Pour}, où i va de 1 à 100.

\medskip

\begin{center}


\begin{tabular}{|c|c|} \hline
\textbf{Valeur de i}& \textbf{Valeur de s}\\ \hline
\cellcolor{gray}& 0\\ \hline
1& $0+1=1$\\ \hline
2&$1+2=3$\\ \hline
3&$3+3=6$\\ \hline
4&$6+4=10$ \\ \hline
$\cdots$&$\cdots$\\ \hline
99&$\cdots$\\ \hline
100&$\cdots$\\ \hline
\end{tabular}
\end{center}
\vspace*{3.5cm}


\columnbreak

\begin{center}
\textbf{Explications}
\end{center}

\begin{itemize}
\item[\textbullet] La 1\up{re} ligne du tableau ci-contre correspond à la 1\up{re} ligne du code~: $\text{s}=0$ et i n'existe pas encore.
\item[\textbullet] Dans la boucle, i va de $1$ à $100,$ donc on écrit les valeurs de 1 jusqu'à 100 dans la 1\up{re} colonne.
\item[\textbullet] À chaque étape de la boucle \textbf{Pour}, s reçoit la valeur $\text{s}+\text{i}.$

 Donc au début, lorsque $\text{i}=1,$ s reçoit la valeur $\text{s}+\text{i}=0+1=1.$ La valeur de s a donc été modifiée et il vaut maintenant 1 (et non plus 0).
\item[\textbullet] Ensuite, lorsque $\text{i}=2,$ s reçoit la nouvelle valeur $\text{s}+\text{i}=1+2=3.$ La valeur de s a été une nouvelle fois modifiée.
\item[\textbullet] Puis quand $\text{i}=3,$ s reçoit la nouvelle valeur $\text{s}+\text{i}=3+3=6.$ Cela continue ainsi de suite jusqu'en bas du tableau.
\end{itemize}
\end{multicols}

\medskip

Finalement, on part de 0, puis on ajoute 1, puis 2, puis 3~; et ainsi de suite jusque 100. On calcule donc
\[1+2+3+\cdots+ 99+100.\] Le résultat, \np{5050}, s'affiche en fin de programme.

\end{exo}


\begin{exo}

On édite en machine un programme Python qui calcule~:
\[10!=1\times 2\times 3\times \cdots\times 10.\]

On s'inspire pour cela du programme précédent, avec trois différences~:

\begin{itemize}
\item[\textbullet] on \textbf{multiplie} par i à chaque étape, au lieu \textbf{d'ajouter} i~;
\item[\textbullet] au début $\text{s}=1,$ élément neutre de la multiplication (si on démarrait avec $\text{s}=0,$ la valeur de s vaudrait toujours 0)~;
\item[\textbullet] la boucle ne va que de 1 à 10.
\end{itemize}

\medskip

\begin{lstlisting}
s=1
for i in range(1,11):
	s=s*i
print(s)
\end{lstlisting}


\end{exo}

%\newpage

\begin{exo}

La suite $(u_n)_{n\in\mathbb{N}}$ est définie par $u_0=3$ et la formule de récurrence
\[u_{n+1}=2u_n-1\] pour tout $n\in\mathbb{N}.$

\medskip

Nous allons calculer les premiers termes, puis, comme dans l'exercice 39, compléter un tableau avec les calculs à chaque étape de la boucle \textbf{Pour}.


\setlength{\columnseprule}{1pt}

\begin{multicols}{3}

\begin{center}
\textbf{Calcul des premiers termes}
\end{center}

\medskip

\begin{align*}
u_0&=3\\
u_1&=2\times 3-1=5\\
u_2&=2\times 5-1=9\\
u_3&=2\times 9-1=17\\
u_4&=2\times 17-1=33
\end{align*}

\vspace*{0.cm}

\columnbreak
\begin{center}
\textbf{Programme}
\end{center}

\begin{lstlisting}
u=3
for i in range(4):
	u=2*u-1
print(u)
\end{lstlisting}



\vspace*{1.cm}


\columnbreak



\begin{center}
\textbf{Tableau}
\end{center}


On a une boucle \textbf{Pour}, où i va de 0 à 3.\footnote{La commande \[\text{for i in range(n):}\] signifie que i va de 0 à $\text{n}-1.$}

\medskip

\begin{center}


\begin{tabular}{|c|cc|} \hline
\textbf{Valeur de i}& \textbf{Valeur de u}&\\ \hline
\cellcolor{gray}& 3&\textcolor{red}{$\leftarrow {u_0}$}\\ \hline
0& $2\times 3-1=5$&\textcolor{red}{$\leftarrow {u_1}$}\\ \hline
1&$2\times 5-1=9$&\textcolor{red}{$\leftarrow {u_2}$}\\ \hline
2& $2\times 9-1=17$&\textcolor{red}{$\leftarrow {u_3}$}\\ \hline
3&$2\times 17-1=33$&\textcolor{red}{$\leftarrow {u_4}$} \\ \hline

\end{tabular}
\end{center}
%\vspace*{3.5cm}


\end{multicols}


\medskip

Dans la boucle \textbf{Pour}, u prend successivement les valeurs $u_0,$ $u_1,$ $u_2,$ $u_3$ et $u_4.$ C'était prévisible, puisque l'instruction \begin{lstlisting}
u=2*u-1
\end{lstlisting} est la même que la formule de récurrence.

\medskip

Conclusion~: la valeur affichée en sortie est $u_4=33.$

\end{exo}

\begin{exo}

Commençons par des rappels concernant les listes, avec quelques exemples~:

\begin{itemize}
\item[\textbullet] La commande \[\text{L}=\left[5,6,10\right]\] crée une liste L de trois éléments. Le premier, $\text{L}\left[0\right],$ est égal à 5~; le deuxième, $\text{L}\left[1\right],$ est égal à 6~; le troisième, $\text{L}\left[2\right],$ est égal à 10. On notera en particulier l'indexation des termes  à partir de $0.$
\item[\textbullet] La commande \[\text{L.append(2)}\] ajoute un terme à la liste, égal à 2. On aura donc ensuite une liste de 4 éléments~: $\text{L}=\left[5,6,10,2\right].$
\item[\textbullet] La commande \[\text{len(L)}\] renvoie la longueur de la liste.
\item[\textbullet] La commande \[\text{L}=\left[~\right]\] crée une liste vide (donc de longueur 0).
\end{itemize}

\medskip

Venons-en à l'exercice. On souhaite afficher la liste des nombres de la table de 8~:
\[\left[8~,~16~,~24~,~\cdots~,~80\right].\]

Pour cela, on crée une liste vide L, puis on reprend le programme de l'exercice 38, en ajoutant les nombres de la table à la liste L au fur et à mesure de leur calcul~:

\medskip

\begin{lstlisting}
L=[]
for i in range(1,11):
	L.append(8*i)
print(L)
\end{lstlisting}



\end{exo}



\begin{exo}

On reprend la suite de l'exercice 41~: $u_0=3$ et $u_{n+1}=2u_n-1$ pour tout $n\in\mathbb{N}.$

\medskip

Pour afficher la liste des termes de $u_0$ à $u_6,$ on crée d'abord une liste qui contient le premier terme avec la commande  \begin{lstlisting}
L=[3]
\end{lstlisting} Ensuite, on reprend le programme de l'exercice 41, en ajoutant à la liste L chacun des termes de la suite au fur et à mesure de leur calcul.

\medskip

Pour plus de clarté, on a ajouté un tableau explicatif~:

\medskip

\setlength{\columnseprule}{1pt}

\begin{multicols}{2}


\begin{center}
\textbf{Programme}
\end{center}

\begin{lstlisting}
u=3
L=[3]
for i in range(6):
	u=2*u-1
	L.append(u)
print(u)
\end{lstlisting}



\vspace*{0.25cm}


\columnbreak



\begin{center}
\textbf{Tableau}
\end{center}



\begin{center}


\begin{tabular}{|c|c|l|} \hline
\textbf{Valeur de i}& \textbf{Valeur de u}&\text{liste L}\\ \hline
\cellcolor{gray}& 3&$\text{L}=\left[3\right]$\\ \hline
0& $2\times 3-1=5$ & $\text{L}=\left[3,5\right]$ \\ \hline
1&$2\times 5-1=9$&$\text{L}=\left[3,5,9\right]$\\ \hline
2& $2\times 9-1=17$&$\text{L}=\left[3,5,9,17\right]$\\ \hline
3&$2\times 17-1=33$&$\text{L}=\left[3,5,9,17,33\right]$ \\ \hline
4&$2\times 33-1=65$&$\text{L}=\left[3,5,9,17,33,65\right]$ \\ \hline
5&$2\times 65-1=129$&$\text{L}=\left[3,5,9,17,33,65,129\right]$ \\ \hline
\end{tabular}
\end{center}
%\vspace*{3.5cm}


\end{multicols}

\end{exo}

\begin{exo}

~{}

\setlength{\columnseprule}{1pt}

\begin{multicols}{2}

\begin{center}\textbf{Programme 1}\end{center}

\begin{lstlisting}
x=3
if x==4:
	print(5*x)
else:
	print(2*x)
\end{lstlisting}

\medskip

Puisque $\text{x}\not=4,$ le programme affiche
\[2\times \text{x}=2\times 3=6.\]

\begin{center}\textbf{Programme 2}\end{center}

\begin{lstlisting}
x=3
if x<=4:
	print(5*x)
else:
	print(2*x)
\end{lstlisting}

\medskip

Puisque $\text{x}\leq 4,$ le programme affiche
\[5\times \text{x}=5\times 3=15.\]

\end{multicols}

\end{exo}

\begin{exo}



On commence par deux remarques~:

\begin{itemize}
\item[\textbullet] Un entier $\text{n}\geq 1$ est un diviseur de 30 si, et seulement si, $30\%\text{n}=0.$
\item[\textbullet] En python, on teste les égalités avec $==.$ Par exemple, la commande \begin{lstlisting}
4==4
\end{lstlisting} renvoie \textbf{True}~; tandis que
\begin{lstlisting}
4==5
\end{lstlisting} renvoie \textbf{False}.
\end{itemize}

\medskip

On édite un programme Python qui renvoie la liste des diviseurs positifs de 30~:

\medskip

\begin{lstlisting}
L=[]
for i in range(1,31):
	if 30%i==0:
		L.append(i)
print(L)
\end{lstlisting}



\end{exo}

\begin{exo}

On édite la fonction~:

\begin{lstlisting}
def f(x):
	return x**2
\end{lstlisting}



\medskip

On obtient \begin{align*}
\text{f}(3)&=3^2=9\\
\text{f}(-2)&=(-2)^2=4
\end{align*}


\end{exo}

\begin{exo}

La fonction

\begin{lstlisting}
def g():
	return 5
\end{lstlisting}

\medskip

renvoie toujours la valeur 5.

\medskip

\textbf{Remarque~:} Pour lancer la fonction, il faut entrer la commande \begin{lstlisting}
g()
\end{lstlisting} sans oublier les parenthèses, mais sans rien écrire à l'intérieur\footnote{Cela fait une différence notable avec les mathématiques,  où une fonction dépend forcément d'une (ou plusieurs) variables.}.


\end{exo}



\begin{exo}

~{}

\begin{lstlisting}
def moyenne(a,b):
	return (a+b)/2
\end{lstlisting}


 \end{exo}
 
 
 \begin{exo}
 
 ~{}
 
 \begin{lstlisting}
def transforme(note):
	x=1.2*note
	if x<=20:
		return x
	else:
		return 20
\end{lstlisting}


 
 \end{exo}
 
 \begin{exo}

On reprend encore la suite définie par  $u_0=3$ et $u_{n+1}=2u_n-1$ pour tout $n\in\mathbb{N}.$

\medskip

On recopie quasiment à l'identique les programmes que nous avons écrits dans les exercices 41 et 43. Il y a tout de même trois différences~:

\begin{itemize}
\item[\textbullet] on utilise une fonction~;
\item[\textbullet] la boucle \textbf{Pour} a n étapes, et non plus 4 (ex 41) ou  6 (ex 43)~;
\item[\textbullet] on utilise \textbf{return} au lieu de \textbf{print} pour renvoyer le résultat.
\end{itemize}

\begin{enumerate}
\item Fonction qui renvoie la valeur de $u_{\text{n}}~:$

\begin{lstlisting}
def terme(n):
	u=3
	for i in range(n):
		u=2*u-1
	return u
\end{lstlisting}


 \item Fonction qui renvoie la liste de tous les termes de $u_0$ à $u_{\text{n}}~:$
 
 
 \begin{lstlisting}
def liste(n):
	u=3
	L=[3]
	for i in range(n):
		u=2*u-1
		L.append(u)
	return L
\end{lstlisting}

 
 \end{enumerate}

\end{exo}


\begin{exo}

On explique le fonctionnement du programme avec un tableau~:

\medskip

\setlength{\columnseprule}{1pt}

\begin{multicols}{2}

\begin{center}
\textbf{Programme}
\end{center}

\begin{lstlisting}
def somme(n):
	s=0
	for k in range(1,n+1):
		s=s+1/k
	return s
\end{lstlisting}



\vspace*{3cm}


\columnbreak

\begin{center}
\textbf{Tableau}
\end{center}

Avec la commande somme(100), k va de 1 à 100, puisque $\text{n}+1=100+1=101.$

\medskip

On laisse volontairement les résultats sous forme de sommes de fractions.

\medskip

\begin{center}

\renewcommand{\arraystretch}{1.25}

\begin{tabular}{|c|c|} \hline
\textbf{Valeur de k}& \textbf{Valeur de s}\\ \hline
\cellcolor{gray}& 0\\ \hline
1& $0+\frac{1}{1}=\frac{1}{1}$\\ \hline
2&$\frac{1}{1}+\frac{1}{2}$\\ \hline
3&$\frac{1}{1}+\frac{1}{2}+\frac{1}{3}$\\ \hline
$\cdots$&$\cdots$\\ \hline
100&$\frac{1}{1}+\frac{1}{2}+\frac{1}{3}+\cdots+\frac{1}{100}$\\ \hline
\end{tabular}
\end{center}


\end{multicols}

\medskip

Conclusion~: somme(100) renvoie la valeur de la somme
\[\frac{1}{1}+\frac{1}{2}+\frac{1}{3}+\cdots+\frac{1}{100}\] 
(qui vaut environ $5,19$).




\end{exo}

\begin{exo}


~{}

\begin{lstlisting}
def mystere(L):
	M=L[0]
	for i in range(len(L)):
		if L[i]>M:
			M=L[i]
	return M
\end{lstlisting}




On explique encore une fois le programme avec un tableau~:

\medskip

\item \setlength{\columnseprule}{1pt}

\begin{multicols}{2}




\begin{center}
\textbf{Commentaires}
\end{center}


\medskip

On entre la commande mystere([2, 3, 7,0]). Dans ce cas, en notant $\text{L}=\left[2,3,7,0\right]~:$

\medskip

\begin{itemize}
\item[\textbullet] la longueur de la liste L est 4, donc $\text{len(L)}=4~;$ et i va de 0 à 3~;
\item[\textbullet] $\text{L}\left[0\right]=2,$ $\text{L}\left[1\right]=3,$ $\text{L}\left[2\right]=7$ et $\text{L}\left[3\right]=0~;$
\item[\textbullet] Au départ, $\text{M}=\text{L}\left[0\right]=2.$ Puis, à chaque étape de la boucle, on regarde si $\text{L}\left[\text{i}\right]>\text{M}.$ Si c'est le cas, on remplace M par $\text{L}\left[\text{i}\right].$
\end{itemize}



\columnbreak

\begin{center}
\textbf{Tableau}
\end{center}

\begin{center}

%\renewcommand{\arraystretch}{1.25}


\begin{tabular}{|c|cc|c|} \hline
\textbf{Valeur de i}&\textbf{A-t-on L[i]>M~?}& & \textbf{Valeur de M}\\ \hline
\cellcolor{gray}&\cellcolor{gray}&\cellcolor{gray} &$2$\\ \hline
0&  A-t-on $\text{L}\left[0\right]>2~?$ $2>2~?$ &\textcolor{blue}{non}&2\\ \hline
1& A-t-on $\text{L}\left[1\right]>2~?$ $3>2~?$ &\textcolor{red}{oui}&$3$\\ \hline
2& A-t-on $\text{L}\left[2\right]>3~?$ $7>3~?$ &\textcolor{red}{oui}&$7$\\ \hline
3& A-t-on $\text{L}\left[3\right]>7~?$ $0>7~?$ &\textcolor{blue}{non}&$7$\\ \hline
\end{tabular}
\end{center}

\vspace*{1cm}



\end{multicols}

\medskip

Conclusion~: la valeur renvoyée en sortie est 7, maximum de la liste L. Notez que l'on aurait pu obtenir ce maximum avec la simple commande

\begin{lstlisting}
max(L)
\end{lstlisting}



\end{exo}


\begin{exo}


\begin{enumerate}
\item Les termes successifs sont~:
\[26 - 13 - 40 - 20 - 10 -5 -16 - 8 - 4 - 2 - 1 - 4 - 2 - 1- \cdots\]

On aboutit à une suite périodique~; phénomène qui semble d'ailleurs avoir lieu quel que soit le nombre de départ (on invite le lecteur curieux à faire des essais avec d'autres entiers et à lire l'article Wikipédia sur la conjecture de Syracuse).
\item Avec Thonny, un entier n est pair si, et seulement si, $\text{n}\%2=0.$ On utilise ce résultat pour tester la parité~:

\medskip

\begin{lstlisting}
def syracuse():
	u=26
	L=[26]
	for i in range(10):
		if n%2==0:
			u=u/2
		else:
			u=3*u+1
		L.append(u)
	return L
\end{lstlisting}


 \end{enumerate}

\end{exo} 


\section{Dénombrement}


\begin{exo}



\begin{enumerate}
\item Pour chacun des 4 symboles, il y a 12 possibilités, donc au total $12\times 12\times 12\times 12=12^4=\np{20736}$ codes différents possibles.

\medskip

\textbf{Remarque~:} Chaque code est ce que l'on appelle une \textbf{4-liste d'un ensemble à 12 éléments}.
\item On raisonne comme dans la question 1~: il y a $11^4=\np{14641}$ codes d'entrée ne comportant pas la lettre A.
\item Il y a 12 possibilités pour le 1\up{er} symbole, 11 pour le 2\up{e} (car il diffère du premier symbole), puis 10 pour le 3\up{e}~; et enfin 9 pour le 4\up{e}. Donc au total,
\[12\times 11\times 10\times 9=\np{11880}\] codes avec 4 symboles différents.

\medskip

\textbf{Remarque~:} Chaque code est ce que l'on appelle un \textbf{arrangement de 4 éléments d'un ensemble à 12 éléments}. Le cours donne directement la réponse~:
\[\text{nombre d'arrangements}=\frac{12!}{(12-4)!}=\frac{12!}{8!}=
\frac{12\times 11\times 10\times 9\times \cancel{8}\times \cancel{7}\times\cdots\times \cancel{1}}{\cancel{8}\times \cancel{7}\times\cdots\times \cancel{1}}=12\times 11\times 10\times 9.\]
\end{enumerate}

\end{exo}

\begin{exo}




\begin{enumerate}
\item On obtient les anagrammes de VOYAGE en permutant les lettres de toutes les façons possibles. La lettre V peut prendre 6 positions différentes, puis il reste 5 positions possibles pour le O, puis 4 pour le Y, etc. Au final, il y a
\[6!=6\times 5\times 4\times 3\times 2\times 1=720\] anagrammes.

\medskip

\textbf{Remarque~:} On dit qu'il y a 720 \textbf{permutations} possibles des lettres.
\item Si les 8 lettres du mot ANTILLES étaient toutes différentes, il y aurait $8!=\np{40320}$ anagrammes différentes. Mais il y a deux \og L \fg , donc chaque anagramme est comptée deux fois. En effet, si on différencie les deux \og L \fg~{} en les coloriant, les mots INA\textcolor{red}{L}S\textcolor{green}{L}ET et INA\textcolor{green}{L}S\textcolor{red}{L}ET, par exemple, semblent différents~; mais ils représentent en réalité le même mot INALSLET. Finalement, il n'y a que
\[\np{40320}\div 2=\np{20160}\] anagrammes différentes.
\end{enumerate}

\end{exo}




\begin{exo}

Il y a 2 possibilités pour la 1\up{re} réponse, 2 possibilités pour la 2\up{e}, 2 pour la 3\up{e}, etc. Donc au total $2^{10}=\np{1024}$ façons possibles de remplir le questionnaire.

\medskip

\textbf{Remarques~:}

\begin{itemize}
\item[\textbullet] Il s'agit du nombre de 10-listes d'un ensemble à deux éléments (ces deux éléments étant Vrai/Faux).
\item[\textbullet] Au collège, vous auriez pu présenter la solution avec un arbre~:

\begin{center}
\begin{tikzpicture}[
  level 1/.style={level distance=6em, sibling distance=5em},
  level 2/.style={level distance=6em, sibling distance=2.5em},
  level 3/.style={level distance=6em, sibling distance=1em},
  grow'=right,
  ]
  \coordinate                   % joli sommet de l'arbre
    child foreach \evenemi in {V, F}
      {
        node {$\evenemi$}
        child foreach \evenemii in {V, F}
          {
            node {$\evenemii$}
            child foreach \evenemiii in {$\cdots$, $\cdots$}
              { node {\evenemiii} }
          }
      };
\end{tikzpicture}
\end{center}

\end{itemize}


\end{exo}

\begin{exo}



\begin{enumerate}
\item Les podiums sont les arrangements de 3 éléments d'un ensemble à 8 éléments (car les trois premiers de la course sont différents), donc il y en a 
\[\frac{8!}{(8-3)!}=\frac{8!}{5!}=8\times 7\times 6=336.\]
\item On s'intéresse à l'événement contraire~: on compte le nombre de podiums sans aucun Américain. Il s'agit du nombre d'arrangements de 3 éléments d'un ensemble à 5 éléments (les 5 non Américains)~; il y en a 
\[\frac{5!}{(5-3)!}=\frac{5!}{2!}=5\times 4\times 3=60.\]

Conclusion~: il reste \[336-60=276\] podiums comportant au moins un Américain.
\end{enumerate}

\end{exo}


\begin{exo}

On a déjà rencontré ce code dans l'exercice 40 (à quelques différences près). On explique son fonctionnement avec un tableau~:

\medskip

\setlength{\columnseprule}{1pt}

\begin{multicols}{2}

\begin{center}
\textbf{Programme}
\end{center}

\begin{lstlisting}
def fact(n):
	p=1
	for i in range(1,n+1):
		p=p*i
	return p
\end{lstlisting}



\vspace*{0.5cm}

\columnbreak

\begin{center}
\textbf{Tableau}
\end{center}

\medskip

On rentre par exemple la commande fact(4) -- donc i va alors de 1 à 4~:


\begin{center}


\begin{tabular}{|c|c|} \hline
\textbf{Valeur de i}& \textbf{Valeur de p}\\ \hline
\cellcolor{gray}& 1\\ \hline
1& $1\times 1=1$\\ \hline
2&$1\times 2=2$\\ \hline
3&$2\times 3=6$\\ \hline
4&$6\times 4=24$ \\ \hline
\end{tabular}
\end{center}

\end{multicols}

\medskip

La valeur renvoyée est 24, qui correspond à 
\[4!=1\times 2\times 3\times 4.\]




\end{exo}


\begin{exo}

Pour simplifier et sans rien enlever à la généralité du raisonnement, on suppose que les questions sont numérotées de 1 à 6 en histoire et de 1 à 5 en géographie, et que le candidat connaît les questions n°1, 2, 3 en histoire, n°1 et 2 en géographie.

Dans le tableau ci-dessous, les questions connues sont écrites en bleu, les questions inconnues sont écrites en rouge.

\medskip



On a colorié les cases de trois couleurs~:
\begin{itemize}
\item[\textbullet] en vert~: le candidat connaît les deux questions~;
\item[\textbullet] en orange~: le candidat connaît une seule des deux questions~;
\item[\textbullet] en magenta~: le candidat ne connaît aucune des deux questions.
\end{itemize}


\begin{center}
\begin{tabular}{|c|c|c|c|c|c|c|}\hline
\backslashbox{Géo}{Hist}&\textcolor{blue}{1}&\textcolor{blue}{2}&\textcolor{blue}{3}&\textcolor{red}{4}&\textcolor{red}{5}&\textcolor{red}{6} \\ \hline
\textcolor{blue}{1}&\cellcolor{green}&\cellcolor{green}&\cellcolor{green}&\cellcolor{orange}&\cellcolor{orange}&\cellcolor{orange}	\\ \hline
\textcolor{blue}{2}&\cellcolor{green}&\cellcolor{green}&\cellcolor{green}&\cellcolor{orange}&\cellcolor{orange}&\cellcolor{orange}\\ \hline
\textcolor{red}{3}&\cellcolor{orange}&\cellcolor{orange}&\cellcolor{orange}&\cellcolor{magenta}&\cellcolor{magenta}&\cellcolor{magenta}\\ \hline
\textcolor{red}{4}&\cellcolor{orange}&\cellcolor{orange}&\cellcolor{orange}&\cellcolor{magenta}&\cellcolor{magenta}&\cellcolor{magenta}\\ \hline
\textcolor{red}{5}&\cellcolor{orange}&\cellcolor{orange}&\cellcolor{orange}&\cellcolor{magenta}&\cellcolor{magenta}&\cellcolor{magenta}\\ \hline
\end{tabular}
\end{center}




Conclusion~: il y a $6\times 5=30$ cases au total, 6 vertes et 15 oranges, donc~:

\begin{itemize}
\item[\textbullet] la probabilité que le candidat connaisse les deux questions est $\frac{6}{30}=\frac{1}{5}~;$
\item[\textbullet] la probabilité que le candidat connaisse au moins l'une des deux questions est $\frac{6+15}{30}=\frac{21}{30}=\frac{7}{10}.$
\end{itemize}

\medskip

\textbf{Remarque~:} On aurait pu se passer du tableau~:

\begin{itemize}
\item[\textbullet] il y a $6\times 5=30$ tirages possibles~;
\item[\textbullet] il y a $3\times 2=6$ cas favorables à l'événement \og le candidat connaît les deux questions \fg~;
\item[\textbullet] il y a $3\times 3=9$ cas favorables à l'événement \og le candidat ne connaît aucune des deux questions \fg, donc $30-9=21$ cas favorables à l'événement contraire \og le candidat connaît au moins l'une des deux questions \fg.
\end{itemize}

\end{exo}


\begin{exo}

\begin{itemize}
\item[\textbullet] Les cas possibles sont les 3-listes d'un ensemble à 4 éléments~; il y en a $4^3=64.$
\item[\textbullet] Les cas favorables à G sont les arrangements de 3 éléments d'un ensemble à 4 éléments ($\heartsuit$, $\diamondsuit$, $\spadesuit$, $\clubsuit$)~; il y en a $\frac{4!}{(4-3)!}=\frac{4!}{1!}=4\times 3\times 2=24.$ On a donc $P(\text{G})=\frac{24}{64}=\frac{3}{8}.$
\item[\textbullet] Pour calculer $P(H),$ on prend l'événement contraire $\overline{\text{H}}$~: \og aucun cœur n'apparaît à l'écran \fg.

Les cas favorables à $\overline{\text{H}}$ sont les 3-listes d'un ensemble à 3 éléments ($\diamondsuit$, $\spadesuit$, $\clubsuit$), il y en a donc $3^3=27.$ Il reste $64-27=37$ cas favorables à H, et ainsi $P(\text{H})=\frac{37}{64}.$ 
\end{itemize}
\end{exo}

\begin{exo}

\begin{enumerate}
\item \begin{itemize}
\item[\textbullet] Les cas possibles sont les 30-listes d'un ensemble à 200 éléments, il y en a $200^{30}.$
\item[\textbullet] Les cas favorables sont les arrangements de 30 éléments d'un ensemble à 200 éléments~; il y en a \[\frac{200!}{(200-30)!}=\frac{200!}{170!}=200\times 199\times 198\times \cdots\times 172\times 171.\]
\item[\textbullet] La probabilité que les élèves choisissent tous un nombre différent est donc
\[p=\frac{200\times 199\times 198\times \cdots\times 172\times 171}{200^{30}}.\]
\end{itemize}
\item On remarque que

\[
p=\frac{200\times 199\times 198\times \cdots\times 172\times 171}{200\times 200\times 200\times\cdots\times 200\times 200}=\frac{200}{200}\times \frac{199}{200}\times \frac{198}{200}\times \frac{172}{200}\times \frac{171}{200}.
\]

C'est sous cette forme, en calculant le produit de proche en proche, que l'on peut obtenir la réponse avec un programme\footnote{Le nombre $200^{30}$ dépasse les capacités de votre calculatrice, faisant du calcul de proche en proche une nécessité.}~: on part de la valeur 1, puis on multiplie par $\frac{171}{200},$ puis par $\frac{172}{200},$ puis par $\frac{173}{200},$ ... jusqu'à $\frac{200}{200}.$

\medskip

\begin{lstlisting}
def proba():
	p=1
	for i in range(171,201):
		p=p*i/200
	return p
\end{lstlisting}


\end{enumerate}

\medskip

La réponse obtenue en sortie est $p\approx 0,10.$

\end{exo}

\begin{exo}

\begin{itemize}
\item[\textbullet] Les podiums possibles sont les 3-listes d'un ensemble à 12 éléments~; il y en a $12^3=\np{1728}.$
\item[\textbullet] Les cas favorables à l'événement A : \og Le joueur obtient le tiercé \fg~{} sont toutes les permutations possibles des 3 premiers de la course~; il y en a $3!=3\times 2\times 1=6.$ On peut d'ailleurs les énumérer rapidement~:
\[(7,4,10)~;~(7,10,4)~;(4,7,10)~;(4,10,7)~;(10,4,7)~;(10,7,4).\]
\item[\textbullet] Conclusion~: $P\left(\text{A}\right)=\frac{6}{\np{1728}}=\frac{1}{288}.$
\end{itemize}

\end{exo}

\begin{exo}

\begin{enumerate}
\item On  commence par $A=\binom{5}{2}.$ Il y a trois méthodes~:

\begin{itemize}
\item[\textbullet] avec la formule~:
\[A=\binom{5}{2}=\frac{5!}{2!\times 3!}=\frac{5\times 4\times\cancel{3}\times \cancel{2}\times \cancel{1}}{2\times 1\times \cancel{3}\times \cancel{2}\times \cancel{1}}=\frac{20}{2}=10.\]
\item[\textbullet] avec le triangle de Pascal~:

\begin{center}
\begin{tabular}{|l|c c c c c c|}
\hline
         & 0        & 1        & \red{$\boxed{2}$} &3 &4    & 5   \\
\hline
$0$ &  1 &0&0&0&0  &0      \\
$1$ &   1&1&0&0&0  &  0   \\
$2$ &   1&2&1&0&0     &0  \\
$3$ &  1&3&3&1&0       &0 \\
$4$ &    1&4&6&4&1&0      \\
\red{$\boxed{5}$} &    1&5&\red{$\boxed{10}$}&10&5&1      \\
\hline
\end{tabular}
\end{center}
\item[\textbullet] avec la calculatrice~:

\small

\setlength{\columnseprule}{1pt}
\begin{multicols}{4}

\begin{center}\textbf{Calculatrices collège}\end{center}

\medskip


Il faut écrire le calcul (le symbole ! est sur le clavier)~:

\[\frac{5!}{2!\times 3!}\]
\vspace*{1cm}
\columnbreak

\begin{center}\textbf{NUMWORKS}\end{center}

\medskip

\begin{itemize}
\item[\textbullet] \fbox{\textcolor{yellow}{\faHome}}
\item[\textbullet] Calculs \fbox{EXE} puis \fbox{\textcolor{black}{\faToolbox }} (boîte à outils)
\item[\textbullet] choisir Dénombrement \fbox{EXE}
\item[\textbullet] choisir  binomial(n,k) \fbox{EXE}
\item[\textbullet] compléter $\binom{5}{2}$ \fbox{EXE}
\end{itemize}

\columnbreak

\begin{center}\textbf{TI graphiques}\end{center}

\medskip


\begin{itemize}
\item[\textbullet] \fbox{math} puis \fbox{PROB} 
\item[\textbullet] 3:Combinaison

\fbox{EXE}
\item[\textbullet] $~_{5}\text{C}_2$ \fbox{EXE}
\end{itemize}
\vspace*{1cm}

\columnbreak

\begin{center}\textbf{CASIO graphiques}\end{center}

\medskip


\begin{itemize}
\item[\textbullet] \fbox{MENU} puis \fbox{RUN} \fbox{EXE}
\item[\textbullet] 5 \fbox{OPTN}  \fbox{$\triangleright$}
\item[\textbullet] \fbox{F3} (on choisit donc PROB)
\item[\textbullet] \fbox{F3} (on choisit donc nCr)
\item[\textbullet] 2 \fbox{EXE} (on affiche 5\textbf{C}2 à l'écran avant d'exécuter)
\end{itemize}






\end{multicols}
\end{itemize}

\normalsize

\medskip

Quelle que soit la méthode, on obtient \[A=\binom{5}{2}=10.\]


\medskip

On obtient également~:

\[B=\binom{6}{3}=20,\quad C=\binom{50}{1}=50,\quad D=\binom{4}{0}=1.\]

\medskip

\textbf{Remarque~:} Pour $C$ et $D,$ le résultat s'obtient sans calcul~:

\begin{itemize}
\item[\textbullet] pour $C,$ on choisit 1 élément parmi 50, donc il y a 50 choix possibles~;
\item[\textbullet] pour $D,$ on on sait (cf cours) que $\binom{n}{0}=1$ quelle que soit la valeur de $n.$
\end{itemize}

\item \begin{itemize}
\item[\textbullet] $\binom{10}{3}=\binom{10}{7}=120.$ L'égalité était prévisible~: choisir 3 éléments que l'on conserve dans un ensemble à 10 éléments revient à choisir les 7 éléments que l'on met de côté.
\item[\textbullet] Par le même raisonnement, vu que $100-60=40,$ $\binom{100}{60}=\binom{100}{40}.$ Et d'une manière plus générale, si $0\leq k\leq n~:$
\[\binom{n}{k}=\binom{n}{n-k}.\]
\end{itemize}

\end{enumerate}

\end{exo}



\begin{exo}

On choisit trois numéros sur une grille de neuf cases. Il y a $\binom{9}{3}=84$ grilles possibles.

\medskip



\begin{center}
 \begin{tabular}{|c|c|c|}\hline
1& 2&$\textcolor{red}{\xcancel{\black{3}}}$ \\ \hline
4&$\textcolor{red}{\xcancel{\black{5}}}$&6\\ \hline
7&$\textcolor{red}{\xcancel{\black{8}}}$&9\\ \hline
\end{tabular}
\end{center}

\end{exo}



\begin{exo}

On prend 5 cartes dans un jeu de 32. Il y a $\binom{32}{5}=\np{201376}$ mains possibles.

\end{exo}

\begin{exo}

Lorsqu'ils se rencontrent en arrivant le matin au lycée, les 24 élèves d'une classe se serrent la main.

\medskip

Choisir une poignée de main, c'est choisir deux personnes dans la classe. Il s'échange donc  $\binom{24}{2}=\np{276}$ poignées de mains au total.

\end{exo}


\begin{exo}


\begin{enumerate}
\item Si $p\geq 2~:$ \[(p-1)!\times \textcolor{red}{p}=1\times 2\times\cdots\times (p-1)\times \textcolor{red}{p}=p!\]
(l'égalité est également vraie lorsque $p=1,$ puisque $0!\times 1=1=1!$).
\item Si $n\geq 2~:$ 
\[\binom{n+1}{2}=\frac{(n+1)!}{2!\times(n+1-2)!}=\frac{(n+1)!}{2!\times(n-1)!}=
\frac{\cancel{(n-1)!}\times n\times (n+1)}{2\times\cancel{(n-1)!}}=\frac{n^2+n}{2}.\]

\medskip

\textbf{Remarque~:} On peut aussi obtenir la réponse par un raisonnement de dénombrement~: choisir 2 éléments parmi $n+1$ revient à compter le nombre de poignées de mains lorsque $n+1$ personnes se serrent la main les unes les autres (comme dans l'exercice précédent). Chacune des $n+1$ personnes donne $n$ poignées de main~; et on divise par 2, parce que sinon chaque poignée de main est comptée deux fois. On total, on en dénombre $(n+1)\times n\div 2=\frac{n^2+n}{2}.$


\item Soit $n\geq k\geq 1.$ On calcule séparément~:

 \setlength{\columnseprule}{1pt}
\begin{multicols}{2}

\begin{align*}
n\times \binom{n-1}{k-1}
&=n\times \frac{(n-1)!}{(k-1)!\times((n-1)-(k-1))!}
\\&=\frac{(n-1)!\times n}{(k-1)!\times(n-\cancel{1}-k+\cancel{1})!}
\\&=\frac{n!}{(k-1)!\times(n-k)!}
\end{align*}

\begin{align*}
k\times \binom{n}{k}
&=k\times \frac{n!}{k!\times(n-k)!}
\\&=\frac{n!\times \cancel{k}}{(k-1)!\times \cancel{k}\times(n-k)!}
\\&=\frac{n!}{(k-1)!\times(n-k)!}
\end{align*}
\end{multicols}

\medskip

On a donc bien
\[n\times \binom{n-1}{k-1}=k\times \binom{n}{k}\] (formule du pion).
\end{enumerate}


\end{exo}

\begin{exo}


\begin{enumerate}
\item Il faut choisir 12 personnes parmi 20, donc on peut constituer $\binom{20}{12}$ groupes différents.
\item David est un des membres de l'association. Il y a~:
\begin{itemize}
\item[\textbullet] $\binom{19}{11}$ groupes de 12 personnes contenant David (puisque si David est pris, il reste 11 personnes à choisir parmi les 19 autres)~;
\item[\textbullet] $\binom{19}{12}$ groupes de 12 personnes ne contenant pas David (puisque si David n'est pas pris, il reste encore 12 personnes à choisir parmi les 19 autres).
\end{itemize}
\item D'après les questions 1 et 2~:
\[\binom{20}{12}=\binom{19}{11}+\binom{19}{12}.\]
\item On généralise~: si $1\leq k\leq n-1,$
\[\binom{n}{k}= \binom{n-1}{k-1} + \binom{n-1}{k}.\]
\item On redémontre par le calcul la formule obtenue à la question précédente~:

\begin{align*}
\binom{n-1}{k-1} + \binom{n-1}{k}
&=\frac{(n-1)!}{(k-1)!\times ((n-1)-(k-1))!}+\frac{(n-1)!}{k!\times ((n-1)-k)!}
\\&=\frac{(n-1)!}{(k-1)!\times (n-\cancel{1}-k+\cancel{1})!}+\frac{(n-1)!}{k!\times (n-1-k)!}
\\&=\frac{(n-1)!\textcolor{red}{\times k}}{(k-1)!\textcolor{red}{\times k}\times (n-k)!}+\frac{(n-1)!\textcolor{blue}{\times (n-k)}}{k!\times (n-k-1)!\textcolor{blue}{\times (n-k)}}
\\&=\frac{(n-1)!\times k}{k!\times (n-k)!}+\frac{(n-1)!\times(n-k)}{k!\times (n-k)!}
\\&=\frac{(n-1)!\times(k+n-k)}{k!\times (n-k)!}
\\&=\frac{(n-1)!\times n}{k!\times (n-k)!}
\\&=\frac{n!}{k!\times (n-k)!}
\\&=\binom{n}{k}
\end{align*}

\end{enumerate}

\medskip

\textbf{Remarque~:} Cette égalité, appelée formule de Pascal, permet de faire le lien entre le triangle de Pascal et les $\binom{n}{k}.$ En effet~:

\begin{itemize}
\item[\textbullet] La première colonne  du triangle de Pascal contient uniquement des 1, qui correspondent bien aux $\binom{n}{0}.$
\item[\textbullet] La diagonale  du triangle de Pascal contient uniquement des 1, qui correspondent bien aux $\binom{n}{n}.$
\item[\textbullet] On remplit chacune des cases \og centrales \fg~{} en ajoutant le nombre au-dessus et celui au-dessus à gauche. Si les nombres de la ligne $n-1$ correspondent aux $\binom{n-1}{k},$ alors ceux de la ligne du dessous correspondront également, via la formule de Pascal~:

\LARGE
\[\boxed{\begin{matrix}
\binom{n-1}{k-1}&\textcolor{blue}{+}&\binom{n-1}{k}\\
& &\textcolor{blue}{=}\\
& & \binom{n}{k}
\end{matrix}}\]
\normalsize

C'est une sorte de raisonnement par récurrence~: la correspondance entre les termes du triangle de Pascal et les $\binom{n}{k}$ \og se propage de ligne en ligne \fg.
\end{itemize}

\end{exo}



\begin{exo}

Le sélectionneur choisit 10 joueurs de champ parmi 17, puis 1 gardien parmi 3~; il a donc
\[\binom{17}{3}\times \binom{3}{1}=680\times 3=\np{2040}~\text{équipes possibles.}\]

\end{exo}

\begin{exo}

Il faut choisir 2 moniteurs parmi 5, puis 10 enfants parmi 40. Il y a donc

\[\binom{5}{2}\times \binom{40}{10}~\text{groupes possibles}\]

(la valeur explicite est entre 8 et 9 milliards).


\end{exo}

\begin{exo}


On écrit les formules, mais on ne fait pas les calculs explicites (sans grand intérêt mathématique à ce stade du cours).

\begin{enumerate}
\item Il y a $\binom{32}{5}$ mains possibles.

\item La probabilité qu'une main contienne~:
\begin{itemize}
\item[\textbullet] exactement 3 dames est 
\[\frac{\binom{4}{3}\times \binom{28}{2}}{\binom{32}{5}}\] (on choisit 3 dames parmi 4, puis 2 cartes parmi les 28 autres)~;
\item[\textbullet] trois cœurs et deux carreaux est
\[\frac{\binom{8}{3}\times \binom{8}{2}}{\binom{32}{5}}\] (on choisit 3 cœurs parmi 8, puis 2 carreaux parmi 8)\footnote{On pourrait multiplier par $\binom{16}{0}$ au numérateur, puisqu'on ne choisit aucune carte parmi les piques et les trèfles. Bien sûr, cela ne changerait rien à la réponse, puisque $\binom{16}{0}=1.$}~;
\item[\textbullet] exactement un roi et deux valets est
\[\frac{\binom{4}{1}\times \binom{4}{2}\times\binom{24}{2}}{\binom{32}{5}}\] (on choisit 1 roi parmi 4, puis 2 valets parmi 4~; et enfin 2 cartes parmi les 24 autres).
\end{itemize}
\end{enumerate}
\end{exo}

\begin{exo}


\begin{enumerate}
\item 

\begin{enumerate}
\item Le recrutement de 3 candidats peut se faire de $\binom{8}{3}=56$ façons possibles.
\item Le recrutement de 7 candidats peut se faire de $\binom{8}{7}=8$ façons possibles.
\item On peut recruter entre 0 et 8 candidats, donc en raisonnant comme dans les questions 1 et 2, on voit qu'il y a
\[\binom{8}{0}+\binom{8}{1}+\binom{8}{2}+\binom{8}{3}+\binom{8}{4}+\binom{8}{5}+\binom{8}{6}+\binom{8}{7}+\binom{8}{8}\] recrutements différents possibles.
\item Chaque candidat est soit accepté (A), soir refusé (R). On peut donc assimiler le recrutement à une liste de 8 éléments à choisir parmi A/R. Ainsi y a-t-il \[2^8=256\] recrutements différents possibles (nombre de 2-listes d'un ensemble à 8 éléments)\footnote{Le lecteur qui n'est pas convaincu peut faire un arbre.}.
\end{enumerate}
\item Soit $n\geq 1.$ On imagine $n$ candidats au lieu 8 et on raisonne comme dans la question 1~: le nombre de recrutements différents possibles est
\[\binom{n}{0}+\binom{n}{1}+\binom{n}{2}+\cdots +\binom{n}{n}=2^n.\]  

\medskip

Avec le symbole $\sum,$ cette formule (fort connue) se réécrit \[\sum\limits_{k=0}^n\binom{n}{k}=2^n.\]

(On vérifie sans peine qu'elle est également vraie lorsque $n=0.$)

\end{enumerate}

\end{exo}




\begin{exo}~{}


\begin{center}
\psset{xunit=1.0cm,yunit=1.0cm,algebraic=true,dimen=middle,dotstyle=o,dotsize=5pt 0,linewidth=2.pt,arrowsize=3pt 2,arrowinset=0.25}
\begin{pspicture*}(-2.52,-0.34)(7.52,4.56)
\psline[linewidth=2.pt](-2.,4.)(-2.,1.)
\psline[linewidth=2.pt](-2.,1.)(1.,1.)
\psline[linewidth=2.pt](1.,1.)(1.,4.)
\psline[linewidth=2.pt](1.,4.)(-2.,4.)
\psline[linewidth=2.pt](3.,4.)(3.,1.)
\psline[linewidth=2.pt](3.,1.)(6.,1.)
\psline[linewidth=2.pt](6.,1.)(6.,4.)
\psline[linewidth=2.pt](6.,4.)(3.,4.)
\begin{LARGE}
\rput[tl](-1.56,3.48){\ding{172}}
\rput[tl](-1.4,1.78){\ding{173}}
\rput[tl](-0.56,2.78){\ding{174}}
\rput[tl](-0.24,1.94){\ding{185}}
\rput[tl](0.04,3.42){\ding{186}}
\rput[tl](4.3,3.44){\ding{177}}
\rput[tl](5.18,2.02){\ding{178}}
\rput[tl](3.62,2.84){\ding{189}}
\rput[tl](3.78,1.6){\ding{190}}
\rput[tl](4.92,2.8){\ding{191}}
\end{LARGE}
\rput[tl](-0.55,0.45){$U_1$}
\rput[tl](4.5,0.45){$U_2$}
\end{pspicture*}
\end{center}

\begin{itemize}
\item[\textbullet] Il y a \[\binom{5}{2}\times \binom{5}{2}=10\times 10=100\] tirages possibles (puisqu'on choisit 2 des 5 boules de l'urne  $U_1,$ 2 des 5 boules de l'urne  $U_2$).
\item[\textbullet] Pour avoir 2 boules blanches il faut, au choix~:
\begin{itemize}
\item Tirer 2 blanches et 0 noire dans l'urne $U_1,$ 0 blanche et 2 noires dans l'urne $U_2.$ Le nombre de façons différentes de faire ce tirage est \[\binom{3}{2}\times \binom{2}{0}\times \binom{2}{0}\times \binom{3}{2}=3\times 1\times 1\times 3=9.\]
\item Tirer 1 blanche et 1 noire dans l'urne $U_1,$ 1 blanche et 1 noire dans l'urne $U_2.$ Le nombre de façons différentes de faire ce tirage est \[\binom{3}{1}\times \binom{2}{1}\times \binom{2}{1}\times \binom{3}{1}=3\times 2\times 2\times 3=36.\]
\item Tirer 0 blanche et 2 noires dans l'urne $U_1,$ 2 blanches et 0 noire dans l'urne $U_2.$ Le nombre de façons différentes de faire ce tirage est \[\binom{3}{0}\times \binom{2}{2}\times \binom{2}{2}\times \binom{3}{0}=1\times 1\times 1\times 1=1.\]
\end{itemize}
\item[\textbullet] Conclusion~: $P(A)=\frac{9+36+1}{100}=0,46.$
\end{itemize}



\end{exo}

\section{Limites de suites}


\begin{exo}



\begin{enumerate}
\item $u_n=3+\dfrac{1}{n}.$

\[
\left.
    \begin{array}{ll}
        \lim\limits_{n\to +\infty}3&= 3 \\
        \lim\limits_{n\to +\infty}\dfrac{1}{n}&= 0
    \end{array}
\right \}\implies \lim\limits_{n\to +\infty}\left(3+\dfrac{1}{n}\right)=3+0=3.
\]

On s'autorise à aller un peu plus vite~: on écrit simplement
\[\lim\limits_{n\to +\infty}u_n=3+0=3.\] 
\item On écrit $v_n=4-\dfrac{1}{n^2}=4-\dfrac{1}{n}\times \dfrac{1}{n}.$ On a donc
\[\lim\limits_{n\to +\infty}v_n=4-0\times 0=4.\] 

\item On écrit $w_n=\left(5+\dfrac{3}{n}\right)\left(2+\dfrac{1}{n}\right)=\left(5+3\times \dfrac{1}{n}\right)\left(2+\dfrac{1}{n}\right).$ On a donc
\[\lim\limits_{n\to +\infty}w_n=\left(5+3\times 0\right)\left(2+0\right)=10.\] 


\item On écrit $x_n=\dfrac{1}{1+\dfrac{2}{n}}=\dfrac{1}{1+2\times \dfrac{1}{n}}.$

On a donc
\[\lim\limits_{n\to +\infty}x_n=\dfrac{1}{1+2\times 0}=1.\] 
\item On met $n$ en facteur au numérateur et au dénominateur~:
\[
y_n=\dfrac{3n-5}{4n+1}
=\dfrac{\cancel{n}\left(3-\frac{5}{n}\right)}{\cancel{n}\left(4+\frac{1}{n}\right)}
=\dfrac{3-5\times \frac{1}{n}}{4+\frac{1}{n}}.
\]

On a donc
\[\lim\limits_{n\to +\infty}y_n=\dfrac{3-5\times 0}{4+0}=\dfrac{3}{4}.\]


\end{enumerate}

\medskip

\textbf{Bilan~:} Pour calculer les limites, il suffit de faire apparaître des $\frac{1}{n}$ et de les remplacer par 0 lorsqu'on \og passe à la limite \fg.

\end{exo}



\begin{exo}

La suite $(u_n)_{n\in\mathbb{N}}$ est définie par $u_0=6$ et la relation de récurrence \[u_{n+1}=0,6u_n-4\] pour tout $n\in\mathbb{N}.$ On admet qu'elle converge et on note $\ell$ sa limite.

\medskip

Les suites $\left(u_n\right)_{n\in\mathbb{N}}=\left(u_0,u_1,u_2,\cdots\right)$ et $\left(u_{n+1}\right)_{n\in\mathbb{N}}=\left(u_1,u_2,u_3,\cdots\right)$ ont la même limite   puisque les indices sont simplement décalés~:

\begin{align*}
\lim\limits_{n\to +\infty}u_n&=\ell,\\
\lim\limits_{n\to +\infty}u_{n+1}&=\ell.
\end{align*}

 Par opération sur les limites, on peut \og passer à la limite \fg~{} dans la relation de récurrence~:
\[u_{n+1}=0,6u_n-4\qquad\text{pour tout }n\in\mathbb{N},\] donc
\[\ell=0,6\ell-4.\]

On résout cette équation~:
\[\ell=0,6\ell-4\iff \ell-0,6\ell=-4\iff 0,4\ell=-4\iff\ell=\frac{-4}{0,4}\iff \ell=-10.\]

Conclusion~: $\lim\limits_{n\to +\infty}u_n=-10.$
\end{exo}


\begin{exo}

La suite $\left(v_n\right)_{n\in\mathbb{N}}$ est géométrique de premier terme $v_0=20$ et de raison $q=-0,5.$

\begin{enumerate}
\item $v_0=20~;~v_1=20\times (-0,5)=-10~;~v_2=-10\times (-0,5)=5~;~v_3=5\times (-0,5)=-2,5~;~v_4=-2,5\times (-0,5)=1,25.$
\item On admet que $\left(v_n\right)_{n\in\mathbb{N}}$ converge, on note $\ell$ sa limite.

\medskip
On \og passe à la limite \fg~{} dans la relation de récurrence~: $\left(v_n\right)_{n\in\mathbb{N}}$ est géométrique de raison $q=-0,5,$ donc
\[v_{n+1}=-0,5\times v_n\qquad\text{pour tout }n\in\mathbb{N}~;\] et donc
\[\ell=-0,5\times\ell.\]

On résout~:
\[\ell=-0,5\times\ell\iff\ell+0,5\ell=0\iff 1,5\ell=0\iff \ell=\frac{0}{1,5}\iff\ell=0.\]

Conclusion~: $\lim\limits_{n\to +\infty}v_n=0.$

\end{enumerate}

\end{exo}

\begin{exo}


\begin{enumerate}
\item Il est clair que $0\leq \frac{1}{n+\sqrt{n}}$ pour tout entier $n\geq 1.$ Pour l'autre inégalité, on part de \[n+\sqrt{n}\geq n.\] Deux nombres strictement positifs sont rangés en sens contraire de leurs inverses, donc
\[\frac{1}{n+\sqrt{n}}\leq \frac{1}{n}\] (\danger en prenant l'inverse, le sens de l'inégalité est renversé).

\item $0\leq \frac{1}{n+\sqrt{n}}\leq \frac{1}{n}$ pour tout entier $n\geq 1$ et
\[\lim\limits_{n\to +\infty}0=0\qquad , \qquad \lim\limits_{n\to +\infty}\frac{1}{n}=0.\] Donc d'après le théorème des gendarmes~:
\[\lim\limits_{n\to +\infty}\dfrac{1}{n+\sqrt{n}}=0.\]
\end{enumerate}

\end{exo}

\begin{exo}

La suite $(u_n)_{n\in\mathbb{N}}$ est définie par $u_0=1$ et la relation de récurrence

\[u_{n+1}=0,6u_n+0,4n+1\] pour tout $n\in\mathbb{N}.$

\begin{enumerate}
\item \begin{align*}
u_1&=0,6u_0+0,4\times 0+1=0,6\times 1+0+1=1,6\\
u_2&=0,6u_1+0,4\times 1+1=0,6\times 1,6+0,4+1=2,36
\end{align*}
\item Pour tout $n\in\mathbb{N},$ on note $\mathcal{P}_n$ la propriété \[n\leq u_n\leq n+1.\]





\begin{itemize}
\item[{\textbullet}] \textbf{Initialisation.} On prouve que $\mathcal{P}_0$ est vraie.

\[
\left.
    \begin{array}{ll}
        u_0&=1 \\
        0&\leq u_0\leq 0+1
    \end{array}
\right \}\implies \mathcal{P}_0~\text{est vraie}.
\]



\item[{\textbullet}] \textbf{Hérédité.} Soit $k\in\mathbb{N}$ tel que $\mathcal{P}_k$ soit vraie. On a donc
\[k\leq u_k\leq k+1.\]
%\medskip

\newtcolorbox{mybox}[1]{colback=green!10!white,colframe=green!80!white,fonttitle=\bfseries,title=#1}
\begin{mybox}{Objectif}{Prouver que $\mathcal{P}_{k+1}$ est vraie, c'est-à-dire que \[k+1\leq u_{k+1}\leq k+2.\]
}\end{mybox}



%\medskip

On part de 
\[k\leq u_k\leq k+1.\]


On multiplie par $\textcolor{red}{0,6}~:$

\begin{align*}k\textcolor{red}{\times 0,6}&\leq u_k\textcolor{red}{\times 0,6}\leq (k+1)\textcolor{red}{\times 0,6}\\
0,6k&\leq 0,6u_k\leq 0,6k+0,6\end{align*}

Puis on ajoute  $\textcolor{blue}{0,4k+1}~:$

\begin{align*}
0,6k\textcolor{blue}{+0,4k+1}&\leq 0,6u_k\textcolor{blue}{+0,4k+1}\leq 0,6k+0,6\textcolor{blue}{+0,4k+1}\\
k+1&\leq 0,6u_k+0,4k+1\leq k+1,6\\
k+1&\leq u_{k+1}\leq k+1,6
\end{align*}

Or $k+1,6\leq k+2,$ donc la propriété $\mathcal{P}_{k+1}$ est vraie.
\item[{\textbullet}] \textbf{Conclusion.} $\mathcal{P}_0$ est vraie et $\mathcal{P}_n$ est héréditaire, donc elle est vraie pour tout $n\in\mathbb{N}.$
\end{itemize}
\item Soit $n\geq 1.$ On reprend l'inégalité de la question précédente et on divise par $n~:$

\begin{align*}
n&\leq u_n\leq n+1\\
\frac{n}{\textcolor{red}{n}}&\leq \frac{u_n}{\textcolor{red}{n}}\leq \frac{n+1}{\textcolor{red}{n}}\\
1&\leq \frac{u_n}{n}\leq 1+\frac{1}{n}
\end{align*}

\medskip

On conclut~:

\[\lim\limits_{n\to +\infty}1=1\qquad , \qquad \lim\limits_{n\to +\infty}\left(1+\frac{1}{n}\right)=1+0=1,\] donc d'après le théorème des gendarmes
\[\lim\limits_{n\to +\infty}\frac{u_n}{n}=1.\]

\end{enumerate}

\end{exo}


\begin{exo}

On veut prouver que $\lim\limits_{n\to +\infty}\frac{n-3}{4}=+\infty.$ On se donne pour cela un réel $M> 0$ et on écrit les équivalences~:

\[\frac{n-3}{4}\geq M \iff n-3\geq 4M\iff n\geq 4M+3.\]

Conclusion~: quand $n$ dépasse $ 4M+3,$ $\frac{n-3}{4}$ dépasse $M.$ On a donc bien $\lim\limits_{n\to +\infty}\frac{n-3}{4}=+\infty.$



\end{exo}

\begin{exo}

Soit $M>0.$ On sait que $\lim\limits_{n\to +\infty}u_n=+\infty,$ donc $u_n\geq M$ à partir d'un certain rang $N.$ On a donc $v_n\geq u_n\geq M$ à partir du rang $N.$ On en déduit $\lim\limits_{n\to +\infty}v_n=+\infty.$



\end{exo}




\begin{exo}

Il est conseillé de calculer les premiers termes des suites pour se faire une idée des variations et de l'existence éventuelle d'un majorant ou d'un minorant. Cela permet ensuite de traiter les questions avec efficacité~: si par exemple on a prouvé qu'une suite était croissante, alors on est certain (théorème du cours) qu'elle est minorée par son premier terme.

\medskip 


\begin{enumerate}
\item $u_n=\text{e}^{-n}$ pour tout $n\in\mathbb{N}.$

\begin{itemize}
\item[\textbullet] Une exponentielle est strictement positive, donc $\left(u_n\right)_{n\in\mathbb{N}}$ est minorée par $0.$
%\item[\textbullet] La fonction exponentielle est strictement décroissante, donc \[n\geq 0\implies -n\leq 0\implies \text{e}^{-n}\leq \text{e}^{0}\implies u_n\leq 1.\]
%La suite $\left(u_n\right)_{n\in\mathbb{N}}$ est donc majorée par $1.$
\item[\textbullet] Pour tout $n\in\mathbb{N}~:$
\begin{align*}u_{n+1}-u_{n}&=\text{e}^{-(n+1)}-\text{e}^{-n}
\\&=\text{e}^{-n-1}-\text{e}^{-n}
\\&=\text{e}^{-n}\times\text{e}^{-1} -\text{e}^{-n}\times 1
\\&=\underbrace{\text{e}^{-n}}_{\oplus}\underbrace{\left(\text{e}^{-1}-1\right)}_{\ominus}
\end{align*}
Conclusion~: $u_{n+1}-u_{n}\leq 0,$ donc $\left(u_n\right)_{n\in\mathbb{N}}$ est décroissante.
\item[\textbullet] Comme $\left(u_n\right)_{n\in\mathbb{N}}$ est décroissante, elle est majorée par son premier terme, $u_0=\text{e}^{-0}=1.$
\end{itemize}
\item $v_n=n^2$ pour tout $n\in\mathbb{N}.$


\begin{itemize}
\item[\textbullet] Un carré est positif, donc $\left(v_n\right)_{n\in\mathbb{N}}$ est minorée par $0.$
\item[\textbullet] Soit $M>0.$ La fonction carré est strictement croissante sur $\left[0;+\infty\right[,$ donc \[n> \sqrt{M}\implies n^2> \sqrt{M}^2\implies  u_n> M.\] Le réel $M$ ne peut donc être un majorant. Et comme cela est vrai quel que soit $M,$ la suite $\left(v_n\right)_{n\in\mathbb{N}}$ n'est pas majorée. 
\item[\textbullet] Pour tout $n\in\mathbb{N}~:$
\begin{align*}u_{n+1}-u_{n}&=(n+1)^2-n^2
\\&=\cancel{n^2}+2n+1-\cancel{n^2}
\\&=\underbrace{2n+1}_{\oplus}
\end{align*}
Conclusion~: $u_{n+1}-u_{n}\geq 0,$ donc $\left(u_n\right)_{n\in\mathbb{N}}$ est croissante.
\end{itemize}
\item $(w_n)_{n\in\mathbb{N}}$ est définie par $w_0=1$ et la relation de récurrence \[w_{n+1}=\dfrac{w_n}{1+2w_n}\] pour tout $n\in\mathbb{N}.$


\begin{itemize}
\item[\textbullet] L'énoncé nous dit d'admettre que $(w_n)_{n\in\mathbb{N}}$ est à termes positifs, donc elle est minorée par $0.$
\item[\textbullet] Pour tout $n\in\mathbb{N}~:$
\begin{align*}w_{n+1}-w_{n}&=\dfrac{w_n}{1+2w_n}-w_n
\\&=\dfrac{w_n}{1+2w_n}-\dfrac{w_n\textcolor{red}{\left(1+2w_n\right)}}{\textcolor{red}{1+2w_n}}
\\&=\dfrac{w_n}{1+2w_n}-\dfrac{w_n+2w_n^2}{1+2w_n}
\\&=\dfrac{\cancel{w_n}-\cancel{w_n}-2w_n^2}{1+2w_n}
\\&=\dfrac{\overbrace{-2w_n^2}^{\ominus}}{\underbrace{1+2w_n}_{\oplus~\text{car}~w_n\oplus}}
\end{align*}
Conclusion~: $w_{n+1}-w_{n}\leq 0,$ donc $\left(w_n\right)_{n\in\mathbb{N}}$ est décroissante.
\item[\textbullet] Comme $\left(w_n\right)_{n\in\mathbb{N}}$ est décroissante, elle est majorée par son premier terme, $w_0=1.$
\end{itemize}
\end{enumerate}

\end{exo}

\begin{exo}

La suite $(w_n)_{n\in\mathbb{N}}$ de l'exercice précédent est décroissante et minorée par 0. Or d'après le théorème de limite monotone, toute suite décroissante minorée converge, donc $(w_n)_{n\in\mathbb{N}}$ converge.

\end{exo}

\begin{exo}

La suite $(v_n)_{n\in\mathbb{N}^*}$ est définie par $v_n=\dfrac{n}{2^n}$ pour tout $n\in\mathbb{N}^*.$ 


\begin{enumerate}
\item Pour tout $n\in\mathbb{N}^*~:$
\begin{align*}v_{n+1}-v_{n}&=\dfrac{n+1}{2^{n+1}}-\dfrac{n}{2^n}
\\&=\dfrac{n+1}{2^{n+1}}-\dfrac{n\textcolor{red}{\times 2}}{2^n\textcolor{red}{\times 2}}
\\&=\dfrac{n+1}{2^{n+1}}-\dfrac{2n}{2^{n+1}}
\\&=\dfrac{\overbrace{-n+1}^{\ominus}}{\underbrace{2^{n+1}}_{\oplus}}
\end{align*}


Conclusion~: $v_{n+1}-v_{n}\leq 0,$ donc $\left(v_n\right)_{n\in\mathbb{N}^*}$ est décroissante.
\medskip

La suite $(v_n)_{n\in\mathbb{N}^*}$ est décroissante et elle est clairement minorée par 0. D'après le théorème de limite monotone, toute suite décroissante minorée converge, donc $(v_n)_{n\in\mathbb{N}^*}$ converge. De plus, comme elle est minorée par 0, sa limite $\ell$ est supérieure ou égale à $0.$
\item Pour tout $n\in\mathbb{N}^*~:$
\[v_{n+1}=\frac{n+1}{2^{n+1}}=\frac{1}{2}\times(n+1)\times\frac{1}{2^n}=\frac{1}{2}\times\frac{n+1}{\textcolor{red}{n}}\times\frac{\textcolor{red}{n}}{2^n}=\frac{1}{2}\times\left(1+\frac{1}{n}\right)\times v_n.\]
\item On sait que~:
\begin{itemize}
\item[\textbullet] $\lim\limits_{n\to +\infty}v_n=\lim\limits_{n\to +\infty} v_{n+1}=\ell.$
\item[\textbullet] $\lim\limits_{n\to +\infty}\frac{1}{n}=0.$
\end{itemize}

On \og passe à la limite  \fg~{} dans l'égalité de la question précédente~:
\[v_{n+1}=\frac{1}{2}\times\left(1+\frac{1}{n}\right)\times v_n\qquad\text{pour tout }n\in\mathbb{N},\]
donc
\[\ell=\frac{1}{2}\times\left(1+0\right)\times \ell.\]
On résout cette équation~:
\[\ell=\frac{1}{2}\ell\iff \ell-\frac{1}{2}\ell=0\iff \frac{1}{2}\ell=0\iff \ell=2\times 0\iff \ell=0.\]



Conclusion~: $\lim\limits_{n\to +\infty}v_n=0.$


\end{enumerate}

\end{exo}


\begin{exo}

La suite $(u_n)_{n\in\mathbb{N}}$ est définie par $u_0=10$ et la relation de récurrence

\[u_{n+1}=0,5u_n+2\] pour tout $n\in\mathbb{N}.$

\begin{enumerate}
\item Pour tout $n\in\mathbb{N},$ on note $\mathcal{P}_n$ la propriété \[4\leq u_{n+1}\leq u_{n}.\]

\begin{itemize}
\item[{\textbullet}] \textbf{Initialisation.} On prouve que $\mathcal{P}_0$ est vraie.


\[
\left.
    \begin{array}{ll}
        u_0&=10 \\
        u_1&=0,5\times 10+2=7\\
        4&\leq 7\leq 10
    \end{array}
\right \}\implies 4\leq u_{1}\leq u_{0}\implies\mathcal{P}_0~\text{est vraie}.
\]



\item[{\textbullet}] \textbf{Hérédité.} Soit $k\in\mathbb{N}$ tel que $\mathcal{P}_k$ soit vraie. On a donc
\[4\leq u_{k+1}\leq u_{k}.\]
%\medskip

\newtcolorbox{mybox}[1]{colback=green!10!white,colframe=green!80!white,fonttitle=\bfseries,title=#1}
\begin{mybox}{Objectif}{Prouver que $\mathcal{P}_{k+1}$ est vraie, c'est-à-dire que \[4\leq u_{k+2}\leq u_{k+1}.\]
}\end{mybox}



%\medskip

On part de 
\[4\leq u_{k+1}\leq u_{k}.\]


On multiplie par $\textcolor{red}{0,5}~:$

\begin{align*}4\textcolor{red}{\times 0,5}&\leq u_{k+1}\textcolor{red}{\times 0,5}\leq u_{k}\textcolor{red}{\times 0,5}\\
2&\leq 0,5 u_{k+1}\leq 0,5u_{k}.
\end{align*}

Puis on ajoute  $\textcolor{blue}{2}~:$

\begin{align*}
2\textcolor{blue}{+2}&\leq 0,5u_{k+1}\textcolor{blue}{+2}\leq 0,5u_k\textcolor{blue}{+2}\\
4&\leq u_{k+2}\leq u_{k+1}.
\end{align*}

La propriété $\mathcal{P}_{k+1}$ est donc vraie.
\item[{\textbullet}] \textbf{Conclusion.} $\mathcal{P}_0$ est vraie et $\mathcal{P}_n$ est héréditaire, donc elle est vraie pour tout $n\in\mathbb{N}.$
\end{itemize}
\item D'après la question précédente~:

\begin{itemize}
\item[{\textbullet}] $u_{n+1}\leq u_n$ pour tout $n\in\mathbb{N},$ donc $(u_n)_{n\in\mathbb{N}}$ est décroissante.
\item[{\textbullet}] $4\leq u_n$ pour tout $n\in\mathbb{N},$ donc $(u_n)_{n\in\mathbb{N}}$ est minorée par 4.
\end{itemize}

Or d'après le théorème de limite monotone, toute suite décroissante minorée converge, donc $(u_n)_{n\in\mathbb{N}}$ converge.
\item On note $\ell$ la limite de $\left(u_n\right)_{n\in\mathbb{N}}$ et \og on passe à la limite \fg~{} dans la formule de récurrence~:

\[u_{n+1}=0,5u_n+2\qquad\text{pour tout }n\in\mathbb{N},\] donc
\[\ell=0,5\ell+2.\]

On résout cette équation~:
\[\ell=0,5\ell+2\iff \ell-0,5\ell=2\iff 0,5\ell=2\iff\ell=\frac{2}{0,5}\iff \ell=4.\]

Conclusion~: $\lim\limits_{n\to +\infty}u_n=4.$

\end{enumerate}

\end{exo}








\begin{exo}

 


\begin{enumerate}
\item Pour tout $n\in\mathbb{N}~:$
\[w_{n+1}-w_n=\left(w_n+\frac{1}{w_n}\right)-w_n=\frac{1}{\underbrace{w_n}_{\oplus}}.\]

\medskip

$w_{n+1}-w_n\geq 0,$ donc $\left(w_n\right)_{n\in\mathbb{N}}$ est croissante.

\item La suite $\left(w_n\right)_{n\in\mathbb{N}}$ est croissante, donc d'après le théorème de limite monotone, il y a deux possibilités~:

\begin{itemize}
\item[{\textbullet}] soit elle est majorée, et dans ce cas elle converge~;
\item[{\textbullet}] soit elle n'est pas majorée, et dans ce cas elle a pour limite $+\infty.$
\end{itemize}

\medskip

Par conséquent, si $(\clubsuit)$ n'est pas vraie (et donc que $\left(w_n\right)_{n\in\mathbb{N}}$ n'a pas pour limite $+\infty$), elle converge  vers une limite finie $\ell.$

\medskip

De plus, $\left(w_n\right)_{n\in\mathbb{N}}$ étant croissante, $\ell\geq w_0=2.$
\item On \og passe à la limite \fg~{} dans la formule de récurrence~:

\[w_{n+1}=w_n+\frac{1}{w_n}\qquad\text{pour tout }n\in\mathbb{N},\] donc
\[\ell=\ell+\frac{1}{\ell}.\]

On résout cette équation~:
\[\ell=\ell+\frac{1}{\ell} \iff 0=\frac{1}{\ell}\iff 0\times\ell=\frac{1}{\ell}\times\ell\iff \underbrace{0=1}_{\text{absurde}}.\]

\medskip

Conclusion~: il n'y a pas de solution, donc en supposant que $(\clubsuit)$ est fausse, on aboutit à une absurdité. C'est donc que $(\clubsuit)$ est vraie~: \[\lim\limits_{n\to +\infty}w_n=+\infty.\]

\end{enumerate}

\end{exo}

\begin{exo}

On détermine les limites des suites de terme général~:

\begin{enumerate}
\item $u_n=0,8^n+(-0,2)^n.$
\[
\left.
    \begin{array}{ll}
        -1<0,8<1&\implies \lim\limits_{n\to +\infty} 0,8^n = 0 \\
        -1<-0,2<1&\implies \lim\limits_{n\to +\infty} (-0,2)^n = 0
    \end{array}
\right \}\implies \lim\limits_{n\to +\infty}\left(0,8^n+(-0,2)^n\right)=0+0=0.
\]

\item $v_n=4-3\times \left(\dfrac{2}{3}\right)^n.$

\[-1<\dfrac{2}{3}<1\implies \lim\limits_{n\to +\infty} \left(\dfrac{2}{3}\right)^n = 0\implies \lim\limits_{n\to +\infty} \left(4-3\times \left(\dfrac{2}{3}\right)^n\right)=4-3\times 0=4.\]
\item $w_n=\dfrac{0,5^n-1}{0,5^n+1}.$

\[-1<0,5<1\implies \lim\limits_{n\to +\infty} 0,5^n = 0\implies \lim\limits_{n\to +\infty} \left(\dfrac{0,5^n-1}{0,5^n+1}\right)=\dfrac{0-1}{0+1}=-1.\]
\item $x_n=\dfrac{2^n-1}{2^n+1}.$

\medskip

Cette fois, on ne peut pas conclure directement, car $2>1,$ donc $\lim\limits_{n\to +\infty} 2^n=+\infty.$ L'astuce consiste à mettre $2^n$ en facteur au numérateur et au dénominateur~:
\[x_n=\dfrac{2^n-1}{2^n+1}=\dfrac{\cancel{2^n}\left(1-\frac{1}{2^n}\right)}{\cancel{2^n}\left(1+\frac{1}{2^n}\right)}=\dfrac{1-\left(\frac{1}{2}\right)^n}{1+\left(\frac{1}{2}\right)^n}.\]

On peut alors conclure~:

\[-1<\dfrac{1}{2}<1\implies \lim\limits_{n\to +\infty} \left(\dfrac{1}{2}\right)^n = 0\implies \lim\limits_{n\to +\infty} x_n=\frac{1-0}{1+0}=1.\]

\end{enumerate}

\end{exo}

\begin{exo}

La suite $(u_n)_{n\in\mathbb{N}}$ est définie par $u_0=10$ et $u_{n+1}=0,5u_n+2$ pour tout $n\in\mathbb{N}.$ On pose \[v_n=u_n-4\] pour tout $n\in\mathbb{N}.$

\begin{enumerate}
\item Pour tout $n\in\mathbb{N}~:$



\begin{alignat*}{3}
&v_{n+1}&& =u_{n+1}-4 && \text{  (déf. de } (v_n)_{n\in\mathbb{N}})\\
& && =(0,5u_n+2)-4 && \text{  (rel. réc. pour } (u_n)_{n\in\mathbb{N}})\\
& && =0,5u_n-2 && \text{  (calcul)}\\
& && =0,5\left(u_n-\frac{2}{0,5}\right) && \text{  (factorisation)}\\
& && =0,5(u_n-4) && \text{  (calcul)}\\
& && =0,5v_n&& \text{  (déf. de } (v_n)_{n\in\mathbb{N}})\\
\end{alignat*}


Conclusion~: pour tout $n\in\mathbb{N},$ $v_{n+1}=0,5v_n,$
donc $(v_n)_{n\in\mathbb{N}}$ est géométrique de raison $q=0,5.$
\item La suite $(v_n)_{n\in\mathbb{N}}$ est géométrique de raison $q=0,5,$ et  $v_0=u_0-4=10-4=6,$ donc pour tout $n\in\mathbb{N}~:$
\[v_n=v_0\times q^n=6\times 0,5^n.\]
\item Enfin $v_n=u_n-4$ donc
\[u_n=v_n+4=6\times 0,5^n+4.\]
\item Pour tout $n\in\mathbb{N},$ $u_n=6\times 0,5^n+4.$
\[-1<0,5<1\implies \lim\limits_{n\to +\infty} 0,5^n = 0\implies \lim\limits_{n\to +\infty} u_n=6\times 0+4=4.\]

\end{enumerate}
\end{exo}



\begin{exo}

\begin{enumerate}
\item La zone grise fait 1/2 disque, la zone quadrillée 1/4 du disque, la zone hachurée 1/8, la zone noircie 1/16, etc. Si on continue indéfiniment, on obtiendra le disque entier, si bien que
\[\frac{1}{2}+\frac{1}{4}+\frac{1}{8}+\frac{1}{16}+\cdots =1,\] ou encore
\[\frac{1}{2}+\left(\frac{1}{2}\right)^2+\left(\frac{1}{2}\right)^3+\left(\frac{1}{2}\right)^4+\cdots =1.\]

Dans les questions suivantes, on justifie ce résultat de façon rigoureuse.

\item Soit $q$ un réel dans l'intervalle $\left]-1;1\right[.$ On sait  que pour tout $n\in\mathbb{N}~:$
\[1+q+q^2+\cdots+q^n=\dfrac{q^{n+1}-1}{q-1}~;\]

on a donc
\[q+q^2+\cdots+q^n=\dfrac{q^{n+1}-1}{q-1}-1.\]

Or
\[-1<q<1\implies \lim\limits_{n\to +\infty} q^{n+1} = 0\implies \lim\limits_{n\to +\infty} \left(\dfrac{q^{n+1}-1}{q-1}-1\right)=\dfrac{0-1}{q-1}-1=\dfrac{-1}{q-1}-1,\]

donc \[\lim\limits_{n\to +\infty}\left(q+q^2+\cdots+q^n\right)=\dfrac{-1}{q-1}-1.\]

\medskip

\textbf{Remarque~:} Il est agréable de réécrire la réponse sous une forme un peu différente~:

\[\dfrac{-1}{q-1}-1=\dfrac{-1}{q-1}-\dfrac{q-1}{q-1}=\dfrac{-1-(q-1)}{q-1}=\dfrac{-1-q+1}{q-1}=\dfrac{-q}{q-1}=\dfrac{-q\textcolor{red}{\times (-1)}}{(q-1)\textcolor{red}{\times (-1)}}=\dfrac{q}{1-q}.\]

\item On applique la formule de la question 2 à $q=\frac{1}{2}~:$
\[\lim\limits_{n\to +\infty}\left(\frac{1}{2}+\left(\frac{1}{2}\right)^2+\cdots+\left(\frac{1}{2}\right)^n\right)=\dfrac{\frac{1}{2}}{1-\frac{1}{2}}=1.\] On retrouve bien le résultat de la question 1.
\end{enumerate}

\end{exo}


\begin{exo}


Nous rencontrons pour la première fois une boucle \textbf{Tant que}. Il est sûrement nécessaire de faire quelques rappels.

\medskip


\setlength{\columnseprule}{1pt}

\begin{multicols}{4}

\begin{center}
\textbf{Programme}
\end{center}

\begin{lstlisting}
n=10
while n<=14:
	n=n+1
print(n)
\end{lstlisting}



\vspace*{2cm}


\columnbreak

\begin{center}
\textbf{Traduction en français}
\end{center}

\newtcolorbox{mybox}[1]{colback=white,colframe=black!80!white,fonttitle=\bfseries,title=#1}\begin{mybox}{}

$\text{n}=10$

$\text{Tant que n}\leq\text{14:}$

$\qquad\text{n}=\text{n}+1$

$\text{afficher n}$

\end{mybox}

\vspace*{2cm}


\columnbreak

\begin{center}
\textbf{Explications}
\end{center}

\begin{itemize}
\item[\textbullet] Au départ, $\text{n}=10~;$
\item[\textbullet] Puis, tant que $\text{n}\leq 14,$ n augmente de 1. \item[\textbullet] n prend donc successivement les valeurs 11, 12, 13, 14 et 15
\item[\textbullet] Lorsque $\text{n}=15,$ on sort de la boucle \textbf{Tant que}. La valeur affichée en sortie est 15 (valeur finale de n).
\end{itemize}

\columnbreak

\begin{center}
\textbf{Tableau}
\end{center}

Il est agréable d'expliquer avec un tableau~:

\begin{center}


\begin{tabular}{|c|c|} \hline
\textbf{Valeur de n}& $\underset{\mathbf{\text{n}\leq 14~?}}{\textbf{Boucle à continuer~?}}$\\ \hline
10& oui\\ \hline
$10+1=11$& oui\\ \hline
$11+1=12$&oui\\ \hline
$12+1=13$&oui\\ \hline
$13+1=14$&oui \\ \hline
$14+1=15$&non\\ \hline
\end{tabular}
\end{center}
%\vspace*{3.5cm}

\end{multicols}

\end{exo}

\begin{exo}

~{}

\begin{lstlisting}
def div(x):
	while x>=1:
		x=x/2
	return x
\end{lstlisting}



\medskip

\danger On s'arrête quand le résultat est strictement inférieur à 1, donc \textbf{on continue tant qu'il est supérieur ou égal à 1}.
\end{exo}

\begin{exo}

On reprend la suite $(u_n)_{n\in\mathbb{N}}$ de l'exercice 84~: $u_0=10$ et $u_{n+1}=0,5u_n+2$ pour tout $n\in\mathbb{N}.$ On a vu qu'elle était décroissante, et qu'elle convergeait vers 4.

\medskip

Calculons les premiers termes~:

\begin{align*}
u_0&=10\\
u_1&=0,5\times 10+2=7\\
u_2&=0,5\times 7+2=5,5\\
u_3&=0,5\times 5,5+2=4,75\\
u_4&=0,5\times 4,75+2=4,375
\end{align*}




On explique maintenant le programme avec un tableau~:

\medskip

\setlength{\columnseprule}{1pt}
\begin{multicols}{2}

\begin{center}
\textbf{Programme}
\end{center}




\begin{lstlisting}
def seuil():
	u=10
	n=0
	while u>=4.5:
		u=0.5*u+2
		n=n+1
	return n
\end{lstlisting}








\columnbreak

\begin{center}
\textbf{Tableau}
\end{center}

\medskip

\begin{center}


\begin{tabular}{|cc|c|c|} \hline
\textbf{u}& &\textbf{n}& $\underset{\mathbf{\text{u}\geq 4,5~?}}{\textbf{Boucle à continuer~?}}$\\ \hline
10&\textcolor{red}{$\leftarrow {u_0}$}& 0&oui\\ \hline
$0,5\times 10+2=7$&\textcolor{red}{$\leftarrow {u_1}$}& $0+1=1$&oui\\ \hline
$0,5\times 7+2=5,5$&\textcolor{red}{$\leftarrow {u_2}$}& $1+1=2$&oui\\ \hline
$0,5\times 5,5+2=4,75$&\textcolor{red}{$\leftarrow {u_3}$}& $2+1=3$&oui\\ \hline
$0,5\times 4,75+2=4,375$&\textcolor{red}{$\leftarrow {u_4}$}& $3+1=4$&non\\ \hline
\end{tabular}
\end{center}

\vspace*{0.25cm}


\end{multicols}


\medskip

Conclusion~: la valeur en sortie est $\text{n}=4.$ C'est l'indice du premier terme de la suite strictement inférieur à $4,5~:$
\begin{itemize}
\item[\textbullet] $u_0,$ $u_1,$ $u_2$ et $u_3$ sont supérieurs ou égaux à 4,5~;
\item[\textbullet] $u_4$ est strictement inférieur à 4,5.
\end{itemize}

\medskip

Il s'agit donc de déterminer l'indice (ou le rang) à partir duquel la suite descend en-dessous du seuil  4,5 -- d'où le nom de la fonction.

\end{exo}

\begin{exo}

On reprend les idées de l'exercice précédent~:

\begin{lstlisting}
def seuil():
	somme=100
	annees=0
	while somme<200:
		somme=somme*1.5
		annees=annees+1
	return annees
\end{lstlisting}




\end{exo}

\begin{exo}

~{}

\begin{lstlisting}
def syr(a):
	u=a
	L=[a]
	while u!=1:
		if u%2==0:
			u=u/2
		else
			u=3*u+1
		L.append(u)
	return L
\end{lstlisting}



\medskip

\textbf{Remarque~:} Quand on lance la fonction, on a une surprise désagréable~: toutes les réponses apparaissent avec un chiffre après la virgule. Par exemple, \textbf{syr(26)} renvoie
\[\left[26.0~,~13.0~,~40.0~,~20.0~,~10.0~,~5.0~,~16.0~,~8.0~,~4.0~,~2.0~,~1.0\right]\]

Pour éviter cela, il faut remplacer l'avant-dernière ligne par 


\begin{lstlisting}
L.append(int(u))
\end{lstlisting}


\medskip

Les élèves qui font (ou ont fait) la spécialité NSI reconnaîtront le type \textbf{int} (entier), alors que Python considère par défaut que a est de type \textbf{float} (réel).


\end{exo} 

\newpage

\section{Géométrie repérée dans l'espace}










\begin{exo}


\begin{enumerate}

\item Commençons par deux remarques concernant les figures en perspective cavalière~:
\begin{itemize}
%\item[\textbullet] Faire une figure en perspective cavalière, c'est dessiner son \og ombre \fg~{} lorsqu'elle est éclairée par des rayons lumineux parallèles.
\item[\textbullet] Dans une figure en perspective cavalière, on respecte la proportionnalité. Par exemple, un point situé au milieu d'un segment dans la réalité doit être représenté au milieu du segment sur la figure.
\item[\textbullet] Dans une figure en perspective cavalière, deux droites parallèles sont représentées par des droites parallèles.
\end{itemize}


\begin{center}
\psset{xunit=1.0cm,yunit=1.0cm,algebraic=true,dimen=middle,dotstyle=o,dotsize=5pt 0,linewidth=2.pt,arrowsize=3pt 2,arrowinset=0.25}
\begin{pspicture*}(1.12,0.16)(7.26,5.54)
\psline[linewidth=2.pt,linestyle=dashed,dash=2pt 2pt,linecolor=red](2.,3.25)(3.,5.)
\psline[linewidth=2.pt,linestyle=dashed,dash=2pt 2pt,linecolor=red](3.,5.)(6.,2.75)
\psline[linewidth=2.pt,linestyle=dashed,dash=2pt 2pt](3.,2.)(2.,1.)
\psline[linewidth=2.pt](2.,1.)(5.,1.)
\psline[linewidth=2.pt](5.,1.)(6.,2.)
\psline[linewidth=2.pt,linestyle=dashed,dash=2pt 2pt](6.,2.)(3.,2.)
\psline[linewidth=2.pt](3.,5.)(2.,4.)
\psline[linewidth=2.pt](2.,4.)(5.,4.)
\psline[linewidth=2.pt](5.,4.)(6.,5.)
\psline[linewidth=2.pt](6.,5.)(3.,5.)
\psline[linewidth=2.pt,linestyle=dashed,dash=2pt 2pt](3.,5.)(3.,2.)
\psline[linewidth=2.pt](2.,4.)(2.,1.)
\psline[linewidth=2.pt](5.,4.)(5.,1.)
\psline[linewidth=2.pt](6.,5.)(6.,2.)
\psline[linewidth=2.pt,linecolor=red](5.,1.)(6.,2.75)
\psline[linewidth=2.pt,linecolor=red](5.,1.)(2.,3.25)
\psdots[dotsize=1pt 0,dotstyle=*](3.,2.)
\rput[bl](2.64,2.06){$D$}
\psdots[dotsize=1pt 0,dotstyle=*](2.,1.)
\rput[bl](1.7,0.88){$A$}
\psdots[dotsize=1pt 0,dotstyle=*](5.,1.)
\rput[bl](5.1,0.74){$B$}
\psdots[dotsize=1pt 0,dotstyle=*](6.,2.)
\rput[bl](6.08,2.04){$C$}
\psdots[dotsize=1pt 0,dotstyle=*](3.,5.)
\rput[bl](2.62,5.1){$H$}
\psdots[dotsize=1pt 0,dotstyle=*](2.,4.)
\rput[bl](1.74,4.08){$E$}
\psdots[dotsize=1pt 0,dotstyle=*](5.,4.)
\rput[bl](4.78,4.1){$F$}
\psdots[dotsize=1pt 0,dotstyle=*](6.,5.)
\rput[bl](6.08,5.04){$G$}
\psdots[dotsize=1pt 0,dotstyle=*](2.,3.25)
\rput[bl](1.76,3.3){$I$}
\psdots[dotsize=1pt 0,dotstyle=*](6.,2.75)
\rput[bl](6.18,2.76){$J$}
\end{pspicture*}
\end{center}


\item Dans le repère $\left(A,\overrightarrow{AB},\overrightarrow{AD},\overrightarrow{AE}\right)~:$
\[B(1;0;0),\quad H(0;1;1),\quad I(0;0;~0,75),\quad J(1;1;~0,25).\]

\begin{itemize}
\item[\textbullet] \textbf{Parallélogramme.} On prouve que deux vecteurs sont égaux~:



\begin{align*}&\overrightarrow{BJ}\begin{pmatrix} x_J-x_B\\y_J-y_B\\z_J-z_B \end{pmatrix}\qquad 
\overrightarrow{BJ}\begin{pmatrix} 1-1\\1-0\\0,25-0\end{pmatrix}\qquad \overrightarrow{BJ}\begin{pmatrix} 0
\\1\\0,25 \end{pmatrix}\\
&\overrightarrow{IH}\begin{pmatrix} x_H-x_I\\y_H-y_I\\z_H-z_I \end{pmatrix}\qquad 
\overrightarrow{IH}\begin{pmatrix} 0-0\\1-0\\1-0,75\end{pmatrix}\qquad \overrightarrow{IH}\begin{pmatrix} 0
\\1\\0,25 \end{pmatrix}
\end{align*}

\medskip

Conclusion~: $\overrightarrow{BJ}=\overrightarrow{IH},$ donc $BJHI$ est un parallélogramme.

\medskip

\danger La colinéarité des vecteurs ne suffit pas, il doivent être \textbf{égaux}.

\item[\textbullet] \textbf{Losange.} Un losange a 4 côtés égaux. Or 

\begin{align*}
BI&=\sqrt{\left(x_I-x_B\right)^2+\left(y_I-y_B\right)^2+\left(z_I-z_B\right)^2}=\sqrt{(0-1)^2+(0-0)^2+(0,75-0)^2}=\sqrt{\np{1,5625}}\\
BJ&=\sqrt{\left(x_J-x_B\right)^2+\left(y_J-y_B\right)^2+\left(z_J-z_B\right)^2}=\sqrt{(1-1)^2+(1-0)^2+(0,25-0)^2}=\sqrt{\np{1,0625}}
\end{align*}

\medskip

Conclusion~:  $BJHI$ a deux côtés de longueurs différentes, donc \textbf{ce n'est pas} un losange.
\end{itemize}
\end{enumerate}



\end{exo}

\newpage
\begin{exo}

\setlength{\columnseprule}{1pt}

\begin{multicols}{2}
\begin{enumerate}
\item ~{}


\begin{center}
\psset{xunit=1cm,yunit=1cm,algebraic=true,dimen=middle,dotstyle=o,dotsize=5pt 0,linewidth=2.pt,arrowsize=3pt 2,arrowinset=0.25}
\begin{pspicture*}(-1.56,-2.18)(4.04,5.54)
\psline[linewidth=2.pt,linestyle=dashed,dash=2pt 2pt](0.,2.)(-1.,1.)
\psline[linewidth=2.pt](-1.,1.)(2.,1.)
\psline[linewidth=2.pt](2.,1.)(3.,2.)
\psline[linewidth=2.pt,linestyle=dashed,dash=2pt 2pt](3.,2.)(0.,2.)
\psline[linewidth=2.pt](0.,5.)(-1.,4.)
\psline[linewidth=2.pt](-1.,4.)(2.,4.)
\psline[linewidth=2.pt](2.,4.)(3.,5.)
\psline[linewidth=2.pt](3.,5.)(0.,5.)
\psline[linewidth=2.pt,linestyle=dashed,dash=2pt 2pt](0.,5.)(0.,2.)
\psline[linewidth=2.pt](-1.,4.)(-1.,1.)
\psline[linewidth=2.pt](2.,4.)(2.,1.)
\psline[linewidth=2.pt](3.,5.)(3.,2.)
\psline[linewidth=2.pt,linecolor=red](-1.,4.)(3.,5.)
\psline[linewidth=2.pt,linestyle=dashed,dash=2pt 2pt,linecolor=red](0.5,1.)(2.5,1.5)
\psline[linewidth=2.pt,linecolor=blue](-1.,4.)(2.,-2.)
\psline[linewidth=2.pt,linecolor=blue](2.,-2.)(3.,5.)
\psdots[dotsize=1pt 0,dotstyle=*](0.,2.)
\rput[bl](0.12,2.16){$A$}
\psdots[dotsize=1pt 0,dotstyle=*](-1.,1.)
\rput[bl](-1.34,1.1){$B$}
\psdots[dotsize=1pt 0,dotstyle=*](2.,1.)
\rput[bl](1.82,0.7){$C$}
\psdots[dotsize=1pt 0,dotstyle=*](3.,2.)
\rput[bl](3.08,2.04){$D$}
\psdots[dotsize=1pt 0,dotstyle=*](0.,5.)
\rput[bl](-0.32,5.02){$E$}
\psdots[dotsize=1pt 0,dotstyle=*](-1.,4.)
\rput[bl](-1.18,4.06){$F$}
\psdots[dotsize=1pt 0,dotstyle=*](2.,4.)
\rput[bl](1.7,4.12){$G$}
\psdots[dotsize=1pt 0,dotstyle=*](3.,5.)
\rput[bl](3.08,5.04){$H$}
\psdots[dotsize=1pt 0,dotstyle=*](2.5,1.5)
\rput[bl](2.8,1.46){$K$}
\psdots[dotsize=1pt 0,dotstyle=*](0.5,1.)
\rput[bl](0.3,0.6){$J$}
\psdots[dotsize=1pt 0,dotstyle=*,linecolor=blue](2.,-2.)
\rput[bl](2.28,-1.92){\blue{$L$}}
\end{pspicture*}
\end{center}

\item Dans le repère $\left(A,\overrightarrow{AB},\overrightarrow{AD},\overrightarrow{AE}\right)~:$
\begin{align*}&A(0;0;0),\quad B(1;0;0),\quad C(1;1;0),\quad D(0;1;0),\\
& E(0;0;1),\quad F(1;0;1),\quad G(1;1;1),\quad H(0;1;1),\\
& J(1;~0,5~;0),\quad K(0,5~;1;0).
\end{align*}
\item On utilise la colinéarité~:



\begin{align*}&\overrightarrow{FH}\begin{pmatrix} x_H-x_F\\y_H-y_F\\z_H-z_F \end{pmatrix}\qquad 
\overrightarrow{FH}\begin{pmatrix} 0-1\\1-0\\1-1\end{pmatrix}\qquad \overrightarrow{FH}\begin{pmatrix} -1
\\1\\0 \end{pmatrix}\\
&\overrightarrow{JK}\begin{pmatrix} x_K-x_J\\y_K-y_J\\z_K-z_J \end{pmatrix}\qquad 
\overrightarrow{JK}\begin{pmatrix} 0,5-1\\1-0,5\\0-0\end{pmatrix}\qquad \overrightarrow{JK}\begin{pmatrix} -0,5
\\0,5\\0 \end{pmatrix}
\end{align*}

\medskip

On voit que $\overrightarrow{FH}=2\overrightarrow{JK},$ donc $\overrightarrow{FH}$ et $\overrightarrow{JK}$ sont colinéaires~; et donc les droites $(FH)$ et $(JK)$ sont parallèles.
\item Les droites $(FH)$ et $(JK)$ étant parallèles, les quatre points $F,$ $H,$ $J$ et $K$ sont coplanaires (dans un même plan)~; et les droites $(FJ)$ et $(HK)$ aussi. Pour prouver qu'elles se coupent, il suffit donc de prouver qu'elles ne sont pas parallèles. On utilise à nouveau la colinéarité~:

\begin{align*}&\overrightarrow{FJ}\begin{pmatrix} x_J-x_F\\y_J-y_F\\z_J-z_F \end{pmatrix}\qquad 
\overrightarrow{FJ}\begin{pmatrix} 1-1\\0,5-0\\0-1\end{pmatrix}\qquad \overrightarrow{FJ}\begin{pmatrix} 0
\\0,5\\-1 \end{pmatrix}\\
&\overrightarrow{HK}\begin{pmatrix} x_K-x_H\\y_K-y_H\\z_K-z_H \end{pmatrix}\qquad 
\overrightarrow{HK}\begin{pmatrix} 0,5-0\\1-1\\0-1\end{pmatrix}\qquad \overrightarrow{HK}\begin{pmatrix} 0,5
\\0\\-1 \end{pmatrix}
\end{align*}


\medskip

Les vecteurs $\overrightarrow{FJ}$ et $\overrightarrow{HK}$ ne sont pas colinéaires, puisque le tableau \[\begin{tabular}{|c|c|}\hline
   $0$ & $0,5$  \\ \hline
   $0,5$ & $0$  \\ \hline
   $-1$ & $-1$  \\ \hline
   \end{tabular}\]
   
   n'est pas un tableau de proportionnalité. Les droites $(FJ)$ et $(HK)$ sont donc bien sécantes. On a noté $L$ leur point d'intersection sur la figure.

\end{enumerate}

\end{multicols}
\end{exo}

\begin{exo}

\setlength{\columnseprule}{1pt}

\begin{multicols}{2}

\begin{enumerate}
\item ~{}


\begin{center}
\psset{xunit=0.75cm,yunit=0.75cm,algebraic=true,dimen=middle,dotstyle=o,dotsize=5pt 0,linewidth=2.pt,arrowsize=3pt 2,arrowinset=0.25}
\begin{pspicture*}(-6.5,-0.3)(4.58,8.56)
\psline[linewidth=2.pt,linestyle=dashed,dash=2pt 2pt](1.,2.)(-1.,1.)
\psline[linewidth=2.pt](-1.,1.)(2.,1.)
\psline[linewidth=2.pt](2.,1.)(4.,2.)
\psline[linewidth=2.pt,linestyle=dashed,dash=2pt 2pt](4.,2.)(1.,2.)
\psline[linewidth=2.pt](1.,5.)(-1.,4.)
\psline[linewidth=2.pt](-1.,4.)(2.,4.)
\psline[linewidth=2.pt](2.,4.)(4.,5.)
\psline[linewidth=2.pt](4.,5.)(1.,5.)
\psline[linewidth=2.pt,linestyle=dashed,dash=2pt 2pt](1.,5.)(1.,2.)
\psline[linewidth=2.pt,linecolor=red](-1.,4.)(-1.,1.)
\psline[linewidth=2.pt](2.,4.)(2.,1.)
\psline[linewidth=2.pt,linecolor=red](4.,5.)(4.,2.)
\psline[linewidth=2.pt,linecolor=red](4.,5.)(4.,8.)
\psline[linewidth=2.pt,linecolor=red](-6.,0.)(-1.,1.)
\psline[linewidth=2.pt,linecolor=red,linestyle=dashed,dash=2pt 2pt](-1.,1.)(4.,2.)
\psline[linewidth=2.pt,linecolor=red](-6.,0.)(4.,8.)
\psdots[dotsize=1pt 0,dotstyle=*](1.,2.)
\rput[bl](1.12,2.16){$A$}
\psdots[dotsize=1pt 0,dotstyle=*](-1.,1.)
\rput[bl](-1.44,1.1){$B$}
\psdots[dotsize=1pt 0,dotstyle=*](2.,1.)
\rput[bl](1.82,0.6){$C$}
\psdots[dotsize=1pt 0,dotstyle=*](4.,2.)
\rput[bl](4.08,2.04){$D$}
\psdots[dotsize=1pt 0,dotstyle=*](1.,5.)
\rput[bl](0.68,5.02){$E$}
\psdots[dotsize=1pt 0,dotstyle=*](-1.,4.)
\rput[bl](-1.28,4.06){$F$}
\psdots[dotsize=1pt 0,dotstyle=*](2.,4.)
\rput[bl](1.7,4.12){$G$}
\psdots[dotsize=1pt 0,dotstyle=*](4.,5.)
\rput[bl](4.08,5.04){$H$}
\psdots[dotsize=1pt 0,dotstyle=*](4.,8.)
\rput[bl](4.08,8.04){$K$}
\psdots[dotsize=1pt 0,dotstyle=*](-6.,0.)
\rput[bl](-6.2,0.12){$J$}
\end{pspicture*}
\end{center}

\item Dans le repère $\left(A,\overrightarrow{AB},\overrightarrow{AD},\overrightarrow{AE}\right)~:$
\[F(1;0;1),\quad K(0;1;2).\]

\medskip


On détermine ensuite les coordonnées de $~J~:$

On sait que $\overrightarrow{DJ}=2\overrightarrow{DB}.$ Or 
\[\overrightarrow{DJ}\begin{pmatrix} x_J-x_D\\y_J-y_D\\z_J-z_D\end{pmatrix}\quad \overrightarrow{DJ}\begin{pmatrix} x_J-0\\y_J-1\\z_J-0\end{pmatrix}\quad
\overrightarrow{DJ}\begin{pmatrix} x_J\\y_J-1\\z_J\end{pmatrix}\] et

\[\overrightarrow{DB}\begin{pmatrix} x_B-x_D\\y_B-y_D\\z_B-z_D\end{pmatrix}\quad
\overrightarrow{DB}\begin{pmatrix} 1-0\\0-1\\0-0\end{pmatrix}\quad
\overrightarrow{DB}\begin{pmatrix} 1\\-1\\0\end{pmatrix}\quad \text{et donc}\quad
2\overrightarrow{DB}\begin{pmatrix} 2\\-2\\0\end{pmatrix}.\]

On a donc \begin{align*}
x_J&=2\\ y_J-1&=-2\iff y_J=-2+1=-1\\ z_J&=0.\end{align*} Conclusion~: $J(2;-1;0).$

\item On utilise la colinéarité~:

\begin{align*}
&\overrightarrow{KF}\begin{pmatrix} x_F-x_K\\y_F-y_K\\z_F-z_K \end{pmatrix}\qquad 
\overrightarrow{KF}\begin{pmatrix} 1-0\\0-1\\1-2\end{pmatrix}\qquad \overrightarrow{KF}\begin{pmatrix} 1
\\-1\\-1 \end{pmatrix}\\
&\overrightarrow{KJ}\begin{pmatrix} x_J-x_K\\y_J-y_K\\z_J-z_K \end{pmatrix}\qquad 
\overrightarrow{KJ}\begin{pmatrix} 2-0\\-1-1\\0-2\end{pmatrix}\qquad \overrightarrow{KJ}\begin{pmatrix} 2
\\-2\\-2 \end{pmatrix}
\end{align*}

\medskip

On voit que $\overrightarrow{KJ}=2\overrightarrow{KF},$ donc $\overrightarrow{KJ}$ et $\overrightarrow{KF}$ sont colinéaires~; et donc les points $K,$ $F$ et $J$ sont alignés.
\end{enumerate}

\medskip

\textbf{Remarque~:} On a tracé une partie de la figure en rouge pour mettre en évidence une configuration de Thalès.

\end{multicols}
\end{exo}

\begin{exo}

\setlength{\columnseprule}{1pt}

\begin{multicols}{2}

\danger Sur la figure ci-contre, on représente les plans $(ABC)$ et $(MNC)$ par des triangles colorés. Mais ces plans \og ne s'arrêtent pas aux triangles \fg~{}, ils continuent indéfiniment dans toutes les directions.

\medskip

Pour construire la droite d'intersection des plans $(ABC)$ et $(MNC),$ il suffit d'avoir deux points de cette droite~; et donc deux points communs à chacun des deux plans. Le point $\mathbf{C},$ évidemment, appartient à chacun des deux plans $(AB\mathbf{C})$ et $(MN\mathbf{C})~;$ il reste donc à trouver un deuxième point.

\medskip

Les droites $(MN)$ et $(AB)$ sont toutes deux dans le plan $(ABD)$ et elles ne sont pas parallèles d'après l'énoncé~; elles se coupent donc en un point $K.$ Ce point $K$ appartenant à $(AB),$ il appartient également au plan $(ABC).$ Mais $K$ appartient aussi à ($MN),$ donc au plan $(MNC).$ Finalement, $K$ appartient à chacun des deux plans $(ABC)$ et $(MNC).$

\medskip

Chacun des points $C$ et $K$ appartient à la fois aux plans $(ABC)$ et $(MNC),$ donc l'intersection de ces deux plans est la droite $(CK).$


\begin{center}
\psset{xunit=1.0cm,yunit=1.0cm,algebraic=true,dimen=middle,dotstyle=o,dotsize=5pt 0,linewidth=2.pt,arrowsize=3pt 2,arrowinset=0.25}
\begin{pspicture*}(-0.36,-2.64)(6.22,3.24)
\pspolygon[linewidth=2.pt,linecolor=white,fillcolor=blue!20!white,fillstyle=solid,opacity=0.1](0.18,-0.44)(4.66,1.08)(5.215081510069391,-2.232147656126393)
\pspolygon[linewidth=2.pt,linecolor=white,fillcolor=red!20!white,fillstyle=solid,opacity=0.1](1.3403076923076926,1.545415384615385)(3.1109036346307413,-0.18075964040081338)(4.66,1.08)
%\psline[linewidth=2.pt,linecolor=blue](0.18,-0.44)(3.72,-1.7)
\psline[linewidth=2.pt,linecolor=red](3.1109036346307413,-0.18075964040081338)(4.66,1.08)
\psline[linewidth=2.pt,linecolor=red,linestyle=dashed,dash=2pt 2pt](1.3403076923076926,1.545415384615385)(4.66,1.08)
\psline[linewidth=2.pt](3.72,-1.7)(4.66,1.08)
\psline[linewidth=2.pt](4.66,1.08)(1.98,2.64)
\psline[linewidth=2.pt](1.98,2.64)(0.18,-0.44)
\psline[linewidth=2.pt](1.98,2.64)(3.72,-1.7)
\psline[linewidth=2.pt,linestyle=dashed,dash=2pt 2pt](0.18,-0.44)(4.66,1.08)
\psline[linewidth=2.pt,linecolor=blue](0.18,-0.44)(5.215081510069391,-2.232147656126393)
\psline[linewidth=2.pt,linecolor=red](1.3403076923076926,1.545415384615385)(5.215081510069391,-2.232147656126393)
\psline[linewidth=2.pt,linecolor=green](5.215081510069391,-2.232147656126393)(4.66,1.08)
\psdots[dotsize=1pt 0,dotstyle=*](0.18,-0.44)
\rput[bl](-0.16,-0.36){$A$}
\psdots[dotsize=1pt 0,dotstyle=*](3.72,-1.7)
\rput[bl](3.64,-2.12){$B$}
\psdots[dotsize=1pt 0,dotstyle=*](4.66,1.08)
\rput[bl](4.74,1.12){$C$}
\psdots[dotsize=1pt 0,dotstyle=*](1.98,2.64)
\rput[bl](1.58,2.54){$D$}
\psdots[dotsize=1pt 0,dotstyle=*](1.3403076923076926,1.545415384615385)
\rput[bl](0.9,1.44){$M$}
\psdots[dotsize=1pt 0,dotstyle=*](3.1109036346307413,-0.18075964040081338)
\rput[bl](2.68,-0.38){$N$}
\psdots[dotsize=1pt 0,dotstyle=*](5.215081510069391,-2.232147656126393)
\rput[bl](5.3,-2.2){$K$}
\end{pspicture*}
\end{center}

\end{multicols}

\end{exo}

\begin{exo}

\setlength{\columnseprule}{1pt}

\begin{multicols}{2}

Le point $G$ est sur la face $ABC,$ donc la droite $(AG)$ coupe le segment $\left[BC\right]$ en un point $J.$ Les cinq points $A,$ $I,$ $D,$ $G,$ $J$ sont dans un même plan (le plan $(ADJ)$), coloré en rose sur la figure. Il s'ensuit que les droites $(IG)$ et $(DJ)$ sont dans ce plan. De plus, $(IG)$ et $(DJ)$ ne sont pas parallèles (sinon la droite $(IG)$ serait parallèle à une droite du plan $(BCD),$ et donc parallèle au plan $(BCD)$ -- ce que l'énoncé exclut). On en déduit que $(IG)$ et $(DJ)$ se coupent en un point $K$\footnote{Deux droites non parallèles de l'espace ne se coupent pas forcément. Ce qui fait que $(IG)$ et $(DJ)$ se coupent, c'est qu'elles sont coplanaires ($=$dans un même plan) et non parallèles.}.

\medskip

Par construction, le point $K$ appartient à la droite $(IG).$ Il appartient également à la droite $(DJ),$ qui est incluse dans le plan $(BCD)~;$ le point $K$ appartient donc au plan $(BCD).$

\medskip

Conclusion~: le point $K$ appartient à la droite $(IG)$ et au plan $(BCD),$ donc c'est leur point d'intersection.


\begin{center}
\psset{xunit=1.0cm,yunit=1.0cm,algebraic=true,dimen=middle,dotstyle=o,dotsize=5pt 0,linewidth=2.pt,arrowsize=3pt 2,arrowinset=0.25}
\begin{pspicture*}(0.48,0.02)(7.66,6.56)
\pspolygon[linewidth=0.pt,linecolor=white,fillcolor=red!20!white,fillstyle=solid,opacity=0.1](1.,3.)(3.68,5.98)(4.420263113767394,2.753162862476028)(6.506887384729074,1.011014716945642)
\psline[linewidth=2.pt](3.68,5.98)(3.72,0.6)
\psline[linewidth=2.pt](3.72,0.6)(1.,3.)
\psline[linewidth=2.pt](1.,3.)(3.68,5.98)
\psline[linewidth=2.pt](3.68,5.98)(6.24,3.46)
\psline[linewidth=2.pt](6.24,3.46)(3.72,0.6)
\psline[linewidth=2.pt](3.68,5.98)(4.66752,1.67536)
\psline[linewidth=2.pt](1.,3.)(6.506887384729074,1.011014716945642)
\psline[linewidth=2.pt,linestyle=dashed,dash=2pt 2pt](1.,3.)(6.24,3.46)
\psline[linewidth=2.pt](4.420263113767394,2.753162862476028)(6.506887384729074,1.011014716945642)
\psline[linewidth=2.pt,linestyle=dashed,dash=2pt 2pt](4.420263113767394,2.753162862476028)(2.34,4.49)
\psdots[dotsize=1pt 0,dotstyle=*](3.68,5.98)
\rput[bl](3.42,6.1){$A$}
\psdots[dotsize=1pt 0,dotstyle=*](3.72,0.6)
\rput[bl](3.76,0.24){$B$}
\psdots[dotsize=1pt 0,dotstyle=*](1.,3.)
\rput[bl](0.84,3.24){$D$}
\psdots[dotsize=1pt 0,dotstyle=*](6.24,3.46)
\rput[bl](6.32,3.5){$C$}
\psdots[dotsize=1pt 0,dotstyle=*](4.66752,1.67536)
\rput[bl](4.58,1.1){$J$}
\psdots[dotsize=1pt 0,dotstyle=*](2.34,4.49)
\rput[bl](2.12,4.58){$I$}
\psdots[dotsize=1pt 0,dotstyle=*](4.420263113767394,2.753162862476028)
\rput[bl](4.5,2.8){$G$}
\psdots[dotsize=1pt 0,dotstyle=*](6.506887384729074,1.011014716945642)
\rput[bl](6.58,1.06){$K$}
\end{pspicture*}
\end{center}


\end{multicols}


\end{exo}

\newpage




\begin{exo}

\setlength{\columnseprule}{1pt}

\begin{multicols}{2}

\begin{enumerate}
\item ~{}


\begin{center}
\psset{xunit=1.0cm,yunit=1.0cm,algebraic=true,dimen=middle,dotstyle=o,dotsize=5pt 0,linewidth=2.pt,arrowsize=3pt 2,arrowinset=0.25}
\begin{pspicture*}(-1.56,0.32)(3.96,5.54)
\pspolygon[linewidth=2.pt,linecolor=white,fillcolor=blue!20!white,fillstyle=solid,opacity=0.1](0.,5.)(2.,4.)(1.25,1.)(-0.25,1.75)
\psline[linewidth=2.pt,linestyle=dashed,dash=2pt 2pt](0.,2.)(-1.,1.)
\psline[linewidth=2.pt](-1.,1.)(2.,1.)
\psline[linewidth=2.pt](2.,1.)(3.,2.)
\psline[linewidth=2.pt,linestyle=dashed,dash=2pt 2pt](3.,2.)(0.,2.)
\psline[linewidth=2.pt](0.,5.)(-1.,4.)
\psline[linewidth=2.pt](-1.,4.)(2.,4.)
\psline[linewidth=2.pt](2.,4.)(3.,5.)
\psline[linewidth=2.pt](3.,5.)(0.,5.)
\psline[linewidth=2.pt,linestyle=dashed,dash=2pt 2pt](0.,5.)(0.,2.)
\psline[linewidth=2.pt](-1.,4.)(-1.,1.)
\psline[linewidth=2.pt](2.,4.)(2.,1.)
\psline[linewidth=2.pt](3.,5.)(3.,2.)
\psline[linewidth=2.pt,linecolor=blue](0.,5.)(2.,4.)
\psline[linewidth=2.pt,linecolor=blue](2.,4.)(1.25,1.)
\psline[linewidth=2.pt,linecolor=blue,linestyle=dashed,dash=2pt 2pt](1.25,1.)(-0.25,1.75)
\psline[linewidth=2.pt,linecolor=blue,linestyle=dashed,dash=2pt 2pt](-0.25,1.75)(0.,5.)
\psline[linewidth=2.pt,linecolor=red](3.,5.)(2.75,1.75)
\psdots[dotsize=1pt 0,dotstyle=*](0.,2.)
\rput[bl](0.12,2.16){$A$}
\psdots[dotsize=1pt 0,dotstyle=*](-1.,1.)
\rput[bl](-1.36,0.84){$B$}
\psdots[dotsize=1pt 0,dotstyle=*](2.,1.)
\rput[bl](2.04,0.78){$C$}
\psdots[dotsize=1pt 0,dotstyle=*](3.,2.)
\rput[bl](3.08,2.04){$D$}
\psdots[dotsize=1pt 0,dotstyle=*](0.,5.)
\rput[bl](-0.32,5.02){$E$}
\psdots[dotsize=1pt 0,dotstyle=*](-1.,4.)
\rput[bl](-1.18,4.06){$F$}
\psdots[dotsize=1pt 0,dotstyle=*](2.,4.)
\rput[bl](1.8,4.12){$G$}
\psdots[dotsize=1pt 0,dotstyle=*](3.,5.)
\rput[bl](3.08,5.04){$H$}
\psdots[dotsize=1pt 0,dotstyle=*](2.75,1.75)
\rput[bl](2.8,1.35){$I$}
\psdots[dotsize=1pt 0,dotstyle=*](1.25,1.)
\rput[bl](1.04,0.62){$J$}
\psdots[dotsize=1pt 0,dotstyle=*](-0.25,1.75)
\rput[bl](-0.54,1.78){$K$}
\end{pspicture*}
\end{center}

\item Dans le repère $\left(A,\overrightarrow{AB},\overrightarrow{AD},\overrightarrow{AE}\right)~:$
\small
\[H(0;1;1),\quad I(0,25~;1;0),\quad J(1;~0,75~;0),\quad E(0;0;1),\quad G(1;1;1).\]
 \normalsize
\item Les faces $ABCD$ et $EFGH$ sont parallèles, donc le plan $(EGJ)$ (en bleu) les coupe suivant des segments parallèles. Pour construire la section du cube par le plan $(EGJ),$ on trace donc la parallèle à $(EG)$ passant par $J.$ Elle coupe $\left[AB\right]$ en $K.$

\medskip

Le point $K$ appartient au segment $\left[AB\right],$ donc il a des coordonnées de la forme $(x;0;0).$ Pour déterminer la valeur de $x,$ on utilise la colinéarité~:

\begin{align*}&\overrightarrow{EG}\begin{pmatrix} x_G-x_E\\y_G-y_E\\z_G-z_E\end{pmatrix}\qquad 
\overrightarrow{EG}\begin{pmatrix} 1-0\\1-0\\1-1\end{pmatrix}\qquad \overrightarrow{EG}\begin{pmatrix} 1
\\1\\0 \end{pmatrix}\\
&\overrightarrow{JK}\begin{pmatrix} x_K-x_J\\y_K-y_J\\z_K-z_J \end{pmatrix}\qquad 
\overrightarrow{JK}\begin{pmatrix} x-1\\0-0,75\\0-0\end{pmatrix}\qquad \overrightarrow{JK}\begin{pmatrix} x-1
\\-0,75\\0 \end{pmatrix}
\end{align*}

\medskip

Or $(EG)$ est parallèle à $(JK),$ donc $\overrightarrow{EG}$ et $\overrightarrow{JK} $ sont colinéaires, et le tableau \[\begin{tabular}{|c|c|}\hline
   $1$ & $x-1$  \\ \hline
  $1$ & $-0,75$  \\ \hline
   $0$ & $0$  \\ \hline
   \end{tabular}\]
   
   est un tableau de proportionnalité. On a donc \[1\times (-0,75)=1\times (x-1).\] On développe et on résout~:
   \[-0,75=x-1\iff x=-0,75+1=0,25.\]
   
   Conclusion~: $K(0,25~;0;0).$


\item Pour prouver que la droite $(HI)$ est parallèle au plan $(EGJ),$ il suffit de prouver qu'elle est parallèle à une droite de ce plan. La bonne candidate est la droite $(EK),$ et l'outil est la colinéarité. On obtient (en accélérant un petit peu)~:

\[\overrightarrow{HI}\begin{pmatrix} 0,25
\\0\\-1 \end{pmatrix}\qquad,\qquad \overrightarrow{EK}\begin{pmatrix} 0,25
\\0\\-1 \end{pmatrix}.\]

Conclusion~: les vecteurs $\overrightarrow{HI}$ et $\overrightarrow{EK}$ sont égaux (et donc colinéaires~!), donc $(HI)$ est parallèle à $(EK)~;$ et donc au plan $(EGJ).$


\end{enumerate}

\end{multicols}
\end{exo}


\begin{exo}


\setlength{\columnseprule}{1pt}

\begin{multicols}{2}
Les faces $ABFE$ et $DCGH$ sont parallèles, donc le plan $(EBI)$ les coupe suivant des segments parallèles. Pour construire la section du parallélépipède par le plan $(EBI),$ il suffit donc de tracer la parallèle à $(EB)$ passant par $I.$ Elle coupe $\left[GC\right]$ en $J,$ et la section est le quadrilatère $EBJI.$

\begin{center}
\psset{xunit=1.0cm,yunit=1.0cm,algebraic=true,dimen=middle,dotstyle=o,dotsize=5pt 0,linewidth=2.pt,arrowsize=3pt 2,arrowinset=0.25}
\begin{pspicture*}(-1.56,0.6)(7.76,5.54)
\pspolygon[linewidth=2.pt,linecolor=white,fillcolor=blue!20!white,fillstyle=solid,opacity=0.1](-1.,1.)(1.,5.)(6.,4.5)(5.,2.5)
\psline[linewidth=2.pt,linecolor=blue,linestyle=dashed,dash=2pt 2pt](-1.,1.)(1.,5.)
\psline[linewidth=2.pt,linecolor=blue](1.,5.)(6.,4.5)
\psline[linewidth=2.pt,linecolor=blue](6.,4.5)(5.,2.5)
\psline[linewidth=2.pt,linecolor=blue](5.,2.5)(-1.,1.)
\psline[linewidth=2.pt,linestyle=dashed,dash=2pt 2pt](1.,2.)(-1.,1.)
\psline[linewidth=2.pt](-1.,1.)(5.,1.)
\psline[linewidth=2.pt](5.,1.)(7.,2.)
\psline[linewidth=2.pt,linestyle=dashed,dash=2pt 2pt](7.,2.)(1.,2.)
\psline[linewidth=2.pt](1.,5.)(-1.,4.)
\psline[linewidth=2.pt](-1.,4.)(5.,4.)
\psline[linewidth=2.pt](5.,4.)(7.,5.)
\psline[linewidth=2.pt](7.,5.)(1.,5.)
\psline[linewidth=2.pt,linestyle=dashed,dash=2pt 2pt](1.,5.)(1.,2.)
\psline[linewidth=2.pt](-1.,4.)(-1.,1.)
\psline[linewidth=2.pt](5.,4.)(5.,1.)
\psline[linewidth=2.pt](7.,5.)(7.,2.)
\psdots[dotsize=1pt 0,dotstyle=*](1.,2.)
\rput[bl](1.12,2.16){$A$}
\psdots[dotsize=1pt 0,dotstyle=*](-1.,1.)
\rput[bl](-1.36,0.84){$B$}
\psdots[dotsize=1pt 0,dotstyle=*](5.,1.)
\rput[bl](5.14,0.74){$C$}
\psdots[dotsize=1pt 0,dotstyle=*](7.,2.)
\rput[bl](7.08,2.04){$D$}
\psdots[dotsize=1pt 0,dotstyle=*](1.,5.)
\rput[bl](0.68,5.02){$E$}
\psdots[dotsize=1pt 0,dotstyle=*](-1.,4.)
\rput[bl](-1.18,4.06){$F$}
\psdots[dotsize=1pt 0,dotstyle=*](5.,4.)
\rput[bl](4.7,4.12){$G$}
\psdots[dotsize=1pt 0,dotstyle=*](7.,5.)
\rput[bl](7.08,5.04){$H$}
\psdots[dotsize=1pt 0,dotstyle=*](6.,4.5)
\rput[bl](6.28,4.28){$I$}
\psdots[dotsize=1pt 0,dotstyle=*](5.,2.5)
\rput[bl](5.22,2.3){$J$}
\end{pspicture*}
\end{center}

\end{multicols}

\end{exo}

\newpage

\begin{exo}


Dans l'espace muni d'un repère orthonormé de centre $O,$ on considère les points $A(4;0;0),$ $B(0;3;0),$ $C(0;0;4)$ et $D(4;3;0).$

\setlength{\columnseprule}{1pt}

\begin{multicols}{2}
\begin{enumerate}
\item ~{}


\begin{center}
\psset{xunit=1.0cm,yunit=1.0cm,algebraic=true,dimen=middle,dotstyle=o,dotsize=5pt 0,linewidth=2.pt,arrowsize=3pt 2,arrowinset=0.25}
\begin{pspicture*}(-1.2317356201376504,-0.6694401600652891)(5.774359079670679,5.501490933143359)
\pspolygon[linewidth=1.pt,linecolor=blue,fillcolor=blue!25!white,fillstyle=solid,opacity=0.1](2.6666666666666665,0.)(2.309778562339617,0.)(2.450617203277112,0.28167728187498925)(2.8075053076041616,0.28167728187498925)
\pspolygon[linewidth=1.pt,linecolor=blue,fillcolor=blue!25!white,fillstyle=solid,opacity=0.1](3.3333333333333335,1.3333333333333333)(3.0702295980701004,1.2280918392280402)(3.0702295980701004,1.5695113214409624)(3.3333333333333335,1.6747528155462557)
\rput[tl](4.8154675313525415,0.30491738096765536){$x$}
\rput[tl](2.062520828116597,3.800231734514408){$y$}
\rput[tl](-0.5357659479712601,4.851919239121396){$z$}
\rput[tl](0.933503359935564,-0.1590624004766039){1}
\rput[tl](-0.3037760572491299,1.04728503127847){1}
\rput[tl](0.4540575857764951,0.70703319155268){1}
\psline[linewidth=2.pt,linestyle=dashed,dash=2pt 2pt,linecolor=red](1.,2.)(5.,2.)
\psline[linewidth=2.pt,linecolor=red](5.,2.)(3.98,0.)
\psline[linewidth=2.pt,linecolor=red](0.,4.)(5.,2.)
\psline[linewidth=2.pt,linestyle=dashed,dash=2pt 2pt,linecolor=red](0.,4.)(1.,2.)
\psline[linewidth=2.pt,linecolor=red](0.,4.)(3.98,0.)
\psline[linewidth=2.pt,linestyle=dashed,dash=2pt 2pt](0.,0.)(5.,2.)
\psline[linewidth=2.pt,linestyle=dashed,dash=2pt 2pt](2.6666666666666665,0.)(3.3333333333333335,1.3333333333333333)
\psline[linewidth=2.pt,linestyle=dashed,dash=2pt 2pt](3.3333333333333335,2.6666666666666665)(3.3333333333333335,1.3333333333333333)
\psline[linewidth=2.pt,linecolor=blue](2.6666666666666665,0.)(2.309778562339617,0.)
\psline[linewidth=2.pt,linecolor=blue](2.309778562339617,0.)(2.450617203277112,0.28167728187498925)
\psline[linewidth=2.pt,linecolor=blue](2.450617203277112,0.28167728187498925)(2.8075053076041616,0.28167728187498925)
\psline[linewidth=2.pt,linecolor=blue](2.8075053076041616,0.28167728187498925)(2.6666666666666665,0.)
\psline[linewidth=2.pt,linecolor=blue](3.3333333333333335,1.3333333333333333)(3.0702295980701004,1.2280918392280402)
\psline[linewidth=2.pt,linecolor=blue](3.0702295980701004,1.2280918392280402)(3.0702295980701004,1.5695113214409624)
\psline[linewidth=2.pt,linecolor=blue](3.0702295980701004,1.5695113214409624)(3.3333333333333335,1.6747528155462557)
\psline[linewidth=2.pt,linecolor=blue](3.3333333333333335,1.6747528155462557)(3.3333333333333335,1.3333333333333333)
\psline[linewidth=2.pt,linecolor=red](0.,0.)(3.98,0.)
\psline[linewidth=2.pt,linestyle=dashed,dash=2pt 2pt,linecolor=red](0.,0.)(1.,2.)
\psline[linewidth=2.pt,linestyle=dashed,dash=2pt 2pt]{->}(1.,2.)(2.,4.)
\psline[linewidth=2.pt]{->}(0.,4.)(0.,5.)
\psline[linewidth=2.pt,linecolor=red](0.,0.)(0.,4.)
\psline[linewidth=2.pt]{->}(3.98,0.)(5.,0.)
\psdots[dotsize=1pt 0,dotstyle=*](0.,0.)
\rput[bl](-0.3656400281083646,-0.08173243690256067){$O$}
\psdots[dotsize=2pt 0,dotstyle=+](3.98,0.)
\rput[bl](3.980303924752873,-0.2982563349098816){$A$}
\psdots[dotsize=2pt 0,dotstyle=+](1.,0.)
\psdots[dotsize=2pt 0,dotstyle=+](0.,1.)
\psdots[dotsize=2pt 0,dotstyle=+](0.3333333333333333,0.6666666666666666)
\psdots[dotsize=2pt 0,dotstyle=+](0.6666666666666666,1.3333333333333333)
\psdots[dotsize=2pt 0,dotstyle=+](1.,2.)
\rput[bl](0.63,1.8515166524485194){$B$}
\psdots[dotsize=2pt 0,dotstyle=+](2.,0.)
\psdots[dotsize=2pt 0,dotstyle=+](0.,2.)
\psdots[dotsize=2pt 0,dotstyle=+](0.,3.)
\psdots[dotsize=1pt 0,dotstyle=*](0.,4.)
\rput[bl](-0.37,4.00128963980692){$C$}
\psdots[dotsize=2pt 0,dotstyle=+](5.,2.)
\rput[bl](5.06292341478948,2.0680405504558403){$D$}
\psdots[dotsize=2pt 0,dotstyle=+](3.3333333333333335,2.6666666666666665)
\rput[bl](3.3925962015901434,2.733078237192612){$M$}
\psdots[dotsize=2pt 0,dotstyle=+](3.3333333333333335,1.3333333333333333)
\rput[bl](3.454460172449378,1.1091490021377046){$P$}
\psdots[dotsize=2pt 0,dotstyle=+](2.6666666666666665,0.)
\rput[bl](2.696626529423753,-0.31372232762469027){$H$}
\end{pspicture*}
\end{center}

\medskip
On prend la même unité de longueur pour graduer les axes $(Ox)$ et $(Oz),$ qui sont vus de face, puis une unité arbitraire (plus petite) pour graduer l'axe $(Oy),$ qui est une ligne de fuite.

\columnbreak
\item On utilise la colinéarité~:

\begin{align*}&\overrightarrow{CD}\begin{pmatrix} 4-0
\\3-0\\0-4 \end{pmatrix}\qquad \overrightarrow{CD}\begin{pmatrix} 4
\\3\\-4 \end{pmatrix}\\
&\overrightarrow{CM}\begin{pmatrix} \dfrac{8}{3}-0
\\2-0\\ \dfrac{4}{3}-4 \end{pmatrix}\qquad \overrightarrow{CM}\begin{pmatrix} \dfrac{8}{3}
\\2\\ \dfrac{4}{3}-\dfrac{12}{3} \end{pmatrix}
\qquad \overrightarrow{CM}\begin{pmatrix} \dfrac{8}{3}
\\2\\ -\dfrac{8}{3} \end{pmatrix}.
\end{align*}

On voit que $\overrightarrow{CM}=\dfrac{2}{3}\overrightarrow{CD},$ donc $M$ appartient au segment $\left[CD\right]~;$ et il est aux $2/3$ de ce segment en partant de $C.$

\item \begin{itemize}
\item[\textbullet] La parallèle à $(Oz)$ passant par $M$ coupe $\left[OD\right]$ en $P,$ dont les coordonnées sont $\left(\dfrac{8}{3};2;0\right).$
 \item[\textbullet] La parallèle à $(Oy)$ passant par $P$ coupe $\left[OA\right]$ en $H,$ dont les coordonnées sont $\left(\dfrac{8}{3};0;0\right).$
\end{itemize}

\end{enumerate}
\end{multicols}

\end{exo}



\begin{exo}

~{}


\begin{multicols}{2}
\begin{center}
\psset{xunit=1.0cm,yunit=1.0cm,algebraic=true,dimen=middle,dotstyle=o,dotsize=5pt 0,linewidth=2.pt,arrowsize=3pt 2,arrowinset=0.25}
\begin{pspicture*}(-1.16,0.44)(5.42,5.62)
\psline[linewidth=2.pt,linestyle=dashed,dash=2pt 2pt,linecolor=green](1.,5.)(1.,2.)
\psline[linewidth=2.pt,linestyle=dashed,dash=2pt 2pt,linecolor=red](0.,1.)(4.,2.)
\pspolygon[linewidth=2.pt,linecolor=white,hatchcolor=blue,fillstyle=hlines,hatchangle=135.0,hatchsep=0.3](1.,2.)(0.,1.)(3.,1.)(4.,2.)
\psline[linewidth=2.pt,linecolor=blue,linestyle=dashed,dash=2pt 2pt](1.,2.)(0.,1.)
\psline[linewidth=2.pt](0.,1.)(3.,1.)
\psline[linewidth=2.pt](3.,1.)(4.,2.)
\psline[linewidth=2.pt,linecolor=blue,linestyle=dashed,dash=2pt 2pt](4.,2.)(1.,2.)
\psline[linewidth=2.pt](1.,5.)(0.,4.)
\psline[linewidth=2.pt](0.,4.)(3.,4.)
\psline[linewidth=2.pt](3.,4.)(4.,5.)
\psline[linewidth=2.pt](4.,5.)(1.,5.)
\psline[linewidth=2.pt](0.,4.)(0.,1.)
\psline[linewidth=2.pt](3.,4.)(3.,1.)
\psline[linewidth=2.pt](4.,5.)(4.,2.)
\psdots[dotsize=1pt 0,dotstyle=*](1.,2.)
\rput[bl](0.64,2.08){$A$}
\psdots[dotsize=1pt 0,dotstyle=*](0.,1.)
\rput[bl](-0.4,0.82){$B$}
\psdots[dotsize=1pt 0,dotstyle=*](3.,1.)
\rput[bl](3.06,0.74){$C$}
\psdots[dotsize=1pt 0,dotstyle=*](4.,2.)
\rput[bl](4.06,1.68){$D$}
\psdots[dotsize=1pt 0,dotstyle=*](1.,5.)
\rput[bl](0.72,5.02){$E$}
\psdots[dotsize=1pt 0,dotstyle=*](0.,4.)
\rput[bl](-0.24,4.02){$F$}
\psdots[dotsize=1pt 0,dotstyle=*](3.,4.)
\rput[bl](3.14,3.8){$G$}
\psdots[dotsize=1pt 0,dotstyle=*](4.,5.)
\rput[bl](4.08,5.04){$H$}
\end{pspicture*}

\textit{Figure pour les questions 1 et 2}
\end{center}

\columnbreak


\begin{center}
\psset{xunit=1.0cm,yunit=1.0cm,algebraic=true,dimen=middle,dotstyle=o,dotsize=5pt 0,linewidth=2.pt,arrowsize=3pt 2,arrowinset=0.25}
\begin{pspicture*}(-1.16,0.44)(5.,5.62)
\pspolygon[linewidth=2.pt,linecolor=white,hatchcolor=green,fillstyle=hlines,hatchangle=130.0,hatchsep=0.2](1.,2.)(3.,1.)(3.,4.)(1.,5.)
\psline[linewidth=2.pt,linestyle=dashed,dash=2pt 2pt,linecolor=red](0.,1.)(4.,2.)
\psline[linewidth=2.pt,linestyle=dashed,dash=2pt 2pt](1.,2.)(0.,1.)
\psline[linewidth=2.pt](0.,1.)(3.,1.)
\psline[linewidth=2.pt](3.,1.)(4.,2.)
\psline[linewidth=2.pt,linestyle=dashed,dash=2pt 2pt](4.,2.)(1.,2.)
\psline[linewidth=2.pt](1.,5.)(0.,4.)
\psline[linewidth=2.pt](0.,4.)(3.,4.)
\psline[linewidth=2.pt](3.,4.)(4.,5.)
\psline[linewidth=2.pt](4.,5.)(1.,5.)
\psline[linewidth=2.pt](0.,4.)(0.,1.)
\psline[linewidth=2.pt,linestyle=dashed,dash=2pt 2pt,linecolor=green](1.,5.)(1.,2.)
\psline[linewidth=2.pt](3.,4.)(3.,1.)
\psline[linewidth=2.pt](4.,5.)(4.,2.)
\psline[linewidth=2.pt,linestyle=dashed,dash=2pt 2pt,linecolor=green](1.,2.)(3.,1.)
\psline[linewidth=2.pt,linecolor=green](3.,4.)(1.,5.)
\psline[linewidth=2.pt,linestyle=dashed,dash=2pt 2pt,linecolor=orange](1.,2.)(3.,4.)
\psdots[dotsize=1pt 0,dotstyle=*](1.,2.)
\rput[bl](0.64,2.08){$A$}
\psdots[dotsize=1pt 0,dotstyle=*](0.,1.)
\rput[bl](-0.4,0.82){$B$}
\psdots[dotsize=1pt 0,dotstyle=*](3.,1.)
\rput[bl](3.06,0.74){$C$}
\psdots[dotsize=1pt 0,dotstyle=*](4.,2.)
\rput[bl](4.06,1.68){$D$}
\psdots[dotsize=1pt 0,dotstyle=*](1.,5.)
\rput[bl](0.72,5.02){$E$}
\psdots[dotsize=1pt 0,dotstyle=*](0.,4.)
\rput[bl](-0.24,4.02){$F$}
\psdots[dotsize=1pt 0,dotstyle=*](3.,4.)
\rput[bl](3.14,3.8){$G$}
\psdots[dotsize=1pt 0,dotstyle=*](4.,5.)
\rput[bl](4.08,5.04){$H$}
\end{pspicture*}

\textit{Figure pour les questions 3 et 4}

\end{center}

\end{multicols}

\begin{enumerate}
\item La droite $(AE)$ est orthogonale aux droites $(AB)$ et $(AD)$ (deux droites sécantes du plan $(ABD)$), car $ABFE$ et $ADHE$ sont deux carrés. La droite $(AE)$ est donc orthogonale au plan $(ABD).$
\item La droite $(AE)$ est orthogonale au plan $(ABD),$ donc elle est orthogonale à toute droite de ce plan. En particulier, elle est orthogonale à la droite $(BD).$
\item La droite $(BD)$ est orthogonale à la droite $(AE)$ d'après la question précédente, mais aussi à la droite $(AC)$ (car $(BD)$ et $(AC)$ sont deux diagonales du carré $ABCD$).

Conclusion~: $(BD)$ est orthogonale à deux droites sécantes du plan $(AEC),$ donc elle est orthogonale au plan $(AEC).$
\item Comme $(BD)$ est orthogonale au plan $(AEC),$ elle est orthogonale à toute droite de ce plan, et donc en particulier à la droite $(AG).$
\end{enumerate}



\end{exo}

\begin{exo}

Dans le repère $\left(A,\overrightarrow{AB},\overrightarrow{AD},\overrightarrow{AE}\right)~:$ \[A(0;0;0),\quad B(1;0;0),\quad D(0;1;0),\quad G(1;1;1),\] donc

\[\overrightarrow{AG}\begin{pmatrix} x_G-x_A\\y_G-y_A\\z_G-z_A \end{pmatrix}\qquad
\overrightarrow{AG}\begin{pmatrix} 1-0\\1-0\\1-0\end{pmatrix}\qquad \overrightarrow{AG}\begin{pmatrix} 1
\\1\\1 \end{pmatrix}\begin{matrix}\textcolor{red}{x}\\\textcolor{red}{y}\\\textcolor{red}{z}\end{matrix} \qquad\qquad \overrightarrow{BD}\begin{pmatrix} x_D-x_B\\y_D-y_B\\z_D-z_B \end{pmatrix}\qquad
\overrightarrow{BD}\begin{pmatrix} 0-1\\1-0\\0-0\end{pmatrix}\qquad \overrightarrow{BD}\begin{pmatrix} -1
\\1\\0 \end{pmatrix}\begin{matrix}\textcolor{red}{x'}\\\textcolor{red}{y'}\\\textcolor{red}{z'}\end{matrix}.\]
On calcule le produit scalaire~:
\[\overrightarrow{AG}\cdot \overrightarrow{BD}=\textcolor{red}{xx'+yy'+zz'}=1\times (-1)+1\times 1+1\times 0=0.\] On en déduit que $(AG)$ est orthogonale à $(BD).$

\end{exo}



\begin{exo}

\setlength{\columnseprule}{1pt}

\begin{multicols}{2}
\begin{enumerate}
\item ~{}


\begin{center}
\psset{xunit=1.0cm,yunit=1.0cm,algebraic=true,dimen=middle,dotstyle=o,dotsize=5pt 0,linewidth=2.pt,arrowsize=3pt 2,arrowinset=0.25}
\begin{pspicture*}(-1.56,0.62)(3.82,5.54)
\pspolygon[linewidth=2.pt,linecolor=white,fillcolor=red!30!white,fillstyle=solid,opacity=0.1](-0.5,4.5)(-0.5,1.5)(3.,2.)
\psline[linewidth=2.pt,linestyle=dashed,dash=2pt 2pt,linecolor=red](3.,2.)(-0.5,1.5)
\psline[linewidth=2.pt,linestyle=dashed,dash=2pt 2pt,linecolor=red](-0.5,1.5)(-0.5,4.5)
\psline[linewidth=2.pt,linestyle=dashed,dash=2pt 2pt,linecolor=red](-0.5,4.5)(3.,2.)
\psline[linewidth=2.pt,linestyle=dashed,dash=2pt 2pt,linecolor=red](3.,2.)(-0.5,4.5)
\psline[linewidth=2.pt,linestyle=dashed,dash=2pt 2pt](0.,2.)(-1.,1.)
\psline[linewidth=2.pt](-1.,1.)(2.,1.)
\psline[linewidth=2.pt](2.,1.)(3.,2.)
\psline[linewidth=2.pt,linestyle=dashed,dash=2pt 2pt](3.,2.)(0.,2.)
\psline[linewidth=2.pt](0.,5.)(-1.,4.)
\psline[linewidth=2.pt](-1.,4.)(2.,4.)
\psline[linewidth=2.pt](2.,4.)(3.,5.)
\psline[linewidth=2.pt](3.,5.)(0.,5.)
\psline[linewidth=2.pt,linestyle=dashed,dash=2pt 2pt](0.,5.)(0.,2.)
\psline[linewidth=2.pt](-1.,4.)(-1.,1.)
\psline[linewidth=2.pt](2.,4.)(2.,1.)
\psline[linewidth=2.pt](3.,5.)(3.,2.)
\psline[linewidth=2.pt,linestyle=dashed,dash=2pt 2pt,linecolor=blue](0.,2.)(0.5,1.)
\psdots[dotsize=1pt 0,dotstyle=*](0.,2.)
\rput[bl](0.12,2.16){$A$}
\psdots[dotsize=1pt 0,dotstyle=*](-1.,1.)
\rput[bl](-1.34,1.1){$B$}
\psdots[dotsize=1pt 0,dotstyle=*](2.,1.)
\rput[bl](1.82,0.7){$C$}
\psdots[dotsize=1pt 0,dotstyle=*](3.,2.)
\rput[bl](3.08,2.04){$D$}
\psdots[dotsize=1pt 0,dotstyle=*](0.,5.)
\rput[bl](-0.32,5.02){$E$}
\psdots[dotsize=1pt 0,dotstyle=*](-1.,4.)
\rput[bl](-1.18,4.06){$F$}
\psdots[dotsize=1pt 0,dotstyle=*](2.,4.)
\rput[bl](1.7,4.12){$G$}
\psdots[dotsize=1pt 0,dotstyle=*](3.,5.)
\rput[bl](3.08,5.04){$H$}
\psdots[dotsize=1pt 0,dotstyle=*](-0.5,4.5)
\rput[bl](-0.72,4.56){$I$}
\psdots[dotsize=1pt 0,dotstyle=*](-0.5,1.5)
\rput[bl](-0.84,1.58){$J$}
\psdots[dotsize=1pt 0,dotstyle=*](0.5,1.)
\rput[bl](0.58,1.12){$K$}
\end{pspicture*}
\end{center}

\item Dans le repère $\left(A,\overrightarrow{AB},\overrightarrow{AD},\overrightarrow{AE}\right)~:$
\small
\[A(0;0;0),\quad K(1;~0,5~;0),\quad D(0;1;0),\quad I(0,5~;0;1),\quad J(0,5~;0;0).\]
\normalsize

On calcule les coordonnées de $\overrightarrow{AK}$ et $\overrightarrow{DI},$ puis le produit scalaire $\overrightarrow{AK}\cdot \overrightarrow{DI}~:$

\[\overrightarrow{AK}\begin{pmatrix} 1-0\\0,5-0\\0-0 \end{pmatrix}\quad \overrightarrow{AK}\begin{pmatrix} 1\\0,5\\0 \end{pmatrix}\qquad \overrightarrow{DI}\begin{pmatrix} 0,5-0\\0-1\\1-0 \end{pmatrix}\quad \overrightarrow{DI}\begin{pmatrix} 0,5\\-1\\1 \end{pmatrix}.\]
donc
\[\overrightarrow{AK}\cdot \overrightarrow{DI}=1\times 0,5+0,5\times (-1)+0\times 1=0.\]

On en déduit que les droites $(AK)$ et $(DI)$ sont orthogonales.
\item On calcule les coordonnées de $\overrightarrow{DJ},$ puis le produit scalaire $\overrightarrow{AK}\cdot \overrightarrow{DJ}~:$

\[\overrightarrow{DJ}\begin{pmatrix} 0,5-0\\0-1\\0-0 \end{pmatrix}\quad \overrightarrow{DJ}\begin{pmatrix} 0,5\\-1\\0 \end{pmatrix}\]
donc
\[\overrightarrow{AK}\cdot \overrightarrow{DJ}=1\times 0,5+0,5\times (-1)+0\times 0=0.\]
On en déduit que les droites $(AK)$ et $(DJ)$ sont orthogonales.

\medskip

Conclusion~: $(AK)$ est orthogonale à deux droites sécantes du plan $(DIJ)$ (les droites $(DI)$ et $(DJ)$), donc $(AK)$ est orthogonale au plan $(DIJ).$
\end{enumerate}

\end{multicols}
\end{exo}


\begin{exo}

On reprend l'énoncé de l'exercice précédent. On calcule une mesure à $1^{\circ}$ près de l'angle $\widehat{DIJ}.$

\medskip

Pour cela, on utilise la formulation du produit scalaire avec le cosinus~:
\[\overrightarrow{ID}\cdot\overrightarrow{IJ}=ID\times IJ\times \cos\widehat{DIJ}.\]

On calcule donc successivement (à ce stade de la leçon, on s'autorise à aller un peu plus vite)~:

\medskip
\begin{itemize}
\item[\textbullet] $\overrightarrow{ID}\begin{pmatrix} -0,5\\1\\-1\end{pmatrix}\qquad \overrightarrow{IJ}\begin{pmatrix} 0\\0\\-1\end{pmatrix}.$
\medskip
\item[\textbullet] $\overrightarrow{ID}\cdot\overrightarrow{IJ}=-0,5\times 0+1\times 0+(-1)\times (-1)=1.$
\medskip
\item[\textbullet] $ID=\sqrt{\left(x_D-x_I\right)^2+\left(y_D-y_I\right)^2+\left(z_D-z_I\right)^2}=\sqrt{(0-0,5)^2+(1-0)^2+(0-1)^2}=\sqrt{2,25}=1,5.$
\medskip
\item[\textbullet] On trouve de même $IJ=1.$
\end{itemize}

\medskip

On remplace dans la formule~:
\begin{align*}
\overrightarrow{ID}\cdot\overrightarrow{IJ}&=ID\times IJ\times \cos\widehat{DIJ}\\
1&=1,5\times 1\times \cos\widehat{DIJ}\\
\dfrac{1}{1,5}&=\cos\widehat{DIJ}\\
\dfrac{2}{3}&=\cos\widehat{DIJ}.
\end{align*}

On calcule le membre de gauche, puis on utilise la touche $\arccos$ de la calculatrice (attention, elle doit être en mode degrés). On obtient
\[\widehat{DIJ}=\arccos\left(\dfrac{2}{3}\right)\approx 48^{\circ}.\]

\end{exo}



\newpage


\begin{exo}

\begin{enumerate}
\item Pour prouver que les points $A(2;4;5)$ et $B(6;3;0)$ sont situés sur une même sphère $\mathcal{S}$ de centre $O(0;0;0),$ il suffit de prouver que les longueurs $OA$ et $OB$ sont égales. C'est bien le cas puisque~:

\begin{align*}
 OA&=\sqrt{\left(x_A-x_O\right)^2+\left(y_A-y_O\right)^2+\left(z_A-z_O\right)^2}=\sqrt{(2-0)^2+(4-0)^2+(5-0)^2}=\sqrt{45},\\
  OB&=\sqrt{\left(x_B-x_O\right)^2+\left(y_B-y_O\right)^2+\left(z_B-z_O\right)^2}=\sqrt{(6-0)^2+(3-0)^2+(0-0)^2}=\sqrt{45}.
  \end{align*}

\item

\setlength{\columnseprule}{1pt}

\begin{multicols}{2}

Calculer la distance géodésique entre $A$ et $B$ (notée $\wideparen{AB}$ dans la suite) revient à calculer l'angle $\widehat{AOB}.$ En effet, si cet angle valait $180^{\circ},$ alors $\wideparen{AB}$ serait le demi-périmètre de la sphère et on aurait $\wideparen{AB}=\pi\times R=\pi\times \sqrt{45}.$ Dans le cas général, il y a proportionnalité~:

\medskip

\begin{center}

\begin{tabular}{|c|c|c|}\hline
$\widehat{AOB}$&$180^{\circ}$&~à calculer~\\ \hline
$\wideparen{AB}$&$\pi\times \sqrt{45}$&~?~\\ \hline
\end{tabular}

\end{center}


\begin{center}
\psset{xunit=1.0cm,yunit=1.0cm,algebraic=true,dimen=middle,dotstyle=o,dotsize=5pt 0,linewidth=2.pt,arrowsize=3pt 2,arrowinset=0.25}
\begin{pspicture*}(-2.12,-2.06)(2.2,2.1)
\pscircle[linewidth=2.pt](0.,0.){2.}
\psplot[linewidth=2.pt,linestyle=dashed,dash=2pt 2pt,plotpoints=200]{-1.9999990400000067}{1.9999960457152133}{sqrt(1.0-x^(2.0)/4.0)}
\psplot[linewidth=2.pt,plotpoints=200]{-1.9999990400000067}{-0.99}{-sqrt(1.0-x^(2.0)/4.0)}
\psplot[linewidth=2.pt,plotpoints=200]{0.9999990400000067}{1.99}{-sqrt(1.0-x^(2.0)/4.0)}
\psplot[linewidth=2.pt,plotpoints=200,linecolor=red]{-1}{1}{-sqrt(1.0-x^(2.0)/4.0)}
\psline[linewidth=2.pt](-1.,-0.8660254037844386)(0.,0.)
\psline[linewidth=2.pt](0.,0.)(1.,-0.8660254037844386)
\pscustom[linewidth=2.pt,linecolor=red,fillcolor=red!25!white,fillstyle=solid,opacity=0.25]{
\parametricplot{-2.1}{-1.05}{0.8*cos(t)+0.|0.4*sin(t)+0.}
\lineto(0.,0.)\closepath}
\psdots[dotsize=1pt 0,dotstyle=*](0.,0.)
\rput[bl](-0.2,0.08){$O$}
\psdots[dotsize=1pt 0,dotstyle=*](-1.,-0.8660254037844386)
\rput[bl](-1.16,-1.24){$A$}
\psdots[dotsize=1pt 0,dotstyle=*](1.,-0.8660254037844386)
\rput[bl](1.08,-1.12){$B$}
\rput[tl](-0.1,-1.2){\red{$\wideparen{AB}$}}

\end{pspicture*}
\end{center}
\medskip


%\begin{center}
%\psset{xunit=1.0cm,yunit=1.0cm,algebraic=true,dimen=middle,dotstyle=o,dotsize=5pt 0,linewidth=2.pt,arrowsize=3pt 2,arrowinset=0.25}
%\begin{pspicture*}(-2.8,-2.98)(2.76,3.36)
%\psline[linewidth=2.pt](0.,0.)(1.4455927369592911,2.039671943928863)
%\psline[linewidth=2.pt](0.6248921151710598,1.0892244233384776)(0.8207006217882312,0.9504475205903856)
%\psline[linewidth=2.pt](0.,0.)(-1.059487685394873,2.264395249177275)
%\psline[linewidth=2.pt](-0.6384348146579455,1.081342215689684)(-0.4210528707369271,1.1830530334875917)
%\pscustom[linewidth=2.pt,linecolor=red,fillcolor=red!30!white,fillstyle=solid,opacity=0.1]{
%\parametricplot{0.9542300654085691}{2.00842750233533}{0.6*cos(t)+0.|0.6*sin(t)+0.}
%\lineto(0.,0.)\closepath}
%\rput[tl](0.82,1.14){$\sqrt{45}$}
%\parametricplot[linewidth=2.pt,linecolor=red]{0.954230065408569}{2.00842750233533}{1.*2.5*cos(t)+0.*2.5*sin(t)+0.|0.*2.5*cos(t)+1.*2.5*sin(t)+0.}
%\parametricplot[linewidth=2.pt]{-4.274757804844256}{0.954230065408569}{1.*2.5*cos(t)+0.*2.5*sin(t)+0.|0.*2.5*cos(t)+1.*2.5*sin(t)+0.}
%\rput[tl](-0.17,3){\red{$\wideparen{AB}$}}
%\psdots[dotsize=1pt 0,dotstyle=*](0.,0.)
%\rput[bl](0.02,-0.54){$O$}
%\psdots[dotsize=1pt 0,dotstyle=*](1.4455927369592911,2.039671943928863)
%\rput[bl](1.52,2.08){$A$}
%\psdots[dotsize=1pt 0,dotstyle=*](-1.059487685394873,2.264395249177275)
%\rput[bl](-1.32,2.42){$B$}
%\end{pspicture*}
%\end{center}

%\medskip

Pour calculer $\widehat{AOB},$ on utilise le produit scalaire et on raisonne comme dans l'exercice précédent~:

\medskip
\begin{itemize}
\item[\textbullet] $\overrightarrow{OA}\begin{pmatrix} 2\\4\\5\end{pmatrix}\qquad ,\qquad \overrightarrow{OB}\begin{pmatrix} 6\\3\\0\end{pmatrix}.$
\medskip
\item[\textbullet] $\overrightarrow{OA}\cdot\overrightarrow{OB}=2\times 6+4\times 3+5\times 0=24.$
\medskip
\item[\textbullet] $OA=OB=\sqrt{45}.$

\end{itemize}

\medskip

On remplace dans la formule~:
\begin{align*}
\overrightarrow{OA}\cdot\overrightarrow{OB}&=OA\times OB\times \cos\widehat{AOB}\\
24&=\sqrt{45}\times\sqrt{45}\times \cos\widehat{AOB}\\
\frac{24}{45}&=\cos\widehat{AOB}\\
\frac{8}{15}&=\cos\widehat{AOB}.
\end{align*}

D'où $\widehat{AOB}=\arccos\left(\frac{8}{15}\right).$

\medskip

Finalement on obtient le tableau 

\medskip

\begin{center}

\begin{tabular}{|c|c|c|}\hline
$\widehat{AOB}$&$180^{\circ}$&$\arccos\left(\frac{8}{15}\right)$\\ \hline
$\wideparen{AB}$&$\pi\times \sqrt{45}$&~?~\\ \hline
\end{tabular}

\end{center}

\medskip

Et donc~:
 
\[\wideparen{AB}=\pi\times \sqrt{45}\times \arccos\left(\frac{8}{15}\right)\div 180\approx 6,76\]

(on prend garde à travailler en mode degrés et on n'arrondit qu'une seule fois, à la fin du calcul, pour éviter que des erreurs d'arrondis s'ajoutent).

\end{multicols}


\end{enumerate}


\end{exo}

%\newpage

\section{Continuité et limites de suites}




\begin{exo}

La fonction $f$ est définie sur $\left[0;2\right]$ par \[f(x)=x^3 - 2x^2 + 2x-1 .\]

\begin{enumerate}
  \item \medskip

Pour tout $x\in\left[0;2\right]~:$

\[f'(x)=3x^2-2\times 2x+2\times 1-0=3x^2-4x+2.\]

La dérivée est du second degré, donc on utilise le discriminant~:

\begin{itemize}
\item[\textbullet] $a=3,$ $b=-4,$ $c=2.$
\item[\textbullet] le discriminant est $\Delta=b^2-4ac=(-4)^2-4\times 3\times 2=-8.$
\item[\textbullet] $\Delta<0,$ donc il n'y a pas de racine.
\end{itemize}

\medskip



$a=3$  $\oplus$ donc $f'$ est du signe $\oplus$ sur $\left[0;2\right]~:$

\newpage

\setlength{\columnseprule}{1pt}

\begin{multicols}{2}
\begin{center}
\begin{tikzpicture}[scale=1]
\tkzTabInit{$x$/1,$f'(x)$/1,$f(x)$/2}{$0$,$2$}
\tkzTabLine{,+,}
\tkzTabVar{-/$-1$,+/$3$}
\tkzTabVal[draw]{1}{2}{0.6}{$x_0$}{$1$}
\end{tikzpicture}
\end{center}

\columnbreak

\begin{itemize}
\item[\textbullet] $f(0)=0^3 - 2\times 0^2 + 2\times 0-1=-1$
\item[\textbullet] $f(2)=2^3 - 2\times 2^2 + 2\times 2-1=3$
\end{itemize}


\end{multicols}
  \item \begin{itemize}
\item[\textbullet] La fonction $f$ est continue et strictement croissante sur $\left[0;2\right]~;$
\item[\textbullet] $f(0)=-1,$ $f(2)=3~;$
\item[\textbullet] $1\in\left[-1;3\right].$
\end{itemize}

D'après le théorème de la bijection, l'équation $f(x)=1$ a exactement une solution  $x_0$ dans $\left[0;2\right].$


\medskip

\Large \danger \normalsize \`A la 3\up{e} ligne \og $1\in\left[-1;3\right]$ \fg , les extrémités $-1$ et $3$ sont les images de $0$ et $2.$ Ne \textbf{surtout pas écrire} \og $1\in\left[0;2\right]$ \fg.
  \item  \begin{itemize}
\item[\textbullet] On commence par un tableau de valeurs sur $\left[0;2\right]$ avec un pas de $0,1.$ On obtient~:

\medskip
\begin{center}
\begin{tabular}{|c|c|c|c|c|c|c|c|c|}
\hline
$x$&$0$&$0,1$&$\dots$&$1,5$&$1,6$&$\dots$&$1,9$&$2$\\ \hline
$f(x)$&$-1$&$-0,819$&$\dots$&$0,875$&$1,176$&$\dots$&$2,439$&$3$\\ \hline
\end{tabular}
\end{center}

\medskip


On en déduit~:\[1,5<x_0<1,6.\]
\item[\textbullet] Ensuite on fait un tableau de valeurs sur $\left[1,5~;~1,6\right]$ avec un pas de $0,01.$ On obtient (en arrondissant les résultats au millième)~:

\medskip

\begin{center}

\begin{tabular}{|c|c|c|c|c|c|c|c|c|}
\hline
$x$&$1,50$&$1,51$&$\dots$&$1,54$&$1,55$&$\dots$&$1,59$&$1,60$\\ \hline
$f(x)$&$0,875$&$ 0,903$&$\dots$&$0,989$&$1,019$&$\dots$&$1,143$&$1,176$\\ \hline
\end{tabular}
\end{center}

\medskip



On en déduit~:\[1,54<x_0<1,55.\]
\end{itemize}
 \end{enumerate}
 
 

\end{exo}


\begin{exo}

\begin{enumerate}
\item La fonction $f$ est définie sur $\left[0;1\right]$ par \[f(x)=2\text{e}^{-2x}-1.\]

\begin{enumerate}
  \item On étudie les variations~: pour tout $x\in\left[0;1\right],$
  \[f'(x)=2\times \left(-2\text{e}^{-2x}\right)-0=\underbrace{-4}_{\ominus}\underbrace{\text{e}^{-2x}}_{\oplus}.\]
  
 \medskip

\setlength{\columnseprule}{1pt}

\begin{multicols}{2}
\begin{center}
\begin{tikzpicture}[scale=1]
\tkzTabInit{$x$/1,$f'(x)$/1,$f(x)$/2}{$0$,$1$}
\tkzTabLine{,-,}
\tkzTabVar{+/$1$,-/$\underbrace{2\text{e}^{-2}-1}_{\ominus}$}
\tkzTabVal[draw]{1}{2}{0.4}{$\alpha$}{$0$}
\end{tikzpicture}
\end{center}

\columnbreak

\begin{itemize}
\item[\textbullet] $f(0)=2\text{e}^{-2\times 0}-1=2\text{e}^{0}-1=2\times 1-1=1$
\item[\textbullet] $f(1)=2\text{e}^{-2\times 1}-1=2\text{e}^{-2}-1\approx -0,73$
\end{itemize}


\end{multicols}
  
  \begin{itemize}
\item[\textbullet] La fonction $f$ est continue et strictement décroissante sur $\left[0;1\right]~;$
\item[\textbullet] $f(0)=1,$ $f(1)=2\text{e}^{-2}-1~;$
\item[\textbullet] $0\in\left[2\text{e}^{-2}-1;1\right].$
\end{itemize}

D'après le théorème de la bijection, l'équation $f(x)=0$ a exactement une solution  $\alpha$ dans $\left[0;1\right].$

  \item \begin{itemize}
\item[\textbullet] On commence par un tableau de valeurs sur $\left[0;1\right]$ avec un pas de $0,1.$ On obtient (en arrondissant les résultats au millième)~:

\begin{center}
\begin{tabular}{|c|c|c|c|c|c|c|c|c|}
\hline
$x$&$0$&$0,1$&$\dots$&$0,3$&$0,4$&$\dots$&$0,9$&$1$\\ \hline
$f(x)$&$1$&$0,637$&$\dots$&$0,098$&$-0,101$&$\dots$&$-0,669$&$-0,729$\\ \hline
\end{tabular}
\end{center}

\medskip


On en déduit~:\[0,3<\alpha<0,4.\]
\item[\textbullet] Ensuite on fait un tableau de valeurs sur $\left[0,3~;~0,4\right]$ avec un pas de $0,01.$ On obtient (en arrondissant les résultats au millième)~:

\begin{center}

\begin{tabular}{|c|c|c|c|c|c|c|c|c|}
\hline
$x$&$0,30$&$0,31$&$\dots$&$0,34$&$0,35$&$\dots$&$0,39$&$0,40$\\ \hline
$f(x)$&$0,098$&$ 0,076$&$\dots$&$0,013$&$-0,007$&$\dots$&$-0,083$&$-0,101$\\ \hline
\end{tabular}
\end{center}

\medskip



On en déduit~:\[0,34<\alpha<0,35.\]
\end{itemize}
  \item 
  
  \setlength{\columnseprule}{1pt}
  
  \begin{multicols}{2}

  Reprenons le tableau de variations de $f~:$


\begin{center}
\begin{tikzpicture}[scale=1]
  \draw[draw=red] (2.1,-2.8) rectangle ++(1.4,2.6);
  \draw[draw=blue] (4.2,-3.8) rectangle ++(1.7,3.6);
\tkzTabInit{$x$/1,$f'(x)$/1,$f(x)$/2}{$0$,$1$}
\tkzTabLine{,-,}
\tkzTabVar{+/$1$,-/$\underbrace{2\text{e}^{-2}-1}_{\ominus}$}
\tkzTabVal[draw]{1}{2}{0.4}{$\alpha$}{$0$}
\end{tikzpicture}
\end{center}

\columnbreak

On voit que~:

\begin{itemize}
  \item[\textbullet] $f(\alpha)=0~;$
  \item[\textbullet] $f$ est strictement positive dans la zone rouge~;
    \item[\textbullet] $f$ est strictement négative dans la zone bleue.
    \end{itemize}
    
    \medskip

 On a donc le tableau de signe~:
  
  \medskip
  
  \begin{center}
\begin{tikzpicture}[scale=1]
\tkzTabInit{$x$/1,$f(x)$/1}{$0$,$\alpha$,$1$}
\tkzTabLine{,+,z,-,}
\end{tikzpicture}
\end{center}


\end{multicols}
  
  

  
 

  
 \end{enumerate}
 \item \begin{enumerate}
 \item La fonction $g$ est définie sur $\left[0;1\right]$ par \[g(x)=-\text{e}^{-2x}-x+2.\]
 
 \medskip
 
 Pour tout $x\in\left[0;1\right]~:$
 \[g'(x)=-\left(-2\text{e}^{-2x}\right)-1+0=2\text{e}^{-2x}-1.\]
 
 La dérivée de $g'$ est $f,$ dont on a étudié le signe dans la question  1.(c). On peut donc construire le tableau de variations de $g~:$
 
 \begin{center}
\begin{tikzpicture}[scale=1]
\tkzTabInit{$x$/1,$g'(x)=f(x)$/1,$g(x)$/2}{$0$,$\alpha$,$1$}
\tkzTabLine{,+,z,-,}
\tkzTabVar{-/,+/,-/}
\end{tikzpicture}
\end{center}
 
 
 \item $f(x)=2\text{e}^{-2x}-1$ et $f(\alpha)=0,$ donc
 \begin{align*}
 2\text{e}^{-2\alpha}-1&=0\\
 2\text{e}^{-2\alpha}&=1\\
 \text{e}^{-2\alpha}&=\frac{1}{2}
 \end{align*}
 
 \medskip
 
 On en déduit~: \begin{align*}
 g(\alpha)&=-\text{e}^{-2\alpha}-\alpha+2
 \\&=-\frac{1}{2}-\alpha+2\\&=\frac{3}{2}-\alpha
 \\&=1,5-\alpha. 
 \end{align*}
 On sait que $0,34<\alpha<0,35,$ donc
 
  \begin{align*}
  0,34\textcolor{red}{\times (-1)}&>\alpha\textcolor{red}{\times (-1)}>0,35\textcolor{red}{\times (-1)}&&(\text{\danger $-1~\ominus,$ donc les $<$ deviennent des $>$})\\
  -0,34&>-\alpha>-0,35\\
 1,5-0,34&>1,5-\alpha>1,5-0,35&&(\text{on ajoute 1,5})\\
 1,16&>g(\alpha)>1,15.
 \end{align*}
 
 \end{enumerate}
 \end{enumerate}
 \end{exo}
 

 
 
 

 \begin{exo}
 
 La suite $(u_n)_{n\in\mathbb{N}}$ est définie par $u_0=1$ et la relation de récurrence \[u_{n+1}=0,5u_n+3\] pour tout $n\in\mathbb{N}.$


\begin{enumerate}
\item \begin{align*}
u_0&=1\\
u_1&=0,5u_0+3=0,5\times 1+3=3,5\\
u_2&=0,5u_1+3=0,5\times 3,5+3=4,75\\
u_3&=0,5u_2+3=0,5\times 4,75+3=5,375
\end{align*}
\item On trace la droite d'équation $y=0,5x+3$ (cf question suivante) à partir d'un tableau de valeurs avec deux valeurs~:


\begin{center}
\begin{tabular}{|c|c|c|}\hline
$x$ & $0$ & $10$\\ \hline
$y$ & $3$ & $8$\\ \hline
\end{tabular}
\end{center}

\item ~{}

\begin{center}
\newrgbcolor{ududff}{0.30196078431372547 0.30196078431372547 1.}
\newrgbcolor{xfqqff}{0.4980392156862745 0. 1.}
\psset{xunit=0.75cm,yunit=0.75cm,algebraic=true,dimen=middle,dotstyle=o,dotsize=5pt 0,linewidth=1.6pt,arrowsize=3pt 2,arrowinset=0.25}
\begin{pspicture*}(-0.82,-1.06)(10.58,10.58)
\multips(0,0)(0,1.0){12}{\psline[linestyle=dashed,linecap=1,dash=1.5pt 1.5pt,linewidth=0.4pt,linecolor=lightgray]{c-c}(0,0)(10.58,0)}
\multips(0,0)(1.0,0){11}{\psline[linestyle=dashed,linecap=1,dash=1.5pt 1.5pt,linewidth=0.4pt,linecolor=lightgray]{c-c}(0,0)(0,10.58)}
\psaxes[labelFontSize=\scriptstyle,xAxis=true,yAxis=true,Dx=5.,Dy=5.,ticksize=-2pt 0,subticks=2]{->}(0,0)(0.,0.)(10.58,10.56)
\psplot[linewidth=2.pt]{0}{10}{(-0.--1.*x)/1.}
\psplot[linewidth=2.pt,linecolor=blue]{0}{10}{(3.+0.5*x)/1.}
\rput[tl](6.5,8.16){$y=x$}
\rput[tl](7.86,6.5){\blue{$y=0,5x+3$}}
\psline[linewidth=2.pt,linestyle=dotted,linecolor=red](1.,0.)(1.,3.5)
\psline[linewidth=2.pt,linestyle=dotted,linecolor=red](0.,3.5)(3.5,3.5)
\psline[linewidth=2.pt,linestyle=dotted,linecolor=red](3.5,0.)(3.5,4.75)
\psline[linewidth=2.pt,linestyle=dotted,linecolor=red](0,4.75)(4.75,4.75)
\psline[linewidth=2.pt,linestyle=dotted,linecolor=red](4.75,5.375)(4.75,0.)
\psline[linewidth=2.pt,linestyle=dotted,linecolor=red](5.375,5.375)(0,5.375)
\psline[linewidth=2.pt,linestyle=dotted,linecolor=red](5.375,5.375)(5.375,0)
\psline[linewidth=2.pt,linestyle=dotted,linecolor=xfqqff](6,6)(6,0)
\psdots[dotstyle=*,dotsize=6pt,linecolor=xfqqff](6,6)
\rput[tl](0.8,-0.4){\red{$u_0$}}
\rput[tl](3.25,-0.4){\red{$u_1$}}
\rput[tl](4.4,-0.4){\red{$u_2$}}
\rput[tl](5.25,-0.4){\red{$u_3$}}
\rput[tl](-0.7,3.6){\red{$u_1$}}
\rput[tl](-0.7,4.85){\red{$u_2$}}
\rput[tl](-0.7,5.475){\red{$u_3$}}
\rput[tl](5.8,-0.4){\xfqqff{$\ell=6$}}
\end{pspicture*}
\end{center}


\item Pour tout $n\in\mathbb{N},$ on note $\mathcal{P}_n$ la propriété \[ u_{n}\leq u_{n+1}\leq 6.\]

\begin{itemize}
\item[{\textbullet}] \textbf{Initialisation.} On prouve que $\mathcal{P}_0$ est vraie.


\[
\left.
    \begin{array}{ll}
        u_0&=1 \\
        u_1&=3,5\\
        1&\leq 3,5\leq 6
    \end{array}
\right \}\implies  u_{0}\leq u_{1}\leq 6\implies\mathcal{P}_0~\text{est vraie}.
\]



\item[{\textbullet}] \textbf{Hérédité.} Soit $k\in\mathbb{N}$ tel que $\mathcal{P}_k$ soit vraie. On a donc
\[ u_{k}\leq u_{k+1}\leq 6.\]
%\medskip

\newtcolorbox{mybox}[1]{colback=green!10!white,colframe=green!80!white,fonttitle=\bfseries,title=#1}
\begin{mybox}{Objectif}{Prouver que $\mathcal{P}_{k+1}$ est vraie, c'est-à-dire que \[u_{k+1}\leq u_{k+2}\leq 6.\]
}\end{mybox}



%\medskip

On part de 
\[u_{k}\leq u_{k+1}\leq 6.\]


On multiplie par $\textcolor{red}{0,5}~:$

\begin{align*} u_{k}\textcolor{red}{\times 0,5}&\leq u_{k+1}\textcolor{red}{\times 0,5}\leq 6\textcolor{red}{\times 0,5}\\
 0,5 u_{k}&\leq 0,5u_{k+1}\leq 3.
\end{align*}

Puis on ajoute  $\textcolor{blue}{3}~:$

\begin{align*}
0,5 u_{k}\textcolor{blue}{+3}&\leq 0,5u_{k+1}\textcolor{blue}{+3}\leq 3\textcolor{blue}{+3}\\
u_{k+1}&\leq u_{k+2}\leq 6.
\end{align*}

La propriété $\mathcal{P}_{k+1}$ est donc vraie.
\item[{\textbullet}] \textbf{Conclusion.} $\mathcal{P}_0$ est vraie et $\mathcal{P}_n$ est héréditaire, donc elle est vraie pour tout $n\in\mathbb{N}.$
\end{itemize}
\item D'après la question précédente~:

\begin{itemize}
\item[{\textbullet}] $u_{n}\leq u_{n+1}$ pour tout $n\in\mathbb{N},$ donc $(u_n)_{n\in\mathbb{N}}$ est croissante.
\item[{\textbullet}] $u_n\leq 6$ pour tout $n\in\mathbb{N},$ donc $(u_n)_{n\in\mathbb{N}}$ est majorée par 6.
\end{itemize}

Or d'après le théorème de limite monotone, toute suite croissante majorée converge, donc $(u_n)_{n\in\mathbb{N}}$ converge.
\item On note $\ell$ la limite de $\left(u_n\right)_{n\in\mathbb{N}}$ et \og on passe à la limite \fg~{} dans la formule de récurrence (autre justification~: \og la fonction $f:x\mapsto 0,5x+3$ est continue \fg)~:

\[u_{n+1}=0,5u_n+3\qquad\text{pour tout }n\in\mathbb{N},\] donc
\[\ell=0,5\ell+3.\]

On résout cette équation~:
\[\ell=0,5\ell+3\iff \ell-0,5\ell=3\iff 0,5\ell=3\iff\ell=\frac{3}{0,5}\iff \ell=6.\]

Conclusion~: $\lim\limits_{n\to +\infty}u_n=6.$

\end{enumerate}
 
 \end{exo}
 
 \begin{exo}

La fonction $f$ est définie sur $\mathbb{R}$ par $f(x)=-x^2+2x.$


La suite $(u_n)_{n\in\mathbb{N}}$ est définie par $u_0=0,5$ et la relation de récurrence

\[u_{n+1}=f\left(u_n\right)\] pour tout $n\in\mathbb{N}.$

\begin{enumerate}
\item Pour tout $x\in\left[0;1\right]~:$

\[f'(x)=-2x+2\times 1=-2x+2.\]

On résout~:
\[-2x+2=0\iff -2x=-2\iff x=\frac{-2}{-2}=1.\]


On en déduit le tableau de signe de $f'$ et le tableau de variations de $f~:$


\medskip

\setlength{\columnseprule}{1pt}

\begin{multicols}{2}

\begin{center}
\begin{tikzpicture}[scale=0.8]
\tkzTabInit{$x$/1,$f'(x)$/1,$f(x)$/2}{$0$,$1$}
\tkzTabLine{,+,z}
\tkzTabVar{-/$0$,+/$1$}
\end{tikzpicture}
\end{center}

\columnbreak


\begin{itemize}
\item[\textbullet] $f(0)=-0^2+2\times 0=0$
\item[\textbullet] $f(1)=-1^2+2\times 1=1$
\end{itemize}



\end{multicols}


\item 

\medskip

\setlength{\columnseprule}{1pt}

\begin{multicols}{2}
La fonction $f$ est du second degré, donc sa courbe représentative est une parabole. On la trace (en bleu) à partir d'un tableau de valeurs~:

\medskip

\begin{center}
\begin{tabular}{|c|c|c|c|c|c|c|}
\hline
   $x$ &$0$ &$0,2$ &$0,4$ &$0,6$ &$0,8$&$1$ \\
	\hline
	$f(x)$ &$0$ &$0,36$ &$0,64$ &$0,84$ &$0,96$&$1$ \\
	\hline
\end{tabular}
\end{center}

\medskip


\begin{center}
\newrgbcolor{xfqqff}{0.4980392156862745 0. 1.}
\psset{xunit=5cm,yunit=5cm,algebraic=true,dimen=middle,dotstyle=o,dotsize=5pt 0,linewidth=2.pt,arrowsize=3pt 2,arrowinset=0.25}
\begin{pspicture*}(-0.22543720763631073,-0.1530496292958985)(1.9131271900939086,1.2380553672859034)
\multips(0,0)(0,0.2){6}{\psline[linestyle=dashed,linecap=1,dash=1.5pt 1.5pt,linewidth=0.4pt,linecolor=lightgray]{c-c}(0,0)(1.05,0)}
\multips(0,0)(0.2,0){6}{\psline[linestyle=dashed,linecap=1,dash=1.5pt 1.5pt,linewidth=0.4pt,linecolor=lightgray]{c-c}(0,0)(0,1.05)}
\psaxes[labelFontSize=\scriptstyle,xAxis=true,yAxis=true,Dx=0.2,Dy=0.2,ticksize=-2pt 0,subticks=2]{->}(0,0)(0.,0.)(1.23,1.23)
\rput{-180.}(1.,1.){\psplot[linewidth=2.pt,linecolor=blue]{0.}{1.}{x^2/2/0.5}}
\psplot[linewidth=2.pt]{0}{1}{(-0.--1.*x)/1.}
%\rput[tl](1.15,0.63){$y=x$}
%\rput[tl](1.15,0.45){\blue{$y=-x^2+2x$}}
%\psline[linewidth=2.pt](1.2,0.6)(1.4,0.6)
\psline[linewidth=2.pt,linestyle=dotted,linecolor=red](0.5,0.)(0.5,0.75)
\psline[linewidth=2.pt,linestyle=dotted,linecolor=red](0.,0.75)(0.75,0.75)
\psline[linewidth=2.pt,linestyle=dotted,linecolor=red](0.75,0.9375)(0.75,0.)
\psline[linewidth=2.pt,linestyle=dotted,linecolor=red](0.,0.9375)(0.9375,0.9375)
\psline[linewidth=2.pt,linestyle=dotted,linecolor=red](0.9375,0.99609375)(0.9375,0.)
\psline[linewidth=2.pt,linestyle=dotted,linecolor=red](0.,0.99609375)(0.99609375,0.99609375)
\psline[linewidth=2.pt,linestyle=dotted,linecolor=red](0.99609375,0.99609375)(1.,0.)
\rput[tl](0.45,-0.05){\red{$u_0$}}
\rput[tl](0.6846638224657082,-0.05){\red{$u_1$}}
\rput[tl](0.9,-0.05){\red{$u_2$}}
\rput[tl](1.0341703017064077,-0.05){\red{$u_3$}}
\rput[tl](-0.13,0.7605118609966282){\red{$u_1$}}
\rput[tl](-0.13,0.97){\red{$u_2$}}
\rput[tl](-0.13,1.07){\red{$u_3$}}
\rput[tl](1.0341703017064077,0.08){\xfqqff{$\ell=1$}}
%\psline[linewidth=2.pt,linecolor=blue](1.2,0.4)(1.4,0.4)
\begin{scriptsize}
\psdots[dotsize=4pt 0,dotstyle=*,linecolor=blue](0.2,0.36)
\psdots[dotsize=4pt 0,dotstyle=*,linecolor=blue](0.4,0.64)
\psdots[dotsize=4pt 0,dotstyle=*,linecolor=blue](0.6,0.84)
\psdots[dotsize=4pt 0,dotstyle=*,linecolor=blue](0.8,0.96)
\psdots[dotsize=4pt 0,dotstyle=*,linecolor=blue](1,1)
\end{scriptsize}
\psline[linewidth=2.pt,linestyle=dotted,linecolor=xfqqff](1,0.02)(1.,0.98)
\end{pspicture*}
\end{center}

($u_3\approx 0,996$ et $\ell=1$ se confondent presque.)
\columnbreak

Parallèlement au graphique, calculons les premiers termes de la suite~:

\medskip

\begin{align*}
u_0&=0,5\\
u_1&=f\left(u_0\right)=f(0,5)=-0,5^2+2\times 0,5=0,75\\
u_2&=f\left(u_1\right)=f(0,75)=-0,75^2+2\times 0,75=\np{0,9375}\\
u_3&=f\left(u_2\right)=f(\np{0,9375})=-\np{0,9375}^2+2\times \np{0,9375}\approx 0,996
\end{align*}

\medskip

Insistons sur le fait -- très important dans la suite de l'exercice -- que 
$u_1=f\left(u_0\right),$ $u_2=f\left(u_1\right),$ $u_3=f\left(u_2\right)~;$ et plus généralement, pour tout $n\in\mathbb{N}~:$
\[u_{n+1}=f\left(u_n\right).\]

On a donc aussi
\[u_{n+2}=f\left(u_{n+1}\right).\]



\end{multicols}

\item Pour tout $n\in\mathbb{N},$ on note $\mathcal{P}_n$ la propriété \[0\leq u_n\leq u_{n+1}\leq 1.\]

\begin{itemize}
\item[{\textbullet}] \textbf{Initialisation.} On prouve que $\mathcal{P}_0$ est vraie.


\[
\left.
    \begin{array}{ll}
        u_0&=0,5 \\
        u_1&=0,75\\
        0&\leq 0,5\leq 0,75\leq 1
    \end{array}
\right \}\implies 0\leq u_{0}\leq u_{1}\leq 1\implies\mathcal{P}_0~\text{est vraie}.
\]



\item[{\textbullet}] \textbf{Hérédité.} Soit $k\in\mathbb{N}$ tel que $\mathcal{P}_k$ soit vraie. On a donc
\[0\leq u_k\leq u_{k+1}\leq 1.\]%\medskip

\newtcolorbox{mybox}[1]{colback=green!10!white,colframe=green!80!white,fonttitle=\bfseries,title=#1}
\begin{mybox}{Objectif}{Prouver que $\mathcal{P}_{k+1}$ est vraie, c'est-à-dire que \[0\leq u_{k+1}\leq u_{k+2}\leq 1.\]
}\end{mybox}



%\medskip

\setlength{\columnseprule}{1pt}

\begin{multicols}{2}

On part de 
\[0\leq u_k\leq u_{k+1}\leq 1.\]

La fonction $f$ est strictement croissante sur $\left[0;1\right],$ donc
\[f(0)\leq f\left(u_k\right)\leq f\left(u_{k+1}\right)\leq f(1).\]

\medskip

Autrement dit~:

\[0\leq u_{k+1}\leq u_{k+2}\leq 1\]

(cf illustration avec le tableau de variations ci-contre), c'est-à-dire que la propriété $\mathcal{P}_{k+1}$ est  vraie.

\begin{center}
\begin{tikzpicture}[scale=1.]
\tkzTabInit{$x$/1,$f(x)$/2}{$0$,$1$}
\tkzTabVar{-/$0$,+/$1$}
\tkzTabVal[draw]{1}{2}{0.33}{$u_k$}{$u_{k+1}$}
\tkzTabVal[draw]{1}{2}{0.66}{$u_{k+1}$}{$u_{k+2}$}
\end{tikzpicture}
\end{center}
\medskip

\textbf{N.B.} $f\left(u_k\right)=u_{k+1},\qquad f\left(u_{k+1}\right)=u_{k+2}$

\end{multicols}



\medskip

\textbf{Remarque~:} D'habitude, l'hérédité demande de manipuler les inégalités en faisant certaines opérations (multiplier, ajouter, etc.). Ici, toutes ces opérations sont \og incluses dans les variations de $f$ \fg. Rappelons-nous l'exercice précédent ($u_0=1$ et $u_{n+1}=0,5u_n+3$) et sa preuve par récurrence~: là aussi, on aurait pu raccourcir l'hérédité en invoquant la croissance de $x\mapsto 0,5x+3,$ plutôt que de multiplier par $0,5,$ puis d'ajouter $3.$ C'est en fait le cas dans bon nombre de démonstrations par récurrence que nous avons faites jusqu'ici.


\item[{\textbullet}] \textbf{Conclusion.} $\mathcal{P}_0$ est vraie et $\mathcal{P}_n$ est héréditaire, donc elle est vraie pour tout $n\in\mathbb{N}.$
\end{itemize}
\item D'après la question précédente~:

\begin{itemize}
\item[{\textbullet}] $u_{n}\leq u_{n+1}$ pour tout $n\in\mathbb{N},$ donc $(u_n)_{n\in\mathbb{N}}$ est croissante.
\item[{\textbullet}] $u_n\leq 1$ pour tout $n\in\mathbb{N},$ donc $(u_n)_{n\in\mathbb{N}}$ est majorée par 1.
\end{itemize}

Or d'après le théorème de limite monotone, toute suite croissante majorée converge, donc $(u_n)_{n\in\mathbb{N}}$ converge. De plus, $(u_n)_{n\in\mathbb{N}}$ étant croissante, sa limite $\ell$ est supérieure ou égale à $u_0~:$ \[\ell\geq 0,5.\]
\item On note $\ell$ la limite de $\left(u_n\right)_{n\in\mathbb{N}}$ et \og on passe à la limite \fg~{} dans la formule de récurrence (autre justification~: \og la fonction $f:x\mapsto -x^2+2x$ est continue \fg)~:

\[u_{n+1}=-u_n^2+2u_n\qquad\text{pour tout }n\in\mathbb{N},\] donc
\[\ell=-\ell^2+2\ell.\]

On résout cette équation~:
\[\ell=-\ell^2+2\ell \iff \ell^2+\ell-2\ell=0\iff \ell^2-\ell=0\iff\ell(\ell-1)=0\iff \left(\ell=0~\text{ou}~\ell=1\right).\]

Or on a vu à la fin de la question précédente que $\ell\geq 0,5, $ donc $\ell=1.$

Conclusion~: $\lim\limits_{n\to +\infty}u_n=1.$

\end{enumerate}

\end{exo}
 
 \begin{exo}
 
 On définit deux fonctions $f$ et $g$ sur $\mathbb{R}$ par 
 
 \[ f(x)=\text{e}^{-x},\qquad g(x)=x-\text{e}^{-x}.\]
 
 La suite $(u_n)_{n\in\mathbb{N}}$ est définie par $u_0=1$ et la relation de récurrence \[u_{n+1}=f\left(u_n\right)\] pour tout $n\in\mathbb{N}.$
 
 \begin{enumerate}
  \item On trace la courbe de la fonction $f$ (en bleu) à partir d'un tableau de valeurs~:

\medskip
\begin{center}
 \begin{tabular}{|c|c|c|c|c|c|}\hline
$x$&$0$& $0,5$ &$1$&$1,5$&$2$ \\ \hline 
$f(x)$&$1$&$0,61$ &$0,37$&$0,22$&$0,14$  \\ \hline
\end{tabular}
\medskip

(valeurs arrondies au centième)

\end{center}





\begin{center}
\newrgbcolor{xfqqff}{0.4980392156862745 0. 1.}
\psset{xunit=5.0cm,yunit=5.0cm,algebraic=true,dimen=middle,dotstyle=o,dotsize=5pt 0,linewidth=2.pt,arrowsize=3pt 2,arrowinset=0.25}
\begin{pspicture*}(-0.2420552743558184,-0.1693265672575209)(1.2547571686399135,1.2126996454471273)
\multips(0,0)(0,0.2){7}{\psline[linestyle=dashed,linecap=1,dash=1.5pt 1.5pt,linewidth=0.4pt,linecolor=lightgray]{c-c}(0,0)(1.2547571686399135,0)}
\multips(0,0)(0.2,0){8}{\psline[linestyle=dashed,linecap=1,dash=1.5pt 1.5pt,linewidth=0.4pt,linecolor=lightgray]{c-c}(0,0)(0,1.2126996454471273)}
\psaxes[labelFontSize=\scriptstyle,xAxis=true,yAxis=true,Dx=1.,Dy=1.,ticksize=-2pt 0,subticks=2]{->}(0,0)(0.,0.)(1.2547571686399135,1.2126996454471273)
\psplot[linewidth=2.pt,linecolor=blue,plotpoints=200]{0}{1}{EXP(-x)}
\psplot[linewidth=2.pt]{0}{1}{(-0.--1.*x)/1.}
\psline[linewidth=2.pt,linestyle=dotted,linecolor=red](1.,0.)(1.,0.36787944117144233)
\psline[linewidth=2.pt,linestyle=dotted,linecolor=red](1.,0.36787944117144233)(0.,0.36787944117144233)
\psline[linewidth=2.pt,linestyle=dotted,linecolor=red](0.36788,0.6922002407339761)(0.36787944117144233,0.)
\psline[linewidth=2.pt,linestyle=dotted,linecolor=red](0.,0.6922002407339761)(0.6922002407339761,0.6922002407339761)
\psline[linewidth=2.pt,linestyle=dotted,linecolor=red](0.,0.5004736941575195)(0.6922002407339761,0.5004736941575195)
\psline[linewidth=2.pt,linestyle=dotted,linecolor=red](0.5004736941575195,0.5004736941575195)(0.5004736941575195,0.)
\psline[linewidth=2.pt,linestyle=dotted,linecolor=red](0.6922002407339761,0.6922002407339761)(0.6922002407339761,0.)
\psline[linewidth=2.pt,linestyle=dotted,linecolor=xfqqff](0.5671432904153082,0.5671432904153082)(0.5671432904153082,0.)
\rput[tl](1.0160018096344594,-0.03158309090822043){\red{$u_0$}}
\rput[tl](0.33,-0.03158309090822043){\red{$u_1$}}
\rput[tl](0.6532773219146347,-0.03158309090822043){\red{$u_2$}}
\rput[tl](0.46,-0.03617454011986378){\red{$u_3$}}
\rput[tl](-0.1,0.38){\red{$u_1$}}
\rput[tl](-0.1,0.7){\red{$u_2$}}
\rput[tl](-0.1,0.52){\red{$u_3$}}
\rput[tl](0.52,-0.03158309090822043){\xfqqff{$\underset{\approx 0,57}{\ell}$}}
\end{pspicture*}
\end{center}
\item On admet que $\lim\limits_{n\to +\infty}u_n=\ell,$ où $\ell\in \left[0;1\right].$
\begin{enumerate}
\item On étudie les variations de $g$ sur $\left[0;1\right]~:$ pour tout $x\in\left[0;1\right],$
\[g'(x)=1-\left(-\text{e}^{-x}\right)=\underbrace{1+\text{e}^{-x}}_{\oplus}.\]

\medskip

\setlength{\columnseprule}{1pt}

\begin{multicols}{2}
\begin{center}
\begin{tikzpicture}[scale=1]
\tkzTabInit{$x$/1,$g'(x)$/1,$g(x)$/2}{$0$,$1$}
\tkzTabLine{,+,}
\tkzTabVar{-/$-1$,+/$\underbrace{1-\text{e}^{-1}}_{\oplus}$}
\tkzTabVal[draw]{1}{2}{0.4}{$\alpha$}{$0$}
\end{tikzpicture}
\end{center}

\columnbreak

\begin{itemize}
\item[\textbullet] $g(0)=0-\text{e}^{-0}=0-1=-1$
\item[\textbullet] $g(1)=1-\text{e}^{-1}\approx 0,63$
\end{itemize}


\end{multicols}
  \item \begin{itemize}
\item[\textbullet] La fonction $g$ est continue et strictement croissante sur $\left[0;1\right]~;$
\item[\textbullet] $g(0)=-1,$ $g(1)=1-\text{e}^{-1}~;$
\item[\textbullet] $0\in\left[-1;1-\text{e}^{-1}\right].$
\end{itemize}

D'après le théorème de la bijection, l'équation $g(x)=0$ a exactement une solution  $\alpha$ dans $\left[0;1\right].$

\medskip

Il reste à prouver que $\alpha=\ell.$ Pour cela, on se rappelle que $(u_n)_{n\in\mathbb{N}}$ est définie par $u_0=1$ et la relation de récurrence $u_{n+1}=f\left(u_n\right).$ La fonction $f$ étant continue sur $\left[0;1\right],$ on peut \og passer à la limite \fg~{} dans la formule de récurrence~:

\[u_{n+1}=f\left(u_n\right)\qquad\text{pour tout }n\in\mathbb{N},\] donc
\begin{equation}\label{Z}\ell=f(\ell).
\end{equation}

Or $f(x)=\text{e}^{-x},$ donc (\ref{Z}) ci-dessus se réécrit
\[\ell=\text{e}^{-\ell},\] ou encore \begin{equation}\label{Y}\ell-\text{e}^{-\ell}=0.\end{equation} En se rappelant que $g(x)=x-\text{e}^{-x},$ on a donc $g(\ell)=0.$ Or l'unique solution dans $\left[0;1\right]$ de l'équation $g(x)=0$ est $\alpha,$ donc $\ell=\alpha.$

%\medskip

%En particulier, $\lim\limits_{n\to +\infty}u_n=\alpha.$

\item Pour déterminer une valeur approchée de $\ell$ au centième, il y a deux méthodes possibles~:
\begin{itemize}
\item[\textbullet] Calculer les termes successifs de la suite $(u_n)_{n\in\mathbb{N}}.$ Comme la spirale \og s'enroule autour de $\ell$ \fg, il suffit que l'écart entre deux termes successifs de la suite soit inférieur à $0,01$ pour avoir une valeur approchée de $\ell$ au  centième. Avec la calculatrice, on trouve $u_8\approx 0,576$ et $u_9\approx 0,562,$ donc $\ell\approx 0,57.$\footnote{Il est certain que l'écart entre la valeur que nous venons de donner et la valeur réelle de $\ell$ est inférieur à $0,01,$ puisque $0,576-0,57=0,006$ et $0,57-0,562=0,008.$}
\item[\textbullet] Utiliser la méthode du début de la leçon pour trouver une valeur approchée de $\alpha,$ solution de l'équation $g(x)=0$\footnote{On sait que $\alpha=\ell,$ donc trouver une valeur approchée de $\ell$ revient à trouver une valeur approchée de $\alpha.$}~: on commence par un tableau de valeurs pour $g$ sur $\left[0;1\right]$ avec un pas de $0,1.$ On obtient (en arrondissant les résultats au millième)~:

\medskip
\begin{center}
\begin{tabular}{|c|c|c|c|c|c|c|c|c|}
\hline
$x$&$0$&$0,1$&$\dots$&$0,5$&$0,6$&$\dots$&$0,9$&$1$\\ \hline
$g(x)$&$-1$&$-0,804$&$\dots$&$-0,106$&$0,051$&$\dots$&$0,493$&$0,632$\\ \hline
\end{tabular}
\end{center}

\medskip


On en déduit~:\[0,5<\alpha<0,6.\]
\item[\textbullet] Ensuite on fait un tableau de valeurs sur $\left[0,5~;~0,6\right]$ avec un pas de $0,01.$ On obtient (en arrondissant les résultats au millième)~:

\medskip

\begin{center}

\begin{tabular}{|c|c|c|c|c|c|c|c|c|}
\hline
$x$&$0,50$&$0,51$&$\dots$&$0,56$&$0,57$&$\dots$&$0,59$&$0,60$\\ \hline
$f(x)$&$-0,106$&$ -0,090$&$\dots$&$-0,011$&$0,004$&$\dots$&$0,036$&$0,051$\\ \hline
\end{tabular}
\end{center}

\medskip



On en déduit~:\[0,56<\alpha<0,57.\]

\`A nouveau, on peut écrire $\ell\approx 0,57.$\footnote{$\ell\approx 0,56$ conviendrait également.}
\end{itemize}
\end{enumerate} 
\end{enumerate}
 \end{exo}
 
 \begin{exo}

La partie entière d'un nombre réel $x,$ notée $E(x),$ est le plus grand nombre entier inférieur ou égal à $x.$

\medskip

Commençons par quelques exemples~: on calcule la partie entière de $2,5,$ de $\frac{3}{4},$ de $4,$ puis de $-2,5.$

\medskip

\begin{itemize}
\item[\textbullet] $2\leq 2,5<3,$ donc le plus grand entier inférieur ou égal à $2,5$ est $2~:$
\[E(2,5)=2.\]
\item[\textbullet] $0\leq \frac{3}{4}<1,$ donc le plus grand entier inférieur ou égal à $\frac{3}{4}$ est $0~:$
\[E\left(\frac{3}{4}\right)=0.\]
\item[\textbullet] $4\leq 4<5,$ donc le plus grand entier inférieur ou égal à $4$ est $4$ lui-même~:
\[E(4)=4.\]
\item[\textbullet] Le dernier exemple est piégeux~: $-3\leq -2,5<-2,$ donc le plus grand entier inférieur ou égal à $-2,5$ est $-3~:$
\[E(-2,5)=-3.\]
\end{itemize}

\medskip
\

La fonction partie entière est une fonction \og en escalier \fg~{}: elle est constante sur chaque intervalle $\left[n;n+1\right[,$ où $n\in\mathbb{Z},~$ et elle fait des \og sauts \fg~{} de hauteur 1 à chaque point entier (où elle est donc discontinue).

\begin{center}
\psset{xunit=1.0cm,yunit=1.0cm,algebraic=true,dimen=middle,dotstyle=o,dotsize=5pt 0,linewidth=2.pt,arrowsize=3pt 2,arrowinset=0.25}
\begin{pspicture*}(-3.52,-3.76)(4.96,3.52)
\multips(0,-3)(0,1.0){8}{\psline[linestyle=dashed,linecap=1,dash=1.5pt 1.5pt,linewidth=0.4pt,linecolor=lightgray]{c-c}(-3.52,0)(4.96,0)}
\multips(-3,0)(1.0,0){9}{\psline[linestyle=dashed,linecap=1,dash=1.5pt 1.5pt,linewidth=0.4pt,linecolor=lightgray]{c-c}(0,-3.76)(0,3.52)}
\psaxes[labelFontSize=\scriptstyle,xAxis=true,yAxis=true,Dx=1.,Dy=1.,ticksize=-2pt 0,subticks=2]{->}(0,0)(-3.52,-3.76)(4.96,3.52)
\psline[linewidth=2.pt,linecolor=blue](-3.,-3.)(-2.,-3.)
\psline[linewidth=2.pt,linecolor=blue](-2.,-2.)(-1.,-2.)
\psline[linewidth=2.pt,linecolor=blue](-1.,-1.)(0.,-1.)
\psline[linewidth=2.pt,linecolor=blue](0.,0.)(1.,0.)
\psline[linewidth=2.pt,linecolor=blue](1.,1.)(2.,1.)
\psline[linewidth=2.pt,linecolor=blue](2.,2.)(3.,2.)
\psline[linewidth=2.pt,linecolor=blue](3.,3.)(4.,3.)
\rput[tl](-2.1,-2.72){\huge {\blue{[}}}
\rput[tl](-1.1,-1.72){\huge {\blue{[}}}
\rput[tl](-0.1,-0.72){\huge {\blue{[}}}
\rput[tl](0.9,0.28){\huge {\blue{[}}}
\rput[tl](1.9,1.28){\huge {\blue{[}}}
\rput[tl](2.9,2.28){\huge {\blue{[}}}
\rput[tl](3.9,3.28){\huge {\blue{[}}}
\psdots[dotstyle=*,linecolor=blue](-3.,-3.)
\psdots[dotstyle=*,linecolor=blue](-2.,-2.)
\psdots[dotstyle=*,linecolor=blue](-1.,-1.)
\psdots[dotstyle=*,linecolor=blue](0.,0.)
\psdots[dotstyle=*,linecolor=blue](1.,1.)
\psdots[dotstyle=*,linecolor=blue](2.,2.)
\psdots[dotstyle=*,linecolor=blue](3.,3.)
\end{pspicture*}
\end{center}


\end{exo}



\begin{exo}

Sur chacune des figures ci-dessous, on a tracé la courbe d'une fonction $f$ en traits pleins et la droite d'équation $y=x$ en pointillés. Dans chaque cas, une suite $(u_n)_{n\in\mathbb{N}}$ est définie par la relation de récurrence 

\[u_{n+1}=f\left(u_n\right)\] pour tout $n\in\mathbb{N}.$

\medskip

On allège un peu les constructions~: on dessine des escaliers, des spirales, [...] en traits pleins, sans aller jusqu'aux axes et sans y faire apparaître $u_0,$ $u_1,$ etc.

\medskip

\begin{multicols}{2}


\begin{center}
\newrgbcolor{uququq}{0.25098039215686274 0.25098039215686274 0.25098039215686274}
\psset{xunit=4.0cm,yunit=4.0cm,algebraic=true,dimen=middle,dotstyle=o,dotsize=5pt 0,linewidth=2.pt,arrowsize=3pt 2,arrowinset=0.25}
\begin{pspicture*}(-0.178981823685607,-0.18017215627121094)(1.2,1.2)
\multips(0,0)(0,0.5){3}{\psline[linestyle=dashed,linecap=1,dash=1.5pt 1.5pt,linewidth=0.4pt,linecolor=gray]{c-c}(0,0)(1.2,0)}
\multips(0,0)(0.5,0){4}{\psline[linestyle=dashed,linecap=1,dash=1.5pt 1.5pt,linewidth=0.4pt,linecolor=gray]{c-c}(0,0)(0,1.2)}
\psaxes[labelFontSize=\scriptstyle,xAxis=true,yAxis=true,Dx=0.5,Dy=0.5,ticksize=-2pt 0,subticks=2]{->}(0,0)(0.,0.)(1.2,1.2)
\psplot[linewidth=2.pt,linestyle=dashed,dash=1pt 1pt]{0}{1.2}{(-0.--1.*x)/1.}
\psplot[linewidth=2.pt,linecolor=uququq]{0}{1.2}{(-0.05--1.2*x)/1.}
\rput[tl](0.65,-0.06167120265581232){$u_0$}
\begin{scriptsize}
\psdots[dotstyle=+](0.7,0.)
\end{scriptsize}
\psline[linewidth=2.pt,linecolor=red](0.7,0.)(0.7,0.79)
\psline[linewidth=2.pt,linecolor=red](0.7,0.79)(0.79,0.79)
\psline[linewidth=2.pt,linecolor=red](0.79,0.79)(0.79,0.898)
\psline[linewidth=2.pt,linecolor=red](0.79,0.898)(0.898,0.898)
\psline[linewidth=2.pt,linecolor=red](0.898,0.898)(0.898,1.0276)
\psline[linewidth=2.pt,linecolor=red](1.0276,1.0276)(0.898,1.0276)
\psline[linewidth=2.pt,linecolor=red](1.0276,1.18312)(1.0276,1.0276)
\end{pspicture*}

\medskip

L'escalier va \og monter vers l'infini \fg ~: $\textcolor{red}{\lim\limits_{n\to +\infty}u_n=+\infty}.$
\end{center}

\medskip


\begin{center}
\newrgbcolor{uququq}{0.25098039215686274 0.25098039215686274 0.25098039215686274}
\psset{xunit=4.0cm,yunit=4.0cm,algebraic=true,dimen=middle,dotstyle=o,dotsize=5pt 0,linewidth=2.pt,arrowsize=3pt 2,arrowinset=0.25}
\begin{pspicture*}(-0.178981823685607,-0.18017215627121094)(1.2,1.2)
\multips(0,0)(0,0.5){3}{\psline[linestyle=dashed,linecap=1,dash=1.5pt 1.5pt,linewidth=0.4pt,linecolor=gray]{c-c}(0,0)(1.2,0)}
\multips(0,0)(0.5,0){4}{\psline[linestyle=dashed,linecap=1,dash=1.5pt 1.5pt,linewidth=0.4pt,linecolor=gray]{c-c}(0,0)(0,1.2)}
\psaxes[labelFontSize=\scriptstyle,xAxis=true,yAxis=true,Dx=0.5,Dy=0.5,ticksize=-2pt 0,subticks=2]{->}(0,0)(0.,0.)(1.2,1.2)
\psplot[linewidth=2.pt,linestyle=dashed,dash=1pt 1pt]{0}{1}{(-0.--1.*x)/1.}
\psplot[linewidth=2.pt,linecolor=uququq]{0}{1}{1-x}
\rput[tl](0.65,-0.06167120265581232){$u_0$}
\begin{scriptsize}
\psdots[dotstyle=+](0.7,0.)
\end{scriptsize}
\psline[linewidth=2.pt,linecolor=red](0.7,0.)(0.7,0.7)
\psline[linewidth=2.pt,linecolor=red](0.7,0.3)(0.3,0.3)
\psline[linewidth=2.pt,linecolor=red](0.3,0.3)(0.3,0.7)
\psline[linewidth=2.pt,linecolor=red](0.3,0.7)(0.7,0.7)
\end{pspicture*}

\medskip

On revient au même endroit toutes les deux étapes, donc \textcolor{red}{la suite $(u_n)_{n\in\mathbb{N}}$ est périodique.}
\end{center}

\vspace{1cm}

\begin{center}
\newrgbcolor{uququq}{0.25098039215686274 0.25098039215686274 0.25098039215686274}
\psset{xunit=4.0cm,yunit=4.0cm,algebraic=true,dimen=middle,dotstyle=o,dotsize=5pt 0,linewidth=2.pt,arrowsize=3pt 2,arrowinset=0.25}
\begin{pspicture*}(-0.178981823685607,-0.18017215627121094)(1.2,1.2)
\multips(0,0)(0,0.5){3}{\psline[linestyle=dashed,linecap=1,dash=1.5pt 1.5pt,linewidth=0.4pt,linecolor=gray]{c-c}(0,0)(1.2,0)}
\multips(0,0)(0.5,0){4}{\psline[linestyle=dashed,linecap=1,dash=1.5pt 1.5pt,linewidth=0.4pt,linecolor=gray]{c-c}(0,0)(0,1.2)}
\psaxes[labelFontSize=\scriptstyle,xAxis=true,yAxis=true,Dx=0.5,Dy=0.5,ticksize=-2pt 0,subticks=2]{->}(0,0)(0.,0.)(1.2,1.2)
\psplot[linewidth=2.pt,linestyle=dashed,dash=1pt 1pt]{0}{1}{(-0.--1.*x)/1.}
\psplot[linewidth=2.pt,linecolor=uququq]{0}{1}{EXP(-x)}
\rput[tl](0.75,-0.06167120265581232){$u_0$}
\begin{scriptsize}
\psdots[dotstyle=+](0.8,0.)
\end{scriptsize}
\psline[linewidth=2.pt,linecolor=red](0.8,0.)(0.8,0.44932896411722156)
\psline[linewidth=2.pt,linecolor=red](0.8,0.44932896411722156)(0.44932896411722156,0.44932896411722156)
\psline[linewidth=2.pt,linecolor=red](0.44932896411722156,0.44932896411722156)(0.44932896411722156,0.6380561665820187)
\psline[linewidth=2.pt,linecolor=red](0.44932896411722156,0.6380561665820187)(0.6380561665820187,0.6380561665820187)
\psline[linewidth=2.pt,linecolor=red](0.6380561665820187,0.6380561665820187)(0.6380561665820187,0.52831838950788)
\psline[linewidth=2.pt,linecolor=red](0.6380561665820187,0.52831838950788)(0.52831838950788,0.52831838950788)
\psline[linewidth=2.pt,linecolor=red](0.52831838950788,0.52831838950788)(0.52831838950788,0.5895956066704154)
\psline[linewidth=2.pt,linecolor=red](0.5283183895078806,0.5895956066704151)(0.5895956066704151,0.5895956066704151)
\psline[linewidth=2.pt,linecolor=red](0.5895956066704151,0.5895956066704151)(0.589595606670415,0.5545514963226404)
\end{pspicture*}

\medskip

La spirale s'enroule autour du point de coordonnées $(0,57~;~0,57)$ environ~: $\textcolor{red}{\lim\limits_{n\to +\infty}u_n\approx 0,57.}$\end{center}

\begin{center}
\newrgbcolor{uququq}{0.25098039215686274 0.25098039215686274 0.25098039215686274}
\psset{xunit=4.0cm,yunit=4.0cm,algebraic=true,dimen=middle,dotstyle=o,dotsize=5pt 0,linewidth=2.pt,arrowsize=3pt 2,arrowinset=0.25}
\begin{pspicture*}(-0.178981823685607,-0.18017215627121094)(1.2,1.2)
\multips(0,0)(0,0.5){3}{\psline[linestyle=dashed,linecap=1,dash=1.5pt 1.5pt,linewidth=0.4pt,linecolor=gray]{c-c}(0,0)(1.2,0)}
\multips(0,0)(0.5,0){4}{\psline[linestyle=dashed,linecap=1,dash=1.5pt 1.5pt,linewidth=0.4pt,linecolor=gray]{c-c}(0,0)(0,1.2)}
\psaxes[labelFontSize=\scriptstyle,xAxis=true,yAxis=true,Dx=0.5,Dy=0.5,ticksize=-2pt 0,subticks=2]{->}(0,0)(0.,0.)(1.2,1.2)
\psplot[linewidth=2.pt,linestyle=dashed,dash=1pt 1pt]{0}{1}{(-0.--1.*x)/1.}
\psplot[linewidth=2.pt,linecolor=uququq]{0}{1}{-2*abs(x-0.5)+1}
\rput[tl](0.2,-0.06167120265581232){$u_0$}
\begin{scriptsize}
\psdots[dotstyle=+](0.23,0.)
\end{scriptsize}
\psline[linewidth=2.pt,linecolor=red](0.23,0.)(0.23,0.46)
\psline[linewidth=2.pt,linecolor=red](0.23,0.46)(0.46,0.46)
\psline[linewidth=2.pt,linecolor=red](0.46,0.46)(0.46,0.92)
\psline[linewidth=2.pt,linecolor=red](0.46,0.92)(0.92,0.92)
\psline[linewidth=2.pt,linecolor=red](0.92,0.92)(0.92,0.16)
\psline[linewidth=2.pt,linecolor=red](0.92,0.16)(0.16,0.16)
\psline[linewidth=2.pt,linecolor=red](0.16,0.16)(0.16,0.32)
\psline[linewidth=2.pt,linecolor=red](0.16,0.32)(0.32,0.32)
\psline[linewidth=2.pt,linecolor=red](0.32,0.64)(0.64,0.64)
\psline[linewidth=2.pt,linecolor=red](0.64,0.64)(0.64,0.72)
\psline[linewidth=2.pt,linecolor=red](0.64,0.72)(0.72,0.72)
\psline[linewidth=2.pt,linecolor=red](0.72,0.72)(0.72,0.56)
\psline[linewidth=2.pt,linecolor=red](0.72,0.56)(0.56,0.56)
\psline[linewidth=2.pt,linecolor=red](0.56,0.56)(0.56,0.88)
\psline[linewidth=2.pt,linecolor=red](0.56,0.88)(0.88,0.88)
\psline[linewidth=2.pt,linecolor=red](0.32,0.32)(0.32,0.64)
\psline[linewidth=2.pt,linecolor=red](0.88,0.88)(0.88,0.24)
\end{pspicture*}

\medskip

On observe un phénomène \og chaotique \fg. \textcolor{red}{La suite $(u_n)_{n\in\mathbb{N}}$ n'a pas de limite.} \end{center}


\end{multicols}

\end{exo}



\newpage

\section{Variables aléatoires, loi binomiale}







\begin{exo}

\begin{enumerate}
\item \setlength{\columnseprule}{1pt}

\begin{multicols}{2}

On rappelle les zones de tir à 2 et 3 points~:


\begin{center}
\psset{xunit=1.0cm,yunit=1.0cm,algebraic=true,dimen=middle,dotstyle=o,dotsize=5pt 0,linewidth=2.pt,arrowsize=3pt 2,arrowinset=0.25}
\begin{pspicture*}(-0.5006126521330643,-3.458863355936357)(4.527406792181142,3.3532920202312764)
\pspolygon[linewidth=2.pt,linecolor=red,fillcolor=red!30!white,fillstyle=solid,opacity=0.1](0.,3.)(4.,3.)(4.,-3.)(0.,-3.)
\pscustom[linewidth=0.8pt,linecolor=blue,fillcolor=blue!30!white,fillstyle=solid,opacity=0.1]{\psplot{0.}{2.}{-sqrt(4.0-x^(2.0))}\lineto(2.,0.)\psplot{2.}{0.}{sqrt(4.0-x^(2.0))}\lineto(0.,-2.)\closepath}
\parametricplot[linewidth=2.pt,linecolor=blue]{-1.5707963267948966}{1.5707963267948966}{1.*2.*cos(t)+0.*2.*sin(t)+0.|0.*2.*cos(t)+1.*2.*sin(t)+0.}
\psline[linewidth=2.pt](0.,3.)(4.,3.)
\psline[linewidth=2.pt](4.,3.)(4.,-3.)
\psline[linewidth=2.pt](0.,-3.)(4.,-3.)
\psline[linewidth=2.pt,linecolor=red](0.,3.)(4.,3.)
\psline[linewidth=2.pt,linecolor=red](4.,3.)(4.,-3.)
\psline[linewidth=2.pt,linecolor=red](4.,-3.)(0.,-3.)
\psline[linewidth=2.pt,linecolor=red](0.,-3.)(0.,3.)
\psline[linewidth=2.pt](0.,0.)(0.31047517904295696,0.)
\pscircle[linewidth=2.pt](0.31047517904295696,0.){0.1515996751199221}
\rput[tl](0.2382719257052241,-0.8817781210369613){\blue{tir à 2 pts}}
\rput[tl](1.9322999822125193,-2.1252668008135926){\red{tir à 3 pts}}
\end{pspicture*}
\end{center}
\columnbreak

On représente la situation par un arbre pondéré~:

\medskip

\begin{center}
\pstree[treemode=R,treesep=1,levelsep=3]{\TR{}}%
{
\pstree{\Tr{$D$}\taput{$0,53$}}
	{
	\Tr{$M$}\taput{$0,52$} 
	\Tr{$\overline{M}$}\tbput{$0,48$}
		}	
\pstree{\Tr{$\overline{D}$}\tbput{$0,47$}}
	{
	\Tr{$M$}\taput{$0,44$} 
	\Tr{$\overline{M}$}\tbput{$0,56$}
		}
}
\end{center}

\medskip

\textbf{Remarque~:} $\overline{D}$ signifie \og Stephen Curry tire à 3 points \fg.
\end{multicols}
\item La probabilité que Stephen Curry tire à 2 points et marque est
\[P(D\cap M)=0,53\times 0,52=\np{0,2756}.\]
\item D'après la formule des probabilités totales, la probabilité que Stephen Curry marque est~:

\begin{align*}P(M)&=P\left(D\cap M\right)+P\left(\overline{D}\cap M\right)\\
&=0,53\times 0,52+0,47\times 0,44=0,\np{4824}.\end{align*}
\item Stephen Curry a marqué. La probabilité qu'il ait tiré à deux points est
\[P_M(D)=\frac{P(M\cap D)}{P(M)}=\frac{\np{0,2756}}{\np{0,4824}}\approx 0,57.\]
\item On reprend l'arbre pondéré de la question 1 et on indique à l'extrémité droite des branches le nombre de points marqués suivant la situation

\medskip

\begin{center}
\pstree[treemode=R,treesep=1,levelsep=3]{\TR{}}%
{
\pstree{\Tr{$D$}\taput{$0,53$}}
	{
	\Tr{$M$}~[tnpos=r]{~~\textcolor{blue}{S.C. marque 2 pts}}\taput{$0,52$} 
	\Tr{$\overline{M}$}~[tnpos=r]{~~\textcolor{green}{S.C. marque 0 pt}}\tbput{$0,48$}
		}	
\pstree{\Tr{$\overline{D}$}\tbput{$0,47$}}
	{
	\Tr{$M$}~[tnpos=r]{~~\textcolor{red}{S.C. marque 3 pts}}\taput{$0,44$} 
	\Tr{$\overline{M}$}~[tnpos=r]{~~\textcolor{green}{S.C. marque 0 pt}}\tbput{$0,56$}
		}
}
\end{center}



\begin{itemize}
\item[\textbullet] la probabilité que Stéphane Curry marque 3 pts est
\[0,47\times 0,44=\np{0,2068}~;\]
\item[\textbullet] la probabilité que Stéphane Curry marque 2 pts est
\[0,53\times 0,52=\np{0,2756}\] (on l'a déjà calculée dans la question 2)~;
\item[\textbullet] d'après la formule des probabilités totales, la probabilité que Stéphane Curry marque 0 pt est
\[0,53\times 0,48+0,47\times 0,56=\np{0,5176}.\]
\end{itemize}

\medskip

\textbf{Remarque~:} Pour le 3\up{e} calcul, on aurait pu utiliser le résultat de la question 3 et faire le calcul \[1-\np{0,4824}=\np{0,5176}.\]

On aurait aussi pu calculer \[1-\np{0,2068}-\np{0,2756}=\np{0,5176}.\]


\medskip

La loi de $X$ est donc donnée par le tableau~:

\begin{center}
\renewcommand{\arraystretch}{1.5}
\begin{tabular}{|c|c|c|c|}
\hline
   $x$     & 3      & 2       & 0   \\
\hline
$P(X=x)$ & $\np{0,2068}$ &$\np{0,2756}$ &$\np{0,5176}$      \\
\hline
\end{tabular}
\end{center}

\medskip

Enfin, l'espérance de $X$ est \[E(X)=\np{0,2068}\times 3+\np{0,2756}\times 2+\np{0,5176}\times 0= \np{1,1716}.\]

En moyenne, Stephen Curry marque \np{1,1716}~point par tir.

\end{enumerate}


\end{exo}



\begin{exo}




 
\begin{enumerate}
\item On indique à l'extrémité droite des branches le coût des trajets (utile pour la question 3).

\medskip

\begin{center}
\pstree[treemode=R,treesep=1,levelsep=3]{\TR{}}%
{
\pstree{\Tr{$A$}\taput{$0,65$}}
	{
	\Tr{$R$}~[tnpos=r]{~~\textcolor{red}{\np{1000}~\euro}}\taput{$0,90$} 
	\Tr{$\overline{R}$}~[tnpos=r]{~~\textcolor{blue}{800~\euro}}\tbput{$0,10$}
		}	
\pstree{\Tr{$\overline{A}$}\tbput{$0,35$}}
	{
	\Tr{$R$}~[tnpos=r]{~~\textcolor{blue}{800~\euro}}\taput{$0,70$} 
	\Tr{$\overline{R}$}~[tnpos=r]{~~\textcolor{green}{600~\euro}}\tbput{$0,30$}
		}
}
\end{center}
\item 
	\begin{enumerate}
		\item La probabilité que le client fasse l'aller-retour en bateau est
		\[P(A\cap R)=0,65\times 0,90=0,585.\]
		\item La probabilité que le client utilise les deux moyens de transport est
		\[P\left(A\cap\overline{R}\right)+P\left(\overline{A}\cap R\right)=0,65\times 0,10+0,35\times 0,70=0,31.\]
		\item On calcule d'abord $P(R)~:$ d'après la formule des probabilités totales, 
		\[P(R)=P\left(A\cap R\right)+P\left(\overline{A}\cap R\right)=0,65\times 0,90+0,35\times 0,70=0,83.\]
		
		\medskip
		
		Le client a fait le retour en bateau. La probabilité qu'il ait aussi fait l'aller en bateau est
		\[P_R(A)=\frac{P(A\cap R)}{P(R)}=\frac{0,585}{0,83}\approx 0,705.\]
	\end{enumerate}
\item 
	\begin{enumerate}
		\item On a déjà fait deux des trois calculs utiles dans les questions précédentes~:
		
		\begin{center}
\renewcommand{\arraystretch}{1.5}
\begin{tabular}{|c|c|c|c|}
\hline
   $y$     & \np{1000}      & 800       & 600   \\
\hline
$P(Y=y)$ & $\np{0,585}$ &$\np{0,31}$ &$?$      \\
\hline
\end{tabular}
\end{center}

\medskip

\[P(Y=600)=1-0,585-0,31=0,105.\]
		\item L'espérance mathématique de $Y$ est
		\[E(Y)=0,585\times\np{1000}+0,31\times 800+0,105\times 600=896.\] C'est la dépense moyenne des clients pour leurs transports du week-end.
	\end{enumerate}
\end{enumerate}
\end{exo}



\begin{exo}




\begin{enumerate}
\item On construit l'arbre et on le complète à partir des données de l'énoncé~:

\medskip

\begin{center}
\pstree[treemode=R,treesep=1,levelsep=3]{\TR{}}%
{
\pstree{\Tr{$R$}\taput{$0,17$}}
	{
	\Tr{$J$}\taput{$0,32$} 
	\Tr{$\overline{J}$}\tbput{$0,68$}
		}	
\pstree{\Tr{$\overline{R}$}\tbput{$0,83$}}
	{
	\Tr{$J$}\taput{\red{$?$}} 
	\Tr{$\overline{J}$}\tbput{\blue{$?$}}
		}
}
\end{center}

\medskip

\item $P(R \cap J)=0,17\times 0,32=\np{0,0544}.$
\item L'énoncé donne $P(J)=0,11.$ On utilise la formule des probabilités totales~:

\begin{align*}
P(J)&=P\left(R\cap J\right)+P\left(\overline{R}\cap J\right)\\
0,11&=\np{0,0544}+P\left(\overline{R}\cap J\right)\\
0,11-\np{0,0544}&=P\left(\overline{R}\cap J\right)\\
\np{0,0556}&=P\left(\overline{R}\cap J\right)
\end{align*}

\medskip

Conclusion~: $P\left(\overline{R}\cap J\right)= \np{0,0556}.$
\item La proportion de jeunes de 18 à 24 ans parmi les utilisateurs non réguliers des transports en commun est égale à la probabilité qu'un utilisateur non régulier soit un jeune. Cette proportion vaut donc
\[P_{\overline{R}}(J)=\frac{P\left(\overline{R}\cap J\right)}{P\left(\overline{R}\right)}=\frac{\np{0,0556}}{0,83}\approx \np{0,0670}.\]

C'est le point d'interrogation rouge de l'arbre pondéré du début.

\end{enumerate}


\end{exo}

\begin{exo}

Il faut prendre l'initiative de nommer des événements et de construire un arbre pondéré. On pose ainsi~:


\begin{itemize}
\item[\textbullet] $E~:$ \og le dé est équilibré \fg ,
\item[\textbullet] $\overline{E}~:$ \og le dé est pipé \fg ,
\item[\textbullet] $S~:$ \og on obtient 6 \fg .
\end{itemize}

\medskip

\begin{center}
\pstree[treemode=R,treesep=1.5,levelsep=2.5]{\TR{}}%
{
\pstree{\Tr{$E$}\taput{$\frac{75}{100}=\frac{3}{4}$}}
	{
	\Tr{$S$}\taput{$\frac{1}{6}$} 
	\Tr{$\overline{S}$}\tbput{$\frac{5}{6}$}
		}	
\pstree{\Tr{$\overline{E}$}\tbput{$\frac{25}{100}=\frac{1}{4}$}}
	{
	\Tr{$S$}\taput{$\frac{1}{2}$} 
	\Tr{$\overline{S}$}\tbput{$\frac{1}{2}$}
		}
}
\end{center}

\medskip

La probabilité qu'il faut calculer est $P_S\left(\overline{E}\right).$ On utilise la formule des probabilités conditionnelles~:
\[P_S\left(\overline{E}\right)=\frac{P\left(\overline{E}\cap S\right)}{P(S)}.\]

Or $P\left(\overline{E}\cap S\right)=\frac{1}{4}\times\frac{1}{2}=\frac{1}{8}$ et $P(S)=\frac{3}{4}\times\frac{1}{6}+\frac{1}{4}\times\frac{1}{2}=\frac{3}{24}+\frac{1}{8}
=\frac{1}{8}+\frac{1}{8}=\frac{2}{8}=\frac{1}{4}$ (formule des probabilités totales), donc~:
\[P_S\left(\overline{E}\right)=\frac{\frac{1}{8}}{\frac{1}{4}}=\frac{1}{8}\times \frac{4}{1}=\frac{4}{8}
=\frac{1}{2}.\]



\end{exo}

\begin{exo} ~{}


\begin{center}
\psset{xunit=1.0cm,yunit=1.0cm,algebraic=true,dimen=middle,dotstyle=o,dotsize=5pt 0,linewidth=2.pt,arrowsize=3pt 2,arrowinset=0.25}
\begin{pspicture*}(0.8,1.9)(5.7,5.4)
\psline[linewidth=2.pt](1,5.2)(1,2.)
\psline[linewidth=2.pt](1,2.)(5.5,2.)
\psline[linewidth=2.pt](5.5,2.)(5.5,5.2)
\psline[linewidth=2.pt](5.5,5.2)(1,5.2)
\rput[tl](1.28,4.88){\red{$\boxed{1}$}}
\rput[tl](2.95,3.84){\green{$\boxed{10}$}}
\rput[tl](3.94,4.92){\green{$\boxed{9}$}}
\rput[tl](1.44,3.88){\blue{$\boxed{2}$}}
\rput[tl](2.26,4.74){\green{$\boxed{6}$}}
\rput[tl](3.86,4.12){\green{$\boxed{11}$}}
\rput[tl](2.38,3.38){\blue{$\boxed{5}$}}
\rput[tl](3.28,4.5){\green{$\boxed{8}$}}
\rput[tl](2.06,4.04){\blue{$\boxed{3}$}}
\rput[tl](1.64,3.14){\blue{$\boxed{4}$}}
\rput[tl](3.64,3.38){\green{$\boxed{13}$}}
\rput[tl](4.52,4.48){\green{$\boxed{12}$}}
\rput[tl](4.36,3.56){\green{$\boxed{14}$}}
\rput[tl](3.,5.02){\green{$\boxed{7}$}}
\rput[tl](4.08,2.88){\green{$\boxed{15}$}}
\end{pspicture*}
\end{center}


\begin{enumerate}
\item \begin{enumerate}
\item La probabilité que la bille tirée soit bleue ou numérotée d'un nombre pair est $\frac{9}{15}.$

(Billes concernées~: \blue{$\boxed{2}$} \blue{$\boxed{3}$} \blue{$\boxed{4}$} \blue{$\boxed{5}$} \green{$\boxed{6}$} \green{$\boxed{8}$} \green{$\boxed{10}$} \green{$\boxed{12}$} \green{$\boxed{14}$}\black .)

\medskip

\textbf{Remarque~:} Si on note A~: \og le numéro est pair \fg~{} et B~: \og la bille est bleue \fg, alors on a calculé la probabilité de l'événement $A\cup B.$ On aurait aussi pu utiliser la formule du cours de 2\up{de} (pas franchement judicieuse ici, mais dont il faut se rappeler)~:
 \[P(A\cup B)=P(A)+P(B)-P(A\cap B).\]
\item Sachant que la bille tirée est verte, la probabilité qu'elle soit numérotée 7 est $\frac{1}{10}.$

(Parmi les 10 boules vertes, une seule porte le numéro 7.)
\end{enumerate}
\item \begin{itemize}
\item[\textbullet] $G=5$ lorsque le joueur remporte 15 euros. Ce n'est possible que s'il tire la bille \blue{$\boxed{5}$}\black ~ (puisque $3\times 5=15$) ou la bille \green{$\boxed{15}$}\black.

Conclusion~: $P(G=5)=\frac{2}{15}.$

\item[\textbullet] Si le joueur choisit une bille rouge, il ne remporte rien, donc $G=-10.$ On a donc $P_R(G = 0)=0.$
\item[\textbullet] Il y a deux façons possibles d'avoir $G=-4~:$
\begin{itemize}
\item soit le joueur tire la bille \blue{$\boxed{2}$}\black, et dans ce cas $G=3\times 2-10=6-10=-4~;$
\item soit le joueur tire la bille \green{$\boxed{6}$}\black, et dans ce cas $G=6-10=-4.$
\end{itemize}
Donc sachant que $G=-4$ il y a une chance sur deux que la bille soit bleue, une chance sur deux qu'elle soit verte~: $P_{(G = - 4)}(V)=\frac{1}{2}.$
\end{itemize}
\end{enumerate}


\end{exo}


\begin{exo} 

\danger J'ai remplacé le mot \og pair \fg~{} par le mot \og impair \fg~{} dans l'énoncé.


\begin{center}
\psset{xunit=1.0cm,yunit=1.0cm,algebraic=true,dimen=middle,dotstyle=o,dotsize=5pt 0,linewidth=2.pt,arrowsize=3pt 2,arrowinset=0.25}
\begin{pspicture*}(0.8,1.9)(5.7,5.4)
\psline[linewidth=2.pt](1,5.2)(1,2.)
\psline[linewidth=2.pt](1,2.)(5.5,2.)
\psline[linewidth=2.pt](5.5,2.)(5.5,5.2)
\psline[linewidth=2.pt](5.5,5.2)(1,5.2)
\rput[tl](2.95,3.84){\blue{$\boxed{1}$}}
\rput[tl](1.44,3.88){\red{$\boxed{1}$}}
\rput[tl](3.86,4.12){\blue{$\boxed{2}$}}
\rput[tl](2.38,3.38){\red{$\boxed{2}$}}
\rput[tl](3.28,4.5){\blue{$\boxed{3}$}}
\rput[tl](2.06,4.04){\red{$\boxed{3}$}}
\rput[tl](1.64,3.14){\red{$\boxed{4}$}}
\rput[tl](3.64,3.38){\blue{$\boxed{4}$}}
\rput[tl](4.52,4.48){\blue{$\boxed{5}$}}
\rput[tl](4.36,3.56){\blue{$\boxed{6}$}}
\end{pspicture*}
\end{center}

On note

\begin{itemize}
\item[\textbullet] $B~:$ \og les deux boules sont bleues \fg ,
\item[\textbullet] $I~:$ \og les deux boules portent un numéro impair \fg .
\end{itemize}

\medskip

Il faut calculer $P_I(B)=\frac{P(B\cap I)}{P(I)}.$ Comme le tirage des boules est simultané, l'utilisation d'un arbre pondéré  n'est pas adaptée à la situation~; il faut utiliser des combinaisons pour comptabiliser les cas possibles et les cas favorables aux événements~:

\begin{itemize}
\item[\textbullet] Il y a $\binom{10}{2}=\frac{10\times 9}{2}=45$ tirages possibles.
\item[\textbullet] L'événement $I$ est réalisé quand on tire 2 des 5 boules portent un numéro impair \textcolor{red}{$\boxed{1}$} \textcolor{red}{$\boxed{3}$} \textcolor{blue}{$\boxed{1}$} \textcolor{blue}{$\boxed{3}$} \textcolor{blue}{$\boxed{5}$}, donc il y a $\binom{5}{2}=\frac{5\times 4}{2}=10$ cas favorables à $I.$
\item[\textbullet] L'événement $B\cap I$ est réalisé quand on tire deux boules parmi \textcolor{blue}{$\boxed{1}$} \textcolor{blue}{$\boxed{3}$} \textcolor{blue}{$\boxed{5}$}, donc il y a $\binom{3}{2}=\frac{3\times 2}{2}=3$ cas favorables à $B\cap I.$
\end{itemize}

\medskip

On en déduit $P(I)=\frac{10}{45}$ et $P(B\cap I)=\frac{3}{45},$ puis
\[P_I(B)=\frac{P(B\cap I)}{P(I)}=\frac{\frac{3}{45}}{\frac{10}{45}}=\frac{3}{45}\times \frac{45}{10}
=\frac{3}{10}
.\]

%\medskip

%\textbf{Remarque~:} Pour traiter l'exercice plus rapidement, on pouvait dire que le \og sachant $I$ \fg~{} revenait à considérer que l'on tirait deux boules parmi \textcolor{red}{$\boxed{1}$} \textcolor{red}{$\boxed{3}$} \textcolor{blue}{$\boxed{1}$} \textcolor{blue}{$\boxed{3}$} \textcolor{blue}{$\boxed{5}$}~; il y avait donc $\binom{5}{2}=10$ cas possibles. Et parmi ces cas possibles, $\binom{3}{2}$ cas favorables à l'événement $B,$ puisque 3 des 5 boules impaires sont bleues.



\end{exo}

\begin{exo}


\begin{enumerate}
\item On répète $10$ épreuves indépendantes de Bernoulli de paramètre $0,04,$ donc $X$ suit la loi binomiale de paramètres $n=10,$ $p=0,04.$ On a donc $E(X)=n\times p=10\times 0,04=0,4.$

\medskip

\danger $p=0,04$ et non $0,96$ ($X$ désigne le nombre d'adresses \textbf{illisibles}).
\item 
 $P(X=2)=\binom{10}{2}\times 0,04^{2}\times (1-0,04)^{10-2}=45\times 0,04^2\times 0,96^8\approx 0,052.$
 
 \medskip
 
 Schéma pour retenir~:
 
 \begin{center}
\psset{xunit=1.0cm,yunit=1.0cm,algebraic=true,dimen=middle,dotstyle=o,dotsize=5pt 0,linewidth=2.pt,arrowsize=3pt 2,arrowinset=0.25}
\begin{pspicture*}(-4.3,0.82)(5.7,5.1)
\psline[linewidth=2.pt](-4.25,0.9)(-4.25,4.9)
\psline[linewidth=2.pt](-4.25,4.9)(5.6,4.9)
\psline[linewidth=2.pt](5.6,0.9)(5.6,4.9)
\psline[linewidth=2.pt](-4.25,0.9)(5.6,0.9)
\rput[tl](-3.2,3.2){\Large $P(X=\textcolor{violet}{2})=\binom{\textcolor{green}{10}}{\textcolor{violet}{2}}\times\textcolor{blue}{0,04}^{\textcolor{violet}{2}}\times \textcolor{red}{0,96}^{\textcolor{orange}{8}}$}
\psline[linewidth=2.pt,linecolor={violet}]{->}(-1.22,1.54)(-0.6,2.4)
\rput[tl](-2.46,1.34){\textcolor{violet}{nb adresses illisibles}}
\psline[linewidth=2.pt,linecolor=blue]{->}(-0.02,4.02)(0.55,3.2)
\rput[tl](-1.2,4.46){\blue{proba adresse illisible}}
\psline[linewidth=2.pt,linecolor=green]{->}(-2.06,4.08)(-0.6,3.45)
\rput[tl](-3.94,4.5){\green{nb adresses lues}}
\psline[linewidth=2.pt,linecolor=red]{->}(2.14,1.8)(2.4,2.55)
\rput[tl](1.34,1.56){\red{proba adresse lisible}}
\psline[linewidth=2.pt,linecolor=orange]{->}(3.1,3.92)(2.9,3.25)
\rput[tl](2.68,4.44){\textcolor{orange}{nb adresses lisibles}}
\end{pspicture*}
\end{center}
 \item La probabilité qu'au moins une adresse soit illisible est \[P(X\geq 1)=1-P(X=0)=1-\underbrace{\binom{10}{0}}_{=1}\times \underbrace{0,04^{0}}_{=1}\times (1-0,04)^{10-0}=1-0,96^{10}\approx 0,335.\]
\end{enumerate}

\end{exo}




\begin{exo}



\begin{enumerate}
\item Pour chaque question, l'élève a une chance sur 4 de donner la bonne réponse, 3 chances sur 4 de donner une mauvaise réponse.

On répète $10$ épreuves indépendantes de Bernoulli de paramètre $\frac{1}{4}=0,25,$ donc $X$ suit la loi binomiale de paramètres $n=10, $ $p=0,25.$

\item \begin{itemize}
\item[\textbullet] $A$ est réalisé quand $X=3,$ donc \[P(A)=P(X=3)=\binom{10}{3}\times 0,25^3\times (1-0,25)^{10-3}=120\times 0,25^3\times 0,75^7\approx 0,250.\]
\item[\textbullet] On utilise l'événement contraire~:
\[P(B)=P(X\geq 1)=1-P(X=0)=1-\underbrace{\binom{10}{0}}_{=1}\times \underbrace{0,25^{0}}_{=1}\times 0,75^{10-0}=1- 0,75^{10}\approx \np{0,944}.\]
\end{itemize}

\item L'élève a au moins une bonne réponse. La probabilité qu'il en ait exactement trois est
\[P_{(X\geq 1)}(X=3)=\frac{P((X\geq 1)\cap (X=3))}{P(X\geq 1)}.\]


Ici, il y a une petite subtilité~: l'événement $(X\geq 1)\cap (X=3)$ est égal à $(X=3).$ En effet, dire que $X$ est à la fois plus grand que 1 et égal à 3 revient simplement à dire qu'il est égal à 3. On a donc finalement
\[P_{(X\geq 1)}(X=3)=\frac{P(X=3)}{P(X\geq 1)}\approx \frac{0,250}{0,944}\approx 0,265.\]

\end{enumerate}


\end{exo}





\begin{exo}


 
\begin{enumerate}
\item On répète $100$ épreuves indépendantes de Bernoulli de paramètre $\frac{1}{10},$ donc $X$ suit la loi binomiale de paramètres $n=100, $ $p=\frac{1}{10}=0,1.$

\medskip

D'après le  cours~:
\[E(X)=n\times p=100\times 0,1=10.\]%\qquad, \qquad V(X)=n\times p\times (1-p)=100\times 0,1\times (1-0,1)=9.\]
\item À l'aide de la calculatrice, on obtient $P(5\leq X\leq 15)\approx 0,936.$
\item On cherche par tâtonnement, à l'aide de la calculatrice, le plus petit entier naturel $k$ tel que \[P(X> k)\leq 0,01.\] C'est très rapide avec une NUMWORKS, un peu moins avec les CASIO et les TI. Une méthode qui fonctionne quel que soit le modèle consiste à utiliser l'événement contraire~:
\[ P(X> k)\leq 0,01\iff P(X\leq k)\geq 0,99.\]

Par tâtonnement, on trouve 
\[P(X\leq 17)=0,989\cdots\qquad\text{et}\qquad P(X\leq 18)=0,995\cdots,\] donc le plus petit entier $k$ qui convienne est $18.$

\end{enumerate}
\end{exo}

\begin{exo}

Soit $X$ une variable aléatoire suivant la loi binomiale de paramètres $n=50,$ $p=0,8.$

\`A l'aide de la calculatrice, on obtient~:
\begin{align*}
P(X\geq 35)&\approx 0,969\\
P(X\geq 40)&\approx 0,584.
\end{align*}
On raisonne comme dans l'exercice 121~: l'événement  $(X\geq 35)\cap (X\geq 40)$ est égal à $(X\geq 40),$ donc

\[P_{(X\geq 35)}(X\geq 40)=\frac{P((X\geq 35)\cap (X\geq 40))}{P(X\geq 35)}
=\frac{P(X\geq 40)}{P(X\geq 35)}\approx \frac{0,584}{0,969}\approx 0,603.\]

\end{exo}




\begin{exo}


On note $X$ le nombre de 6. On répète $n$ épreuves indépendantes de Bernoulli de paramètre $\frac{1}{6},$ donc $X$ suit la loi binomiale de paramètres $n, $ $p=\frac{1}{6}.$

\medskip

La probabilité de faire au moins un 6 est 
\[P(X\geq 1)=1-P(X=0)=1-\underbrace{\binom{n}{0}}_{=1}\times \underbrace{\left(\frac{1}{6}\right)^0}_{=1}\times\left(1-\frac{1}{6}\right)^{n-0}=1-\left(\frac{5}{6}\right)^{n}.\]
\end{exo}

\begin{exo}



\begin{enumerate}
\item 	\begin{enumerate}
		\item On répète $5$ épreuves indépendantes de Bernoulli de paramètre $0,103,$ donc $X$ suit la loi binomiale de paramètres $n=5, $ $p=0,103.$
		\item L'espérance de $X$ est \[E(X)=n\times p=5\times 0,103=0,515.\] 
		
		\item La probabilité qu'au moins un des 5 athlètes contrôlés présente un test positif est 
		\[P(X\geq 1)=1-P(X=0)=1-\underbrace{\binom{5}{0}}_{=1}\times \underbrace{0,103^{0}}_{=1}\times (1-0,103)^{5-0}=1- 0,897^{5}\approx \np{0,419}.\]
	\end{enumerate}
\item Si on interroge $n$ athlètes, la probabilité qu'au moins l'un  d'entre eux présente un test positif est \[1-0,897^n\] (c'est le même calcul que dans la question précédente).

\medskip

On cherche donc par tâtonnement la plus petite valeur de $n$ telle que $1-0,897^n\geq 0,75~:$ on trouve $1-0,897^{12}\approx 0,729$ et $1-0,897^{12}\approx 0,757.$

\medskip

Conclusion~: il faut contrôler au moins 13 athlètes pour que la probabilité de l'événement \og au moins un athlète contrôlé présente un test positif \fg{} soit supérieure ou égale à $0,75.$
\end{enumerate}

\end{exo}

\begin{exo}



\begin{enumerate}
\item Au départ, la trottinette est en bon état, donc la probabilité qu'elle le soit encore après 1 semaine est $p_1=0,9.$ 
\medskip

Pour calculer $p_2,$ on utilise un arbre pondéré~:

\begin{center}
\pstree[treemode=R,nodesepA=0pt,nodesepB=2.5pt,treesep = 0.5cm,levelsep=2.5cm]{\TR{}}
{\pstree{\TR{$B_1$~}\taput{$0,9$}}
{\TR{$B_{2}$}\taput{$0,9$}
\TR{$\overline{B_{2}}$}\tbput{$0,1$}
}
\pstree{\TR{$\overline{B_1}$~}\tbput{$0,1$}}
{\TR{$B_{2}$}\taput{$0,4$}
\TR{$\overline{B_{2}}$}\tbput{$0,6$}
}}
\end{center}

\medskip

D'après la formule des probabilités totales~:
\[p_2=P\left(B_2\right)=P\left(B_1\cap B_2\right)+P\left(\overline{B_1}\cap B_2\right)=0,9\times 0,9+0,1\times 0,4=0,85.\]


\item La méthode de la question 1 se généralise~: la probabilité de l'événement $B_n$ est $p_n,$ celle de l'événement contraire $\overline{B_{n}}$ est $1-p_n.$ On a donc l'arbre~:

\begin{center}
\pstree[treemode=R,nodesepA=0pt,nodesepB=2.5pt,treesep = 0.5cm,levelsep=2.5cm]{\TR{}}
{\pstree{\TR{$B_n$~}\taput{$p_n$}}
{\TR{$B_{n+1}$}\taput{$0,9$}
\TR{$\overline{B_{n+1}}$}\tbput{$0,1$}
}
\pstree{\TR{$\overline{B_n}$~}\tbput{$1-p_n$}}
{\TR{$B_{n+1}$}\taput{$0,4$}
\TR{$\overline{B_{n+1}}$}\tbput{$0,6$}
}}

\end{center}

\item D'après la formule des probabilités totales~:

\begin{align*}p_{n+1}&=P\left(B_{n+1}\right)\\
&=P\left(B_n\cap B_{n+1}\right)+P\left(\overline{B_n}\cap B_{n+1}\right)\\
&=p_n\times 0,9+\left(1-p_n\right)\times 0,4\\
&=0,9p_n+0,4-0,4p_n\\
&=0,5p_n + 0,4.
\end{align*}
\item On admet que la suite $\left(p_{n}\right)_{n\in\mathbb{N}}$ converge, on note $\ell$ sa limite. On \og passe à la limite \fg~{} dans la formule de récurrence~:

\[p_{n+1}=0,5u_n+0,4\qquad\text{pour tout }n\in\mathbb{N},\] donc
\[\ell=0,5\ell+0,4.\]

On résout cette équation~:
\[\ell=0,5\ell+0,4\iff \ell-0,5\ell=0,4\iff 0,5\ell=0,4\iff\ell=\frac{0,4}{0,5}\iff \ell=0,8.\]

Conclusion~: $\lim\limits_{n\to +\infty}p_n=0,8.$

\medskip

\textbf{Interprétation~:} \`A long terme, la probabilité qu'une trottinette prise au hasard soit en bon état est égale à 80~\%. Autre façon de dire~: sur le long terme, 80~\% des trottinettes seront en bon état de fonctionnement.
\end{enumerate}

\end{exo}

\begin{exo}




\begin{enumerate}
\item \begin{enumerate}
\item En 2024, 85~\% des \np{1700} sportifs qui sont dans le club A en 2023 restent dans ce club, et 10~\% des \np{1300} sportifs qui sont dans le club B en 2023 quittent ce club pour adhérer au club A. Il y a donc
\[a_1=0,85\times\np{1700}+0,10\times \np{1300}=\np{1575}\] membres dans le club A en 2024.

\medskip

De plus, le nombre total de membres dans le groupe n'évolue pas au cours des années, car les membres se contentent de changer de club (il n'y a aucune arrivée, ni aucun départ)~; il y a donc toujours $\np{1700}+\np{1300}=\np{3000}$ membres au total. Par conséquent, il y a
\[b_1=\np{3000}-\np{1575}=\np{1425}\] membres dans le club B en 2024.
\item On reprend l'argument que l'on a utilisé juste au-dessus~: le nombre total de membres n'évolue pas au cours du temps, donc pour tout entier naturel $n,$ 
\[a_n+b_n=\np{3000}.\]
\item D'une année à la suivante~:
\begin{itemize}
\item[\textbullet] 85~\% des membres du club A y restent~;
\item[\textbullet] 10~\% des membres du club B  adhèrent au club A.
\end{itemize}

On a donc
\[a_{n+1}=0,85a_n+0,10b_n.\] Or on a vu que $a_n+b_n=\np{3000},$ donc $b_n=\np{3000}-a_n$ et
\begin{align*}
a_{n+1}&=0,85a_n+0,10b_n
\\&0,85a_n+0,10\left(\np{3000}-a_n\right)
\\&=0,85a_n+0,10\times\np{3000}-0,10 a_n
\\&=0,75a_n+300.
\end{align*}

\end{enumerate}
\item 
	\begin{enumerate}
		\item Pour tout $n\in\mathbb{N},$ on note $\mathcal{P}_n$ la propriété \[\np{1200} \leq a_{n+1} \leq a_{n} \leq \np{1700}.\]

\begin{itemize}
\item[{\textbullet}] \textbf{Initialisation.} On prouve que $\mathcal{P}_0$ est vraie.


\[
\left.
    \begin{array}{ll}
        a_0&=\np{1700} \\
        a_1&=\np{1575}\\
        \np{1200}&\leq \np{1575}\leq \np{1700}
    \end{array}
\right \}\implies  \np{1200} \leq a_{1}\leq a_{0}\leq \np{1700}\implies\mathcal{P}_0~\text{est vraie}.
\]



\item[{\textbullet}] \textbf{Hérédité.} Soit $k\in\mathbb{N}$ tel que $\mathcal{P}_k$ soit vraie. On a donc
\[\np{1200} \leq a_{k+1} \leq a_{k} \leq \np{1700}.\]

\newtcolorbox{mybox}[1]{colback=green!10!white,colframe=green!80!white,fonttitle=\bfseries,title=#1}
\begin{mybox}{Objectif}{Prouver que $\mathcal{P}_{k+1}$ est vraie, c'est-à-dire que \[\np{1200} \leq a_{k+2} \leq a_{k+1} \leq \np{1700}.\]
}\end{mybox}



%\medskip

On part de 
\[\np{1200} \leq a_{k+1} \leq a_{k} \leq \np{1700}.\]


On multiplie par $\textcolor{red}{0,75}~:$

\begin{align*} \np{1200}\textcolor{red}{\times 0,75}&\leq a_{k+1}\textcolor{red}{\times 0,75}\leq a_{k}\textcolor{red}{\times 0,75}\leq \np{1700}\textcolor{red}{\times 0,75}\\
 900&\leq 0,75 a_{k+1}\leq 0,75 a_{k}\leq \np{1275}.
\end{align*}

Puis on ajoute  $\textcolor{blue}{300}~:$

\begin{align*}
900\textcolor{blue}{+300}&\leq 0,75 a_{k+1}\textcolor{blue}{+300}\leq 0,75 a_{k}\textcolor{blue}{+300}\leq \np{1275}\textcolor{blue}{+300}\\
\np{1200} &\leq a_{k+2} \leq a_{k+1} \leq \np{1575}.
\end{align*}

La propriété $\mathcal{P}_{k+1}$ est donc vraie.
\item[{\textbullet}] \textbf{Conclusion.} $\mathcal{P}_0$ est vraie et $\mathcal{P}_n$ est héréditaire, donc elle est vraie pour tout $n\in\mathbb{N}.$
\end{itemize}
		\item D'après la question précédente~:

\begin{itemize}
\item[{\textbullet}] $a_{n+1}\leq a_{n}$ pour tout $n\in\mathbb{N},$ donc $(a_n)_{n\in\mathbb{N}}$ est décroissante.
\item[{\textbullet}] $a_n\geq \np{1200}$ pour tout $n\in\mathbb{N},$ donc $(a_n)_{n\in\mathbb{N}}$ est minorée par \np{1200}.
\end{itemize}

Or toute suite décroissante minorée converge, donc $(a_n)_{n\in\mathbb{N}}$ converge.
		\item On note $\ell$ la limite de $\left(a_n\right)_{n\in\mathbb{N}}$ et \og on passe à la limite \fg~{} dans la formule de récurrence~:

\[a_{n+1}=0,75a_n+300\qquad\text{pour tout }n\in\mathbb{N},\] donc
\[\ell=0,75\ell+300.\]

On résout cette équation~:
\[\ell=0,75\ell+300\iff \ell-0,75\ell=300\iff 0,25\ell=300\iff\ell=\frac{300}{0,25}\iff \ell=\np{1200}.\]

Conclusion~: $\lim\limits_{n\to +\infty}a_n=\np{1200}.$
	\end{enumerate}
\item \begin{enumerate}
		\item On complète la fonction \textbf{seuil} afin qu'elle renvoie la plus petite valeur de n à partir de laquelle le nombre de membres du club A est strictement inférieur à \np{1280}.
		
		
\begin{lstlisting}
def seuil():
	A=1700
	n=0
	while A>=1280:
		A=0.75*A+300
		n=n+1
	return n
\end{lstlisting}




		\item Il y a essentiellement deux méthodes pour déterminer la valeur renvoyée par la fonction~: soit on rentre le programme en machine, soit on calcule les termes successifs de la suite.
		
		\medskip
		
		Avec la deuxième méthode, on obtient $a_6\approx \np{1289}$ et $a_7\approx \np{1267}.$ C'est donc après 7 ans qu'il y aura moins de \np{1280} inscrits dans le club A -- et donc la valeur renvoyée par la fonction est \og 7 \fg.
	\end{enumerate}
\end{enumerate}	

\end{exo}

\section{Limites de fonctions}



\begin{exo}

\begin{multicols}{2}
On trace un exemple de courbe correspondant aux conditions de l'énoncé. On indique par une couleur quelle condition correspond à quel élément du graphique.

\begin{enumerate}
\item \textcolor{blue!60!white}{La fonction $h$ est définie sur $\left]1;+\infty\right[.$}

\begin{enumerate}
\item \textcolor{green}{La droite d'équation $y=-2$ est asymptote à $\mathcal{C}$ en $+\infty.$}
\item \textcolor{blue}{La droite d'équation $x=1$ est asymptote à $\mathcal{C}.$}
\item \textcolor{red}{$h$ est strictement croissante sur $\left]1;3\right]$} et \textcolor{violet!50!blue}{strictement décroissante sur $\left[3;+\infty\right[.$}
\item \textcolor{orange}{$h(3)=4.$}
\item \textcolor{pink!20!magenta}{Les solutions de l'équation $h(x)=0$ sont $1,5$ et $6.$}
\item \textcolor{black}{Le point de coordonnées $(5;2)$ est un point d'inflexion de $\mathcal{C}.$}
\end{enumerate}


\item $\lim\limits_{x\to +\infty} f(x)=-2\qquad ,\qquad\lim\limits_{x\to 1} f(x)=-\infty.$
\end{enumerate}


\begin{center}
\newrgbcolor{ffxfqq}{1. 0.4980392156862745 0.}
\newrgbcolor{ccqqww}{0.8 0. 0.4}
\newrgbcolor{ccqqww}{0.8 0. 0.4}
\newrgbcolor{ccqqqq}{0.8 0. 0.}
\newrgbcolor{xfqqff}{0.4980392156862745 0. 1.}
\psset{xunit=0.75cm,yunit=0.75cm,algebraic=true,dimen=middle,dotstyle=o,dotsize=5pt 0,linewidth=2.pt,arrowsize=3pt 2,arrowinset=0.25}
\begin{pspicture*}(-1.06,-2.66)(10.22,4.46)
\pspolygon[linewidth=0.pt,linecolor=white,fillcolor=blue!30!white,fillstyle=solid,opacity=0.25](-1.26,4.66)(-1.26,-2.86)(1.,-2.86)(1.,4.66)
%\multips(0,-2)(0,1.0){8}{\psline[linestyle=dashed,linecap=1,dash=1.5pt 1.5pt,linewidth=0.4pt,linecolor=lightgray]{c-c}(-1.06,0)(10.22,0)}
%\multips(1.03,0)(1.0,0){10}{\psline[linestyle=dashed,linecap=1,dash=1.5pt 1.5pt,linewidth=0.4pt,linecolor=lightgray]{c-c}(0,-2.66)(0,4.46)}
\psaxes[labelFontSize=\scriptstyle,xAxis=true,yAxis=true,Dx=1.,Dy=1.,ticksize=-2pt 0,subticks=2]{->}(0,0)(-1.06,-2.66)(10.22,4.46)
%\psplot[linewidth=2.pt,linecolor=green]{-1.06}{10.22}{(-2.-0.*x)/1.}
%\input{courbe_ex8.1}
\psplot[linewidth=2.pt,linestyle=dashed,dash=2pt 2pt,linecolor=green]{-1.06}{10.22}{(-2.-0.*x)/1.}
\psplot[linewidth=2.pt,linecolor=ccqqqq,plotpoints=200]{1.1}{3}{4.0-4.0/1.5^(3.0)*(abs(x-3.0))^(3.0)}
\psplot[linewidth=2.pt,linecolor=xfqqff,plotpoints=200]{3}{5}{4.0-0.25*(x-3.0)^(3.0)}
\psplot[linewidth=2.pt,linecolor=xfqqff,plotpoints=200]{5}{10.220000000000011}{2.0*(2.0^(6.0-x)-1.0)}
\psline[linewidth=2.pt,linestyle=dashed,dash=2pt 2pt,linecolor=blue](1.,-2.66)(1.,4.46)
\begin{large}
\psdots[dotstyle=*,linecolor=ffxfqq](3.,4.)
\psdots[dotstyle=*](5.,2.)
\psdots[dotstyle=*,linecolor=ccqqww!50!white](1.5,0.)
\psdots[dotstyle=*,linecolor=ccqqww!50!white](6.,0.)
\end{large}
\end{pspicture*}
\end{center}

\end{multicols}

\end{exo}

\begin{exo}

\begin{multicols}{2}

~{}

\begin{tikzpicture}[scale=0.6]
\tkzTabInit{$x$/1,$i(x)$/2}{$-\infty$,$1$,$3$,$+\infty$}
\tkzTabVar{+/$2$,-D+/$-\infty$/$+\infty$,-/$-1$,+/$1$}
\end{tikzpicture}



\begin{itemize}
\item[\textbullet] la droite d'équation $y=1$ (en rose) est asymptote à $\mathcal{C}$ en $+\infty.$
\item[\textbullet] la droite d'équation $y=2$ (en vert) est asymptote à $\mathcal{C}$ en $-\infty.$
\item[\textbullet] la droite d'équation $x=1$ (en orange) est asymptote à $\mathcal{C}.$ 
\end{itemize}
\columnbreak
\begin{center}
\newrgbcolor{ffxfqq}{1. 0.4980392156862745 0.}
\psset{xunit=0.75cm,yunit=0.75cm,algebraic=true,dimen=middle,dotstyle=o,dotsize=5pt 0,linewidth=2.pt,arrowsize=3pt 2,arrowinset=0.25}
\begin{pspicture*}(-3.52,-2.48)(7.88,3.84)
%\multips(0,-2)(0,1.0){7}{\psline[linestyle=dashed,linecap=1,dash=1.5pt 1.5pt,linewidth=0.4pt,linecolor=lightgray]{c-c}(-3.52,0)(7.88,0)}
%\multips(-3,0)(1.0,0){12}{\psline[linestyle=dashed,linecap=1,dash=1.5pt 1.5pt,linewidth=0.4pt,linecolor=lightgray]{c-c}(0,-2.48)(0,3.84)}
\psaxes[labelFontSize=\scriptstyle,xAxis=true,yAxis=true,Dx=1.,Dy=1.,ticksize=-2pt 0,subticks=2]{->}(0,0)(-3.52,-2.48)(7.88,3.84)
\psplot[linewidth=2.pt,linecolor=blue,plotpoints=200]{-4}{0.96}{2.0/(x-1.0)+2.0}
\psplot[linewidth=2.pt,linecolor=blue,plotpoints=200]{1.2}{3}{(x-3.0)^(4.0)-1.0}
\psplot[linewidth=2.pt,linecolor=blue,plotpoints=200]{3}{7.880000000000006}{1.13*(EXP(x-6.0)-1.0)/(EXP(x-6.0)+1.0)}
\psplot[linewidth=2.pt,linestyle=dashed,dash=2pt 2pt,linecolor=magenta]{-3.52}{7.88}{(--1.-0.*x)/1.}
\psplot[linewidth=2.pt,linestyle=dashed,dash=2pt 2pt,linecolor=green]{-3.52}{7.88}{(--2.-0.*x)/1.}
\psline[linewidth=2.pt,linestyle=dashed,dash=2pt 2pt,linecolor=ffxfqq](1.,-2.48)(1.,3.84)
\rput[tl](-2.92,2.5){\green{$y=2$}}
\rput[tl](6.08,1.5){\magenta{$y=1$}}
\rput[tl](1.08,-1.68){\ffxfqq{$x=1$}}
\rput[tl](4.88,-0.65){\blue{$\mathcal{C}$}}
\psdots[dotstyle=*,linecolor=red](0.,0.)
\psdots[dotstyle=*,linecolor=red](2.,0.)
\psdots[dotstyle=*,linecolor=red](3.,-1.)
\psdots[dotstyle=*,linecolor=red](6.,0.)
\end{pspicture*}
\end{center}
\end{multicols}

\end{exo}


\begin{exo}

\begin{enumerate}
\item $\lim\limits_{x\to +\infty}\frac{1}{x}=0,$ donc 
\[\lim\limits_{x\to +\infty}\left(2+\frac{3}{x}\right)=\lim\limits_{x\to +\infty}\left(2+3\times \frac{1}{x}\right)=2+3\times 0=2.\]
\item $\lim\limits_{x\to +\infty}\frac{1}{x}=0,$ donc
\[
\left.
    \begin{array}{ll}
        \lim\limits_{x\to +\infty}\left(1-\frac{1}{x}\right)&= 1-0=1 \\
        \lim\limits_{x\to +\infty}\left(1+\frac{1}{x}\right)&= 1+0=1
    \end{array}
\right \}\implies \lim\limits_{x\to +\infty}\left(1-\frac{1}{x}\right)\left(1+\frac{1}{x}\right)=  1\times 1=1.
\]
\item $\lim\limits_{x\to 0,~x>0}\frac{1}{x}=+\infty,$ donc
\[
\left.
    \begin{array}{ll}
        \lim\limits_{x\to 0,~x>0}\left(1-\frac{1}{x}\right)= & \text{\og}~1-(+\infty)~\text{ \fg}=-\infty\\
        \lim\limits_{x\to 0,~x>0}\left(1+\frac{1}{x}\right)= & \text{\og}~1+(+\infty)~\text{ \fg}=+\infty
    \end{array}
\right \}\implies \lim\limits_{x\to 0,~x>0}\left(1-\frac{1}{x}\right)\left(1+\frac{1}{x}\right)= \text{\og}~(-\infty)\times(+\infty)~\text{ \fg}=-\infty.
\]
\item \[
\left.
    \begin{array}{ll}
        \lim\limits_{x\to -\infty}\text{e}^{x}&= 0 \\
        \lim\limits_{x\to -\infty}3x&= \text{\og}~3\times(-\infty)~\text{ \fg}=-\infty\\
        \lim\limits_{x\to -\infty}1&= 1
    \end{array}
\right \}\implies \lim\limits_{x\to -\infty}\left(\text{e}^{x}+3x-1\right)= \text{\og}~0+(-\infty)-1~\text{ \fg}=-\infty.
\]
\item $\lim\limits_{x\to -\infty}\left(1+x\right)=\text{\og}~1+(-\infty)~\text{ \fg}=-\infty,$ donc
\[\lim\limits_{x\to -\infty}\frac{2}{1+x}=\text{\og}~\frac{2}{-\infty}~\text{ \fg}=0.\]
\item $\lim\limits_{x\to 0,~x<0}\frac{1}{x^2}=+\infty,$ donc
\[\lim\limits_{x\to 0,~x<0}\left(5+\frac{1}{x^2}\right)=\text{\og}~5+(+\infty)~\text{ \fg}=+\infty.\]
\item
\[\left.
    \begin{array}{ll}
        \lim\limits_{x\to +\infty}\text{e}^{-x}&= 0 \\
        \lim\limits_{x\to +\infty}x&=+\infty
    \end{array}
\right \}\implies \lim\limits_{x\to +\infty}\frac{\exp(-x)}{x}= \text{\og}~\frac{0}{+\infty}~\text{ \fg}=0.
\]
\end{enumerate}

\end{exo}

\begin{exo}

La fonction $f$ est définie sur $\mathbb{R}^*$ par $f(x)=x+\frac{4}{x}.$ On note $\mathcal{C}$ sa courbe représentative.

\medskip

\textbf{Remarque~:} $\mathbb{R}^*=\mathbb{R}-\left\{0\right\}$ est l'ensemble de tous les nombres sauf 0 -- ici 0 est \og valeur interdite \fg.

\begin{enumerate}
\item On peut écrire $f(x)=1+4\times\frac{1}{x},$ donc pour tout $x\in \mathbb{R}^*~:$
\[f'(x)=1+4\times\left(-\frac{1}{x^2}\right)=1-\frac{4}{x^2}=\frac{x^2}{x^2}-\frac{4}{x^2}=\frac{x^2-4}{x^2}.\]

\medskip

\textbf{Remarque~:} On aurait pu s'arrêter à $1-\frac{4}{x^2},$ mais c'eût été malcommode pour étudier le signe.

\item Les racines de $x^2-4$ sont évidentes~: ce sont $x=2$ et $x=-2.$ On a donc le tableau~:

\medskip
\begin{center}
\begin{tikzpicture}[scale=0.75]
\tkzTabInit{$x$/1,$x^2-4$/1,$x^2$/1,$f'(x)$/1,$f(x)$/2}{$-\infty$,$-2$,$0$,$2$,$+\infty$}
\tkzTabLine{,+,z,-,,-,z,+,}
\tkzTabLine{,+,,+,z,+,,+,}
\tkzTabLine{,+,z,-,d,-,z,+,}

\tkzTabVar{-/$-\infty$,+/$-4$,-D+/$-\infty$/$+\infty$,-/$4$,+/$+\infty$}
\end{tikzpicture}
\end{center}
\item Les bornes de l'ensemble de définition sont $-\infty,$ $0$ et $+\infty.$ Cela fait quatre limites à calculer (car il faut distinguer les limites à gauche et à droite en $0$). Pour les calculer, on écrit $f(x)=x+4\times\frac{1}{x}.$

\medskip

\begin{itemize}
\item[\textbullet] \[
\left.
    \begin{array}{ll}
        \lim\limits_{x\to -\infty}x&=-\infty \\
        \lim\limits_{x\to -\infty}\frac{1}{x}&=0
    \end{array}
\right \}\implies \lim\limits_{x\to -\infty}\left(x+4\times \frac{1}{x}\right)=\text{\og}~-\infty+4\times 0~\text{ \fg}=-\infty.
\] Cette limite ne permet de mettre aucune asymptote en évidence.
\item[\textbullet] \[
\left.
    \begin{array}{ll}
        \lim\limits_{x\to +\infty}x&=+\infty \\
        \lim\limits_{x\to +\infty}\frac{1}{x}&=0
    \end{array}
\right \}\implies \lim\limits_{x\to +\infty}\left(x+4\times \frac{1}{x}\right)=\text{\og}~+\infty+4\times 0~\text{ \fg}=+\infty.
\] Cette limite ne permet de mettre aucune asymptote en évidence.
\item[\textbullet] \[
\left.
    \begin{array}{ll}
        \lim\limits_{x\to 0,~x<0}
x&=0 \\
        \lim\limits_{x\to 0,~x<0}
\frac{1}{x}&=-\infty
    \end{array}
\right \}\implies \lim\limits_{x\to 0,~x<0}
\left(x+4\times \frac{1}{x}\right)=\text{\og}~0+4\times (-\infty)~\text{ \fg}=-\infty.
\] 

On en déduit que la droite d'équation $x=0$ est asymptote (verticale) à $\mathcal{C}.$
\item[\textbullet] \[
\left.
    \begin{array}{ll}
        \lim\limits_{x\to 0,~x>0}
x&=0 \\
        \lim\limits_{x\to 0,~x>0}
\frac{1}{x}&=+\infty
    \end{array}
\right \}\implies \lim\limits_{x\to 0,~x>0}
\left(x+4\times \frac{1}{x}\right)=\text{\og}~0+4\times (+\infty)~\text{ \fg}=+\infty.
\] 

On en déduit (pour la 2\up{e} fois) que la droite d'équation $x=0$ est asymptote (verticale) à $\mathcal{C}.$
\end{itemize}
\item Avant de tracer la courbe, on construit la droite d'équation $x=0$ (axe des ordonnées, en pointillés rouge)~; puis on fait un petit tableau de valeurs. Il semble judicieux de choisir 1 carreau ou 1 cm pour 1 en abscisse, et 1 cm ou 1 carreau pour 2 en ordonnée.


\begin{center}
\psset{xunit=1.0cm,yunit=0.5cm,algebraic=true,dimen=middle,dotstyle=o,dotsize=5pt 0,linewidth=2.pt,arrowsize=3pt 2,arrowinset=0.25}
\begin{pspicture*}(-5.2294168176799944,-9.843110727720001)(5.41057854832,9.994829350280005)
\multips(0,-10)(0,2.0){10}{\psline[linestyle=dashed,linecap=1,dash=1.5pt 1.5pt,linewidth=0.4pt,linecolor=lightgray]{c-c}(-5.2294168176799944,0)(5.41057854832,0)}
\multips(-5,0)(1.0,0){11}{\psline[linestyle=dashed,linecap=1,dash=1.5pt 1.5pt,linewidth=0.4pt,linecolor=lightgray]{c-c}(0,-9.843110727720001)(0,9.994829350280005)}
\psaxes[labelFontSize=\scriptstyle,xAxis=true,yAxis=true,Dx=1.,Dy=2.,ticksize=-2pt 0,subticks=2]{->}(0,0)(-5.2294168176799944,-9.843110727720001)(5.41057854832,9.994829350280005)
\psline[linewidth=2pt,linestyle=dashed,dash=5pt 5pt,linecolor=red](0.,-9.843110727720001)(0.,9.994829350280005)
\rput[tl](0.226991062320003,-2.7108061417199987){\red{$x=0$}}
\psplot[linewidth=2pt,linestyle=dashed,dash=5pt 5pt,linecolor=green]{-5.2294168176799944}{5.41057854832}{(-0.--1.*x)/1.}
\rput[tl](3.5203229613200016,3.213293842280003){\green{$y=x$}}
\psplot[linewidth=2.pt,linecolor=blue,plotpoints=200]{-5.2294168176799944}{-0.01}{x+4.0/x}
\psplot[linewidth=2.pt,linecolor=blue,plotpoints=200]{0.01}{5.41}{x+4.0/x}
\rput[tl](2.117246649320002,6.5){\blue{$y=x+\frac{4}{x}$}}
\end{pspicture*}
\end{center}

\medskip

\textbf{Remarque~:} La droite d'équation $y=x$ (tracée en pointillés verts) est asymptote en $+\infty$ et en $-\infty.$ Mais c'est une asymptote \og oblique \fg , et l'étude de ces asymptotes n'est pas au programme.\end{enumerate}


\end{exo}



\begin{exo}

La fonction $g$ est définie sur $\mathbb{R}$ par $g(x)=\frac{\text{e}^{x}-1}{\text{e}^{x}+1}.$ On note $\mathcal{C}$ sa courbe représentative.

\begin{enumerate}
\item On utilise la formule pour la dérivée d'un quotient avec

\begin{align*}
&u(x)=\text{e}^{x}-1&&,&& v(x)=\text{e}^{x}+1, \\
& u'(x)=\text{e}^{x}&&, &&v'(x)=\text{e}^{x}.\\
\end{align*}

On obtient, pour tout $x\in \mathbb{R}~:$
\[g'(x)=\frac{\text{e}^{x}\times\left(\text{e}^{x}+1\right)-(\text{e}^{x}-1)\times \text{e}^{x}}{\left(\text{e}^{x}+1\right)^2}=\frac{\text{e}^{x}\times \text{e}^{x}+\text{e}^{x}\times 1-\text{e}^{x}\times \text{e}^{x}+1\times \text{e}^{x}}{\left(\text{e}^{x}+1\right)^2}=\frac{\cancel{\text{e}^{2x}}+\text{e}^{x}-\cancel{\text{e}^{2x}}+\text{e}^{x}}{\left(\text{e}^{x}+1\right)^2}=\frac{2\text{e}^{x}}{\left(\text{e}^{x}+1\right)^2}.\]

\item Une exponentielle est strictement positive, donc on obtient le tableau~:

\medskip
\begin{center}
\begin{tikzpicture}[scale=1]
\tkzTabInit{$x$/1,$g'(x)$/1,$g(x)$/2}{$-\infty$,$+\infty$}
\tkzTabLine{,+,}
\tkzTabVar{-/$-1$,+/$1$}
\end{tikzpicture}
\end{center}
\item $\lim\limits_{x\to -\infty}\text{e}^{x}= 0,$ donc 
\[
\left.
    \begin{array}{ll}
        \lim\limits_{x\to -\infty}\left(\text{e}^{x}-1\right)&= 0-1=-1 \\
        \lim\limits_{x\to -\infty}\left(\text{e}^{x}+1\right)&= 0+1=1
    \end{array}
\right \}\implies \lim\limits_{x\to -\infty}\frac{\text{e}^{x}-1}{\text{e}^{x}+1}=\frac{0-1}{0+1}=-1.
\]
On en déduit que la droite d'équation $y=-1$ est asymptote à $\mathcal{C}$ en $-\infty.$
\item On multiplie le numérateur et le dénominateur par $\text{e}^{-x}.$ Pour tout réel $x~:$
\[
\frac{\text{e}^{x}-1}{\text{e}^x+1}
=\frac{\left(\text{e}^{x}-1\right)\times\text{e}^{-x}}{\left(\text{e}^{x}+1\right)\times\text{e}^{-x}}
=\frac{\text{e}^{x-x}-\text{e}^{-x}}{\text{e}^{x-x}+\text{e}^{-x}}
=\frac{\text{e}^{0}-\text{e}^{-x}}{\text{e}^{0}+\text{e}^{-x}}
=\frac{1-\text{e}^{-x}}{1+\text{e}^{-x}}.
\]
On peut alors calculer la limite~: $\lim\limits_{x\to +\infty}\text{e}^{-x}= 0,$ donc 
\[
\left.
    \begin{array}{ll}
        \lim\limits_{x\to +\infty}\left(1-\text{e}^{-x}\right)&= 1-0=1 \\
        \lim\limits_{x\to +\infty}\left(1+\text{e}^{-x}\right)&= 1+0=1
    \end{array}
\right \}\implies \lim\limits_{x\to +\infty}g(x)=\lim\limits_{x\to +\infty}\frac{1-\text{e}^{-x}}{1+\text{e}^{-x}}=\frac{1-0}{1+0}=1.
\]
On en déduit que la droite d'équation $y=1$ est asymptote à $\mathcal{C}$ en $+\infty.$

\item L'équation de $T$ est \[y=g'(0)(x-0)+g(0).\]

Or $g(0)=\frac{\text{e}^{0}-1}{\text{e}^{0}+1}=\frac{1-1}{1+1}=\frac{0}{2}=0$ et $g'(0)=\frac{2\text{e}^{0}}{\left(\text{e}^{0}+1\right)^2}=\frac{2\times 1}{\left(1+1\right)^2}=\frac{2}{4}=0,5,$ donc 
\begin{align*}
T:y&=g'(0)(x-0)+g(0)\\
T:y&=0,5(x-0)+0\\
T:y&=0,5x.\end{align*}
\item Avant de construire $\mathcal{C},$ on trace les deux asymptotes et la tangente $T.$


\begin{center}
\psset{xunit=1.5cm,yunit=1.5cm,algebraic=true,dimen=middle,dotstyle=o,dotsize=5pt 0,linewidth=2.pt,arrowsize=3pt 2,arrowinset=0.25}
\begin{pspicture*}(-4.750325797418209,-1.805256471552488)(4.866503654121987,1.5961505361655612)
\multips(0,-1)(0,1.0){4}{\psline[linestyle=dashed,linecap=1,dash=1.5pt 1.5pt,linewidth=0.4pt,linecolor=lightgray]{c-c}(-4.750325797418209,0)(4.866503654121987,0)}
\multips(-4,0)(1.0,0){10}{\psline[linestyle=dashed,linecap=1,dash=1.5pt 1.5pt,linewidth=0.4pt,linecolor=lightgray]{c-c}(0,-1.805256471552488)(0,1.5961505361655612)}
\psaxes[labelFontSize=\scriptstyle,xAxis=true,yAxis=true,Dx=1.,Dy=1.,ticksize=-2pt 0,subticks=2]{->}(0,0)(-4.750325797418209,-1.805256471552488)(4.866503654121987,1.5961505361655612)
\psplot[linewidth=2.pt,linestyle=dashed,dash=2pt 2pt,linecolor=green]{-4.750325797418209}{4.866503654121987}{(--1.-0.*x)/1.}
\psplot[linewidth=2.pt,linestyle=dashed,dash=2pt 2pt,linecolor=green]{-4.750325797418209}{4.866503654121987}{(-1.-0.*x)/1.}
\psplot[linewidth=2.pt,linestyle=dashed,dash=2pt 2pt,linecolor=red]{-4.750325797418209}{4.866503654121987}{(-0.--0.5*x)/1.}
\psplot[linewidth=2.pt,linecolor=blue,plotpoints=200]{-4.750325797418209}{4.866503654121987}{(EXP(x)-1.0)/(EXP(x)+1.0)}
\rput[tl](1.5,1.3502656922341358){\red{$y=0,5x$}}
\rput[tl](3.8829642783962846,1.3502656922341358){\green{$y=1$}}
\rput[tl](-4.203915033126152,-1.1632238235093222){\green{$y=-1$}}
\rput[tl](3,0.8){\blue{$y=f(x)$}}
\end{pspicture*}
\end{center}
\end{enumerate}

\end{exo}



\begin{exo}

\begin{enumerate}
\item On met le terme de plus haut degré, $x^2,$ en facteur~:
\[x^2-5x+3=x^2\left(1-\frac{5x}{x^2}+\frac{3}{x^2}\right)=x^2\left(1-\frac{5}{x}+\frac{3}{x^2}\right).\]
On a donc

\[
\left.
    \begin{array}{ll}
        \lim\limits_{x\to +\infty}x^2&=  +\infty\\
        \lim\limits_{x\to +\infty}\left(1-\frac{5}{x}+\frac{3}{x^2}\right) &= 1-5\times 0+3\times 0=1
    \end{array}
\right \}\implies \lim\limits_{x\to +\infty}\left(x^2-5x+3\right)=\lim\limits_{x\to +\infty}x^2\left(1-\frac{5}{x}+\frac{3}{x^2}\right)= \text{\og}~(+\infty)\times 1~\text{ \fg}=+\infty.
\]

\item On met le terme de plus haut degré, $x^3,$ en facteur~:
\[x^3-2x=x^3\left(1-\frac{2x}{x^3}\right)=x^3\left(1-\frac{2}{x^2}\right).\]
On a donc

\[
\left.
    \begin{array}{ll}
        \lim\limits_{x\to -\infty}x^3&=  -\infty\\
        \lim\limits_{x\to -\infty}\left(1-\frac{2}{x^2}\right) &= 1-2\times 0=1
    \end{array}
\right \}\implies \lim\limits_{x\to -\infty}\left(x^3-2x\right)=\lim\limits_{x\to -\infty}x^3\left(1-\frac{2}{x^2}\right)= \text{\og}~-\infty\times 1~\text{ \fg}=-\infty.
\]
\item On met $x$ en facteur au numérateur et au dénominateur~:
\[\frac{6x-1}{2x+1}=\frac{\cancel{x}\left(6-\frac{1}{x}\right)}{\cancel{x}\left(2+\frac{1}{x}\right)}=\frac{6-\frac{1}{x}}{2+\frac{1}{x}}.\]

On a donc

\[
\left.
    \begin{array}{ll}
        \lim\limits_{x\to +\infty}\left(6-\frac{1}{x}\right)&= 6-0=6\\
        \lim\limits_{x\to +\infty}\left(2+\frac{1}{x}\right) &= 2+0=2
    \end{array}
\right \}\implies \lim\limits_{x\to +\infty}\frac{6x-1}{2x+1}=\lim\limits_{x\to +\infty}\frac{6-\frac{1}{x}}{2+\frac{1}{x}}= \frac{6}{2}=3.
\]

\end{enumerate}


\end{exo}



\begin{exo}

~{}


\begin{enumerate}
\setItemnumber{1}
\item \item ~{}

\begin{center}
\newcommand*{\z}{\colorbox{green}{$0$}}
\begin{tikzpicture}[scale=0.8]
\tkzTabInit{$x$/1,$x-1$/1}{$-\infty$,$\textcolor{orange}{\blacktriangleright}~ 1~\textcolor{blue!50!black}{\blacktriangleleft}$,$+\infty$}
\draw[fill=red!50,opacity=0.4](N21) rectangle (N32);
\draw[fill=yellow!50,opacity=0.4](N11) rectangle (N22);
\tkzTabLine{,-,\z, +,}
\end{tikzpicture}
\end{center}

\setlength{\columnseprule}{1pt}
\begin{multicols}{2}


Quand $x$ se rapproche de $1$ en étant supérieur à $1$, donc par la droite (flèche \textcolor{blue!50!black}{$\blacktriangleleft$}), $x-1$ se rapproche de $\colorbox{green}{0}$ en étant \colorbox{red!50}{positif}, donc

\[\lim\limits_{x\to 1,x>1}\frac{1}{x-1}=\text{\og}~\frac{1}{\colorbox{green}{0}^{\colorbox{red!50}{+}}}~\text{ \fg}=+\infty.\]


%\begin{center}
%\newcommand*{\z}{\colorbox{green}{$0$}}
%\begin{tikzpicture}[scale=0.8]
%\tkzTabInit{$x$/1,$x-1$/1}{$-\infty$,$\textcolor{orange}{\blacktriangleright}~ 1~\textcolor{white}{\blacktriangleleft}$,$+\infty$}
%\draw[fill=red!50,opacity=0.4](N11) rectangle (N22);
%\tkzTabLine{,-,\z, +,}
%\end{tikzpicture}
%\end{center}


%\medskip

\columnbreak


Quand $x$ se rapproche de $1$ en étant inférieur à $1$, donc par la gauche (flèche \textcolor{orange}{$\blacktriangleright$}), $x-1$ se rapproche de $\colorbox{green}{0}$ en étant \colorbox{yellow!50}{négatif}, donc

\[\lim\limits_{x\to 1,x<1}\frac{1}{x-1}=\text{\og}~\frac{1}{\colorbox{green}{0}^{\colorbox{yellow!50}{-}}}~\text{ \fg}=-\infty.\]
\end{multicols}
\end{enumerate}

\begin{enumerate}
\setItemnumber{3}
\item \item ~{}

\begin{center}
\newcommand*{\z}{\colorbox{green}{$0$}}
\begin{tikzpicture}[scale=0.8]
\tkzTabInit{$x$/1,$-x+4$/1}{$-\infty$,$\textcolor{orange}{\blacktriangleright}~ 1~\textcolor{blue!50!black}{\blacktriangleleft}$,$+\infty$}
\draw[fill=yellow!50,opacity=0.4](N21) rectangle (N32);
\draw[fill=red!50,opacity=0.4](N11) rectangle (N22);
\tkzTabLine{,+,\z, -,}
\end{tikzpicture}
\end{center}

\setlength{\columnseprule}{1pt}
\begin{multicols}{2}


Quand $x$ se rapproche de $4$ en étant supérieur à $4$, donc par la droite (flèche \textcolor{blue!50!black}{$\blacktriangleleft$}), $-x+4$ se rapproche de $\colorbox{green}{0}$ en étant \colorbox{yellow!50}{négatif}, donc

\[\lim\limits_{x\to 4,x>4}\frac{x+2}{-x+4}=\text{\og}~\frac{4+2}{\colorbox{green}{0}^{\colorbox{yellow!50}{-}}}~\text{ \fg}=\text{\og}~\frac{6}{0^-}~\text{ \fg}=-\infty.\]


%\begin{center}
%\newcommand*{\z}{\colorbox{green}{$0$}}
%\begin{tikzpicture}[scale=0.8]
%\tkzTabInit{$x$/1,$x-1$/1}{$-\infty$,$\textcolor{orange}{\blacktriangleright}~ 1~\textcolor{white}{\blacktriangleleft}$,$+\infty$}
%\draw[fill=red!50,opacity=0.4](N11) rectangle (N22);
%\tkzTabLine{,-,\z, +,}
%\end{tikzpicture}
%\end{center}


%\medskip
\columnbreak
Quand $x$ se rapproche de $4$ en étant inférieur à $4$, donc par la gauche (flèche \textcolor{orange}{$\blacktriangleright$}), $-x+4$ se rapproche de $\colorbox{green}{0}$ en étant \colorbox{red!50}{positif}, donc


\[\lim\limits_{x\to 4,x<4}\frac{x+2}{-x+4}=\text{\og}~\frac{4+2}{\colorbox{green}{0}^{\colorbox{red!50}{+}}}~\text{ \fg}=\text{\og}~\frac{6}{0^+}~\text{ \fg}=+\infty.\]

\end{multicols}
\end{enumerate}

\begin{enumerate}


\setItemnumber{5}
\item \item ~{}

\begin{itemize}
\item[\textbullet] On résout l'équation $\text{e}^{x}-\text{e}^{-x}=0$ pour construire le tableau de signe de $\text{e}^{x}-\text{e}^{-x}~:$
\[\text{e}^{x}-\text{e}^{-x}=0\iff \text{e}^{x}=\text{e}^{-x}\iff x=-x\iff x+x=0\iff 2x=0\iff x=0.\]

\item[\textbullet] Pour obtenir le signe, on remplace par des valeurs de $x~:$
\begin{itemize}
\item Par exemple, pour la case de droite, $\text{e}^{1}-\text{e}^{-1}\approx 3,09~;$ donc il faut mettre un $\oplus.$
\item Par exemple, pour la case de gauche, $\text{e}^{-1}-\text{e}^{-(-1)}\approx -3,09~;$ donc il faut mettre un $\ominus.$
\end{itemize}
\end{itemize}

\begin{center}
\newcommand*{\z}{\colorbox{green}{$0$}}
\begin{tikzpicture}[scale=0.8]
\tkzTabInit{$x$/1,$\text{e}^{x}-\text{e}^{-x}$/1}{$-\infty$,$\textcolor{orange}{\blacktriangleright}~ 0~\textcolor{blue!50!black}{\blacktriangleleft}$,$+\infty$}
\draw[fill=red!50,opacity=0.4](N21) rectangle (N32);
\draw[fill=yellow!50,opacity=0.4](N11) rectangle (N22);
\tkzTabLine{,-,\z, +,}
\end{tikzpicture}
\end{center}

\begin{align*}
\lim\limits_{x\to 0,x>0}\frac{\text{e}^{x}+\text{e}^{-x}}{\text{e}^{x}-\text{e}^{-x}}&=\text{\og}~\frac{\text{e}^{0}+\text{e}^{-0}}{\colorbox{green}{0}^{\colorbox{red!50}{+}}}~\text{ \fg}=\text{\og}~\frac{2}{0^+}~\text{ \fg}=+\infty,
\\
\lim\limits_{x\to 0,x<0}\frac{\text{e}^{x}+\text{e}^{-x}}{\text{e}^{x}-\text{e}^{-x}}&=\text{\og}~\frac{\text{e}^{0}+\text{e}^{-0}}{\colorbox{green}{0}^{\colorbox{yellow!50}{-}}}~\text{ \fg}=\text{\og}~\frac{2}{0^-}~\text{ \fg}=-\infty.
\end{align*}


\end{enumerate}


\end{exo}

\begin{exo}

La fonction $i$ est définie sur $\mathbb{R}-\left\{-\frac{1}{2}\right\}$ par $i(x)=\frac{x}{2x+1}.$ On note $\mathcal{C}$ sa courbe représentative.


\medskip
 
\textbf{Remarque~:} $-\frac{1}{2}$ est valeur interdite, car le dénominateur de la fraction $\frac{x}{2x+1}$ s'annule lorsque $x=-\frac{1}{2}~:$
\[2x+1=0\iff 2x=-1\iff x=-\frac{1}{2}.\]
\begin{enumerate}
\item

\setlength{\columnseprule}{1pt}

\begin{multicols}{2}

On utilise la formule pour la dérivée d'un quotient avec

\begin{align*}
&u(x)=x&&,&& v(x)=2x+1, \\
& u'(x)=1&&, &&v'(x)=2.\\
\end{align*}


On obtient, pour $x\in \mathbb{R}-\left\{-\frac{1}{2}\right\}~:$
\[i'(x)=\frac{1\times(2x+1)-x\times 2}{\left(2x+1\right)^2}=\frac{2x+1-2x}{\left(2x+1\right)^2}=\frac{1}{\left(2x+1\right)^2}.\]

On a donc le tableau~:

\medskip
\begin{center}
\begin{tikzpicture}[scale=0.75]
\tkzTabInit{$x$/1,$i'(x)$/1,$i(x)$/2}{$-\infty$,$-\frac{1}{2}$,$+\infty$}
\tkzTabLine{,+,d,+,}
\tkzTabVar{-/$\frac{1}{2}$,+D-/$+\infty$/$-\infty$,+/$\frac{1}{2}$}
\end{tikzpicture}
\end{center}
\end{multicols}
\item \begin{itemize}
\item[\textbullet] On calcule d'abord les limites en $\pm \infty.$

On met $x$ en facteur au numérateur et au dénominateur~:
\[\frac{x}{2x+1}=\frac{\cancel{x}\left(1\right)}{\cancel{x}\left(2+\frac{1}{x}\right)}=\frac{1}{2+\frac{1}{x}}.\]

On a donc~:


\[
\left.
    \begin{array}{ll}
        \lim\limits_{x\to +\infty}1&= 1\\
        \lim\limits_{x\to +\infty}\left(2+\frac{1}{x}\right) &= 2+0=2
    \end{array}
\right \}\implies \lim\limits_{x\to +\infty}\frac{x}{2x+1}=\lim\limits_{x\to +\infty}\frac{1}{2+\frac{1}{x}}= \frac{1}{2}.
\]

\medskip

\[
\left.
    \begin{array}{ll}
        \lim\limits_{x\to -\infty}1&= 1\\
        \lim\limits_{x\to -\infty}\left(2+\frac{1}{x}\right) &= 2+0=2
    \end{array}
\right \}\implies \lim\limits_{x\to -\infty}\frac{x}{2x+1}=\lim\limits_{x\to -\infty}\frac{1}{2+\frac{1}{x}}= \frac{1}{2}.
\]

\medskip

On en déduit que la droite d'équation $y=\frac{1}{2}$ est asymptote  à la courbe $\mathcal{C}$ en $+\infty$ et en $-\infty.$
\item[\textbullet] On calcule ensuite les limites à droite et à gauche en $0.$

On construit le tableau de signe~:

\medskip
\begin{center}
\newcommand*{\z}{\colorbox{green}{$0$}}
\begin{tikzpicture}[scale=0.8]
\tkzTabInit{$x$/1,$2x+1$/1}{$-\infty$,$\textcolor{orange}{\blacktriangleright}~ -\frac{1}{2}~\textcolor{blue!50!black}{\blacktriangleleft}$,$+\infty$}
\draw[fill=red!50,opacity=0.4](N21) rectangle (N32);
\draw[fill=yellow!50,opacity=0.4](N11) rectangle (N22);
\tkzTabLine{,-,\z, +,}
\end{tikzpicture}
\end{center}

\medskip




\[\lim\limits_{x\to -\frac{1}{2},x>-\frac{1}{2}}\frac{x}{2x+1}=\text{\og}~\frac{-\frac{1}{2}}{\colorbox{green}{0}^{\colorbox{red!50}{+}}}~\text{ \fg}=-\infty\qquad,\qquad\lim\limits_{x\to -\frac{1}{2},x<-\frac{1}{2}}\frac{x}{2x+1}=\text{\og}~\frac{-\frac{1}{2}}{\colorbox{green}{0}^{\colorbox{yellow!50}{-}}}~\text{ \fg}=+\infty.\]


\medskip

On en déduit que la droite d'équation $x=-\frac{1}{2}$ est asymptote à la courbe $\mathcal{C}.$

\end{itemize}

\item On construit les asymptotes d'équations $y=\frac{1}{2}$ (en pointillés verts), $x=-\frac{1}{2}$ (en pointillés rouges), puis la courbe $\mathcal{C}$ (un tableau de valeurs avec $x=-1$  et $x=0$ est suffisant pour avoir un bon tracé).

\medskip


\begin{center}
\psset{xunit=1.0cm,yunit=1.0cm,algebraic=true,dimen=middle,dotstyle=o,dotsize=5pt 0,linewidth=2.pt,arrowsize=3pt 2,arrowinset=0.25}
\begin{pspicture*}(-4.54,-3.5)(3.52,3.72)
\multips(0,-3)(0,1.0){8}{\psline[linestyle=dashed,linecap=1,dash=1.5pt 1.5pt,linewidth=0.4pt,linecolor=lightgray]{c-c}(-4.54,0)(3.52,0)}
\multips(-4,0)(1.0,0){9}{\psline[linestyle=dashed,linecap=1,dash=1.5pt 1.5pt,linewidth=0.4pt,linecolor=lightgray]{c-c}(0,-3.5)(0,3.72)}
\psaxes[labelFontSize=\scriptstyle,xAxis=true,yAxis=true,Dx=1.,Dy=1.,ticksize=-2pt 0,subticks=2]{->}(0,0)(-4.54,-3.5)(3.52,3.72)
\psline[linewidth=2.pt,linestyle=dashed,dash=2pt 2pt,linecolor=red](-0.5,-3.5)(-0.5,3.72)
\psplot[linewidth=2.pt,linestyle=dashed,dash=2pt 2pt,linecolor=green]{-4.54}{3.52}{(--0.5-0.*x)/1.}
\psplot[linewidth=2.pt,linecolor=blue,plotpoints=200]{-4.539999999999997}{-0.51}{x/(2.0*x+1.0)}
\psplot[linewidth=2.pt,linecolor=blue,plotpoints=200]{-0.49}{3.5}{x/(2.0*x+1.0)}
\rput[tl](1.56,1.3){\green{$y=\frac{1}{2}$}}
\rput[tl](-1.65,-1.02){\red{$x=-\frac{1}{2}$}}
\rput[tl](-3.4,1.18){\blue{$y=f(x)$}}
\end{pspicture*}
\end{center}

\end{enumerate}


\end{exo}



\begin{exo}



\begin{enumerate}
\item On calcule d'abord $\lim\limits_{x\to +\infty}\left(x^2-2x\right).$ Pour cela, on met le terme de plus haut degré, $x^2,$ en facteur~:
\[x^2-2x=x^2\left(1-\frac{2x}{x^2}\right)=x^2\left(1-\frac{2}{x}\right).\]
On a donc

\[
\left.
    \begin{array}{ll}
        \lim\limits_{x\to +\infty}x^2&=  +\infty\\
        \lim\limits_{x\to +\infty}\left(1-\frac{2}{x}\right) &= 1-2\times 0=1
    \end{array}
\right \}\implies \lim\limits_{x\to +\infty}\left(x^2-2x\right)=+\infty.
\]

On en déduit~:
\[\lim\limits_{x\to +\infty}\text{e}^{x^2-2x}=\text{\og}~\text{e}^{+\infty}~\text{\fg}=+\infty.\]
\item $\lim\limits_{x\to +\infty}\frac{1}{x}=0,$ donc $\lim\limits_{x\to +\infty}\text{e}^{1/x}=\text{e}^0=1.$

\medskip

\textbf{Remarque~:} Pour justifier rigoureusement la 2\up{e} limite, il faut invoquer \textbf{la continuité en 0} de la fonction exponentielle.
\item On factorise comme dans la question 1~: $x^2-2x=x^2\left(1-\frac{2}{x}\right).$

\[
\left.
    \begin{array}{ll}
        \lim\limits_{x\to -\infty}x^2&=  +\infty\\
        \lim\limits_{x\to -\infty}\left(1-\frac{2}{x}\right) &= 1-2\times 0=1
    \end{array}
\right \}\implies \lim\limits_{x\to -\infty}\left(x^2-2x\right)=+\infty.
\]

On en déduit~:

\[\lim\limits_{x\to -\infty}\left(x^2-2x\right)^3=\text{\og}~\left({+\infty}\right)^3~\text{\fg}=+\infty.\]
\item $\lim\limits_{x\to 0,x<0}\frac{1}{x}=-\infty,$ donc \[\lim\limits_{x\to 0,x<0}\text{e}^{1/x}=\text{\og}~\text{e}^{-\infty}~\text{\fg}=0.\]
\end{enumerate}

\end{exo}


\begin{exo}

La fonction $k$ est définie sur $\left[0;+\infty\right[$ par \[k(x)=x^2\text{e}^{-x}.\]



\begin{enumerate}
\item On utilise la formule pour la dérivée d'un produit, avec

\begin{align*}
&u(x)=x^2&&,&& v(x)=\text{e}^{-x}, \\
& u'(x)=2x&&, &&v'(x)=-\text{e}^{-x}.\\
\end{align*}


On obtient, pour tout $x\in \left[0;+\infty\right[~:$
\[k'(x)=2x\times \text{e}^{-x}+x^2\times \left(-\text{e}^{-x}\right)=\left(2x-x^2\right)\text{e}^{-x}.\]
\item $2x-x^2=x(2-x),$ donc les racines de $2x-x^2$ sont 0 et 2 et on a le tableau~:

\medskip
\begin{center}
\begin{tikzpicture}[scale=1]
\tkzTabInit{$x$/1,$2x-x^2$/1,$\text{e}^{-x}$/1,$k'(x)$/1,$k(x)$/2}{$0$,$2$,$+\infty$}
\tkzTabLine{z,+,z,-,}
\tkzTabLine{,+,,+,}
\tkzTabLine{z,+,z,-,}
\tkzTabVar{-/$0$,+/$4\text{e}^{-2}$,-/}
\end{tikzpicture}
\end{center}
\medskip

\[k(0)=0^2\text{e}^{-0}=0\qquad \qquad k(2)=2^2\text{e}^{-2}=4\text{e}^{-2}.\]

\item La fonction $k$ est positive (car un carré et une exponentielle le sont) et son maximum est $4\text{e}^{-2},$ donc pour tout $x\in\left]0;+\infty\right[~:$
\[0\leq x^2\text{e}^{-x}\leq 4\text{e}^{-2}.\] On divise par $x$ (qui est strictement positif, donc le sens des inégalités ne change pas)~:
\begin{alignat*}{3}
\frac{0}{x}&\leq \frac{x^2\text{e}^{-x}}{x}&\leq \frac{4\text{e}^{-2}}{x}\\
0&\leq x \text{e}^{-x}&\leq  \frac{4\text{e}^{-2}}{x}.
\end{alignat*}
\item On sait que $0\leq x\text{e}^{-x}\leq \frac{4\text{e}^{-2}}{x}$ pour tout $x>0.$ Or $\lim\limits_{x\to +\infty}0=0$ et $\lim\limits_{x\to +\infty}\frac{4\text{e}^{-2}}{x}=0,$ donc d'après le théorème des gendarmes~:

\[\lim\limits_{x\to +\infty}x\text{e}^{-x}=0.\]
\item \textbf{Application~:} On met $\text{e}^{-x}$ en facteur~: pour tout $x\in\left[0;+\infty\right[,$
\[\text{e}^x-x-2=\text{e}^x\left(1-\frac{x}{\text{e}^x}-\frac{2}{\text{e}^x}\right)=
\text{e}^x\left(1-x\text{e}^{-x}-2\text{e}^{-x}\right).\]
 
 Or $\lim\limits_{x\to +\infty}x\text{e}^{-x}=0$ (d'après la question précédente) et  $\lim\limits_{x\to +\infty}\text{e}^{-x}=0,$ donc~:
 
 \[
\left.
    \begin{array}{ll}
        \lim\limits_{x\to +\infty}\text{e}^x&=  +\infty\\
        \lim\limits_{x\to +\infty}\left(1-x\text{e}^{-x}-2\text{e}^{-x}\right) &= 1-0-2\times 0=1
    \end{array}
\right \}\implies \lim\limits_{x\to +\infty}\left(\text{e}^x-x-2\right)=\text{\og}~{+\infty}\times 1~\text{\fg}=+\infty.
\]
 
\end{enumerate}

\end{exo}

\begin{exo}
%Polynésie, septembre 2010
~{}

\textbf{Partie 1}

\medskip 
Soit $g$ la fonction définie sur $\left[0;+\infty\right[$ par \[g(x) = \text{e}^x - x\text{e}^x + 1.\]


\begin{enumerate}
\item On met $\text{e}^x$ en facteur~:
\[\text{e}^x - x\text{e}^x + 1=\text{e}^x(1-x) + 1.\]

\[
\left.
    \begin{array}{ll}
        \lim\limits_{x\to +\infty}\text{e}^x&=  +\infty\\
        \lim\limits_{x\to +\infty}\left(1-x\right) &= -\infty
    \end{array}
\right \}\implies \lim\limits_{x\to +\infty}\text{e}^x(1-x) + 1=\text{\og}~{+\infty}\times (-\infty)+1~\text{\fg}=-\infty.
\]

Conclusion~: $\lim\limits_{x\to +\infty}g(x)=-\infty.$

\item 



\setlength{\columnseprule}{1pt}

\begin{multicols}{2}

On dérive $x\mapsto x\text{e}^{x}$ grâce à la formule pour la dérivée d'un produit~:

\begin{align*}
&u(x)=x&&,&& v(x)=\text{e}^{x}, \\
& u'(x)=1&&, &&v'(x)=\text{e}^{x}.\\
\end{align*}

On a donc, pour tout $x\in \left[0;+\infty\right[~:$

\begin{align*}
g'(x)&=\text{e}^{x}-\left(1\times \text{e}^{x}+x\times \text{e}^{-x}\right)
\\&=\text{e}^{x}-\text{e}^{x}-x\text{e}^{x}
\\&=-x\text{e}^{x}.
\end{align*}
\columnbreak

On obtient le tableau~:

\medskip
\begin{center}
\begin{tikzpicture}[scale=0.8]
\tkzTabInit{$x$/1,$-x$/1,$\text{e}^{x}$/1,$g'(x)$/1,$g(x)$/2}{$0$,$+\infty$}
\tkzTabLine{z,-,}
\tkzTabLine{,+,}
\tkzTabLine{z,-,}
\tkzTabVar{+/$2$,-/$-\infty$}
\end{tikzpicture}
\end{center}
\medskip

\[g(0)=\text{e}^0 - 0\times \text{e}^0 + 1=2.\]

\end{multicols}
\item 
	\begin{enumerate}
		\item \begin{itemize}
\item[\textbullet] La fonction $g$ est continue et strictement décroissante sur $\left[0;+\infty\right[~;$
\item[\textbullet] $g(0)=2,$ $\lim\limits_{x\to +\infty}g(x)=-\infty~;$
\item[\textbullet] $0\in\left]-\infty;2\right].$
\end{itemize}

D'après le théorème de la bijection, l'équation $g(x)=0$ a exactement une solution  $\alpha$ dans $\left[0;+\infty\right[.$
		
		
		\item On obtient grâce à la calculatrice~:
		\[1,27\leq \alpha\leq 1,28.\] 
		%\item On sait que $g(\alpha)=0,$ soit $ \text{e}^{\alpha} - \alpha\text{e}^{\alpha} + 1=0.$
		
		%On isole $\text{e}^{\alpha}$ dans cette égalité~:
		%\[\text{e}^{\alpha} - \alpha\text{e}^{\alpha} + 1=0\implies \text{e}^{\alpha}(1-\alpha) =-1
		%\implies \text{e}^{\alpha} = \frac{-1\textcolor{red}{\times (-1)}}{(1-\alpha)\textcolor{red}{\times (-1)}}\implies \text{e}^{\alpha} = \frac{1}{\alpha-1}.\]
	\end{enumerate} 
\item \setlength{\columnseprule}{1pt}

\begin{multicols}{2} 

On complète le tableau de variations de $g$ en faisant apparaître $\alpha~:$

\medskip
\begin{center}
\begin{tikzpicture}[scale=1]
\tkzTabInit{$x$/1,$g(x)$/2}{$0$,$+\infty$}
\tkzTabVar{+/$2$,-/$-\infty$}
\tkzTabVal[draw]{1}{2}{0.4}{$\alpha$}{$0$}
\end{tikzpicture}
\end{center}
\medskip

On en déduit le tableau de signe~:

\medskip
\begin{center}
\begin{tikzpicture}[scale=0.8]
\tkzTabInit{$x$/1,$g(x)$/1}{$0$,$\alpha$,$+\infty$}
\tkzTabLine{,+,z,-,}
\end{tikzpicture}
\end{center}

\end{multicols}
\end{enumerate}

\medskip
 
\textbf{Partie 2}

\medskip
 
Soit $A$ la fonction définie sur $\left[0;+\infty\right[$ par \[A(x) = \frac{4x}{\text{e}^x + 1}.\]

\medskip

\setlength{\columnseprule}{1pt}

\begin{multicols}{2}

On utilise la formule pour la dérivée d'un quotient~:

\begin{align*}
&u(x)=4x&&,&& v(x)=\text{e}^x + 1, \\
& u'(x)=4&&, &&v'(x)=\text{e}^{x}.\\
\end{align*}

On a donc, pour tout $x\in \left[0;+\infty\right[~:$

\begin{align*}
A'(x)&=\frac{4\times \left(\text{e}^x + 1\right)-4x\times \text{e}^x }{\left(\text{e}^x + 1\right)^2}
\\&=\frac{4\text{e}^x+4-4x\text{e}^x}{\left(\text{e}^x + 1\right)^2}
\\&=\frac{4\left(\text{e}^x+1-x\text{e}^x\right)}{\left(\text{e}^x + 1\right)^2}
\\&=g(x)\times\frac{4}{\left(\text{e}^x + 1\right)^2}.
\end{align*}

Or $\frac{4 }{\left(\text{e}^x + 1\right)^2}$ est strictement positif sur $\left[0;+\infty\right[,$ donc $A'(x)$ a le même signe que $g(x).$ On a donc, grâce à la question 4 de la partie 1~:

\medskip
\begin{center}
\begin{tikzpicture}[scale=0.8]
\tkzTabInit{$x$/1,$A'(x)$/1,$A(x)$/2}{$0$,$\alpha$,$+\infty$}
\tkzTabLine{,+,z,-,}
\tkzTabVar{-/,+/,-/}
\end{tikzpicture}
\end{center}
\end{multicols}
%\medskip

\newpage
 
\textbf{Partie 3}

\medskip
 
On considère la fonction $f$ définie sur $\left[0;+\infty\right[$ par \[f(x) = \frac{4}{\text{e}^x + 1}.\]


\medskip
 
\begin{enumerate}
\item 

\begin{multicols}{2}

L'aire du rectangle $OPMQ$ est 
\begin{align*}OP\times OQ&=x\times f(x)
\\&=x\times \frac{4}{\text{e}^x + 1}
\\&= \frac{4x}{\text{e}^x + 1}
\\&=A(x).
\end{align*}

Or $A$ atteint son maximum pour $x=\alpha$ (cf partie 2), donc l'aire  de $OPMQ$ est maximale lorsque $M$ a pour abscisse $\alpha$. 

\columnbreak

\begin{center}
\psset{xunit=2.0cm,yunit=2.0cm,algebraic=true,dimen=middle,dotstyle=o,dotsize=5pt 0,linewidth=2.pt,arrowsize=3pt 2,arrowinset=0.25}
\begin{pspicture*}(-0.814109783146838,-0.35450297114661766)(3.3768323351326335,2.2)
\pscustom[linewidth=0.8pt,linecolor=blue,fillcolor=blue!30!white,fillstyle=solid,opacity=0.1]{\psplot{0.}{1.3}{0.8566600678297656}\lineto(1.3,0)\lineto(0.,0)\closepath}
\psaxes[labelFontSize=\scriptstyle,xAxis=true,yAxis=true,Dx=1,Dy=1,ticksize=-2pt 0,subticks=2]{->}(0,0)(-0.814109783146838,-0.35450297114661766)(3.3768323351326335,2.2)
\psplot[linewidth=2.pt,linecolor=red,plotpoints=200]{-0.814109783146838}{3.3768323351326335}{4.0/(EXP(x)+1.0)}
\rput[tl](2.2126817467216693,0.7902451074574979){$\red{(\mathcal{C})}$}
\rput[tl](1.3261411975381024,-0.09256908875415056){\blue{$x$}}
\rput[tl](-0.47456586152697344,0.9357639310088685){\blue{$f(x)$}}
%\begin{scriptsize}
\psdots[dotsize=3pt 0,dotstyle=*,linecolor=blue](0.,0.)
\rput[bl](0.04,0.05){\blue{$O$}}
\psdots[dotsize=3pt 0,dotstyle=*,linecolor=blue](1.3,0.)
\rput[bl](1.33,0.05){\blue{$P$}}
\psdots[dotsize=3pt 0,dotstyle=*,linecolor=blue](1.3,0.8566600678297656)
\rput[bl](1.33,0.9){\blue{$M$}}
\psdots[dotsize=3pt 0,dotstyle=*,linecolor=blue](0.,0.8566600678297656)
\rput[bl](0.04,0.9){\blue{$Q$}}
%\end{scriptsize}
\end{pspicture*}
\end{center}

\end{multicols}
\item 

\begin{multicols}{2}

On suppose dans cette question que le point $M$ a pour abscisse $\alpha$.

\medskip

On rappelle que deux droites (non verticales) sont parallèles si, et seulement si, elles ont le même coefficient directeur. On rappelle également que la tangente à la courbe d'une fonction $f$ au point d'abscisse $a$ a pour coefficient directeur $f'(a).$ Ceci étant dit~:


\begin{itemize}
\item[\textbullet] La tangente $(T)$ en $M$ à la courbe $(\mathcal{C})$ a pour coefficient
$f'(\alpha).$ Or on obtient facilement (par exemple avec la dérivée d'un quotient)
\[f'(x)=-\frac{4\text{e}^x}{\left(\text{e}^x+1\right)^2},\]
donc
\[f'(\alpha)=-\frac{4\text{e}^\alpha}{\left(\text{e}^{\alpha}+1\right)^2}.\]
%On a vu dans la question 3.(c) de la partie 1 que $\text{e}^{\alpha}=\frac{1}{\alpha -1}~;$ il s'ensuit que
%\begin{align*}
%f'(\alpha)&=-\frac{4\times \frac{1}{\alpha -1}}{\left(\frac{1}{\alpha -1}+1\right)^2}
%=-\frac{\frac{4}{\alpha -1}}{\left(\frac{1}{\alpha -1}+\frac{\alpha -1}{\alpha -1}\right)^2}
%=-\frac{\frac{4}{\alpha -1}}{\left(\frac{\alpha}{\alpha -1}\right)^2}
%\\&=-\frac{\frac{4}{\alpha -1}}{\frac{\alpha^2}{(\alpha -1)^2}}
%=-\frac{4}{\alpha -1}\times \frac{(\alpha -1)^2}{\alpha^2}
%=-\frac{4(\alpha-1)}{\alpha^2}.
%\end{align*}
\item[\textbullet] La droite $(PQ)$ a pour coefficient directeur
\[
c_{(PQ)}=\frac{y_P-y_Q}{x_P-x_Q}=\frac{0-f(\alpha)}{\alpha-0}=\frac{-\frac{4}{\text{e}^{\alpha}+1}}{\alpha}=-\frac{4}{\alpha\left(\text{e}^{\alpha}+1\right)}.
\]
%donc en utilisant à nouveau $\text{e}^{\alpha}=\frac{1}{\alpha -1}~:$
%\[
%c_{(PQ)}=\frac{-\frac{4}{\frac{1}{\alpha -1}+1}}{\alpha}=\frac{-\frac{4}{\frac{\alpha}{\alpha -1}}}{\alpha}=\frac{\frac{-4(\alpha-1)}{\alpha}}{\alpha}=-\frac{4(\alpha-1)}{\alpha^2}.\]
\end{itemize}

\columnbreak

Pour prouver l'égalité $f'(\alpha)=c_{(PQ)},$ on réduit au même dénominateur et on met $-4$ en facteur~:
\begin{align*}
f'(\alpha)&=-\frac{4\text{e}^\alpha}{\left(\text{e}^{\alpha}+1\right)^2}=-4\times \frac{\alpha\text{e}^\alpha}{\alpha\left(\text{e}^{\alpha}+1\right)^2}
\\ c_{(PQ)}&=-\frac{4}{\alpha\left(\text{e}^{\alpha}+1\right)}=-\frac{4\textcolor{red}{\times\left(\text{e}^{\alpha}+1\right)}}{\alpha\left(\text{e}^{\alpha}+1\right)\textcolor{red}{\times\left(\text{e}^{\alpha}+1\right)}}=-4\times \frac{\text{e}^\alpha+1}{\alpha\left(\text{e}^{\alpha}+1\right)^2}
\end{align*}

On sait que $g(\alpha)=0,$ donc $\text{e}^{\alpha}-\alpha\text{e}^{\alpha}+1=0~;$ et donc
\[\alpha\text{e}^{\alpha}=\text{e}^{\alpha}+1.\] Il s'ensuit que $f'(\alpha)=c_{(PQ)},$ et donc que la tangente $(T)$ et la droite $(PQ)$ ont le même coefficient directeur. Elles sont donc bien parallèles.


\begin{center}
\psset{xunit=2.0cm,yunit=2.0cm,algebraic=true,dimen=middle,dotstyle=o,dotsize=5pt 0,linewidth=2.pt,arrowsize=3pt 2,arrowinset=0.25}
\begin{pspicture*}(-0.814109783146838,-0.25749042211237005)(3.5805586881045524,1.8767856566410657)
\pscustom[linewidth=0.8pt,linecolor=blue,fillcolor=blue!30!white,fillstyle=solid,opacity=0.1]{\psplot{0.}{1.3}{0.8566600678297656}\lineto(1.3,0)\lineto(0.,0)\closepath}
\psaxes[labelFontSize=\scriptstyle,xAxis=true,yAxis=true,Dx=1,Dy=1,ticksize=-2pt 0,subticks=2]{->}(0,0)(-0.814109783146838,-0.25749042211237005)(3.5805586881045524,1.8767856566410657)
\psplot[linewidth=2.pt,linecolor=red,plotpoints=200]{-0.814109783146838}{3.5805586881045524}{4.0/(EXP(x)+1.0)}
\rput[tl](2.2126817467216693,0.7902451074574984){$\red{(\mathcal{C})}$}
\rput[tl](1.174647472055226,-0.08205349950749474){\blue{$\alpha$}}
\rput[tl](-0.43576084191327463,0.92){\blue{$f(\alpha)$}}
\rput[tl](1.8,0.35){\magenta{$(T)$}}
\psplot[linewidth=2.pt,linecolor=magenta]{-0.814109783146838}{3.5805586881045524}{(--1.1136580881786953-0.8566600678297656*x)/1.3}
\psplot[linewidth=2.pt,linecolor=magenta]{-0.814109783146838}{3.5805586881045524}{(--2.2273161763573905-0.8566600678297656*x)/1.3}
%\begin{scriptsize}
\psdots[dotsize=3pt 0,dotstyle=*,linecolor=blue](0.,0.)
\rput[bl](0.04,0.05){\blue{$O$}}
\psdots[dotsize=3pt 0,dotstyle=*,linecolor=blue](1.3,0.)
\rput[bl](1.33,0.05){\blue{$P$}}
\psdots[dotsize=3pt 0,dotstyle=*,linecolor=blue](1.3,0.8566600678297656)
\rput[bl](1.33,0.9){\blue{$M$}}
\psdots[dotsize=3pt 0,dotstyle=*,linecolor=blue](0.,0.8566600678297656)
\rput[bl](0.04,0.9){\blue{$Q$}}
%\end{scriptsize}
\end{pspicture*}
\end{center}

\medskip

%\textbf{Remarque~:} Pour établir l'égalité $f'(\alpha)=c_{(PQ)},$ il était plus rapide (mais moins naturel) de %montrer que $f'(\alpha)-c_{(PQ)}=0.$
 \end{multicols}
\end{enumerate}

\end{exo}

\newpage

\begin{exo}



La fonction $g$ est définie sur $\left]0;+\infty\right[$ par
\[g(x)=\frac{1}{2}x+\frac{1}{x}.\]


\begin{enumerate}
\item Pour tout $x\in \left]0;+\infty\right[~:$
\[g'(x)=\frac{1}{2}+\left(-\frac{1}{x^2}\right)=\frac{1\textcolor{red}{\times x^2}}{2\textcolor{red}{\times x^2}}-\frac{1\textcolor{red}{\times 2}}{x^2\textcolor{red}{\times 2}}=\frac{x^2}{2x^2}-\frac{2}{2x^2}=\frac{x^2-2}{2x^2}.\]
\item On étudie le signe de $\frac{x^2-2}{2x^2}~:$

\begin{itemize}
\item[\textbullet] Les racines de $x^2-2$ sont $\sqrt{2}$ et $-\sqrt{2}~;$ seule la première est dans $\left[0;+\infty\right[.$
\item[\textbullet] Sur $\mathbb{R}$ tout entier, le signe serait de la forme \fbox{$+~\upphi~-~\upphi~+$}, avec un $\upphi$ au niveau des racines~; mais comme on travaille dans $\left[0;+\infty\right[$ seulement,  le signe est de la forme \fbox{$-~\upphi~+$}.
%\item[\textbullet] $x^2$ est toujours $\ominus,$ et il s'annule pour $x=0.$
\end{itemize}

\medskip

On obtient finalement~:

\begin{center}
\begin{tikzpicture}[scale=0.8]
\tkzTabInit{$x$/1,$x^2-2$/1,$x^2$/1,$g'(x)$/1,$g(x)$/2}{$0$,$\sqrt{2}$,$+\infty$}
\tkzTabLine{,-,z,+,}
\tkzTabLine{z,+,,+,}
\tkzTabLine{d,-,z,+,}
\tkzTabVar{D+/$+\infty$,-/$\sqrt{2}$,+/$+\infty$}
\end{tikzpicture}
\end{center}

\medskip

\[f\left(\sqrt{2}\right)=\frac{1\textcolor{red}{\times\sqrt{2}}}{\sqrt{2}\textcolor{red}{\times\sqrt{2}}}+\frac{\sqrt{2}}{2}=\frac{\sqrt{2}}{2}+\frac{\sqrt{2}}{2}=\frac{\cancel{2}\sqrt{2}}{\cancel{2}}=\sqrt{2}.\]

\item
\[
\left.
    \begin{array}{ll}
        \lim\limits_{x\to +\infty}\frac{1}{2}x&=  +\infty\\
        \lim\limits_{x\to +\infty}\frac{1}{x} &= 0
    \end{array}
\right \}\implies \lim\limits_{x\to +\infty}g(x)=+\infty.
\hspace*{2cm}
\left.
    \begin{array}{ll}
        \lim\limits_{x\to 0,~x>0}\frac{1}{2}x&=  0\\
        \lim\limits_{x\to 0,~x>0}\frac{1}{x} &= +\infty
    \end{array}
\right \}\implies \lim\limits_{x\to 0,~x>0}g(x)=+\infty.
\]

%Conclusion~: $\lim\limits_{x\to 0,~x>0}g(x)=+\infty.$


%\medskip
%On définit une suite $\left(a_n\right)_{n\in\mathbb{N}}$ par $a_0=2$ et la relation de récurrence

%\[a_{n+1}=g\left(a_n\right)\] pour tout $n\in\mathbb{N}.$

\item Pour tout $n\in\mathbb{N},$ on note $\mathcal{P}_n$ la propriété \[\sqrt{2}\leq a_n\leq 2.\]

\begin{itemize}
\item[{\textbullet}] \textbf{Initialisation.} On prouve que $\mathcal{P}_0$ est vraie.

\[a_0=2\implies \sqrt{2}\leq a_0\leq 2 \implies\mathcal{P}_0~\text{est vraie}.\]




\item[{\textbullet}] \textbf{Hérédité.}

\setlength{\columnseprule}{1pt}

\begin{multicols}{2}

Soit $k\in\mathbb{N}$ tel que $\mathcal{P}_k$ soit vraie. On a donc
\[\sqrt{2}\leq a_k\leq 2.\]%\medskip







La fonction $g$ est strictement croissante sur $\left[\sqrt{2};+\infty\right[,$ donc
\begin{align*}g\left(\sqrt{2}\right)&\leq g\left(a_k\right)\leq g(2)
\\ \sqrt{2}&\leq a_{k+1}\leq \frac{3}{2}.
\end{align*}

(cf illustration avec le tableau de variations ci-contre).

\medskip

Or $\frac{3}{2}\leq 2,$ donc

\[\sqrt{2}\leq a_{k+1}\leq 2,\]

c'est-à-dire que la propriété $\mathcal{P}_{k+1}$ est  vraie.

\columnbreak

\vspace*{1cm}
\begin{center}
\begin{tikzpicture}[scale=1]
\tkzTabInit{$x$/1,$g(x)$/2}{$0$,$\sqrt{2}$,$+\infty$}
\tkzTabVar{D+/$+\infty$,-/$\sqrt{2}$,+/$+\infty$}
\tkzTabVal[draw]{2}{3}{0.33}{$a_k$}{$a_{k+1}$}
\tkzTabVal[draw]{2}{3}{0.66}{$2$}{$\frac{3}{2}$}
\end{tikzpicture}
\end{center}


\medskip

\textbf{Remarque~:} $g\left(a_k\right)=a_{k+1}.$

\end{multicols}




\item[{\textbullet}] \textbf{Conclusion.} $\mathcal{P}_0$ est vraie et $\mathcal{P}_n$ est héréditaire, donc elle est vraie pour tout $n\in\mathbb{N}.$
\end{itemize}
\item Pour tout $n\in\mathbb{N},$
\[a_{n+1}-a_n=\left(\frac{1}{2}a_n+\frac{1}{a_n}\right)-a_n=\frac{1}{a_n}-\frac{a_n}{2}=
\frac{1\textcolor{red}{\times 2}}{a_n\textcolor{red}{\times 2}}-\frac{a_n\textcolor{red}{\times a_n}}{2\textcolor{red}{\times a_n}}=\frac{2-a_n^2}{2a_n}.\]

Or $\sqrt{2}\leq a_n$ d'après la question précédente, donc $\sqrt{2}^2\leq a_n^2$ (deux nombres positifs sont rangés dans le même ordre que leurs carrés). On a donc $2\leq a_n^2~;$ et ainsi
\[\frac{\overbrace{2-a_n^2}^{\ominus}}{\underbrace{2a_n}_{\oplus}}\leq 0.\]

Conclusion~: $a_{n+1}-a_n\leq 0,$ donc $\left(a_n\right)_{n\in\mathbb{N}}$ est décroissante.
\item La suite $\left(a_n\right)_{n\in\mathbb{N}}$ est~:

\begin{itemize}
\item[{\textbullet}] décroissante d'après la question précédente~;
\item[{\textbullet}] minorée par $\sqrt{2}$ d'après la question 4.
\end{itemize}

Or toute suite décroissante minorée converge (théorème de limite monotone du chapitre 4), donc $\left(a_n\right)_{n\in\mathbb{N}}$ converge. De plus, sa limite $\ell$ est supérieure ou égale au minorant $\sqrt{2}.$

\medskip

Pour déterminer $\ell,$ on reprend un raisonnement du chapitre 6~:

\medskip

La fonction $g:x\mapsto \frac{1}{2}x+\frac{1}{x}$ est continue sur $\left[\sqrt{2};+\infty\right[,$ donc on peut \og passer à la limite \fg~{} dans la formule de récurrence~:

\[a_{n+1}=\frac{1}{2}a_n+\frac{1}{a_n}\qquad\text{pour tout }n\in\mathbb{N},\] donc
\begin{equation}\ell=\frac{1}{2}\ell+\frac{1}{\ell}.
\end{equation}



On résout cette équation dans $\left[\sqrt{2};+\infty\right[~:$
\[\ell=\frac{1}{2}\ell+\frac{1}{\ell} \iff \ell-\frac{1}{2}\ell=\frac{1}{\ell}
\iff \frac{\ell}{2}=\frac{1}{\ell}\iff \ell\times\ell=2\times 1\iff \ell^2=2\iff \ell=\sqrt{2}~\text{ou}~\underbrace{\ell=-\sqrt{2}}_{\text{impossible}}.\]

Conclusion~: $\lim\limits_{n\to +\infty}a_n=\sqrt{2}.$
\end{enumerate}
\end{exo}


\section{Équations de plans, représentations de droites}





Dans tout ce chapitre, on n'a plus intérêt à placer les points précisément dans un repère. On fait tout de même des figures pour illustrer les situations et guider les raisonnements.

\begin{exo}



\begin{enumerate}
\item Soient $A(2;1;-1),$ $B(0;2;3),$ $C(1;-1;0)$ et le vecteur $\overrightarrow{n}\begin{pmatrix}9\\-2\\5\end{pmatrix}.$
\begin{enumerate}
\item On calcule les coordonnées des vecteurs~:
\[\overrightarrow{AB}\begin{pmatrix} x_B-x_A\\y_B-y_A\\z_B-z_A\end{pmatrix}\qquad\overrightarrow{AB}\begin{pmatrix} 0-2\\2-1\\3-(-1)\end{pmatrix}\qquad\overrightarrow{AB}\begin{pmatrix} -2\\1\\4\end{pmatrix}\qquad;\qquad\overrightarrow{AC}\begin{pmatrix} x_C-x_A\\y_C-y_A\\z_C-z_A\end{pmatrix}\qquad\overrightarrow{AC}\begin{pmatrix} 1-2\\-1-1\\0-(-1)\end{pmatrix}\qquad\overrightarrow{AC}\begin{pmatrix} -1\\-2\\1\end{pmatrix}.\] Les vecteurs $\overrightarrow{AB}$ et $\overrightarrow{AC}$ ne sont pas colinéaires, donc les points $A,$ $B,$ $C$ ne sont pas alignés et déterminent un plan.

\medskip

Le vecteur $\overrightarrow{n}$ est orthogonal aux vecteurs $\overrightarrow{AB}$ et $\overrightarrow{AC}$ puisque
\begin{align*}
\overrightarrow{n}\cdot\overrightarrow{AB}&=9\times (-2)+(-2)\times 1+5\times 4=0,\\
\overrightarrow{n}\cdot\overrightarrow{AC}&=9\times (-1)+(-2)\times (-2)+5\times 1=0.
\end{align*}

Conclusion~: le vecteur $\overrightarrow{n}$ est orthogonal aux vecteurs  directeurs $\overrightarrow{AB}$ et $\overrightarrow{AC}$ du plan $(ABC),$ donc il est  orthogonal au plan $(ABC).$


\begin{center}
\psset{xunit=1.8cm,yunit=1.8cm,algebraic=true,dimen=middle,dotstyle=o,dotsize=5pt 0,linewidth=2.pt,arrowsize=3pt 2,arrowinset=0.25}
\begin{pspicture*}(2.356763410104027,-0.3400624724627181)(8.578677948246996,2.5595659457871553)
\pspolygon[linewidth=2.pt,linecolor=blue,fillcolor=blue!10!white,fillstyle=solid,opacity=0.25](2.5,0.)(7.,0.)(8.,1.5)(4.,1.5)
\pspolygon[linewidth=2.pt,linecolor=red,fillcolor=red!10!white,fillstyle=solid,opacity=0.25](5.200035732456475,0.9000119108188249)(5.5,1.)(5.5,1.2241019560476185)(5.2,1.12)
\pspolygon[linewidth=2.pt,linecolor=red,fillcolor=red!10!white,fillstyle=solid,opacity=0.25](5.5,1.)(5.5,1.2241019560476185)(5.75358,1.02)(5.754874936005471,0.7961000511956229)
\psline[linewidth=2.pt,linecolor=blue](2.5,0.)(7.,0.)
\psline[linewidth=2.pt,linecolor=blue](7.,0.)(8.,1.5)
\psline[linewidth=2.pt,linecolor=blue](8.,1.5)(4.,1.5)
\psline[linewidth=2.pt,linecolor=blue](4.,1.5)(2.5,0.)
\psline[linewidth=2.pt]{->}(5.5,1.)(5.5,2.5)
\psline[linewidth=2.pt](5.5,1.)(4.,0.5)
\psline[linewidth=2.pt](5.5,1.)(6.,0.6)
\psline[linewidth=2.pt](5.200035732456475,0.9000119108188249)(5.2,1.12)
\psline[linewidth=2.pt](5.2,1.12)(5.5,1.2241019560476185)
\psline[linewidth=2.pt](5.754874936005471,0.7961000511956229)(5.75358,1.02)
\psline[linewidth=2.pt](5.5,1.2241019560476185)(5.75358,1.02)
\psline[linewidth=2.pt,linecolor=red](5.200035732456475,0.9000119108188249)(5.5,1.)
\psline[linewidth=2.pt,linecolor=red](5.5,1.)(5.5,1.2241019560476185)
\psline[linewidth=2.pt,linecolor=red](5.5,1.2241019560476185)(5.2,1.12)
\psline[linewidth=2.pt,linecolor=red](5.2,1.12)(5.200035732456475,0.9000119108188249)
\psline[linewidth=2.pt,linecolor=red](5.5,1.)(5.5,1.2241019560476185)
\psline[linewidth=2.pt,linecolor=red](5.5,1.2241019560476185)(5.75358,1.02)
\psline[linewidth=2.pt,linecolor=red](5.75358,1.02)(5.754874936005471,0.7961000511956229)
\psline[linewidth=2.pt,linecolor=red](5.754874936005471,0.7961000511956229)(5.5,1.)
\rput[tl](5.561098543492033,1.9698110132617574){$\overrightarrow{n}$}
\psdots[dotstyle=*,linecolor=blue](5.5,1.)
\rput[bl](5.384172063734414,0.7){\blue{$A$}}
\psdots[dotstyle=*,linecolor=blue](4.,0.5)
\rput[bl](3.850809239168374,0.5838869218270721){\blue{$B$}}
\psdots[dotstyle=*,linecolor=blue](6.,0.6)
\rput[bl](6.0427317383877766,0.7018379083321518){\blue{$C$}}
\end{pspicture*}
\end{center}

\item Le vecteur $\overrightarrow{n}\begin{pmatrix}9\\-2\\5\end{pmatrix}~\textcolor{red}{\begin{matrix}a\\b\\c\end{matrix}}$ est orthogonal au plan $(ABC),$ donc $(ABC)$ a une équation de la forme
\begin{align*}ax+by+cz+d&=0\\9x-2y+5z+d&=0.\end{align*}
Et comme $(ABC)$ passe par $A(2;1;-1)~:$
\begin{align*}
9\times 2-2\times 1+5\times (-1)+d&=0\\
11+d&=0\\
d&=-11.\end{align*}

Conclusion~:
\[(ABC):9x-2y+5z-11=0.\]

\end{enumerate}
\item Soient $A(2;-1;3)$ et $B(2;5;0).$ On note $P$ le plan médiateur de $\left[AB\right].$ Ce plan passe par le milieu $I$ de $\left[AB\right]$ et il est orthogonal à $\overrightarrow{AB}.$


\begin{center}
\psset{xunit=1.8cm,yunit=1.8cm,algebraic=true,dimen=middle,dotstyle=o,dotsize=5pt 0,linewidth=2.pt,arrowsize=3pt 2,arrowinset=0.25}
\begin{pspicture*}(2.356763410104027,-0.7922079207321905)(8.205166490980906,2.7168339277939277)
\pspolygon[linewidth=2.pt,linecolor=blue,fillcolor=blue!10!white,fillstyle=solid,opacity=0.25](2.5,0.)(7.,0.)(8.,1.5)(4.,1.5)
\pspolygon[linewidth=2.pt,linecolor=red,fillcolor=red!10!white,fillstyle=solid,opacity=0.25](5.200035732456475,0.9000119108188249)(5.5,1.)(5.5,1.2241019560476185)(5.2,1.12)
\pspolygon[linewidth=2.pt,linecolor=red,fillcolor=red!10!white,fillstyle=solid,opacity=0.25](5.5,1.)(5.5,1.2241019560476185)(5.75358,1.02)(5.754874936005471,0.7961000511956229)
\psline[linewidth=2.pt,linecolor=blue](2.5,0.)(7.,0.)
\psline[linewidth=2.pt,linecolor=blue](7.,0.)(8.,1.5)
\psline[linewidth=2.pt,linecolor=blue](8.,1.5)(4.,1.5)
\psline[linewidth=2.pt,linecolor=blue](4.,1.5)(2.5,0.)
\psline[linewidth=2.pt](5.5,1.)(4.,0.5)
\psline[linewidth=2.pt](5.5,1.)(6.,0.6)
\psline[linewidth=2.pt](5.200035732456475,0.9000119108188249)(5.2,1.12)
\psline[linewidth=2.pt](5.2,1.12)(5.5,1.2241019560476185)
\psline[linewidth=2.pt](5.754874936005471,0.7961000511956229)(5.75358,1.02)
\psline[linewidth=2.pt](5.5,1.2241019560476185)(5.75358,1.02)
\psline[linewidth=2.pt,linecolor=red](5.200035732456475,0.9000119108188249)(5.5,1.)
\psline[linewidth=2.pt,linecolor=red](5.5,1.)(5.5,1.2241019560476185)
\psline[linewidth=2.pt,linecolor=red](5.5,1.2241019560476185)(5.2,1.12)
\psline[linewidth=2.pt,linecolor=red](5.2,1.12)(5.200035732456475,0.9000119108188249)
\psline[linewidth=2.pt,linecolor=red](5.5,1.)(5.5,1.2241019560476185)
\psline[linewidth=2.pt,linecolor=red](5.5,1.2241019560476185)(5.75358,1.02)
\psline[linewidth=2.pt,linecolor=red](5.75358,1.02)(5.754874936005471,0.7961000511956229)
\psline[linewidth=2.pt,linecolor=red](5.754874936005471,0.7961000511956229)(5.5,1.)
\psline[linewidth=2.pt,linecolor=red](5.5,2.5)(5.5,1.)
\psline[linewidth=2.pt,linestyle=dashed,dash=2pt 2pt,linecolor=red](5.5,1.)(5.5,0.)
\psline[linewidth=2.pt,linecolor=red](5.5,0.)(5.5,-0.5)
\rput[tl](6.8585593950479105,0.8492766414635007){\blue{$P$}}
\psdots[dotstyle=*,linecolor=blue](5.5,1.)
\rput[bl](5.3,0.7){\blue{$I$}}
\psdots[dotstyle=*,linecolor=red](5.5,2.5)
\rput[bl](5.541440045741185,2.559565945787155){\red{$A$}}
\psdots[dotstyle=*,linecolor=red](5.5,-0.5)
\rput[bl](5.541440045741185,-0.39903796571525846){\red{$B$}}
\end{pspicture*}
\end{center}

On calcule donc, avec les formules habituelles~:
\begin{align*}
&I\left(\frac{x_A+x_B}{2};\frac{y_A+y_B}{2};\frac{z_A+z_B}{2}\right)\qquad
I\left(\frac{2+2}{2};\frac{-1+5}{2};\frac{3+0}{2}\right)\qquad
I(2;2;1,5),\\
&\overrightarrow{AB}\begin{pmatrix} x_B-x_A\\y_B-y_A\\z_B-z_A\end{pmatrix}\qquad
\overrightarrow{AB}\begin{pmatrix} 2-2\\5-(-1)\\0-3\end{pmatrix}\qquad\overrightarrow{AB}\begin{pmatrix} 0\\6\\-3\end{pmatrix}~\textcolor{red}{\begin{matrix}a\\b\\c\end{matrix}}~.\end{align*}

Le plan $P$ a donc une équation de la forme \[0x+6y-3z+d=0.\] Et comme il passe par $I~:$
\begin{align*}
0\times 2+6\times 2-3\times 1,5+d&=0\\
7,5+d&=0\\
d&=-7,5.\end{align*}

Conclusion~:
\[P:6y-3z-7,5=0.\]

\medskip

\textbf{Remarque~:} En multipliant par $\frac{2}{3},$ on a une écriture plus agréable~: $P:4y-2z-5=0.$

\item Soient $(P)$ le plan d'équation $4x+y-z-3=0,$ et soient $M(1;2;3)$ et $N(9;4;1).$

\begin{enumerate}
\item Le point $M$ appartient au plan $(P),$ car (on remplace $x,$ $y,$ $z$ par les coordonnées de $M$)~:
\[4\times 1+2-3-3=0.\]

\medskip

Le point $N$ n'appartient pas au plan $(P),$ car~:
\[4\times 9+4-1-3=36\not=0.\]


\item Le plan $(P)$ a pour équation $\textcolor{red}{4}x+\textcolor{red}{1}y+\textcolor{red}{(-1)}z-3=0,$ donc le vecteur $\overrightarrow{n}\begin{pmatrix}\textcolor{red}{4}\\ \textcolor{red}{1}\\ \textcolor{red}{-1}\end{pmatrix}$ est normal au plan $(P).$

\medskip

Par ailleurs 
\[\overrightarrow{MN}\begin{pmatrix}x_N-x_M\\y_N-y_M\\z_N-z_M\end{pmatrix}\qquad 
\overrightarrow{MN}\begin{pmatrix}9-1\\4-2\\1-3\end{pmatrix}\qquad \overrightarrow{MN}\begin{pmatrix}8\\2\\-2\end{pmatrix}.\]
\medskip

On remarque que $ \overrightarrow{MN}=2\overrightarrow{n},$ donc $\overrightarrow{MN}$ et $\overrightarrow{n}$ sont colinéaires. Or le vecteur $\overrightarrow{n}$ est orthogonal au plan $(P),$ donc la droite $(MN)$ est orthogonale à $(P).$


\begin{center}
\newrgbcolor{ududff}{0.30196078431372547 0.30196078431372547 1.}
\psset{xunit=1.0cm,yunit=1.0cm,algebraic=true,dimen=middle,dotstyle=o,dotsize=5pt 0,linewidth=2.pt,arrowsize=3pt 2,arrowinset=0.25}
\begin{pspicture*}(0.52,0.46)(9.88,6.58)
\psaxes[labelFontSize=\scriptstyle,xAxis=true,yAxis=true,Dx=1.,Dy=1.,ticksize=-2pt 0,subticks=2]{->}(0,0)(0.52,0.46)(9.88,6.58)
\pspolygon[linewidth=2.pt,linecolor=blue,fillcolor=blue!10!white,fillstyle=solid,opacity=0.1](1.,1.)(7.,1.)(9.,4.)(3.,4.)
\psline[linewidth=2.pt,linecolor=blue](1.,1.)(7.,1.)
\psline[linewidth=2.pt,linecolor=blue](7.,1.)(9.,4.)
\psline[linewidth=2.pt,linecolor=blue](9.,4.)(3.,4.)
\psline[linewidth=2.pt,linecolor=blue](3.,4.)(1.,1.)
\psline[linewidth=2.pt]{->}(5.,3.)(5.,5.)
\psline[linewidth=2.pt]{->}(6.,2.)(6.,6.)
\rput[tl](4.38,3.84){$\overrightarrow{n}$}
\rput[tl](1.86,1.58){\ududff{$(P)$}}
\psdots[dotstyle=*,linecolor=ududff](6.,2.)
\rput[bl](6.22,2.18){\ududff{$M$}}
\psdots[dotstyle=*,linecolor=red](6.,6.)
\rput[bl](6.24,5.62){\red{$N$}}
\end{pspicture*}
\end{center}
\end{enumerate}
\end{enumerate}
\end{exo}




\begin{exo}

$OABC$ est un tétraèdre trirectangle en $O,$ c'est-à-dire que $OAB,$ $OAC$ et $OBC$ sont rectangles en $O.$ On suppose de plus que $OA=OB=OC=1.$ On note $H$ le projeté orthogonal de $O$ sur le plan $(ABC),$ c'est-à-dire que $H\in(ABC)$ et $(OH)\perp (ABC).$


\begin{center}
\psset{xunit=1cm,yunit=1cm,algebraic=true,dimen=middle,dotstyle=o,dotsize=5pt 0,linewidth=1.6pt,arrowsize=3pt 2,arrowinset=0.25}
\begin{pspicture*}(-0.26,0.84)(6.64,6.22)
\psline[linewidth=2.pt,linestyle=dashed,dash=2pt 2pt](2.,6.)(2.,2.)
\psline[linewidth=2.pt,linestyle=dashed,dash=2pt 2pt](2.,2.)(6.,2.)
\psline[linewidth=2.pt,linestyle=dashed,dash=2pt 2pt](2.,2.)(0.8,1.28)
\psline[linewidth=2.pt](0.8,1.28)(2.,6.)
\psline[linewidth=2.pt](0.8,1.28)(6.,2.)
\psline[linewidth=2.pt](2.,6.)(6.,2.)
\psline[linewidth=2.pt,linestyle=dashed,dash=2pt 2pt](2.,2.)(3.,3.)
\rput[bl](1.56,5.8){$C$}
\rput[bl](1.55,2.14){$O$}
\rput[bl](5.92,1.6){$B$}
\rput[bl](0.42,1.4){$A$}
\psdots[dotsize=1pt 0,dotstyle=*](3.,3.)
\rput[bl](3.08,3.04){$H$}
\end{pspicture*}
\end{center}

On travaille dans le repère $\left(O,\overrightarrow{OA},\overrightarrow{OB},\overrightarrow{OC}\right).$



\begin{enumerate}
\item Pour prouver que $(ABC)$ a pour équation $x+y+z=1,$ il suffit de prouver que les coordonnées des points $A,$ $B,$ $C$ vérifient cette égalité. On le fait pour $A$ (le travail est le même pour $B$ et $C$)~:
\[A(1;0;0)\qquad\text{et}\qquad 1+0+0=1\qquad\text{donc}~A~\text{appartient au plan}.\]
\item Notons $(x_H;y_H;z_H)$ les coordonnées du point $H.$ Puisque ce point est sur le plan $(ABC):x+y+z=1,$ on a $x_H+y_H+z_H=1,$ et donc en transposant~: \[z_H=1-x_H-y_H.\]

Il existe donc bien deux réels $x_H,$ $y_H$ tels que $H$ ait pour coordonnées $(x_H;y_H;1-x_H-y_H).$
\item On a $O(0;0;0),$ $A(1;0;0),$ $B(0;1;0),$ $C(0;0;1)$ et $H\left(x_H;y_H;1-x_H-y_H\right).$ Donc

\[\overrightarrow{OH}\begin{pmatrix} x_H\\y_H\\1-x_H-y_H\end{pmatrix}\qquad\overrightarrow{AB}\begin{pmatrix}-1\\1\\0\end{pmatrix} \qquad\overrightarrow{AC}\begin{pmatrix}-1\\0\\1\end{pmatrix},\]

puis
\begin{align*}
\overrightarrow{OH}\cdot\overrightarrow{AB}&=x_H\times (-1)+y_H\times 1+\left(1-x_H-y_H\right)\times 0=-x_H+y_H,\\
\overrightarrow{OH}\cdot\overrightarrow{AC}&=x_H\times (-1)+y_H\times 0+\left(1-x_H-y_H\right)\times 1=-x_H+1-x_H-y_H=1-2x_H-y_H.
\end{align*}

\item La droite $(OH)$ est orthogonale au plan $(ABC),$ donc elle est orthogonale à $(AB)$ et à $(AC).$ On a donc $\overrightarrow{OH}\cdot \overrightarrow{AB}=0$ et $\overrightarrow{OH}\cdot \overrightarrow{AC}=0.$ Soit, d'après la question précédente~:

\[\begin{cases}
-x_H+y_H&=0
\\ 1-2x_H-y_H&=0
\end{cases}
\]

La première ligne donne $y_H=x_H,$ puis en remplaçant dans la deuxième~:
\[-2x_H-x_H+1=0\qquad -3x_H+1=0\qquad 1=3x_H\qquad x_H=\frac{1}{3}.\]
Finalement $x_H=y_H=\frac{1}{3},$ puis \[z_H=1-x_H-y_H=1-\frac{1}{3}-\frac{1}{3}=1-\frac{2}{3}=\frac{1}{3}.\]

Conclusion~: le point $H$ a pour coordonnées $\left(\frac{1}{3};\frac{1}{3};\frac{1}{3}\right).$

\end{enumerate}


\end{exo}





\begin{exo}

Faire une figure n'est pas évident ici, parce que l'énoncé part de la droite $d$ et du point $A,$ puis définit $B$ et enfin le plan $P.$ On préférerait que ce soit l'inverse~; et c'est justement en faisant tout à l'envers que l'on construit la figure \footnote{La figure, que nous faisons en premier, n'est là que pour illustrer l'exercice et guider l'intuition. Lorsque nous ferons les calculs, en revanche, nous irons \og dans l'ordre \fg.}~: on trace un plan $P,$ on y place un point $B,$ puis on trace une droite $d$ orthogonale à $P$ et ne passant pas par $B,$ sur laquelle on place un point $A.$


\begin{center}
\newrgbcolor{ududff}{0.30196078431372547 0.30196078431372547 1.}
\psset{xunit=1.8cm,yunit=1.8cm,algebraic=true,dimen=middle,dotstyle=o,dotsize=5pt 0,linewidth=2.pt,arrowsize=3pt 2,arrowinset=0.25}
\begin{pspicture*}(2.4157389033565653,-0.910158907237268)(8.382092970738524,3.2843469772753022)
\pspolygon[linewidth=2.pt,linecolor=blue,fillcolor=blue!10!white,fillstyle=solid,opacity=0.25](2.5,0.)(7.,0.)(8.,1.5)(4.,1.5)
\pspolygon[linewidth=2.pt,linecolor=red,fillcolor=red!10!white,fillstyle=solid,opacity=0.25](5.200035732456475,0.9000119108188249)(5.5,1.)(5.5,1.2988026080634913)(5.2,1.17)
\pspolygon[linewidth=2.pt,linecolor=red,fillcolor=red!10!white,fillstyle=solid,opacity=0.25](5.5,1.)(5.5,1.2988026080634913)(5.75358,1.09)(5.754874936005471,0.7961000511956229)
\psline[linewidth=2.pt,linecolor=blue](2.5,0.)(7.,0.)
\psline[linewidth=2.pt,linecolor=blue](7.,0.)(8.,1.5)
\psline[linewidth=2.pt,linecolor=blue](8.,1.5)(4.,1.5)
\psline[linewidth=2.pt,linecolor=blue](4.,1.5)(2.5,0.)
\psline[linewidth=2.pt](5.5,1.)(4.,0.5)
\psline[linewidth=2.pt](5.5,1.)(6.,0.6)
\psline[linewidth=2.pt](5.200035732456475,0.9000119108188249)(5.2,1.17)
\psline[linewidth=2.pt](5.2,1.17)(5.5,1.2988026080634913)
\psline[linewidth=2.pt](5.754874936005471,0.7961000511956229)(5.75358,1.09)
\psline[linewidth=2.pt](5.5,1.2988026080634913)(5.75358,1.09)
\psline[linewidth=2.pt,linecolor=red](5.200035732456475,0.9000119108188249)(5.5,1.)
\psline[linewidth=2.pt,linecolor=red](5.5,1.)(5.5,1.2988026080634913)
\psline[linewidth=2.pt,linecolor=red](5.5,1.2988026080634913)(5.2,1.17)
\psline[linewidth=2.pt,linecolor=red](5.2,1.17)(5.200035732456475,0.9000119108188249)
\psline[linewidth=2.pt,linecolor=red](5.5,1.)(5.5,1.2988026080634913)
\psline[linewidth=2.pt,linecolor=red](5.5,1.2988026080634913)(5.75358,1.09)
\psline[linewidth=2.pt,linecolor=red](5.75358,1.09)(5.754874936005471,0.7961000511956229)
\psline[linewidth=2.pt,linecolor=red](5.754874936005471,0.7961000511956229)(5.5,1.)
\rput[tl](6.711120661916558,0.2){\blue{$P$}}
\psline[linewidth=2.pt,linecolor=green](5.5,1.)(5.5,3.)
\psline[linewidth=2.pt,linestyle=dashed,dash=2pt 2pt,linecolor=green](5.5,1.)(5.5,0.)
\psline[linewidth=2.pt,linecolor=green](5.5,0.)(5.502123050239489,-0.733232427479649)
\rput[tl](5.64956178337084,-0.41869646346610334){\green{$d$}}
\psdots[dotstyle=*,linecolor=ududff](6.553852679909785,1.1048371122245073)
\rput[bl](6.593169675411478,1.20312960097874){\ududff{$B$}}
\psdots[dotstyle=*,linecolor=green](5.5,2.5)
\rput[bl](5.541440045741183,2.5975912486422597){\green{$A$}}
\psline[linewidth=2.pt,linecolor=black]{->}(5.5,1.)(5.5,2)
\rput[tl](5.6,1.8){$\overrightarrow{u}$}
\end{pspicture*}
\end{center}

Les vecteurs $\overrightarrow{AB}\begin{pmatrix} 1\\-4\\3\end{pmatrix}$ et $\overrightarrow{u}\begin{pmatrix}-1\\5\\3\end{pmatrix}$ ne sont pas colinéaires, donc $B$ n'appartient pas à la droite $d$ passant par $A$ et dirigée par $\overrightarrow{u}.$

\medskip

Le plan $P$ est orthogonal à $d,$ donc il est orthogonal à $\overrightarrow{u}\begin{pmatrix} -1\\5\\3\end{pmatrix}~\textcolor{red}{\begin{matrix}a\\b\\c\end{matrix}}.$ Il a donc une équation de la forme
\[-x+5y+3z+d=0.\]
Et comme $P$ passe par $B(2;-4;1),$ on a 
\begin{align*}
-2+5\times (-4)+3\times 1+d&=0\\
-19+d&=0\\
d&=19.
\end{align*}
Finalement
\[P:-x+5y+3z+19=0.\]

\end{exo}



\begin{exo}


\begin{enumerate}
\item On considère les points $A(2;-1;4)$ et $B(1;3;4).$
\begin{enumerate}
\item Comme $\overrightarrow{AB}\begin{pmatrix}-1\\ 4\\ 0\end{pmatrix},$ une représentation paramétrique de $(AB)$ est
\[\begin{cases}x&=x_A+t\times (-1)\\y&=y_A+t\times 4\\z&=z_A+t\times 0\end{cases}\hspace{1cm}\text{soit}\hspace{1cm}\begin{cases}x&=2-t\\y&=-1+4t\\z&=4\end{cases}.\]

\item Savoir si $K$ appartient à $(AB)$ revient à savoir s'il existe un réel $t$ tel que \[\begin{cases}0&=2-t\\7&=-1+4t\\4&=4\end{cases}.\]

Il est clair que $t=2$ est solution, donc $K\in (AB).$
\end{enumerate}
\item La droite $(D)$ passe par $C\left(2;-1;0\right)$ et elle est  dirigée par $\overrightarrow{u}\begin{pmatrix}2\\-8\\0\end{pmatrix},$ donc sa représentation paramétrique est

\[\begin{cases}x&=x_C+t\times 2\\y&=y_C+t\times (-8)\\z&=z_C+t\times 0\end{cases}\hspace{1cm}\text{soit}\hspace{1cm}\begin{cases}x&=2+2t\\y&=-1-8t\\z&=0\end{cases}.\]

\item On remarque que $\overrightarrow{u}=-2\overrightarrow{AB},$ donc $\overrightarrow{u}$ et $\overrightarrow{AB}$ sont colinéaires~; et donc $(D)$ est parallèle à $(AB).$
\end{enumerate}

\end{exo}

\begin{exo}

Soient $D$ et $D'$ les droites de représentations paramétriques respectives
\[\begin{cases}x&=\textcolor{red}{1}t+1\\y&=\textcolor{red}{2}t-1\\z&=\textcolor{red}{3}t+2\end{cases}\qquad,\qquad \begin{cases}x&=\textcolor{blue}{1}t+1\\y&=\textcolor{blue}{1}t+3\\z&=\textcolor{blue}{-1}t+4\end{cases}.\]

\begin{enumerate}
\item D'après leur représentation paramétrique~:
\begin{itemize}
\item[\textbullet] un vecteur directeur de $D$ est $\overrightarrow{u}\begin{pmatrix}\textcolor{red}{1}\\\textcolor{red}{2}\\\textcolor{red}{3}\end{pmatrix}~;$
\item[\textbullet] un vecteur directeur de $D'$ est $\overrightarrow{v}\begin{pmatrix}\textcolor{blue}{1}\\\textcolor{blue}{1}\\\textcolor{blue}{-1}\end{pmatrix}.$
\end{itemize}

Les vecteurs $\overrightarrow{u}$ et $\overrightarrow{v}$ ne sont pas colinéaires, donc $D$ et $D'$ ne sont pas parallèles.

\medskip

En revanche, $\overrightarrow{u}$ et $\overrightarrow{v}$ sont orthogonaux, puisque
\[\overrightarrow{u}\cdot \overrightarrow{v}=1\times 1+2\times 1+3\times (-1)=0.\]

Les droites $D$ et $D'$ sont donc orthogonales.
\item Pour prouver que le plan $P:x+y-z+2=0$ contient la droite $D,$ il suffit de prouver que tous les points de la droite $D$ sont dans le plan $P.$ Or la représentation paramétrique de $D$ dit que ces points ont des coordonnées de la forme $\begin{cases}x&=t+1\\y&=2t-1\\z&=3t+2\end{cases}.$ Il suffit donc de prouver qu'ils vérifient l'équation de $P$ (on \og injecte \fg~{} la représentation paramétrique de $D$ dans l'équation de $P$)~:
\[x+y-z+2=(t+1)+(2t-1)-(3t+2)+2=t+1+2t-1-3t-2+2=0.\] La droite $D$ est donc bien incluse dans le plan $P.$
\medskip

Pour prouver que le plan $P$ est orthogonal à la droite $D',$ on utilise un vecteur normal~: $P$ a pour équation $\underset{\textcolor{red}{a}}{1}x+\underset{\textcolor{red}{b}}{1}y+\underset{\textcolor{red}{c}}{-1}z+2=0,$ donc un vecteur normal à $P$ est
$\overrightarrow{n}\begin{pmatrix}1\\1\\-1\end{pmatrix}~\textcolor{red}{\begin{matrix}a\\b\\c\end{matrix}}.$

Ce vecteur est égal\footnote{La colinéarité suffirait.} au vecteur directeur $\overrightarrow{v}\begin{pmatrix}1\\1\\-1\end{pmatrix}$ de la droite $D',$ donc $D'$ est orthogonale au plan $P.$
\end{enumerate}


\end{exo}

\begin{exo}

Soient $P:-3x+y+2z-10=0$ et $A(2;0;1).$ On cherche les coordonnées du projeté orthogonal $H$ de $A$ sur le plan $P.$

\begin{center}
\newrgbcolor{ududff}{0.30196078431372547 0.30196078431372547 1.}
\psset{xunit=1.8cm,yunit=1.8cm,algebraic=true,dimen=middle,dotstyle=o,dotsize=5pt 0,linewidth=2.pt,arrowsize=3pt 2,arrowinset=0.25}
\begin{pspicture*}(2.4157389033565653,-0.910158907237268)(8.382092970738524,3.2843469772753022)
\pspolygon[linewidth=2.pt,linecolor=blue,fillcolor=blue!10!white,fillstyle=solid,opacity=0.25](2.5,0.)(7.,0.)(8.,1.5)(4.,1.5)
\pspolygon[linewidth=2.pt,linecolor=red,fillcolor=red!10!white,fillstyle=solid,opacity=0.25](5.200035732456475,0.9000119108188249)(5.5,1.)(5.5,1.2988026080634913)(5.2,1.17)
\pspolygon[linewidth=2.pt,linecolor=red,fillcolor=red!10!white,fillstyle=solid,opacity=0.25](5.5,1.)(5.5,1.2988026080634913)(5.75358,1.09)(5.754874936005471,0.7961000511956229)
\psline[linewidth=2.pt,linecolor=blue](2.5,0.)(7.,0.)
\psline[linewidth=2.pt,linecolor=blue](7.,0.)(8.,1.5)
\psline[linewidth=2.pt,linecolor=blue](8.,1.5)(4.,1.5)
\psline[linewidth=2.pt,linecolor=blue](4.,1.5)(2.5,0.)
\psline[linewidth=2.pt](5.5,1.)(4.,0.5)
\psline[linewidth=2.pt](5.5,1.)(6.,0.6)
\psline[linewidth=2.pt](5.200035732456475,0.9000119108188249)(5.2,1.17)
\psline[linewidth=2.pt](5.2,1.17)(5.5,1.2988026080634913)
\psline[linewidth=2.pt](5.754874936005471,0.7961000511956229)(5.75358,1.09)
\psline[linewidth=2.pt](5.5,1.2988026080634913)(5.75358,1.09)
\psline[linewidth=2.pt,linecolor=red](5.200035732456475,0.9000119108188249)(5.5,1.)
\psline[linewidth=2.pt,linecolor=red](5.5,1.)(5.5,1.2988026080634913)
\psline[linewidth=2.pt,linecolor=red](5.5,1.2988026080634913)(5.2,1.17)
\psline[linewidth=2.pt,linecolor=red](5.2,1.17)(5.200035732456475,0.9000119108188249)
\psline[linewidth=2.pt,linecolor=red](5.5,1.)(5.5,1.2988026080634913)
\psline[linewidth=2.pt,linecolor=red](5.5,1.2988026080634913)(5.75358,1.09)
\psline[linewidth=2.pt,linecolor=red](5.75358,1.09)(5.754874936005471,0.7961000511956229)
\psline[linewidth=2.pt,linecolor=red](5.754874936005471,0.7961000511956229)(5.5,1.)
\rput[tl](6.711120661916558,0.25){\blue{$P$}}
\psline[linewidth=2.pt,linecolor=green](5.5,1.)(5.5,2.5)
\rput[tl](5.4,0.85){\green{$H$}}
\psdots[dotstyle=*,linecolor=green](5.5,1)
\psdots[dotstyle=*,linecolor=green](5.5,2.5)
\rput[bl](5.541440045741183,2.5975912486422597){\green{$A$}}
\end{pspicture*}
\end{center}

\begin{enumerate}
\item La droite $(AH)$ est orthogonale au plan $P,$ donc elle est dirigée par un vecteur normal au plan $P.$

L'équation de $P$ est $-3x+y+2z-10=0,$ donc un vecteur normal à ce plan est $\overrightarrow{n}\begin{pmatrix}-3\\1\\2\end{pmatrix}.$ La droite $(AH)$ a donc pour représentation paramétrique \[\begin{cases}x&=x_A+t\times (-3)\\y&=y_A+t\times 1\\z&=z_A+t\times 2\end{cases}\hspace{1cm}\text{soit}\hspace{1cm}\begin{cases}x&=2-3t\\y&=t\\z&=1+2t\end{cases}.\]


\item Le point $H$ est le point d'intersection de la droite $(AH)$ avec le plan $P,$ donc pour obtenir ses coordonnées, on \og injecte \fg~{} la représentation paramétrique de $(AH)$  dans l'équation de $P,$ puis on résout~:
\begin{align*}
-3x+y+2z-10&=0\\
-3(2-3t)+(t)+2(1+2t)-10&=0\\
-6+9t+t+2+4t-10&=0\\
14t-14&=0\\
t=\frac{14}{14}&=1.
\end{align*}

Enfin on remplace $t$ par $1$ dans la représentation paramétrique~:
\[\begin{cases}x&=2-3t=2-3\times 1=-1\\y&=t=1\\z&=1+2t=1+2\times 1=3\end{cases}.\]

Conclusion~: $H(-1;1;3).$
\end{enumerate}

\end{exo}

\begin{exo}

Soient $A(1;2;3),$ $B(0;1;4),$ $C(-1;-3;2),$ $D(4;-2;5)$ et le vecteur $\overrightarrow{n}\begin{pmatrix}2\\-1\\1\end{pmatrix}.$

\begin{enumerate}
\item 
\begin{enumerate}
\item On calcule les coordonnées des vecteurs~:
\[\overrightarrow{AB}\begin{pmatrix} -1\\-1\\1\end{pmatrix}\qquad;\qquad \overrightarrow{AC}\begin{pmatrix} -2\\-5\\-1\end{pmatrix}.\] Les vecteurs $\overrightarrow{AB}$ et $\overrightarrow{AC}$ ne sont pas colinéaires, donc les points $A,$ $B,$ $C$ ne sont pas alignés (et déterminent un plan).

\item Le vecteur $\overrightarrow{n}$ est orthogonal au plan $(ABC),$ puisqu'il est orthogonal aux vecteurs  directeurs du plan $\overrightarrow{AB}$ et $\overrightarrow{AC}.$ En effet~:
\begin{align*}
\overrightarrow{n}\cdot\overrightarrow{AB}&=2\times (-1)+(-1)\times (-1)+1\times 1=0,\\
\overrightarrow{n}\cdot\overrightarrow{AC}&=2\times (-2)+(-1)\times (-5)+1\times (-1)=0.
\end{align*}

Le vecteur $\overrightarrow{n}\begin{pmatrix}2\\-1\\1\end{pmatrix}~\textcolor{red}{\begin{matrix}a\\b\\c\end{matrix}}$ étant orthogonal au plan $(ABC),$ ce dernier a une équation de la forme
\begin{align*}ax+by+cz+d&=0\\2x-y+z+d&=0.\end{align*}
Et comme $(ABC)$ passe par $A(1;2;3)~:$
\begin{align*}
2\times 1-2+3+d&=0\\
3+d&=0\\
d&=-3.\end{align*}

Conclusion~:
\[(ABC):2x-y+z-3=0.\]



\end{enumerate}
\item Soit $\Delta$ la droite dont une représentation paramétrique est~:

\[\begin{cases}x&=2-2t\\y&=-1+t\\z&=4-t\end{cases},~t\in\mathbb{R}.\]
\begin{enumerate}
\item Lorsqu'on prend $t=-1$ dans la représentation paramétrique de $\Delta,$ on obtient~:
\[\begin{cases}x&=2-2t=2-2\times(-1)=4\\y&=-1+t=-1+(-1)=-2\\z&=4-t=4-(-1)=5\end{cases}.\] Ce sont les coordonnées de $D,$ donc $D$ appartient bien à $\Delta.$\footnote{J'ai \og deviné \fg~{} qu'il fallait prendre $t=-1,$ mais ce n'était vraiment pas difficile.}

\medskip

D'après sa représentation paramétrique, un vecteur directeur de $\Delta$ est $\overrightarrow{u}\begin{pmatrix}-2\\1\\-1\end{pmatrix}.$ On sait aussi que le vecteur $\overrightarrow{n}\begin{pmatrix}2\\-1\\1\end{pmatrix}$ est orthogonal au plan $(ABC).$ On remarque que $\overrightarrow{n}=-\overrightarrow{u},$ donc $\overrightarrow{u}$ est aussi orthogonal au plan $(ABC).$ Comme c'est un vecteur directeur de $\Delta,$ la droite $\Delta$ est orthogonale au plan $(ABC).$


\item Le projeté orthogonal $E$ de $D$ sur le plan $(ABC)$ est le point d'intersection de $\Delta$ et de $(ABC),$ donc pour obtenir ses coordonnées, on \og injecte \fg~{} la représentation paramétrique de $\Delta$ dans l'équation de $(ABC),$ puis on résout~:
\begin{align*}
2x-y+z-3&=0\\
2(2-2t)-(-1+t)+(4-t)-3&=0\\
4-4t+1-t+4-t-3&=0\\
-6t+6&=0\\
t=\frac{-6}{-6}&=1.
\end{align*}

Enfin on remplace $t$ par $1$ dans la représentation paramétrique~:
\[\begin{cases}x&=2-2t=2-2\times 1=0\\y&=-1+t=-1+1=0\\z&=4-t=4-1=3\end{cases}.\]

Conclusion~: $E(0;0;3).$
\item La distance du point $D$ au plan $(ABC)$ est la longueur $DE~:$
\[d(D,(ABC))=DE=\sqrt{(0-4)^2+(0-(-2))^2+(3-5)^2}=\sqrt{24}.\]
\end{enumerate}


\begin{center}
\newrgbcolor{ffxfqq}{1. 0.4980392156862745 0.}
\newrgbcolor{ududff}{0.30196078431372547 0.30196078431372547 1.}
\psset{xunit=1.8cm,yunit=1.8cm,algebraic=true,dimen=middle,dotstyle=o,dotsize=5pt 0,linewidth=2.pt,arrowsize=3pt 2,arrowinset=0.25}
\begin{pspicture*}(2.4157389033565653,-0.910158907237268)(8.382092970738524,2.2843469772753022)
\pspolygon[linewidth=2.pt,linecolor=blue,fillcolor=blue!10!white,fillstyle=solid,opacity=0.25](2.5,0.)(7.,0.)(8.,1.5)(4.,1.5)
\pspolygon[linewidth=2.pt,linecolor=red,fillcolor=red!10!white,fillstyle=solid,opacity=0.25](5.200035732456475,0.9000119108188249)(5.5,1.)(5.5,1.2988026080634913)(5.2,1.17)
\pspolygon[linewidth=2.pt,linecolor=red,fillcolor=red!10!white,fillstyle=solid,opacity=0.25](5.5,1.)(5.5,1.2988026080634913)(5.75358,1.09)(5.754874936005471,0.7961000511956229)
\psline[linewidth=2.pt,linecolor=blue](2.5,0.)(7.,0.)
\psline[linewidth=2.pt,linecolor=blue](7.,0.)(8.,1.5)
\psline[linewidth=2.pt,linecolor=blue](8.,1.5)(4.,1.5)
\psline[linewidth=2.pt,linecolor=blue](4.,1.5)(2.5,0.)
\psline[linewidth=2.pt](5.5,1.)(4.,0.5)
\psline[linewidth=2.pt](5.5,1.)(6.,0.6)
\psline[linewidth=2.pt](5.200035732456475,0.9000119108188249)(5.2,1.17)
\psline[linewidth=2.pt](5.2,1.17)(5.5,1.2988026080634913)
\psline[linewidth=2.pt](5.754874936005471,0.7961000511956229)(5.75358,1.09)
\psline[linewidth=2.pt](5.5,1.2988026080634913)(5.75358,1.09)
\psline[linewidth=2.pt,linecolor=red](5.200035732456475,0.9000119108188249)(5.5,1.)
\psline[linewidth=2.pt,linecolor=red](5.5,1.)(5.5,1.2988026080634913)
\psline[linewidth=2.pt,linecolor=red](5.5,1.2988026080634913)(5.2,1.17)
\psline[linewidth=2.pt,linecolor=red](5.2,1.17)(5.200035732456475,0.9000119108188249)
\psline[linewidth=2.pt,linecolor=red](5.5,1.)(5.5,1.2988026080634913)
\psline[linewidth=2.pt,linecolor=red](5.5,1.2988026080634913)(5.75358,1.09)
\psline[linewidth=2.pt,linecolor=red](5.75358,1.09)(5.754874936005471,0.7961000511956229)
\psline[linewidth=2.pt,linecolor=red](5.754874936005471,0.7961000511956229)(5.5,1.)
\rput[tl](6.711120661916558,0.2){\blue{$P$}}
\psline[linewidth=2.pt,linecolor=ffxfqq](5.5,1.)(5.5,2.)
\psline[linewidth=2.pt,linestyle=dashed,dash=2pt 2pt,linecolor=ffxfqq](5.5,1.)(5.5,0.)
\psline[linewidth=2.pt,linecolor=ffxfqq](5.5,0.)(5.502123050239489,-0.733232427479649)
\rput[tl](5.5610985434920295,-0.3498917213381403){\ffxfqq{$\Delta$}}
\psdots[dotstyle=*,linecolor=ffxfqq](5.5,1.)
\rput[bl](5.32519657048187,0.7804718993355384){\ffxfqq{$E$}}
\psdots[dotstyle=*,linecolor=ududff](6.365389440030975,0.9885936369651944)
\rput[bl](6.504706435532668,0.9868861257194276){\ududff{$B$}}
\psdots[dotstyle=*,linecolor=ffxfqq](5.5,2.)
\rput[bl](5.541440045741183,2.0975912486422597){\ffxfqq{$D$}}
\psdots[dotstyle=*,linecolor=ududff](4.705953891330201,1.2646663610999301)
\rput[bl](4.445270886831894,1.2129588498541632){\ududff{$C$}}
\psdots[dotstyle=*,linecolor=ududff](4.430734922818347,0.3774726954431831)
\rput[bl](4.60051918320041,0.51576518419741637){\ududff{$A$}}
\end{pspicture*}
\end{center}
\item 
\begin{enumerate}
\item Pour prouver que le point $H(-1;0;5)$ est le projeté orthogonal de $C$ sur $(AB),$ il suffit de prouver que~:

\begin{itemize}
\item[\textbullet] $H$ appartient à $(AB),$
\item[\textbullet] $(CH)$ est orthogonale à $(AB).$
\end{itemize}

\medskip

Pour le premier point, on utilise la colinéarité~: $\overrightarrow{AB}\begin{pmatrix}-1\\-1\\1\end{pmatrix}$ et $\overrightarrow{AH}\begin{pmatrix}-2\\-2\\2\end{pmatrix}.$ On remarque que $\overrightarrow{AH}=2\overrightarrow{AB},$ donc $\overrightarrow{AH}$ et $\overrightarrow{AB}$ sont colinéaires~; et donc $H$ appartient à $(AB).$

\medskip

Pour le deuxième point, on utilise le produit scalaire~: $\overrightarrow{AB}\begin{pmatrix}-1\\-1\\1\end{pmatrix}$ et $\overrightarrow{CH}\begin{pmatrix}0\\3\\3\end{pmatrix},$ donc
\[\overrightarrow{AB}\cdot \overrightarrow{CH}=(-1)\times 0+(-1)\times 3+1\times 3=0.\]
On en déduit que $(CH)$ est orthogonale à $(AB).$

\medskip

Finalement, comme $H$ appartient à $(AB)$ et que $(CH)$ est orthogonale à $(AB),$ $H$ est bien le projeté orthogonal de $C$ sur $(AB).$

\item L'aire du triangle $ABC$ est 
\[\mathcal{A}_{ABC}=\frac{1}{2}\times\text{base}\times h=\frac{1}{2}\times AB\times h.\]

Or $AB=\sqrt{(-1)^2+(-1)^2+1^2}=\sqrt{3}$ et $CH=\sqrt{0^2+3^2+3^2}=\sqrt{18},$ donc
\[\mathcal{A}_{ABC}=\frac{1}{2}\times \sqrt{3}\times \sqrt{18}=\frac{1}{2}\times \sqrt{54}.\]
\item La formule pour le volume d'un tétraèdre (cas particulier de pyramide) est 
\[\mathcal{V}=\frac{1}{3}\times\mathcal{B}\times h,\]
où $\mathcal{B}$ désigne l'aire de la base et $h$ la hauteur.

\medskip
Le volume du tétraèdre $ABCD$ est donc~:
\[\mathcal{V}=\frac{1}{3}\times \mathcal{A}_{ABC}\times DE=\frac{1}{3}\times \frac{1}{2}\times \sqrt{54}\times \sqrt{24}=\frac{1}{6}\times\sqrt{54\times 24}=\frac{1}{6}\times\sqrt{1296}=\frac{1}{6}\times 36=6.\]
\end{enumerate}
\end{enumerate}

\end{exo}



\begin{exo}

Dans l'espace rapporté à un repère orthonormé de centre O, on considère les points:

$A(2;0;0),$ $B(0;3;0)$ et $C(0;0;1).$





\begin{enumerate}
\item 
	\begin{enumerate}
		\item On calcule les coordonnées des vecteurs~:
\[\overrightarrow{AB}\begin{pmatrix} -2\\3\\0\end{pmatrix}\qquad;\qquad \overrightarrow{AC}\begin{pmatrix} -2\\0\\1\end{pmatrix}.\] 

Le vecteur $\overrightarrow{n}\begin{pmatrix}3\\2\\6\end{pmatrix}$ est orthogonal au plan $(ABC),$ puisqu'il est orthogonal aux vecteurs  directeurs du plan $\overrightarrow{AB}$ et $\overrightarrow{AC}.$ En effet~:
\begin{align*}
\overrightarrow{n}\cdot\overrightarrow{AB}&=3\times (-2)+2\times 3+6\times 0=0,\\
\overrightarrow{n}\cdot\overrightarrow{AC}&=3\times (-2)+2\times 0+6\times 1=0.
\end{align*}
\item Le vecteur $\overrightarrow{n}\begin{pmatrix}3\\2\\6\end{pmatrix}~\textcolor{red}{\begin{matrix}a\\b\\c\end{matrix}}$ étant orthogonal au plan $(ABC),$ ce dernier a une équation de la forme
\begin{align*}ax+by+cz+d&=0\\3x+2y+6z+d&=0.\end{align*}
Et comme $(ABC)$ passe par $A(2;0;0)~:$
\begin{align*}
3\times 2+2\times 0+6\times 0+d&=0\\
6+d&=0\\
d&=-6.\end{align*}

Conclusion~:
\[(ABC):3x+2y+6z-6=0.\]
		
		
	\end{enumerate}
\item  
	\begin{enumerate}
		\item La droite $d$ est orthogonale au plan $(ABC),$ donc elle est dirigée par le vecteur $\overrightarrow{n}\begin{pmatrix}3\\2\\6\end{pmatrix}.$ Elle passe par $O(0;0;0),$ donc sa représentation paramétrique est

\[\begin{cases}x&=x_O+t\times 3\\y&=y_O+t\times 2\\z&=z_O+t\times 6\end{cases}\hspace{1cm}\text{soit}\hspace{1cm}\begin{cases}x&=3t\\y&=2t\\z&=t\end{cases}.\]

		\item On \og injecte \fg~{} la représentation paramétrique de $d$ dans l'équation de $(ABC),$ puis on résout~:
\begin{align*}
3x+2y+6z-6&=0\\
3(3t)+2(2t)+6(6t)-6&=0\\
9t+4t+36t-6&=0\\
49t-6&=0\\
t&=\frac{6}{49}.
\end{align*}

Enfin on remplace $t$ par $\frac{6}{49}$ dans la représentation paramétrique~:
\[\begin{cases}x&=3t=3\times \frac{6}{49}=\frac{18}{49}\\y&=2t=2\times \frac{6}{49}=\frac{12}{49}\\z&=6t=6\times \frac{6}{49}=\frac{36}{49}\end{cases}.\]

Conclusion~: $H\left(\frac{18}{49};\frac{12}{49};\frac{36}{49}\right)$.
		\item \begin{align*}OH&=\sqrt{\left(x_H-x_O\right)^2+\left(x_H-x_O\right)^2+\left(x_H-x_O\right)^2}=
		\sqrt{\left(\frac{18}{49}-0\right)^2+\left(\frac{12}{49}-0\right)^2+\left(\frac{36}{49}-0\right)^2}
		\\&=\sqrt{\frac{324}{2401}+\frac{144}{2401}+\frac{1296}{2401}}=\sqrt{\frac{1764}{2401}}=\frac{\sqrt{1764}}{\sqrt{2401}}=\frac{42}{49}=\frac{6}{7}.
		\end{align*}
	\end{enumerate}
\item  La formule pour le volume d'une pyramide est 
\[\mathcal{V}=\frac{1}{3}\times\mathcal{B}\times h,\]
où $\mathcal{B}$ désigne l'aire de la base et $h$ la hauteur.

\medskip

On fait le calcul avec deux choix de bases et de hauteurs différents~:

\setlength{\columnseprule}{1pt}
\begin{multicols}{2}

On prend comme base le rectangle $OAEB$ et comme hauteur $OC.$ On obtient~:

\begin{align*}\mathcal{V}&=\frac{1}{3}\times\mathcal{A}_{OAEB}\times OC
\\&=\frac{1}{3}\times(3\times 2)\times 1=6.
\end{align*}

\begin{center}
\newrgbcolor{ududff}{0.30196078431372547 0.30196078431372547 1.}
\psset{xunit=1.0cm,yunit=1.0cm,algebraic=true,dimen=middle,dotstyle=o,dotsize=5pt 0,linewidth=2.pt,arrowsize=3pt 2,arrowinset=0.25}
\begin{pspicture*}(1.660508382909681,0.45693888588426146)(10.836495402920503,4.530924824229315)
\pspolygon[linewidth=2.pt,linecolor=blue,linestyle=dotted,fillcolor=blue!10!white,fillstyle=solid,opacity=0.1](2.,1.)(4.,2.)(10.,2.)(8.,1.)
\psline[linewidth=2.pt,linecolor=red](4.,4.)(4.,2.)
%\psline[linewidth=2.pt,linestyle=dotted,linecolor=blue](2.,1.)(4.,2.)
%\psline[linewidth=2.pt,linestyle=dotted,linecolor=blue](4.,2.)(10.,2.)
\psline[linewidth=2.pt,linecolor=blue](10.,2.)(8.,1.)
\psline[linewidth=2.pt,linecolor=blue](8.,1.)(2.,1.)
\rput[tl](5.620270416441322,1.6466630611141162){\blue{$\mathcal{B}$}}
\rput[tl](3.65,2.931790156841537){\red{$h$}}
\psline[linewidth=2.pt](2.,1.)(4.,4.)
\psline[linewidth=2.pt](4.,4.)(8.,1.)
\psline[linewidth=2.pt](4.,4.)(10.,2.)
\psdots[dotstyle=*,linecolor=ududff](4.,4.)
\rput[bl](3.583277447268795,4.074029204975664){\ududff{$C$}}
\psdots[dotstyle=*,linecolor=ududff](4.,2.)
\rput[bl](4.07824770146025,2.189334775554354){\ududff{$O$}}
\psdots[dotstyle=*,linecolor=ududff](2.,1.)
\rput[bl](1.8128069226608978,1.2184315846403462){\ududff{$A$}}
\psdots[dotstyle=*,linecolor=ududff](8.,1.)
\rput[bl](7.976084369929696,1.1993942671714442){\ududff{$E$}}
\psdots[dotstyle=*,linecolor=ududff](10.,2.)
\rput[bl](10.075002704164419,2.2274094104921587){\ududff{$B$}}
\end{pspicture*}
\end{center}

\columnbreak

On prend comme base le triangle $ABC$ et comme hauteur $OH.$ On obtient~:

\begin{align*}\mathcal{V}&=\frac{1}{3}\times\mathcal{A}_{ABC}\times OH
\\&=\frac{1}{3}\times\mathcal{A}_{ABC}\times \frac{6}{7}=\frac{2}{7}\mathcal{A}_{ABC}.
\end{align*}


\begin{center}
\newrgbcolor{ududff}{0.30196078431372547 0.30196078431372547 1.}
\psset{xunit=1.0cm,yunit=1.0cm,algebraic=true,dimen=middle,dotstyle=o,dotsize=5pt 0,linewidth=2.pt,arrowsize=3pt 2,arrowinset=0.25}
\begin{pspicture*}(1.4701352082206574,0.3807896160086535)(10.55093564088696,4.664186046511631)
\pspolygon[linewidth=2.pt,linecolor=ududff,fillcolor=ududff!10!white,fillstyle=solid,opacity=0.1](2.,1.)(4.,4.)(10.,2.)
\rput[tl](5.620270416441322,2.8466630611141162){\blue{$\mathcal{B}$}}
\rput[tl](4.35,2.4){\red{$h$}}
\psline[linewidth=2.pt,linestyle=dotted](4.,4.)(4.,2.)
\psline[linewidth=2.pt,linestyle=dotted](4.,2.)(2.,1.)
\psline[linewidth=2.pt,linestyle=dotted](4.,2.)(10.,2.)
\psline[linewidth=2.pt,linecolor=ududff](2.,1.)(4.,4.)
\psline[linewidth=2.pt,linecolor=ududff](4.,4.)(10.,2.)
\psline[linewidth=2.pt,linecolor=ududff](10.,2.)(2.,1.)
\psline[linewidth=2.pt,linecolor=red](4.,2.)(4.5732179556517005,2.7985289345592226)
\psdots[dotstyle=*,linecolor=ududff](4.,4.)
\rput[bl](4.078247701460246,4.188253109789078){\ududff{$C$}}
\psdots[dotstyle=*,linecolor=ududff](4.,2.)
\rput[bl](3.6165386695511056,2.189334775554355){\ududff{$O$}}
\psdots[dotstyle=*,linecolor=ududff](2.,1.)
\rput[bl](1.6795457003785805,1.3136181719848574){\ududff{$A$}}
\psdots[dotstyle=*,linecolor=ududff](10.,2.)
\rput[bl](10.075002704164406,2.189334775554355){\ududff{$B$}}
\psdots[dotstyle=*,linecolor=ududff](4.5732179556517005,2.7985289345592226)
\rput[bl](4.649367225527309,2.9889021092482437){\ududff{$H$}}
\end{pspicture*}
\end{center}


\end{multicols} 

On compare les deux calculs du volume. On obtient~:
\begin{align*}
\frac{2}{7}\mathcal{A}_{ABC}&=6
\\ \mathcal{A}_{ABC}&=\frac{6}{\frac{2}{7}}
\\ \mathcal{A}_{ABC}&=6\times\frac{7}{2}=21.
\end{align*}


\end{enumerate}

\end{exo}




\begin{exo}



\begin{enumerate}
\item

\begin{itemize}
\item[\textbullet] Un vecteur normal à $P:x+2y-z+1=0$ est $\overrightarrow{n}\begin{pmatrix}1\\2\\-1\end{pmatrix}.$
\item[\textbullet] Un vecteur normal à $P':-x+y+z=0$ est $\overrightarrow{n}\begin{pmatrix}-1\\1\\1\end{pmatrix}.$
\end{itemize}

Les vecteurs $\overrightarrow{n}$ et $\overrightarrow{n}'$ sont orthogonaux, puisque
\[\overrightarrow{n}\cdot \overrightarrow{n}'=1\times (-1)+2\times 1+(-1)\times 1=0,\]
donc $P$ est orthogonal à $P'.$
\item Pour prouver que $P$ et $P'$ se coupent suivant la droite $d,$ il suffit de prouver que cette droite est incluse à la fois dans $P$ et dans $P'.$ On le fait en \og injectant \fg~{} la représentation paramétrique dans les équations de plans~:

\medskip
\begin{itemize}
\item[\textbullet] Pour le plan $P~:$
\[x+2y-z+1=\left(-\frac{1}{3}+t\right)+2\left(-\frac{1}{3}\right)-t+1=-\frac{1}{3}+t-\frac{2}{3}-t+1=0,\]
donc $d\subset P.$
\item[\textbullet] Pour le plan $P'~:$
\[-x+y+z=-\left(-\frac{1}{3}+t\right)+\left(-\frac{1}{3}\right)+t=\frac{1}{3}-t-\frac{1}{3}+t=0,\]
donc $d\subset P'.$
\end{itemize}

\medskip

Conclusion~: $P$ et $P'$ se coupent bien suivant $d.$


\item Pour déterminer les coordonnées du projeté orthogonal $H$ de $A$ sur le plan $P,$ on cherche d'abord la représentation paramétrique de $(AH).$

\medskip

Cette droite est orthogonale à $P,$ donc elle est dirigée par $\overrightarrow{n}\begin{pmatrix}1\\2\\-1\end{pmatrix}$ et elle a pour représentation


\[\begin{cases}x&=x_A+t\times 1\\y&=y_A+t\times 2\\z&=z_A+t\times (-1)\end{cases}\hspace{1cm}\text{soit}\hspace{1cm}\begin{cases}x&=t\\y&=1+2t\\z&=1-t\end{cases}.\]

\medskip

On \og injecte \fg~{} ensuite cette représentation dans l'équation de $P$ et on résout~:

\begin{align*}
x+2y-z+1&=0\\
t+2(1+2t)-(1-t)+1&=0\\
t+2+4t-1+t+1&=0\\
6t+2&=0\\
t=-\frac{2}{6}&=-\frac{1}{3}.
\end{align*}

Enfin on remplace $t$ par $-\frac{1}{3}$ dans la représentation paramétrique~:
\[\begin{cases}x&=-\frac{1}{3}\\y&=1+2t=1+2\times \left(-\frac{1}{3}\right)=1-\frac{2}{3}=\frac{1}{3}\\z&=1-t=1-\left( -\frac{1}{3}\right)=\frac{3}{3}+\frac{1}{3}=\frac{4}{3}\end{cases}.\]

Conclusion~: $H\left(-\frac{1}{3};\frac{1}{3};\frac{4}{3}\right)$.

\item On admet que $K\left(\frac{2}{3};\frac{1}{3};\frac{1}{3}\right)$ est le projeté orthogonal de $A$ sur le plan $P'.$ On note $L$ le projeté orthogonal de $A$ sur $d~;$ la distance du point $A$ à la droite $d$ est donc la longueur $AL.$

\medskip


\begin{center}
\newrgbcolor{ududff}{0.30196078431372547 0.30196078431372547 1.}
\psset{xunit=1.0cm,yunit=1.0cm,algebraic=true,dimen=middle,dotstyle=o,dotsize=5pt 0,linewidth=2.pt,arrowsize=3pt 2,arrowinset=0.25}
\begin{pspicture*}(2.38,0.8)(11.28,7.24)
\pspolygon[linewidth=2.pt,linecolor=green,fillcolor=green!20!white,fillstyle=solid,opacity=0.1](3.,5.)(5.,7.)(5.,3.)(3.,1.)
\pspolygon[linewidth=2.pt,linecolor=green,fillcolor=green!20!white,fillstyle=solid,opacity=0.1](3.,1.)(9.,1.)(11.,3.)(5.,3.)
\psline[linewidth=2.pt,linecolor=green](3.,5.)(5.,7.)
\psline[linewidth=2.pt,linecolor=green](5.,7.)(5.,3.)
\psline[linewidth=2.pt,linecolor=green](5.,3.)(3.,1.)
\psline[linewidth=2.pt,linecolor=green](3.,1.)(3.,5.)
\psline[linewidth=2.pt,linecolor=green](3.,1.)(9.,1.)
\psline[linewidth=2.pt,linecolor=green](9.,1.)(11.,3.)
\psline[linewidth=2.pt,linecolor=green](11.,3.)(5.,3.)
\psline[linewidth=2.pt,linecolor=magenta](5.,3.)(3.,1.)
\psline[linewidth=2.pt,linestyle=dotted,linecolor=red](8.,2.)(4.,2.)
\psline[linewidth=2.pt,linestyle=dotted,linecolor=red](4.,2.)(4.,4.)
\psline[linewidth=2.pt,linestyle=dotted,linecolor=red](4.,4.)(8.,4.)
\psline[linewidth=2.pt,linestyle=dotted,linecolor=red](8.,4.)(8.,2.)
\psline[linewidth=2.pt,linestyle=dotted,linecolor=red](4.,2.)(8.,4.)
\rput[tl](3.6,1.52){\magenta{$d$}}
\rput[tl](9.74,2.8){\green{$P$}}
\rput[tl](4.26,6.08){\green{$P'$}}
\psdots[dotstyle=*,linecolor=ududff](8.,2.)
\rput[bl](8.08,2.2){\ududff{$H$}}
\psdots[dotstyle=*,linecolor=ududff](4.,2.)
\rput[bl](3.7,2.2){\ududff{$L$}}
\psdots[dotstyle=*,linecolor=ududff](4.,4.)
\rput[bl](4.08,4.2){\ududff{$K$}}
\psdots[dotstyle=*,linecolor=ududff](8.,4.)
\rput[bl](8.08,4.2){\ududff{$A$}}
\end{pspicture*}
\end{center}

\medskip

D'après le théorème de Pythagore~:
\[AL=\sqrt{AH^2+HL^2}=\sqrt{AH^2+AK^2}.\]

Or \begin{align*}
AH&=\sqrt{\left(-\frac{1}{3}-0\right)^2+\left(\frac{1}{3}-1\right)^2+\left(\frac{4}{3}-1\right)^2}
=\sqrt{\frac{1}{9}+ \frac{4}{9}+\frac{1}{9}}=\sqrt{\frac{6}{9}},
\\AK&=\sqrt{\left(\frac{2}{3}-0\right)^2+\left(\frac{1}{3}-1\right)^2+\left(\frac{1}{3}-1\right)^2}
=\sqrt{\frac{4}{9}+ \frac{4}{9}+\frac{4}{9}}=\sqrt{\frac{12}{9}},
\end{align*}
donc la distance du point $A$ à la droite $d$ est
\[AL=\sqrt{\frac{6}{9}+\frac{12}{9}}=\sqrt{2}.\]


\end{enumerate}

\end{exo}

\begin{exo}

~{}

\begin{center}
\psset{unit=1cm}
\begin{pspicture}(6,6.5)
\pspolygon(0.5,1.5)(3.5,1)(5.6,2.1)(5.6,2.7)(2.6,5.2)(0.5,4.7)%BCDLKFB
\psline(3.5,1)(3.5,2.1)(0.5,4.7)%CJF
\psline(0.5,4.7)(3.5,2.1)(5.6,2.7)%FJL
\psline[linestyle=dashed](0.5,1.5)(2.6,2.6)(5.6,2.1)%BAD
\psline[linestyle=dashed](2.6,2.6)(2.6,5.2)%%AK
\uput[ul](2.6,2.6){A} \uput[l](0.5,1.5){B} \uput[d](3.5,1){C} 
\uput[r](5.6,2.1){D}  \uput[ul](0.5,4.7){F} \uput[ur](2.6,5.2){K}
 \uput[ur](4.1,3.95){I} \psdots[dotstyle=+,dotangle=45,dotscale=1.5](4.1,3.95)
\uput[dr](3.5,2.1){J} \uput[r](5.6,2.7){L} 
\end{pspicture}
\end{center}

\begin{enumerate}
\item 
	\begin{enumerate}
		\item Les coordonnées sont $F(1;0;1)$ et $I(0;0,5;0,5)$. 
		\item On a $\overrightarrow{FI}\begin{pmatrix} -1\\0,5\\-0,5\end{pmatrix}$ et $\overrightarrow{FJ}\begin{pmatrix} 0\\1\\-0,6\end{pmatrix}.$
		On calcule les produits scalaires~:
			
		\begin{align*}		
		\overrightarrow{FI}\cdot \overrightarrow{n}&=-1\times (-1)+0,5\times 3+(-0,5)\times 5=0,\\
		\overrightarrow{FJ}\cdot \overrightarrow{n}&=0\times (-1)+1\times 3+(-0,6)\times 5=0.
		\end{align*}
		
		On en déduit que $\overrightarrow{n}$ est orthogonal aux vecteurs directeurs $\overrightarrow{FI}$ et $\overrightarrow{FJ}$ du plan $(FIJ),$ et donc aussi au plan $(FIJ).$
		\item Le vecteur $\begin{pmatrix}-1\\3\\5\end{pmatrix}~\textcolor{red}{\begin{matrix}a\\b\\c\end{matrix}}$ est orthogonal au plan $(FIJ),$ donc ce plan a une équation de la forme \[-x +3y+ 5z +d = 0.\]
		$F(1;0;1)\in(FIJ)$ donc $-1+3\times 0+5\times 1+d=0,$ ce qui donne $d=-4.$
		
		\medskip
		
		Finalement $(FIJ):-x +3y+ 5z -4=0.$
\item Pour prouver que $d$ coupe le plan $(FIJ)$ en $M\left(\dfrac{6}{7}~;~\dfrac{3}{7}~;~\dfrac{5}{7}\right),$ il suffit de prouver que~:
\begin{itemize}
\item[i.] $M$ appartient au plan $(FIJ)~;$
\item[ii.] $(MB)$ est orthogonale au plan $(FIJ).$ 
\end{itemize}

\medskip

Pour le point i, il suffit de remplacer $x,$ $y,$ $z$ par les coordonnées de $M$ dans l'équation de $(FIJ)~:$
\[-\frac{6}{7}+3\times \frac{3}{7}+5\times \frac{5}{7}-4=-\frac{6}{7}+\frac{9}{7}+\frac{25}{7}-\frac{28}{7}=0,\] donc $M\in (FIJ).$

Pour le point ii, on utilise les vecteurs~: d'une part $\overrightarrow{MB}\begin{pmatrix} 1-\frac{6}{7}\\0-\frac{3}{7}\\0-\frac{5}{7}\end{pmatrix},$ soit $\overrightarrow{MB}\begin{pmatrix} \frac{1}{7}\\-\frac{3}{7}\\ -\frac{5}{7}\end{pmatrix},$ d'autre part un vecteur normal à $(FIJ)$ est  $\overrightarrow{n}\begin{pmatrix} -1\\3\\5\end{pmatrix}.$

Conclusion~: $\overrightarrow{MB}=-\dfrac{1}{7}\overrightarrow{n},$ donc les vecteurs $\overrightarrow{MB}$ et $\overrightarrow{n}$ sont colinéaires. Et comme $\overrightarrow{n}$ est orthogonal à $(FIJ),$ le vecteur $\overrightarrow{MB}$ (et la droite $(MB)$) l'est aussi.

\medskip

\textbf{Remarque~:} Une autre méthode consistait à trouver la représentation paramétrique de $d,$ puis à injecter cette représentation dans l'équation du plan $(FIJ).$ C'est un peu plus long à faire.
	\end{enumerate}
\item  
	\begin{enumerate}
		\item Les faces $ABFE$ et $CDHG$ sont parallèles, donc le plan $(FIJ)$ les coupe suivant deux segments parallèles. Les segments $\left[FK\right]$ et $\left[JL\right]$ sont donc parallèles. On prouve de même que $\left[FJ\right]$ est parallèle à $\left[KL\right].$
		
		\medskip
		
		Conclusion~: les côtés opposés de $FKLJ$ sont parallèles deux à deux, donc c'est un parallélogramme.
		\item Le point $L$ a pour abscisse 0 et pour ordonnée 1, donc il a des coordonnées de la forme $L(0;1;b).$ De plus les vecteurs  $\overrightarrow{FJ}\begin{pmatrix} 0\\1\\a-1\end{pmatrix}$ et $\overrightarrow{IL}\begin{pmatrix} 0\\0,5\\b-0,5\end{pmatrix}$ sont colinéaires puisque $(IL)$ est parallèle à $(FJ).$ Donc le tableau 

\begin{tabular}{|c|c|}
\hline $1$&$a-1$\\ \hline $0,5$&$b-0,5$\\ \hline
\end{tabular}
est un tableau de proportionnalité. On a donc $b-0,5=(a-1)\times 0,5,$ d'où \[b=0,5+0,5a-0,5=0,5a=\frac{a}{2}.\]
		
		Conclusion~: $L\left(0;1;\frac{a}{2}\right).$
		\item Un parallélogramme est un losange lorsqu'il a deux côtés consécutifs égaux. On calcule donc~:
		\begin{align*}
		FJ&=\sqrt{(1-1)^2+(1-0)^2+\left(a-1\right)^2}=\sqrt{0+1+a^2-2a+1}=\sqrt{a^2-2a+2}\\
		LJ&=\sqrt{(1-0)^2+(0-0)^2+\left(a-\frac{1}{2}a\right)^2}=\sqrt{1+0+\left(\frac{1}{2}a\right)^2}=\sqrt{1+\frac{1}{4}a^2}.
		\end{align*}
		
		Il s'ensuit~:
		\[FKJL~\text{parallélogramme}\iff FJ=LJ\iff a^2-2a+2=1+\frac{1}{4}a^2\iff a^2-2a+2-1-\frac{1}{4}a^2=0\iff \frac{3}{4}a^2-2a+1=0.\]
		
		On aboutit à une équation du second degré. Le discriminant est  $\Delta=(-2)^2-4\times\frac{3}{4}\times 1=1.$
		
		\medskip
		
		$\Delta>0,$ donc il y a deux solutions~:
		\[a_1=\frac{2-\sqrt{1}}{2\times\frac{3}{4}}=\frac{1}{\frac{6}{4}}=\frac{4}{6}=\frac{2}{3}\qquad,\qquad a_2=\frac{2+\sqrt{1}}{2\times\frac{3}{4}}=\frac{3}{\frac{6}{4}}=3\times \frac{4}{6}=\frac{12}{6}=2.\]
		
		On sait que $a\in\left[0;1\right],$ donc seule la valeur $a_1=\frac{2}{3}$ convient.
		
		\medskip
		
		Conclusion~: $FKJL$ est un losange lorsque $a=\frac{2}{3}.$
	\end{enumerate}
\end{enumerate}
\end{exo}


\newpage

\section{Le logarithme népérien}




\begin{exo}

Pour tout $x\in \left]0;+\infty\right[~:$ \[f'(x)=1-2\times \frac{1}{x}=\frac{x}{x}-\frac{2}{x}=\frac{x-2}{x}.\] On a donc le tableau~:

\medskip
\begin{center}
\begin{tikzpicture}[scale=1]
\tkzTabInit{$x$/1,$x-2$/1,$x$/1,$f'(x)$/1,$f(x)$/2}{$0$,$2$,$+\infty$}
\tkzTabLine{,-,z,+,}
\tkzTabLine{z,+,,+,}
\tkzTabLine{d,-,z,+,}

\tkzTabVar{D+/,-/$2-2\ln 2$,+/}
\end{tikzpicture}
\end{center}

\medskip

\danger Ne pas oublier la double-barre en 0~: la fonction $\ln$ n'est pas définie en $0,$ qui est donc \og valeur interdite \fg.

\end{exo}

\begin{exo}

Pour tout $x\in\mathbb{R}~:$

\[g'(x)=2\text{e}^{2x}-8.\] Jusqu'à présent, nous n'aurions pas pu résoudre l'équation $g'(x)=0.$ Avec le logarithme népérien, désormais, nous pouvons le faire~:

\begin{align*}
2\text{e}^{2x}-8&=0\\
\frac{\cancel{2}\text{e}^{2x}}{\cancel{2}}&=\frac{8}{2}\\
\text{e}^{2x}&=4\\
\ln\left(\text{e}^{2x}\right)&=\ln 4\qquad\text{(on prend le logarithme pour \og éliminer \fg~{} l'exponentielle)}\\
2x&=\ln 4\qquad \text{(le logarithme et l'exponentielle s'éliminent)}\\
\frac{\cancel{2}x}{\cancel{2}}&=\frac{\ln 4}{2}\\
x&=\frac{1}{2}\ln 4.
\end{align*}

On a donc le tableau~:

\medskip
\begin{center}
\begin{tikzpicture}[scale=1]
\tkzTabInit{$x$/1,$g'(x)$/1,$g(x)$/2}{$-\infty$,$\frac{1}{2}\ln 4$,$+\infty$}
\tkzTabLine{,-,z,+,}
\tkzTabVar{+/,-/$4-4\ln 4$,+/}
\end{tikzpicture}
\end{center}

\medskip

\textbf{Remarques~:}

\begin{itemize}
\item[\textbullet] $\frac{1}{2}\ln 4 \approx 0,69.$
\item[\textbullet] $g\left(\frac{1}{2}\ln 4\right)=\text{e}^{2\times \frac{1}{2}\ln 4}-8\times \frac{1}{2}\ln 4=\text{e}^{\ln 4}-4\ln 4=4-4\ln 4.$
\item[\textbullet] Pour avoir le signe de la dérivée, on remplace par une valeur de $x$ dans chacune des deux cases. Par exemple, pour avoir le signe dans la case de droite (sur $\left]\frac{1}{2}\ln 4;+\infty\right[$), on calcule (la calculatrice s'avère nécessaire)~:
\[g'(1)=2\text{e}^{2\times 1}-8=2\text{e}^2-8\approx 6,8\qquad \text{signe} ~\oplus\]
\end{itemize}


\end{exo}

\begin{exo}


On utilise la formule pour la dérivée d'un produit avec
\begin{alignat*}{3}
&u(x)=x,&& \hspace{1cm}&&v(x)=\ln x, \\
& u'(x)=1,&& &&v'(x)=\frac{1}{x}.\\
\end{alignat*}
On obtient, pour tout $x\in \left]0;+\infty\right[~:$
\[h'(x)=1\times \ln x+x\times \frac{1}{x}=\ln x+1.\]

On résout l'équation~:

\begin{align*}
\ln x+1&=0\\
\ln x&=-1\\
\text{e}^{\ln x}&=\text{e}^{-1}\qquad\text{(on prend l'exponentielle pour \og éliminer \fg~{} le logarithme)}\\
x&=\text{e}^{-1}\qquad \text{(le logarithme et l'exponentielle s'éliminent)}.
\end{align*}


On a donc le tableau~:

\medskip
\begin{center}
\begin{tikzpicture}[scale=1]
\tkzTabInit{$x$/1,$\ln x+1$/1,$h(x)$/2}{$0$,$\text{e}^{-1}$,$+\infty$}
\tkzTabLine{d,-,z,+,}

\tkzTabVar{D+/,-/$-\text{e}^{-1}$,+/}
\end{tikzpicture}
\end{center}

\medskip

\textbf{Remarques~:}

\begin{itemize}
\item[\textbullet] $h\left(\text{e}^{-1}\right)=\text{e}^{-1}\times \ln \left(\text{e}^{-1}\right)=\text{e}^{-1}\times(-1)=-\text{e}^{-1}.$
\item[\textbullet] Pour avoir le signe de la dérivée, on remplace par une valeur de $x$ dans chacune des deux cases. Par exemple, pour avoir le signe dans la case de droite (sur $\left]\text{e}^{-1};+\infty\right[$), on calcule~:
\[h'(1)=\ln 1+1=1\qquad \text{signe} ~\oplus\] 
\end{itemize}

\end{exo}

\begin{exo}

On pose $f(x)=\ln x$ pour $x\in\left]0;+\infty\right[$ et on calcule les dérivées 1\up{re} et 2\up{de}~:

\medskip
Pour tout $x\in\left]0;+\infty\right[~:$
\begin{align*}
f'(x)&=\frac{1}{x},
\\ f''(x)&=-\frac{1}{x^2}.
\end{align*}

\medskip

Clairement $f''$ est strictement négative sur $\left]0;+\infty\right[,$ donc $f$ est concave sur $\left]0;+\infty\right[.$

\medskip

\textbf{Illustration graphique~:}


\begin{center}
\psset{xunit=1.0cm,yunit=1.0cm,algebraic=true,dimen=middle,dotstyle=o,dotsize=5pt 0,linewidth=2.pt,arrowsize=3pt 2,arrowinset=0.25}
\begin{pspicture*}(-0.72,-2.68)(9.74,2.64)
\multips(0,-2)(0,1.0){6}{\psline[linestyle=dashed,linecap=1,dash=1.5pt 1.5pt,linewidth=0.4pt,linecolor=lightgray]{c-c}(-0.72,0)(9.74,0)}
\multips(0,0)(1.0,0){11}{\psline[linestyle=dashed,linecap=1,dash=1.5pt 1.5pt,linewidth=0.4pt,linecolor=lightgray]{c-c}(0,-2.68)(0,2.64)}
\psaxes[labelFontSize=\scriptstyle,xAxis=true,yAxis=true,Dx=1.,Dy=1.,ticksize=-2pt 0,subticks=2]{->}(0,0)(-0.72,-2.68)(9.74,2.64)
\psplot[linewidth=2.pt,linecolor=blue,plotpoints=200]{8.359999868878054E-8}{9.739999999999998}{ln(x)}
\rput[tl](5.94,1.64){\blue{$y=\ln (x)$}}
\end{pspicture*}
\end{center}

\end{exo}



\begin{exo}

On résout les équations et inéquations~:



\begin{enumerate}
\item \begin{align*}\text{e}^x&=5\\
\ln\left(\text{e}^x\right)&=\ln 5\\
x&=\ln 5.
\end{align*}

Il y a une seule solution~: $x=\ln 5.$
\item \begin{align*}\text{e}^x-4&=0\\
\text{e}^x&=4\\
\ln\left(\text{e}^x\right)&=\ln 4\\
x&=\ln 4.
\end{align*}

Il y a une seule solution~: $x=\ln 4.$

\item \begin{align*}\text{e}^{2x}&\leq 2\\
\ln\left(\text{e}^{2x}\right)&\leq \ln 2\qquad \text{(car la fonction $\ln$ est strictement croissante sur }\left]0;+\infty\right[)\\
\frac{\cancel{2}x}{\cancel{2}}&\leq \frac{\ln 2}{2}\\
x&\leq \frac{\ln 2}{2}.
\end{align*}

Les solutions sont les nombres de l'intervalle $\left]-\infty;\frac{\ln 2}{2}\right].$


\item $\text{e}^{-2x}=-5.$

Il n'y a pas de solution car une exponentielle est strictement positive.

\item On résout dans $\left]0;+\infty\right[$ (c'est-à-dire que les $x$ que l'on considère doivent être strictement positifs -- la raison en est le terme $\ln x$)~:
\begin{align*}\ln x&=3\\
\text{e}^{\ln x}&=\text{e}^{3}\\
x&=\text{e}^{3}.
\end{align*}

Il y a une seule solution~: $x=\text{e}^{3}.$

\item On résout dans $\left]0;+\infty\right[~:$
\begin{align*}2\ln x-3&=0\\
\frac{\cancel{2}\ln x}{\cancel{2}}&=\frac{3}{2}
\\ \ln x&=\frac{3}{2}\\
\text{e}^{\ln x}&=\text{e}^{\frac{3}{2}}\\
x&=\text{e}^{\frac{3}{2}}.
\end{align*}

Il y a une seule solution~: $x=\text{e}^{\frac{3}{2}}.$



\item $2x-3>0$ lorsque $x>\frac{3}{2},$ donc on résout dans $\left]\frac{3}{2};+\infty\right[~:$

\begin{align*}
\ln(2x-3)&= 0\\
\text{e}^{\ln (2x-3)}&=\text{e}^{0}\\
2x-3&= 1\\
\frac{\cancel{2}x}{\cancel{2}}&=\frac{3+1}{2}\\
x&= 2.
\end{align*}

Il y a une seule solution~: $x=2.$


\item On résout dans $\left]0;+\infty\right[~:$
\begin{align*}\ln x&\leq 2\\
\text{e}^{\ln x}&\leq \text{e}^{2}\qquad \text{(car la fonction $\exp$ est strictement croissante sur }\mathbb{R})\\
x&\leq \text{e}^{2}.
\end{align*}

\danger Les solutions ne sont pas les nombres de l'intervalle $\left]-\infty;\text{e}^{2}\right],$ car on résout dans $\left]0;+\infty\right[.$

\medskip

Les solutions sont les nombres de l'intervalle $\left]0;\text{e}^{2}\right].$


\item On note $\left(E_1\right)$ l'équation à résoudre~: \[\text{e}^{2x}-3\text{e}^x-4=0\qquad \left(E_1\right).\] Cette équation se réécrit \[\left(\text{e}^{x}\right)^2-3\text{e}^x-4=0,\] donc en posant $X=\text{e}^x~:$
\[X^2-3X-4=0\qquad \left(E_2\right).\]

\medskip

Nous allons résoudre $\left(E_2\right).$ Les solutions que nous trouverons seront les valeurs possibles de $X.$ Mais attention, c'est $\left(E_1\right)$ que l'on nous a demandé de résoudre, donc il faudra déduire de la résolution de $\left(E_2\right)$ celle de $\left(E_1\right).$

\medskip

Allons-y pour $\left(E_2\right)~:$ c'est une équation habituelle du second degré. Le discriminant est $\Delta=b^2-4ac=(-3)^2-4\times 1\times (-4)=25,$ donc $\left(E_2\right)$ a deux solutions~:
\[X_1=\frac{-b+\sqrt{\Delta}}{2a}=\frac{-(-3)+\sqrt{25}}{2}=\frac{3+5}{2}=4~,~X_2=\frac{-b-\sqrt{\Delta}}{2a}=\frac{3-5}{2}=-1
.\]

Revenons à $\left(E_1\right).$ Puisque  $X=\text{e}^x$ et compte tenu des solutions que l'on vient de trouver pour $\left(E_2\right),$ il y a deux possibilités~:
\[\text{e}^x=4~\text{ou}~\text{e}^x=-1.\]

La première de ces deux équations a pour solution $x=\ln 4.$ La deuxième, elle, n'a pas de solution, car une exponentielle est strictement positive.

\medskip

Conclusion~: $\left(E_1\right)$ a une seule solution~: $x=\ln 4.$
\end{enumerate}



\end{exo}

\begin{exo}

On cherche l'ensemble de définition de la fonction définie par
\[f(x)=\ln\left(-x^2+2x+3\right).\]

Le seul problème concerne le logarithme~: on ne peut calculer $\ln X$ que quand $X>0.$ Donc $f(x)$ est bien défini si, et seulement si, $-x^2+2x+3>0.$

\medskip

Nous sommes ramenés à l'étude du signe de $-x^2+2x+3~:$ le discriminant est $\Delta=2^2-4\times (-1)\times 3=16,$ donc il y a deux racines~:
\[x_1=\frac{-2-\sqrt{16}}{2\times (-1)}=\frac{-2-4}{-2}=\frac{-6}{-2}=3\qquad,\qquad x_2=\frac{-2+4}{-2}=\frac{2}{-2}=-1.\]
On a donc le tableau de signe~:

\begin{center}
\begin{tikzpicture}[scale=0.9]
\tkzTabInit{$x$/1,$-x^2+2x+3$/1}{$-\infty$,$-1$,$3$,$+\infty$}
\tkzTabLine{,-,z,+,z,-,}
\end{tikzpicture}
\end{center}
\medskip

Conclusion~: $-x^2+2x+3$ est strictement positif lorsque $x\in\left]-1;3\right[,$ donc l'ensemble de définition de $f$ est $\left]-1;3\right[.$

\end{exo}

\begin{exo}

Dans chaque question, on utilise la formule 
\[(\ln u)'=\frac{u'}{u}.\]

\begin{enumerate}
\item $f(x)=\ln(2x-1),$ définie pour $x\in\left]\frac{1}{2};+\infty\right[$ (car $2x-1>0\iff x>\frac{1}{2}$).
\medskip

On a \[u(x)=2x-1\qquad ,\qquad u'(x)=2.\] Donc pour tout $x\in \left]\frac{1}{2};+\infty\right[~:$
\[f'(x)=\frac{2}{2x-1}.\]
\item $g(x)=\ln\left(1+\text{e}^{-x}\right),$ définie pour $x\in\mathbb{R}$ (car $1+\text{e}^{-x}>0$ pour tout réel $x$).

\medskip

On a \[u(x)=1+\text{e}^{-x}\qquad ,\qquad u'(x)=-\text{e}^{-x}.\] Donc pour tout $x\in \mathbb{R}~:$
\[g'(x)=\frac{-\text{e}^{-x}}{1+\text{e}^{-x}}.\]
\end{enumerate}


\end{exo}





\begin{exo}

~{}
\begin{enumerate}
\item \begin{align*}
\ln 8&=\ln\left(2^3\right)=3\ln 2\\
\ln\left(\frac{1}{4}\right)&=-\ln 4=-\ln\left(2^2\right)=-2\ln 2\\
\ln 6 -\ln 3+\ln\left(\sqrt{2}\right)&=\ln\left(\frac{6}{3}\right)+\frac{1}{2}\ln 2=\ln 2+\frac{1}{2}\ln 2=\frac{3}{2}\ln 2.
\end{align*}
\item ~{}

\begin{align*}
2\ln a+\ln b&=\ln\left(a^2\right)+\ln b=\ln\left(a^2\times b\right)\\
\ln a-2\ln b&=\ln a-\ln\left(b^2\right)=\ln\left(\frac{a}{b^2}\right)\\
\ln a+\frac{1}{2}\ln b&=\ln a+\ln\left(\sqrt{b}\right)=\ln\left(a\times \sqrt{b}\right).
\end{align*}
\item 

\begin{itemize}
\item[\textbullet] On résout l'équation $\text{e}^{-x}-1=3~:$

\begin{align*}\text{e}^{-x}-1&=3\\
\text{e}^{-x}&=1+3\\
\ln\left(\text{e}^{-x}\right)&=\ln 4\\
-x&=\ln 4\\
x&=-\ln 4\\
x&=\ln\left(\frac{1}{4}\right).
\end{align*}

Il y a une seule solution~: $x=\ln\left(\frac{1}{4}\right).$

\item[\textbullet] On résout l'équation $\text{e}^{2x}+1=10~:$

\begin{align*}\text{e}^{2x}+1&=10\\
\text{e}^{2x}&=10-1\\
\ln\left(\text{e}^{2x}\right)&=\ln 9\\
2x&=\ln 9\\
x&=\frac{1}{2}\ln 9\\
x&=\ln\left(\sqrt{9}\right).
\end{align*}

Il y a une seule solution~: $x=\ln\left(\sqrt{9}\right).$
\end{itemize}

\end{enumerate}

\end{exo}

\begin{exo}


On résout l'équation \[\left(\ln x\right)^2-3\ln x-4=0\qquad (E)\] dans $\left]0;+\infty\right[.$

\medskip

On pose $X=\ln x~;$ l'équation se réécrit alors $X^2-3X-4=0.$

Le discriminant est $\Delta=(-3)^2-4\times 1\times (-4)=25,$ donc il y a deux racines~:
\[X_1=\frac{3-\sqrt{25}}{2\times 1}=\frac{3-5}{2}=-1\qquad,\qquad X_2=\frac{3+5}{2}=4.\]

Revenons à $(E).$ Il y a deux possibilités~:
\begin{align*}\ln x&=-1&\qquad&\text{ou}&\qquad& \ln x=4
\\ x&=\text{e}^{-1}&\qquad&\text{ou}&\qquad&x=\text{e}^4
\end{align*}

Il y a donc deux solutions (qui sont bien dans $\left]0;+\infty\right[$)~: $x=\text{e}^{-1}$ et $x=\text{e}^4.$

\end{exo}

\begin{exo}

D'après le codage, $\ln a$ et $\ln b$ sont deux nombres opposés~: 

\[\ln b=-\ln a,\]

donc\[\ln b=\ln\left(\frac{1}{a}\right).\]

On en déduit
\begin{align*}
\text{e}^{\ln b}&=\text{e}^{\ln \left(\frac{1}{a}\right)}\\
b&=\frac{1}{a}.
\end{align*}



\begin{center}
\psset{xunit=2cm,yunit=2cm,algebraic=true,dimen=middle,dotstyle=o,dotsize=3pt 0,linewidth=0.8pt,arrowsize=3pt 2,arrowinset=0.25}
\begin{pspicture*}(-0.58,-1.24)(3.17,1.54)
\psaxes[labelFontSize=\scriptstyle,xAxis=true,yAxis=true,Dx=10,Dy=10,ticksize=-2pt 0,subticks=2]{->}(0,0)(-0.58,-1.24)(3.17,1.54)
\psplot[linewidth=1.6pt,plotpoints=200]{0.2}{3.1735615974827756}{ln(x)}
\rput[tl](2.3,1.25){$y=\ln x$}
\psline[linewidth=1.2pt,linestyle=dashed,dash=2pt 2pt](0,-0.88)(0.42,-0.88)
\psline[linewidth=1.2pt,linestyle=dashed,dash=2pt 2pt](0.42,-0.88)(0.42,0)
%\psline[linewidth=1.2pt](0,0.88)(0,0)
\psline[linewidth=0.8pt](0.05,0.46)(-0.05,0.46)
\psline[linewidth=0.8pt](0.05,0.42)(-0.05,0.42)
%\psline[linewidth=1.2pt](0,0)(0,-0.88)
\psline[linewidth=0.8pt](0.05,-0.42)(-0.05,-0.42)
\psline[linewidth=0.8pt](0.05,-0.46)(-0.05,-0.46)
\psline[linewidth=1.2pt,linestyle=dashed,dash=2pt 2pt](2.4,0.88)(2.4,0)
\psline[linewidth=1.2pt,linestyle=dashed,dash=2pt 2pt](0,0.88)(2.4,0.88)
\rput[tl](2.36,-0.1){$a$}
\rput[tl](-0.48,0.95){$\ln a$}
\rput[tl](-0.48,-0.8){$\ln b$}
\rput[tl](0.36,0.2){$b$}
\begin{scriptsize}
\psdots[dotstyle=*](2.4,0.88)
\psdots[dotstyle=*](0,0.88)
\psdots[dotstyle=*](0,-0.88)
\psdots[dotstyle=*](0.42,-0.88)
\psdots[dotstyle=*](0,0)
\end{scriptsize}
\end{pspicture*}
\end{center}

\end{exo}


\begin{exo}

Voici comment on construit le produit de deux réels $x,$ $y,$ à partir de la courbe $\Gamma$ de la fonction $\ln~:$

\begin{itemize}
\item[\textbullet] Étape 1~: on place $x,$ $y$ sur l'axe des abscisses et leurs images $\ln x,$ $\ln y$ sur l'axe des ordonnées.
\item[\textbullet] Étape 2~: on construit $z=\ln x+\ln y$ sur l'axe des ordonnées, en reportant une longueur au compas.
\item[\textbullet] Étape 3~: on a aussi $z=\ln (x\times y),$ donc on obtient $x\times y$ en reportant sur l'axe des abscisses.
\end{itemize}



\begin{center}
\newrgbcolor{xfqqff}{0.4980392156862745 0. 1.}
\psset{xunit=1.0cm,yunit=1.0cm,algebraic=true,dimen=middle,dotstyle=o,dotsize=5pt 0,linewidth=2.pt,arrowsize=3pt 2,arrowinset=0.25}
\begin{pspicture*}(-2.,-0.86)(19.54,3.68)
\psaxes[labelFontSize=\scriptstyle,xAxis=true,yAxis=true,labels=none,Dx=20.,Dy=20.,ticksize=-2pt 0,subticks=2]{->}(0,0)(-2.,-0.86)(13.0,3.68)
\psplot[linewidth=2.pt,linecolor=blue,plotpoints=200]{2.007999983167282E-7}{12.5}{ln(x)}
\psline[linewidth=2.pt,linestyle=dotted,linecolor=red](2.64,0.)(2.64,0.9707789171582248)
\psline[linewidth=2.pt,linestyle=dotted,linecolor=red](0.,0.9707789171582248)(2.64,0.9707789171582248)
\psline[linewidth=2.pt,linestyle=dotted,linecolor=red](0.,1.5129270120532565)(4.54,1.5129270120532565)
\psline[linewidth=2.pt,linestyle=dotted,linecolor=red](4.54,1.5129270120532565)(4.54,0.)
\psline[linewidth=2.pt,linestyle=dashed,dash=2pt 2pt,linecolor=green](0.,2.4829270120532567)(11.97626784477273,2.4829270120532567)
\psline[linewidth=1.6pt,linecolor=xfqqff](0.,0.)(0.,0.97)
\psline[linewidth=2.pt,linecolor=xfqqff](-0.12,0.4353894585791126)(0.12,0.4353894585791126)
\psline[linewidth=2.pt,linecolor=xfqqff](-0.12,0.5353894585791127)(0.12,0.5353894585791127)
\psline[linewidth=1.6pt,linecolor=xfqqff](0.,1.5129270120532565)(0.,2.4829270120532567)
\psline[linewidth=2.pt,linecolor=xfqqff](-0.12,1.9479270120532564)(0.12,1.9479270120532564)
\psline[linewidth=2.pt,linecolor=xfqqff](-0.12,2.0479270120532562)(0.12,2.0479270120532562)
\psline[linewidth=2.pt,linecolor=xfqqff](14.,1.2)(16.,1.2)
\psline[linewidth=2.pt,linecolor=xfqqff](14.95,1.32)(14.95,1.08)
\psline[linewidth=2.pt,linecolor=xfqqff](15.05,1.32)(15.05,1.08)
\psline[linewidth=2.pt,linestyle=dashed,dash=2pt 2pt,linecolor=green](11.97626784477273,2.4829270120532567)(11.97626784477273,0.)
\rput[tl](2.45,-0.2){\red{$x$}}
\rput[tl](4.4,-0.2){\red{$y$}}
\rput[tl](-0.8,1.15){\red{$\ln x$}}
\rput[tl](-0.8,1.75){\red{$\ln y$}}
\psline[linewidth=2.pt,linestyle=dotted,linecolor=red](14.,2.)(16.,2.)
\psline[linewidth=2.pt,linestyle=dashed,dash=2pt 2pt,linecolor=green](14.,0.4)(16.,0.4)
\rput[tl](-2,2.8){\xfqqff{$z=\ln x + \ln y$}}
\rput[tl](-1.8,2.4){\xfqqff{$=\ln(x\times y)$}}
\rput[tl](11.6,-0.2){\green{$x\times y$}}
\rput[tl](16.6,2.2){\red{étape 1}}
\rput[tl](16.6,1.4){\xfqqff{étape 2}}
\rput[tl](16.6,0.6){\green{étape 3}}
\rput[tl](8.06,1.9){\blue{$\Gamma$}}
\psline[linewidth=2.pt](13.8,2.4)(18.,2.4)
\psline[linewidth=2.pt](18.,2.4)(18.,0.)
\psline[linewidth=2.pt](18.,0.)(13.8,0.)
\psline[linewidth=2.pt](13.8,0.)(13.8,2.4)
\psdots[dotsize=6pt 0,dotstyle=*,linecolor=xfqqff](0.,2.5)
\psdots[dotsize=6pt 0,dotstyle=*,linecolor=green](12,0.)
\psdots[dotsize=6pt 0,dotstyle=*,linecolor=xfqqff](16.,1.2)
\psdots[dotsize=6pt 0,dotstyle=*,linecolor=green](16.,0.4)
\end{pspicture*}
\end{center}


\end{exo}

\begin{exo}

La fonction $f$ est définie sur $\left]0;+\infty\right[$ par
\[f(x)=x^2\ln x-x^2.\]

\begin{enumerate}
\item $f\left(\text{e}^{\frac{1}{2}}\right)=\left(\text{e}^{\frac{1}{2}}\right)^2\times \ln\left(\text{e}^{\frac{1}{2}}\right)-\left(\text{e}^{\frac{1}{2}}\right)^2=\text{e}^{1}\times \frac{1}{2}-\text{e}^{1}=-\frac{1}{2}\text{e}.$
\item On utilise la formule pour la dérivée d'un produit avec



\begin{alignat*}{3}
&u(x)=x^2,&& \hspace{1cm}&&v(x)=\ln x, \\
& u'(x)=2x,&& &&v'(x)=\frac{1}{x}.\\
\end{alignat*}
On obtient, pour tout $x\in \left]0;+\infty\right[~:$
\begin{align*}f'(x)&=2x\times \ln x+x^2\times \frac{1}{x}-2x\\
&=2x\ln x+x-2x\\
&=2x\ln x-x\\
&=x(2\ln x-1).
\end{align*}
\item 
On résout l'équation~:

\[2\ln x-1=0\iff 2\ln x=1\iff \ln x=\frac{1}{2}\iff
\text{e}^{\ln x}=\text{e}^{\frac{1}{2}}\iff
x=\text{e}^{\frac{1}{2}}.\]


On a donc le tableau~:

\medskip
\begin{center}
\begin{tikzpicture}[scale=1]
\tkzTabInit{$x$/1,$x$/1,$2\ln x-1$/1,$f'(x)$/1,$f(x)$/2}{$0$,$\text{e}^{\frac{1}{2}}$,$+\infty$}
\tkzTabLine{z,+,,+,}
\tkzTabLine{d,-,z,+,}
\tkzTabLine{d,-,z,+,}
\tkzTabVar{D+/,-/$-\frac{1}{2}\text{e}$,+/}
\end{tikzpicture}
\end{center}
\end{enumerate}



\end{exo}

\begin{exo}

On résout l'inéquation d’inconnue $n\in\mathbb{N}~:$

\[2^n\geq 10^{100}.\]

\begin{align*}
2^n\geq 10^{100}&\iff\ln\left(2^n\right)\geq \ln \left(10^{100}\right)\qquad\text{(par stricte croissance de la fonction $\ln$)}\\
&\iff n\ln 2 \geq 100\ln 10\\
&\iff n\geq \frac{100\ln 10}{\ln 2} \qquad\qquad~~\text{(\danger car $\ln 2>0,$ donc $\geq$ reste $\geq$)}.
\end{align*}

Avec la calculatrice on trouve $\frac{100\ln 10}{\ln 2}\approx 332,19,$ donc l'ensemble des solutions est $\llbracket 333;+\infty \llbracket.$


\end{exo}

\begin{exo}



\begin{enumerate}
\item $100~\%-8~\%=92~\%=0,92,$ donc une diminution de $8~\%$ revient à faire une multiplication par $0,92.$ La suite $\left(v_n\right)_{n\in\mathbb{N}}$ est donc géométrique de raison $q=0,92$ et l'on a, pour tout $n\in\mathbb{N}~:$
\[v_n=v_0\times q^n=10\times 0,92^n.\]
\item On cherche la demi-vie de l'iode 131, c'est-à-dire le nombre de jours après lequel il restera la moitié de l'iode 131 ingérée au départ.

\medskip

$10\div 2=5,$ donc il s'agit de résoudre l'inéquation $v_n\leq 5.$

\begin{align*}
10\times 0,92^n\leq 5&\iff 0,92^n\leq \frac{5}{10}\\&\iff 0,92^n\leq 0,5\\
&\iff \ln\left(0,92^n\right)\leq \ln \left(0,5\right)\qquad\text{(par stricte croissance de la fonction $\ln$)}\\
&\iff n\ln 0,92 \leq \ln 0,5\\
&\iff n\geq \frac{\ln 0,5}{\ln 0,92} \qquad\qquad~~\text{(\danger car $\ln 0,92<0,$ donc $\leq$ devient $\geq$)}.
\end{align*}

Avec la calculatrice on trouve $\frac{\ln 0,5}{\ln 0,92}\approx 8,31,$ donc la demi-vie de l'iode 131 est de 8 jours et quelques.

\end{enumerate}



\end{exo}

\begin{exo}


	\begin{enumerate}
		\item On répète 10 épreuves indépendantes de Bernoulli de paramètre $7~\%=0,07,$ donc $X$ suit la loi binomiale de paramètres $n=10,$ $p=0,07.$
		\item La probabilité pour qu'exactement deux personnes portent le gène ZXC est
		\[P(X=2)=\binom{10}{2}\times 0,07^2\times (1-0,07)^{10-2}
		=\binom{10}{2}\times 0,07^2\times 0,93^8
		\approx 0,12.
		\]

\item On cherche le nombre minimum de personnes à tester dans ce pays pour que la probabilité qu'au moins une de ces personnes porte le gène ZXC, soit supérieure à 99\,\%.

\medskip

Le test auprès de $n$ personnes revient à répéter $n$ épreuves indépendantes de Bernoulli de paramètre $0,07.$ Donc si l'on note $X_n$ le nombre de porteurs du gène ZXC parmi elles, $X_n$ suit la loi binomiale de paramètres $n$ et $p=0,07.$

\medskip

La probabilité qu'au moins une de ces personnes porte le gène ZXC est 
\[P\left(X_n\geq 1\right)=1-P\left(X_n=0\right)=1-\underbrace{\binom{10}{0}}_{=1}\times \underbrace{0,07^0}_{=1}\times 0,93^n=1-0,93^n.\]

Nous sommes ainsi ramenés à résoudre dans $\mathbb{N}$ l'équation $1-0,93^n\geq 0,99.$


\begin{align*}
1-0,93^n\geq 0,99&\iff 1-0,99\geq 0,93^n
\\ &\iff 0,01\geq 0,93^n
\\&\iff 
\ln 0,01\geq \ln \left(0,93^n\right)\qquad\text{(par stricte croissance de la fonction $\ln$)}\\
&\iff \ln 0,01\geq n\ln 0,93\\
&\iff\frac{\ln 0,01}{\ln 0,93}\leq n \qquad\qquad~~\text{(\danger car $\ln 0,93<0,$ donc $\geq$ devient $\leq$)}.
\end{align*}

Avec la calculatrice on trouve $\frac{\ln 0,01}{\ln 0,93}\approx 63,46,$ donc il faut interroger au moins 64 personnes.

\end{enumerate}

\end{exo}

\begin{exo}

On abrégera \og croissance comparée \fg~{} en C.C.

\begin{enumerate}
\item $\lim\limits_{x\to +\infty}(-x+1)=-\infty$ donc \[\lim\limits_{x\to +\infty}\text{e}^{-x+1}=\text{\og}~\text{e}^{-\infty}~\text{ \fg}=0.\]
\item$\lim\limits_{x\to -\infty}(-x+1)=+\infty$ donc \[\lim\limits_{x\to -\infty}\text{e}^{-x+1}=\text{\og}~\text{e}^{+\infty}~\text{ \fg}=+\infty.\]
\item $\lim\limits_{x\to +\infty}\frac{x}{\text{e}^x}=0$ par croissance comparée.
\item $\lim\limits_{x\to -\infty}\frac{x}{\text{e}^x}=\lim\limits_{x\to -\infty}\left(x\times \text{e}^{-x}\right)=\text{\og}~(-\infty)\times\left(+\infty\right)~\text{ \fg}=-\infty.$
\item $\lim\limits_{x\to -\infty}\left(\text{e}^x-x+1\right)=\text{\og}~0-(-\infty)+ 1~\text{\fg}=+\infty.$
\item  On met $\text{e}^{x}$ en facteur~: pour tout $x\in\mathbb{R},$
\[\text{e}^x-x+1=\text{e}^x\left(1-\frac{x}{\text{e}^x}+\frac{1}{\text{e}^x}\right)=
\text{e}^x\left(1-x\text{e}^{-x}+\text{e}^{-x}\right).\]

\[
\left.
    \begin{array}{ll}
        \lim\limits_{x\to +\infty}\text{e}^x&=  +\infty\\
        \lim\limits_{x\to +\infty}\left(1-x\text{e}^{-x}+\text{e}^{-x}\right) &= 1-0+0=1~\text{(par C.C. pour le terme }x\text{e}^{-x})
    \end{array}
\right \}\implies \lim\limits_{x\to +\infty}\left(\text{e}^x-x+1\right)=\text{\og}~{+\infty}\times 1~\text{\fg}=+\infty.
\]
\item ~{} \[
\left.
    \begin{array}{ll}
        \lim\limits_{x\to -\infty}\left(\text{e}^{x}+1\right)&= 0+1=1 \\
        \lim\limits_{x\to -\infty}\left(\text{e}^{x}+2\right)&= 0+2=2
    \end{array}
\right \}\implies \lim\limits_{x\to -\infty}\frac{\text{e}^{x}+1}{\text{e}^{x}+2}=\frac{0+1}{0+2}=\frac{1}{2}.
\]

\item $\lim\limits_{x\to -\infty}\frac{\text{e}^{x}+1}{\text{e}^{x}+2}=\frac{1}{2}$ d'après la question précédente, donc \[\lim\limits_{x\to -\infty}\ln\left(\frac{\text{e}^x+1}{\text{e}^x+2}\right)=\ln\left(\frac{1}{2}\right)=-\ln 2\] par continuité de la fonction $\ln$ en $\frac{1}{2}.$ 
\item $\lim\limits_{x\to +\infty}\frac{\ln x}{x}=0$ par croissance comparée.
\item Si $x>1,$ alors $0<\frac{1}{x+1}<\frac{1}{x}$ (car deux nombres strictement positifs sont rangés en sens contraire de leurs inverses), donc en multipliant par $\ln x$ (qui est strictement positif, puisqu'on a supposé $x>1$)~:
\[0<\frac{\ln x}{x+1}<\frac{\ln x}{x}.\]

Or $\lim\limits_{x\to +\infty}\frac{\ln x}{x}=0$ et $\lim\limits_{x\to +\infty}0=0,$ donc d'après le théorème des gendarmes~:
\[\lim\limits_{x\to +\infty}\frac{\ln x}{x+1}=0.\]

\item $\lim\limits_{x\to 0,~x>0}\frac{\ln x}{x}=\lim\limits_{x\to 0,~x>0}\left(\ln x\times\frac{1}{x}\right)=\text{\og}~(-\infty)\times (+\infty)~\text{ \fg}=-\infty.$
\item On développe~: pour tout $x>0,$
\[x\left(\ln x+1\right)=x\ln x+x.\]

\[
\left.
    \begin{array}{ll}
        \lim\limits_{x\to 0,~x>0}x\ln x&= 0~(\text{par C.C.}) \\
        \lim\limits_{x\to 0,~x>0}x&= 0
    \end{array}
\right \}\implies \lim\limits_{x\to 0,~x>0}x(\ln x+1)=0+0=0.
\]

\item D'après le point 6, $\lim\limits_{x\to +\infty}\left(\text{e}^{x}-x+1\right)=+\infty,$ donc

\[\lim\limits_{x\to +\infty}\ln\left(\text{e}^{x}-x+1\right)=\text{\og}~\ln(+\infty)~\text{ \fg}=+\infty.\]
\item $\lim\limits_{x\to +\infty}x\text{e}^{-x}=0$ par C.C., donc $\lim\limits_{x\to +\infty}\ln\left(1+x\text{e}^{-x}\right)=\ln(1+0)=0,$ par continuité de la fonction $x\mapsto \ln(1+x)$ en $0.$
\end{enumerate}


\end{exo}

\begin{exo}

Soit $q>1$ et soit $n\in\mathbb{N}.$ On peut écrire
\[q^n=\text{e}^{\ln\left(q^n\right)}=\text{e}^{n\ln q}.\]

Or $q>1\implies \ln q>0,$ donc $\lim\limits_{n\to +\infty}\left(n\ln q\right)=+\infty,$ et donc
\[\lim\limits_{n\to +\infty}q^n=\lim\limits_{n\to +\infty}\text{e}^{n\ln q}=\text{\og}~\text{e}^{+\infty}~\text{ \fg}=+\infty.\]




\end{exo}

\begin{exo}

Les fonctions $f$ et $g$ sont définies sur $\left]0;+\infty\right[$ par
\[f(x)=\ln x\hspace{2cm} g(x)=\left(\ln x\right)^2.\]

On note $C$ et $C'$ leurs courbes représentatives respectives, tracées ci-dessous.


\begin{center}
\psset{xunit=1.5cm,yunit=1.5cm,algebraic=true,dimen=middle,dotstyle=o,dotsize=5pt 0,linewidth=2.pt,arrowsize=3pt 2,arrowinset=0.25}
\begin{pspicture*}(-0.3817498172327687,-1.2839722274510572)(3.7692989053959343,1.820690252077334)
\psaxes[labelFontSize=\scriptstyle,xAxis=true,yAxis=true,Dx=1.,Dy=1.,ticksize=-2pt 0,subticks=2]{->}(0,0)(-0.3817498172327687,-1.2839722274510572)(3.7692989053959343,1.820690252077334)
\psplot[linewidth=2.pt,plotpoints=200]{1.3785437799784155E-6}{3.7692989053959343}{ln(x)}
\psplot[linewidth=2.pt,linestyle=dashed,dash=2pt 2pt,plotpoints=200]{1.3785437799784155E-6}{3.7692989053959343}{ln(x)^(2.0)}
\rput[tl](0.6186414261708245,-0.6975359813179167){$C$}
\rput[tl](0.5726464264741076,0.7168102593561283){$C'$}
\end{pspicture*}
\end{center}

\begin{enumerate}
\item On utilise la formule $\left(u^n\right)'=n\times u'\times u^{n-1},$ avec
\[u(x)=\ln x,\qquad u'(x)=\frac{1}{x},\qquad n=2.\]

Pour tout $x\in\left]0;+\infty\right[~:$
\[g'(x)=2\times\frac{1}{x}\times\ln x.\] 

On a donc le tableau~:

\medskip
\begin{center}
\begin{tikzpicture}[scale=1]
\tkzTabInit{$x$/1,$2\times\frac{1}{x}$/1,$\ln x$/1,$g'(x)$/1,$g(x)$/2}{$0$,$1$,$+\infty$}
\tkzTabLine{d,+,,+,}
\tkzTabLine{d,-,z,+,}
\tkzTabLine{d,-,z,+,}

\tkzTabVar{D+/$+\infty$,-/$0$,+/$+\infty$}
\end{tikzpicture}
\end{center}

\medskip

\[g(1)=(\ln 1)^2=0^2=0.\]

On calcule les limites en $+\infty$ et à droite en $0~:$

\begin{itemize}
\item[\textbullet] $\lim\limits_{x\to +\infty}\ln x=+\infty,$ donc $\lim\limits_{x\to +\infty}\left(\ln x\right)^2=\text{\og}~\left(+\infty\right)^2~\text{ \fg}=+\infty.$
\item[\textbullet] $\lim\limits_{x\to 0,~x>0}\ln x=-\infty,$ donc $\lim\limits_{x\to 0,~x>0}\left(\ln x\right)^2=\text{\og}~\left(-\infty\right)^2~\text{ \fg}=+\infty.$
\end{itemize}


\item Étudier les positions relatives de $C$ et $C'$ revient à étudier le signe de leur différence. Pour tout $x\in\left]0;+\infty\right[:~$
\[\ln x-(\ln x)^2=1\times\ln x-\ln x\times\ln x=\ln x(1-\ln x).\] On résout les équations~:

\begin{itemize}
\item[\textbullet] $\ln x=0\iff \text{e}^{\ln x}=\text{e}^0\iff x=1,$
\item[\textbullet] $1-\ln x=0\iff 1=\ln x\iff \text{e}^1=\text{e}^{\ln x}\iff \text{e}^{1}=x.$
\end{itemize}

\medskip

On a donc le tableau~:

\medskip
\begin{center}
\begin{tikzpicture}[scale=1.2]
\tkzTabInit{$x$/1,$\ln x$/1,$1-\ln x$/1,$\ln x(1-\ln x)$/1}{$0$,$1$,$\text{e}^1$,$+\infty$}
\tkzTabLine{d,-,z,+,,+,}
\tkzTabLine{d,+,,+,z,-,}
\tkzTabLine{d,-,z,+,z,-,}
\end{tikzpicture}
\end{center}

\medskip

Conclusion~:

\begin{itemize}
\item[\textbullet] les courbes $C$ et $C'$ se coupent aux points d'abscisses $1$ et $\text{e}^1~;$
\item[\textbullet] la courbe $C$ est au dessus de $C'$ sur l'intervalle $\left]1;\text{e}^1\right[~;$
\item[\textbullet] la courbe $C$ est en dessous de $C'$ sur les intervalles $\left]0;
1\right[$ et $\left]\text{e}^1;+\infty\right[.$
\end{itemize}



\item Pour tout réel $x \in\left[ 1;\text{e}\right],$ on note $M$ (respectivement $N$) le point de $C$ (resp. $C'$) d'abscisse $x.$


\begin{center}
\newrgbcolor{ccqqqq}{0.8 0. 0.}
\psset{xunit=4.0cm,yunit=4.0cm,algebraic=true,dimen=middle,dotstyle=o,dotsize=5pt 0,linewidth=2.pt,arrowsize=3pt 2,arrowinset=0.25}
\begin{pspicture*}(-0.36104147549967824,-0.20709880681088047)(2.6335761997331653,0.9511967090810663)
\psaxes[labelFontSize=\scriptstyle,xAxis=true,yAxis=true,labels=none,Dx=10,Dy=10,ticksize=-2pt 0,subticks=2]{->}(0,0)(-0.36104147549967824,-0.20709880681088047)(2.6335761997331653,0.9511967090810663)
\psplot[linewidth=2.pt,linecolor=ccqqqq,plotpoints=200]{1.6098970942798202E-6}{2.6335761997331653}{ln(x)}
\psplot[linewidth=2.pt,linecolor=blue,plotpoints=200]{1.6098970942798202E-6}{2.6335761997331653}{ln(x)^(2.0)}
\psline[linewidth=2.pt,linestyle=dashed,dash=2pt 2pt](1.85667,0.6187845606016078)(0.,0.6187845606016078)
\psline[linewidth=2.pt,linestyle=dashed,dash=2pt 2pt](1.85667,0.38289433243892484)(0.,0.38289433243892484)
\rput[tl](1.85,-0.06584325609235037){$x$}
\psline[linewidth=2.pt,linestyle=dashed,dash=2pt 2pt](1.8566706707812488,0.)(1.85667,0.38289433243892484)
\psline[linewidth=2.pt,linecolor=green]{->}(1.85667,0.49604077827179993)(1.85667,0.6187845606016078)
\psline[linewidth=2.pt,linecolor=green]{->}(1.85667,0.49604077827179993)(1.85667,0.38289433243892484)
\rput[tl](-0.20566036970929485,0.65){\ccqqqq{$\ln x$}}
\rput[tl](-0.3045392552122661,0.45){\blue{$(\ln x)^2$}}
\psline[linewidth=2.pt,linecolor=green]{->}(0.,0.5)(0.,0.6187845606016078)
\psline[linewidth=2.pt,linecolor=green]{->}(0.,0.5)(0.,0.38289433243892484)
\rput[tl](2.2,0.88998597043637){\ccqqqq{$C$}}
\rput[tl](2.256894731150418,0.6310174607857315){\blue{$C'$}}
\psdots[dotsize=4pt 0,dotstyle=*,linecolor=darkgray](1.8566706707812488,0.)
\psdots[dotsize=4pt 0,dotstyle=*,linecolor=blue](1.85667,0.38289433243892484)
\rput[bl](1.9178814094259449,0.3249637675622496){\blue{$N$}}
\psdots[dotsize=4pt 0,dotstyle=*,linecolor=ccqqqq](1.85667,0.6187845606016078)
\rput[bl](1.851962152423964,0.7){\ccqqqq{$M$}}
\psdots[dotsize=4pt 0,dotstyle=*,linecolor=ccqqqq](0.,0.6187845606016078)
\psdots[dotsize=4pt 0,dotstyle=*,linecolor=blue](0.,0.38289433243892484)
\end{pspicture*}
\end{center}

La longueur $MN$ est la longueur verte. Comme $y_M=f(x)=\ln x$ et $y_N=g(x)=(\ln x)^2~:$
\[MN=y_M-y_N=\ln x-(\ln x )^2.\]

Étudier la valeur de $x$ pour laquelle la longueur $MN$ est maximale revient donc à déterminer la valeur de $x$ pour laquelle la fonction $h,$ définie sur $\left[1;\text{e}^1\right]$ par $h(x)=\ln x-(\ln x)^2,$ atteint son maximum. Pour résoudre le problème, on utilise la dérivation~: pour tout $x\in \left[1;\text{e}^1\right],$
\begin{align*}h'(x)&=\frac{1}{x}-2\times\frac{1}{x}\times\ln x\\
&=\frac{1-2\ln x}{x}.
\end{align*}

Or \[1-2\ln x=0\iff 1=2\ln x\iff \frac{1}{2}=\ln x\iff \text{e}^{\frac{1}{2}}=\text{e}^{\ln x}\iff \text{e}^{\frac{1}{2}}=x.\]

On a donc le tableau (il est inutile ici de compléter l'extrémité des flèches)~:

\medskip
\begin{center}
\begin{tikzpicture}[scale=1]
\tkzTabInit{$x$/1,$1-2\ln x$/1,$x$/1,$h'(x)$/1,$h(x)$/2}{$1$,$\text{e}^{\frac{1}{2}}$,$\text{e}^{1}$}
\tkzTabLine{,+,z,-,}
\tkzTabLine{,+,,+,}
\tkzTabLine{,+,z,-,}
\tkzTabVar{-/,+/,-/}
\end{tikzpicture}
\end{center}

\medskip

Conclusion~: $h$ atteint son maximum pour $x=\text{e}^{\frac{1}{2}},$ donc la longueur $MN$ est maximale lorsque $x=\text{e}^{\frac{1}{2}}.$


\end{enumerate}
\end{exo}



\begin{exo}

\begin{enumerate}

\item Soit $u$ la fonction définie sur $\left]0;+\infty\right[$ par
\[u(x)=x^3-1+2\ln x.\]
\begin{enumerate}
\item Pour tout $x\in \left]0;+\infty\right[~:$
\[u'(x)=3x^2+2\times\frac{1}{x}.\] Clairement $u'$ est strictement positive. On a donc le tableau~:

\medskip
\begin{center}
\begin{tikzpicture}[scale=1]
\tkzTabInit{$x$/1,$u'(x)$/1,$u(x)$/2}{$0$,$+\infty$}
\tkzTabLine{d,}
\tkzTabVar{D-/,+/}
\tkzTabVal[draw]{1}{2}{0.6}{$1$}{$0$}
\end{tikzpicture}
\end{center}

\medskip

\item $u(1)=1^3-1+2\ln 1=1-1+2\times 0=0.$ On a donc le tableau de signes~:

\medskip
\begin{center}
\begin{tikzpicture}[scale=1]
\tkzTabInit{$x$/1,$u(x)$/1}{$0$,$1$,$+\infty$}
\tkzTabLine{d,-,0,+,}
\end{tikzpicture}
\end{center}

\medskip

\end{enumerate}
\item La fonction $f$ est définie sur $\left]0;+\infty\right[$ par \[f(x)=x-\frac{\ln x}{x^2}.\]
\begin{enumerate}
\item On utilise la formule pour la dérivée d'un quotient $\left(\frac{u}{v}\right)'=\frac{u'v-uv'}{v^2},$ avec
\begin{alignat*}{3}
&u(x)=\ln x,&& \hspace{1cm}&&v(x)=x^2, \\
& u'(x)=\frac{1}{x},&& &&v'(x)=2x.\\
\end{alignat*}
On obtient, pour tout $x\in \left]0;+\infty\right[~:$
\begin{align*}
f'(x)&=1-\frac{\frac{1}{x}\times x^2-\ln x\times 2x}{\left(x^2\right)^2}\\
&=1-\frac{x-\ln x\times 2x}{x^4}\\
&=1-\frac{\cancel{x}\left(1-2\ln x\right)}{\cancel{x}\times x^3}\\
&=\frac{x^3}{x^3}-\frac{1-2\ln x}{x^3}\\
&=\frac{x^3-1+2\ln x}{x^3}\\
&=\frac{u(x)}{x^3}.\end{align*}


\item On a étudié le signe de $u$ dans la question 1.(b). On a donc le tableau~:

\medskip
\begin{center}
\begin{tikzpicture}[scale=1]
\tkzTabInit{$x$/1,$u(x)$/1,$x^3$/1,$f'(x)=\frac{u(x)}{x^3}$/1,$f(x)$/2}{$0$,$1$,$+\infty$}
\tkzTabLine{d,-,z,+,}
\tkzTabLine{z,+,,+,}
\tkzTabLine{d,-,z,+,}
\tkzTabVar{D+/$+\infty$,-/$1$,+/$+\infty$}
\end{tikzpicture}
\end{center}

\medskip

\[f(1)=1-\frac{\ln 1}{1^2}=1-\frac{0}{1}=1.\]

\item On commence par calculer $\lim\limits_{x\to 0,~x>0}f(x).$ Pour cela on écrit $f(x)=x-\ln x\times \frac{1}{x^2}.$

\medskip

On a

\[\left.
    \begin{array}{ll}
        \lim\limits_{x\to 0,~x>0}\ln x&= -\infty\\
        \lim\limits_{x\to 0,~x>0}\frac{1}{x^2}&= +\infty
    \end{array}
\right \}\implies \lim\limits_{x\to 0,~x>0}\ln x\times \frac{1}{x^2}= -\infty,\]

et donc

\[\lim\limits_{x\to 0,~x>0}\left(x-\ln x\times \frac{1}{x^2}\right)=\og~ 0-(-\infty)~\fg=+\infty.\]

\medskip

Ensuite on calcule $\lim\limits_{x\to +\infty}f(x).$

\medskip

\[\left.
    \begin{array}{ll}
        \lim\limits_{x\to +\infty}x&= +\infty\\
        \lim\limits_{x\to +\infty}\frac{\ln x}{x^2}&= 0\qquad \text{par croissance comparée}~
    \end{array}
\right \}\implies \lim\limits_{x\to +\infty}\left(x-\frac{\ln x}{x^2}\right)= +\infty.\]


\end{enumerate}
\end{enumerate}

\end{exo}


\begin{exo}



\begin{enumerate}
\item On considère la fonction $g$ définie sur l'intervalle $\left]0;+\infty\right[$ par
\[g(x) =2\ln x+x-2.\]
\begin{enumerate}
\item  ~{}
\begin{itemize}
\item[\textbullet] \[\left.
    \begin{array}{ll}
        \lim\limits_{x\to 0,~x>0}2\ln x&= -\infty\\
        \lim\limits_{x\to 0,~x>0}(x-2)&= 0-2=-2
    \end{array}
\right \}\implies \lim\limits_{x\to 0,~x>0}g(x)= -\infty.\]
\item[\textbullet] \[\left.
    \begin{array}{ll}
        \lim\limits_{x\to +\infty}2\ln x&= +\infty\\
        \lim\limits_{x\to +\infty}(x-2)&= +\infty
    \end{array}
\right \}\implies \lim\limits_{x\to +\infty}g(x)= +\infty.\]
\end{itemize}

\item  Pour tout $x\in\left]0;+\infty\right[~:$
\[g'(x)=2\times\frac{1}{x}+1=\frac{2}{x}+1.\]

Clairement $g'$ est strictement positive sur $\left]0;+\infty\right[,$ donc on a le tableau~:

\medskip
\begin{center}
\begin{tikzpicture}[scale=1]
\tkzTabInit{$x$/1,$g'(x)$/1,$g(x)$/2}{$0$,$+\infty$}
\tkzTabLine{d,+,}
\tkzTabVar{D-/$-\infty$,+/$+\infty$}
\tkzTabVal[draw]{1}{2}{0.6}{$\alpha$}{$0$}
\end{tikzpicture}
\end{center}

\item \begin{itemize}
\item[\textbullet] La fonction $g$ est continue et strictement croissante sur $\left]0;+\infty\right[~;$
\item[\textbullet] $\lim\limits_{x\to 0,~x>0}g(x)= -\infty\qquad,\qquad \lim\limits_{x\to +\infty}g(x)= +\infty~;$
\item[\textbullet] $0\in\left]-\infty;+\infty\right[.$
\end{itemize}

D'après le théorème de la bijection, l'équation $g(x)=0$ a exactement une solution  $\alpha$ dans $\left]0;+\infty\right[.$

\medskip

\`A l'aide de la calculatrice, on obtient l'encadrement
\[1,37<\alpha<1,38.\]

\item On déduit des questions précédentes le tableau de signe de la fonction $g~:$ 

\medskip
\begin{center}
\begin{tikzpicture}[scale=1]
\tkzTabInit{$x$/1,$g(x)$/1}{$0$,$\alpha$,$+\infty$}
\tkzTabLine{d,-,z,+,}
\end{tikzpicture}
\end{center}

\end{enumerate}
\item On considère la fonction $f$ définie sur l'intervalle $\left]0;+\infty\right[$ par:
\[f(x) = \frac{x-2}{x}\ln x.\]
On note $\mathcal{C}_f$ sa courbe représentative dans un repère orthonormé.
\begin{enumerate}
\item ~{} \[\left.
    \begin{array}{ll}
        \lim\limits_{x\to 0,~x>0}\frac{x-2}{x}&=\og~ \frac{0-2}{0^+}~\fg=\og~ \frac{-2}{0^+}~\fg= -\infty\\
        \lim\limits_{x\to 0,~x>0}\ln x&= -\infty
    \end{array}
\right \}\implies \lim\limits_{x\to 0,~x>0}f(x)= +\infty.\]

On en déduit que la droite d'équation $x=0$ est asymptote (verticale) à $\mathcal{C}_f.$
\item On écrit, pour $x>0~:$
\[\frac{x-2}{x}=\frac{x}{x}-\frac{2}{x}=1-2\times\frac{1}{x}.\] On peut dès lors calculer la limite~:

\[\left.
    \begin{array}{ll}
        \lim\limits_{x\to +\infty}\frac{x-2}{x}&= 1-2\times 0=1\\
        \lim\limits_{x\to +\infty}\ln x&= +\infty
    \end{array}
\right \}\implies \lim\limits_{x\to +\infty}f(x)= +\infty.\]

\item Pour tout $x\in\left]0;+\infty\right[,$ $f(x)=\left(1-2\times\frac{1}{x}\right)\ln x.$

\medskip

On utilise la formule pour la dérivée d'un produit, avec~:

\begin{alignat*}{3}
&u(x)=1-2\times\frac{1}{x},&& \hspace{1cm}&&v(x)=\ln x, \\
& u'(x)=-2\times\left(-\frac{1}{x^2}\right)=\frac{2}{x^2},&& &&v'(x)=\frac{1}{x}.\\
\end{alignat*}

On obtient, pour tout $x\in \left]0;+\infty\right[~:$

\begin{align*}
f'(x)&=\frac{2}{x^2}\times \ln x+\left(1-\frac{2}{x}\right)\times\frac{1}{x}\\
&=\frac{2\ln x}{x^2}+\frac{1\textcolor{red}{\times x}}{x\textcolor{red}{\times x}}-\frac{2}{x}\times \frac{1}{x}\\
&=\frac{2\ln x}{x^2}+ \frac{x}{x^2}-\frac{2}{x^2}\\
&=\frac{2\ln x+x-2}{x^2}\\
&=\frac{g(x)}{x^2}.
\end{align*}

\item Comme $f'(x)=\frac{g(x)}{x^2}$ pour tout $x>0$ et que $x^2>0,$ $f'$ et $g$ sont du même signe.

\medskip

On a étudié le signe de $g$ dans la question 1.(d)~; on en déduit le signe de $f'$ et les variations de $f~:$

\medskip
\begin{center}
\begin{tikzpicture}[scale=1]
\tkzTabInit{$x$/1,$f'(x)$/1,$f(x)$/2}{$0$,$\alpha$,$+\infty$}
\tkzTabLine{d,-,z,+,}
\tkzTabVar{D+/$+\infty$,-/$f(\alpha)$,+/$+\infty$}
\end{tikzpicture}
\end{center}

\end{enumerate}
\item Pour étudier la position relative de la courbe $\mathcal{C}_f$ et de la courbe représentative $\Gamma$ de la fonction $\ln$ sur l'intervalle $\left]0;+\infty\right[,$ on étudie le signe de la différence~: pour tout $x>0,$
\[f(x)-\ln x=\frac{x-2}{x}\ln x-\ln x=\ln x\left(\frac{x-2}{x}-1\right)=
\ln x\left(\frac{x-2}{x}-\frac{x}{x}\right)=\ln x\times \frac{-2}{x}.\]

Or 
\[\ln x=0\iff\text{e}^{\ln x}=\text{e}^0\iff x=1,\] donc on obtient~:

\begin{center}
\hspace*{-1cm}
\begin{tikzpicture}[scale=1.2]
\tkzTabInit{$x$/1,$\ln x$/1,$\frac{-2}{x}$/1,$\ln x\times \frac{-2}{x}$/1,Positions relatives des courbes/3}{$-\infty$,$1$,$+\infty$}
\tkzTabLine{d,-,z,+,}
\tkzTabLine{d,-,,-,}
\tkzTabLine{d,+,z,-,}
\tkzTabLine{d,\mathcal{C}_f\text{ au-dessus de }\Gamma,\scriptsize{\Longstack{S\\e\\ \\c\\o\\u\\p\\e\\n\\t}},\mathcal{C}_f\text{ en-dessous de }\Gamma,}
\end{tikzpicture}
\end{center}
\end{enumerate}

\end{exo}

\begin{exo}

Soit $f$ la fonction définie sur $\left[0;+\infty\right[$ par \[f(x)=\ln (1+x).\] On note $(C)$ sa courbe représentative et $(T)$ sa tangente au point de coordonnées $(0;0).$

\begin{enumerate}
\item \begin{enumerate}
\item Pour avoir l'équation de la tangente, il faut d'abord calculer la dérivée. Pour cela, on utilise la formule $(\ln u)'=\frac{u'}{u},$ avec \[u(x)=1+x,\qquad u'(x)=1.\] Pour tout $x\in \left[0;+\infty\right[~:$
\[f'(x)=\frac{1}{1+x}.\]

L'équation de $T$ est \[y=f'(0)(x-0)+f(0).\]

On calcule donc~: $f(0)=\ln (1+0)=\ln 1=0$ et $f'(0)=\frac{1}{1+0}=1,$ et l'on obtient~:

\begin{align*}
T:y&=f'(0)(x-0)+f(0)\\
T:y&=1(x-0)+0\\
T:y&=x.
\end{align*}

Pour tracer la courbe $(C),$ il suffit de translater la courbe de la fonction $\ln$ de 1 carreau vers la gauche\footnote{En effet, en posant $f(x)=\ln(1+x)$ et $g(x)=\ln x,$ les fonctions $f $et $g$ prennent les mêmes valeurs, mais $g$ a \og un temps de retard \fg~{} sur $f.$ Par exemple, $f(2)= \ln(1+2)=\ln(3)=g(3),$ ou encore $f(5)=\ln(5+1)=\ln 6=g(6).$}. Quant à la droite d'équation $y=x,$ c'est la première bissectrice -- on l'a déjà tracée maintes fois.

\medskip


\begin{center}
\newrgbcolor{ccqqqq}{0.8 0. 0.}
\psset{xunit=1.0cm,yunit=1.0cm,algebraic=true,dimen=middle,dotstyle=o,dotsize=5pt 0,linewidth=2.pt,arrowsize=3pt 2,arrowinset=0.25}
\begin{pspicture*}(-0.64,-0.7)(6.54,5.84)
\multips(0,-2)(0,1.0){9}{\psline[linestyle=dashed,linecap=1,dash=1.5pt 1.5pt,linewidth=0.4pt,linecolor=lightgray]{c-c}(-0.64,0)(6.54,0)}
\multips(0,0)(1.0,0){8}{\psline[linestyle=dashed,linecap=1,dash=1.5pt 1.5pt,linewidth=0.4pt,linecolor=lightgray]{c-c}(0,-2.88)(0,5.84)}
\psaxes[labelFontSize=\scriptstyle,xAxis=true,yAxis=true,Dx=1.,Dy=1.,ticksize=-2pt 0,subticks=2]{->}(0,0)(-0.64,-2.88)(6.54,5.84)
\psplot[linewidth=2.pt,linecolor=ccqqqq,plotpoints=200]{0}{6.540000000000001}{ln(1.0+x)}
\psplot[linewidth=2.pt,linecolor=blue]{0}{6.54}{(-0.--1.*x)/1.}
\rput[tl](2.04,2.85){\blue{$(T)$}}
\rput[tl](3.5,2.0){\ccqqqq{$(C)$}}
\end{pspicture*}
\end{center}

\item Pour étudier la convexité, on utilise la dérivée seconde. On sait déjà que $f'(x)=\frac{1}{1+x}$ pour tout $x\in\left[0;+\infty\right[.$ On utilise donc la formule $\left(\frac{1}{u}\right)'=-\frac{u'}{u^2},$ avec
\[u(x)=1+x,\qquad u'(x)=1.\] Pour tout $x\in\left[0;+\infty\right[~:$
\[f''(x)=-\frac{1}{(x+1)^2}.\]

\medskip

La dérivée seconde est clairement strictement négative sur $\left[0;+\infty\right[,$ donc $f$ est concave sur cet intervalle. D'après une propriété du cours de début d'année, la tangente $(T)$ est au dessus de $(C)$ sur $\left[0;+\infty\right[.$ Autrement dit, pour tout $x\in \left[0;+\infty\right[~:$
\[\ln(1+x)\leq x.\]
\end{enumerate}
\item On définit une suite $(u_n)_{n\in\mathbb{N}}$ par $u_0=5$ et la relation de récurrence \[u_{n+1}=\ln\left(1+u_n\right).\]
\begin{enumerate}
\item Prouver que $(u_n)_{n\in\mathbb{N}}$ est à termes positifs revient à prouver que $u_n\geq 0$ pour tout $n\in\mathbb{N}.$ On fait une démonstration par récurrence~: pour tout $n\in\mathbb{N},$ on note 
\[\mathcal{P}_n~:~u_n\geq 0.\]

\begin{itemize}
\item[\textbullet] \textbf{Initialisation.} $u_0=5\geq 0$ donc $\mathcal{P}_0$ est vraie.


\item[\textbullet] \textbf{Hérédité.} Soit $k\in\mathbb{N}$ tel que $\mathcal{P}_k$ soit vraie. On a donc \[u_k\geq 0.\] Donc~:
\begin{align*}
1+u_k&\geq 1+0\\
1+u_k&\geq 1\\
\ln\left(1+u_k\right)&\geq \ln 1\qquad \text{(car la fonction $\ln$ est strictement croissante sur $\left]0;+\infty\right[$)}\\
u_{k+1}&\geq 0.
\end{align*}

La propriété $\mathcal{P}_{k+1}$ est donc vraie.


\item[\textbullet] \textbf{Conclusion.} $\mathcal{P}_0$ est vraie et $\mathcal{P}_n$ est héréditaire, donc elle est vraie pour tout $n\in \mathbb{N}.$ La suite $(u_n)_{n\in\mathbb{N}}$ est donc bien à termes positifs.

\end{itemize}
\item Pour tout $n\in\mathbb{N}~:$
\[u_{n+1}-u_n=\ln\left(1+u_n\right)-u_n.\]

\medskip



On sait d'après la question 1.(b) que $\ln(1+x)\leq x$ pour tout $x\in\left[0;+\infty\right[.$ Donc \[\ln(1+x)-x\leq 0,\] et en appliquant ce résultat avec $x=u_n$ (ce qui est licite, puisqu'on a vu que tous les $u_n$ étaient positifs dans la question précédente)~:
\[u_{n+1}-u_n=\ln\left(1+u_n\right)-u_n\leq 0.\] La suite $(u_n)_{n\in\mathbb{N}}$ est donc décroissante. 
Comme de plus elle est minorée par $0$ (une nouvelle fois parce qu'on a prouvé qu'elle était à termes positifs), elle converge\footnote{Toute suite décroissante minorée converge, c'est le théorème de limite monotone.}. De plus, sa limite $\ell$ est un nombre réel positif.
\item On fait le raisonnement désormais classique~:

\begin{itemize}
\item[\textbullet] D'une part, $\lim\limits_{n\to +\infty}u_{n+1}=\lim\limits_{n\to +\infty}u_n=\ell.$
\item[\textbullet] D'autre part, $\lim\limits_{n\to +\infty}u_{n+1}=\lim\limits_{n\to +\infty}f\left(u_n\right)=f(\ell)=\ln(1+\ell)$ car $f$ est continue sur $\left[0;+\infty\right[.$
\end{itemize}

On en déduit $\ell=\ln(1+\ell),$ et donc $\ell=0.$

\medskip

\textbf{Remarque~:} Je suis allé un peu vite en besogne dans la conclusion ci-dessus. Pourquoi a-t-on $\ell =0~?$ Parce que $\ell$ est solution de l'équation $x=\ln(1+x),$ et donc les courbes $(T)$ et $(C)$ se coupent au point d'abscisse $\ell.$ Il semble clair sur le graphique et compte tenu de la concavité de $f$ que cela ne se produit que lorsque $\ell=0.$ Mais raisonner à partir du graphique est interdit et il faudrait une preuve plus rigoureuse. Je m'arrête là cependant et je vous laisse chercher cette preuve plus rigoureuse (indication~: poser $h(x)=x-\ln(1+x)$ pour $x\in \left[0;+\infty\right[$).

\end{enumerate}
\end{enumerate}

\end{exo}



\section{Équations différentielles}





\begin{exo}

À chaque fois, $C$ désigne une constante réelle. Les primitives de $f$ sont notées $F,$ celles $g$ sont notée $G,$ etc.

\begin{enumerate}
\item $f(x)=2x-3,~x\in\mathbb{R}.$
\medskip \[F(x)=x^2-3x+C.\]
\item $g(x)=x^2,~x\in\mathbb{R}.$

\medskip \[G(x)=\frac{1}{3}x^3+C.\]
\item $h(x)=6x^3-3x^2+4x-1,~x\in\mathbb{R}.$

\medskip \[H(x)=6\times\frac{1}{4}x^4-3\times\frac{1}{3}x^3+4\times\frac{1}{2}x^2-x+C=\frac{3}{2}x^4-x^3+2x^2-x+C.\]

\item $i(x)=\text{e}^{x}+x,~x\in\mathbb{R}.$

\medskip \[I(x)=\text{e}^{x}+\frac{1}{2}x^2+C.\]

\item $j(x)=\text{e}^{-2x},~x\in\mathbb{R}.$

\medskip \[J(x)=\frac{1}{-2}\text{e}^{-2x}+C=-\frac{1}{2}\text{e}^{-2x}+C.\]

\item $k(x)=3\text{e}^{0,1x},~x\in\mathbb{R}.$

\medskip \[K(x)=3\times \frac{1}{0,1}\text{e}^{0,1x}+C=30\text{e}^{0,1x}+C.\]

\item $\ell(x)=\frac{2x}{x^2+1},~x\in\mathbb{R}.$
\medskip

On reconnaît la formule $\frac{u'}{u},$ avec $u(x)=x^2+1,$ donc les primitives sont de la forme \[L(x)=\ln\left(x^2+1\right)+C.\]

\item $m(x)=\frac{\text{e}^{x}+1}{\text{e}^{x}+x},~x\in\mathbb{R}.$

On reconnaît encore la formule $\frac{u'}{u}~;$ les primitives sont donc de la forme \[M(x)=\ln\left(\text{e}^{x}+x\right)+C.\]
\end{enumerate}

\end{exo}


\begin{exo}

\medskip

On utilise la formule pour la dérivée d'un produit avec

\begin{alignat*}{3}
&u(x)=x,&& \hspace{1cm}&&v(x)=\ln x, \\
& u'(x)=1,&& &&v'(x)=\frac{1}{x}.\\
\end{alignat*}

On obtient, pour tout $x\in \left]0;+\infty\right[~:$

\begin{align*}
F'(x)&=1\times \ln x+x\times \frac{1}{x}-1\\
&=\ln x+1-1\\
&=\ln x.
\end{align*}

La fonction $F$ est donc bien une primitive de la fonction $\ln.$ 


\end{exo}


\begin{exo}


\begin{enumerate}
\item Les solutions de l'équation différentielle $y'(t)=t-1$ sont les fonctions de la forme
\[y(t)=\frac{1}{2}t^2-t+C,\]
où $C$ est une constante.

\medskip

De plus, on a les équivalences
\[y(0)=-5\iff \frac{1}{2}\times 0^2-0+C=-5\iff C=-5.\]

Conclusion~: l'unique solution de l'équation différentielle $y'(t)=t-1$ vérifiant de plus la condition initiale $y(0)=-5$ est
\[y(t)=\frac{1}{2}t^2-t-5.\]


\item Les solutions de l'équation différentielle $z'(x)=x^2$ sont les fonctions de la forme
\[z(x)=\frac{1}{3}x^3+C,\]
où $C$ est une constante.

\end{enumerate}

\end{exo}


\begin{exo}

\begin{enumerate}
\item D'après le théorème 3 du cours, l'équation différentielle \[(E)\qquad y'=3y\] a pour solutions les fonctions de la forme $y(x)=C\text{e}^{3x},$ où $C$ est une constante.
\item On a les équivalences~:
\[y(0)=4\iff C\text{e}^{3\times 0}=4\iff C\times 1=4\iff C=4.\]

\medskip

Conclusion~: la seule solution de $(E)$ vérifiant la condition initiale $y(0)=4$ est la fonction définie par
\[y(x)=4\text{e}^{3x}.\]
\end{enumerate}

\end{exo}
\begin{exo}

\begin{enumerate}
\item L'équation différentielle \[(E)\qquad y'+2y=0\] se réécrit $y'=-2y,$ donc d'après le théorème 3 du cours, ses solutions sont les fonctions de la forme $y(x)=C\text{e}^{-2x},$ où $C$ est une constante.
\item On a les équivalences~:

\[y(1)=10\iff C\text{e}^{-2\times 1}=10\iff C\underbrace{\text{e}^{-2}\times\text{e}^{2}}_{=1}=10\times\text{e}^{2}\iff C=10\text{e}^{2}.\]

Conclusion~: la seule solution de $(E)$ vérifiant la condition initiale $y(1)=10$ est la fonction définie par
\[y(x)=10\text{e}^{2}\text{e}^{-2x}=10\text{e}^{-2x+2}.\]
\end{enumerate}

\end{exo}




\begin{exo}






\begin{enumerate}
\item La fonction $f$ est solution de l'équation différentielle $y'=-0,12y$ donc $f$ est définie  par
\[f(x)=C\text{e}^{-0,12x},\] où $C$ est une constante.

On sait de plus que $f(0)=\np{1013,25},$ donc $C\text{e}^{-0,12\times 0}=\np{1013,25},$ et ainsi $C=\np{1013,25}.$

Conclusion~: \[f(x)=\np{1013,25}\text{e}^{-0,12x}.\]
\item La pression atmosphérique à 150~m d'altitude est\[f(0,150)=\np{1013,25}\text{e}^{-0,12\times 0,150}\approx 995,17~\text{hPa}.\]
\item On résout l'équation $f(x)=700~:$
\begin{align*}\np{1013,25}\text{e}^{-0,12x}=700 &\iff \text{e}^{-0,12x}=\frac{700}{\np{1013,25}}\iff
\ln\left( \text{e}^{-0,12x}\right)=\ln\left(\frac{700}{\np{1013,25}}\right)\\ &\iff
-0,12x=\ln\left(\frac{700}{\np{1013,25}}\right)\iff x=-\frac{\ln\left(\frac{700}{\np{1013,25}}\right)}{-0,12}\iff x\approx 3,082.
\end{align*}

Conclusion~: c'est à \np{3082}~m d'altitude que la pression atmosphérique est égale à 700 hPa.
\end{enumerate}

\end{exo}

\begin{exo}

\begin{enumerate}
\item D'après le théorème 4 du cours avec $a=2$ et $b=6,$ les solutions de l'équation différentielle \[(E)\qquad y'=2y+6\] sont les fonctions de la forme
\[y(x)=C\text{e}^{2x}-\frac{6}{2}=C\text{e}^{2x}-3,\] où $C$ est une constante.
\item On a les équivalences~:
\[y(0)=2\iff C\text{e}^{2\times 0}-3=2\iff C-3=2\iff C=5.\]

Conclusion~: la seule solution de $(E)$ vérifiant la condition initiale $y(0)=2$ est la fonction définie par
\[y(x)=5\text{e}^{2x}-3.\]
\end{enumerate}

\end{exo}

\begin{exo}

\begin{enumerate}
\item L'équation différentielle \[(E)\qquad y'+y=1\] se réécrit \[y'=-y+1.\] D'après le théorème 4 du cours, ses solutions sont de la forme
\[y(x)=C\text{e}^{-x}-\frac{1}{-1}=C\text{e}^{-x}+1,\] où $C$ est une constante.
\item On a les équivalences~:
\[y(2)=10\iff C\text{e}^{-2}+1=10\iff C\text{e}^{-2}=9\iff C=\frac{9}{\text{e}^{-2}}=9\text{e}^{2}.\]

Conclusion~: la seule solution de $(E)$ vérifiant la condition initiale $y(0)=2$ est la fonction définie par
\[y(x)=9\text{e}^{2}\times \text{e}^{-x}+1=9\text{e}^{-x+2}+1.\]

\end{enumerate}

\end{exo}


\begin{exo}

Un corps de masse $m$ est lâché en chute libre sans vitesse initiale. Pour $t\geq 0,$ sa vitesse $v$ est solution de l'équation différentielle \[(E)~:~y'(t)+\frac{k}{m}y(t)=g,\] où $k$ est le coefficient de frottement et $g$ l'accélération de la pesanteur.

\begin{enumerate}
\item L'équation différentielle $ y'(t)+\frac{k}{m}y(t)=g$ se réécrit $ y'(t)=-\frac{k}{m}y(t)+g,$ donc  d'après le théorème 4 du cours, la vitesse est une fonction de la forme
\[v(t)=C\text{e}^{-\frac{k}{m}t}-\frac{g}{-\frac{k}{m}}=C\text{e}^{-\frac{k}{m}t}+\frac{mg}{k}.\]

De plus, le corps est lâché sans vitesse initiale, donc $v(0)=0.$ On a ainsi les équivalences~:
\[C\text{e}^{-\frac{k}{m}\times 0}+\frac{mg}{k}=0\iff C+\frac{mg}{k}=0\iff C=-\frac{mg}{k}.\]

Conclusion~: la vitesse du corps au temps $t$ est
\[v(t)=-\frac{mg}{k}\text{e}^{-\frac{k}{m}t}+\frac{mg}{k}.\]

\item Pour tout $t\geq 0~:$

\[v'(t)=-\frac{mg}{k}\times\left(-\frac{k}{m}\right)\text{e}^{-\frac{k}{m}t}=g\text{e}^{-\frac{k}{m}t}. \]

Clairement $v'$ est strictement positive, donc $v$ est strictement croissante\footnote{C'est bien normal pour une masse en chute libre~!}~:

\medskip
\begin{center}
\begin{tikzpicture}[scale=1]
\tkzTabInit{$t$/1,$v'(t)$/1,$v(t)$/2}{$0$,$+\infty$}
\tkzTabLine{,+,}
\tkzTabVar{-/$0$,+/$\frac{mg}{k}$}
\end{tikzpicture}
\end{center}
\medskip

$-\frac{k}{m}<0,$ donc $\lim\limits_{t\to +\infty}\text{e}^{-\frac{k}{m}t}=0.$ On en déduit
\[\lim\limits_{t\to +\infty}v(t)=\frac{mg}{k}.\]
\item Allure du graphe de $v~:$

\begin{center}
\psset{xunit=1.0cm,yunit=1.0cm,algebraic=true,dimen=middle,dotstyle=o,dotsize=5pt 0,linewidth=2.pt,arrowsize=3pt 2,arrowinset=0.25}
\begin{pspicture*}(-1.26,-0.62)(10.74,6.02)
\psaxes[labelFontSize=\scriptstyle,xAxis=true,yAxis=true,Dx=20.,Dy=10.,ticksize=-2pt 0,subticks=2]{->}(0,0)(-1.26,-0.62)(10.74,6.02)
\psplot[linewidth=2.pt,linecolor=red,plotpoints=200]{0}{10.4}{-4.0*EXP(-0.25*x)+4.0}
\psplot[linewidth=2.pt,linestyle=dotted,linecolor=blue]{0}{10.74}{(--4.-0.*x)/1.}
\rput[tl](10.06,-0.14){{$t$}}
\rput[tl](-1.02,5.42){\red{$v(t)$}}
\rput[tl](-0.82,4.3){\blue{$\frac{mg}{k}$}}
\end{pspicture*}
\end{center}

\end{enumerate}

\medskip

\textbf{Remarque pour les élèves qui font la spécialité sciences physiques~:}

%On note $z(t)$ la hauteur de la masse au temps $t,$  on a donc $v(t)=-z'(t)$ (on ajoute un \og $-$ \fg , car la vitesse est positive, mais la hauteur diminue au cours du temps). On a alors également $v'(t)=-z''(t).$

\medskip

\begin{multicols}{2}

D'après la deuxième loi de Newton~: 
\begin{equation}\label{N}\sum \overrightarrow{F}=m\overrightarrow{a},
\end{equation} où l'on somme sur toutes les forces qui s'appliquent sur la masse. Ces dernières sont au nombre de deux~:

\begin{itemize}
\item[\textbullet] le poids, dirigé vers le bas, d'intensité $mg~;$
\item[\textbullet] les forces de frottement, dirigées vers le haut (car elles freinent la chute), et que l'on suppose proportionnelles à la vitesse de la masse, avec un coefficient de proportionnalité égal à $k.$
\end{itemize}




\medskip

Quant à l'accélération, on sait que c'est la dérivée de la vitesse~: $a=v'.$ Donc en prenant comme sens positif le sens de haut en bas et en appliquant (\ref{N}), on obtient~:

\[mg-kv=mv'.\]

En transposant et en divisant par $m,$ on aboutit finalement à l'équation $(E)~:$
\[v'+\frac{k}{m}v=g.\] %c'est-à-dire l'équation $(E).$

\begin{center}
\psset{xunit=1.0cm,yunit=1.0cm,algebraic=true,dimen=middle,dotstyle=o,dotsize=5pt 0,linewidth=2.pt,arrowsize=3pt 2,arrowinset=0.25}
\begin{pspicture*}(1.56,-2.2)(6.06,6.44)
\psline[linewidth=2.pt,linecolor=red]{->}(3.,3.)(3.,0.)
\rput[tl](3.38,3.02){masse}
\psline[linewidth=2.pt,linecolor=blue]{->}(3.,3.)(3.,5.)
\rput[tl](3.22,1.66){\red{$\overrightarrow{F}$}}
\rput[tl](3.12,4.04){\blue{$\overrightarrow{F}_{\text{frottement}}$}}
\psdots[dotsize=7pt 0,dotstyle=*](3.,3.)
\end{pspicture*}
\end{center}
\end{multicols}


\end{exo}

\begin{exo}

L'octane est un hydrocarbure qui entre dans la composition de l'essence. Lorsqu'on chauffe un mélange d'octane et de solvant dans une cuve, une réaction chimique transforme progressivement l'octane en un carburant plus performant, appelé iso-octane.

La concentration d'octane, en moles par litre, dans la cuve est modélisée par une fonction $C$ du temps $t,$ exprimé en minutes.

On admet que la fonction $C$ est solution de l'équation différentielle \[y'+0,12y=0,003\] sur l'intervalle $\left[0;+\infty\right[.$

À l'instant $t=0,$ la concentration d'octane dans la cuve est de 0,5 moles par litre.

\begin{enumerate}
\item L'équation différentielle $y'+0,12y=0,003$ se réécrit
\[y'=-0,12y+0,003,\] donc d'après le théorème 4 du cours, ses solutions sont de la forme
\[C(t)=C\text{e}^{-0,12t}-\frac{0,003}{-0,12}=C\text{e}^{-0,12t}+0,025,\] où $C$ est une constante.

\medskip

On sait que la concentration au temps $t=0$ est de 0,5 mol/$\ell$, donc $y(0)=0,5.$ On résout donc une équation pour trouver $C~:$
\[C\text{e}^{-0,12\times 0}+0,025=0,5\iff C+0,025=0,5\iff C=0,5-0,025=0,475.\] Finalement la concentration au temps $t$ est
\[C(t)=0,475\text{e}^{-0,12t}+0,025~\text{mol}/\ell.\]
\item $\lim\limits_{t\to +\infty}\text{e}^{-0,12t}=0$ donc
\[\lim\limits_{t\to +\infty}C(t)=0,475\times 0+0,025=0,025.\]
\item Au bout d'une heure (60 minutes), la concentration d'octane est
\[C(60)=0,475\text{e}^{-0,12\times 60}+0,025\approx 0,0256~\text{mol}/\ell,\] ce qui est tout proche de la concentration limite (0,025~mol/$\ell$). Plus précisément~:

\begin{itemize}
\item[\textbullet] entre le temps $t=0$ et le temps $t=+\infty,$ la concentration baisse de $0,5-0,025=0,475$~mol/$\ell~;$
\item[\textbullet] entre le temps $t=0$ et le temps $t=60,$ la concentration baisse de $0,5-0,0256=0,4744$~mol/$\ell$ environ~;
\item[\textbullet] $\frac{0,4744\times 100}{0,475}\approx 99,9.$
\end{itemize}

Conclusion~: 99,9~\% du produit pouvant être transformé s'est déjà transformé au bout d'une heure. Inutile donc de prolonger la transformation au-delà de cette durée.
\end{enumerate}

\end{exo}

\begin{exo}

On considère l'équation différentielle
\[(E)\qquad y'(x)=y(x)-x^2+x.\]

\begin{enumerate}
\item On remplace $y$ par $y_P$ dans chacun des deux membres~:
\begin{itemize}
\item[\textbullet] $y_P'(x)=2x+1~;$
\item[\textbullet] $ y_P(x)-x^2+x=x^2+x+1-x^2+x=2x+1.$
\end{itemize}

On a bien $y_P'(x)=y_P(x)-x^2+x,$ c'est-à-dire que $y_P$ est une solution particulière de $(E).$
\item D'après le théorème 3 du cours, les solutions de $\left(E_0\right): y'(x)=y(x)$ sont les fonctions de la forme $y(x)=C\text{e}^{x},$ où $C$ est une constante.

\item D'après le théorème 5 du cours, les solutions de $(E)$ sont les fonctions de la forme 
\[y(x)=C\text{e}^{x}+y_P(x)=C\text{e}^{x}+x^2+x+1,\] où $C$ est une constante.

\medskip

De plus,
\[y(0)=-1\iff C\text{e}^0+0^2+0+1=-1\iff C+1=-1\iff C=-2,\]
donc l'unique solution de $(E)$ vérifiant la condition initiale $y(0)=-1$ est la fonction définie par
\[y(x)=-2\text{e}^{x}+x^2+x+1.\]

\end{enumerate}
\end{exo}



\begin{exo}

On considère l'équation différentielle
\[(E)\qquad y'+y=\text{e}^{-x}\]



\begin{enumerate}
\item On calcule d'abord la dérivée de $f~:$ pour cela, on utilise la formule pour la dérivée d'un produit avec

\begin{alignat*}{3}
&u(x)=x,&& \hspace{1cm}&&v(x)=\text{e}^{-x}, \\
& u'(x)=1,&& &&v'(x)=-\text{e}^{-x}.\\
\end{alignat*}

On obtient, pour tout $x\in \mathbb{R}~:$

\begin{align*}
f'(x)&=1\times \text{e}^{-x}+x\times \left(-\text{e}^{-x}\right)\\
&=\text{e}^{-x}-x\text{e}^{-x}.
\end{align*}

\medskip

Ensuite on remplace $y$ par $f$ dans le membre de gauche de l'équation~:
\[f'(x)+f(x)=\text{e}^{-x}-x\text{e}^{-x}+x\text{e}^{-x}=\text{e}^{-x}.\]

Conclusion~: $f$ est bien solution de $(E).$

\item L'équation $\left(E_0\right)$ se réécrit $y'(x)=-y(x),$ donc d'après le théorème 3 du cours, ses solutions sont les fonctions de la forme $y(x)=C\text{e}^{-x},$ où $C$ est une constante.

\item L'équation $\left(E\right)$ se réécrit $y'(x)=-y(x)+\text{e}^{-x},$ donc d'après le théorème 5 du cours, ses solutions sont les fonctions de la forme 
\[y(x)=C\text{e}^{-x}+f(x)=C\text{e}^{-x}+x\text{e}^{-x},\] où $C$ est une constante.

\end{enumerate}

\end{exo}



\textbf{Remarque~:} Donnons-nous une équation différentielle $(E)$ de la forme $y'=ay+g,$ où $g$ est une fonction continue, ainsi qu'un point du plan de coordonnées $\left(x_0;y_0\right).$ Dans ce cas, il existe une unique solution $y$ de $(E)$ vérifiant la condition initiale $y\left(x_0\right)=y_0$ (théorème de Cauchy-Lipschitz).

\medskip

Par exemple, avec l'équation $(E)~:~ y'+y=\text{e}^{-x}$ de l'exercice précédent, le lecteur vérifiera facilement que~:

\begin{itemize}
\item[\textbullet] l'unique solution de $(E)$ vérifiant la condition initiale $y(0)=4$ est donnée par
\[y(x)=4\text{e}^{-x}+x\text{e}^{-x}.\]
\item[\textbullet] l'unique solution de $(E)$ vérifiant la condition initiale $y(2)=3$ est donnée par
\[y(x)=3\text{e}^{-x+2}+x\text{e}^{-x}-2\text{e}^{-x}.\]
\end{itemize}

\medskip

Une conséquence du théorème de Cauchy-Lipschitz est que les courbes des différentes solutions ne se coupent pas. On l'illustre ci-dessous en traçant différentes solutions de $(E),$ correspondant à des conditions initiales différentes.


\begin{center}
\newrgbcolor{qqwuqq}{0. 0.39215686274509803 0.}
\newrgbcolor{ccqqqq}{0.8 0. 0.}
\newrgbcolor{ffvvqq}{1. 0.3333333333333333 0.}
\newrgbcolor{zzttff}{0.6 0.2 1.}
\newrgbcolor{wwwwww}{0.4 0.4 0.4}
\newrgbcolor{zzttqq}{0.6 0.2 0.}
\psset{xunit=1.0cm,yunit=1.0cm,algebraic=true,dimen=middle,dotstyle=o,dotsize=5pt 0,linewidth=2.pt,arrowsize=3pt 2,arrowinset=0.25}
\begin{pspicture*}(-4.3,-2.34)(6.12,6.3)
\multips(0,-2)(0,1.0){9}{\psline[linestyle=dashed,linecap=1,dash=1.5pt 1.5pt,linewidth=0.4pt,linecolor=lightgray]{c-c}(-4.3,0)(6.12,0)}
\multips(-4,0)(1.0,0){11}{\psline[linestyle=dashed,linecap=1,dash=1.5pt 1.5pt,linewidth=0.4pt,linecolor=lightgray]{c-c}(0,-2.34)(0,6.3)}
\psaxes[labelFontSize=\scriptstyle,xAxis=true,yAxis=true,Dx=1.,Dy=1.,ticksize=-2pt 0,subticks=2]{->}(0,0)(-4.3,-2.34)(6.12,6.3)
\psplot[linewidth=2.pt,linecolor=qqwuqq,plotpoints=200]{-4.300000000000001}{6.120000000000001}{(1.0+x)*EXP(-x)}
\psplot[linewidth=2.pt,linecolor=ccqqqq,plotpoints=200]{-4.300000000000001}{6.120000000000001}{(2.0+x)*EXP(-x)}
\psplot[linewidth=2.pt,linecolor=blue,plotpoints=200]{-4.300000000000001}{6.120000000000001}{(1.5+x)*EXP(-x)}
\psplot[linewidth=2.pt,linecolor=ffvvqq,plotpoints=200]{-4.300000000000001}{6.120000000000001}{(2.5+x)*EXP(-x)}
\psplot[linewidth=2.pt,linecolor=zzttff,plotpoints=200]{-4.300000000000001}{6.120000000000001}{(0.5+x)*EXP(-x)}
\psplot[linewidth=2.pt,linecolor=wwwwww,plotpoints=200]{-4.300000000000001}{6.120000000000001}{(0.0+x)*EXP(-x)}
\psplot[linewidth=2.pt,linecolor=zzttqq,plotpoints=200]{-4.300000000000001}{6.120000000000001}{(-0.5+x)*EXP(-x)}
\psplot[linewidth=2.pt,linecolor=qqwuqq,plotpoints=200]{-4.300000000000001}{6.120000000000001}{(-1.0+x)*EXP(-x)}
\psplot[linewidth=2.pt,linecolor=ccqqqq,plotpoints=200]{-4.300000000000001}{6.120000000000001}{(-2.5+x)*EXP(-x)}
\psplot[linewidth=2.pt,linecolor=blue,plotpoints=200]{-4.300000000000001}{6.120000000000001}{(-2.0+x)*EXP(-x)}
\psplot[linewidth=2.pt,linecolor=ffvvqq,plotpoints=200]{-4.300000000000001}{6.120000000000001}{(-1.5+x)*EXP(-x)}
\psplot[linewidth=2.pt,linecolor=zzttff,plotpoints=200]{-4.300000000000001}{6.120000000000001}{(-3.0+x)*EXP(-x)}
\psplot[linewidth=2.pt,linecolor=zzttqq,plotpoints=200]{-4.300000000000001}{6.120000000000001}{(2.75+x)*EXP(-x)}
\end{pspicture*}
\end{center}



\begin{exo}

On considère l'équation différentielle
\[(E)\qquad y'-2y=-x.\]

\begin{enumerate}
\item On cherche une solution particulière de $(E)$ de la forme $y_P(x)=ax+b.$ Pour cela, on remplace dans le membre de gauche de $(E)~:$
\[y_P'(x)-2y_P(x)=a-2(ax+b)=a-2ax-2b=\textcolor{red}{-2a}x+\textcolor{blue}{a-2b}.\]

Donc pour que $y_P$ soit solution de $(E)~;$ autrement dit pour que \[y_P'(x)-2y_P(x)=\textcolor{red}{-1}x+\textcolor{blue}{0},\]
il suffit que
\[\begin{cases}
-2a&=-1\\a-2b&=0\end{cases}\] (on dit qu'on a \og identifié \fg~{} les coefficients écrits en rouge et en bleu).

On obtient $a=\frac{-1}{-2}=\frac{1}{2},$ puis $a=2b,$ donc $b=\frac{a}{2}=\frac{\frac{1}{2}}{2}=\frac{1}{4}.$

\medskip

Conclusion~: la fonction définie $y_P(x)=\frac{1}{2}x+\frac{1}{4}$ est une solution particulière de $(E).$ 
\item L'équation $\left(E\right): y'-2y=-x$ se réécrit $y'=2y-x,$ donc d'après le théorème 5 du cours, ses solutions sont les fonctions de la forme 
\[y(x)=C\text{e}^{2x}+y_P(x)=C\text{e}^{2x}+\frac{1}{2}x+\frac{1}{4},\] où $C$ est une constante.
\item On raisonne par équivalences~:

\[y(0)=1\iff C\text{e}^{2\times 0}+\frac{1}{2}\times 0+\frac{1}{4}=1\iff C+\frac{1}{4}=1\iff C=1-\frac{1}{4}=\frac{3}{4},\]


Conclusion~: l'unique solution de $(E)$ vérifiant la condition initiale $y(0)=1$ est la fonction définie par
\[y(x)=\frac{3}{4}\text{e}^{2x}+\frac{1}{2}x+\frac{1}{4}.\]
\end{enumerate}
\end{exo}




\end{document}