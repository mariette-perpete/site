\documentclass[10pt]{article}
\usepackage[T1]{fontenc}
\usepackage[utf8]{inputenc}
\usepackage{fourier}
\usepackage[scaled=0.875]{helvet}
\renewcommand{\ttdefault}{lmtt}
\usepackage{amsmath,amssymb,makeidx}
\usepackage[normalem]{ulem}
\usepackage{fancybox}
\usepackage{cancel}
\usepackage{stmaryrd}
\usepackage{ulem}
\usepackage{tabularx}
\usepackage{geometry}
\usepackage{enumerate}
\geometry{hmargin=1.5cm,vmargin=1.5cm}
\usepackage{dcolumn}
\usepackage{textcomp}
\usepackage{lscape}
\usepackage{eurosym}
%\newcommand{\euro}{\eurologo{}}
\usepackage[dvips]{color}
\usepackage[all]{xy}

\usepackage{tikz,tkz-tab}

\usepackage{systeme}
\usepackage{ upgreek }


\usepackage{pstricks,pst-plot,pst-text,pst-tree,pstricks-add}
\usepackage{colortbl}
\usepackage{diagbox}
\usepackage{fontawesome5}
\usepackage{pifont}
\usepackage{wasysym}


\usepackage{theorem}
\theorembodyfont{\upshape}
\newtheorem{exo}{Exercice}
%\newtheorem{exo}{Exercice}%[section]
\usepackage{hyperref}
\hypersetup{
    colorlinks=true,       % false: liens encadrés; true: liens colorés
    linkcolor=blue,          % couleur des liens (ou bordures) internes
}

%\setlength{\voffset}{-1,5cm}
\usepackage{fancyhdr} 
\usepackage{graphicx}
\usepackage[frenchb]{babel}
\usepackage[np]{numprint}
\usepackage{multicol}
\usepackage{xlop}
\usepackage{soul}
\usepackage{etoolbox}
\usepackage{multirow}
\usepackage{diagbox}

\usepackage{tcolorbox}

\usepackage{xcolor}
\usepackage{stackengine}
    \setstackEOL{\\}

\title{Mathématiques -- Terminale spécialité}

\date{Corrigés des exercices}
\begin{document}
\setlength\parindent{0mm}
\renewcommand \footrulewidth{.2pt}

\maketitle

\tableofcontents


\newpage

\section{Compléments sur la dérivation}


\begin{exo}

La fonction $f$ est définie sur l'intervalle $\left[-2;6\right]$ par

\[ f(x) = 0,5x^2-2x-4.\]

\medskip

Pour tout $x\in\mathbb{R}~:$
\[f'(x)=0,5\times 2x-2\times 1-0=x-2.\]

La dérivée est du premier degré, donc pour obtenir le tableau de signe, il faut résoudre une équation, puis regarder le signe de $a~:$
\begin{align*}x-2&=0\\
 x-\cancel{2}+\cancel{2}&=0+2\\
 x&=2.
 \end{align*}

$a=1$ (puisque $x-2$ signifie $\textcolor{red}{1}x-2$), $a$ est $\oplus$ donc le signe est de la forme \fbox{$-~\upphi~+$}

\medskip


On en déduit le tableau de signe de $f'$ et le tableau de variations de $f~:$


\medskip

\setlength{\columnseprule}{1pt}

\begin{multicols}{2}

\begin{center}
\begin{tikzpicture}[scale=0.8]
\tkzTabInit{$x$/1,$f'(x)$/1,$f(x)$/2}{$-2$,$2$,$6$}
\tkzTabLine{,-,z,+,}
\tkzTabVar{+/$2$,-/$-6$,+/$2$}
\end{tikzpicture}
\end{center}

\columnbreak

Pour compléter l'extrémité des flèches, on calcule~:

\begin{itemize}
\item[\textbullet] $f(-2)=0,5\times (-2)^2-2\times (-2)-4=2$
\item[\textbullet] $f(2)=0,5\times 2^2-2\times 2-4=-6$
\item[\textbullet] $f(6)=0,5\times 6^2-2\times 6-4=2$
\end{itemize}

\medskip

On peut aussi faire un tableau de valeurs à la calculatrice.


\end{multicols}

\medskip

\textbf{Remarque~:} La courbe représentative est une parabole, dont le sommet $S$ a pour coordonnées $(2;-6).$


\begin{center}
\psset{xunit=1.0cm,yunit=0.5cm,algebraic=true,dimen=middle,dotstyle=o,dotsize=5pt 0,linewidth=2.pt,arrowsize=3pt 2,arrowinset=0.25}
\begin{pspicture*}(-2.36,-6.24)(6.46,2.42)
\multips(0,-6)(0,1.0){9}{\psline[linestyle=dashed,linecap=1,dash=1.5pt 1.5pt,linewidth=0.4pt,linecolor=lightgray]{c-c}(-2.36,0)(6.46,0)}
\multips(-2,0)(1.0,0){9}{\psline[linestyle=dashed,linecap=1,dash=1.5pt 1.5pt,linewidth=0.4pt,linecolor=lightgray]{c-c}(0,-6.24)(0,2.42)}
\psaxes[labelFontSize=\scriptstyle,xAxis=true,yAxis=true,Dx=1.,Dy=1.,ticksize=-2pt 0,subticks=2]{->}(0,0)(-2.36,-6.24)(6.46,2.42)
\rput{0.}(2.,-6.){\psplot[linewidth=2.pt,linecolor=blue]{-4.}{4.}{x^2/2/1.}}
\psdots[dotstyle=*,linecolor=red](2.,-6.)
\rput[bl](2.08,-5.8){\red{$S$}}
\end{pspicture*}
\end{center}


\end{exo}

\begin{exo}

On considère un segment $\left[AB\right]$ de longueur 4 et un point mobile $M$ pouvant se déplacer librement sur ce segment.

\begin{center}
\psset{xunit=1cm,yunit=1cm,algebraic=true,dimen=middle,dotstyle=o,dotsize=3pt 0,linewidth=0.8pt,arrowsize=3pt 2,arrowinset=0.25}
\begin{pspicture*}(0.6,0.27)(5.41,1.85)
\psline[linewidth=1.2pt](1,1)(5,1)
\rput[tl](2.92,1.5){$4$}
\rput[tl](1.5,0.7){$x$}
\psline{->}(3.2,1.4)(5,1.4)
\psline{->}(2.8,1.4)(1,1.4)
\psline{->}(1.4,0.6)(1,0.6)
\psline{->}(1.8,0.6)(2.2,0.6)
\psdots[dotstyle=*](1,1)
\rput[bl](0.85,1.08){$A$}
\psdots[dotstyle=*](5,1)
\rput[bl](5.06,1.08){$B$}
\psdots[dotstyle=*](2.2,1)
\rput[bl](2.26,1.08){$M$}
\end{pspicture*}
\end{center}
 

On note  $x$ la longueur du segment $\left[AM\right]$  et $f(x)$  le produit des longueurs $AM\times BM.$

\begin{enumerate}
\item $BM=AB-AM=4-x,$ donc
\begin{align*}
f(x)&=AM\times BM\\
&=x\times (4-x)\\
&=x\times 4+x\times (-x)\\
&=4x-x^2.
\end{align*}
\item Le produit des longueurs  $AM\times BM$ est donné par $f(x),$ donc maximiser ce produit revient à maximiser la fonction $f.$ On étudie donc les variations~ : pour tout $x\in\left[0;4\right],$
\[f'(x)=4\times 1-2x=-2x+4.\]

On résout~:

\begin{align*}-2x+4&=0\\
 -2x+\cancel{4}-\cancel{4}&=0-4\\
 \frac{\cancel{-2}x}{\cancel{-2}}&=\frac{-4}{-2}\\
 x&=2.
 \end{align*}

$a=-2,$ $a$ est $\ominus$ donc le signe est de la forme \fbox{$+~\upphi~-$}

\medskip


On obtient le tableau de signe de $f'$ et le tableau de variations de $f~:$


\medskip

\setlength{\columnseprule}{1pt}

\begin{multicols}{2}

\begin{center}
\begin{tikzpicture}[scale=0.8]
\tkzTabInit{$x$/1,$f'(x)$/1,$f(x)$/2}{$0$,$2$,$4$}
\tkzTabLine{,+,z,-,}
\tkzTabVar{-/,+/,-/}
\end{tikzpicture}
\end{center}

\columnbreak

Il n'est pas utile ici de compléter l'extrémité des flèches~: tout ce qui nous intéresse, c'est la valeur de $x$ pour laquelle $f$ atteint son maximum.
\end{multicols}

Conclusion~: $f$ atteint son maximum lorsque $x=2,$ donc le produit $AM\times BM$ est maximal lorsque $x=2~;$ c'est-à-dire quand $M$ est le milieu de $\left[AB\right].$
\end{enumerate}

\medskip

\textbf{Remarque~:} Cet exemple est celui qu'a choisi Fermat vers 1637 pour exposer sa méthode de l'adégalité -- ancêtre de la dérivation -- pour déterminer le maximum et le minimum d'une fonction.
\end{exo}

\begin{exo}

La fonction $g$ est définie sur $\mathbb{R}$ par 

\[g(x)=0,5x^3+0,75x^2-3x-1.\]

\medskip

Pour tout $x\in\mathbb{R}~:$

\[g'(x)=0,5\times 3x^2+0,75\times 2x-3\times 1-0=1,5x^2+1,5x-3.\]

La dérivée est du second degré, donc on utilise la méthode de la classe de première~:

\begin{itemize}
\item[\textbullet] $a=1,5,$ $b=1,5,$ $c=-3.$
\item[\textbullet] le discriminant est $\Delta=b^2-4ac=1,5^2-4\times 1,5\times (-3)=20,25.$
\item[\textbullet] $\Delta>0,$ donc il y a deux racines~:

\begin{align*}x_1&=\frac{-b-\sqrt{\Delta}}{2a}=\frac{-1,5-\sqrt{20,25}}{2\times 1,5}=\frac{-1,5-4,5}{3}=\frac{-6}{3}=-2,\\
x_2&=\frac{-b+\sqrt{\Delta}}{2a}=\frac{-1,5+\sqrt{20,25}}{2\times 1,5}=\frac{-1,5+4,5}{3}=\frac{3}{3}=1.
\end{align*}
\end{itemize}

\medskip

$a=1,5$  $a$ est $\oplus$ donc le signe est de la forme \fbox{$+~\upphi~-~\upphi~+$}

\medskip

\setlength{\columnseprule}{1pt}

\begin{multicols}{2}
\begin{center}
\begin{tikzpicture}[scale=0.7]
\tkzTabInit{$x$/1,$g'(x)$/1,$g(x)$/2}{$-\infty$,$-2$,$1$,$+\infty$}
\tkzTabLine{,+,z,-,z,+}
\tkzTabVar{-/,+/$4$,-/$-2.75$,+/}
\end{tikzpicture}
\end{center}

\columnbreak

\begin{itemize}
\item[\textbullet] $g(-2)=0,5\times(-2)^3+0,75\times (-2)^2-3\times (-2)-1=4$
\item[\textbullet] $g(1)=0,5\times 1^3+0,75\times 1^2-3\times 1-1=-2,75$
\end{itemize}


\end{multicols}

 \medskip
 
 \textbf{Remarque~:} Voici à quoi ressemble la courbe représentative~:
 

\begin{center}
\psset{xunit=1.0cm,yunit=0.5cm,algebraic=true,dimen=middle,dotstyle=o,dotsize=5pt 0,linewidth=2.pt,arrowsize=3pt 2,arrowinset=0.25}
\begin{pspicture*}(-4.08,-4.14)(4.,4.44)
\multips(0,-4)(0,1.0){9}{\psline[linestyle=dashed,linecap=1,dash=1.5pt 1.5pt,linewidth=0.4pt,linecolor=lightgray]{c-c}(-4.08,0)(4.,0)}
\multips(-4,0)(1.0,0){9}{\psline[linestyle=dashed,linecap=1,dash=1.5pt 1.5pt,linewidth=0.4pt,linecolor=lightgray]{c-c}(0,-4.14)(0,4.44)}
\psaxes[labelFontSize=\scriptstyle,xAxis=true,yAxis=true,Dx=1.,Dy=1.,ticksize=-2pt 0,subticks=2]{->}(0,0)(-4.08,-4.14)(4.,4.44)
\psplot[linewidth=2.pt,linecolor=blue,plotpoints=200]{-4.08}{4.0}{0.5*x^(3.0)+0.75*x^(2.0)-3.0*x-1.0}
\end{pspicture*}
\end{center}


\end{exo}

\begin{exo}

La fonction $h$ est définie sur $\left[1;+\infty\right[$ par 

\[h(x)=(x-6)\sqrt{x}.\]
 
 
 On utilise la formule pour la dérivée d'un produit avec
\begin{align*}
&u(x)=x-6&&,&& v(x)=\sqrt{x}, \\
& u'(x)=1&&, &&v'(x)=\frac{1}{2\sqrt{x}}.\\
\end{align*}

On obtient, pour tout $x\in \left[1;+\infty\right[~:$
\begin{align*}h'(x)&=u'(x)\times v(x)+u(x)\times v'(x)\\&=1\times\sqrt{x}+(x-6)\times\frac{1}{2\sqrt{x}}\\&=\frac{\sqrt{x}\times 2\sqrt{x}}{2\sqrt{x}}+\frac{x-6}{2\sqrt{x}}\\&=\frac{2x}{2\sqrt{x}}+\frac{x-6}{2\sqrt{x}}\qquad\qquad\qquad\left(\text{rappel~:}~\sqrt{x}\times\sqrt{x}=\sqrt{x}^2=x\right)\\&=\frac{3x-6}{2\sqrt{x}}.\end{align*}

\medskip

\begin{itemize}
\item[\textbullet] On résout rapidement~:
\[3x-6=0\iff 3x=6\iff x=\frac{6}{3}=2.\]
\item[\textbullet] Dans $3x-6,$ $a=3$ $\oplus$ , donc \fbox{$-~\upphi~+$}
\item[\textbullet] $2\sqrt{x}$ est strictement positif pour tout $x\in \left[1;+\infty\right[.$
\end{itemize}

\medskip

On a donc le tableau~:

\medskip

\setlength{\columnseprule}{1pt}

\begin{multicols}{2}
\begin{center}
\begin{tikzpicture}[scale=0.8]
\tkzTabInit{$x$/1,$3x-6$/1,$2\sqrt{x}$/1,$h'\left(x\right)$/1,$h\left(x\right)$/2}{$1$,$2$,$+\infty$}
\tkzTabLine{,-,z,+,}
\tkzTabLine{,+,,+,}
\tkzTabLine{,-,z,+,}
\tkzTabVar{+/$-5$,-/$-4\sqrt{2} $,+/}
\end{tikzpicture}
\end{center}

\columnbreak

\begin{itemize}
\item[\textbullet] $h(1)=(1-6)\times\sqrt{1}=-5\times 1=-5~;$
\item[\textbullet] $h(2)=(2-6)\times\sqrt{2}=-4\sqrt{2}.$
\end{itemize}

\end{multicols}

\end{exo}




\begin{exo}

La fonction $f$ est définie sur $\left[1;4\right]$ par $f(x)=x+\dfrac{4}{x}-3.$ On note $\mathcal{C}$ sa courbe représentative, $A,$ $B,$ $C$ les points de $\mathcal{C}$ d'abscisses respectives 1, 2, 4~; et $T_A,$ $T_B,$ $T_C$ les tangentes à $\mathcal{C}$ en ces points.
 
\begin{enumerate}
\item Pour dériver, le plus simple est de réécrire $f(x)$ sous la forme \[f(x)=x+4\times\dfrac{1}{x}-3.\] On obtient alors, pour tout $x\in \left[1;4\right]~:$
\begin{align*}
f'(x)&=1+4\times\left(-\frac{1}{x^2}\right)-0\\
&=1-\frac{4}{x^2}\\
&=\frac{x^2}{x^2}-\frac{4}{x^2}\\
&=\dfrac{x^2-4}{x^2}
\end{align*} 
\item \begin{itemize}
\item[\textbullet] Les racines de $x^2-4$ sont évidentes~: ce sont $x_1=-2$ et $x_2=2.$ Seule la deuxième est dans l'intervalle $\left[1;4\right].$
\item[\textbullet] $x^2$ est strictement positif pour tout $x\in \left[1;4\right].$
\end{itemize}

On obtient donc le tableau~:

\medskip

\setlength{\columnseprule}{1pt}

\begin{multicols}{2}
\begin{center}
\begin{tikzpicture}[scale=0.8]
\tkzTabInit{$x$/1,$x^2-4$/1,$x^2$/1,$f'\left(x\right)$/1,$f\left(x\right)$/2}{$1$,$2$,$4$}
\tkzTabLine{,-,z,+,}
\tkzTabLine{,+,,+,}
\tkzTabLine{,-,z,+,}
\tkzTabVar{+/$2$,-/$1$,+/$2$}
\end{tikzpicture}
\end{center}

\columnbreak

Le signe de $x^2-4$ sur $\left]-\infty;+\infty\right[$ est de la forme \fbox{$+~\upphi~-~\upphi~+$} Mais comme on travaille sur l'intervalle $\left[1;4\right],$ il ne reste plus que la partie droite \fbox{$-~\upphi~+$}

\medskip

On calcule les valeurs aux extrémités des flèches~:

\begin{itemize}
\item[\textbullet] $f(1)=1+\frac{4}{1}-3=2~;$
\item[\textbullet] $f(2)=2+\frac{4}{2}-3=1~;$
\item[\textbullet] $f(4)=4+\frac{4}{4}-3=2.$
\end{itemize}

\end{multicols}
\item On rappelle que la tangente à la courbe en un point d'abscisse $a$ a pour équation
\[y=f'(a)(x-a)+f(a).\]

Appliquons cette formule avec $a=1$ -- puisque le point $A$ a pour abscisse $1~:$

\medskip

$f(1)=2$ (déjà calculé) et $f'(1)=\frac{1^2-4}{1^2}=\frac{-3}{1}=-3,$ donc l'équation de $T_A$ est
\begin{align*}
y&=f'(1)(x-1)+f(1)\\
y&=-3(x-1)+2\\
y&=-3x+3+2\\
y&=-3x+5.
\end{align*}

Le point $A$ a pour coordonnées $(1;2),$ puisque $f(1)=2~;$ la tangente $T_A$ passe donc par ce point. Pour la tracer, il faut placer un deuxième point (c'est une droite)~; ce que l'on peut faire de trois façons différentes~:

\begin{enumerate}[(a)]
\item L'ordonnée à l'origine est $\textcolor{red}{5}$ (puisque $T_A:y=-3x\textcolor{red}{+5}$), donc $T_A$ passe par le point de coordonnées $(0;5).$
\item Le coefficient directeur de $T_A$ est $\textcolor{blue}{-3}$ (puisque $T_A:y=\textcolor{blue}{-3}x+5$), donc en partant de $A,$ il suffit d'avancer de $1$ carreau en abscisse et de descendre de $3$ carreaux en ordonnée -- $T_A$ passe donc par le point de coordonnées $(2;-1).$
\item On calcule un deuxième point avec la formule~: par exemple, si $x=2,$ $y=-3\times 2+5=-1.$ On obtient le point de coordonnées $(2;-1)$ (le même qu'avec la méthode (b)) et on trace la tangente.
\end{enumerate}

\item \begin{itemize}
\item[\textbullet] $f(2)=1$ et $f'(2)=\frac{2^2-4}{2^2}=\frac{0}{4}=0,$ donc l'équation de $T_B$ est
\begin{align*}
y&=f'(2)(x-2)+f(2)\\
y&=0(x-1)+1\\
y&=1.
\end{align*}

Le coefficient directeur étant égal à 0, la tangente $T_B$ est horizontale.
\item[\textbullet]  $f(4)=2$ et $f'(4)=\frac{4^2-4}{4^2}=\frac{12}{16}=0,75,$ donc l'équation de $T_C$ est
\begin{align*}
y&=f'(4)(x-4)+f(4)\\
y&=0,75(x-4)+2\\
y&=0,75x-3+2\\
y&=0,75x-1.\end{align*}

On trace la tangente $T_C$ par la même méthode que $T_A$ (le plus simple et le plus précis est d'utiliser l'ordonnée à l'origine).
\end{itemize}

\item On place les points $A,$ $B,$ $C,$ on trace les trois tangentes et on construit la courbe de la fonction $f$ (en bleu) en s'appuyant sur ces tangentes.


\begin{center}
\newrgbcolor{ffxfqq}{1. 0.4980392156862745 0.}
\psset{xunit=1.0cm,yunit=1.0cm,algebraic=true,dimen=middle,dotstyle=o,dotsize=5pt 0,linewidth=2.pt,arrowsize=3pt 2,arrowinset=0.25}
\begin{pspicture*}(-1.32,-1.88)(6.02,5.6)
\multips(0,-1)(0,1.0){8}{\psline[linestyle=dashed,linecap=1,dash=1.5pt 1.5pt,linewidth=0.4pt,linecolor=lightgray]{c-c}(-1.32,0)(6.02,0)}
\multips(-1,0)(1.0,0){8}{\psline[linestyle=dashed,linecap=1,dash=1.5pt 1.5pt,linewidth=0.4pt,linecolor=lightgray]{c-c}(0,-1.88)(0,5.6)}
\psaxes[labelFontSize=\scriptstyle,xAxis=true,yAxis=true,Dx=1.,Dy=1.,ticksize=-2pt 0,subticks=2]{->}(0,0)(-1.32,-1.88)(6.02,5.6)
\psline[linewidth=2.pt,linecolor=ffxfqq](0.,5.)(2.,-1.)
\rput[tl](2.1,-0.64){\ffxfqq{$T_A$}}
\psline[linewidth=2.pt,linecolor=green](0.,1.)(4.,1.)
\rput[tl](3.26,0.72){\green{$T_B$}}
\psplot[linewidth=2.pt,linecolor=magenta]{0.}{6.02}{(-4.--3.*x)/4.}
\rput[tl](0.48,-0.84){\magenta{$T_C$}}
\psplot[linewidth=2.pt,linecolor=blue,plotpoints=200]{1}{4}{x+4.0/x-3.0}
\rput[bl](1.08,2.2){\ffxfqq{$A$}}
\rput[bl](2.08,1.2){\green{$B$}}
\rput[bl](3.74,2.28){\magenta{$C$}}
\psdots[dotstyle=*,linecolor=ffxfqq](1.,2.)
\psdots[dotstyle=*,linecolor=green](2.,1.)
\psdots[dotstyle=*,linecolor=magenta](4.,2.)
\end{pspicture*}
\end{center}

\end{enumerate}

\end{exo}

\begin{exo}

La fonction $i$ est définie sur $\mathbb{R}$ par 

\[i(x)=\frac{2x}{x^2+1}.\]
 
\begin{enumerate}
\item On utilise la formule pour la dérivée d'un quotient avec
\begin{align*}
&u(x)=2x&&,&& v(x)=x^2+1, \\
& u'(x)=2&&, &&v'(x)=2x.\\
\end{align*}

On obtient, pour tout $x\in \mathbb{R}~:$
\begin{align*}i'(x)&=\frac{u'(x)\times v(x)-u(x)\times v'(x)}{(v(x))^2}
\\&=\frac{2\times\left(x^2+1\right)-2x\times 2x }{\left(x^2+1\right)^2}
\\&=\frac{2x^2+2-4x^2}{\left(x^2+1\right)^2}
\\&=\frac{-2x^2+2}{\left(x^2+1\right)^2}
.
\end{align*}


\item \begin{itemize}
\item[\textbullet] Les racines de $-2x^2+2$ sont assez évidentes~: 
\[-2x^2+2=0\iff 2=2x^2\iff 1=x^2\iff \left(x=1~\text{ou}~x=-1\right).\] 
\item[\textbullet] $\left(x^2+1\right)^2$ est strictement positif pour tout réel $x.$
\end{itemize}

On obtient donc le tableau~:

\medskip

\setlength{\columnseprule}{1pt}

\begin{multicols}{2}
\begin{center}
\begin{tikzpicture}[scale=0.7]
\tkzTabInit{$x$/1,$-2x^2+2$/1,$\left(x^2+1\right)^2$/1,$i'\left(x\right)$/1,$i\left(x\right)$/2}{$-\infty$,$-1$,$1$,$+\infty$}
\tkzTabLine{,-,z,+,z,-}
\tkzTabLine{,+,,+,,+}
\tkzTabLine{,-,z,+,z,-}
\tkzTabVar{+/,-/$-1$,+/$1$,-/}
\end{tikzpicture}
\end{center}

\columnbreak




\begin{itemize}
\item[\textbullet] $i(-1)=\frac{2\times(-1)}{(-1)^2+1}=\frac{-2}{2}=-1~;$
\item[\textbullet] $i(1)=\frac{2\times 1)}{1^2+1}=\frac{2}{2}=1.$
\end{itemize}

\end{multicols}


\item
\begin{enumerate}
\item $i(0)=\frac{2\times 0}{0^2+1}=\frac{0}{1}=0$ et $i'(0)=\frac{-2\times 0^2+2}{\left(0^2+1\right)^2}=\frac{2}{1}=2,$ donc l'équation de $(T)$ est
\begin{align*}
y&=f'(0)(x-0)+f(0)\\
y&=2x+0\\
y&=2x.\end{align*}

\item Pour étudier les positions relatives de $(C):y=\frac{2x}{x^2+1}$ et $(T):y=2x,$ on étudie \textbf{le signe de la différence}~:
\[\frac{2x}{x^2+1}-2x.\]
\begin{itemize}
\item[\textbullet] Pour les valeurs de $x$ pour lesquelles cette différence vaut 0, les deux courbes se coupent~;
\item[\textbullet] pour les valeurs de $x$ pour lesquelles cette différence est strictement positive, $(C)$ est au-dessus de $(T)~;$
\item[\textbullet] pour les valeurs de $x$ pour lesquelles cette différence est strictement négative, $(C)$ est en-dessous de $(T).$
\end{itemize}


\medskip

On commence par calculer la différence~:

\begin{align*}
\frac{2x}{x^2+1}-2x
&= \frac{2x}{x^2+1}-\frac{2x\left(x^2+1\right)}{x^2+1}
\\&=\frac{2x}{x^2+1}-\frac{2x^3+2x}{x^2+1}
\\&=\frac{\cancel{2x}-2x^3-\cancel{2x}}{x^2+1}
\\&=\frac{-2x^3}{x^2+1}.
\end{align*}

\medskip

\footnotesize
\setlength{\columnseprule}{1pt}

\begin{multicols}{2}

\begin{center}
\hspace*{-1cm}
\begin{tikzpicture}[scale=1]
\tkzTabInit{$x$/1,$-2x^3$/1,$\left(x^2+1\right)^2$/1,$\frac{-2x^3}{x^2+1}$/1,Positions relatives des courbes/3}{$-\infty$,$0$,$+\infty$}
\tkzTabLine{,+,z,-,}
\tkzTabLine{,+,,+,}
\tkzTabLine{,+,z,-,}
\tkzTabLine{,(C)\text{ au-dessus de }(T),\scriptsize{\Longstack{S\\e\\ \\c\\o\\u\\p\\e\\n\\t}},(C)\text{ en-dessous de }(T),}
\end{tikzpicture}
\end{center}
\normalsize

\columnbreak

Pour compléter le tableau de signe~:

\begin{itemize}
\item[\textbullet] $-2x^3=0$ lorsque $x=0~;$
\item[\textbullet] $-2x^3$ est $\ominus$ lorsque $x$ est strictement positif~;
\item[\textbullet] $-2x^3$ est $\oplus$ lorsque $x$ est strictement négatif~;
\item[\textbullet] $\left(x^2+1\right)^2$ est strictement positif pour tout réel $x.$
\end{itemize}

\end{multicols}
\end{enumerate}
\item ~{}


\begin{center}
\psset{xunit=1.0cm,yunit=1.0cm,algebraic=true,dimen=middle,dotstyle=o,dotsize=5pt 0,linewidth=2.pt,arrowsize=3pt 2,arrowinset=0.25}
\begin{pspicture*}(-5.92,-2.46)(5.98,2.64)
\multips(0,-2)(0,1.0){6}{\psline[linestyle=dashed,linecap=1,dash=1.5pt 1.5pt,linewidth=0.4pt,linecolor=lightgray]{c-c}(-5.92,0)(5.98,0)}
\multips(-5,0)(1.0,0){12}{\psline[linestyle=dashed,linecap=1,dash=1.5pt 1.5pt,linewidth=0.4pt,linecolor=lightgray]{c-c}(0,-2.46)(0,2.64)}
\psaxes[labelFontSize=\scriptstyle,xAxis=true,yAxis=true,Dx=1.,Dy=1.,ticksize=-2pt 0,subticks=2]{->}(0,0)(-5.92,-2.46)(5.98,2.64)
\psplot[linewidth=2.pt,linecolor=blue,plotpoints=200]{-5.919999999999999}{5.979999999999995}{2.0*x/(x^(2.0)+1.0)}
\psplot[linewidth=2.pt,linecolor=red]{-5.92}{5.98}{(-0.--2.*x)/1.}
\rput[tl](1.22,1.9){\red{$(T)$}}
\rput[tl](3.44,1.04){\blue{$(C)$}}
\end{pspicture*}
\end{center}
\end{enumerate}

\end{exo}

\begin{exo}

La distance (en m) parcourue au temps $t$ (en s) par une pierre en chute libre est $d(t)=5t^2.$

On lance cette pierre d'une hauteur de 20~m.



\begin{enumerate}
\item La pierre arrive au sol quand elle a parcouru 20~m. Il faut donc résoudre l'équation $5t^2=20~:$
\[5t^2=20\iff t^2=\frac{20}{5}\iff t^2=4\iff \left(t=2\quad\text{ou}\quad\underbrace{t=-2}_{\text{impossible}}\right).\]

Conclusion~: la pierre arrive au sol après 2~s.
\item On construit la courbe à partir d'un tableau de valeurs (avec un pas de 0,4 par exemple).

\begin{center}
\begin{tabular}{|l|c|c|c|c|c|c|}
\hline
   $t$ &$0$ &$0,4$ &$0,8$ &$1,2$ &$1,6$&$2$ \\
	\hline
	$d(t)$ &$0$ &$0,8$ &$3,2$ &$7,2$ &$12,8$&$20$ \\
	\hline
\end{tabular}
\end{center}

Pour obtenir ce tableau, on utilise la calculatrice (bien sûr, on met des $x$ à la place des $t$)~:

\medskip

\small

\setlength{\columnseprule}{1pt}
\begin{multicols}{4}

\begin{center}\textbf{Calculatrices collège}\end{center}

\medskip


\begin{itemize}
\item[\textbullet] \fbox{MODE}
\item[\textbullet] 4 : TABLE ou 4 : Tableau
\item[\textbullet] f(X)=$5\text{X}^2$ \fbox{EXE}

(si on demande g(X)=, ne rien rentrer)
\item[\textbullet] Début? $0$ \fbox{EXE}
\item[\textbullet] Fin? $2$ \fbox{EXE}
\item[\textbullet] Pas? $0,4$ \fbox{EXE}
\end{itemize}

\columnbreak

\begin{center}\textbf{NUMWORKS}\end{center}

\medskip

x s'obtient avec les touches  \fbox{alpha}  \fbox{x}

\begin{itemize}
\item[\textbullet] \fbox{\textcolor{yellow}{\faHome}}
\item[\textbullet] Fonctions \fbox{EXE} puis choisir Fonctions \fbox{EXE}
\item[\textbullet] f(x)=$5\text{x}^2$ \fbox{EXE}
\item[\textbullet] choisir Tableau \fbox{EXE} puis Régler l'intervalle \fbox{EXE}

\item[\textbullet] X début\qquad$0$ \fbox{EXE}
\item[\textbullet] X fin\qquad $2$ \fbox{EXE}
\item[\textbullet] Pas\qquad $0.4$ \fbox{EXE}
\item[\textbullet] choisir Valider
\end{itemize}

\columnbreak

\begin{center}\textbf{TI graphiques}\end{center}

\medskip


X s'obtient avec la touche \fbox{$x,t,\theta,n$}
\begin{itemize}
\item[\textbullet] \fbox{$f(x)$}
\item[\textbullet] $\text{Y}_1=5\text{X}^2$ \fbox{EXE}
\item[\textbullet] \fbox{2nde} \fbox{déf table}
\item[\textbullet] DébTable=$0$ \fbox{EXE}
\item[\textbullet] PasTable=$0.4$ \fbox{EXE}

ou

\tiny{$\Delta$}\normalsize Tbl=$0.4$ \fbox{EXE}
\item[\textbullet] \fbox{2nde} \fbox{table}
\end{itemize}

\columnbreak

\begin{center}\textbf{CASIO graphiques}\end{center}

\medskip


X s'obtient avec la touche \fbox{$\text{X},\theta,\text{T}$}
\begin{itemize}
\item[\textbullet] \fbox{MENU} puis choisir TABLE \fbox{EXE}
\item[\textbullet] $\text{Y}_1:5\text{X}^2$ \fbox{EXE}
\item[\textbullet] \fbox{F5} (on choisit donc SET)
\item[\textbullet] Start:$0$ \fbox{EXE}
\item[\textbullet] End:$2$ \fbox{EXE}
\item[\textbullet] Step:$0.4$ \fbox{EXE}
\item[\textbullet]\fbox{EXIT}
\item[\textbullet] \fbox{F6} (on choisit donc TABLE)
\end{itemize}
\end{multicols}

\normalsize


\begin{center}
\psset{xunit=2.5cm,yunit=0.25cm,algebraic=true,dimen=middle,dotstyle=o,dotsize=5pt 0,linewidth=2.pt,arrowsize=3pt 2,arrowinset=0.25}
\begin{pspicture*}(-0.28472279556053043,-2.064821322818694)(2.336654362677241,21.51292853619145)
\multips(0,0)(0,4.0){6}{\psline[linestyle=dashed,linecap=1,dash=1.5pt 1.5pt,linewidth=0.4pt,linecolor=lightgray]{c-c}(0,0)(2.336654362677241,0)}
\multips(0,0)(0.4,0){7}{\psline[linestyle=dashed,linecap=1,dash=1.5pt 1.5pt,linewidth=0.4pt,linecolor=lightgray]{c-c}(0,0)(0,21.51292853619145)}
\psaxes[labelFontSize=\scriptstyle,xAxis=true,yAxis=true,Dx=0.4,Dy=4.,ticksize=-2pt 0,subticks=2]{->}(0,0)(0.,0.)(2.336654362677241,21.51292853619145)
\rput[lt](1.8,2.9){\parbox{1.2343130420771191 cm}{temps \\  (en s)}}
\rput[lt](0.06674676755514843,19.46268941801665){\parbox{1.3124173894361588 cm}{distance \\  (en m)}}
\rput{0.}(0.,0.){\psplot[linewidth=2.pt,linecolor=blue]{0.}{2.}{x^2/2/0.1}}
\psplot[linewidth=2.pt,linecolor=red]{-0.28472279556053043}{2.336654362677241}{(-10.--10.*x)/0.5}
\psdots[dotsize=4pt 0,dotstyle=*,linecolor=blue](0.4,0.8)
\psdots[dotsize=4pt 0,dotstyle=*,linecolor=blue](0.8,3.2)
\psdots[dotsize=4pt 0,dotstyle=*,linecolor=blue](1.2,7.2)
\psdots[dotsize=4pt 0,dotstyle=*,linecolor=blue](1.6,12.8)
\psdots[dotsize=4pt 0,dotstyle=*,linecolor=red](2.,20.)
\end{pspicture*}
\end{center}


\item La vitesse de la pierre au moment de l'impact au sol est $d'(2).$

Or $d'(t)=5\times 2t=10t,$ donc $d'(2)=10\times 2=20.$ Ainsi la vitesse au moment de l'impact est de 20~m/s.

\medskip

\textbf{Remarques~:}

\begin{itemize}
\item[\textbullet] cette vitesse instantanée est le coefficient directeur de la tangente au point $A$ d'abscisse 2 (en rouge).
\item[\textbullet] la \og vraie formule \fg~{}(valable en l'absence de frottements) est $d(t)=4,9t^2.$ Dans l'exercice, on a pris 5 au lieu de 4,9 pour simplifier les calculs.
\end{itemize}
\end{enumerate}

\end{exo}


\begin{exo}



Dans cet exercice, on utilise deux propriétés du cours~:

\begin{itemize}
\item[\textbullet] la dérivée de $x\mapsto \text{e}^{ax+b}$ est $x\mapsto a\text{e}^{ax+b}~;$
\item[\textbullet] une exponentielle est strictement positive.
\end{itemize}

\medskip



\medskip

\small

\setlength{\columnseprule}{1pt}
\begin{multicols}{4}

Pour tout $x\in \mathbb{R}~:$

\begin{align*}
f(x)&=\text{e}^{0,5x+1}\\
f'(x)&=\underbrace{0,5}_{\oplus}\underbrace{\text{e}^{0,5x+1}}_{\oplus}
\end{align*}
\medskip

$f'>0$ donc $f$ strictement croissante sur $\mathbb{R}.$


\begin{center}
\psset{xunit=0.75cm,yunit=0.75cm,algebraic=true,dimen=middle,dotstyle=o,dotsize=5pt 0,linewidth=2.pt,arrowsize=3pt 2,arrowinset=0.25}
\begin{pspicture*}(-2.94,-0.5)(2.9,5.32)
\multips(0,0)(0,1.0){6}{\psline[linestyle=dashed,linecap=1,dash=1.5pt 1.5pt,linewidth=0.4pt,linecolor=lightgray]{c-c}(-2.94,0)(2.9,0)}
\multips(-2,0)(1.0,0){6}{\psline[linestyle=dashed,linecap=1,dash=1.5pt 1.5pt,linewidth=0.4pt,linecolor=lightgray]{c-c}(0,-0.5)(0,5.32)}
\psaxes[labelFontSize=\scriptstyle,xAxis=true,yAxis=true,Dx=1.,Dy=1.,ticksize=-2pt 0,subticks=2]{->}(0,0)(-2.94,-0.5)(2.9,5.32)
\psplot[linewidth=2.pt,linecolor=red,plotpoints=200]{-2.94}{2.900000000000002}{EXP(0.5*x+1.0)}
\end{pspicture*}
\end{center}
\columnbreak

Pour tout $x\in \mathbb{R}~:$

\begin{align*}
g(x)&=\text{e}^{-1,5x}\\
g'(x)&=\underbrace{-1,5}_{\ominus}\underbrace{\text{e}^{-1,5x}}_{\oplus}
\end{align*}
\medskip

$g'<0$ donc $g$ strictement décroissante sur $\mathbb{R}.$


\begin{center}
\psset{xunit=0.75cm,yunit=0.75cm,algebraic=true,dimen=middle,dotstyle=o,dotsize=5pt 0,linewidth=2.pt,arrowsize=3pt 2,arrowinset=0.25}
\begin{pspicture*}(-2.94,-0.5)(2.9,5.32)
\multips(0,0)(0,1.0){6}{\psline[linestyle=dashed,linecap=1,dash=1.5pt 1.5pt,linewidth=0.4pt,linecolor=lightgray]{c-c}(-2.94,0)(2.9,0)}
\multips(-2,0)(1.0,0){6}{\psline[linestyle=dashed,linecap=1,dash=1.5pt 1.5pt,linewidth=0.4pt,linecolor=lightgray]{c-c}(0,-0.5)(0,5.32)}
\psaxes[labelFontSize=\scriptstyle,xAxis=true,yAxis=true,Dx=1.,Dy=1.,ticksize=-2pt 0,subticks=2]{->}(0,0)(-2.94,-0.5)(2.9,5.32)
\psplot[linewidth=2.pt,linecolor=green,plotpoints=200]{-2.94}{2.900000000000002}{EXP(-1.5*x+0.0)}
\end{pspicture*}
\end{center}
\columnbreak
Pour tout $x\in \mathbb{R}~:$

\begin{align*}
h(x)&=\text{e}^{2x-2}\\
h'(x)&=\underbrace{2}_{\oplus}\underbrace{\text{e}^{2x-2}}_{\oplus}
\end{align*}
\medskip

$h'>0$ donc $h$ strictement croissante sur $\mathbb{R}.$


\begin{center}
\psset{xunit=0.75cm,yunit=0.75cm,algebraic=true,dimen=middle,dotstyle=o,dotsize=5pt 0,linewidth=2.pt,arrowsize=3pt 2,arrowinset=0.25}
\begin{pspicture*}(-2.94,-0.5)(2.9,5.32)
\multips(0,0)(0,1.0){6}{\psline[linestyle=dashed,linecap=1,dash=1.5pt 1.5pt,linewidth=0.4pt,linecolor=lightgray]{c-c}(-2.94,0)(2.9,0)}
\multips(-2,0)(1.0,0){6}{\psline[linestyle=dashed,linecap=1,dash=1.5pt 1.5pt,linewidth=0.4pt,linecolor=lightgray]{c-c}(0,-0.5)(0,5.32)}
\psaxes[labelFontSize=\scriptstyle,xAxis=true,yAxis=true,Dx=1.,Dy=1.,ticksize=-2pt 0,subticks=2]{->}(0,0)(-2.94,-0.5)(2.9,5.32)
\psplot[linewidth=2.pt,linecolor=orange,plotpoints=200]{-2.94}{2.900000000000002}{EXP(2*x-2.0)}
\end{pspicture*}
\end{center}
\columnbreak

Pour tout $x\in \mathbb{R}~:$

\begin{align*}
i(x)&=\text{e}^{-1x+1}\\
i'(x)&=\underbrace{-1}_{\ominus}\underbrace{\text{e}^{-1x+1}}_{\oplus}
\end{align*}
\medskip

$i'<0$ donc $i$ strictement décroissante sur $\mathbb{R}.$


\begin{center}
\psset{xunit=0.75cm,yunit=0.75cm,algebraic=true,dimen=middle,dotstyle=o,dotsize=5pt 0,linewidth=2.pt,arrowsize=3pt 2,arrowinset=0.25}
\begin{pspicture*}(-2.94,-0.5)(2.9,5.32)
\multips(0,0)(0,1.0){6}{\psline[linestyle=dashed,linecap=1,dash=1.5pt 1.5pt,linewidth=0.4pt,linecolor=lightgray]{c-c}(-2.94,0)(2.9,0)}
\multips(-2,0)(1.0,0){6}{\psline[linestyle=dashed,linecap=1,dash=1.5pt 1.5pt,linewidth=0.4pt,linecolor=lightgray]{c-c}(0,-0.5)(0,5.32)}
\psaxes[labelFontSize=\scriptstyle,xAxis=true,yAxis=true,Dx=1.,Dy=1.,ticksize=-2pt 0,subticks=2]{->}(0,0)(-2.94,-0.5)(2.9,5.32)
\psplot[linewidth=2.pt,linecolor=blue,plotpoints=200]{-2.94}{2.900000000000002}{EXP(-1*x+1.0)}
\end{pspicture*}
\end{center}



\end{multicols}

\medskip

\`A titre d'illustration, on a tracé les courbes des quatre fonctions. Elles ont toutes une allure très similaire, à deux différences près~:

\begin{itemize}
\item[\textbullet] elles montent lorsque $a>0,$ elles descendent lorsque $a<0~;$
\item[\textbullet] plus $|a|$ est grand, plus la pente de la partie inclinée est forte.
\end{itemize}


\end{exo}

\begin{exo}

La fonction $f$ est définie sur l'intervalle $\left[0;4\right]$ par

\[ f(x) = (-2x+1)\text{e}^{-x}.\]

\begin{enumerate}
\item  On utilise la formule pour la dérivée d'un produit avec

\begin{align*}
&u(x)=-2x+1&&,&& v(x)=\text{e}^{-x}, \\
& u'(x)=-2&&, &&v'(x)=-\text{e}^{-x}.\\
\end{align*}

On obtient, pour tout $x\in\left[0;4\right]~:$


\begin{align*}f'(x)&=u'(x)\times v(x)+u(x)\times v'(x)\\
&=-2\times\text{e}^{-x}+\left(-2x+1\right)\times \left(-\text{e}^{-x}\right)\\
&=-2\times\text{e}^{-x}+(-2x)\times\left(-\text{e}^{-x}\right)+1\times\left(-\text{e}^{-x}\right)\\
&=-2\times\text{e}^{-x}+2x\times\text{e}^{-x}-1\times\text{e}^{-x}\\
&=\left(-2+2x-1\right)\text{e}^{-x}\\
&=\left(2x-3\right)\text{e}^{-x}.
\end{align*}

\item On étudie le signe de $f'$ et on en déduit les variations de $f~:$

\begin{itemize}
\item[\textbullet] $2x-3=0\iff 2x=3\iff x=\frac{3}{2}\iff x=1,5~;$
\item[\textbullet] $\text{e}^{-x}$ est $\oplus$ pour tout réel $x.$
\end{itemize}

\medskip

\setlength{\columnseprule}{1pt}

\begin{multicols}{2}
\begin{center}
\begin{tikzpicture}[scale=0.8]
\tkzTabInit{$x$/1,$2x-3$/1,$\text{e}^{-x}$/1,$f'(x)$/1,$f(x)$/2}{$0$,$1.5$,$4$}
\tkzTabLine{,-,z,+,}
\tkzTabLine{,+,,+,}
\tkzTabLine{,-,z,+,}
\tkzTabVar{+/$1$,-/$-2\text{e}^{-1,5}$,+/$-7\text{e}^{-4}$}
\end{tikzpicture}
\end{center}

\columnbreak

\begin{itemize}
\item[\textbullet] $f(0)=(-2\times 0 +1)\times\underbrace{\text{e}^{-0}}_{=1}=1\times 1=1$
\item[\textbullet] $f(1,5)=(-2\times 1,5 +1)\times\text{e}^{-1,5}=-2\text{e}^{-1,5}\approx -0,45$
\item[\textbullet] $f(4)=(-2\times 4 +1)\times\text{e}^{-4}=-7\text{e}^{-4}\approx -0,13$
\end{itemize}
\end{multicols}

\end{enumerate}


\end{exo}

\begin{exo}

La fonction $g$ est définie sur $\mathbb{R}$ par $g(x)=\text{e}^x-x-1.$

\medskip

Pour tout $x\in\mathbb{R}~:$
\[g'(x)=\text{e}^x-1-0=\text{e}^x-1.\]

\medskip

\setlength{\columnseprule}{1pt}
\begin{multicols}{2}

On résout l'équation~:
\[\text{e}^x-1=0\iff \text{e}^x=1 \iff x=0.\]

\danger On a utilisé la propriété~: le seul nombre dont l'exponentielle est égale à 1 est 0.

\medskip

Pour avoir les signes dans chaque case du tableau, on remplace par des valeurs de $x~:$

\begin{itemize}
\item[\textbullet] pour l'intervalle $\left]-\infty;0\right[,$ on prend (par exemple) $x=-1$ et on calcule avec la calculatrice~:
\[g'(-1)=\text{e}^{-1}-1\approx -0,63\qquad \ominus ~;\]
\item[\textbullet] pour l'intervalle $\left]0;+\infty\right[,$ on prend (par exemple) $x=1$ et on calcule avec la calculatrice~:
\[g'(1)=\text{e}^{1}-1\approx 3,72\qquad \oplus .\]
\end{itemize}

\columnbreak

\begin{center}
\begin{tikzpicture}[scale=1]
\tkzTabInit{$x$/1,$g'(x)=\text{e}^x-1$/1,$g(x)$/2}{$-\infty$,$0$,$+\infty$}
\tkzTabLine{,-,z,+,}
\tkzTabVar{+/,-/$0$,+/}
\end{tikzpicture}

\medskip

\[g(0)=\text{e}^0-0-1=1-1=0.\]
\end{center}



\end{multicols}

\medskip

\textbf{Remarque~:} Le minimum de $g$ est $0,$ donc $g(x)\geq 0$ pour tout réel $x~;$ autrement dit $\text{e}^x-x-1\geq 0.$ Cette inégalité se réécrit
\[\text{e}^x\geq x+1.\]
On obtiendra ce résultat par une autre méthode dans l'exercice 18 (utilisation de la convexité). Cette inégalité sera utilisée plus tard dans l'année, pour démontrer des résultats sur les limites.


\end{exo}

\begin{exo}


\begin{align*}
\dfrac{\text{e}^8}{\text{e}^2\times \text{e}^1\times \text{e}^3}&=\dfrac{\text{e}^8}{\text{e}^{2+1+3}}=\dfrac{\text{e}^8}{\text{e}^{6}}=\text{e}^{8-6}=\text{e}^{2}\\
\dfrac{\text{e}\times\text{e}^2}{\left(\text{e}^2\right)^2}
&=\dfrac{\text{e}^1\times\text{e}^2}{\text{e}^{2\times 2}}
=\dfrac{\text{e}^{1+2}}{\text{e}^{4}}=\text{e}^{3-4}=\text{e}^{-1}\\
\left(\text{e}^2\right)^3\times\text{e}^{-5}&=\text{e}^{2\times 3}\times\text{e}^{-5}=\text{e}^{6-5}=\text{e}^{1}
\end{align*}

\end{exo}

\begin{exo}

Dans chaque cas, on note $\mathcal{S}$ l'ensemble des solutions.

\begin{enumerate}

\item \begin{align*}
&\text{e}^{x}=-3\\
&\text{Impossible, car une exponentielle est strictement positive}\\
&\mathcal{S}=\emptyset
\end{align*}

\item \begin{align*}
&\text{e}^{2x-1}=1\\
&2x-1=0&&\text{(le seul nombre dont l'exponentielle vaut 1 est 0)}\\
&x=\frac{1}{2}\\
&\mathcal{S}=\left\{\frac{1}{2}\right\}.
\end{align*}

\item L'équation $\text{e}^{2x}+2\text{e}^{x}=3$ se réécrit
\[\left(\text{e}^x\right)^2+2\text{e}^{x}-3=0.\]

Pour résoudre, il est astucieux de noter $X=\text{e}^x~;$ l'équation se réécrit alors sous la forme
\[X^2+2X-3=0.\]
On résout avec la méthode de la classe de première~:

\begin{itemize}
\item[\textbullet] $a=1,$ $b=2,$ $c=-3.$
\item[\textbullet] le discriminant est $\Delta=b^2-4ac=2^2-4\times 1\times (-3)=16.$
\item[\textbullet] $\Delta>0,$ donc il y a deux racines~:

\begin{align*}X_1&=\frac{-b-\sqrt{\Delta}}{2a}=\frac{-2-\sqrt{16}}{2\times 1}=\frac{-2-4}{2}=\frac{-6}{2}=-3,\\
X_2&=\frac{-b+\sqrt{\Delta}}{2a}=\frac{-2+\sqrt{16}}{2\times 1}=\frac{-2+4}{2}=\frac{2}{2}=1.
\end{align*}

\medskip

On a posé $X=\text{e}^x,$ donc il y a deux possibilités~:
\[\text{e}^x=-3\qquad\text{ou}\qquad \text{e}^x=1.\] La première équation n'a pas de solution, car une exponentielle est strictement positive~; la deuxième équation a une seule solution~: $x=0.$

\medskip

Conclusion~: L'unique solution de l'équation $\text{e}^{2x}+2\text{e}^{x}=3$ est $x=0~:$
\[\mathcal{S}=\left\{0\right\}.\]


\end{itemize}



\end{enumerate}
\end{exo}

\begin{exo}

On utilisera la propriété~: pour tout nombre réel $x,$\[\text{e}^x\times \text{e}^{-x}=1.\]


\begin{enumerate}
\item D'après l'identité remarquable $(a+b)^2=a^2+2ab+b^2~:$



\[\left(\text{e}^{x}+ \text{e}^{-x}\right)^2=\left(\text{e}^{x}\right)^2+2\times \underbrace{\text{e}^{x}\times \text{e}^{-x}}_{=1}+\left( \text{e}^{-x}\right)^2=\text{e}^{2x}+2+\text{e}^{-2x}.\]
\item On multiplie le numérateur et le dénominateur par $\text{e}^x~:$
\begin{align*}
\dfrac{\text{e}^{x}-\text{e}^{-x}}{\text{e}^x+\text{e}^{-x}}
&=\dfrac{\left(\text{e}^{x}-\text{e}^{-x}\right)\times\text{e}^x}{\left(\text{e}^{x}+\text{e}^{-x}\right)\times\text{e}^x}\\
&=\dfrac{\text{e}^{x}\times \text{e}^{x}-\text{e}^{-x}\times \text{e}^{x}}{\text{e}^{x}\times \text{e}^{x}-\text{e}^{-x}\times \text{e}^{x}}\\
&=\dfrac{\text{e}^{x+x}-\text{e}^{-x+x}}{\text{e}^{x+x}+\text{e}^{-x+x}}\\
&=\dfrac{\text{e}^{2x}-\text{e}^{0}}{\text{e}^{2x}+\text{e}^{0}}\\
&=\dfrac{\text{e}^{2x}-1}{\text{e}^{2x}+1}.
\end{align*}
\end{enumerate}


\end{exo}



\begin{exo}


\begin{enumerate}
\item La fonction $f$ est de la forme $f(x)=\text{e}^{u(x)},$ avec \[u(x)=-x^2,\qquad u'(x)=-2x.\]
 On a donc, pour tout $x\in\mathbb{R}~:$ \[f'(x)=u'(x)\times\text{e}^{u(x)}=-2x\text{e}^{-x^2}.\]
\item La fonction $h$ est de la forme $h(x)=\left(u(x)\right)^n,$ avec \[u(x)=-4x+1,\qquad u'(x)=-4,\qquad n=3.\]
 On a donc, pour tout $x\in\mathbb{R}~:$ \[h'(x)=n\times u'(x)\times \left(u(x)\right)^{n-1}=3\times (-4)\times (-4x+1)^{3-1}=-12\left(-4x+1\right)^2.\]
\item La fonction $i$ est de la forme $i(x)=\text{e}^{u(x)},$ avec \[u(x)=5x-9,\qquad u'(x)=5.\]
 On a donc, pour tout $x\in\mathbb{R}~:$ \[i'(x)=u'(x)\times\text{e}^{u(x)}=5\text{e}^{5x-9}.\]
\item La fonction $j$ est de la forme $j(x)=\left(u(x)\right)^n,$ avec \[u(x)=x^2-3x,\qquad u'(x)=2x-3,\qquad n=5.\]
 On a donc, pour tout $x\in\mathbb{R}~:$ \[j'(x)=n\times u'(x)\times \left(u(x)\right)^{n-1}=5\times \left(2x-3\right)\times \left(x^2-3x\right)^{5-1}=\left(10x-15\right)\times \left(x^2-3x\right)^4.\]
\item L'énoncé nous donne \[k(x)=\sqrt{x^2-x+2}.\] Il faut se méfier~: on ne peut calculer la racine carrée d'un nombre que si celui-ci est positif~; et on ne peut dériver une fonction de la forme $\sqrt{u}$ que lorsqu'elle est strictement positive. Intéressons-nous donc au signe de $x^2-x+2~:$

\medskip

Le discriminant est $\Delta=b^2-4ac=(-1)^2-4\times 1\times 2=-7.$ Il s'ensuit qu'il n'y a pas de racine, et que  $x^2-x+2$ est strictement positif sur $\mathbb{R}.$ La fonction $k$ est donc bien définie sur $\mathbb{R},$ mais aussi dérivable.

\medskip

Elle est de la forme $k(x)=\sqrt{u(x)},$ avec  \[u(x)=x^2-x+2,\qquad u'(x)=2x-1.\]
 On a donc, pour tout $x\in\mathbb{R}~:$ \[k'(x)=\dfrac{u'(x)}{2\sqrt{u(x)}}=\dfrac{2x-1}{2\sqrt{x^2-x+2}}.\]
\end{enumerate}


\medskip

\textbf{Remarque informelle~:} On a déjà vu les dérivées suivantes dans le cours de première~:

\begin{align*}
\left(x^n\right)'&=nx^{n-1}\\
\left(\text{e}^x\right)'&=\text{e}^x\\
\left(\sqrt{x}\right)'&=\dfrac{1}{2\sqrt{x}}
\end{align*}

\medskip

Les trois nouvelles formules du cours de terminale peuvent se réécrire

\begin{align*}
\left(u^n\right)'&=nu^{n-1}\times u'\\
\left(\text{e}^u\right)'&=\text{e}^u\times u'\\
\left(\sqrt{u}\right)'&=\dfrac{1}{2\sqrt{u}}\times u'
\end{align*}

\medskip

On voit qu'il suffit de remplacer $x$ par $u,$ et de multiplier par $u'.$

\end{exo}




\begin{exo}



\begin{enumerate}

\item Pour tout $x\in\mathbb{R}~:$
\begin{align*}
f(x)&=x^2\\
f'(x)&=2x\\
f''(x)&=2.
\end{align*}

\medskip

Conclusion~: $f''$ est strictement positive, donc $f$ est convexe sur $\mathbb{R}.$

\medskip

On peut aussi présenter les choses  avec un tableau de signe~:

\begin{center}
\hspace*{-1cm}
\begin{tikzpicture}[scale=1]
\tkzTabInit{$x$/1,$f''(x)=2$/1,Convexité/3}{$-\infty$,$+\infty$}
\tkzTabLine{,+,}

\tkzTabLine{,f\text{ convexe },}
\end{tikzpicture}
\end{center}

\medskip


\begin{center}
\psset{xunit=1.0cm,yunit=1.0cm,algebraic=true,dimen=middle,dotstyle=o,dotsize=5pt 0,linewidth=2.pt,arrowsize=3pt 2,arrowinset=0.25}
\begin{pspicture*}(-2.46,-0.74)(2.68,3.4)
\multips(0,0)(0,1.0){5}{\psline[linestyle=dashed,linecap=1,dash=1.5pt 1.5pt,linewidth=0.4pt,linecolor=lightgray]{c-c}(-2.46,0)(2.68,0)}
\multips(-2,0)(1.0,0){6}{\psline[linestyle=dashed,linecap=1,dash=1.5pt 1.5pt,linewidth=0.4pt,linecolor=lightgray]{c-c}(0,-0.74)(0,3.4)}
\psaxes[labelFontSize=\scriptstyle,xAxis=true,yAxis=true,Dx=1.,Dy=1.,ticksize=-2pt 0,subticks=2]{->}(0,0)(-2.46,-0.74)(2.68,3.4)
\rput{0.}(0.,0.){\psplot[linewidth=2.pt,linecolor=blue]{-3.}{3.}{x^2/2/0.5}}
\rput[tl](0.26,2.86){\blue{convexe}}
\end{pspicture*}
\end{center}



\item Pour tout $x\in\mathbb{R}~:$
\begin{align*}
g(x)&=x^3\\
g'(x)&=3x^2\\
g''(x)&=6x.
\end{align*}

\medskip

Cette fois, le tableau de signe est fortement recommandé~:


\begin{center}
\hspace*{-1cm}
\begin{tikzpicture}[scale=1.2]
\tkzTabInit{$x$/1,$g''(x)=6x$/1,Convexité/3}{$-\infty$,$0$,$+\infty$}
\tkzTabLine{,-,z,+,}

\tkzTabLine{,g\text{ concave},\scriptsize{\Longstack{P\\t\\ \\i\\n\\f\\l\\e\\x\\i\\o\\n}},g\text{ convexe},}
\end{tikzpicture}
\end{center}

Conclusion~:

\begin{itemize}
\item[\textbullet] $g$ est concave sur $\left]-\infty;0\right]~;$
\item[\textbullet] $g$ est convexe sur $\left[0;+\infty\right[~;$
\item[\textbullet] le point de coordonnées $(0;0)$ est un point d'inflexion.
\end{itemize}

\medskip


\begin{center}
\newrgbcolor{ffxfqq}{1. 0.4980392156862745 0.}
\psset{xunit=1.0cm,yunit=1.0cm,algebraic=true,dimen=middle,dotstyle=o,dotsize=5pt 0,linewidth=2.pt,arrowsize=3pt 2,arrowinset=0.25}
\begin{pspicture*}(-2.46,-1.76)(3.44,1.88)
\multips(0,-1)(0,1.0){4}{\psline[linestyle=dashed,linecap=1,dash=1.5pt 1.5pt,linewidth=0.4pt,linecolor=lightgray]{c-c}(-2.46,0)(3.44,0)}
\multips(-2,0)(1.0,0){6}{\psline[linestyle=dashed,linecap=1,dash=1.5pt 1.5pt,linewidth=0.4pt,linecolor=lightgray]{c-c}(0,-1.76)(0,1.88)}
\psaxes[labelFontSize=\scriptstyle,xAxis=true,yAxis=true,Dx=1.,Dy=1.,ticksize=-2pt 0,subticks=2]{->}(0,0)(-2.46,-1.76)(3.44,1.88)
\rput[tl](1.06,0.86){\blue{convexe}}
\psplot[linewidth=2.pt,linecolor=blue,plotpoints=200]{0}{3.4400000000000035}{x^(3.0)}
\psplot[linewidth=2.pt,linecolor=red,plotpoints=200]{-2.460000000000003}{0}{x^(3.0)}
\rput[tl](-2.34,-0.66){\red{concave}}
\rput[tl](0.58,-1.22){\green{point d'inflexion}}
\psline[linewidth=2.pt,linecolor=green]{->}(1.,-1.)(0.,0.)
\psdots[dotstyle=*,linecolor=green](0.,0.)
\end{pspicture*}
\end{center}
\item Pour tout $x\in\mathbb{R}~:$
\begin{align*}
h(x)&=\text{e}^{x}\\
h'(x)&=\text{e}^{x}\\
h''(x)&=\text{e}^{x}.
\end{align*}

\medskip

Conclusion~: $h''$ est strictement positive, donc $h$ est convexe sur $\mathbb{R}$ (cette fois, on se passe du tableau de signes).


\begin{center}
\psset{xunit=1.0cm,yunit=1.0cm,algebraic=true,dimen=middle,dotstyle=o,dotsize=5pt 0,linewidth=2.pt,arrowsize=3pt 2,arrowinset=0.25}
\begin{pspicture*}(-3.74,-0.86)(2.58,3.22)
\multips(0,0)(0,1.0){5}{\psline[linestyle=dashed,linecap=1,dash=1.5pt 1.5pt,linewidth=0.4pt,linecolor=lightgray]{c-c}(-3.74,0)(2.58,0)}
\multips(-3,0)(1.0,0){7}{\psline[linestyle=dashed,linecap=1,dash=1.5pt 1.5pt,linewidth=0.4pt,linecolor=lightgray]{c-c}(0,-0.86)(0,3.22)}
\psaxes[labelFontSize=\scriptstyle,xAxis=true,yAxis=true,Dx=1.,Dy=1.,ticksize=-2pt 0,subticks=2]{->}(0,0)(-3.74,-0.86)(2.58,3.22)
\rput[tl](-2.3,0.9){\blue{convexe}}
\psplot[linewidth=2.pt,linecolor=blue,plotpoints=200]{-3.7400000000000038}{2.5800000000000027}{EXP(x)}
\end{pspicture*}
\end{center}
\end{enumerate}

\end{exo}




\begin{exo}

La fonction $g$ est définie sur l'intervalle $\left[-1;3\right]$ par

\[ g(x) = -0,5 x^3+2x^2-2x.\]



\begin{enumerate}
\item \medskip

Pour tout $x\in\left[-1;3\right]~:$

\[g'(x)=-0,5\times 3x^2+2\times 2x-2\times 1=-1,5x^2+4x-2.\]

La dérivée est du second degré, donc on utilise la méthode de la classe de première~:

\begin{itemize}
\item[\textbullet] $a=-1,5,$ $b=4,$ $c=-2.$
\item[\textbullet] le discriminant est $\Delta=b^2-4ac=4^2-4\times (-1,5)\times (-2)=4.$
\item[\textbullet] $\Delta>0,$ donc il y a deux racines~:

\begin{align*}x_1&=\frac{-b-\sqrt{\Delta}}{2a}=\frac{-4-\sqrt{4}}{2\times (-1,5)}=\frac{-4-2}{-3}=\frac{-6}{-3}=2,\\
x_2&=\frac{-b+\sqrt{\Delta}}{2a}=\frac{-4+\sqrt{4}}{2\times (-1,5)}=\frac{-4+2}{-3}=\frac{-2}{-3}=\frac{2}{3}.
\end{align*}
\end{itemize}

\medskip

$a=-1,5$  $a$ est $\ominus$ donc le signe est de la forme \fbox{$-~\upphi~+~\upphi~-$}

\medskip

\setlength{\columnseprule}{1pt}

\begin{multicols}{2}
\begin{center}
\begin{tikzpicture}[scale=0.7]
\tkzTabInit{$x$/1,$g'(x)$/1,$g(x)$/2}{$-1$,$\frac{2}{3}$,$2$,$3$}
\tkzTabLine{,-,z,+,z,-}
\tkzTabVar{+/$3.5$,-/$-\frac{16}{27}$,+/$0$,-/$-1.5$}
\end{tikzpicture}
\end{center}

\columnbreak

\begin{itemize}
\item[\textbullet] $g(-1)=-0,5\times (-1)^3+2\times (-1)^2-2\times (-1)=3,5$
\item[\textbullet] $g\left(\frac{2}{3}\right)=-0,5\times \left(\frac{2}{3}\right)^3+2\times \left(\frac{2}{3}\right)^2-2\times \left(\frac{2}{3}\right)=-\frac{16}{27}$
\item[\textbullet] $g(2)=-0,5\times 2^3+2\times 2^2-2\times 2=0$
\item[\textbullet] $g(3)=-0,5\times 3^3+2\times 3^2-2\times 3=-1,5$

\end{itemize}


\end{multicols}

\item Pour tout $x\in\left[-1;3\right]~:$
\[g''(x)=-1,5\times 2x+4\times 1-0=-3x+4.\]


On étudie le signe de $g''~:$
\[-3x+4=0\iff -3x=-4\iff x=\frac{-4}{-3}=\frac{4}{3}.\]

\begin{center}
\begin{tikzpicture}[scale=1.2]
\tkzTabInit{$x$/1,$-3x+4$/1,Convexité/3}{$-1$,$\frac{4}{3}$,$3$}
\tkzTabLine{,+,z,-,}
\tkzTabLine{,g\text{ convexe},\scriptsize{\Longstack{P\\t\\ \\i\\n\\f\\l\\e\\x\\i\\o\\n}},g\text{ concave},}
\end{tikzpicture}
\end{center}

$g\left(\frac{4}{3}\right)=\left[\cdots\right]=-\frac{8}{27},$ donc le point de coordonnées $\left(\frac{4}{3};-\frac{8}{27}\right)$ est un point d'inflexion (noté $I$ sur la figure ci-dessous).
\item ~{}

\begin{center}
\psset{xunit=1.0cm,yunit=1.0cm,algebraic=true,dimen=middle,dotstyle=o,dotsize=5pt 0,linewidth=2.pt,arrowsize=3pt 2,arrowinset=0.25}
\begin{pspicture*}(-1.44,-1.64)(4.44,3.24)
\multips(0,-1)(0,1.0){5}{\psline[linestyle=dashed,linecap=1,dash=1.5pt 1.5pt,linewidth=0.4pt,linecolor=gray]{c-c}(-1.44,0)(4.44,0)}
\multips(-1,0)(1.0,0){6}{\psline[linestyle=dashed,linecap=1,dash=1.5pt 1.5pt,linewidth=0.4pt,linecolor=gray]{c-c}(0,-1.64)(0,3.24)}
\psaxes[labelFontSize=\scriptstyle,xAxis=true,yAxis=true,Dx=1.,Dy=1.,ticksize=-2pt 0,subticks=2]{->}(0,0)(-1.44,-1.64)(4.44,3.24)
\psplot[linewidth=2.pt,plotpoints=200,linecolor=blue]{-1.0}{1.33333}{-0.5*x^(3.0)+2.0*x^(2.0)-2.0*x+0}
\psplot[linewidth=2.pt,plotpoints=200,linecolor=red]{1.33333}{3}{-0.5*x^(3.0)+2.0*x^(2.0)-2.0*x+0.0}
\rput[tl](0.14,0.88){\blue{convexe}}
\rput[tl](3.16,-0.72){\red{concave}}
\psdots[dotstyle=*,linecolor=green](1.3333333333333333,-0.2777777777777777)
\rput[bl](1.42,-0.72){\green{$I$}}
\end{pspicture*}
\end{center}

\end{enumerate}

\end{exo}




\begin{exo}

La fonction $h$ est définie sur l'intervalle $\left[-1;4\right]$ par

\[ h(x) = (2x+3)\text{e}^{-x}.\]

On calcule les dérivées première et seconde~:

\medskip

\begin{enumerate}
\item \textbf{Dérivée première.} On utilise la formule pour la dérivée d'un produit avec

\begin{align*}
&u(x)=2x+3&&,&& v(x)=\text{e}^{-x}, \\
& u'(x)=2&&, &&v'(x)=-\text{e}^{-x}.\\
\end{align*}

On obtient, pour tout $x\in\left[-1;4\right]~:$


\begin{align*}h'(x)&=u'(x)\times v(x)+u(x)\times v'(x)\\
&=2\times\text{e}^{-x}+\left(2x+3\right)\times \left(-\text{e}^{-x}\right)\\
&=2\times\text{e}^{-x}+2x\times\left(-\text{e}^{-x}\right)+3\times\left(-\text{e}^{-x}\right)\\
&=2\times\text{e}^{-x}-2x\times\text{e}^{-x}-3\times\text{e}^{-x}\\
&=\left(2-2x-3\right)\text{e}^{-x}\\
&=\left(-2x-1\right)\text{e}^{-x}.
\end{align*}
\item \textbf{Dérivée seconde.}  On utilise la formule pour la dérivée d'un produit avec

\begin{align*}
&u(x)=-2x-1&&,&& v(x)=\text{e}^{-x}, \\
& u'(x)=-2&&, &&v'(x)=-\text{e}^{-x}.\\
\end{align*}

On obtient, pour tout $x\in\left[-1;4\right]~:$


\begin{align*}h''(x)&=u'(x)\times v(x)+u(x)\times v'(x)\\
&=-2\times\text{e}^{-x}+\left(-2x-1\right)\times \left(-\text{e}^{-x}\right)\\
&=-2\times\text{e}^{-x}+(-2x)\times\left(-\text{e}^{-x}\right)+(-1)\times\left(-\text{e}^{-x}\right)\\
&=-2\times\text{e}^{-x}+2x\times\text{e}^{-x}+1\times\text{e}^{-x}\\
&=\left(-2+2x+1\right)\text{e}^{-x}\\
&=\left(2x-1\right)\text{e}^{-x}.
\end{align*}
\end{enumerate}

\medskip

On étudie le signe de la dérivée seconde~:
\[h''(x)=\left(2x-1\right)\text{e}^{-x}.\]

\begin{itemize}
\item[\textbullet] $2x-1=0\iff 2x=1\iff x=\frac{1}{2}.$
\item[\textbullet] $\text{e}^{-x}$ est $\oplus$ pour tout $x\in\left[-1;4\right].$
\end{itemize}

\medskip

On a donc le tableau~:

\begin{center}
\begin{tikzpicture}[scale=1.2]
\tkzTabInit{$x$/1,$2x-1$/1,$\text{e}^{-x}$/1,$h''(x)$/1,Convexité/3}{$-1$,$\frac{1}{2}$,$4$}
\tkzTabLine{,-,z,+,}
\tkzTabLine{,+,,+,}
\tkzTabLine{,-,z,+,}
\tkzTabLine{,h\text{ concave},\scriptsize{\Longstack{P\\t\\ \\i\\n\\f\\l\\e\\x\\i\\o\\n}},h\text{ convexe},}
\end{tikzpicture}
\end{center}



\begin{center}
\psset{xunit=1.0cm,yunit=0.8cm,algebraic=true,dimen=middle,dotstyle=o,dotsize=5pt 0,linewidth=1.6pt,arrowsize=3pt 2,arrowinset=0.25}
\begin{pspicture*}(-2.26,-0.5)(5.3,3.8)
\multips(0,0)(0,1.0){6}{\psline[linestyle=dashed,linecap=1,dash=1.5pt 1.5pt,linewidth=0.4pt,linecolor=gray]{c-c}(-2.26,0)(6.18,0)}
\multips(-2,0)(1.0,0){9}{\psline[linestyle=dashed,linecap=1,dash=1.5pt 1.5pt,linewidth=0.4pt,linecolor=gray]{c-c}(0,0)(0,3.8)}
\psaxes[labelFontSize=\scriptstyle,xAxis=true,yAxis=true,Dx=1.,Dy=1.,ticksize=-2pt 0,subticks=2]{->}(0,0)(-2.26,-1.98)(5.3,3.8)
\psplot[linewidth=2.pt,plotpoints=200,linecolor=red]{-1}{0.5}{(2.0*x+3.0)*2.718281828459045^(-x)}
\psplot[linewidth=2.pt,plotpoints=200,linecolor=blue]{0.5}{4}{(2.0*x+3.0)*2.718281828459045^(-x)}
\psline[linewidth=2.pt,linecolor=green]{->}(1.1,3.3)(0.5,2.4261226388505337)
\rput[tl](0.3,3.8){\green{Point d'inflexion}}
\rput[tl](-1.68,1.46){\red{concave}}
\rput[tl](2.58,1.34){\blue{convexe}}
\psdots[dotsize=4pt 0,dotstyle=*,linecolor=green](0.5,2.4261226388505337)
\end{pspicture*}
\end{center}


\end{exo}

\begin{exo}

On note $\mathcal{C}$ la courbe de la fonction exponentielle et $T$ sa tangente au point $A(0;1).$

\begin{enumerate}

\item On pose $f(x)=\text{e}^x$ pour tout $x\in\mathbb{R}.$ On sait que $f'(x)=\text{e}^x$ pour tout $x\in\mathbb{R},$ donc\[f(0)=f'(0)=\text{e}^0=1.\] L'équation de la tangente $T$ est donc
\begin{align*}
y&=f'(0)(x-0)+f(0)\\
y&=1(x-0)+1\\
y&=x+1
\end{align*}
\item On a déjà vu dans un exercice précédent que la fonction exponentielle était convexe sur $\mathbb{R}.$ D'après le théorème 8 du cours, la courbe $\mathcal{C}$ est au-dessus de toutes ses tangentes~; elle est donc en particulier au-dessus de $T.$ Il s'ensuit que
\[\text{e}^x\geq x+1\] pour tout $x\in\mathbb{R}.$


\begin{center}
\psset{xunit=1.0cm,yunit=1.0cm,algebraic=true,dimen=middle,dotstyle=o,dotsize=5pt 0,linewidth=2.pt,arrowsize=3pt 2,arrowinset=0.25}
\begin{pspicture*}(-4.44,-1.04)(4.8,4.42)
\multips(0,-1)(0,1.0){6}{\psline[linestyle=dashed,linecap=1,dash=1.5pt 1.5pt,linewidth=0.4pt,linecolor=lightgray]{c-c}(-4.44,0)(4.8,0)}
\multips(-4,0)(1.0,0){10}{\psline[linestyle=dashed,linecap=1,dash=1.5pt 1.5pt,linewidth=0.4pt,linecolor=lightgray]{c-c}(0,-1.04)(0,4.42)}
\psaxes[labelFontSize=\scriptstyle,xAxis=true,yAxis=true,Dx=1.,Dy=1.,ticksize=-2pt 0,subticks=2]{->}(0,0)(-4.44,-1.04)(4.8,4.42)
\psplot[linewidth=2.pt,linecolor=red,plotpoints=200]{-4.439999999999999}{4.799999999999998}{EXP(x)}
\psplot[linewidth=2.pt,linecolor=blue]{-4.44}{4.8}{(--1.--1.*x)/1.}
\rput[tl](0.74,3.56){\red{$\mathcal{C}$}}
\rput[tl](1.78,2.6){\blue{$T$}}
\psdots[dotsize=5pt 0,dotstyle=*,linecolor=blue](0.,1.)
\rput[bl](0.12,0.68){\blue{$A$}}
\end{pspicture*}
\end{center}

\medskip

\textbf{Remarque~:} On a déjà démontré ce résultat par une étude de fonction, dans l'exercice 10.
\end{enumerate}

\end{exo}


\begin{exo}



\begin{enumerate}
\item Si $u(x)=x^2$ et $ v(x)=4x+1,$ alors
\[v\circ u(x)=v(u(x))=v\left(x^2\right)=4x^2+1.\]
\item Si $u(x)=x+2$ et $v(x)=x^3-3x,$ alors
\[v\circ u(x)=v(u(x))=v\left(x+2\right)=(x+2)^3-3(x+2).\]
\item  Si $u(x)=x-4$ et $ v(x)=\sqrt{x},$ alors
\[v\circ u(x)=v(u(x))=v\left(x-4\right)=\sqrt{x-4}.\]
\item Si $u(x)=2x+3$ et $v(x)=\text{e}^x,$ alors
\[v\circ u(x)=v(u(x))=v\left(2x+3\right)=\text{e}^{2x+3}.\]
\end{enumerate}

\end{exo}

\begin{exo}

\begin{enumerate}
\item Sachant que $v\circ u(x)=\sqrt{x^2+1},$ on peut prendre
\[u(x)=x^2+1\qquad,\qquad v(x)=\sqrt{x}.\]
\item Sachant que $v\circ u(x)=(x-3)^2+5(x-3)+1,$ on peut prendre
\[u(x)=x-3\qquad,\qquad v(x)=x^2+5x+1.\]
\item  Sachant que $v\circ u(x)=\text{e}^{3x-1},$ on peut prendre
\[u(x)=3x-1\qquad,\qquad v(x)=\text{e}^x.\]
\end{enumerate}

\medskip

\textbf{Remarque~:} Il y a une infinité de choix possibles. Par exemple, pour le deuxième, on pourrait prendre
\[u(x)=(x-3)^2+5(x-3)\qquad,\qquad v(x)=x+1~;\]

ou encore 
\[u(x)=(x-3)^2+5(x-3)+1\qquad,\qquad v(x)=x~;\]
etc.


\end{exo}

\begin{exo}

On considère dans un repère orthonormé la parabole $P:y=x^2$ et le point $A(3;0).$


\begin{enumerate}
\item Soit $m$ un réel et soit $M$ le point de $P$ d'abscisse $m.$ L'ordonnée de $M$ est $m^2,$ donc 
\begin{align*}AM&=\sqrt{\left(x_M-x_A\right)^2+\left(y_M-y_A\right)^2}
\\&=\sqrt{\left(m-3\right)^2+\left(m^2-0\right)^2}
\\&=\sqrt{m^2-2\times m\times 3+3^2+m^4}
\\&=\sqrt{m^4+m^2-6m+9}.
\end{align*}

\medskip

On remarque que $AM=f(m),$ où $f$ est la fonction définie dans la question suivante. De ce fait, trouver le point $M$ pour lequel la longueur $AM$ est minimale revient à trouver la valeur de $x$ pour laquelle $f$ atteint son minimum. Nous y reviendrons dans la question 3.
\item On pose $f(x)=\sqrt{x^4+x^2-6x+9}$ pour tout $x\in\mathbb{R}.$


La fonction $f$ est de la forme $f(x)=\sqrt{u(x)},$ avec  \[u(x)=x^4+x^2-6x+9,\qquad u'(x)=4x^3+2x-6.\]
 On a donc, pour tout $x\in\mathbb{R}~:$ \[f'(x)=\dfrac{u'(x)}{2\sqrt{u(x)}}=\dfrac{4x^3+2x-6}{2\sqrt{x^4+x^2-6x+9}}=\dfrac{\cancel{2}\left(2x^3+x-3\right)}{\cancel{2}\sqrt{x^4+x^2-6x+9}}=\dfrac{2x^3+x-3}{\sqrt{x^4+x^2-6x+9}}.\]


\medskip

Pour démontrer la formule de l'énoncé, on développe~:
\[(x-1)\left(2x^2+2x+3\right)=x\times 2x^2+x\times 2x+x\times 3-1\times 2x^2-1\times 2x-1\times 3=2x^3+2x^2+3x-2x^2-2x-3=2x^3+x-3.\]
On retombe sur le numérateur obtenu précédemment~; on a donc bien
\[f'(x)=\dfrac{(x-1)\left(2x^2+2x+3\right)}{\sqrt{x^4+x^2-6x+9}}.\]

Pour construire le tableau de variations de la fonction $f,$ il faut étudier le signe de $2x^2+2x+3.$ Son discriminant est $\Delta=2^2-4\times 2\times 3=-20,$ donc il n'y a pas de racine et $2x^2+2x+3$ est strictement positif pour tout réel $x.$ On peut donc compléter le tableau~:

\medskip

\begin{center}
\begin{tikzpicture}[scale=1.3]
\tkzTabInit{$x$/1,$x-1$/1,$2x^2+2x+3$/1,\footnotesize$\sqrt{x^4+x^2-6x+9}$/1,\normalsize$f'(x)$/1,$f(x)$/2}{$-\infty$,$1$,$+\infty$}
\tkzTabLine{,-,z,+,}
\tkzTabLine{,+,,+,}
\tkzTabLine{,+,,+,}
\tkzTabLine{,-,z,+,}
\tkzTabVar{+/,-/,+/}
\end{tikzpicture}
\end{center}

\item

\begin{itemize}
\item[\textbullet] La fonction $f$ atteint son minimum pour $x=1,$ donc la longueur $AM$ est minimale lorsque $m=1.$ Autrement dit, le point de $P$ le plus proche de $A$ est le point $M(1;1).$

\item[\textbullet] La tangente $(T)$ à la parabole $P$ au point $M$ a pour équation
\[y=g'(1)(x-1)+g(1),\]
avec $g(x)=x^2$ -- donc $g'(x)=2x,$ et $g'(1)=2\times 1=2.$ On a ainsi

\begin{align*}
(T):y&=g'(1)(x-1)+g(1)\\
y&=2(x-1)+1\\
y&=2x-1.
\end{align*}
\item[\textbullet] Pour prouver que $(AM)$ est perpendiculaire à $(T),$ on utilise le produit scalaire~:


$(T)$ passe par $M(1;1)$ et par $N(2;3)$ (puisque $2\times 2-1=3$), donc elle est dirigée par le vecteur $\overrightarrow{MN}\begin{pmatrix}1\\2\end{pmatrix}.$ Par ailleurs $\overrightarrow{AM}\begin{pmatrix}-2\\1\end{pmatrix},$ donc
\[\overrightarrow{MN}\cdot \overrightarrow{AM}=1\times (-2)+2\times 1=0.\] Les droites $(T)$ et $(AM)$ sont donc bien perpendiculaires.
\end{itemize}


\begin{center}
\newrgbcolor{ududff}{0.30196078431372547 0.30196078431372547 1.}
\psset{xunit=1.0cm,yunit=1.0cm,algebraic=true,dimen=middle,dotstyle=o,dotsize=5pt 0,linewidth=2.pt,arrowsize=3pt 2,arrowinset=0.25}
\begin{pspicture*}(-2.34,-0.92)(4.92,5.4)
\multips(0,0)(0,1.0){7}{\psline[linestyle=dashed,linecap=1,dash=1.5pt 1.5pt,linewidth=0.4pt,linecolor=lightgray]{c-c}(-2.34,0)(4.92,0)}
\multips(-2,0)(1.0,0){8}{\psline[linestyle=dashed,linecap=1,dash=1.5pt 1.5pt,linewidth=0.4pt,linecolor=lightgray]{c-c}(0,-0.92)(0,5.4)}
\psaxes[labelFontSize=\scriptstyle,xAxis=true,yAxis=true,Dx=1.,Dy=1.,ticksize=-2pt 0,subticks=2]{->}(0,0)(-2.34,-0.92)(4.92,5.4)
\pspolygon[linewidth=2.pt,linecolor=red,fillcolor=red!30!white,fillstyle=solid,opacity=0.1](1.3794733192202056,0.8102633403898972)(1.5692099788303084,1.1897366596101029)(1.1897366596101029,1.3794733192202056)(1.,1.)
\rput{0.}(0.,0.){\psplot[linewidth=2.pt,linecolor=blue]{-4.}{4.}{x^2/2/0.5}}
\psplot[linewidth=2.pt,linecolor=red]{-2.34}{4.92}{(-0.5--1.*x)/0.5}
\psline[linewidth=2.pt,linestyle=dashed,dash=2pt 2pt,linecolor=red](3.,0.)(1.,1.)
\rput[tl](2.28,3.36){\red{$(T)$}}
\rput[tl](-1.42,2.64){\blue{$P$}}
\psdots[dotstyle=*,linecolor=ududff](1.,1.)
\rput[bl](0.74,1.36){\ududff{$M$}}
\psdots[dotstyle=*,linecolor=red](3.,0.)
\rput[bl](3.08,0.2){\red{$A$}}
\end{pspicture*}
\end{center}
\end{enumerate}

\end{exo}

\end{document}
