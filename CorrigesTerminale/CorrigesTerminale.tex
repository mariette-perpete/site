\documentclass[10pt]{article}
\usepackage[T1]{fontenc}
\usepackage[utf8]{inputenc}
\usepackage{fourier}
\usepackage[scaled=0.875]{helvet}
\renewcommand{\ttdefault}{lmtt}
\usepackage{amsmath,amssymb,makeidx}
\usepackage[normalem]{ulem}
\usepackage{fancybox}
\usepackage{cancel}
\usepackage{stmaryrd}
\usepackage{ulem}
\usepackage{tabularx}
\usepackage{geometry}
\usepackage{enumerate}
\geometry{hmargin=1.5cm,vmargin=1.5cm}
\usepackage{dcolumn}
\usepackage{textcomp}
\usepackage{lscape}
\usepackage{eurosym}
%\newcommand{\euro}{\eurologo{}}
\usepackage[dvips]{color}
\usepackage[all]{xy}

\usepackage{tikz,tkz-tab}

\usepackage{systeme}
\usepackage{ upgreek }


\usepackage{pstricks,pst-plot,pst-text,pst-tree,pstricks-add}
\usepackage{colortbl}
\usepackage{diagbox}
\usepackage{fontawesome5}
\usepackage{pifont}
\usepackage{wasysym}


\usepackage{theorem}
\theorembodyfont{\upshape}
\newtheorem{exo}{Exercice}
%\newtheorem{exo}{Exercice}%[section]
\usepackage{hyperref}
\hypersetup{
    colorlinks=true,       % false: liens encadrés; true: liens colorés
    linkcolor=blue,          % couleur des liens (ou bordures) internes
}

%\setlength{\voffset}{-1,5cm}
\usepackage{fancyhdr} 
\usepackage{graphicx}
\usepackage[frenchb]{babel}
\usepackage[np]{numprint}
\usepackage{multicol}
\usepackage{xlop}
\usepackage{soul}
\usepackage{etoolbox}
\usepackage{multirow}
\usepackage{diagbox}

\usepackage{tcolorbox}

\usepackage{xcolor}
\usepackage{stackengine}
    \setstackEOL{\\}

\title{Mathématiques -- Terminale spécialité}

\date{Corrigés des exercices}
\begin{document}
\setlength\parindent{0mm}
\renewcommand \footrulewidth{.2pt}

\maketitle

\tableofcontents


\newpage

\section{Compléments sur la dérivation}


\begin{exo}

La fonction $f$ est définie sur l'intervalle $\left[-2;6\right]$ par

\[ f(x) = 0,5x^2-2x-4.\]

\medskip

Pour tout $x\in\mathbb{R}~:$
\[f'(x)=0,5\times 2x-2\times 1-0=x-2.\]

La dérivée est du premier degré, donc pour obtenir le tableau de signe, il faut résoudre une équation, puis regarder le signe de $a~:$
\begin{align*}x-2&=0\\
 x-\cancel{2}+\cancel{2}&=0+2\\
 x&=2.
 \end{align*}

$a=1$ (puisque $x-2$ signifie $\textcolor{red}{1}x-2$), $a$ est $\oplus$ donc le signe est de la forme \fbox{$-~\upphi~+$}

\medskip


On en déduit le tableau de signe de $f'$ et le tableau de variations de $f~:$


\medskip

\setlength{\columnseprule}{1pt}

\begin{multicols}{2}

\begin{center}
\begin{tikzpicture}[scale=0.8]
\tkzTabInit{$x$/1,$f'(x)$/1,$f(x)$/2}{$-2$,$2$,$6$}
\tkzTabLine{,-,z,+,}
\tkzTabVar{+/$2$,-/$-6$,+/$2$}
\end{tikzpicture}
\end{center}

\columnbreak

Pour compléter l'extrémité des flèches, on calcule~:

\begin{itemize}
\item[\textbullet] $f(-2)=0,5\times (-2)^2-2\times (-2)-4=2$
\item[\textbullet] $f(2)=0,5\times 2^2-2\times 2-4=-6$
\item[\textbullet] $f(6)=0,5\times 6^2-2\times 6-4=2$
\end{itemize}

\medskip

On peut aussi faire un tableau de valeurs à la calculatrice.


\end{multicols}

\medskip

\textbf{Remarque~:} La courbe représentative est une parabole, dont le sommet $S$ a pour coordonnées $(2;-6).$


\begin{center}
\psset{xunit=1.0cm,yunit=0.5cm,algebraic=true,dimen=middle,dotstyle=o,dotsize=5pt 0,linewidth=2.pt,arrowsize=3pt 2,arrowinset=0.25}
\begin{pspicture*}(-2.36,-6.24)(6.46,2.42)
\multips(0,-6)(0,1.0){9}{\psline[linestyle=dashed,linecap=1,dash=1.5pt 1.5pt,linewidth=0.4pt,linecolor=lightgray]{c-c}(-2.36,0)(6.46,0)}
\multips(-2,0)(1.0,0){9}{\psline[linestyle=dashed,linecap=1,dash=1.5pt 1.5pt,linewidth=0.4pt,linecolor=lightgray]{c-c}(0,-6.24)(0,2.42)}
\psaxes[labelFontSize=\scriptstyle,xAxis=true,yAxis=true,Dx=1.,Dy=1.,ticksize=-2pt 0,subticks=2]{->}(0,0)(-2.36,-6.24)(6.46,2.42)
\rput{0.}(2.,-6.){\psplot[linewidth=2.pt,linecolor=blue]{-4.}{4.}{x^2/2/1.}}
\psdots[dotstyle=*,linecolor=red](2.,-6.)
\rput[bl](2.08,-5.8){\red{$S$}}
\end{pspicture*}
\end{center}


\end{exo}

\begin{exo}

On considère un segment $\left[AB\right]$ de longueur 4 et un point mobile $M$ pouvant se déplacer librement sur ce segment.

\begin{center}
\psset{xunit=1cm,yunit=1cm,algebraic=true,dimen=middle,dotstyle=o,dotsize=3pt 0,linewidth=0.8pt,arrowsize=3pt 2,arrowinset=0.25}
\begin{pspicture*}(0.6,0.27)(5.41,1.85)
\psline[linewidth=1.2pt](1,1)(5,1)
\rput[tl](2.92,1.5){$4$}
\rput[tl](1.5,0.7){$x$}
\psline{->}(3.2,1.4)(5,1.4)
\psline{->}(2.8,1.4)(1,1.4)
\psline{->}(1.4,0.6)(1,0.6)
\psline{->}(1.8,0.6)(2.2,0.6)
\psdots[dotstyle=*](1,1)
\rput[bl](0.85,1.08){$A$}
\psdots[dotstyle=*](5,1)
\rput[bl](5.06,1.08){$B$}
\psdots[dotstyle=*](2.2,1)
\rput[bl](2.26,1.08){$M$}
\end{pspicture*}
\end{center}
 

On note  $x$ la longueur du segment $\left[AM\right]$  et $f(x)$  le produit des longueurs $AM\times BM.$

\begin{enumerate}
\item $BM=AB-AM=4-x,$ donc
\begin{align*}
f(x)&=AM\times BM\\
&=x\times (4-x)\\
&=x\times 4+x\times (-x)\\
&=4x-x^2.
\end{align*}
\item Le produit des longueurs  $AM\times BM$ est donné par $f(x),$ donc maximiser ce produit revient à maximiser la fonction $f.$ On étudie donc les variations~ : pour tout $x\in\left[0;4\right],$
\[f'(x)=4\times 1-2x=-2x+4.\]

On résout~:

\begin{align*}-2x+4&=0\\
 -2x+\cancel{4}-\cancel{4}&=0-4\\
 \frac{\cancel{-2}x}{\cancel{-2}}&=\frac{-4}{-2}\\
 x&=2.
 \end{align*}

$a=-2,$ $a$ est $\ominus$ donc le signe est de la forme \fbox{$+~\upphi~-$}

\medskip


On obtient le tableau de signe de $f'$ et le tableau de variations de $f~:$


\medskip

\setlength{\columnseprule}{1pt}

\begin{multicols}{2}

\begin{center}
\begin{tikzpicture}[scale=0.8]
\tkzTabInit{$x$/1,$f'(x)$/1,$f(x)$/2}{$0$,$2$,$4$}
\tkzTabLine{,+,z,-,}
\tkzTabVar{-/,+/,-/}
\end{tikzpicture}
\end{center}

\columnbreak

Il n'est pas utile ici de compléter l'extrémité des flèches~: tout ce qui nous intéresse, c'est la valeur de $x$ pour laquelle $f$ atteint son maximum.
\end{multicols}

Conclusion~: $f$ atteint son maximum lorsque $x=2,$ donc le produit $AM\times BM$ est maximal lorsque $x=2~;$ c'est-à-dire quand $M$ est le milieu de $\left[AB\right].$
\end{enumerate}

\medskip

\textbf{Remarque~:} Cet exemple est celui qu'a choisi Fermat vers 1637 pour exposer sa méthode de l'adégalité -- ancêtre de la dérivation -- pour déterminer le maximum et le minimum d'une fonction.
\end{exo}

\begin{exo}

La fonction $g$ est définie sur $\mathbb{R}$ par 

\[g(x)=0,5x^3+0,75x^2-3x-1.\]

\medskip

Pour tout $x\in\mathbb{R}~:$

\[g'(x)=0,5\times 3x^2+0,75\times 2x-3\times 1-0=1,5x^2+1,5x-3.\]

La dérivée est du second degré, donc on utilise la méthode de la classe de première~:

\begin{itemize}
\item[\textbullet] $a=1,5,$ $b=1,5,$ $c=-3.$
\item[\textbullet] le discriminant est $\Delta=b^2-4ac=1,5^2-4\times 1,5\times (-3)=20,25.$
\item[\textbullet] $\Delta>0,$ donc il y a deux racines~:

\begin{align*}x_1&=\frac{-b-\sqrt{\Delta}}{2a}=\frac{-1,5-\sqrt{20,25}}{2\times 1,5}=\frac{-1,5-4,5}{3}=\frac{-6}{3}=-2,\\
x_2&=\frac{-b+\sqrt{\Delta}}{2a}=\frac{-1,5+\sqrt{20,25}}{2\times 1,5}=\frac{-1,5+4,5}{3}=\frac{3}{3}=1.
\end{align*}
\end{itemize}

\medskip

$a=1,5$  $a$ est $\oplus$ donc le signe est de la forme \fbox{$+~\upphi~-~\upphi~+$}

\medskip

\setlength{\columnseprule}{1pt}

\begin{multicols}{2}
\begin{center}
\begin{tikzpicture}[scale=0.7]
\tkzTabInit{$x$/1,$g'(x)$/1,$g(x)$/2}{$-\infty$,$-2$,$1$,$+\infty$}
\tkzTabLine{,+,z,-,z,+}
\tkzTabVar{-/,+/$4$,-/$-2.75$,+/}
\end{tikzpicture}
\end{center}

\columnbreak

\begin{itemize}
\item[\textbullet] $g(-2)=0,5\times(-2)^3+0,75\times (-2)^2-3\times (-2)-1=4$
\item[\textbullet] $g(1)=0,5\times 1^3+0,75\times 1^2-3\times 1-1=-2,75$
\end{itemize}


\end{multicols}

 \medskip
 
 \textbf{Remarque~:} Voici à quoi ressemble la courbe représentative~:
 

\begin{center}
\psset{xunit=1.0cm,yunit=0.5cm,algebraic=true,dimen=middle,dotstyle=o,dotsize=5pt 0,linewidth=2.pt,arrowsize=3pt 2,arrowinset=0.25}
\begin{pspicture*}(-4.08,-4.14)(4.,4.44)
\multips(0,-4)(0,1.0){9}{\psline[linestyle=dashed,linecap=1,dash=1.5pt 1.5pt,linewidth=0.4pt,linecolor=lightgray]{c-c}(-4.08,0)(4.,0)}
\multips(-4,0)(1.0,0){9}{\psline[linestyle=dashed,linecap=1,dash=1.5pt 1.5pt,linewidth=0.4pt,linecolor=lightgray]{c-c}(0,-4.14)(0,4.44)}
\psaxes[labelFontSize=\scriptstyle,xAxis=true,yAxis=true,Dx=1.,Dy=1.,ticksize=-2pt 0,subticks=2]{->}(0,0)(-4.08,-4.14)(4.,4.44)
\psplot[linewidth=2.pt,linecolor=blue,plotpoints=200]{-4.08}{4.0}{0.5*x^(3.0)+0.75*x^(2.0)-3.0*x-1.0}
\end{pspicture*}
\end{center}


\end{exo}

\begin{exo}

La fonction $h$ est définie sur $\left[1;+\infty\right[$ par 

\[h(x)=(x-6)\sqrt{x}.\]
 
 
 On utilise la formule pour la dérivée d'un produit avec
\begin{align*}
&u(x)=x-6&&,&& v(x)=\sqrt{x}, \\
& u'(x)=1&&, &&v'(x)=\frac{1}{2\sqrt{x}}.\\
\end{align*}

On obtient, pour tout $x\in \left[1;+\infty\right[~:$
\begin{align*}h'(x)&=u'(x)\times v(x)+u(x)\times v'(x)\\&=1\times\sqrt{x}+(x-6)\times\frac{1}{2\sqrt{x}}\\&=\frac{\sqrt{x}\times 2\sqrt{x}}{2\sqrt{x}}+\frac{x-6}{2\sqrt{x}}\\&=\frac{2x}{2\sqrt{x}}+\frac{x-6}{2\sqrt{x}}\\&=\frac{3x-6}{2\sqrt{x}}.\end{align*}

\medskip

\begin{itemize}
\item[\textbullet] On résout rapidement~:
\[3x-6=0\iff 3x=6\iff x=\frac{6}{3}=2.\]
\item[\textbullet] Dans $3x-6,$ $a=3$ $\oplus$ , donc \fbox{$-~\upphi~+$}
\item[\textbullet] $2\sqrt{x}$ est strictement positif pour tout $x\in \left[1;+\infty\right[.$
\end{itemize}

\medskip

On a donc le tableau~:

\medskip

\setlength{\columnseprule}{1pt}

\begin{multicols}{2}
\begin{center}
\begin{tikzpicture}[scale=0.8]
\tkzTabInit{$x$/1,$3x-6$/1,$2\sqrt{x}$/1,$h'\left(x\right)$/1,$h\left(x\right)$/2}{$1$,$2$,$+\infty$}
\tkzTabLine{,-,z,+,}
\tkzTabLine{,+,,+,}
\tkzTabLine{,-,z,+,}
\tkzTabVar{+/$-5$,-/$-4\sqrt{2} $,+/}
\end{tikzpicture}
\end{center}

\columnbreak

\begin{itemize}
\item[\textbullet] $h(1)=(1-6)\times\sqrt{1}=-5\times 1=-5~;$
\item[\textbullet] $h(2)=(2-6)\times\sqrt{2}=-4\sqrt{2}.$
\end{itemize}

\end{multicols}

\end{exo}




\end{document}
