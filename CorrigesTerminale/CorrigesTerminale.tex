\documentclass[10pt]{article}
\usepackage[T1]{fontenc}
\usepackage[utf8]{inputenc}
\usepackage{fourier}
\usepackage[scaled=0.875]{helvet}
\renewcommand{\ttdefault}{lmtt}
\usepackage{amsmath,amssymb,makeidx}
\usepackage[normalem]{ulem}
\usepackage{fancybox}
\usepackage{cancel}
\usepackage{stmaryrd}
\usepackage{ulem}
\usepackage{tabularx}
\usepackage{geometry}
\usepackage{enumerate}
\geometry{hmargin=1.5cm,vmargin=1.5cm}
\usepackage{dcolumn}
\usepackage{textcomp}
\usepackage{lscape}
\usepackage{eurosym}
%\newcommand{\euro}{\eurologo{}}
\usepackage[dvips]{color}
\usepackage[all]{xy}

\usepackage{tikz,tkz-tab}

\usepackage{systeme}
\usepackage{ upgreek }


\usepackage{pstricks,pst-plot,pst-text,pst-tree,pstricks-add}
\usepackage{colortbl}
\usepackage{diagbox}
\usepackage{fontawesome5}
\usepackage{pifont}
\usepackage{wasysym}


\usepackage{theorem}
\theorembodyfont{\upshape}
\newtheorem{exo}{Exercice}
%\newtheorem{exo}{Exercice}%[section]
\usepackage{hyperref}
\hypersetup{
    colorlinks=true,       % false: liens encadrés; true: liens colorés
    linkcolor=blue,          % couleur des liens (ou bordures) internes
}

%\setlength{\voffset}{-1,5cm}
\usepackage{fancyhdr} 
\usepackage{graphicx}
\usepackage[frenchb]{babel}
\usepackage[np]{numprint}
\usepackage{multicol}
\usepackage{xlop}
\usepackage{soul}
\usepackage{etoolbox}
\usepackage{multirow}
\usepackage{diagbox}

\usepackage{tcolorbox}

\usepackage{xcolor}
\usepackage{stackengine}
    \setstackEOL{\\}
    
    \usepackage{listings}
\lstset{numbers=left, stepnumber=1}

\makeatletter
\patchcmd{\ttlh@hang}{\parindent\z@}{\parindent\z@\leavevmode}{}{}
\patchcmd{\ttlh@hang}{\noindent}{}{}{}
\makeatother
\setlength{\columnseprule}{0.0pt}

\usepackage{listings}
\lstset{%
  language=Python,
  basicstyle   = \ttfamily,
  keywordstyle =    \color{magenta},
  keywordstyle = [2]\color{orange},
  commentstyle =    \color{gray}\itshape,
  stringstyle  =    \color{cyan},
  numbers      = none,
  frame        = single,
  framesep     = 2pt,
  aboveskip    = 1ex
}

\title{Mathématiques -- Terminale spécialité}

\date{Corrigés des exercices}
\begin{document}
\setlength\parindent{0mm}
\renewcommand \footrulewidth{.2pt}

\maketitle

\tableofcontents


\newpage

\section{Compléments sur la dérivation}


\begin{exo}

La fonction $f$ est définie sur l'intervalle $\left[-2;6\right]$ par

\[ f(x) = 0,5x^2-2x-4.\]

\medskip

Pour tout $x\in\mathbb{R}~:$
\[f'(x)=0,5\times 2x-2\times 1-0=x-2.\]

La dérivée est du premier degré, donc pour obtenir le tableau de signe, il faut résoudre une équation, puis regarder le signe de $a~:$
\begin{align*}x-2&=0\\
 x-\cancel{2}+\cancel{2}&=0+2\\
 x&=2.
 \end{align*}

$a=1$ (puisque $x-2$ signifie $\textcolor{red}{1}x-2$), $a$ est $\oplus$ donc le signe est de la forme \fbox{$-~\upphi~+$}

\medskip


On en déduit le tableau de signe de $f'$ et le tableau de variations de $f~:$


\medskip

\setlength{\columnseprule}{1pt}

\begin{multicols}{2}

\begin{center}
\begin{tikzpicture}[scale=0.8]
\tkzTabInit{$x$/1,$f'(x)$/1,$f(x)$/2}{$-2$,$2$,$6$}
\tkzTabLine{,-,z,+,}
\tkzTabVar{+/$2$,-/$-6$,+/$2$}
\end{tikzpicture}
\end{center}

\columnbreak

Pour compléter l'extrémité des flèches, on calcule~:

\begin{itemize}
\item[\textbullet] $f(-2)=0,5\times (-2)^2-2\times (-2)-4=2$
\item[\textbullet] $f(2)=0,5\times 2^2-2\times 2-4=-6$
\item[\textbullet] $f(6)=0,5\times 6^2-2\times 6-4=2$
\end{itemize}

\medskip

On peut aussi faire un tableau de valeurs à la calculatrice.


\end{multicols}

\medskip

\textbf{Remarque~:} La courbe représentative est une parabole, dont le sommet $S$ a pour coordonnées $(2;-6).$


\begin{center}
\psset{xunit=1.0cm,yunit=0.5cm,algebraic=true,dimen=middle,dotstyle=o,dotsize=5pt 0,linewidth=2.pt,arrowsize=3pt 2,arrowinset=0.25}
\begin{pspicture*}(-2.36,-6.24)(6.46,2.42)
\multips(0,-6)(0,1.0){9}{\psline[linestyle=dashed,linecap=1,dash=1.5pt 1.5pt,linewidth=0.4pt,linecolor=lightgray]{c-c}(-2.36,0)(6.46,0)}
\multips(-2,0)(1.0,0){9}{\psline[linestyle=dashed,linecap=1,dash=1.5pt 1.5pt,linewidth=0.4pt,linecolor=lightgray]{c-c}(0,-6.24)(0,2.42)}
\psaxes[labelFontSize=\scriptstyle,xAxis=true,yAxis=true,Dx=1.,Dy=1.,ticksize=-2pt 0,subticks=2]{->}(0,0)(-2.36,-6.24)(6.46,2.42)
\rput{0.}(2.,-6.){\psplot[linewidth=2.pt,linecolor=blue]{-4.}{4.}{x^2/2/1.}}
\psdots[dotstyle=*,linecolor=red](2.,-6.)
\rput[bl](2.08,-5.8){\red{$S$}}
\end{pspicture*}
\end{center}


\end{exo}

\begin{exo}

On considère un segment $\left[AB\right]$ de longueur 4 et un point mobile $M$ pouvant se déplacer librement sur ce segment.

\begin{center}
\psset{xunit=1cm,yunit=1cm,algebraic=true,dimen=middle,dotstyle=o,dotsize=3pt 0,linewidth=0.8pt,arrowsize=3pt 2,arrowinset=0.25}
\begin{pspicture*}(0.6,0.27)(5.41,1.85)
\psline[linewidth=1.2pt](1,1)(5,1)
\rput[tl](2.92,1.5){$4$}
\rput[tl](1.5,0.7){$x$}
\psline{->}(3.2,1.4)(5,1.4)
\psline{->}(2.8,1.4)(1,1.4)
\psline{->}(1.4,0.6)(1,0.6)
\psline{->}(1.8,0.6)(2.2,0.6)
\psdots[dotstyle=*](1,1)
\rput[bl](0.85,1.08){$A$}
\psdots[dotstyle=*](5,1)
\rput[bl](5.06,1.08){$B$}
\psdots[dotstyle=*](2.2,1)
\rput[bl](2.26,1.08){$M$}
\end{pspicture*}
\end{center}
 

On note  $x$ la longueur du segment $\left[AM\right]$  et $f(x)$  le produit des longueurs $AM\times BM.$

\begin{enumerate}
\item $BM=AB-AM=4-x,$ donc
\begin{align*}
f(x)&=AM\times BM\\
&=x\times (4-x)\\
&=x\times 4+x\times (-x)\\
&=4x-x^2.
\end{align*}
\item Le produit des longueurs  $AM\times BM$ est donné par $f(x),$ donc maximiser ce produit revient à maximiser la fonction $f.$ On étudie donc les variations~ : pour tout $x\in\left[0;4\right],$
\[f'(x)=4\times 1-2x=-2x+4.\]

On résout~:

\begin{align*}-2x+4&=0\\
 -2x+\cancel{4}-\cancel{4}&=0-4\\
 \frac{\cancel{-2}x}{\cancel{-2}}&=\frac{-4}{-2}\\
 x&=2.
 \end{align*}

$a=-2,$ $a$ est $\ominus$ donc le signe est de la forme \fbox{$+~\upphi~-$}

\medskip


On obtient le tableau de signe de $f'$ et le tableau de variations de $f~:$


\medskip

\setlength{\columnseprule}{1pt}

\begin{multicols}{2}

\begin{center}
\begin{tikzpicture}[scale=0.8]
\tkzTabInit{$x$/1,$f'(x)$/1,$f(x)$/2}{$0$,$2$,$4$}
\tkzTabLine{,+,z,-,}
\tkzTabVar{-/,+/,-/}
\end{tikzpicture}
\end{center}

\columnbreak

Il n'est pas utile ici de compléter l'extrémité des flèches~: tout ce qui nous intéresse, c'est la valeur de $x$ pour laquelle $f$ atteint son maximum.
\end{multicols}

Conclusion~: $f$ atteint son maximum lorsque $x=2,$ donc le produit $AM\times BM$ est maximal lorsque $x=2~;$ c'est-à-dire quand $M$ est le milieu de $\left[AB\right].$
\end{enumerate}

\medskip

\textbf{Remarque~:} Cet exemple est celui qu'a choisi Fermat vers 1637 pour exposer sa méthode de l'adégalité -- ancêtre de la dérivation -- pour déterminer le maximum et le minimum d'une fonction.
\end{exo}

\begin{exo}

La fonction $g$ est définie sur $\mathbb{R}$ par 

\[g(x)=0,5x^3+0,75x^2-3x-1.\]

\medskip

Pour tout $x\in\mathbb{R}~:$

\[g'(x)=0,5\times 3x^2+0,75\times 2x-3\times 1-0=1,5x^2+1,5x-3.\]

La dérivée est du second degré, donc on utilise la méthode de la classe de première~:

\begin{itemize}
\item[\textbullet] $a=1,5,$ $b=1,5,$ $c=-3.$
\item[\textbullet] le discriminant est $\Delta=b^2-4ac=1,5^2-4\times 1,5\times (-3)=20,25.$
\item[\textbullet] $\Delta>0,$ donc il y a deux racines~:

\begin{align*}x_1&=\frac{-b-\sqrt{\Delta}}{2a}=\frac{-1,5-\sqrt{20,25}}{2\times 1,5}=\frac{-1,5-4,5}{3}=\frac{-6}{3}=-2,\\
x_2&=\frac{-b+\sqrt{\Delta}}{2a}=\frac{-1,5+\sqrt{20,25}}{2\times 1,5}=\frac{-1,5+4,5}{3}=\frac{3}{3}=1.
\end{align*}
\end{itemize}

\medskip

$a=1,5$  $a$ est $\oplus$ donc le signe est de la forme \fbox{$+~\upphi~-~\upphi~+$}

\medskip

\setlength{\columnseprule}{1pt}

\begin{multicols}{2}
\begin{center}
\begin{tikzpicture}[scale=0.7]
\tkzTabInit{$x$/1,$g'(x)$/1,$g(x)$/2}{$-\infty$,$-2$,$1$,$+\infty$}
\tkzTabLine{,+,z,-,z,+}
\tkzTabVar{-/,+/$4$,-/$-2.75$,+/}
\end{tikzpicture}
\end{center}

\columnbreak

\begin{itemize}
\item[\textbullet] $g(-2)=0,5\times(-2)^3+0,75\times (-2)^2-3\times (-2)-1=4$
\item[\textbullet] $g(1)=0,5\times 1^3+0,75\times 1^2-3\times 1-1=-2,75$
\end{itemize}


\end{multicols}

 \medskip
 
 \textbf{Remarque~:} Voici à quoi ressemble la courbe représentative~:
 

\begin{center}
\psset{xunit=1.0cm,yunit=0.5cm,algebraic=true,dimen=middle,dotstyle=o,dotsize=5pt 0,linewidth=2.pt,arrowsize=3pt 2,arrowinset=0.25}
\begin{pspicture*}(-4.08,-4.14)(4.,4.44)
\multips(0,-4)(0,1.0){9}{\psline[linestyle=dashed,linecap=1,dash=1.5pt 1.5pt,linewidth=0.4pt,linecolor=lightgray]{c-c}(-4.08,0)(4.,0)}
\multips(-4,0)(1.0,0){9}{\psline[linestyle=dashed,linecap=1,dash=1.5pt 1.5pt,linewidth=0.4pt,linecolor=lightgray]{c-c}(0,-4.14)(0,4.44)}
\psaxes[labelFontSize=\scriptstyle,xAxis=true,yAxis=true,Dx=1.,Dy=1.,ticksize=-2pt 0,subticks=2]{->}(0,0)(-4.08,-4.14)(4.,4.44)
\psplot[linewidth=2.pt,linecolor=blue,plotpoints=200]{-4.08}{4.0}{0.5*x^(3.0)+0.75*x^(2.0)-3.0*x-1.0}
\end{pspicture*}
\end{center}


\end{exo}

\begin{exo}

La fonction $h$ est définie sur $\left[1;+\infty\right[$ par 

\[h(x)=(x-6)\sqrt{x}.\]
 
 
 On utilise la formule pour la dérivée d'un produit avec
\begin{align*}
&u(x)=x-6&&,&& v(x)=\sqrt{x}, \\
& u'(x)=1&&, &&v'(x)=\frac{1}{2\sqrt{x}}.\\
\end{align*}

On obtient, pour tout $x\in \left[1;+\infty\right[~:$
\begin{align*}h'(x)&=u'(x)\times v(x)+u(x)\times v'(x)\\&=1\times\sqrt{x}+(x-6)\times\frac{1}{2\sqrt{x}}\\&=\frac{\sqrt{x}\times 2\sqrt{x}}{2\sqrt{x}}+\frac{x-6}{2\sqrt{x}}\\&=\frac{2x}{2\sqrt{x}}+\frac{x-6}{2\sqrt{x}}\qquad\qquad\qquad\left(\text{rappel~:}~\sqrt{x}\times\sqrt{x}=\sqrt{x}^2=x\right)\\&=\frac{3x-6}{2\sqrt{x}}.\end{align*}

\medskip

\begin{itemize}
\item[\textbullet] On résout rapidement~:
\[3x-6=0\iff 3x=6\iff x=\frac{6}{3}=2.\]
\item[\textbullet] Dans $3x-6,$ $a=3$ $\oplus$ , donc \fbox{$-~\upphi~+$}
\item[\textbullet] $2\sqrt{x}$ est strictement positif pour tout $x\in \left[1;+\infty\right[.$
\end{itemize}

\medskip

On a donc le tableau~:

\medskip

\setlength{\columnseprule}{1pt}

\begin{multicols}{2}
\begin{center}
\begin{tikzpicture}[scale=0.8]
\tkzTabInit{$x$/1,$3x-6$/1,$2\sqrt{x}$/1,$h'\left(x\right)$/1,$h\left(x\right)$/2}{$1$,$2$,$+\infty$}
\tkzTabLine{,-,z,+,}
\tkzTabLine{,+,,+,}
\tkzTabLine{,-,z,+,}
\tkzTabVar{+/$-5$,-/$-4\sqrt{2} $,+/}
\end{tikzpicture}
\end{center}

\columnbreak

\begin{itemize}
\item[\textbullet] $h(1)=(1-6)\times\sqrt{1}=-5\times 1=-5~;$
\item[\textbullet] $h(2)=(2-6)\times\sqrt{2}=-4\sqrt{2}.$
\end{itemize}

\end{multicols}

\end{exo}




\begin{exo}

La fonction $f$ est définie sur $\left[1;4\right]$ par $f(x)=x+\dfrac{4}{x}-3.$ On note $\mathcal{C}$ sa courbe représentative, $A,$ $B,$ $C$ les points de $\mathcal{C}$ d'abscisses respectives 1, 2, 4~; et $T_A,$ $T_B,$ $T_C$ les tangentes à $\mathcal{C}$ en ces points.
 
\begin{enumerate}
\item Pour dériver, le plus simple est de réécrire $f(x)$ sous la forme \[f(x)=x+4\times\dfrac{1}{x}-3.\] On obtient alors, pour tout $x\in \left[1;4\right]~:$
\begin{align*}
f'(x)&=1+4\times\left(-\frac{1}{x^2}\right)-0\\
&=1-\frac{4}{x^2}\\
&=\frac{x^2}{x^2}-\frac{4}{x^2}\\
&=\dfrac{x^2-4}{x^2}
\end{align*} 
\item \begin{itemize}
\item[\textbullet] Les racines de $x^2-4$ sont évidentes~: ce sont $x_1=-2$ et $x_2=2.$ Seule la deuxième est dans l'intervalle $\left[1;4\right].$
\item[\textbullet] $x^2$ est strictement positif pour tout $x\in \left[1;4\right].$
\end{itemize}

On obtient donc le tableau~:

\medskip

\setlength{\columnseprule}{1pt}

\begin{multicols}{2}
\begin{center}
\begin{tikzpicture}[scale=0.8]
\tkzTabInit{$x$/1,$x^2-4$/1,$x^2$/1,$f'\left(x\right)$/1,$f\left(x\right)$/2}{$1$,$2$,$4$}
\tkzTabLine{,-,z,+,}
\tkzTabLine{,+,,+,}
\tkzTabLine{,-,z,+,}
\tkzTabVar{+/$2$,-/$1$,+/$2$}
\end{tikzpicture}
\end{center}

\columnbreak

Le signe de $x^2-4$ sur $\left]-\infty;+\infty\right[$ est de la forme \fbox{$+~\upphi~-~\upphi~+$} Mais comme on travaille sur l'intervalle $\left[1;4\right],$ il ne reste plus que la partie droite \fbox{$-~\upphi~+$}

\medskip

On calcule les valeurs aux extrémités des flèches~:

\begin{itemize}
\item[\textbullet] $f(1)=1+\frac{4}{1}-3=2~;$
\item[\textbullet] $f(2)=2+\frac{4}{2}-3=1~;$
\item[\textbullet] $f(4)=4+\frac{4}{4}-3=2.$
\end{itemize}

\end{multicols}
\item On rappelle que la tangente à la courbe en un point d'abscisse $a$ a pour équation
\[y=f'(a)(x-a)+f(a).\]

Appliquons cette formule avec $a=1$ -- puisque le point $A$ a pour abscisse $1~:$

\medskip

$f(1)=2$ (déjà calculé) et $f'(1)=\frac{1^2-4}{1^2}=\frac{-3}{1}=-3,$ donc l'équation de $T_A$ est
\begin{align*}
y&=f'(1)(x-1)+f(1)\\
y&=-3(x-1)+2\\
y&=-3x+3+2\\
y&=-3x+5.
\end{align*}

Le point $A$ a pour coordonnées $(1;2),$ puisque $f(1)=2~;$ la tangente $T_A$ passe donc par ce point. Pour la tracer, il faut placer un deuxième point (c'est une droite)~; ce que l'on peut faire de trois façons différentes~:

\begin{enumerate}[(a)]
\item L'ordonnée à l'origine est $\textcolor{red}{5}$ (puisque $T_A:y=-3x\textcolor{red}{+5}$), donc $T_A$ passe par le point de coordonnées $(0;5).$
\item Le coefficient directeur de $T_A$ est $\textcolor{blue}{-3}$ (puisque $T_A:y=\textcolor{blue}{-3}x+5$), donc en partant de $A,$ il suffit d'avancer de $1$ carreau en abscisse et de descendre de $3$ carreaux en ordonnée -- $T_A$ passe donc par le point de coordonnées $(2;-1).$
\item On calcule un deuxième point avec la formule~: par exemple, si $x=2,$ $y=-3\times 2+5=-1.$ On obtient le point de coordonnées $(2;-1)$ (le même qu'avec la méthode (b)) et on trace la tangente.
\end{enumerate}

\item \begin{itemize}
\item[\textbullet] $f(2)=1$ et $f'(2)=\frac{2^2-4}{2^2}=\frac{0}{4}=0,$ donc l'équation de $T_B$ est
\begin{align*}
y&=f'(2)(x-2)+f(2)\\
y&=0(x-1)+1\\
y&=1.
\end{align*}

Le coefficient directeur étant égal à 0, la tangente $T_B$ est horizontale.
\item[\textbullet]  $f(4)=2$ et $f'(4)=\frac{4^2-4}{4^2}=\frac{12}{16}=0,75,$ donc l'équation de $T_C$ est
\begin{align*}
y&=f'(4)(x-4)+f(4)\\
y&=0,75(x-4)+2\\
y&=0,75x-3+2\\
y&=0,75x-1.\end{align*}

On trace la tangente $T_C$ par la même méthode que $T_A$ (le plus simple et le plus précis est d'utiliser l'ordonnée à l'origine).
\end{itemize}

\item On place les points $A,$ $B,$ $C,$ on trace les trois tangentes et on construit la courbe de la fonction $f$ (en bleu) en s'appuyant sur ces tangentes.


\begin{center}
\newrgbcolor{ffxfqq}{1. 0.4980392156862745 0.}
\psset{xunit=1.0cm,yunit=1.0cm,algebraic=true,dimen=middle,dotstyle=o,dotsize=5pt 0,linewidth=2.pt,arrowsize=3pt 2,arrowinset=0.25}
\begin{pspicture*}(-1.32,-1.88)(6.02,5.6)
\multips(0,-1)(0,1.0){8}{\psline[linestyle=dashed,linecap=1,dash=1.5pt 1.5pt,linewidth=0.4pt,linecolor=lightgray]{c-c}(-1.32,0)(6.02,0)}
\multips(-1,0)(1.0,0){8}{\psline[linestyle=dashed,linecap=1,dash=1.5pt 1.5pt,linewidth=0.4pt,linecolor=lightgray]{c-c}(0,-1.88)(0,5.6)}
\psaxes[labelFontSize=\scriptstyle,xAxis=true,yAxis=true,Dx=1.,Dy=1.,ticksize=-2pt 0,subticks=2]{->}(0,0)(-1.32,-1.88)(6.02,5.6)
\psline[linewidth=2.pt,linecolor=ffxfqq](0.,5.)(2.,-1.)
\rput[tl](2.1,-0.64){\ffxfqq{$T_A$}}
\psline[linewidth=2.pt,linecolor=green](0.,1.)(4.,1.)
\rput[tl](3.26,0.72){\green{$T_B$}}
\psplot[linewidth=2.pt,linecolor=magenta]{0.}{6.02}{(-4.--3.*x)/4.}
\rput[tl](0.48,-0.84){\magenta{$T_C$}}
\psplot[linewidth=2.pt,linecolor=blue,plotpoints=200]{1}{4}{x+4.0/x-3.0}
\rput[bl](1.08,2.2){\ffxfqq{$A$}}
\rput[bl](2.08,1.2){\green{$B$}}
\rput[bl](3.74,2.28){\magenta{$C$}}
\psdots[dotstyle=*,linecolor=ffxfqq](1.,2.)
\psdots[dotstyle=*,linecolor=green](2.,1.)
\psdots[dotstyle=*,linecolor=magenta](4.,2.)
\end{pspicture*}
\end{center}

\end{enumerate}

\end{exo}

\begin{exo}

La fonction $i$ est définie sur $\mathbb{R}$ par 

\[i(x)=\frac{2x}{x^2+1}.\]
 
\begin{enumerate}
\item On utilise la formule pour la dérivée d'un quotient avec
\begin{align*}
&u(x)=2x&&,&& v(x)=x^2+1, \\
& u'(x)=2&&, &&v'(x)=2x.\\
\end{align*}

On obtient, pour tout $x\in \mathbb{R}~:$
\begin{align*}i'(x)&=\frac{u'(x)\times v(x)-u(x)\times v'(x)}{(v(x))^2}
\\&=\frac{2\times\left(x^2+1\right)-2x\times 2x }{\left(x^2+1\right)^2}
\\&=\frac{2x^2+2-4x^2}{\left(x^2+1\right)^2}
\\&=\frac{-2x^2+2}{\left(x^2+1\right)^2}
.
\end{align*}


\item \begin{itemize}
\item[\textbullet] Les racines de $-2x^2+2$ sont assez évidentes~: 
\[-2x^2+2=0\iff 2=2x^2\iff 1=x^2\iff \left(x=1~\text{ou}~x=-1\right).\] 
\item[\textbullet] $\left(x^2+1\right)^2$ est strictement positif pour tout réel $x.$
\end{itemize}

On obtient donc le tableau~:

\medskip

\setlength{\columnseprule}{1pt}

\begin{multicols}{2}
\begin{center}
\begin{tikzpicture}[scale=0.7]
\tkzTabInit{$x$/1,$-2x^2+2$/1,$\left(x^2+1\right)^2$/1,$i'\left(x\right)$/1,$i\left(x\right)$/2}{$-\infty$,$-1$,$1$,$+\infty$}
\tkzTabLine{,-,z,+,z,-}
\tkzTabLine{,+,,+,,+}
\tkzTabLine{,-,z,+,z,-}
\tkzTabVar{+/,-/$-1$,+/$1$,-/}
\end{tikzpicture}
\end{center}

\columnbreak




\begin{itemize}
\item[\textbullet] $i(-1)=\frac{2\times(-1)}{(-1)^2+1}=\frac{-2}{2}=-1~;$
\item[\textbullet] $i(1)=\frac{2\times 1)}{1^2+1}=\frac{2}{2}=1.$
\end{itemize}

\end{multicols}


\item
\begin{enumerate}
\item $i(0)=\frac{2\times 0}{0^2+1}=\frac{0}{1}=0$ et $i'(0)=\frac{-2\times 0^2+2}{\left(0^2+1\right)^2}=\frac{2}{1}=2,$ donc l'équation de $(T)$ est
\begin{align*}
y&=f'(0)(x-0)+f(0)\\
y&=2x+0\\
y&=2x.\end{align*}

\item Pour étudier les positions relatives de $(C):y=\frac{2x}{x^2+1}$ et $(T):y=2x,$ on étudie \textbf{le signe de la différence}~:
\[\frac{2x}{x^2+1}-2x.\]
\begin{itemize}
\item[\textbullet] Pour les valeurs de $x$ pour lesquelles cette différence vaut 0, les deux courbes se coupent~;
\item[\textbullet] pour les valeurs de $x$ pour lesquelles cette différence est strictement positive, $(C)$ est au-dessus de $(T)~;$
\item[\textbullet] pour les valeurs de $x$ pour lesquelles cette différence est strictement négative, $(C)$ est en-dessous de $(T).$
\end{itemize}


\medskip

On commence par calculer la différence~:

\begin{align*}
\frac{2x}{x^2+1}-2x
&= \frac{2x}{x^2+1}-\frac{2x\left(x^2+1\right)}{x^2+1}
\\&=\frac{2x}{x^2+1}-\frac{2x^3+2x}{x^2+1}
\\&=\frac{\cancel{2x}-2x^3-\cancel{2x}}{x^2+1}
\\&=\frac{-2x^3}{x^2+1}.
\end{align*}

\medskip

\footnotesize
\setlength{\columnseprule}{1pt}

\begin{multicols}{2}

\begin{center}
\hspace*{-1cm}
\begin{tikzpicture}[scale=1]
\tkzTabInit{$x$/1,$-2x^3$/1,$\left(x^2+1\right)^2$/1,$\frac{-2x^3}{x^2+1}$/1,Positions relatives des courbes/3}{$-\infty$,$0$,$+\infty$}
\tkzTabLine{,+,z,-,}
\tkzTabLine{,+,,+,}
\tkzTabLine{,+,z,-,}
\tkzTabLine{,(C)\text{ au-dessus de }(T),\scriptsize{\Longstack{S\\e\\ \\c\\o\\u\\p\\e\\n\\t}},(C)\text{ en-dessous de }(T),}
\end{tikzpicture}
\end{center}
\normalsize

\columnbreak

Pour compléter le tableau de signe~:

\begin{itemize}
\item[\textbullet] $-2x^3=0$ lorsque $x=0~;$
\item[\textbullet] $-2x^3$ est $\ominus$ lorsque $x$ est strictement positif~;
\item[\textbullet] $-2x^3$ est $\oplus$ lorsque $x$ est strictement négatif~;
\item[\textbullet] $\left(x^2+1\right)^2$ est strictement positif pour tout réel $x.$
\end{itemize}

\end{multicols}
\end{enumerate}
\item ~{}


\begin{center}
\psset{xunit=1.0cm,yunit=1.0cm,algebraic=true,dimen=middle,dotstyle=o,dotsize=5pt 0,linewidth=2.pt,arrowsize=3pt 2,arrowinset=0.25}
\begin{pspicture*}(-5.92,-2.46)(5.98,2.64)
\multips(0,-2)(0,1.0){6}{\psline[linestyle=dashed,linecap=1,dash=1.5pt 1.5pt,linewidth=0.4pt,linecolor=lightgray]{c-c}(-5.92,0)(5.98,0)}
\multips(-5,0)(1.0,0){12}{\psline[linestyle=dashed,linecap=1,dash=1.5pt 1.5pt,linewidth=0.4pt,linecolor=lightgray]{c-c}(0,-2.46)(0,2.64)}
\psaxes[labelFontSize=\scriptstyle,xAxis=true,yAxis=true,Dx=1.,Dy=1.,ticksize=-2pt 0,subticks=2]{->}(0,0)(-5.92,-2.46)(5.98,2.64)
\psplot[linewidth=2.pt,linecolor=blue,plotpoints=200]{-5.919999999999999}{5.979999999999995}{2.0*x/(x^(2.0)+1.0)}
\psplot[linewidth=2.pt,linecolor=red]{-5.92}{5.98}{(-0.--2.*x)/1.}
\rput[tl](1.22,1.9){\red{$(T)$}}
\rput[tl](3.44,1.04){\blue{$(C)$}}
\end{pspicture*}
\end{center}
\end{enumerate}

\end{exo}

\begin{exo}

La distance (en m) parcourue au temps $t$ (en s) par une pierre en chute libre est $d(t)=5t^2.$

On lance cette pierre d'une hauteur de 20~m.



\begin{enumerate}
\item La pierre arrive au sol quand elle a parcouru 20~m. Il faut donc résoudre l'équation $5t^2=20~:$
\[5t^2=20\iff t^2=\frac{20}{5}\iff t^2=4\iff \left(t=2\quad\text{ou}\quad\underbrace{t=-2}_{\text{impossible}}\right).\]

Conclusion~: la pierre arrive au sol après 2~s.
\item On construit la courbe à partir d'un tableau de valeurs (avec un pas de 0,4 par exemple).

\begin{center}
\begin{tabular}{|l|c|c|c|c|c|c|}
\hline
   $t$ &$0$ &$0,4$ &$0,8$ &$1,2$ &$1,6$&$2$ \\
	\hline
	$d(t)$ &$0$ &$0,8$ &$3,2$ &$7,2$ &$12,8$&$20$ \\
	\hline
\end{tabular}
\end{center}

Pour obtenir ce tableau, on utilise la calculatrice (bien sûr, on met des $x$ à la place des $t$)~:

\medskip

\small

\setlength{\columnseprule}{1pt}
\begin{multicols}{4}

\begin{center}\textbf{Calculatrices collège}\end{center}

\medskip


\begin{itemize}
\item[\textbullet] \fbox{MODE}
\item[\textbullet] 4 : TABLE ou 4 : Tableau
\item[\textbullet] f(X)=$5\text{X}^2$ \fbox{EXE}

(si on demande g(X)=, ne rien rentrer)
\item[\textbullet] Début? $0$ \fbox{EXE}
\item[\textbullet] Fin? $2$ \fbox{EXE}
\item[\textbullet] Pas? $0,4$ \fbox{EXE}
\end{itemize}

\columnbreak

\begin{center}\textbf{NUMWORKS}\end{center}

\medskip

x s'obtient avec les touches  \fbox{alpha}  \fbox{x}

\begin{itemize}
\item[\textbullet] \fbox{\textcolor{yellow}{\faHome}}
\item[\textbullet] Fonctions \fbox{EXE} puis choisir Fonctions \fbox{EXE}
\item[\textbullet] f(x)=$5\text{x}^2$ \fbox{EXE}
\item[\textbullet] choisir Tableau \fbox{EXE} puis Régler l'intervalle \fbox{EXE}

\item[\textbullet] X début\qquad$0$ \fbox{EXE}
\item[\textbullet] X fin\qquad $2$ \fbox{EXE}
\item[\textbullet] Pas\qquad $0.4$ \fbox{EXE}
\item[\textbullet] choisir Valider
\end{itemize}

\columnbreak

\begin{center}\textbf{TI graphiques}\end{center}

\medskip


X s'obtient avec la touche \fbox{$x,t,\theta,n$}
\begin{itemize}
\item[\textbullet] \fbox{$f(x)$}
\item[\textbullet] $\text{Y}_1=5\text{X}^2$ \fbox{EXE}
\item[\textbullet] \fbox{2nde} \fbox{déf table}
\item[\textbullet] DébTable=$0$ \fbox{EXE}
\item[\textbullet] PasTable=$0.4$ \fbox{EXE}

ou

\tiny{$\Delta$}\normalsize Tbl=$0.4$ \fbox{EXE}
\item[\textbullet] \fbox{2nde} \fbox{table}
\end{itemize}

\columnbreak

\begin{center}\textbf{CASIO graphiques}\end{center}

\medskip


X s'obtient avec la touche \fbox{$\text{X},\theta,\text{T}$}
\begin{itemize}
\item[\textbullet] \fbox{MENU} puis choisir TABLE \fbox{EXE}
\item[\textbullet] $\text{Y}_1:5\text{X}^2$ \fbox{EXE}
\item[\textbullet] \fbox{F5} (on choisit donc SET)
\item[\textbullet] Start:$0$ \fbox{EXE}
\item[\textbullet] End:$2$ \fbox{EXE}
\item[\textbullet] Step:$0.4$ \fbox{EXE}
\item[\textbullet]\fbox{EXIT}
\item[\textbullet] \fbox{F6} (on choisit donc TABLE)
\end{itemize}
\end{multicols}

\normalsize


\begin{center}
\psset{xunit=2.5cm,yunit=0.25cm,algebraic=true,dimen=middle,dotstyle=o,dotsize=5pt 0,linewidth=2.pt,arrowsize=3pt 2,arrowinset=0.25}
\begin{pspicture*}(-0.28472279556053043,-2.064821322818694)(2.336654362677241,21.51292853619145)
\multips(0,0)(0,4.0){6}{\psline[linestyle=dashed,linecap=1,dash=1.5pt 1.5pt,linewidth=0.4pt,linecolor=lightgray]{c-c}(0,0)(2.336654362677241,0)}
\multips(0,0)(0.4,0){7}{\psline[linestyle=dashed,linecap=1,dash=1.5pt 1.5pt,linewidth=0.4pt,linecolor=lightgray]{c-c}(0,0)(0,21.51292853619145)}
\psaxes[labelFontSize=\scriptstyle,xAxis=true,yAxis=true,Dx=0.4,Dy=4.,ticksize=-2pt 0,subticks=2]{->}(0,0)(0.,0.)(2.336654362677241,21.51292853619145)
\rput[lt](1.8,2.9){\parbox{1.2343130420771191 cm}{temps \\  (en s)}}
\rput[lt](0.06674676755514843,19.46268941801665){\parbox{1.3124173894361588 cm}{distance \\  (en m)}}
\rput{0.}(0.,0.){\psplot[linewidth=2.pt,linecolor=blue]{0.}{2.}{x^2/2/0.1}}
\psplot[linewidth=2.pt,linecolor=red]{-0.28472279556053043}{2.336654362677241}{(-10.--10.*x)/0.5}
\psdots[dotsize=4pt 0,dotstyle=*,linecolor=blue](0.4,0.8)
\psdots[dotsize=4pt 0,dotstyle=*,linecolor=blue](0.8,3.2)
\psdots[dotsize=4pt 0,dotstyle=*,linecolor=blue](1.2,7.2)
\psdots[dotsize=4pt 0,dotstyle=*,linecolor=blue](1.6,12.8)
\psdots[dotsize=4pt 0,dotstyle=*,linecolor=red](2.,20.)
\end{pspicture*}
\end{center}


\item La vitesse de la pierre au moment de l'impact au sol est $d'(2).$

Or $d'(t)=5\times 2t=10t,$ donc $d'(2)=10\times 2=20.$ Ainsi la vitesse au moment de l'impact est de 20~m/s.

\medskip

\textbf{Remarques~:}

\begin{itemize}
\item[\textbullet] cette vitesse instantanée est le coefficient directeur de la tangente au point $A$ d'abscisse 2 (en rouge).
\item[\textbullet] la \og vraie formule \fg~{}(valable en l'absence de frottements) est $d(t)=4,9t^2.$ Dans l'exercice, on a pris 5 au lieu de 4,9 pour simplifier les calculs.
\end{itemize}
\end{enumerate}

\end{exo}


\begin{exo}



Dans cet exercice, on utilise deux propriétés du cours~:

\begin{itemize}
\item[\textbullet] la dérivée de $x\mapsto \text{e}^{ax+b}$ est $x\mapsto a\text{e}^{ax+b}~;$
\item[\textbullet] une exponentielle est strictement positive.
\end{itemize}

\medskip



\medskip

\small

\setlength{\columnseprule}{1pt}
\begin{multicols}{4}

Pour tout $x\in \mathbb{R}~:$

\begin{align*}
f(x)&=\text{e}^{0,5x+1}\\
f'(x)&=\underbrace{0,5}_{\oplus}\underbrace{\text{e}^{0,5x+1}}_{\oplus}
\end{align*}
\medskip

$f'>0$ donc $f$ strictement croissante sur $\mathbb{R}.$


\begin{center}
\psset{xunit=0.75cm,yunit=0.75cm,algebraic=true,dimen=middle,dotstyle=o,dotsize=5pt 0,linewidth=2.pt,arrowsize=3pt 2,arrowinset=0.25}
\begin{pspicture*}(-2.94,-0.5)(2.9,5.32)
\multips(0,0)(0,1.0){6}{\psline[linestyle=dashed,linecap=1,dash=1.5pt 1.5pt,linewidth=0.4pt,linecolor=lightgray]{c-c}(-2.94,0)(2.9,0)}
\multips(-2,0)(1.0,0){6}{\psline[linestyle=dashed,linecap=1,dash=1.5pt 1.5pt,linewidth=0.4pt,linecolor=lightgray]{c-c}(0,-0.5)(0,5.32)}
\psaxes[labelFontSize=\scriptstyle,xAxis=true,yAxis=true,Dx=1.,Dy=1.,ticksize=-2pt 0,subticks=2]{->}(0,0)(-2.94,-0.5)(2.9,5.32)
\psplot[linewidth=2.pt,linecolor=red,plotpoints=200]{-2.94}{2.900000000000002}{EXP(0.5*x+1.0)}
\end{pspicture*}
\end{center}
\columnbreak

Pour tout $x\in \mathbb{R}~:$

\begin{align*}
g(x)&=\text{e}^{-1,5x}\\
g'(x)&=\underbrace{-1,5}_{\ominus}\underbrace{\text{e}^{-1,5x}}_{\oplus}
\end{align*}
\medskip

$g'<0$ donc $g$ strictement décroissante sur $\mathbb{R}.$


\begin{center}
\psset{xunit=0.75cm,yunit=0.75cm,algebraic=true,dimen=middle,dotstyle=o,dotsize=5pt 0,linewidth=2.pt,arrowsize=3pt 2,arrowinset=0.25}
\begin{pspicture*}(-2.94,-0.5)(2.9,5.32)
\multips(0,0)(0,1.0){6}{\psline[linestyle=dashed,linecap=1,dash=1.5pt 1.5pt,linewidth=0.4pt,linecolor=lightgray]{c-c}(-2.94,0)(2.9,0)}
\multips(-2,0)(1.0,0){6}{\psline[linestyle=dashed,linecap=1,dash=1.5pt 1.5pt,linewidth=0.4pt,linecolor=lightgray]{c-c}(0,-0.5)(0,5.32)}
\psaxes[labelFontSize=\scriptstyle,xAxis=true,yAxis=true,Dx=1.,Dy=1.,ticksize=-2pt 0,subticks=2]{->}(0,0)(-2.94,-0.5)(2.9,5.32)
\psplot[linewidth=2.pt,linecolor=green,plotpoints=200]{-2.94}{2.900000000000002}{EXP(-1.5*x+0.0)}
\end{pspicture*}
\end{center}
\columnbreak
Pour tout $x\in \mathbb{R}~:$

\begin{align*}
h(x)&=\text{e}^{2x-2}\\
h'(x)&=\underbrace{2}_{\oplus}\underbrace{\text{e}^{2x-2}}_{\oplus}
\end{align*}
\medskip

$h'>0$ donc $h$ strictement croissante sur $\mathbb{R}.$


\begin{center}
\psset{xunit=0.75cm,yunit=0.75cm,algebraic=true,dimen=middle,dotstyle=o,dotsize=5pt 0,linewidth=2.pt,arrowsize=3pt 2,arrowinset=0.25}
\begin{pspicture*}(-2.94,-0.5)(2.9,5.32)
\multips(0,0)(0,1.0){6}{\psline[linestyle=dashed,linecap=1,dash=1.5pt 1.5pt,linewidth=0.4pt,linecolor=lightgray]{c-c}(-2.94,0)(2.9,0)}
\multips(-2,0)(1.0,0){6}{\psline[linestyle=dashed,linecap=1,dash=1.5pt 1.5pt,linewidth=0.4pt,linecolor=lightgray]{c-c}(0,-0.5)(0,5.32)}
\psaxes[labelFontSize=\scriptstyle,xAxis=true,yAxis=true,Dx=1.,Dy=1.,ticksize=-2pt 0,subticks=2]{->}(0,0)(-2.94,-0.5)(2.9,5.32)
\psplot[linewidth=2.pt,linecolor=orange,plotpoints=200]{-2.94}{2.900000000000002}{EXP(2*x-2.0)}
\end{pspicture*}
\end{center}
\columnbreak

Pour tout $x\in \mathbb{R}~:$

\begin{align*}
i(x)&=\text{e}^{-1x+1}\\
i'(x)&=\underbrace{-1}_{\ominus}\underbrace{\text{e}^{-1x+1}}_{\oplus}
\end{align*}
\medskip

$i'<0$ donc $i$ strictement décroissante sur $\mathbb{R}.$


\begin{center}
\psset{xunit=0.75cm,yunit=0.75cm,algebraic=true,dimen=middle,dotstyle=o,dotsize=5pt 0,linewidth=2.pt,arrowsize=3pt 2,arrowinset=0.25}
\begin{pspicture*}(-2.94,-0.5)(2.9,5.32)
\multips(0,0)(0,1.0){6}{\psline[linestyle=dashed,linecap=1,dash=1.5pt 1.5pt,linewidth=0.4pt,linecolor=lightgray]{c-c}(-2.94,0)(2.9,0)}
\multips(-2,0)(1.0,0){6}{\psline[linestyle=dashed,linecap=1,dash=1.5pt 1.5pt,linewidth=0.4pt,linecolor=lightgray]{c-c}(0,-0.5)(0,5.32)}
\psaxes[labelFontSize=\scriptstyle,xAxis=true,yAxis=true,Dx=1.,Dy=1.,ticksize=-2pt 0,subticks=2]{->}(0,0)(-2.94,-0.5)(2.9,5.32)
\psplot[linewidth=2.pt,linecolor=blue,plotpoints=200]{-2.94}{2.900000000000002}{EXP(-1*x+1.0)}
\end{pspicture*}
\end{center}



\end{multicols}

\medskip

\`A titre d'illustration, on a tracé les courbes des quatre fonctions. Elles ont toutes une allure très similaire, à deux différences près~:

\begin{itemize}
\item[\textbullet] elles montent lorsque $a>0,$ elles descendent lorsque $a<0~;$
\item[\textbullet] plus $|a|$ est grand, plus la pente de la partie inclinée est forte.
\end{itemize}


\end{exo}

\begin{exo}

La fonction $f$ est définie sur l'intervalle $\left[0;4\right]$ par

\[ f(x) = (-2x+1)\text{e}^{-x}.\]

\begin{enumerate}
\item  On utilise la formule pour la dérivée d'un produit avec

\begin{align*}
&u(x)=-2x+1&&,&& v(x)=\text{e}^{-x}, \\
& u'(x)=-2&&, &&v'(x)=-\text{e}^{-x}.\\
\end{align*}

On obtient, pour tout $x\in\left[0;4\right]~:$


\begin{align*}f'(x)&=u'(x)\times v(x)+u(x)\times v'(x)\\
&=-2\times\text{e}^{-x}+\left(-2x+1\right)\times \left(-\text{e}^{-x}\right)\\
&=-2\times\text{e}^{-x}+(-2x)\times\left(-\text{e}^{-x}\right)+1\times\left(-\text{e}^{-x}\right)\\
&=-2\times\text{e}^{-x}+2x\times\text{e}^{-x}-1\times\text{e}^{-x}\\
&=\left(-2+2x-1\right)\text{e}^{-x}\\
&=\left(2x-3\right)\text{e}^{-x}.
\end{align*}

\item On étudie le signe de $f'$ et on en déduit les variations de $f~:$

\begin{itemize}
\item[\textbullet] $2x-3=0\iff 2x=3\iff x=\frac{3}{2}\iff x=1,5~;$
\item[\textbullet] $\text{e}^{-x}$ est $\oplus$ pour tout réel $x.$
\end{itemize}

\medskip

\setlength{\columnseprule}{1pt}

\begin{multicols}{2}
\begin{center}
\begin{tikzpicture}[scale=0.8]
\tkzTabInit{$x$/1,$2x-3$/1,$\text{e}^{-x}$/1,$f'(x)$/1,$f(x)$/2}{$0$,$1.5$,$4$}
\tkzTabLine{,-,z,+,}
\tkzTabLine{,+,,+,}
\tkzTabLine{,-,z,+,}
\tkzTabVar{+/$1$,-/$-2\text{e}^{-1,5}$,+/$-7\text{e}^{-4}$}
\end{tikzpicture}
\end{center}

\columnbreak

\begin{itemize}
\item[\textbullet] $f(0)=(-2\times 0 +1)\times\underbrace{\text{e}^{-0}}_{=1}=1\times 1=1$
\item[\textbullet] $f(1,5)=(-2\times 1,5 +1)\times\text{e}^{-1,5}=-2\text{e}^{-1,5}\approx -0,45$
\item[\textbullet] $f(4)=(-2\times 4 +1)\times\text{e}^{-4}=-7\text{e}^{-4}\approx -0,13$
\end{itemize}
\end{multicols}

\end{enumerate}


\end{exo}

\begin{exo}

La fonction $g$ est définie sur $\mathbb{R}$ par $g(x)=\text{e}^x-x-1.$

\medskip

Pour tout $x\in\mathbb{R}~:$
\[g'(x)=\text{e}^x-1-0=\text{e}^x-1.\]

\medskip

\setlength{\columnseprule}{1pt}
\begin{multicols}{2}

On résout l'équation~:
\[\text{e}^x-1=0\iff \text{e}^x=1 \iff x=0.\]

\danger On a utilisé la propriété~: le seul nombre dont l'exponentielle est égale à 1 est 0.

\medskip

Pour avoir les signes dans chaque case du tableau, on remplace par des valeurs de $x~:$

\begin{itemize}
\item[\textbullet] pour l'intervalle $\left]-\infty;0\right[,$ on prend (par exemple) $x=-1$ et on calcule avec la calculatrice~:
\[g'(-1)=\text{e}^{-1}-1\approx -0,63\qquad \ominus ~;\]
\item[\textbullet] pour l'intervalle $\left]0;+\infty\right[,$ on prend (par exemple) $x=1$ et on calcule avec la calculatrice~:
\[g'(1)=\text{e}^{1}-1\approx 3,72\qquad \oplus .\]
\end{itemize}

\columnbreak

\begin{center}
\begin{tikzpicture}[scale=1]
\tkzTabInit{$x$/1,$g'(x)=\text{e}^x-1$/1,$g(x)$/2}{$-\infty$,$0$,$+\infty$}
\tkzTabLine{,-,z,+,}
\tkzTabVar{+/,-/$0$,+/}
\end{tikzpicture}

\medskip

\[g(0)=\text{e}^0-0-1=1-1=0.\]
\end{center}



\end{multicols}

\medskip

\textbf{Remarque~:} Le minimum de $g$ est $0,$ donc $g(x)\geq 0$ pour tout réel $x~;$ autrement dit $\text{e}^x-x-1\geq 0.$ Cette inégalité se réécrit
\[\text{e}^x\geq x+1.\]
On obtiendra ce résultat par une autre méthode dans l'exercice 18 (utilisation de la convexité). Cette inégalité sera utilisée plus tard dans l'année, pour démontrer des résultats sur les limites.


\end{exo}

\begin{exo}


\begin{align*}
\dfrac{\text{e}^8}{\text{e}^2\times \text{e}^1\times \text{e}^3}&=\dfrac{\text{e}^8}{\text{e}^{2+1+3}}=\dfrac{\text{e}^8}{\text{e}^{6}}=\text{e}^{8-6}=\text{e}^{2}\\
\dfrac{\text{e}\times\text{e}^2}{\left(\text{e}^2\right)^2}
&=\dfrac{\text{e}^1\times\text{e}^2}{\text{e}^{2\times 2}}
=\dfrac{\text{e}^{1+2}}{\text{e}^{4}}=\text{e}^{3-4}=\text{e}^{-1}\\
\left(\text{e}^2\right)^3\times\text{e}^{-5}&=\text{e}^{2\times 3}\times\text{e}^{-5}=\text{e}^{6-5}=\text{e}^{1}
\end{align*}

\end{exo}

\begin{exo}

Dans chaque cas, on note $\mathcal{S}$ l'ensemble des solutions.

\begin{enumerate}

\item \begin{align*}
&\text{e}^{x}=-3\\
&\text{Impossible, car une exponentielle est strictement positive}\\
&\mathcal{S}=\emptyset
\end{align*}

\item \begin{align*}
&\text{e}^{2x-1}=1\\
&2x-1=0&&\text{(le seul nombre dont l'exponentielle vaut 1 est 0)}\\
&x=\frac{1}{2}\\
&\mathcal{S}=\left\{\frac{1}{2}\right\}.
\end{align*}

\item L'équation $\text{e}^{2x}+2\text{e}^{x}=3$ se réécrit
\[\left(\text{e}^x\right)^2+2\text{e}^{x}-3=0.\]

Pour résoudre, il est astucieux de noter $X=\text{e}^x~;$ l'équation se réécrit alors sous la forme
\[X^2+2X-3=0.\]
On résout avec la méthode de la classe de première~:

\begin{itemize}
\item[\textbullet] $a=1,$ $b=2,$ $c=-3.$
\item[\textbullet] le discriminant est $\Delta=b^2-4ac=2^2-4\times 1\times (-3)=16.$
\item[\textbullet] $\Delta>0,$ donc il y a deux racines~:

\begin{align*}X_1&=\frac{-b-\sqrt{\Delta}}{2a}=\frac{-2-\sqrt{16}}{2\times 1}=\frac{-2-4}{2}=\frac{-6}{2}=-3,\\
X_2&=\frac{-b+\sqrt{\Delta}}{2a}=\frac{-2+\sqrt{16}}{2\times 1}=\frac{-2+4}{2}=\frac{2}{2}=1.
\end{align*}

\medskip

On a posé $X=\text{e}^x,$ donc il y a deux possibilités~:
\[\text{e}^x=-3\qquad\text{ou}\qquad \text{e}^x=1.\] La première équation n'a pas de solution, car une exponentielle est strictement positive~; la deuxième équation a une seule solution~: $x=0.$

\medskip

Conclusion~: L'unique solution de l'équation $\text{e}^{2x}+2\text{e}^{x}=3$ est $x=0~:$
\[\mathcal{S}=\left\{0\right\}.\]


\end{itemize}



\end{enumerate}
\end{exo}

\begin{exo}

On utilisera la propriété~: pour tout nombre réel $x,$\[\text{e}^x\times \text{e}^{-x}=1.\]


\begin{enumerate}
\item D'après l'identité remarquable $(a+b)^2=a^2+2ab+b^2~:$



\[\left(\text{e}^{x}+ \text{e}^{-x}\right)^2=\left(\text{e}^{x}\right)^2+2\times \underbrace{\text{e}^{x}\times \text{e}^{-x}}_{=1}+\left( \text{e}^{-x}\right)^2=\text{e}^{2x}+2+\text{e}^{-2x}.\]
\item On multiplie le numérateur et le dénominateur par $\text{e}^x~:$
\begin{align*}
\dfrac{\text{e}^{x}-\text{e}^{-x}}{\text{e}^x+\text{e}^{-x}}
&=\dfrac{\left(\text{e}^{x}-\text{e}^{-x}\right)\times\text{e}^x}{\left(\text{e}^{x}+\text{e}^{-x}\right)\times\text{e}^x}\\
&=\dfrac{\text{e}^{x}\times \text{e}^{x}-\text{e}^{-x}\times \text{e}^{x}}{\text{e}^{x}\times \text{e}^{x}-\text{e}^{-x}\times \text{e}^{x}}\\
&=\dfrac{\text{e}^{x+x}-\text{e}^{-x+x}}{\text{e}^{x+x}+\text{e}^{-x+x}}\\
&=\dfrac{\text{e}^{2x}-\text{e}^{0}}{\text{e}^{2x}+\text{e}^{0}}\\
&=\dfrac{\text{e}^{2x}-1}{\text{e}^{2x}+1}.
\end{align*}
\end{enumerate}


\end{exo}



\begin{exo}


\begin{enumerate}
\item La fonction $f$ est de la forme $f(x)=\text{e}^{u(x)},$ avec \[u(x)=-x^2,\qquad u'(x)=-2x.\]
 On a donc, pour tout $x\in\mathbb{R}~:$ \[f'(x)=u'(x)\times\text{e}^{u(x)}=-2x\text{e}^{-x^2}.\]
\item La fonction $h$ est de la forme $h(x)=\left(u(x)\right)^n,$ avec \[u(x)=-4x+1,\qquad u'(x)=-4,\qquad n=3.\]
 On a donc, pour tout $x\in\mathbb{R}~:$ \[h'(x)=n\times u'(x)\times \left(u(x)\right)^{n-1}=3\times (-4)\times (-4x+1)^{3-1}=-12\left(-4x+1\right)^2.\]
\item La fonction $i$ est de la forme $i(x)=\text{e}^{u(x)},$ avec \[u(x)=5x-9,\qquad u'(x)=5.\]
 On a donc, pour tout $x\in\mathbb{R}~:$ \[i'(x)=u'(x)\times\text{e}^{u(x)}=5\text{e}^{5x-9}.\]
\item La fonction $j$ est de la forme $j(x)=\left(u(x)\right)^n,$ avec \[u(x)=x^2-3x,\qquad u'(x)=2x-3,\qquad n=5.\]
 On a donc, pour tout $x\in\mathbb{R}~:$ \[j'(x)=n\times u'(x)\times \left(u(x)\right)^{n-1}=5\times \left(2x-3\right)\times \left(x^2-3x\right)^{5-1}=\left(10x-15\right)\times \left(x^2-3x\right)^4.\]
\item L'énoncé nous donne \[k(x)=\sqrt{x^2-x+2}.\] Il faut se méfier~: on ne peut calculer la racine carrée d'un nombre que si celui-ci est positif~; et on ne peut dériver une fonction de la forme $\sqrt{u}$ que lorsqu'elle est strictement positive. Intéressons-nous donc au signe de $x^2-x+2~:$

\medskip

Le discriminant est $\Delta=b^2-4ac=(-1)^2-4\times 1\times 2=-7.$ Il s'ensuit qu'il n'y a pas de racine, et que  $x^2-x+2$ est strictement positif sur $\mathbb{R}.$ La fonction $k$ est donc bien définie sur $\mathbb{R},$ mais aussi dérivable.

\medskip

Elle est de la forme $k(x)=\sqrt{u(x)},$ avec  \[u(x)=x^2-x+2,\qquad u'(x)=2x-1.\]
 On a donc, pour tout $x\in\mathbb{R}~:$ \[k'(x)=\dfrac{u'(x)}{2\sqrt{u(x)}}=\dfrac{2x-1}{2\sqrt{x^2-x+2}}.\]
\end{enumerate}


\medskip

\textbf{Remarque informelle~:} On a déjà vu les dérivées suivantes dans le cours de première~:

\begin{align*}
\left(x^n\right)'&=nx^{n-1}\\
\left(\text{e}^x\right)'&=\text{e}^x\\
\left(\sqrt{x}\right)'&=\dfrac{1}{2\sqrt{x}}
\end{align*}

\medskip

Les trois nouvelles formules du cours de terminale peuvent se réécrire

\begin{align*}
\left(u^n\right)'&=nu^{n-1}\times u'\\
\left(\text{e}^u\right)'&=\text{e}^u\times u'\\
\left(\sqrt{u}\right)'&=\dfrac{1}{2\sqrt{u}}\times u'
\end{align*}

\medskip

On voit qu'il suffit de remplacer $x$ par $u,$ et de multiplier par $u'.$

\end{exo}




\begin{exo}



\begin{enumerate}

\item Pour tout $x\in\mathbb{R}~:$
\begin{align*}
f(x)&=x^2\\
f'(x)&=2x\\
f''(x)&=2.
\end{align*}

\medskip

Conclusion~: $f''$ est strictement positive, donc $f$ est convexe sur $\mathbb{R}.$

\medskip

On peut aussi présenter les choses  avec un tableau de signe~:

\begin{center}
\hspace*{-1cm}
\begin{tikzpicture}[scale=1]
\tkzTabInit{$x$/1,$f''(x)=2$/1,Convexité/3}{$-\infty$,$+\infty$}
\tkzTabLine{,+,}

\tkzTabLine{,f\text{ convexe },}
\end{tikzpicture}
\end{center}

\medskip


\begin{center}
\psset{xunit=1.0cm,yunit=1.0cm,algebraic=true,dimen=middle,dotstyle=o,dotsize=5pt 0,linewidth=2.pt,arrowsize=3pt 2,arrowinset=0.25}
\begin{pspicture*}(-2.46,-0.74)(2.68,3.4)
\multips(0,0)(0,1.0){5}{\psline[linestyle=dashed,linecap=1,dash=1.5pt 1.5pt,linewidth=0.4pt,linecolor=lightgray]{c-c}(-2.46,0)(2.68,0)}
\multips(-2,0)(1.0,0){6}{\psline[linestyle=dashed,linecap=1,dash=1.5pt 1.5pt,linewidth=0.4pt,linecolor=lightgray]{c-c}(0,-0.74)(0,3.4)}
\psaxes[labelFontSize=\scriptstyle,xAxis=true,yAxis=true,Dx=1.,Dy=1.,ticksize=-2pt 0,subticks=2]{->}(0,0)(-2.46,-0.74)(2.68,3.4)
\rput{0.}(0.,0.){\psplot[linewidth=2.pt,linecolor=blue]{-3.}{3.}{x^2/2/0.5}}
\rput[tl](0.26,2.86){\blue{convexe}}
\end{pspicture*}
\end{center}



\item Pour tout $x\in\mathbb{R}~:$
\begin{align*}
g(x)&=x^3\\
g'(x)&=3x^2\\
g''(x)&=6x.
\end{align*}

\medskip

Cette fois, le tableau de signe est fortement recommandé~:


\begin{center}
\hspace*{-1cm}
\begin{tikzpicture}[scale=1.2]
\tkzTabInit{$x$/1,$g''(x)=6x$/1,Convexité/3}{$-\infty$,$0$,$+\infty$}
\tkzTabLine{,-,z,+,}

\tkzTabLine{,g\text{ concave},\scriptsize{\Longstack{P\\t\\ \\i\\n\\f\\l\\e\\x\\i\\o\\n}},g\text{ convexe},}
\end{tikzpicture}
\end{center}

Conclusion~:

\begin{itemize}
\item[\textbullet] $g$ est concave sur $\left]-\infty;0\right]~;$
\item[\textbullet] $g$ est convexe sur $\left[0;+\infty\right[~;$
\item[\textbullet] le point de coordonnées $(0;0)$ est un point d'inflexion.
\end{itemize}

\medskip


\begin{center}
\newrgbcolor{ffxfqq}{1. 0.4980392156862745 0.}
\psset{xunit=1.0cm,yunit=1.0cm,algebraic=true,dimen=middle,dotstyle=o,dotsize=5pt 0,linewidth=2.pt,arrowsize=3pt 2,arrowinset=0.25}
\begin{pspicture*}(-2.46,-1.76)(3.44,1.88)
\multips(0,-1)(0,1.0){4}{\psline[linestyle=dashed,linecap=1,dash=1.5pt 1.5pt,linewidth=0.4pt,linecolor=lightgray]{c-c}(-2.46,0)(3.44,0)}
\multips(-2,0)(1.0,0){6}{\psline[linestyle=dashed,linecap=1,dash=1.5pt 1.5pt,linewidth=0.4pt,linecolor=lightgray]{c-c}(0,-1.76)(0,1.88)}
\psaxes[labelFontSize=\scriptstyle,xAxis=true,yAxis=true,Dx=1.,Dy=1.,ticksize=-2pt 0,subticks=2]{->}(0,0)(-2.46,-1.76)(3.44,1.88)
\rput[tl](1.06,0.86){\blue{convexe}}
\psplot[linewidth=2.pt,linecolor=blue,plotpoints=200]{0}{3.4400000000000035}{x^(3.0)}
\psplot[linewidth=2.pt,linecolor=red,plotpoints=200]{-2.460000000000003}{0}{x^(3.0)}
\rput[tl](-2.34,-0.66){\red{concave}}
\rput[tl](0.58,-1.22){\green{point d'inflexion}}
\psline[linewidth=2.pt,linecolor=green]{->}(1.,-1.)(0.,0.)
\psdots[dotstyle=*,linecolor=green](0.,0.)
\end{pspicture*}
\end{center}
\item Pour tout $x\in\mathbb{R}~:$
\begin{align*}
h(x)&=\text{e}^{x}\\
h'(x)&=\text{e}^{x}\\
h''(x)&=\text{e}^{x}.
\end{align*}

\medskip

Conclusion~: $h''$ est strictement positive, donc $h$ est convexe sur $\mathbb{R}$ (cette fois, on se passe du tableau de signes).


\begin{center}
\psset{xunit=1.0cm,yunit=1.0cm,algebraic=true,dimen=middle,dotstyle=o,dotsize=5pt 0,linewidth=2.pt,arrowsize=3pt 2,arrowinset=0.25}
\begin{pspicture*}(-3.74,-0.86)(2.58,3.22)
\multips(0,0)(0,1.0){5}{\psline[linestyle=dashed,linecap=1,dash=1.5pt 1.5pt,linewidth=0.4pt,linecolor=lightgray]{c-c}(-3.74,0)(2.58,0)}
\multips(-3,0)(1.0,0){7}{\psline[linestyle=dashed,linecap=1,dash=1.5pt 1.5pt,linewidth=0.4pt,linecolor=lightgray]{c-c}(0,-0.86)(0,3.22)}
\psaxes[labelFontSize=\scriptstyle,xAxis=true,yAxis=true,Dx=1.,Dy=1.,ticksize=-2pt 0,subticks=2]{->}(0,0)(-3.74,-0.86)(2.58,3.22)
\rput[tl](-2.3,0.9){\blue{convexe}}
\psplot[linewidth=2.pt,linecolor=blue,plotpoints=200]{-3.7400000000000038}{2.5800000000000027}{EXP(x)}
\end{pspicture*}
\end{center}
\end{enumerate}

\end{exo}




\begin{exo}

La fonction $g$ est définie sur l'intervalle $\left[-1;3\right]$ par

\[ g(x) = -0,5 x^3+2x^2-2x.\]



\begin{enumerate}
\item \medskip

Pour tout $x\in\left[-1;3\right]~:$

\[g'(x)=-0,5\times 3x^2+2\times 2x-2\times 1=-1,5x^2+4x-2.\]

La dérivée est du second degré, donc on utilise la méthode de la classe de première~:

\begin{itemize}
\item[\textbullet] $a=-1,5,$ $b=4,$ $c=-2.$
\item[\textbullet] le discriminant est $\Delta=b^2-4ac=4^2-4\times (-1,5)\times (-2)=4.$
\item[\textbullet] $\Delta>0,$ donc il y a deux racines~:

\begin{align*}x_1&=\frac{-b-\sqrt{\Delta}}{2a}=\frac{-4-\sqrt{4}}{2\times (-1,5)}=\frac{-4-2}{-3}=\frac{-6}{-3}=2,\\
x_2&=\frac{-b+\sqrt{\Delta}}{2a}=\frac{-4+\sqrt{4}}{2\times (-1,5)}=\frac{-4+2}{-3}=\frac{-2}{-3}=\frac{2}{3}.
\end{align*}
\end{itemize}

\medskip

$a=-1,5$  $a$ est $\ominus$ donc le signe est de la forme \fbox{$-~\upphi~+~\upphi~-$}

\medskip

\setlength{\columnseprule}{1pt}

\begin{multicols}{2}
\begin{center}
\begin{tikzpicture}[scale=0.7]
\tkzTabInit{$x$/1,$g'(x)$/1,$g(x)$/2}{$-1$,$\frac{2}{3}$,$2$,$3$}
\tkzTabLine{,-,z,+,z,-}
\tkzTabVar{+/$3.5$,-/$-\frac{16}{27}$,+/$0$,-/$-1.5$}
\end{tikzpicture}
\end{center}

\columnbreak

\begin{itemize}
\item[\textbullet] $g(-1)=-0,5\times (-1)^3+2\times (-1)^2-2\times (-1)=3,5$
\item[\textbullet] $g\left(\frac{2}{3}\right)=-0,5\times \left(\frac{2}{3}\right)^3+2\times \left(\frac{2}{3}\right)^2-2\times \left(\frac{2}{3}\right)=-\frac{16}{27}$
\item[\textbullet] $g(2)=-0,5\times 2^3+2\times 2^2-2\times 2=0$
\item[\textbullet] $g(3)=-0,5\times 3^3+2\times 3^2-2\times 3=-1,5$

\end{itemize}


\end{multicols}

\item Pour tout $x\in\left[-1;3\right]~:$
\[g''(x)=-1,5\times 2x+4\times 1-0=-3x+4.\]


On étudie le signe de $g''~:$
\[-3x+4=0\iff -3x=-4\iff x=\frac{-4}{-3}=\frac{4}{3}.\]

\begin{center}
\begin{tikzpicture}[scale=1.2]
\tkzTabInit{$x$/1,$-3x+4$/1,Convexité/3}{$-1$,$\frac{4}{3}$,$3$}
\tkzTabLine{,+,z,-,}
\tkzTabLine{,g\text{ convexe},\scriptsize{\Longstack{P\\t\\ \\i\\n\\f\\l\\e\\x\\i\\o\\n}},g\text{ concave},}
\end{tikzpicture}
\end{center}

$g\left(\frac{4}{3}\right)=\left[\cdots\right]=-\frac{8}{27},$ donc le point de coordonnées $\left(\frac{4}{3};-\frac{8}{27}\right)$ est un point d'inflexion (noté $I$ sur la figure ci-dessous).
\item ~{}

\begin{center}
\psset{xunit=1.0cm,yunit=1.0cm,algebraic=true,dimen=middle,dotstyle=o,dotsize=5pt 0,linewidth=2.pt,arrowsize=3pt 2,arrowinset=0.25}
\begin{pspicture*}(-1.44,-1.64)(4.44,3.24)
\multips(0,-1)(0,1.0){5}{\psline[linestyle=dashed,linecap=1,dash=1.5pt 1.5pt,linewidth=0.4pt,linecolor=gray]{c-c}(-1.44,0)(4.44,0)}
\multips(-1,0)(1.0,0){6}{\psline[linestyle=dashed,linecap=1,dash=1.5pt 1.5pt,linewidth=0.4pt,linecolor=gray]{c-c}(0,-1.64)(0,3.24)}
\psaxes[labelFontSize=\scriptstyle,xAxis=true,yAxis=true,Dx=1.,Dy=1.,ticksize=-2pt 0,subticks=2]{->}(0,0)(-1.44,-1.64)(4.44,3.24)
\psplot[linewidth=2.pt,plotpoints=200,linecolor=blue]{-1.0}{1.33333}{-0.5*x^(3.0)+2.0*x^(2.0)-2.0*x+0}
\psplot[linewidth=2.pt,plotpoints=200,linecolor=red]{1.33333}{3}{-0.5*x^(3.0)+2.0*x^(2.0)-2.0*x+0.0}
\rput[tl](0.14,0.88){\blue{convexe}}
\rput[tl](3.16,-0.72){\red{concave}}
\psdots[dotstyle=*,linecolor=green](1.3333333333333333,-0.2777777777777777)
\rput[bl](1.42,-0.72){\green{$I$}}
\end{pspicture*}
\end{center}

\end{enumerate}

\end{exo}




\begin{exo}

La fonction $h$ est définie sur l'intervalle $\left[-1;4\right]$ par

\[ h(x) = (2x+3)\text{e}^{-x}.\]

On calcule les dérivées première et seconde~:

\medskip

\begin{enumerate}
\item \textbf{Dérivée première.} On utilise la formule pour la dérivée d'un produit avec

\begin{align*}
&u(x)=2x+3&&,&& v(x)=\text{e}^{-x}, \\
& u'(x)=2&&, &&v'(x)=-\text{e}^{-x}.\\
\end{align*}

On obtient, pour tout $x\in\left[-1;4\right]~:$


\begin{align*}h'(x)&=u'(x)\times v(x)+u(x)\times v'(x)\\
&=2\times\text{e}^{-x}+\left(2x+3\right)\times \left(-\text{e}^{-x}\right)\\
&=2\times\text{e}^{-x}+2x\times\left(-\text{e}^{-x}\right)+3\times\left(-\text{e}^{-x}\right)\\
&=2\times\text{e}^{-x}-2x\times\text{e}^{-x}-3\times\text{e}^{-x}\\
&=\left(2-2x-3\right)\text{e}^{-x}\\
&=\left(-2x-1\right)\text{e}^{-x}.
\end{align*}
\item \textbf{Dérivée seconde.}  On utilise la formule pour la dérivée d'un produit avec

\begin{align*}
&u(x)=-2x-1&&,&& v(x)=\text{e}^{-x}, \\
& u'(x)=-2&&, &&v'(x)=-\text{e}^{-x}.\\
\end{align*}

On obtient, pour tout $x\in\left[-1;4\right]~:$


\begin{align*}h''(x)&=u'(x)\times v(x)+u(x)\times v'(x)\\
&=-2\times\text{e}^{-x}+\left(-2x-1\right)\times \left(-\text{e}^{-x}\right)\\
&=-2\times\text{e}^{-x}+(-2x)\times\left(-\text{e}^{-x}\right)+(-1)\times\left(-\text{e}^{-x}\right)\\
&=-2\times\text{e}^{-x}+2x\times\text{e}^{-x}+1\times\text{e}^{-x}\\
&=\left(-2+2x+1\right)\text{e}^{-x}\\
&=\left(2x-1\right)\text{e}^{-x}.
\end{align*}
\end{enumerate}

\medskip

On étudie le signe de la dérivée seconde~:
\[h''(x)=\left(2x-1\right)\text{e}^{-x}.\]

\begin{itemize}
\item[\textbullet] $2x-1=0\iff 2x=1\iff x=\frac{1}{2}.$
\item[\textbullet] $\text{e}^{-x}$ est $\oplus$ pour tout $x\in\left[-1;4\right].$
\end{itemize}

\medskip

On a donc le tableau~:

\begin{center}
\begin{tikzpicture}[scale=1.2]
\tkzTabInit{$x$/1,$2x-1$/1,$\text{e}^{-x}$/1,$h''(x)$/1,Convexité/3}{$-1$,$\frac{1}{2}$,$4$}
\tkzTabLine{,-,z,+,}
\tkzTabLine{,+,,+,}
\tkzTabLine{,-,z,+,}
\tkzTabLine{,h\text{ concave},\scriptsize{\Longstack{P\\t\\ \\i\\n\\f\\l\\e\\x\\i\\o\\n}},h\text{ convexe},}
\end{tikzpicture}
\end{center}



\begin{center}
\psset{xunit=1.0cm,yunit=0.8cm,algebraic=true,dimen=middle,dotstyle=o,dotsize=5pt 0,linewidth=1.6pt,arrowsize=3pt 2,arrowinset=0.25}
\begin{pspicture*}(-2.26,-0.5)(5.3,3.8)
\multips(0,0)(0,1.0){6}{\psline[linestyle=dashed,linecap=1,dash=1.5pt 1.5pt,linewidth=0.4pt,linecolor=gray]{c-c}(-2.26,0)(6.18,0)}
\multips(-2,0)(1.0,0){9}{\psline[linestyle=dashed,linecap=1,dash=1.5pt 1.5pt,linewidth=0.4pt,linecolor=gray]{c-c}(0,0)(0,3.8)}
\psaxes[labelFontSize=\scriptstyle,xAxis=true,yAxis=true,Dx=1.,Dy=1.,ticksize=-2pt 0,subticks=2]{->}(0,0)(-2.26,-1.98)(5.3,3.8)
\psplot[linewidth=2.pt,plotpoints=200,linecolor=red]{-1}{0.5}{(2.0*x+3.0)*2.718281828459045^(-x)}
\psplot[linewidth=2.pt,plotpoints=200,linecolor=blue]{0.5}{4}{(2.0*x+3.0)*2.718281828459045^(-x)}
\psline[linewidth=2.pt,linecolor=green]{->}(1.1,3.3)(0.5,2.4261226388505337)
\rput[tl](0.3,3.8){\green{Point d'inflexion}}
\rput[tl](-1.68,1.46){\red{concave}}
\rput[tl](2.58,1.34){\blue{convexe}}
\psdots[dotsize=4pt 0,dotstyle=*,linecolor=green](0.5,2.4261226388505337)
\end{pspicture*}
\end{center}


\end{exo}

\begin{exo}

On note $\mathcal{C}$ la courbe de la fonction exponentielle et $T$ sa tangente au point $A(0;1).$

\begin{enumerate}

\item On pose $f(x)=\text{e}^x$ pour tout $x\in\mathbb{R}.$ On sait que $f'(x)=\text{e}^x$ pour tout $x\in\mathbb{R},$ donc\[f(0)=f'(0)=\text{e}^0=1.\] L'équation de la tangente $T$ est donc
\begin{align*}
y&=f'(0)(x-0)+f(0)\\
y&=1(x-0)+1\\
y&=x+1
\end{align*}
\item On a déjà vu dans un exercice précédent que la fonction exponentielle était convexe sur $\mathbb{R}.$ D'après le théorème 8 du cours, la courbe $\mathcal{C}$ est au-dessus de toutes ses tangentes~; elle est donc en particulier au-dessus de $T.$ Il s'ensuit que
\[\text{e}^x\geq x+1\] pour tout $x\in\mathbb{R}.$


\begin{center}
\psset{xunit=1.0cm,yunit=1.0cm,algebraic=true,dimen=middle,dotstyle=o,dotsize=5pt 0,linewidth=2.pt,arrowsize=3pt 2,arrowinset=0.25}
\begin{pspicture*}(-4.44,-1.04)(4.8,4.42)
\multips(0,-1)(0,1.0){6}{\psline[linestyle=dashed,linecap=1,dash=1.5pt 1.5pt,linewidth=0.4pt,linecolor=lightgray]{c-c}(-4.44,0)(4.8,0)}
\multips(-4,0)(1.0,0){10}{\psline[linestyle=dashed,linecap=1,dash=1.5pt 1.5pt,linewidth=0.4pt,linecolor=lightgray]{c-c}(0,-1.04)(0,4.42)}
\psaxes[labelFontSize=\scriptstyle,xAxis=true,yAxis=true,Dx=1.,Dy=1.,ticksize=-2pt 0,subticks=2]{->}(0,0)(-4.44,-1.04)(4.8,4.42)
\psplot[linewidth=2.pt,linecolor=red,plotpoints=200]{-4.439999999999999}{4.799999999999998}{EXP(x)}
\psplot[linewidth=2.pt,linecolor=blue]{-4.44}{4.8}{(--1.--1.*x)/1.}
\rput[tl](0.74,3.56){\red{$\mathcal{C}$}}
\rput[tl](1.78,2.6){\blue{$T$}}
\psdots[dotsize=5pt 0,dotstyle=*,linecolor=blue](0.,1.)
\rput[bl](0.12,0.68){\blue{$A$}}
\end{pspicture*}
\end{center}

\medskip

\textbf{Remarque~:} On a déjà démontré ce résultat par une étude de fonction, dans l'exercice 10.
\end{enumerate}

\end{exo}


\begin{exo}



\begin{enumerate}
\item Si $u(x)=x^2$ et $ v(x)=4x+1,$ alors
\[v\circ u(x)=v(u(x))=v\left(x^2\right)=4x^2+1.\]
\item Si $u(x)=x+2$ et $v(x)=x^3-3x,$ alors
\[v\circ u(x)=v(u(x))=v\left(x+2\right)=(x+2)^3-3(x+2).\]
\item  Si $u(x)=x-4$ et $ v(x)=\sqrt{x},$ alors
\[v\circ u(x)=v(u(x))=v\left(x-4\right)=\sqrt{x-4}.\]
\item Si $u(x)=2x+3$ et $v(x)=\text{e}^x,$ alors
\[v\circ u(x)=v(u(x))=v\left(2x+3\right)=\text{e}^{2x+3}.\]
\end{enumerate}

\end{exo}

\begin{exo}

\begin{enumerate}
\item Sachant que $v\circ u(x)=\sqrt{x^2+1},$ on peut prendre
\[u(x)=x^2+1\qquad,\qquad v(x)=\sqrt{x}.\]
\item Sachant que $v\circ u(x)=(x-3)^2+5(x-3)+1,$ on peut prendre
\[u(x)=x-3\qquad,\qquad v(x)=x^2+5x+1.\]
\item  Sachant que $v\circ u(x)=\text{e}^{3x-1},$ on peut prendre
\[u(x)=3x-1\qquad,\qquad v(x)=\text{e}^x.\]
\end{enumerate}

\medskip

\textbf{Remarque~:} Il y a une infinité de choix possibles. Par exemple, pour le deuxième, on pourrait prendre
\[u(x)=(x-3)^2+5(x-3)\qquad,\qquad v(x)=x+1~;\]

ou encore 
\[u(x)=(x-3)^2+5(x-3)+1\qquad,\qquad v(x)=x~;\]
etc.


\end{exo}

\begin{exo}

On considère dans un repère orthonormé la parabole $P:y=x^2$ et le point $A(3;0).$


\begin{enumerate}
\item Soit $m$ un réel et soit $M$ le point de $P$ d'abscisse $m.$ L'ordonnée de $M$ est $m^2,$ donc 
\begin{align*}AM&=\sqrt{\left(x_M-x_A\right)^2+\left(y_M-y_A\right)^2}
\\&=\sqrt{\left(m-3\right)^2+\left(m^2-0\right)^2}
\\&=\sqrt{m^2-2\times m\times 3+3^2+m^4}
\\&=\sqrt{m^4+m^2-6m+9}.
\end{align*}

\medskip

On remarque que $AM=f(m),$ où $f$ est la fonction définie dans la question suivante. De ce fait, trouver le point $M$ pour lequel la longueur $AM$ est minimale revient à trouver la valeur de $x$ pour laquelle $f$ atteint son minimum. Nous y reviendrons dans la question 3.
\item On pose $f(x)=\sqrt{x^4+x^2-6x+9}$ pour tout $x\in\mathbb{R}.$


La fonction $f$ est de la forme $f(x)=\sqrt{u(x)},$ avec  \[u(x)=x^4+x^2-6x+9,\qquad u'(x)=4x^3+2x-6.\]
 On a donc, pour tout $x\in\mathbb{R}~:$ \[f'(x)=\dfrac{u'(x)}{2\sqrt{u(x)}}=\dfrac{4x^3+2x-6}{2\sqrt{x^4+x^2-6x+9}}=\dfrac{\cancel{2}\left(2x^3+x-3\right)}{\cancel{2}\sqrt{x^4+x^2-6x+9}}=\dfrac{2x^3+x-3}{\sqrt{x^4+x^2-6x+9}}.\]


\medskip

Pour démontrer la formule de l'énoncé, on développe~:
\[(x-1)\left(2x^2+2x+3\right)=x\times 2x^2+x\times 2x+x\times 3-1\times 2x^2-1\times 2x-1\times 3=2x^3+2x^2+3x-2x^2-2x-3=2x^3+x-3.\]
On retombe sur le numérateur obtenu précédemment~; on a donc bien
\[f'(x)=\dfrac{(x-1)\left(2x^2+2x+3\right)}{\sqrt{x^4+x^2-6x+9}}.\]

Pour construire le tableau de variations de la fonction $f,$ il faut étudier le signe de $2x^2+2x+3.$ Son discriminant est $\Delta=2^2-4\times 2\times 3=-20,$ donc il n'y a pas de racine et $2x^2+2x+3$ est strictement positif pour tout réel $x.$ On peut donc compléter le tableau~:

\medskip

\begin{center}
\begin{tikzpicture}[scale=1.3]
\tkzTabInit{$x$/1,$x-1$/1,$2x^2+2x+3$/1,\footnotesize$\sqrt{x^4+x^2-6x+9}$/1,\normalsize$f'(x)$/1,$f(x)$/2}{$-\infty$,$1$,$+\infty$}
\tkzTabLine{,-,z,+,}
\tkzTabLine{,+,,+,}
\tkzTabLine{,+,,+,}
\tkzTabLine{,-,z,+,}
\tkzTabVar{+/,-/,+/}
\end{tikzpicture}
\end{center}

\item

\begin{itemize}
\item[\textbullet] La fonction $f$ atteint son minimum pour $x=1,$ donc la longueur $AM$ est minimale lorsque $m=1.$ Autrement dit, le point de $P$ le plus proche de $A$ est le point $M(1;1).$

\item[\textbullet] La tangente $(T)$ à la parabole $P$ au point $M$ a pour équation
\[y=g'(1)(x-1)+g(1),\]
avec $g(x)=x^2$ -- donc $g'(x)=2x,$ et $g'(1)=2\times 1=2.$ On a ainsi

\begin{align*}
(T):y&=g'(1)(x-1)+g(1)\\
y&=2(x-1)+1\\
y&=2x-1.
\end{align*}
\item[\textbullet] Pour prouver que $(AM)$ est perpendiculaire à $(T),$ on utilise le produit scalaire~:


$(T)$ passe par $M(1;1)$ et par $N(2;3)$ (puisque $2\times 2-1=3$), donc elle est dirigée par le vecteur $\overrightarrow{MN}\begin{pmatrix}1\\2\end{pmatrix}.$ Par ailleurs $\overrightarrow{AM}\begin{pmatrix}-2\\1\end{pmatrix},$ donc
\[\overrightarrow{MN}\cdot \overrightarrow{AM}=1\times (-2)+2\times 1=0.\] Les droites $(T)$ et $(AM)$ sont donc bien perpendiculaires.
\end{itemize}


\begin{center}
\newrgbcolor{ududff}{0.30196078431372547 0.30196078431372547 1.}
\psset{xunit=1.0cm,yunit=1.0cm,algebraic=true,dimen=middle,dotstyle=o,dotsize=5pt 0,linewidth=2.pt,arrowsize=3pt 2,arrowinset=0.25}
\begin{pspicture*}(-2.34,-0.92)(4.92,5.4)
\multips(0,0)(0,1.0){7}{\psline[linestyle=dashed,linecap=1,dash=1.5pt 1.5pt,linewidth=0.4pt,linecolor=lightgray]{c-c}(-2.34,0)(4.92,0)}
\multips(-2,0)(1.0,0){8}{\psline[linestyle=dashed,linecap=1,dash=1.5pt 1.5pt,linewidth=0.4pt,linecolor=lightgray]{c-c}(0,-0.92)(0,5.4)}
\psaxes[labelFontSize=\scriptstyle,xAxis=true,yAxis=true,Dx=1.,Dy=1.,ticksize=-2pt 0,subticks=2]{->}(0,0)(-2.34,-0.92)(4.92,5.4)
\pspolygon[linewidth=2.pt,linecolor=red,fillcolor=red!30!white,fillstyle=solid,opacity=0.1](1.3794733192202056,0.8102633403898972)(1.5692099788303084,1.1897366596101029)(1.1897366596101029,1.3794733192202056)(1.,1.)
\rput{0.}(0.,0.){\psplot[linewidth=2.pt,linecolor=blue]{-4.}{4.}{x^2/2/0.5}}
\psplot[linewidth=2.pt,linecolor=red]{-2.34}{4.92}{(-0.5--1.*x)/0.5}
\psline[linewidth=2.pt,linestyle=dashed,dash=2pt 2pt,linecolor=red](3.,0.)(1.,1.)
\rput[tl](2.28,3.36){\red{$(T)$}}
\rput[tl](-1.42,2.64){\blue{$P$}}
\psdots[dotstyle=*,linecolor=ududff](1.,1.)
\rput[bl](0.74,1.36){\ududff{$M$}}
\psdots[dotstyle=*,linecolor=red](3.,0.)
\rput[bl](3.08,0.2){\red{$A$}}
\end{pspicture*}
\end{center}
\end{enumerate}

\end{exo}

\section{Suites et récurrence}


\begin{exo}

On calcule trois ou quatre termes, suivant le cas -- suffisamment pour \og avoir compris le principe \fg.

\begin{enumerate}
\item Pour tout $n\in\mathbb{N}~:$ $u_n=\dfrac{n^2-1}{n+2}.$

\begin{align*}
u_0&=\dfrac{0^2-1}{0+2}=-\dfrac{1}{2}\\
u_1&=\dfrac{1^2-1}{1+2}=\dfrac{0}{3}=0\\
u_2&=\dfrac{2^2-1}{2+2}=\dfrac{3}{4}
\end{align*}

\item Pour tout $n\in\mathbb{N}^*~:$ $v_n=\dfrac{(-1)^n}{n}.$

\danger On \og démarre \fg~{} à $n=1,$ puisqu'on ne peut pas diviser par $0.$

\begin{align*}
v_1&=\dfrac{(-1)^1}{1}=\dfrac{-1}{1}=-1\\
v_2&=\dfrac{(-1)^2}{2}=\dfrac{1}{2}\\
v_3&=\dfrac{(-1)^3}{3}=\dfrac{-1}{3}=-\dfrac{1}{3}\\
v_4&=\dfrac{(-1)^4}{4}=\dfrac{1}{4}
\end{align*}

Les termes sont alternativement positifs et négatifs. On dit que la suite est alternée.
\item $u_0=3$ et pour tout $n\in\mathbb{N}~:$
\[u_{n+1}=2u_n-1.\]


\medskip

\setlength{\columnseprule}{1pt}

\begin{multicols}{4}

\begin{align*}
&\text{On prend}~n=0~:\\
u_{0+1}&=2u_0-1\\
u_{1}&=2\times 3-1\\
u_1&=5
\end{align*}

\begin{align*} 
&\text{On prend}~n=1~:\\
u_{1+1}&=2u_1-1\\
u_{2}&=2\times 5-1\\
u_2&=9
\end{align*}

\begin{align*}
&\text{On prend}~n=2~:\\
u_{2+1}&=2u_2-1\\
u_{3}&=2\times 9-1\\
u_3&=17
\end{align*}

\begin{align*}
&\text{On prend}~n=3~:\\
u_{3+1}&=2u_3-1\\
u_{4}&=2\times 17-1\\
u_4&=33
\end{align*}

\end{multicols}

\item $v_0=-1$ et $v_{n+1}=v_n+n$ \qquad pour tout $n\in\mathbb{N}.$

\medskip

\setlength{\columnseprule}{1pt}

\begin{multicols}{4}

\begin{align*}
&\text{On prend}~n=0~:\\
v_{0+1}&=v_0+0\\
v_{1}&=-1+0\\
v_1&=-1
\end{align*}

\begin{align*} 
&\text{On prend}~n=1~:\\
v_{1+1}&=v_1+1\\
v_{2}&=-1+1\\
v_2&=0
\end{align*}

\begin{align*}
&\text{On prend}~n=2~:\\
v_{2+1}&=v_2+2\\
v_{3}&=0+2\\
v_3&=2
\end{align*}

\begin{align*}
&\text{On prend}~n=3~:\\
v_{3+1}&=v_3+3\\
v_{4}&=2+3\\
v_4&=5
\end{align*}

\end{multicols}
\end{enumerate}

\end{exo}

\begin{exo}



\begin{enumerate}

\item Pour diminuer un nombre de 8~\%, il faut le multiplier par 0,92, car $100~\%-8~\%=92~\%=0,92$. On peut donc compléter le schéma~:

    \medskip


\begin{center}
    $\xymatrix@R=0.5pc@C=3pc{
    *+[F]+{10} \ar@/^0.5cm/[r]|{\red{\times 0,92}} & 
    *+[F]+{9,2} \ar@/^0.5cm/[r]|{\red{\times 0,92}} & *+[F]+{8,464} \\
    \txt{\blue{$v_0$}}&
    \txt{\blue{$v_1$}}&\txt{\blue{$v_2$}}\\
    \txt{\blue{Au départ}}&
    \txt{\blue{Après 1 jour}}&\txt{\blue{Après 2 jours}}
    }$
    \end{center}
   
   Conclusion~: 
    \[v_0=10~;~v_1=9,2~;~v_2=8,464.\]
    
    \medskip

La suite $(v_n)_{n\in\mathbb{N}}$ est géométrique de raison $q=0,92.$
\item La masse d'iode 131 après 10 jours est
\[v_{10}=v_0\times q^{10}=10\times 0,92^{10}\approx 4,3~\mu\text{g}.\]
\item On part de 10 $\mu$g d'iode 131, donc il s'agit de déterminer à partir de quand il en restera moins de 5 $\mu$g. Pour cela, on fait un tableau de valeurs avec la calculatrice, en rentrant la formule 
\[Y=10* 0.92^X\]
(on peut aussi utiliser le mode suite ou le mode tableur, suivant les modèles).

\medskip

Après quelques essais\footnote{On ne peut pas savoir en démarrant jusqu'à quelle valeur de $n$ il faut aller~; il faut donc faire des essais. Lorsque nous connaîtrons le logarithme népérien, nous pourrons donner une méthode plus efficace~; et nous pourrons même donner une formule~: la demi-vie est $-\frac{\ln 2}{\ln 0,92}.$}, on obtient~:

\begin{center}
\begin{tabular}{|c|c|c|}
\hline
	$n$&8&9\\\hline
$v_n$&$5,13$&$4,72$\\\hline
\end{tabular}
\end{center}

\medskip

Conclusion~: la demi-vie de l'iode 131 est de 8 jours et quelques.
\end{enumerate}

\end{exo}

\begin{exo}



\begin{enumerate}
\item $100~\%-15~\%=85~\%=0,85,$ donc pour diminuer un nombre de 15~\%, il faut le multiplier par 0,85. 

Ainsi, dans le schéma ci-dessous, l'intensité lumineuse est-elle multipliée par 0,85 à chaque nouvelle plaque~:

\begin{center}
\newrgbcolor{ududff}{0.30196078431372547 0.30196078431372547 1.}
\newrgbcolor{zzttqq}{0.6 0.2 0.}
\psset{xunit=0.75cm,yunit=0.75cm,algebraic=true,dimen=middle,dotstyle=o,dotsize=5pt 0,linewidth=2.pt,arrowsize=3pt 2,arrowinset=0.25}
\begin{pspicture*}(-1.2095636134454244,-0.8024760882800114)(15,7)
\pspolygon[linewidth=2.pt,linecolor=zzttqq,fillcolor=zzttqq!20!white,fillstyle=solid,opacity=0.1](0.,4.)(5.,4.)(7.,5.)(2.,5.)
\pspolygon[linewidth=2.pt,linecolor=zzttqq,fillcolor=zzttqq!20!white,fillstyle=solid,opacity=0.1](0.,2.)(5.,2.)(7.,3.)(2.,3.)
\pspolygon[linewidth=2.pt,linecolor=zzttqq,fillcolor=zzttqq!20!white,fillstyle=solid,opacity=0.1](0.,0.)(5.,0.)(7.,1.)(2.,1.)
\psline[linewidth=2.pt]{->}(9.006307576513136,3.604370307388384)(6.96981037851486,3.604370307388384)
\psline[linewidth=2.pt]{->}(8.989614976529543,5.607482305419475)(6.986502978498453,5.607482305419475)
\psline[linewidth=2.pt]{->}(8.989614976529543,1.6012583093572934)(6.986502978498453,1.584565709373701)
\psline[linewidth=2.pt]{->}(9.006307576513136,-0.3851610886902048)(6.986502978498453,-0.4018536886737972)
\rput[tl](-0.95,5.15){\textcolor{zzttqq}{1\up{re} plaque}}
\rput[tl](-0.95,3.15){\textcolor{zzttqq}{2\up{e} plaque}}
\rput[tl](-0.95,1.15){\textcolor{zzttqq}{3\up{e} plaque}}
\rput[tl](9.256696576267021,5.8){$v_0=12~\text{lm}$}
\rput[tl](9.25,3.8){$v_1=12\times 0,85=10,2~\text{lm}$}
\rput[tl](9.25,1.8){$v_2=10,2\times 0,85=8,67~\text{lm}$}
\rput[tl](9.25,-0.2){$v_3=8,67\times 0,85=7,3695~\text{lm}$}
\psline[linewidth=2.pt,linecolor=yellow](4.5,7.)(4.5,5.5)
\psline[linewidth=2.pt,linecolor=yellow](4.,7.)(3.5,5.5)
\psline[linewidth=2.pt,linecolor=yellow](5.,7.)(5.5,5.5)
\rput[tl](1.5,6.328984750608794){\yellow{lumière}}
\end{pspicture*}
\end{center}

\medskip

\textbf{Remarque~:} Le lumen est une unité de mesure du flux lumineux, utilisée notamment pour indiquer la capacité d'éclairement des ampoules électriques.
\item La suite $(v_n)_{n\in\mathbb{N}}$ est géométrique de raison $q=0,85,$ donc pour tout $n\in\mathbb{N}~:$
\[v_n=v_0\times q^n=12\times 0,85^n.\]
\item Comme on part de 12~lm, il s'agit de savoir le nombre de plaques nécessaires pour que l'intensité lumineuse soit inférieure à 0,12~lm (puisque $12\div 100=0,12$).

\medskip

Comme dans l'exercice précédent, on rentre la formule \[Y=12*0.85^X\] dans le mode fonction  de la calculatrice, puis on fait des essais. On obtient~:

\begin{center}
\begin{tabular}{|c|c|c|}
\hline
	$n$&28&29\\\hline
$v_n$&$0,13$&$0,11$\\\hline
\end{tabular}
\end{center}

\medskip

Conclusion~: il faut superposer au moins 29 plaques pour que l'intensité lumineuse soit divisée par 100.

\end{enumerate}

\end{exo}

\begin{exo}

Une suite $v$ est définie par $v_0=4$ et la relation de récurrence \[v_{n+1}=2v_n+2\] pour tout entier naturel $n.$

\begin{enumerate}
\item \begin{align*}
v_0&=4\\
v_1&=2\times 4+2=10\\
v_2&=2\times 10+2=22.\end{align*}

\newpage
\item Avec un schéma~:

\setlength{\columnseprule}{1pt}

\begin{multicols}{2}
~{}\begin{center}
    $\xymatrix@R=0.5pc@C=3pc{
    *+[F]+{4} \ar@/^0.5cm/[r]|{\red{+6}} \ar@/_0.5cm/[r]|{\green{\times 2,5}} & 
    *+[F]+{10} \ar@/^0.5cm/[r]|{\red{+12}} \ar@/_0.5cm/[r]|{\green{\times 2,2}} & *+[F]+{22} \\
    \txt{\blue{$v_0$}}&
    \txt{\blue{$v_1$}}&\txt{\blue{$v_2$}}    
    }$
    \end{center}
    
    
    \medskip
    
    Les résultats en rouge (\textcolor{red}{6} et \textcolor{red}{12}) sont différents, donc $u$ \textbf{n'est pas arithmétique.}
    
    \medskip
    
    Les résultats en vert (\textcolor{green}{2,5} et \textcolor{green}{2,2}) sont différents, donc $u$ \textbf{n'est pas géométrique.}
    
  \columnbreak
  
  Calculs utiles~:
  
  \medskip
  
  \begin{align*}
  10-4&=\textcolor{red}{6},\\
  22-10&= \textcolor{red}{12}.
  \end{align*}
  
  \medskip
  
  \begin{align*}
  10\div 4&=\textcolor{green}{2,5},\\
  22\div 10&= \textcolor{green}{2,2}.
  \end{align*}
  
  \end{multicols}


\end{enumerate}
\end{exo}

\begin{exo}

La suite $(u_n)_{n\in\mathbb{N}}$ est définie par $u_0=2$ et la relation de récurrence \[u_{n+1}=3 u_n-1\] pour tout $n\in\mathbb{N}.$


\begin{enumerate}
\item \begin{align*}
u_0&=2\\
u_1&=3\times 2-1=5\\
u_2&=3\times 5-1=14
\end{align*}

\item On pose $v_n=u_n-0,5$ pour tout entier naturel $n.$

\begin{align*}
v_0&=u_0-0,5=2-0,5=1,5\\
v_1&=u_1-0,5=5-0,5=4,5\\
v_2&=u_2-0,5=14-0,5=13,5
\end{align*}
\item Pour tout $n\in\mathbb{N}~:$



\begin{alignat*}{3}
&v_{n+1}&& =u_{n+1}-0,5 && \text{  (déf. de } (v_n)_{n\in\mathbb{N}})\\
& && =(3u_n-1)-0,5 && \text{  (rel. réc. pour } (u_n)_{n\in\mathbb{N}})\\
& && =3u_n-1,5 && \text{  (calcul)}\\
& && =3\left(u_n-\frac{1,5}{3}\right) && \text{  (factorisation)}\\
& && =3(u_n-0,5) && \text{  (calcul)}\\
& && =3v_n&& \text{  (déf. de } (v_n)_{n\in\mathbb{N}})\\
\end{alignat*}


Conclusion~: pour tout $n\in\mathbb{N},$ $v_{n+1}=3v_n,$
donc $(v_n)_{n\in\mathbb{N}}$ est géométrique de raison $q=3.$

\medskip

\textbf{Remarque~:} L'étude d'une suite arithmético-géométrique ($u_{n+1}=au_n+b,$ avec $a\not=1$) se ramène toujours à celle d'une suite géométrique $(v_n)_{n\in\mathbb{N}}.$ Pour prouver que $(v_n)_{n\in\mathbb{N}}$ est géométrique, la méthode est toujours celle que nous venons de donner. \`A la quatrième ligne de calcul, c'est $a$ qu'il faut mettre  en facteur (ici, on a mis 3 en facteur).

\item La suite $(v_n)_{n\in\mathbb{N}}$ est géométrique de raison $q=3,$ et  $v_0=u_0-0,5=2-0,5=1,5,$ donc pour tout $n\in\mathbb{N}~:$
\[v_n=v_0\times q^n=1,5\times 3^n.\]
\item Enfin $v_n=u_n-1,5$ donc
\[u_n=v_n+1,5=1,5\times 3^n+0,5.\]
\end{enumerate}

\end{exo}



\begin{exo}




\begin{enumerate}
\item On complète le schéma ci-dessous pour calculer les termes $u_1$ et $u_2.$ Les sommes écrites dans chaque case sont les sommes restant à rembourser aux dates indiquées.

\medskip

\begin{center}
    $\xymatrix@R=0.5pc@C=3pc{
    *+[F]+{\np{10000}} \ar@/^0.5cm/[r]|{\red{\times 1,02}} & 
    *+[F]+{\np{10200}} \ar@/^0.5cm/[r]|{\red{-~300}} & *+[F]+{\np{9900}}\ar@/^0.5cm/[r]|{\red{\times 1,02}} & *+[F]+{\np{10098}}\ar@/^0.5cm/[r]|{\red{-~300}} & *+[F]+{\np{9798}} \\
    \txt{\blue{$u_0$}}&\txt{\blue{$~$}}&
    \txt{\blue{$u_1$}}&\txt{\blue{$~$}}&\txt{\blue{$u_2$}}\\
    \txt{\blue{01/01/20}}&
    \txt{\blue{31/01/20}}&\txt{\blue{01/02/20}}&
    \txt{\blue{28/02/20}}&\txt{\blue{01/03/20}}
    }$
    \end{center}

Pour passer d'un terme de la suite au terme suivant, on multiplie par $1,02$ (ajout des intérêts) puis on retranche 300 (remboursement mensuel). On peut donc continuer plus rapidement~:

\begin{align*}
u_3&=\np{9798}\times 1,02-300=\np{9693,96}&&\text{(somme à rembourser le 01/04/20)},\\
u_4&=\np{9693,96}\times 1,02-300=\np{9587,84}&&\text{(somme à rembourser le 01/05/20)}
.\end{align*}

\item Pour tout $n\in\mathbb{N}~:$ \[u_{n+1}=1,02u_n-300.\]
\item Pour tout $n\in\mathbb{N}~:$



\begin{alignat*}{3}
&v_{n+1}&& =u_{n+1}-\np{15000} && \text{  (déf. de } (v_n)_{n\in\mathbb{N}})\\
& && =(1,02u_n-300)-\np{15000} && \text{  (rel. réc. pour } (u_n)_{n\in\mathbb{N}})\\
& && =1,02 u_n-\np{15300} && \text{  (calcul)}\\
& && =1,02\left(u_n-\frac{\np{15300}}{1,02}\right) && \text{  (factorisation)}\\
& && =1,02(u_n-\np{15000}) && \text{  (calcul)}\\
& && =1,02v_n&& \text{  (déf. de } (v_n)_{n\in\mathbb{N}})\\
\end{alignat*}


Conclusion~: pour tout $n\in\mathbb{N},$ $v_{n+1}=1,02v_n,$
donc $(v_n)_{n\in\mathbb{N}}$ est géométrique de raison $q=1,02.$


\item La suite $(v_n)_{n\in\mathbb{N}}$ est géométrique de raison $q=1,02,$ et  $v_0=u_0-\np{15000}=\np{10000}-\np{15000}=-\np{5000},$ donc pour tout $n\in\mathbb{N}~:$
\[v_n=v_0\times q^n=-\np{5000}\times 1,02^n.\]

Enfin $v_n=u_n-\np{15000}$ donc
\[u_n=v_n+\np{15000}=-\np{5000}\times 1,02^n+\np{15000}.\]
\item Déterminer la durée du crédit revient à savoir quand la somme restant à rembourser est nulle. En réalité, au bout d'un moment, elle est négative, comme on le voit avec un tableau de valeurs~:

\medskip

\begin{center}
\begin{tabular}{|c|c|c|}\hline
	$n$&$55$&$56$	\\ \hline   
$u_n$&$141,34$&$-155,83$ \\ \hline    
\end{tabular}
\end{center}

\medskip

\`A la fin du 55\up{e} fois, il reste 141,35~\euro~{} à rembourser~; et si on rembourse 300~\euro~{} au début du 56\up{e} mois, la banque nous devra 155,83~\euro.


Conclusion~:

\begin{itemize}
\item[\textbullet] le crédit dure 56 mois~;
\item[\textbullet] on rembourse 56 fois 300~\euro, mais à la fin on a dépassé de 155,83~\euro~{} ce que l'on devait à la banque~;
\item[\textbullet] la somme totale remboursée est donc
\[56\times 300-155,83=\np{16664,17}~\text{\euro}~;\]
\item[\textbullet] le \og coût du crédit \fg~{} est la différence entre ce que l'on a remboursé et ce que la banque nous a prêté~:
\[\text{Coût du crédit}=\text{Somme remboursée}-\text{Somme empruntée}=
\np{16664,17}-\np{10000}=\np{6664,17}~\text{\euro}.\]
\end{itemize}

\end{enumerate}


\end{exo}




\begin{exo}


La suite $(u_n)_{n\in\mathbb{N}}$ est définie par $u_0=0$ et pour tout $n\in\mathbb{N}~:$ \[u_{n+1}=2 u_n+1.\]

Pour tout $n\in\mathbb{N},$ on note $\mathcal{P}_n$ la propriété \[u_n=2^n-1.\]





\begin{itemize}
\item[{\textbullet}] \textbf{Initialisation.} On prouve que $\mathcal{P}_0$ est vraie.

\[
\left.
    \begin{array}{ll}
        u_0&=0 \\
        2^0-1&= 1-1=0
    \end{array}
\right \}\implies \mathcal{P}_0~\text{est vraie}.
\]



\item[{\textbullet}] \textbf{Hérédité.} Soit $k\in\mathbb{N}$ tel que $\mathcal{P}_k$ soit vraie. On a donc
\[u_k=2^k-1.\]

%\medskip

\newtcolorbox{mybox}[1]{colback=green!10!white,colframe=green!80!white,fonttitle=\bfseries,title=#1}
\begin{mybox}{Objectif}{Prouver que $\mathcal{P}_{k+1}$ est vraie, c'est-à-dire que \[u_{k+1}=2^{k+1}-1.\]
}\end{mybox}



%\medskip

On part de 
\[u_k=2^k-1.\]


On a alors~:

\begin{alignat*}{3}
&u_{k+1}&& =2{u_k}+1 && \text{  (rel. réc. pour } (u_n)_{n\in\mathbb{N}})\\
& && =2\left({2^k-1}\right)+1 && \text{ {(H.R.)}}\\
& && =2\times 2^k-2+1 && \text{  (on développe)}\\
& && =2^{k+1}-1 && \text{  (calcul)}.\\
\end{alignat*}



La propriété $\mathcal{P}_{k+1}$ est donc vraie.
\item[{\textbullet}] \textbf{Conclusion.} $\mathcal{P}_0$ est vraie et $\mathcal{P}_n$ est héréditaire, donc elle est vraie pour tout $n\in\mathbb{N}.$
\end{itemize}


\end{exo}

\begin{exo}


La suite $(u_n)_{n\in\mathbb{N}}$ est définie par $u_0=1$ et pour tout $n\in\mathbb{N}~:$ \[u_{n+1}=u_n+2n+3.\]

\begin{enumerate}
\item \begin{align*}
(n=0)\qquad u_{1}&=u_0+2\times 0+3=1+0+3=4\\
(n=1)\qquad u_{2}&=u_1+2\times 1+3=4+2+3=9\\
(n=2)\qquad u_{3}&=u_2+2\times 2+3=9+4+3=16
\end{align*}

\medskip

\textbf{Remarque~:} Pour passer de $u_n$ à $u_{n+1},$ on ajoute $2n+3,$ donc à partir de $u_0=1$ (rond rose ci-dessous)~:

\begin{itemize}
\item[\textbullet] on obtient $u_1$ en ajoutant $2\times 0+3=3$ ronds bleus~;
\item[\textbullet] on obtient $u_2$ en ajoutant $2\times 1+3=5$ ronds oranges~;
\item[\textbullet] on obtient $u_3$ en ajoutant $2\times 2+3=7$ ronds verts~;
\item[\textbullet] etc.
\end{itemize}

\medskip

\begin{center}
\newrgbcolor{ffxfqq}{1. 0.4980392156862745 0.}
\psset{xunit=0.75cm,yunit=0.75cm,algebraic=true,dimen=middle,dotstyle=o,dotsize=5pt 0,linewidth=2.pt,arrowsize=3pt 2,arrowinset=0.25}
\begin{pspicture*}(-4.64,0.)(8.64,5.48)
\pscircle[linewidth=2.pt,linecolor=magenta,fillcolor=magenta,fillstyle=solid,opacity=1](-1.,5.){0.3}
\pscircle[linewidth=2.pt,linecolor=blue,fillcolor=blue,fillstyle=solid,opacity=1](-2.,4.){0.3}
\pscircle[linewidth=2.pt,linecolor=blue,fillcolor=blue,fillstyle=solid,opacity=1](-1.,4.){0.3}
\pscircle[linewidth=2.pt,linecolor=blue,fillcolor=blue,fillstyle=solid,opacity=1](0.,4.){0.3}
\pscircle[linewidth=2.pt,linecolor=ffxfqq,fillcolor=ffxfqq,fillstyle=solid,opacity=1](-3.,3.){0.3}
\pscircle[linewidth=2.pt,linecolor=ffxfqq,fillcolor=ffxfqq,fillstyle=solid,opacity=1](-2.,3.){0.3}
\pscircle[linewidth=2.pt,linecolor=ffxfqq,fillcolor=ffxfqq,fillstyle=solid,opacity=1](-1.,3.){0.3}
\pscircle[linewidth=2.pt,linecolor=ffxfqq,fillcolor=ffxfqq,fillstyle=solid,opacity=1](0.,3.){0.3}
\pscircle[linewidth=2.pt,linecolor=ffxfqq,fillcolor=ffxfqq,fillstyle=solid,opacity=1](1.,3.){0.3}
\pscircle[linewidth=2.pt,linecolor=green,fillcolor=green,fillstyle=solid,opacity=1](-4.,2.){0.3}
\pscircle[linewidth=2.pt,linecolor=green,fillcolor=green,fillstyle=solid,opacity=1](-3.,2.){0.3}
\pscircle[linewidth=2.pt,linecolor=green,fillcolor=green,fillstyle=solid,opacity=1](-2.,2.){0.3}
\pscircle[linewidth=2.pt,linecolor=green,fillcolor=green,fillstyle=solid,opacity=1](-1.,2.){0.3}
\pscircle[linewidth=2.pt,linecolor=green,fillcolor=green,fillstyle=solid,opacity=1](0.,2.){0.3}
\pscircle[linewidth=2.pt,linecolor=green,fillcolor=green,fillstyle=solid,opacity=1](1.,2.){0.3}
\pscircle[linewidth=2.pt,linecolor=green,fillcolor=green,fillstyle=solid,opacity=1](2.,2.){0.3}
\pscircle[linewidth=2.pt,linecolor=magenta,fillcolor=magenta,fillstyle=solid,opacity=1](5.,2.){0.3}
\pscircle[linewidth=2.pt,linecolor=blue,fillcolor=blue,fillstyle=solid,opacity=1](5.,3.){0.3}
\pscircle[linewidth=2.pt,linecolor=blue,fillcolor=blue,fillstyle=solid,opacity=1](6.,3.){0.3}
\pscircle[linewidth=2.pt,linecolor=blue,fillcolor=blue,fillstyle=solid,opacity=1](6.,2.){0.3}
\pscircle[linewidth=2.pt,linecolor=ffxfqq,fillcolor=ffxfqq,fillstyle=solid,opacity=1](5.,4.){0.3}
\pscircle[linewidth=2.pt,linecolor=ffxfqq,fillcolor=ffxfqq,fillstyle=solid,opacity=1](6.,4.){0.3}
\pscircle[linewidth=2.pt,linecolor=ffxfqq,fillcolor=ffxfqq,fillstyle=solid,opacity=1](7.,4.){0.3}
\pscircle[linewidth=2.pt,linecolor=ffxfqq,fillcolor=ffxfqq,fillstyle=solid,opacity=1](7.,3.){0.3}
\pscircle[linewidth=2.pt,linecolor=ffxfqq,fillcolor=ffxfqq,fillstyle=solid,opacity=1](7.,2.){0.3}
\pscircle[linewidth=2.pt,linecolor=green,fillcolor=green,fillstyle=solid,opacity=1](5.,5.){0.3}
\pscircle[linewidth=2.pt,linecolor=green,fillcolor=green,fillstyle=solid,opacity=1](6.,5.){0.3}
\pscircle[linewidth=2.pt,linecolor=green,fillcolor=green,fillstyle=solid,opacity=1](7.,5.){0.3}
\pscircle[linewidth=2.pt,linecolor=green,fillcolor=green,fillstyle=solid,opacity=1](8.,5.){0.3}
\pscircle[linewidth=2.pt,linecolor=green,fillcolor=green,fillstyle=solid,opacity=1](8.,4.){0.3}
\pscircle[linewidth=2.pt,linecolor=green,fillcolor=green,fillstyle=solid,opacity=1](8.,3.){0.3}
\pscircle[linewidth=2.pt,linecolor=green,fillcolor=green,fillstyle=solid,opacity=1](8.,2.){0.3}
\Huge
\rput[tl](-3.44,0.93){$1+3+5+7$}
\rput[tl](4.4,0.93){$=4^2=16$}
\end{pspicture*}
\end{center}

On devine que $u_n$ sera toujours un carré~:

\begin{align*}
u_0&=1=1^2\\
u_1&=4=2^2\\
u_2&=9=3^2\\
u_3&=16=4^2
\end{align*}

Plus généralement, $u_n=(n+1)^2$ pour tout $n\in\mathbb{N}$ -- ce que l'on démontre rigoureusement dans la question suivante.

\item Pour tout $n\in\mathbb{N},$ on note $\mathcal{P}_n$ la propriété \[u_n=(n+1)^2.\]



\begin{itemize}
\item[{\textbullet}] \textbf{Initialisation.} On prouve que $\mathcal{P}_0$ est vraie.

\[
\left.
    \begin{array}{ll}
        u_0&=1 \\
        (0+1)^2&= 1
    \end{array}
\right \}\implies \mathcal{P}_0~\text{est vraie}.
\]



\item[{\textbullet}] \textbf{Hérédité.} Soit $k\in\mathbb{N}$ tel que $\mathcal{P}_k$ soit vraie. On a donc
\[u_k=(k+1)^2.\]

%\medskip

\newtcolorbox{mybox}[1]{colback=green!10!white,colframe=green!80!white,fonttitle=\bfseries,title=#1}
\begin{mybox}{Objectif}{Prouver que $\mathcal{P}_{k+1}$ est vraie, c'est-à-dire que \[u_{k+1}=((k+1)+1)^2,\] ou encore
\[u_{k+1}=(k+2)^2.\]
}\end{mybox}



%\medskip

On part de 
\[u_k=(k+1)^2.\]

On a alors~:


\begin{alignat*}{3}
&u_{k+1}&& =u_k+2k+3 && \text{  (rel. réc. pour } (u_n)_{n\in\mathbb{N}})\\
& && =(k+1)^2+2k+3 && \text{ {(H.R.)}}\\
& && =k^2+2k+1+2k+3 && \text{  (on développe avec l'IR)}\\
& && =k^2+4k+4 && \text{  (on réduit)}\\
& && =(k+2)^2 && \text{  (on factorise avec l'IR)}.\\
\end{alignat*}


La propriété $\mathcal{P}_{k+1}$ est donc vraie.
\item[{\textbullet}] \textbf{Conclusion.} $\mathcal{P}_0$ est vraie et $\mathcal{P}_n$ est héréditaire, donc elle est vraie pour tout $n\in\mathbb{N}.$
\end{itemize}

\end{enumerate}

\end{exo}



\begin{exo}

La suite $(u_n)_{n\in\mathbb{N}}$ est définie par $u_0=1$ et la relation de récurrence

\[u_{n+1}=0,5u_n+3\] pour tout $n\in\mathbb{N}.$

\begin{enumerate}
\item \begin{align*}
u_0&=1\\
u_1&=0,5\times 1+3=3,5\\
u_2&=0,5\times 3,5+3=4,75\\
u_3&=0,5\times 4,75+3=5,375
\end{align*}
\item Pour tout $n\in\mathbb{N},$ on note $\mathcal{P}_n$ la propriété \[u_n\leq 6.\]





\begin{itemize}
\item[{\textbullet}] \textbf{Initialisation.} On prouve que $\mathcal{P}_0$ est vraie.

\[
\left.
    \begin{array}{ll}
        u_0&=1 \\
        1&\leq 6
    \end{array}
\right \}\implies \mathcal{P}_0~\text{est vraie}.
\]



\item[{\textbullet}] \textbf{Hérédité.} Soit $k\in\mathbb{N}$ tel que $\mathcal{P}_k$ soit vraie. On a donc
\[u_k\leq 6.\]

%\medskip

\newtcolorbox{mybox}[1]{colback=green!10!white,colframe=green!80!white,fonttitle=\bfseries,title=#1}
\begin{mybox}{Objectif}{Prouver que $\mathcal{P}_{k+1}$ est vraie, c'est-à-dire que \[u_{k+1}\leq 6.\]
}\end{mybox}



%\medskip

On part de 
\[u_k\leq 6.\]


On multiplie par $\textcolor{red}{0,5}~:$

\begin{align*}u_k\textcolor{red}{\times 0,5}&\leq 6\textcolor{red}{\times 0,5}\\
0,5u_k&\leq 3
\end{align*}

Puis on ajoute  $\textcolor{blue}{3}~:$

\begin{align*}
0,5u_k\textcolor{blue}{+3}&\leq 3\textcolor{blue}{+3}\\
u_{k+1}&\leq 6.
\end{align*}

La propriété $\mathcal{P}_{k+1}$ est donc vraie.
\item[{\textbullet}] \textbf{Conclusion.} $\mathcal{P}_0$ est vraie et $\mathcal{P}_n$ est héréditaire, donc elle est vraie pour tout $n\in\mathbb{N}.$
\end{itemize}

\end{enumerate}

\end{exo}





\begin{exo}

La suite $(u_n)_{n\in\mathbb{N}}$ est définie par $u_0=1$ et la relation de récurrence

\[u_{n+1}=\dfrac{u_n}{u_n+1}\] pour tout $n\in\mathbb{N}.$

\begin{enumerate}
\item \begin{align*}
u_0&=1\\
u_1&=\dfrac{u_0}{u_0+1}=\dfrac{1}{1+1}=\dfrac{1}{2}\\
u_2&=\dfrac{u_1}{u_1+1}=\dfrac{\frac{1}{2}}{\frac{1}{2}+1}=\dfrac{\frac{1}{2}}{\frac{1}{2}+\frac{2}{2}}=\dfrac{\frac{1}{2}}{\frac{3}{2}}=\dfrac{1}{2}\times \dfrac{2}{3}=\dfrac{1}{3}\\
u_3&=\dfrac{u_2}{u_2+1}=\dfrac{\frac{1}{3}}{\frac{1}{3}+1}=\dfrac{\frac{1}{3}}{\frac{1}{3}+\frac{3}{3}}=\dfrac{\frac{1}{3}}{\frac{4}{3}}=\dfrac{1}{3}\times \dfrac{3}{4}=\dfrac{1}{4}
\end{align*}
\item Pour tout $n\in\mathbb{N},$ on note $\mathcal{P}_n$ la propriété \[u_n=\frac{1}{n+1}.\]






\begin{itemize}
\item[{\textbullet}] \textbf{Initialisation.} On prouve que $\mathcal{P}_0$ est vraie.

\[
\left.
    \begin{array}{ll}
        u_0&=1 \\
        \frac{1}{0+1}&=1
    \end{array}
\right \}\implies \mathcal{P}_0~\text{est vraie}.
\]



\item[{\textbullet}] \textbf{Hérédité.} Soit $k\in\mathbb{N}$ tel que $\mathcal{P}_k$ soit vraie. On a donc
\[u_k=\frac{1}{k+1}.\]


%\medskip

\newtcolorbox{mybox}[1]{colback=green!10!white,colframe=green!80!white,fonttitle=\bfseries,title=#1}
\begin{mybox}{Objectif}{Prouver que $\mathcal{P}_{k+1}$ est vraie, c'est-à-dire que \[u_{k+1}=\frac{1}{k+2}.\]
}\end{mybox}



%\medskip

On part de 
\[u_k=\frac{1}{k+1}.\]



On utilise la formule de récurrence et on remplace~:

\[u_{k+1}=\dfrac{u_k}{u_k+1}\overset{\text{H.R.}}{=}\dfrac{\frac{1}{k+1}}{\frac{1}{k+1}+1}=\dfrac{\frac{1}{k+1}}{\frac{1}{k+1}+\frac{k+1}{k+1}}=\dfrac{\frac{1}{k+1}}{\frac{k+2}{k+1}}=\dfrac{1}{\cancel{k+1}}\times \dfrac{\cancel{k+1}}{k+2}=\dfrac{1}{k+2}.\]



La propriété $\mathcal{P}_{k+1}$ est donc vraie.
\item[{\textbullet}] \textbf{Conclusion.} $\mathcal{P}_0$ est vraie et $\mathcal{P}_n$ est héréditaire, donc elle est vraie pour tout $n\in\mathbb{N}.$
\end{itemize}
\end{enumerate}

\end{exo}

\begin{exo}


Soit $q$ un réel différent de 1. Pour tout $n\in\mathbb{N},$ on note $\mathcal{P}_n$ la propriété 

\[1+q+q^2+\cdots+q^n=\frac{q^{n+1}-1}{q-1}.\]







\begin{itemize}
\item[{\textbullet}] \textbf{Initialisation.} On prouve que $\mathcal{P}_0$ est vraie.

La somme dans le membre de gauche va de 1 à $q^n,$ qui, dans le cas où $n=0,$ vaut 1. Autrement dit, la somme dans le membre de gauche est une somme d'un seul terme~: 1.

\medskip

D'un autre côté, $\frac{q^{0+1}-1}{q-1}=\frac{q-1}{q-1}=1.$ $\mathcal{P}_0$ est donc vraie.



\item[{\textbullet}] \textbf{Hérédité.} Soit $k\in\mathbb{N}$ tel que $\mathcal{P}_k$ soit vraie. On a donc
\[1+q+q^2+\cdots+q^k=\frac{q^{k+1}-1}{q-1}.\]

%\medskip

\newtcolorbox{mybox}[1]{colback=green!10!white,colframe=green!80!white,fonttitle=\bfseries,title=#1}
\begin{mybox}{Objectif}{Prouver que $\mathcal{P}_{k+1}$ est vraie, c'est-à-dire que 
\[1+q+q^2+\cdots+q^{k+1}=\frac{q^{k+2}-1}{q-1}.\]

}\end{mybox}



%\medskip

On part de 
\[1+q+q^2+\cdots+q^k=\frac{q^{k+1}-1}{q-1}.\]


On ajoute $\textcolor{blue}{q^{k+1}},$ puis on réduit au même dénominateur~:

\begin{align*}
1+q+q^2+\cdots+q^k\textcolor{blue}{+q^{k+1}}&=\frac{q^{k+1}-1}{q-1}\textcolor{blue}{+q^{k+1}}\\
&=\frac{q^{k+1}-1}{q-1}+\frac{q^{k+1}\textcolor{red}{\times (q-1)}}{\textcolor{red}{q-1}}
\\&=\frac{q^{k+1}-1}{q-1}+\frac{q^{k+1}\times q-q^{k+1}\times 1}{q-1}
\\&=\frac{q^{k+1}-1}{q-1}+\frac{q^{k+2}-q^{k+1}}{q-1}
\\&=\frac{\cancel{q^{k+1}}-1+q^{k+2}-\cancel{q^{k+1}}}{q-1}
\\&=\frac{q^{k+2}-1}{q-1}.
\end{align*}

\medskip

Conclusion~:
\[1+q+q^2+\cdots+q^{k+1}=\frac{q^{k+2}-1}{q-1},\] et la propriété $\mathcal{P}_{k+1}$ est donc vraie.
\item[{\textbullet}] \textbf{Conclusion.} $\mathcal{P}_0$ est vraie et $\mathcal{P}_n$ est héréditaire, donc elle est vraie pour tout $n\in\mathbb{N}.$
\end{itemize}

\medskip

\textbf{Remarque~:} La somme $1+q+q^2+\cdots+q^{n}$ se réécrit sous la forme condensée
\[\sum\limits_{i=0}^nq^i.\]

\end{exo}


\begin{exo}

Soit $x$ un réel positif.

Pour tout $n\in\mathbb{N}^*,$ on note $\mathcal{P}_n$ la propriété 

\[(1+x)^n\geq 1+nx.\]


\begin{itemize}
\item[{\textbullet}] \textbf{Initialisation.} On prouve que $\mathcal{P}_1$ est vraie (\danger Ça démarre à $n=1$ et non pas $n=0.$)


\[
\left.
    \begin{array}{ll}
        (1+x)^1&=1+x \\
        1+1 x&=1+x
    \end{array}
\right \}\implies (1+x)^1\geq 1+1x \implies \mathcal{P}_1~\text{est vraie}.
\]



\item[{\textbullet}] \textbf{Hérédité.} Soit $k\in\mathbb{N}^*$ tel que $\mathcal{P}_k$ soit vraie. On a donc
\[(1+x)^k\geq 1+kx.\]


\newtcolorbox{mybox}[1]{colback=green!10!white,colframe=green!80!white,fonttitle=\bfseries,title=#1}
\begin{mybox}{Objectif}{Prouver que $\mathcal{P}_{k+1}$ est vraie, c'est-à-dire que 
\[(1+x)^{k+1}\geq 1+(k+1)x.\]

}\end{mybox}



%\medskip

On part de 
\[(1+x)^k\geq 1+kx.\]


On multiplie par $\textcolor{red}{1+x}$ et on développe~:

\begin{align*}
(1+x)^k\textcolor{red}{\times (1+x)}&\geq (1+kx)\textcolor{red}{\times (1+x)}\\
(1+x)^{k+1}&\geq 1\times 1+1\times x+kx\times 1+kx\times x\\
(1+x)^{k+1}&\geq 1+ \textcolor{green}{x+kx}+kx^2\\
(1+x)^{k+1}&\geq 1+ \textcolor{green}{(k+1)x}+kx^2\\
\end{align*}

\medskip

Or $kx^2\geq 0,$ car $k$ et $x^2$ sont positifs, donc $1+ (k+1)x+kx^2\geq 1+(k+1)x~;$ et par conséquent

\[(1+x)^{k+1}\geq 1+(k+1)x.\]

La propriété $\mathcal{P}_{k+1}$ est donc vraie.
\item[{\textbullet}] \textbf{Conclusion.} $\mathcal{P}_1$ est vraie et $\mathcal{P}_n$ est héréditaire, donc elle est vraie pour tout $n\in\mathbb{N}^*.$
\end{itemize}


\end{exo}





\begin{exo}

La suite $(u_n)_{n\in\mathbb{N}}$ est définie par $u_0=1$ et la relation de récurrence \[u_{n+1}=\frac{4u_n}{u_n+4}\] pour tout $n\in\mathbb{N}.$ On pose également $v_n=\frac{4}{u_n}$ pour tout $n\in\mathbb{N}.$


\begin{enumerate}
\item Pour démontrer qu'une suite est \textcolor{red}{géométrique}, on part de $v_{n+1}=\cdots$ et on essaye d'aboutir à $\cdots =v_n\textcolor{red}{\times q}.$ Pour une suite \textcolor{blue}{arithmétique}, c'est le même principe~: on part de $v_{n+1}=\cdots$ et on essaye d'aboutir à $\cdots =v_n \textcolor{blue}{+r}.$

\medskip

Pour tout $n\in\mathbb{N}~:$

\[
v_{n+1}=\dfrac{4}{u_{n+1}}=\dfrac{4}{\frac{4u_n}{u_n+4}}=4\times \dfrac{u_n+4}{4u_n}=\dfrac{\cancel{4}\left(u_n+4\right)}{\cancel{4}u_n}=\dfrac{u_n}{u_n}+\dfrac{4}{u_n}=1+v_n.
\]


Conclusion~: pour tout $n\in\mathbb{N},$ $v_{n+1}=v_n+1,$
donc $(v_n)_{n\in\mathbb{N}}$ est arithmétique de raison $r=1.$



\item La suite $(v_n)_{n\in\mathbb{N}}$ est arithmétique de raison $r=1,$ et  $v_0=\frac{4}{u_0}=\frac{4}{1}=4,$ donc pour tout $n\in\mathbb{N}~:$
\[v_n=v_0+n\times r=4+n\times 1=n+4.\]


Enfin $v_n=\dfrac{4}{u_{n}}$ donc
\[u_n=\dfrac{4}{v_{n}}=\dfrac{4}{n+4}.\]

\medskip

\textbf{Remarque~:} On a utilisé~: si $a=\dfrac{b}{c},$ alors $c=\dfrac{b}{a}.$
\end{enumerate}

\end{exo}





\begin{exo}

\begin{enumerate}
\item \begin{enumerate}
\item Chaque année, 80~\% des abonnés se réabonnent (multiplication par $0,8$), puis 40  nouvelles personnes s'inscrivent, donc

\begin{align*}
u_0&=500\\
u_1&=500\times 0,8+40=440\\
u_2&=440\times 0,8+40=392
\end{align*}



Conclusion~: $u_1=440$ et  $u_2=392.$
\item La formule de récurrence est $u_{n+1}=u_n\times 0,8+40,$ ou encore
\[u_{n+1}=0,8u_n+40.\]
\item Avec un schéma~:


~{}\begin{center}
    $\xymatrix@R=0.5pc@C=3pc{
    *+[F]+{500} \ar@/^0.5cm/[r]|{\red{-60}} \ar@/_0.5cm/[r]|{\green{\times 0,88}} & 
    *+[F]+{440} \ar@/^0.5cm/[r]|{\red{-48}} \ar@/_0.5cm/[r]|{\green{\times \approx 0,89}} & *+[F]+{392} \\
    \txt{\blue{$u_0$}}&
    \txt{\blue{$u_1$}}&\txt{\blue{$u_2$}}    
    }$
    \end{center}
    
    
    \medskip
    
    Les résultats en rouge (\textcolor{red}{$-6$} et \textcolor{red}{$-48$}) sont différents, donc $u$ n'est pas arithmétique.  Les résultats en vert (\textcolor{green}{$0,88$} et \textcolor{green}{$0,89$}) sont différents, donc $u$ n'est pas géométrique.
    
\end{enumerate}
\item 
On pose $v_n=u_n -200$ pour tout $n\in\mathbb{N}.$
\begin{enumerate}
\item Pour tout $n\in\mathbb{N}~:$



\begin{alignat*}{3}
&v_{n+1}&& =u_{n+1}-200 && \text{  (déf. de } (v_n)_{n\in\mathbb{N}})\\
& && =(0,8u_n+40)-200 && \text{  (rel. réc. pour } (u_n)_{n\in\mathbb{N}})\\
& && =0,8u_n-160 && \text{  (calcul)}\\
& && =0,8\left(u_n-\frac{160}{0,8}\right) && \text{  (factorisation)}\\
& && =0,8(u_n-200) && \text{  (calcul)}\\
& && =0,8v_n&& \text{  (déf. de } (v_n)_{n\in\mathbb{N}})\\
\end{alignat*}

\medskip

Conclusion~: pour tout $n\in\mathbb{N},$ $v_{n+1}=0,8v_n,$
donc $(v_n)_{n\in\mathbb{N}}$ est géométrique de raison $q=0,8.$



\item La suite $(v_n)_{n\in\mathbb{N}}$ est géométrique de raison $q=0,8,$ et  $v_0=u_0-200=500-200=300,$ donc pour tout $n\in\mathbb{N}~:$
\[v_n=v_0\times q^n=300\times 0,8^n.\]

Enfin $v_n=u_n-200$ donc
\[u_n=v_n+200=300\times 0,8^n+200.\]
\item Suivant ce modèle, en 2030 (donc après 10 ans), il devrait y avoir
\[u_{10}=300\times 0,8^{10}+200\approx 232~\text{abonnés}.\]
\end{enumerate}
\item Pour tout $n\in\mathbb{N},$ on note $\mathcal{P}_n$ la propriété
\[u_n=300\times 0,8^n+200.\]






\begin{itemize}
\item[{\textbullet}] \textbf{Initialisation.} On prouve que $\mathcal{P}_0$ est vraie.

\[
\left.
    \begin{array}{ll}
        u_0&=500 \\
        300\times 0,8^0+200&= 300\times 1+200=500
    \end{array}
\right \}\implies \mathcal{P}_0~\text{est vraie}.
\]



\item[{\textbullet}] \textbf{Hérédité.} Soit $k\in\mathbb{N}$ tel que $\mathcal{P}_k$ soit vraie. On a donc
\[u_k=300\times 0,8^k+200.\]

%\medskip

\newtcolorbox{mybox}[1]{colback=green!10!white,colframe=green!80!white,fonttitle=\bfseries,title=#1}
\begin{mybox}{Objectif}{Prouver que $\mathcal{P}_{k+1}$ est vraie, c'est-à-dire que \[u_{k+1}=300\times 0,8^{k+1}+200.\]

}\end{mybox}



%\medskip

On part de 
\[u_k=300\times 0,8^k+200.\]


On a alors~:


\begin{alignat*}{3}
&u_{k+1}&& =0,8u_k+40 && \text{  (rel. réc. pour } (u_n)_{n\in\mathbb{N}})\\
& && =0,8\left(300\times 0,8^k+200\right)+40 && \text{ {(H.R.)}}\\
& && =300\times 0,8\times 0,8^k+0,8\times 200+40 && \text{  (on développe)}\\
& && =300\times 0,8^{k+1}+200 && \text{  (calcul)}.\\
\end{alignat*}



La propriété $\mathcal{P}_{k+1}$ est donc vraie.
\item[{\textbullet}] \textbf{Conclusion.} $\mathcal{P}_0$ est vraie et $\mathcal{P}_n$ est héréditaire, donc elle est vraie pour tout $n\in\mathbb{N}.$
\end{itemize}
\end{enumerate}

\end{exo}



\begin{exo}

On définit deux suites $\left(u_n\right)_{n\in\mathbb{N}}$ et $\left(v_n\right)_{n\in\mathbb{N}}$ par $u_0=0,$ $v_0=8$ et les relations de récurrence~:

\[\begin{cases}
u_{n+1}&=\frac{3}{4}u_n+\frac{1}{4}v_n\\
v_{n+1}&=\frac{1}{4}u_n+\frac{3}{4}v_n\end{cases}\] pour tout $n\in\mathbb{N}.$

\begin{enumerate}
\item \setlength{\columnseprule}{1pt}

\begin{multicols}{2}
\begin{align*}
u_0&=0\\
u_1&=\frac{3}{4}\times u_0+\frac{1}{4}\times v_0=\frac{3}{4}\times 0+\frac{1}{4}\times 8=2\\
u_2&=\frac{3}{4}\times u_1+\frac{1}{4}\times v_1=\frac{3}{4}\times 2+\frac{1}{4}\times 6=3
\end{align*}

\begin{align*}
v_0&=8\\
v_1&=\frac{1}{4}\times u_0+\frac{3}{4}\times v_0=\frac{1}{4}\times 0+\frac{3}{4}\times 8=6\\
v_2&=\frac{1}{4}\times u_1+\frac{3}{4}\times v_1=\frac{1}{4}\times 2+\frac{3}{4}\times 6=5
\end{align*}
\end{multicols}

\medskip


\begin{center}
\newrgbcolor{ududff}{0.30196078431372547 0.30196078431372547 1.}
\psset{xunit=1.0cm,yunit=1.0cm,algebraic=true,dimen=middle,dotstyle=o,dotsize=5pt 0,linewidth=2.pt,arrowsize=3pt 2,arrowinset=0.25}
\begin{pspicture*}(-0.74,-0.82)(8.84,0.88)
\psaxes[labelFontSize=\scriptstyle,xAxis=true,yAxis=false,Dx=1.,Dy=1.,ticksize=-2pt 0,subticks=2]{->}(0,0)(-0.74,-0.82)(8.84,0.88)
\rput[tl](-0.3,0.64){\ududff{$u_0$}}
\rput[tl](1.7,0.64){\ududff{$u_1$}}
\rput[tl](2.7,0.64){\ududff{$u_2$}}
\rput[tl](4.7,0.64){\red{$v_2$}}
\rput[tl](5.7,0.64){\red{$v_1$}}
\rput[tl](7.7,0.64){\red{$v_0$}}
\psdots[dotstyle=*,linecolor=ududff](0.,0.)
\psdots[dotstyle=*,linecolor=ududff](2.,0.)
\psdots[dotstyle=*,linecolor=ududff](3.,0.)
\psdots[dotstyle=*,linecolor=red](5.,0.)
\psdots[dotstyle=*,linecolor=red](6.,0.)
\psdots[dotstyle=*,linecolor=red](8.,0.)
\end{pspicture*}
\end{center}

\medskip

\textbf{Remarque~:} Pour placer $u_{n+1}$ et $v_{n+1},$ on coupe le segment $\left[u_n;v_n\right]$ en $4~;$ et on place $u_{n+1}$ au quart du segment, $v_{n+1}$ aux trois-quarts du segment.



\item On pose $s_n=v_n+u_n$ et $d_n=v_n-u_n$ pour tout $n\in\mathbb{N}.$

\begin{itemize}
\item[\textbullet] Pour tout $n\in\mathbb{N}~:$
\begin{align*}s_{n+1}&=v_{n+1}+u_{n+1}\\
&=\left(\frac{1}{4}u_n+\frac{3}{4}v_n\right)+\left(\frac{3}{4}u_n+\frac{1}{4}v_n\right)\\
&=\frac{4}{4}v_n+\frac{4}{4}u_n\\
&=v_n+u_n\\
&=s_n.\end{align*}

\medskip

Conclusion~: pour tout $n\in\mathbb{N},$ $s_{n+1}=s_n$ donc $\left(s_n\right)_{n\in\mathbb{N}}$ est constante. Et comme $s_0=v_0+u_0=8+0=8,$ $\left(s_n\right)_{n\in\mathbb{N}}$ est constante égale à 8~:
\[\text{pour tout }n\in\mathbb{N},~s_n=8.\]
\item[\textbullet] Pour tout $n\in\mathbb{N}~:$
\begin{align*}d_{n+1}&=v_{n+1}-u_{n+1}\\
&=\left(\frac{1}{4}u_n+\frac{3}{4}v_n\right)-\left(\frac{3}{4}u_n+\frac{1}{4}v_n\right)\\
&=\frac{2}{4}v_n-\frac{2}{4}u_n\\
&=\frac{1}{2}\left(v_n-u_n\right)\\
&=\frac{1}{2}d_n
.\end{align*}

\medskip

Conclusion~: pour tout $n\in\mathbb{N},$ $d_{n+1}=\frac{1}{2}d_n$ donc $\left(d_n\right)_{n\in\mathbb{N}}$ est géométrique de raison $q=\frac{1}{2}.$
\end{itemize}
\item  La suite $(d_n)_{n\in\mathbb{N}}$ est géométrique de raison $q=\frac{1}{2},$ et  $d_0=v_0-u_0=8-0=8,$ donc pour tout $n\in\mathbb{N}~:$
\[d_n=d_0\times q^n=8\times \left(\frac{1}{2}\right)^n.\] On sait par ailleurs que $s_n=8$ pour tout $n\in\mathbb{N}.$

\medskip
Les relations

\[\begin{cases}
s_n&=v_n+u_n\\
d_n&=v_n-u_n\end{cases}\]

se réécrivent donc

\[\begin{cases}
8&=v_n+u_n\\
8\times \left(\frac{1}{2}\right)^n&=v_n-u_n\end{cases}\]

On ajoute membre à membre~:

\begin{align*}
8+8\times \left(\frac{1}{2}\right)^n&=v_n+\cancel{u_n}+v_n-\cancel{u_n}\\
8+8\times \left(\frac{1}{2}\right)^n&=2v_n\\
\frac{8+8\times \left(\frac{1}{2}\right)^n}{2}&=v_n\\
4+4\times \left(\frac{1}{2}\right)^n&=v_n
\end{align*}

\medskip

Enfin, comme $s_n=v_n+u_n~:$
\[u_n=s_n-v_n=8-\left(4+4\times \left(\frac{1}{2}\right)^n\right)=8-4-4\times \left(\frac{1}{2}\right)^n=4-4\times \left(\frac{1}{2}\right)^n.\]

\medskip

Conclusion~: pour tout $n\in\mathbb{N},$

\[\boxed{v_n=4+4\times \left(\frac{1}{2}\right)^n}\qquad\qquad \boxed{u_n=4-4\times \left(\frac{1}{2}\right)^n}\]




\end{enumerate}


\end{exo}

\begin{exo}

\item \setlength{\columnseprule}{1pt}

\begin{multicols}{3}

\begin{center}
\textbf{Programme}
\end{center}

\vspace*{0.2cm}


\begin{lstlisting}
for i in range(1,6):
	print(i**2)
\end{lstlisting}



\vspace*{2cm}


\columnbreak


\begin{center}
\textbf{Traduction en français}
\end{center}


\newtcolorbox{mybox}[1]{colback=white,colframe=black!80!white,fonttitle=\bfseries,title=#1}\begin{mybox}{}

$\text{Pour i allant de 1 à 5:}$

$\qquad\text{afficher }\text{i}^2$
\end{mybox}

\vspace*{4cm}


\columnbreak

\begin{center}
\textbf{Commentaires}
\end{center}

\begin{itemize}
\item[\textbullet]  \danger La commande \[\text{for i in range(1,6)}\] signifie que i va de 1 à 5 -- il y a un décalage à la fin.
\item[\textbullet] On n’oublie pas les \og  : \fg~{} à la fin de la première ligne. L'incrément qui suit (équivalent à une tabulation sur Thonny) est alors automatiquement inséré lorsqu'on passe à la ligne.
\end{itemize}

\end{multicols}



\end{exo}

\begin{exo}
~{}

\setlength{\columnseprule}{1pt}

\begin{multicols}{2}

\begin{center}
\textbf{Programme}
\end{center}

\begin{lstlisting}
for i in range(1,11):
	print(8*i)
\end{lstlisting}








\columnbreak

\begin{center}
\textbf{Commentaires}
\end{center}

On affiche les résultats les uns en-dessous des autres~:

\[8\times 1=8,~8\times 2=16,~8\times 3=24,~\dots,~8\times 10=80.\]

\end{multicols}

\end{exo}

%\newpage

\begin{exo}

On explique le fonctionnement du programme en remplissant un tableau.


\setlength{\columnseprule}{1pt}

\begin{multicols}{3}

\begin{center}
\textbf{Programme}
\end{center}

\begin{lstlisting}
s=0
for i in range(1,101):
	s=s+i
print(s)
\end{lstlisting}



\vspace*{6cm}


\columnbreak

\begin{center}
\textbf{Tableau}
\end{center}


On a une boucle \textbf{Pour}, où i va de 1 à 100.

\medskip

\begin{center}


\begin{tabular}{|c|c|} \hline
\textbf{Valeur de i}& \textbf{Valeur de s}\\ \hline
\cellcolor{gray}& 0\\ \hline
1& $0+1=1$\\ \hline
2&$1+2=3$\\ \hline
3&$3+3=6$\\ \hline
4&$6+4=10$ \\ \hline
$\cdots$&$\cdots$\\ \hline
99&$\cdots$\\ \hline
100&$\cdots$\\ \hline
\end{tabular}
\end{center}
\vspace*{3.5cm}


\columnbreak

\begin{center}
\textbf{Explications}
\end{center}

\begin{itemize}
\item[\textbullet] La 1\up{re} ligne du tableau ci-contre correspond à la 1\up{re} ligne du code~: $\text{s}=0$ et i n'existe pas encore.
\item[\textbullet] Dans la boucle, i va de $1$ à $100,$ donc on écrit les valeurs de 1 jusqu'à 100 dans la 1\up{re} colonne.
\item[\textbullet] À chaque étape de la boucle \textbf{Pour}, s reçoit la valeur $\text{s}+\text{i}.$

 Donc au début, lorsque $\text{i}=1,$ s reçoit la valeur $\text{s}+\text{i}=0+1=1.$ La valeur de s a donc été modifiée et il vaut maintenant 1 (et non plus 0).
\item[\textbullet] Ensuite, lorsque $\text{i}=2,$ s reçoit la nouvelle valeur $\text{s}+\text{i}=1+2=3.$ La valeur de s a été une nouvelle fois modifiée.
\item[\textbullet] Puis quand $\text{i}=3,$ s reçoit la nouvelle valeur $\text{s}+\text{i}=3+3=6.$ Cela continue ainsi de suite jusqu'en bas du tableau.
\end{itemize}
\end{multicols}

\medskip

Finalement, on part de 0, puis on ajoute 1, puis 2, puis 3~; et ainsi de suite jusque 100. On calcule donc
\[1+2+3+\cdots+ 99+100.\] Le résultat, \np{5050}, s'affiche en fin de programme.

\end{exo}


\begin{exo}

On édite en machine un programme Python qui calcule~:
\[10!=1\times 2\times 3\times \cdots\times 10.\]

On s'inspire pour cela du programme précédent, avec trois différences~:

\begin{itemize}
\item[\textbullet] on \textbf{multiplie} par i à chaque étape, au lieu \textbf{d'ajouter} i~;
\item[\textbullet] au début $\text{s}=1,$ élément neutre de la multiplication (si on démarrait avec $\text{s}=0,$ la valeur de s vaudrait toujours 0)~;
\item[\textbullet] la boucle ne va que de 1 à 10.
\end{itemize}

\medskip

\begin{lstlisting}
s=1
for i in range(1,11):
	s=s*i
print(s)
\end{lstlisting}


\end{exo}

%\newpage

\begin{exo}

La suite $(u_n)_{n\in\mathbb{N}}$ est définie par $u_0=3$ et la formule de récurrence
\[u_{n+1}=2u_n-1\] pour tout $n\in\mathbb{N}.$

\medskip

Nous allons calculer les premiers termes, puis, comme dans l'exercice 39, compléter un tableau avec les calculs à chaque étape de la boucle \textbf{Pour}.


\setlength{\columnseprule}{1pt}

\begin{multicols}{3}

\begin{center}
\textbf{Calcul des premiers termes}
\end{center}

\medskip

\begin{align*}
u_0&=3\\
u_1&=2\times 3-1=5\\
u_2&=2\times 5-1=9\\
u_3&=2\times 9-1=17\\
u_4&=2\times 17-1=33
\end{align*}

\vspace*{0.cm}

\columnbreak
\begin{center}
\textbf{Programme}
\end{center}

\begin{lstlisting}
u=3
for i in range(4):
	u=2*u-1
print(u)
\end{lstlisting}



\vspace*{1.cm}


\columnbreak



\begin{center}
\textbf{Tableau}
\end{center}


On a une boucle \textbf{Pour}, où i va de 0 à 3.\footnote{La commande \[\text{for i in range(n):}\] signifie que i va de 0 à $\text{n}-1.$}

\medskip

\begin{center}


\begin{tabular}{|c|cc|} \hline
\textbf{Valeur de i}& \textbf{Valeur de u}&\\ \hline
\cellcolor{gray}& 3&\textcolor{red}{$\leftarrow {u_0}$}\\ \hline
0& $2\times 3-1=5$&\textcolor{red}{$\leftarrow {u_1}$}\\ \hline
1&$2\times 5-1=9$&\textcolor{red}{$\leftarrow {u_2}$}\\ \hline
2& $2\times 9-1=17$&\textcolor{red}{$\leftarrow {u_3}$}\\ \hline
3&$2\times 17-1=33$&\textcolor{red}{$\leftarrow {u_4}$} \\ \hline

\end{tabular}
\end{center}
%\vspace*{3.5cm}


\end{multicols}


\medskip

Dans la boucle \textbf{Pour}, u prend successivement les valeurs $u_0,$ $u_1,$ $u_2,$ $u_3$ et $u_4.$ C'était prévisible, puisque l'instruction \begin{lstlisting}
u=2*u-1
\end{lstlisting} est la même que la formule de récurrence.

\medskip

Conclusion~: la valeur affichée en sortie est $u_4=33.$

\end{exo}

\begin{exo}

Commençons par des rappels concernant les listes, avec quelques exemples~:

\begin{itemize}
\item[\textbullet] La commande \[\text{L}=\left[5,6,10\right]\] crée une liste L de trois éléments. Le premier, $\text{L}\left[0\right],$ est égal à 5~; le deuxième, $\text{L}\left[1\right],$ est égal à 6~; le troisième, $\text{L}\left[2\right],$ est égal à 10. On notera en particulier l'indexation des termes  à partir de $0.$
\item[\textbullet] La commande \[\text{L.append(2)}\] ajoute un terme à la liste, égal à 2. On aura donc ensuite une liste de 4 éléments~: $\text{L}=\left[5,6,10,2\right].$
\item[\textbullet] La commande \[\text{len(L)}\] renvoie la longueur de la liste.
\item[\textbullet] La commande \[\text{L}=\left[~\right]\] crée une liste vide (donc de longueur 0).
\end{itemize}

\medskip

Venons-en à l'exercice. On souhaite afficher la liste des nombres de la table de 8~:
\[\left[8~,~16~,~24~,~\cdots~,~80\right].\]

Pour cela, on crée une liste vide L, puis on reprend le programme de l'exercice 38, en ajoutant les nombres de la table à la liste L au fur et à mesure de leur calcul~:

\medskip

\begin{lstlisting}
L=[]
for i in range(1,11):
	L.append(8*i)
print(L)
\end{lstlisting}



\end{exo}



\begin{exo}

On reprend la suite de l'exercice 41~: $u_0=3$ et $u_{n+1}=2u_n-1$ pour tout $n\in\mathbb{N}.$

\medskip

Pour afficher la liste des termes de $u_0$ à $u_6,$ on crée d'abord une liste qui contient le premier terme avec la commande  \begin{lstlisting}
L=[3]
\end{lstlisting} Ensuite, on reprend le programme de l'exercice 41, en ajoutant à la liste L chacun des termes de la suite au fur et à mesure de leur calcul.

\medskip

Pour plus de clarté, on a ajouté un tableau explicatif~:

\medskip

\setlength{\columnseprule}{1pt}

\begin{multicols}{2}


\begin{center}
\textbf{Programme}
\end{center}

\begin{lstlisting}
u=3
L=[3]
for i in range(6):
	u=2*u-1
	L.append(u)
print(u)
\end{lstlisting}



\vspace*{0.25cm}


\columnbreak



\begin{center}
\textbf{Tableau}
\end{center}



\begin{center}


\begin{tabular}{|c|c|l|} \hline
\textbf{Valeur de i}& \textbf{Valeur de u}&\text{liste L}\\ \hline
\cellcolor{gray}& 3&$\text{L}=\left[3\right]$\\ \hline
0& $2\times 3-1=5$ & $\text{L}=\left[3,5\right]$ \\ \hline
1&$2\times 5-1=9$&$\text{L}=\left[3,5,9\right]$\\ \hline
2& $2\times 9-1=17$&$\text{L}=\left[3,5,9,17\right]$\\ \hline
3&$2\times 17-1=33$&$\text{L}=\left[3,5,9,17,33\right]$ \\ \hline
4&$2\times 33-1=65$&$\text{L}=\left[3,5,9,17,33,65\right]$ \\ \hline
5&$2\times 65-1=129$&$\text{L}=\left[3,5,9,17,33,65,129\right]$ \\ \hline
\end{tabular}
\end{center}
%\vspace*{3.5cm}


\end{multicols}

\end{exo}

\begin{exo}

~{}

\setlength{\columnseprule}{1pt}

\begin{multicols}{2}

\begin{center}\textbf{Programme 1}\end{center}

\begin{lstlisting}
x=3
if x==4:
	print(5*x)
else:
	print(2*x)
\end{lstlisting}

\medskip

Puisque $\text{x}\not=4,$ le programme affiche
\[2\times \text{x}=2\times 3=6.\]

\begin{center}\textbf{Programme 2}\end{center}

\begin{lstlisting}
x=3
if x<=4:
	print(5*x)
else:
	print(2*x)
\end{lstlisting}

\medskip

Puisque $\text{x}\leq 4,$ le programme affiche
\[5\times \text{x}=5\times 3=15.\]

\end{multicols}

\end{exo}

\begin{exo}



On commence par deux remarques~:

\begin{itemize}
\item[\textbullet] Un entier $\text{n}\geq 1$ est un diviseur de 30 si, et seulement si, $30\%\text{n}=0.$
\item[\textbullet] En python, on teste les égalités avec $==.$ Par exemple, la commande \begin{lstlisting}
4==4
\end{lstlisting} renvoie \textbf{True}~; tandis que
\begin{lstlisting}
4==5
\end{lstlisting} renvoie \textbf{False}.
\end{itemize}

\medskip

On édite un programme Python qui renvoie la liste des diviseurs positifs de 30~:

\medskip

\begin{lstlisting}
L=[]
for i in range(1,31):
	if 30%i==0:
		L.append(i)
print(L)
\end{lstlisting}



\end{exo}

\begin{exo}

On édite la fonction~:

\begin{lstlisting}
def f(x):
	return x**2
\end{lstlisting}



\medskip

On obtient \begin{align*}
\text{f}(3)&=3^2=9\\
\text{f}(-2)&=(-2)^2=4
\end{align*}


\end{exo}

\begin{exo}

La fonction

\begin{lstlisting}
def g():
	return 5
\end{lstlisting}

\medskip

renvoie toujours la valeur 5.

\medskip

\textbf{Remarque~:} Pour lancer la fonction, il faut entrer la commande \begin{lstlisting}
g()
\end{lstlisting} sans oublier les parenthèses, mais sans rien écrire à l'intérieur\footnote{Cela fait une différence notable avec les mathématiques,  où une fonction dépend forcément d'une (ou plusieurs) variables.}.


\end{exo}



\begin{exo}

~{}

\begin{lstlisting}
def moyenne(a,b):
	return (a+b)/2
\end{lstlisting}


 \end{exo}
 
 
 \begin{exo}
 
 ~{}
 
 \begin{lstlisting}
def transforme(note):
	x=1.2*note
	if x<=20:
		return x
	else:
		return 20
\end{lstlisting}


 
 \end{exo}
 
 \begin{exo}

On reprend encore la suite définie par  $u_0=3$ et $u_{n+1}=2u_n-1$ pour tout $n\in\mathbb{N}.$

\medskip

On recopie quasiment à l'identique les programmes que nous avons écrits dans les exercices 41 et 43. Il y a tout de même trois différences~:

\begin{itemize}
\item[\textbullet] on utilise une fonction~;
\item[\textbullet] la boucle \textbf{Pour} a n étapes, et non plus 4 (ex 41) ou  6 (ex 43)~;
\item[\textbullet] on utilise \textbf{return} au lieu de \textbf{print} pour renvoyer le résultat.
\end{itemize}

\begin{enumerate}
\item Fonction qui renvoie la valeur de $u_{\text{n}}~:$

\begin{lstlisting}
def terme(n):
	u=3
	for i in range(n):
		u=2*u-1
	return u
\end{lstlisting}


 \item Fonction qui renvoie la liste de tous les termes de $u_0$ à $u_{\text{n}}~:$
 
 
 \begin{lstlisting}
def liste(n):
	u=3
	L=[3]
	for i in range(n):
		u=2*u-1
		L.append(u)
	return L
\end{lstlisting}

 
 \end{enumerate}

\end{exo}


\begin{exo}

On explique le fonctionnement du programme avec un tableau~:

\medskip

\setlength{\columnseprule}{1pt}

\begin{multicols}{2}

\begin{center}
\textbf{Programme}
\end{center}

\begin{lstlisting}
def somme(n):
	s=0
	for k in range(1,n+1):
		s=s+1/k
	return s
\end{lstlisting}



\vspace*{3cm}


\columnbreak

\begin{center}
\textbf{Tableau}
\end{center}

Avec la commande somme(100), k va de 1 à 100, puisque $\text{n}+1=100+1=101.$

\medskip

On laisse volontairement les résultats sous forme de sommes de fractions.

\medskip

\begin{center}

\renewcommand{\arraystretch}{1.25}

\begin{tabular}{|c|c|} \hline
\textbf{Valeur de k}& \textbf{Valeur de s}\\ \hline
\cellcolor{gray}& 0\\ \hline
1& $0+\frac{1}{1}=\frac{1}{1}$\\ \hline
2&$\frac{1}{1}+\frac{1}{2}$\\ \hline
3&$\frac{1}{1}+\frac{1}{2}+\frac{1}{3}$\\ \hline
$\cdots$&$\cdots$\\ \hline
100&$\frac{1}{1}+\frac{1}{2}+\frac{1}{3}+\cdots+\frac{1}{100}$\\ \hline
\end{tabular}
\end{center}


\end{multicols}

\medskip

Conclusion~: somme(100) renvoie la valeur de la somme
\[\frac{1}{1}+\frac{1}{2}+\frac{1}{3}+\cdots+\frac{1}{100}\] 
(qui vaut environ $5,19$).




\end{exo}

\begin{exo}


~{}

\begin{lstlisting}
def mystere(L):
	M=L[0]
	for i in range(len(L)):
		if L[i]>M:
			M=L[i]
	return M
\end{lstlisting}




On explique encore une fois le programme avec un tableau~:

\medskip

\item \setlength{\columnseprule}{1pt}

\begin{multicols}{2}




\begin{center}
\textbf{Commentaires}
\end{center}


\medskip

On entre la commande mystere([2, 3, 7,0]). Dans ce cas, en notant $\text{L}=\left[2,3,7,0\right]~:$

\medskip

\begin{itemize}
\item[\textbullet] la longueur de la liste L est 4, donc $\text{len(L)}=4~;$ et i va de 0 à 3~;
\item[\textbullet] $\text{L}\left[0\right]=2,$ $\text{L}\left[1\right]=3,$ $\text{L}\left[2\right]=7$ et $\text{L}\left[3\right]=0~;$
\item[\textbullet] Au départ, $\text{M}=\text{L}\left[0\right]=2.$ Puis, à chaque étape de la boucle, on regarde si $\text{L}\left[\text{i}\right]>\text{M}.$ Si c'est le cas, on remplace M par $\text{L}\left[\text{i}\right].$
\end{itemize}



\columnbreak

\begin{center}
\textbf{Tableau}
\end{center}

\begin{center}

%\renewcommand{\arraystretch}{1.25}


\begin{tabular}{|c|cc|c|} \hline
\textbf{Valeur de i}&\textbf{A-t-on L[i]>M~?}& & \textbf{Valeur de M}\\ \hline
\cellcolor{gray}&\cellcolor{gray}&\cellcolor{gray} &$2$\\ \hline
0&  A-t-on $\text{L}\left[0\right]>2~?$ $2>2~?$ &\textcolor{blue}{non}&2\\ \hline
1& A-t-on $\text{L}\left[1\right]>2~?$ $3>2~?$ &\textcolor{red}{oui}&$3$\\ \hline
2& A-t-on $\text{L}\left[2\right]>3~?$ $7>3~?$ &\textcolor{red}{oui}&$7$\\ \hline
3& A-t-on $\text{L}\left[3\right]>7~?$ $0>7~?$ &\textcolor{blue}{non}&$7$\\ \hline
\end{tabular}
\end{center}

\vspace*{1cm}



\end{multicols}

\medskip

Conclusion~: la valeur renvoyée en sortie est 7, maximum de la liste L. Notez que l'on aurait pu obtenir ce maximum avec la simple commande

\begin{lstlisting}
max(L)
\end{lstlisting}



\end{exo}


\begin{exo}


\begin{enumerate}
\item Les termes successifs sont~:
\[26 - 13 - 40 - 20 - 10 -5 -16 - 8 - 4 - 2 - 1 - 4 - 2 - 1- \cdots\]

On aboutit à une suite périodique~; phénomène qui semble d'ailleurs avoir lieu quel que soit le nombre de départ (on invite le lecteur curieux à faire des essais avec d'autres entiers et à lire l'article Wikipédia sur la conjecture de Syracuse).
\item Avec Thonny, un entier n est pair si, et seulement si, $\text{n}\%2=0.$ On utilise ce résultat pour tester la parité~:

\medskip

\begin{lstlisting}
def syracuse():
	u=26
	L=[26]
	for i in range(10):
		if n%2==0:
			u=u/2
		else:
			u=3*u+1
		L.append(u)
	return L
\end{lstlisting}


 \end{enumerate}

\end{exo} 


\section{Dénombrement}


\begin{exo}



\begin{enumerate}
\item Pour chacun des 4 symboles, il y a 12 possibilités, donc au total $12\times 12\times 12\times 12=12^4=\np{20736}$ codes différents possibles.

\medskip

\textbf{Remarque~:} Chaque code est ce que l'on appelle une \textbf{4-liste d'un ensemble à 12 éléments}.
\item On raisonne comme dans la question 1~: il y a $11^4=\np{14641}$ codes d'entrée ne comportant pas la lettre A.
\item Il y a 12 possibilités pour le 1\up{er} symbole, 11 pour le 2\up{e} (car il diffère du premier symbole), puis 10 pour le 3\up{e}~; et enfin 9 pour le 4\up{e}. Donc au total,
\[12\times 11\times 10\times 9=\np{11880}\] codes avec 4 symboles différents.

\medskip

\textbf{Remarque~:} Chaque code est ce que l'on appelle un \textbf{arrangement de 4 éléments d'un ensemble à 12 éléments}. Le cours donne directement la réponse~:
\[\text{nombre d'arrangements}=\frac{12!}{(12-4)!}=\frac{12!}{8!}=
\frac{12\times 11\times 10\times 9\times \cancel{8}\times \cancel{7}\times\cdots\times \cancel{1}}{\cancel{8}\times \cancel{7}\times\cdots\times \cancel{1}}=12\times 11\times 10\times 9.\]
\end{enumerate}

\end{exo}

\begin{exo}




\begin{enumerate}
\item On obtient les anagrammes de VOYAGE en permutant les lettres de toutes les façons possibles. La lettre V peut prendre 6 positions différentes, puis il reste 5 positions possibles pour le O, puis 4 pour le Y, etc. Au final, il y a
\[6!=6\times 5\times 4\times 3\times 2\times 1=720\] anagrammes.

\medskip

\textbf{Remarque~:} On dit qu'il y a 720 \textbf{permutations} possibles des lettres.
\item Si les 8 lettres du mot ANTILLES étaient toutes différentes, il y aurait $8!=\np{40320}$ anagrammes différentes. Mais il y a deux \og L \fg , donc chaque anagramme est comptée deux fois. En effet, si on différencie les deux \og L \fg~{} en les coloriant, les mots INA\textcolor{red}{L}S\textcolor{green}{L}ET et INA\textcolor{green}{L}S\textcolor{red}{L}ET, par exemple, semblent différents~; mais ils représentent en réalité le même mot INALSLET. Finalement, il n'y a que
\[\np{40320}\div 2=\np{20160}\] anagrammes différentes.
\end{enumerate}

\end{exo}




\begin{exo}

Il y a 2 possibilités pour la 1\up{re} réponse, 2 possibilités pour la 2\up{e}, 2 pour la 3\up{e}, etc. Donc au total $2^{10}=\np{1024}$ façons possibles de remplir le questionnaire.

\medskip

\textbf{Remarques~:}

\begin{itemize}
\item[\textbullet] Il s'agit du nombre de 10-listes d'un ensemble à deux éléments (ces deux éléments étant Vrai/Faux).
\item[\textbullet] Au collège, vous auriez pu présenter la solution avec un arbre~:

\begin{center}
\begin{tikzpicture}[
  level 1/.style={level distance=6em, sibling distance=5em},
  level 2/.style={level distance=6em, sibling distance=2.5em},
  level 3/.style={level distance=6em, sibling distance=1em},
  grow'=right,
  ]
  \coordinate                   % joli sommet de l'arbre
    child foreach \evenemi in {V, F}
      {
        node {$\evenemi$}
        child foreach \evenemii in {V, F}
          {
            node {$\evenemii$}
            child foreach \evenemiii in {$\cdots$, $\cdots$}
              { node {\evenemiii} }
          }
      };
\end{tikzpicture}
\end{center}

\end{itemize}


\end{exo}

\begin{exo}



\begin{enumerate}
\item Les podiums sont les arrangements de 3 éléments d'un ensemble à 8 éléments (car les trois premiers de la course sont différents), donc il y en a 
\[\frac{8!}{(8-3)!}=\frac{8!}{5!}=8\times 7\times 6=336.\]
\item On s'intéresse à l'événement contraire~: on compte le nombre de podiums sans aucun Américain. Il s'agit du nombre d'arrangements de 3 éléments d'un ensemble à 5 éléments (les 5 non Américains)~; il y en a 
\[\frac{5!}{(5-3)!}=\frac{5!}{2!}=5\times 4\times 3=60.\]

Conclusion~: il reste \[336-60=276\] podiums comportant au moins un Américain.
\end{enumerate}

\end{exo}


\begin{exo}

On a déjà rencontré ce code dans l'exercice 40 (à quelques différences près). On explique son fonctionnement avec un tableau~:

\medskip

\setlength{\columnseprule}{1pt}

\begin{multicols}{2}

\begin{center}
\textbf{Programme}
\end{center}

\begin{lstlisting}
def fact(n):
	p=1
	for i in range(1,n+1):
		p=p*i
	return p
\end{lstlisting}



\vspace*{0.5cm}

\columnbreak

\begin{center}
\textbf{Tableau}
\end{center}

\medskip

On rentre par exemple la commande fact(4) -- donc i va alors de 1 à 4~:


\begin{center}


\begin{tabular}{|c|c|} \hline
\textbf{Valeur de i}& \textbf{Valeur de p}\\ \hline
\cellcolor{gray}& 1\\ \hline
1& $1\times 1=1$\\ \hline
2&$1\times 2=2$\\ \hline
3&$2\times 3=6$\\ \hline
4&$6\times 4=24$ \\ \hline
\end{tabular}
\end{center}

\end{multicols}

\medskip

La valeur renvoyée est 24, qui correspond à 
\[4!=1\times 2\times 3\times 4.\]




\end{exo}


\begin{exo}

Pour simplifier et sans rien enlever à la généralité du raisonnement, on suppose que les questions sont numérotées de 1 à 6 en histoire et de 1 à 5 en géographie, et que le candidat connaît les questions n°1, 2, 3 en histoire, n°1 et 2 en géographie.

Dans le tableau ci-dessous, les questions connues sont écrites en bleu, les questions inconnues sont écrites en rouge.

\medskip



On a colorié les cases de trois couleurs~:
\begin{itemize}
\item[\textbullet] en vert~: le candidat connaît les deux questions~;
\item[\textbullet] en orange~: le candidat connaît une seule des deux questions~;
\item[\textbullet] en magenta~: le candidat ne connaît aucune des deux questions.
\end{itemize}


\begin{center}
\begin{tabular}{|c|c|c|c|c|c|c|}\hline
\backslashbox{Géo}{Hist}&\textcolor{blue}{1}&\textcolor{blue}{2}&\textcolor{blue}{3}&\textcolor{red}{4}&\textcolor{red}{5}&\textcolor{red}{6} \\ \hline
\textcolor{blue}{1}&\cellcolor{green}&\cellcolor{green}&\cellcolor{green}&\cellcolor{orange}&\cellcolor{orange}&\cellcolor{orange}	\\ \hline
\textcolor{blue}{2}&\cellcolor{green}&\cellcolor{green}&\cellcolor{green}&\cellcolor{orange}&\cellcolor{orange}&\cellcolor{orange}\\ \hline
\textcolor{red}{3}&\cellcolor{orange}&\cellcolor{orange}&\cellcolor{orange}&\cellcolor{magenta}&\cellcolor{magenta}&\cellcolor{magenta}\\ \hline
\textcolor{red}{4}&\cellcolor{orange}&\cellcolor{orange}&\cellcolor{orange}&\cellcolor{magenta}&\cellcolor{magenta}&\cellcolor{magenta}\\ \hline
\textcolor{red}{5}&\cellcolor{orange}&\cellcolor{orange}&\cellcolor{orange}&\cellcolor{magenta}&\cellcolor{magenta}&\cellcolor{magenta}\\ \hline
\end{tabular}
\end{center}




Conclusion~: il y a $6\times 5=30$ cases au total, 6 vertes et 15 oranges, donc~:

\begin{itemize}
\item[\textbullet] la probabilité que le candidat connaisse les deux questions est $\frac{6}{30}=\frac{1}{5}~;$
\item[\textbullet] la probabilité que le candidat connaisse au moins l'une des deux questions est $\frac{6+15}{30}=\frac{21}{30}=\frac{7}{10}.$
\end{itemize}

\medskip

\textbf{Remarque~:} On aurait pu se passer du tableau~:

\begin{itemize}
\item[\textbullet] il y a $6\times 5=30$ tirages possibles~;
\item[\textbullet] il y a $3\times 2=6$ cas favorables à l'événement \og le candidat connaît les deux questions \fg~;
\item[\textbullet] il y a $3\times 3=9$ cas favorables à l'événement \og le candidat ne connaît aucune des deux questions \fg, donc $30-9=21$ cas favorables à l'événement contraire \og le candidat connaît au moins l'une des deux questions \fg.
\end{itemize}

\end{exo}


\begin{exo}

\begin{itemize}
\item[\textbullet] Les cas possibles sont les 3-listes d'un ensemble à 4 éléments~; il y en a $4^3=64.$
\item[\textbullet] Les cas favorables à G sont les arrangements de 3 éléments d'un ensemble à 4 éléments ($\heartsuit$, $\diamondsuit$, $\spadesuit$, $\clubsuit$)~; il y en a $\frac{4!}{(4-3)!}=\frac{4!}{1!}=4\times 3\times 2=24.$ On a donc $P(\text{G})=\frac{24}{64}=\frac{3}{8}.$
\item[\textbullet] Pour calculer $P(H),$ on prend l'événement contraire $\overline{\text{H}}$~: \og aucun cœur n'apparaît à l'écran \fg.

Les cas favorables à $\overline{\text{H}}$ sont les 3-listes d'un ensemble à 3 éléments ($\diamondsuit$, $\spadesuit$, $\clubsuit$), il y en a donc $3^3=27.$ Il reste $64-27=37$ cas favorables à H, et ainsi $P(\text{H})=\frac{37}{64}.$ 
\end{itemize}
\end{exo}

\begin{exo}

\begin{enumerate}
\item \begin{itemize}
\item[\textbullet] Les cas possibles sont les 30-listes d'un ensemble à 200 éléments, il y en a $200^{30}.$
\item[\textbullet] Les cas favorables sont les arrangements de 30 éléments d'un ensemble à 200 éléments~; il y en a \[\frac{200!}{(200-30)!}=\frac{200!}{170!}=200\times 199\times 198\times \cdots\times 172\times 171.\]
\item[\textbullet] La probabilité que les élèves choisissent tous un nombre différent est donc
\[p=\frac{200\times 199\times 198\times \cdots\times 172\times 171}{200^{30}}.\]
\end{itemize}
\item On remarque que

\[
p=\frac{200\times 199\times 198\times \cdots\times 172\times 171}{200\times 200\times 200\times\cdots\times 200\times 200}=\frac{200}{200}\times \frac{199}{200}\times \frac{198}{200}\times \frac{172}{200}\times \frac{171}{200}.
\]

C'est sous cette forme, en calculant le produit de proche en proche, que l'on peut obtenir la réponse avec un programme\footnote{Le nombre $200^{30}$ dépasse les capacités de votre calculatrice, faisant du calcul de proche en proche une nécessité.}~: on part de la valeur 1, puis on multiplie par $\frac{171}{200},$ puis par $\frac{172}{200},$ puis par $\frac{173}{200},$ ... jusqu'à $\frac{200}{200}.$

\medskip

\begin{lstlisting}
def proba():
	p=1
	for i in range(171,201):
		p=p*i/200
	return p
\end{lstlisting}


\end{enumerate}

\medskip

La réponse obtenue en sortie est $p\approx 0,10.$

\end{exo}

\begin{exo}

\begin{itemize}
\item[\textbullet] Les podiums possibles sont les 3-listes d'un ensemble à 12 éléments~; il y en a $12^3=\np{1728}.$
\item[\textbullet] Les cas favorables à l'événement A : \og Le joueur obtient le tiercé \fg~{} sont toutes les permutations possibles des 3 premiers de la course~; il y en a $3!=3\times 2\times 1=6.$ On peut d'ailleurs les énumérer rapidement~:
\[(7,4,10)~;~(7,10,4)~;(4,7,10)~;(4,10,7)~;(10,4,7)~;(10,7,4).\]
\item[\textbullet] Conclusion~: $P\left(\text{A}\right)=\frac{6}{\np{1728}}=\frac{1}{288}.$
\end{itemize}

\end{exo}

\begin{exo}

\begin{enumerate}
\item On  commence par $A=\binom{5}{2}.$ Il y a trois méthodes~:

\begin{itemize}
\item[\textbullet] avec la formule~:
\[A=\binom{5}{2}=\frac{5!}{2!\times 3!}=\frac{5\times 4\times\cancel{3}\times \cancel{2}\times \cancel{1}}{2\times 1\times \cancel{3}\times \cancel{2}\times \cancel{1}}=\frac{20}{2}=10.\]
\item[\textbullet] avec le triangle de Pascal~:

\begin{center}
\begin{tabular}{|l|c c c c c c|}
\hline
         & 0        & 1        & \red{$\boxed{2}$} &3 &4    & 5   \\
\hline
$0$ &  1 &0&0&0&0  &0      \\
$1$ &   1&1&0&0&0  &  0   \\
$2$ &   1&2&1&0&0     &0  \\
$3$ &  1&3&3&1&0       &0 \\
$4$ &    1&4&6&4&1&0      \\
\red{$\boxed{5}$} &    1&5&\red{$\boxed{10}$}&10&5&1      \\
\hline
\end{tabular}
\end{center}
\item[\textbullet] avec la calculatrice~:

\small

\setlength{\columnseprule}{1pt}
\begin{multicols}{4}

\begin{center}\textbf{Calculatrices collège}\end{center}

\medskip


Il faut écrire le calcul (le symbole ! est sur le clavier)~:

\[\frac{5!}{2!\times 3!}\]
\vspace*{1cm}
\columnbreak

\begin{center}\textbf{NUMWORKS}\end{center}

\medskip

\begin{itemize}
\item[\textbullet] \fbox{\textcolor{yellow}{\faHome}}
\item[\textbullet] Calculs \fbox{EXE} puis \fbox{\textcolor{black}{\faToolbox }} (boîte à outils)
\item[\textbullet] choisir Dénombrement \fbox{EXE}
\item[\textbullet] choisir  binomial(n,k) \fbox{EXE}
\item[\textbullet] compléter $\binom{5}{2}$ \fbox{EXE}
\end{itemize}

\columnbreak

\begin{center}\textbf{TI graphiques}\end{center}

\medskip


\begin{itemize}
\item[\textbullet] \fbox{math} puis \fbox{PROB} 
\item[\textbullet] 3:Combinaison

\fbox{EXE}
\item[\textbullet] $~_{5}\text{C}_2$ \fbox{EXE}
\end{itemize}
\vspace*{1cm}

\columnbreak

\begin{center}\textbf{CASIO graphiques}\end{center}

\medskip


\begin{itemize}
\item[\textbullet] \fbox{MENU} puis \fbox{RUN} \fbox{EXE}
\item[\textbullet] 5 \fbox{OPTN}  \fbox{$\triangleright$}
\item[\textbullet] \fbox{F3} (on choisit donc PROB)
\item[\textbullet] \fbox{F3} (on choisit donc nCr)
\item[\textbullet] 2 \fbox{EXE} (on affiche 5\textbf{C}2 à l'écran avant d'exécuter)
\end{itemize}






\end{multicols}
\end{itemize}

\normalsize

\medskip

Quelle que soit la méthode, on obtient \[A=\binom{5}{2}=10.\]


\medskip

On obtient également~:

\[B=\binom{6}{3}=20,\quad C=\binom{50}{1}=50,\quad D=\binom{4}{0}=1.\]

\medskip

\textbf{Remarque~:} Pour $C$ et $D,$ le résultat s'obtient sans calcul~:

\begin{itemize}
\item[\textbullet] pour $C,$ on choisit 1 élément parmi 50, donc il y a 50 choix possibles~;
\item[\textbullet] pour $D,$ on on sait (cf cours) que $\binom{n}{0}=1$ quelle que soit la valeur de $n.$
\end{itemize}

\item \begin{itemize}
\item[\textbullet] $\binom{10}{3}=\binom{10}{7}=120.$ L'égalité était prévisible~: choisir 3 éléments que l'on conserve dans un ensemble à 10 éléments revient à choisir les 7 éléments que l'on met de côté.
\item[\textbullet] Par le même raisonnement, vu que $100-60=40,$ $\binom{100}{60}=\binom{100}{40}.$ Et d'une manière plus générale, si $0\leq k\leq n~:$
\[\binom{n}{k}=\binom{n}{n-k}.\]
\end{itemize}

\end{enumerate}

\end{exo}



\begin{exo}

On choisit trois numéros sur une grille de neuf cases. Il y a $\binom{9}{3}=84$ grilles possibles.

\medskip



\begin{center}
 \begin{tabular}{|c|c|c|}\hline
1& 2&$\textcolor{red}{\xcancel{\black{3}}}$ \\ \hline
4&$\textcolor{red}{\xcancel{\black{5}}}$&6\\ \hline
7&$\textcolor{red}{\xcancel{\black{8}}}$&9\\ \hline
\end{tabular}
\end{center}

\end{exo}



\begin{exo}

On prend 5 cartes dans un jeu de 32. Il y a $\binom{32}{5}=\np{201376}$ mains possibles.

\end{exo}

\begin{exo}

Lorsqu'ils se rencontrent en arrivant le matin au lycée, les 24 élèves d'une classe se serrent la main.

\medskip

Choisir une poignée de main, c'est choisir deux personnes dans la classe. Il s'échange donc  $\binom{24}{2}=\np{276}$ poignées de mains au total.

\end{exo}


\begin{exo}


\begin{enumerate}
\item Si $p\geq 2~:$ \[(p-1)!\times \textcolor{red}{p}=1\times 2\times\cdots\times (p-1)\times \textcolor{red}{p}=p!\]
(l'égalité est également vraie lorsque $p=1,$ puisque $0!\times 1=1=1!$).
\item Si $n\geq 2~:$ 
\[\binom{n+1}{2}=\frac{(n+1)!}{2!\times(n+1-2)!}=\frac{(n+1)!}{2!\times(n-1)!}=
\frac{\cancel{(n-1)!}\times n\times (n+1)}{2\times\cancel{(n-1)!}}=\frac{n^2+n}{2}.\]

\medskip

\textbf{Remarque~:} On peut aussi obtenir la réponse par un raisonnement de dénombrement~: choisir 2 éléments parmi $n+1$ revient à compter le nombre de poignées de mains lorsque $n+1$ personnes se serrent la main les unes les autres (comme dans l'exercice précédent). Chacune des $n+1$ personnes donne $n$ poignées de main~; et on divise par 2, parce que sinon chaque poignée de main est comptée deux fois. On total, on en dénombre $(n+1)\times n\div 2=\frac{n^2+n}{2}.$


\item Soit $n\geq k\geq 1.$ On calcule séparément~:

 \setlength{\columnseprule}{1pt}
\begin{multicols}{2}

\begin{align*}
n\times \binom{n-1}{k-1}
&=n\times \frac{(n-1)!}{(k-1)!\times((n-1)-(k-1))!}
\\&=\frac{(n-1)!\times n}{(k-1)!\times(n-\cancel{1}-k+\cancel{1})!}
\\&=\frac{n!}{(k-1)!\times(n-k)!}
\end{align*}

\begin{align*}
k\times \binom{n}{k}
&=k\times \frac{n!}{k!\times(n-k)!}
\\&=\frac{n!\times \cancel{k}}{(k-1)!\times \cancel{k}\times(n-k)!}
\\&=\frac{n!}{(k-1)!\times(n-k)!}
\end{align*}
\end{multicols}

\medskip

On a donc bien
\[n\times \binom{n-1}{k-1}=k\times \binom{n}{k}\] (formule du pion).
\end{enumerate}


\end{exo}

\begin{exo}


\begin{enumerate}
\item Il faut choisir 12 personnes parmi 20, donc on peut constituer $\binom{20}{12}$ groupes différents.
\item David est un des membres de l'association. Il y a~:
\begin{itemize}
\item[\textbullet] $\binom{19}{11}$ groupes de 12 personnes contenant David (puisque si David est pris, il reste 11 personnes à choisir parmi les 19 autres)~;
\item[\textbullet] $\binom{19}{12}$ groupes de 12 personnes ne contenant pas David (puisque si David n'est pas pris, il reste encore 12 personnes à choisir parmi les 19 autres).
\end{itemize}
\item D'après les questions 1 et 2~:
\[\binom{20}{12}=\binom{19}{11}+\binom{19}{12}.\]
\item On généralise~: si $1\leq k\leq n-1,$
\[\binom{n}{k}= \binom{n-1}{k-1} + \binom{n-1}{k}.\]
\item On redémontre par le calcul la formule obtenue à la question précédente~:

\begin{align*}
\binom{n-1}{k-1} + \binom{n-1}{k}
&=\frac{(n-1)!}{(k-1)!\times ((n-1)-(k-1))!}+\frac{(n-1)!}{k!\times ((n-1)-k)!}
\\&=\frac{(n-1)!}{(k-1)!\times (n-\cancel{1}-k+\cancel{1})!}+\frac{(n-1)!}{k!\times (n-1-k)!}
\\&=\frac{(n-1)!\textcolor{red}{\times k}}{(k-1)!\textcolor{red}{\times k}\times (n-k)!}+\frac{(n-1)!\textcolor{blue}{\times (n-k)}}{k!\times (n-k-1)!\textcolor{blue}{\times (n-k)}}
\\&=\frac{(n-1)!\times k}{k!\times (n-k)!}+\frac{(n-1)!\times(n-k)}{k!\times (n-k)!}
\\&=\frac{(n-1)!\times(k+n-k)}{k!\times (n-k)!}
\\&=\frac{(n-1)!\times n}{k!\times (n-k)!}
\\&=\frac{n!}{k!\times (n-k)!}
\\&=\binom{n}{k}
\end{align*}

\end{enumerate}

\medskip

\textbf{Remarque~:} Cette égalité, appelée formule de Pascal, permet de faire le lien entre le triangle de Pascal et les $\binom{n}{k}.$ En effet~:

\begin{itemize}
\item[\textbullet] La première colonne  du triangle de Pascal contient uniquement des 1, qui correspondent bien aux $\binom{n}{0}.$
\item[\textbullet] La diagonale  du triangle de Pascal contient uniquement des 1, qui correspondent bien aux $\binom{n}{n}.$
\item[\textbullet] On remplit chacune des cases \og centrales \fg~{} en ajoutant le nombre au-dessus et celui au-dessus à gauche. Si les nombres de la ligne $n-1$ correspondent aux $\binom{n-1}{k},$ alors ceux de la ligne du dessous correspondront également, via la formule de Pascal~:

\LARGE
\[\boxed{\begin{matrix}
\binom{n-1}{k-1}&\textcolor{blue}{+}&\binom{n-1}{k}\\
& &\textcolor{blue}{=}\\
& & \binom{n}{k}
\end{matrix}}\]
\normalsize

C'est une sorte de raisonnement par récurrence~: la correspondance entre les termes du triangle de Pascal et les $\binom{n}{k}$ \og se propage de ligne en ligne \fg.
\end{itemize}

\end{exo}



\begin{exo}

Le sélectionneur choisit 10 joueurs de champ parmi 17, puis 1 gardien parmi 3~; il a donc
\[\binom{17}{3}\times \binom{3}{1}=680\times 3=\np{2040}~\text{équipes possibles.}\]

\end{exo}

\begin{exo}

Il faut choisir 2 moniteurs parmi 5, puis 10 enfants parmi 40. Il y a donc

\[\binom{5}{2}\times \binom{40}{10}~\text{groupes possibles}\]

(la valeur explicite est entre 8 et 9 milliards).


\end{exo}

\begin{exo}


On écrit les formules, mais on ne fait pas les calculs explicites (sans grand intérêt mathématique à ce stade du cours).

\begin{enumerate}
\item Il y a $\binom{32}{5}$ mains possibles.

\item La probabilité qu'une main contienne~:
\begin{itemize}
\item[\textbullet] exactement 3 dames est 
\[\frac{\binom{4}{3}\times \binom{28}{2}}{\binom{32}{5}}\] (on choisit 3 dames parmi 4, puis 2 cartes parmi les 28 autres)~;
\item[\textbullet] trois cœurs et deux carreaux est
\[\frac{\binom{8}{3}\times \binom{8}{2}}{\binom{32}{5}}\] (on choisit 3 cœurs parmi 8, puis 2 carreaux parmi 8)\footnote{On pourrait multiplier par $\binom{16}{0}$ au numérateur, puisqu'on ne choisit aucune carte parmi les piques et les trèfles. Bien sûr, cela ne changerait rien à la réponse, puisque $\binom{16}{0}=1.$}~;
\item[\textbullet] exactement un roi et deux valets est
\[\frac{\binom{4}{1}\times \binom{4}{2}\times\binom{24}{2}}{\binom{32}{5}}\] (on choisit 1 roi parmi 4, puis 2 valets parmi 4~; et enfin 2 cartes parmi les 24 autres).
\end{itemize}
\end{enumerate}
\end{exo}

\begin{exo}


\begin{enumerate}
\item 

\begin{enumerate}
\item Le recrutement de 3 candidats peut se faire de $\binom{8}{3}=56$ façons possibles.
\item Le recrutement de 7 candidats peut se faire de $\binom{8}{7}=8$ façons possibles.
\item On peut recruter entre 0 et 8 candidats, donc en raisonnant comme dans les questions 1 et 2, on voit qu'il y a
\[\binom{8}{0}+\binom{8}{1}+\binom{8}{2}+\binom{8}{3}+\binom{8}{4}+\binom{8}{5}+\binom{8}{6}+\binom{8}{7}+\binom{8}{8}\] recrutements différents possibles.
\item Chaque candidat est soit accepté (A), soir refusé (R). On peut donc assimiler le recrutement à une liste de 8 éléments à choisir parmi A/R. Ainsi y a-t-il \[2^8=256\] recrutements différents possibles (nombre de 2-listes d'un ensemble à 8 éléments)\footnote{Le lecteur qui n'est pas convaincu peut faire un arbre.}.
\end{enumerate}
\item Soit $n\geq 1.$ On imagine $n$ candidats au lieu 8 et on raisonne comme dans la question 1~: le nombre de recrutements différents possibles est
\[\binom{n}{0}+\binom{n}{1}+\binom{n}{2}+\cdots +\binom{n}{n}=2^n.\]  

\medskip

Avec le symbole $\sum,$ cette formule (fort connue) se réécrit \[\sum\limits_{k=0}^n\binom{n}{k}=2^n.\]

(On vérifie sans peine qu'elle est également vraie lorsque $n=0.$)

\end{enumerate}

\end{exo}




\begin{exo}~{}


\begin{center}
\psset{xunit=1.0cm,yunit=1.0cm,algebraic=true,dimen=middle,dotstyle=o,dotsize=5pt 0,linewidth=2.pt,arrowsize=3pt 2,arrowinset=0.25}
\begin{pspicture*}(-2.52,-0.34)(7.52,4.56)
\psline[linewidth=2.pt](-2.,4.)(-2.,1.)
\psline[linewidth=2.pt](-2.,1.)(1.,1.)
\psline[linewidth=2.pt](1.,1.)(1.,4.)
\psline[linewidth=2.pt](1.,4.)(-2.,4.)
\psline[linewidth=2.pt](3.,4.)(3.,1.)
\psline[linewidth=2.pt](3.,1.)(6.,1.)
\psline[linewidth=2.pt](6.,1.)(6.,4.)
\psline[linewidth=2.pt](6.,4.)(3.,4.)
\begin{LARGE}
\rput[tl](-1.56,3.48){\ding{172}}
\rput[tl](-1.4,1.78){\ding{173}}
\rput[tl](-0.56,2.78){\ding{174}}
\rput[tl](-0.24,1.94){\ding{185}}
\rput[tl](0.04,3.42){\ding{186}}
\rput[tl](4.3,3.44){\ding{177}}
\rput[tl](5.18,2.02){\ding{178}}
\rput[tl](3.62,2.84){\ding{189}}
\rput[tl](3.78,1.6){\ding{190}}
\rput[tl](4.92,2.8){\ding{191}}
\end{LARGE}
\rput[tl](-0.55,0.45){$U_1$}
\rput[tl](4.5,0.45){$U_2$}
\end{pspicture*}
\end{center}

\begin{itemize}
\item[\textbullet] Il y a \[\binom{5}{2}\times \binom{5}{2}=10\times 10=100\] tirages possibles (puisqu'on choisit 2 des 5 boules de l'urne  $U_1,$ 2 des 5 boules de l'urne  $U_2$).
\item[\textbullet] Pour avoir 2 boules blanches il faut, au choix~:
\begin{itemize}
\item Tirer 2 blanches et 0 noire dans l'urne $U_1,$ 0 blanche et 2 noires dans l'urne $U_2.$ Le nombre de façons différentes de faire ce tirage est \[\binom{3}{2}\times \binom{2}{0}\times \binom{2}{0}\times \binom{3}{2}=3\times 1\times 1\times 3=9.\]
\item Tirer 1 blanche et 1 noire dans l'urne $U_1,$ 1 blanche et 1 noire dans l'urne $U_2.$ Le nombre de façons différentes de faire ce tirage est \[\binom{3}{1}\times \binom{2}{1}\times \binom{2}{1}\times \binom{3}{1}=3\times 2\times 2\times 3=36.\]
\item Tirer 0 blanche et 2 noires dans l'urne $U_1,$ 2 blanches et 0 noire dans l'urne $U_2.$ Le nombre de façons différentes de faire ce tirage est \[\binom{3}{0}\times \binom{2}{2}\times \binom{2}{2}\times \binom{3}{0}=1\times 1\times 1\times 1=1.\]
\end{itemize}
\item[\textbullet] Conclusion~: $P(A)=\frac{9+36+1}{100}=0,46.$
\end{itemize}



\end{exo}

\section{Limites de suites}


\begin{exo}



\begin{enumerate}
\item $u_n=3+\dfrac{1}{n}.$

\[
\left.
    \begin{array}{ll}
        \lim\limits_{n\to +\infty}3&= 3 \\
        \lim\limits_{n\to +\infty}\dfrac{1}{n}&= 0
    \end{array}
\right \}\implies \lim\limits_{n\to +\infty}\left(3+\dfrac{1}{n}\right)=3+0=3.
\]

On s'autorise à aller un peu plus vite~: on écrit simplement
\[\lim\limits_{n\to +\infty}u_n=3+0=3.\] 
\item On écrit $v_n=4-\dfrac{1}{n^2}=4-\dfrac{1}{n}\times \dfrac{1}{n}.$ On a donc
\[\lim\limits_{n\to +\infty}v_n=4-0\times 0=4.\] 

\item On écrit $w_n=\left(5+\dfrac{3}{n}\right)\left(2+\dfrac{1}{n}\right)=\left(5+3\times \dfrac{1}{n}\right)\left(2+\dfrac{1}{n}\right).$ On a donc
\[\lim\limits_{n\to +\infty}w_n=\left(5+3\times 0\right)\left(2+0\right)=10.\] 


\item On écrit $x_n=\dfrac{1}{1+\dfrac{2}{n}}=\dfrac{1}{1+2\times \dfrac{1}{n}}.$

On a donc
\[\lim\limits_{n\to +\infty}x_n=\dfrac{1}{1+2\times 0}=1.\] 
\item On met $n$ en facteur au numérateur et au dénominateur~:
\[
y_n=\dfrac{3n-5}{4n+1}
=\dfrac{\cancel{n}\left(3-\frac{5}{n}\right)}{\cancel{n}\left(4+\frac{1}{n}\right)}
=\dfrac{3-5\times \frac{1}{n}}{4+\frac{1}{n}}.
\]

On a donc
\[\lim\limits_{n\to +\infty}y_n=\dfrac{3-5\times 0}{4+0}=\dfrac{3}{4}.\]


\end{enumerate}

\medskip

\textbf{Bilan~:} Pour calculer les limites, il suffit de faire apparaître des $\frac{1}{n}$ et de les remplacer par 0 lorsqu'on \og passe à la limite \fg.

\end{exo}



\begin{exo}

La suite $(u_n)_{n\in\mathbb{N}}$ est définie par $u_0=6$ et la relation de récurrence \[u_{n+1}=0,6u_n-4\] pour tout $n\in\mathbb{N}.$ On admet qu'elle converge et on note $\ell$ sa limite.

\medskip

Les suites $\left(u_n\right)_{n\in\mathbb{N}}=\left(u_0,u_1,u_2,\cdots\right)$ et $\left(u_{n+1}\right)_{n\in\mathbb{N}}=\left(u_1,u_2,u_3,\cdots\right)$ ont la même limite   puisque les indices sont simplement décalés~:

\begin{align*}
\lim\limits_{n\to +\infty}u_n&=\ell,\\
\lim\limits_{n\to +\infty}u_{n+1}&=\ell.
\end{align*}

 Par opération sur les limites, on peut \og passer à la limite \fg~{} dans la relation de récurrence~:
\[u_{n+1}=0,6u_n-4\qquad\text{pour tout }n\in\mathbb{N},\] donc
\[\ell=0,6\ell-4.\]

On résout cette équation~:
\[\ell=0,6\ell-4\iff \ell-0,6\ell=-4\iff 0,4\ell=-4\iff\ell=\frac{-4}{0,4}\iff \ell=-10.\]

Conclusion~: $\lim\limits_{n\to +\infty}u_n=-10.$
\end{exo}


\begin{exo}

La suite $\left(v_n\right)_{n\in\mathbb{N}}$ est géométrique de premier terme $v_0=20$ et de raison $q=-0,5.$

\begin{enumerate}
\item $v_0=20~;~v_1=20\times (-0,5)=-10~;~v_2=-10\times (-0,5)=5~;~v_3=5\times (-0,5)=-2,5~;~v_4=-2,5\times (-0,5)=1,25.$
\item On admet que $\left(v_n\right)_{n\in\mathbb{N}}$ converge, on note $\ell$ sa limite.

\medskip
On \og passe à la limite \fg~{} dans la relation de récurrence~: $\left(v_n\right)_{n\in\mathbb{N}}$ est géométrique de raison $q=-0,5,$ donc
\[v_{n+1}=-0,5\times v_n\qquad\text{pour tout }n\in\mathbb{N}~;\] et donc
\[\ell=-0,5\times\ell.\]

On résout~:
\[\ell=-0,5\times\ell\iff\ell+0,5\ell=0\iff 1,5\ell=0\iff \ell=\frac{0}{1,5}\iff\ell=0.\]

Conclusion~: $\lim\limits_{n\to +\infty}v_n=0.$

\end{enumerate}

\end{exo}

\begin{exo}


\begin{enumerate}
\item Il est clair que $0\leq \frac{1}{n+\sqrt{n}}$ pour tout entier $n\geq 1.$ Pour l'autre inégalité, on part de \[n+\sqrt{n}\geq n.\] Deux nombres strictement positifs sont rangés en sens contraire de leurs inverses, donc
\[\frac{1}{n+\sqrt{n}}\leq \frac{1}{n}\] (\danger en prenant l'inverse, le sens de l'inégalité est renversé).

\item $0\leq \frac{1}{n+\sqrt{n}}\leq \frac{1}{n}$ pour tout entier $n\geq 1$ et
\[\lim\limits_{n\to +\infty}0=0\qquad , \qquad \lim\limits_{n\to +\infty}\frac{1}{n}=0.\] Donc d'après le théorème des gendarmes~:
\[\lim\limits_{n\to +\infty}\dfrac{1}{n+\sqrt{n}}=0.\]
\end{enumerate}

\end{exo}

\begin{exo}

La suite $(u_n)_{n\in\mathbb{N}}$ est définie par $u_0=1$ et la relation de récurrence

\[u_{n+1}=0,6u_n+0,4n+1\] pour tout $n\in\mathbb{N}.$

\begin{enumerate}
\item \begin{align*}
u_1&=0,6u_0+0,4\times 0+1=0,6\times 1+0+1=1,6\\
u_2&=0,6u_1+0,4\times 1+1=0,6\times 1,6+0,4+1=2,36
\end{align*}
\item Pour tout $n\in\mathbb{N},$ on note $\mathcal{P}_n$ la propriété \[n\leq u_n\leq n+1.\]





\begin{itemize}
\item[{\textbullet}] \textbf{Initialisation.} On prouve que $\mathcal{P}_0$ est vraie.

\[
\left.
    \begin{array}{ll}
        u_0&=1 \\
        0&\leq u_0\leq 0+1
    \end{array}
\right \}\implies \mathcal{P}_0~\text{est vraie}.
\]



\item[{\textbullet}] \textbf{Hérédité.} Soit $k\in\mathbb{N}$ tel que $\mathcal{P}_k$ soit vraie. On a donc
\[k\leq u_k\leq k+1.\]
%\medskip

\newtcolorbox{mybox}[1]{colback=green!10!white,colframe=green!80!white,fonttitle=\bfseries,title=#1}
\begin{mybox}{Objectif}{Prouver que $\mathcal{P}_{k+1}$ est vraie, c'est-à-dire que \[k+1\leq u_{k+1}\leq k+2.\]
}\end{mybox}



%\medskip

On part de 
\[k\leq u_k\leq k+1.\]


On multiplie par $\textcolor{red}{0,6}~:$

\begin{align*}k\textcolor{red}{\times 0,6}&\leq u_k\textcolor{red}{\times 0,6}\leq (k+1)\textcolor{red}{\times 0,6}\\
0,6k&\leq 0,6u_k\leq 0,6k+0,6\end{align*}

Puis on ajoute  $\textcolor{blue}{0,4k+1}~:$

\begin{align*}
0,6k\textcolor{blue}{+0,4k+1}&\leq 0,6u_k\textcolor{blue}{+0,4k+1}\leq 0,6k+0,6\textcolor{blue}{+0,4k+1}\\
k+1&\leq 0,6u_k+0,4k+1\leq k+1,6\\
k+1&\leq u_{k+1}\leq k+1,6
\end{align*}

Or $k+1,6\leq k+2,$ donc la propriété $\mathcal{P}_{k+1}$ est vraie.
\item[{\textbullet}] \textbf{Conclusion.} $\mathcal{P}_0$ est vraie et $\mathcal{P}_n$ est héréditaire, donc elle est vraie pour tout $n\in\mathbb{N}.$
\end{itemize}
\item Soit $n\geq 1.$ On reprend l'inégalité de la question précédente et on divise par $n~:$

\begin{align*}
n&\leq u_n\leq n+1\\
\frac{n}{\textcolor{red}{n}}&\leq \frac{u_n}{\textcolor{red}{n}}\leq \frac{n+1}{\textcolor{red}{n}}\\
1&\leq \frac{u_n}{n}\leq 1+\frac{1}{n}
\end{align*}

\medskip

On conclut~:

\[\lim\limits_{n\to +\infty}1=1\qquad , \qquad \lim\limits_{n\to +\infty}\left(1+\frac{1}{n}\right)=1+0=1,\] donc d'après le théorème des gendarmes
\[\lim\limits_{n\to +\infty}\frac{u_n}{n}=1.\]

\end{enumerate}

\end{exo}


\begin{exo}

On veut prouver que $\lim\limits_{n\to +\infty}\frac{n-3}{4}=+\infty.$ On se donne pour cela un réel $M> 0$ et on écrit les équivalences~:

\[\frac{n-3}{4}\geq M \iff n-3\geq 4M\iff n\geq 4M+3.\]

Conclusion~: quand $n$ dépasse $ 4M+3,$ $\frac{n-3}{4}$ dépasse $M.$ On a donc bien $\lim\limits_{n\to +\infty}\frac{n-3}{4}=+\infty.$



\end{exo}

\begin{exo}

Soit $M>0.$ On sait que $\lim\limits_{n\to +\infty}u_n=+\infty,$ donc $u_n\geq M$ à partir d'un certain rang $N.$ On a donc $v_n\geq u_n\geq M$ à partir du rang $N.$ On en déduit $\lim\limits_{n\to +\infty}v_n=+\infty.$



\end{exo}






\end{document}
