\documentclass[10pt]{article}
\usepackage[T1]{fontenc}
\usepackage[utf8]{inputenc}
\usepackage{fourier}
\usepackage[scaled=0.875]{helvet}
\renewcommand{\ttdefault}{lmtt}
\usepackage{amsmath,amssymb,makeidx}
\usepackage[normalem]{ulem}
\usepackage{fancybox}
\usepackage{cancel}
\usepackage{stmaryrd}
\usepackage{ulem}
\usepackage{tabularx}
\usepackage{geometry}
\usepackage{enumerate}
\geometry{hmargin=1.5cm,vmargin=1.5cm}
\usepackage{dcolumn}
\usepackage{textcomp}
\usepackage{lscape}
\usepackage{eurosym}
%\newcommand{\euro}{\eurologo{}}
\usepackage[dvips]{color}
\usepackage[all]{xy}
\usepackage{xlop}

\usepackage{tikz,tkz-tab}

\usepackage{systeme}


\usepackage{pstricks,pst-plot,pst-text,pst-tree,pstricks-add}
\usepackage{colortbl}
\usepackage{diagbox}
\usepackage{fontawesome5}
\usepackage{pifont}
\usepackage{wasysym}


\usepackage{theorem}
\theorembodyfont{\upshape}
\newtheorem{exo}{Exercice}
%\newtheorem{exo}{Exercice}%[section]
\usepackage{hyperref}
\hypersetup{
    colorlinks=true,       % false: liens encadrés; true: liens colorés
    linkcolor=blue,          % couleur des liens (ou bordures) internes
}

%\setlength{\voffset}{-1,5cm}
\usepackage{fancyhdr} 
\usepackage{graphicx}
\usepackage[frenchb]{babel}
\usepackage[np]{numprint}
\usepackage{multicol}
\usepackage{xlop}
\usepackage{soul}


\title{Mathématiques -- Seconde}

\date{Corrigés des exercices}
\begin{document}
\setlength\parindent{0mm}
\renewcommand \footrulewidth{.2pt}

\maketitle

\tableofcontents


\newpage


\section{Rappels de calcul et de géométrie}

\begin{exo}

Dans chaque question, on obtient la réponse à l'aide d'un tableau de proportionnalité.

\begin{enumerate}
\item ~{}
\begin{center}
\begin{tabular}{|c|c|c|}\hline
Nombre de personnes& 4&6 \\ \hline 
Farine (en g)&250& ? \\ \hline
Lait (en mL)&500& ? \\ \hline
Œufs&4& 6 \\ \hline
\end{tabular}
\end{center}

Pour 6 personnes, il faut $\frac{250\times 6}{4}=\frac{\np{1500}}{4}=375$~g de farine, $\frac{500\times 6}{4}=\frac{\np{3000}}{4}=750$~mL de lait et, bien sûr, 6 œufs.

\item Les 6 yaourts pèsent $6\times 125=750$~g.

\begin{center}
\begin{tabular}{|c|c|c|}\hline
masse (en g)& 1000&750 \\ \hline 
prix (en \euro)&2& ? \\ \hline
\end{tabular}
\end{center}

Je payerai $\frac{750\times 2}{\np{1000}}=\frac{\np{1500}}{\np{1000}}=1,5~\text{\euro}.$

\item Généralement, dans ce type de question, il vaut mieux convertir en minutes\footnote{Les calculs ne sont pas toujours plus faciles en minutes qu'en heures, mais c'est généralement le cas.}.

\begin{center}
\begin{tabular}{|c|c|c|}\hline
temps (en min)& 60&? \\ \hline 
distance (en km)&20& 45 \\ \hline
\end{tabular}
\end{center}

On mettra $\frac{60\times 45}{20}=\frac{\cancel{20}\times 3\times 45}{\cancel{20}}=135$~min, soit 2~h~15~min (puisque $135=120+15$).
\item L'énoncé donne les informations recensées dans le tableau ci-dessous et demande de compléter la case \textcircled{\small{1}}.

\begin{center}
\begin{tabular}{|c|c|c|c|}\hline
Florins& 7&?&\textcircled{\small{1}} \\ \hline 
Pistoles&6& \textcolor{red}{4}&\textcircled{\small{\textcolor{black}{2}}} \\ \hline
Deniers&?& \textcolor{red}{5}&\textcolor{red}{30} \\ \hline
\end{tabular}
\end{center}

On complète d'abord la case \textcircled{\small{2}}~: en échange de 30 deniers, on a $4\times 30\div 5=24$~pistoles~:

\begin{center}
\begin{tabular}{|c|c|c|c|}\hline
Florins& \textcolor{red}{7}&?&\textcircled{\small{\textcolor{black}{1}}} \\ \hline 
Pistoles&\textcolor{red}{6}& 4&\textcolor{red}{24} \\ \hline
Deniers&?& 5&30 \\ \hline
\end{tabular}
\end{center}

On peut alors compléter la case \textcircled{\small{1}}~: en échange de 30 deniers, on a $\frac{7\times 24}{6}=\frac{7\times 4\times \cancel{6}}{\cancel{6}}=28$~florins.

\end{enumerate}
\end{exo}

\begin{exo}

\begin{enumerate}
\item On complète deux tableaux de proportionnalité (on travaille en min et en km)~:

\begin{multicols}{2}

\begin{center}
\begin{tabular}{|c|c|c|}\hline
temps (en min)& 60&? \\ \hline 
distance (en km)&3& 0,5 \\ \hline
\end{tabular}


\begin{tabular}{|c|c|c|}\hline
temps (en min)& 60&? \\ \hline 
distance (en km)&15& 5 \\ \hline
\end{tabular}
\end{center}

\end{multicols}

Stéphane nage $\frac{60\times 0,5}{3}=\frac{30}{3}=10$~min, puis il court $\frac{60\times 5}{15}=\frac{300}{15}=20$~min.


\item Stéphane a parcouru un total de $5+0,5=5,5$~km, en $10+20=30$~min.

\begin{center}
\begin{tabular}{|c|c|c|}\hline
temps (en min)& 30&60 \\ \hline 
distance (en km)&5,5& ? \\ \hline
\end{tabular}
\end{center}

La vitesse moyenne de Stéphane sur l’ensemble de son parcours est donc $\frac{60\times 5,5}{30}=\frac{\cancel{30}\times 2\times 5,5}{\cancel{30}}=11$~km/h.
\end{enumerate}
\end{exo}

\begin{exo}%28
~{}

\begin{center}
\newrgbcolor{xfqqff}{0.4980392156862745 0. 1.}
\psset{xunit=1.0cm,yunit=1.0cm,algebraic=true,dimen=middle,dotstyle=o,dotsize=3pt 0,linewidth=0.8pt,arrowsize=3pt 2,arrowinset=0.25}
\begin{pspicture*}(1.64,0.96)(7.78,5)
\pspolygon(2,2)(7,2)(7,4)(4,4)
\pspolygon(7,2)(6.6,2)(6.6,2.4)(7,2.4)
\pspolygon(7,4)(7,3.6)(6.6,3.6)(6.6,4)
\psline(2,2)(7,2)
\psline(7,2)(7,4)
\psline(7,4)(4,4)
\psline(4,4)(2,2)
\psline(4,4)(2,2)
\psline(4,4)(4,2)
\pspolygon[linewidth=1.pt,linecolor=xfqqff,fillcolor=xfqqff!20!white,fillstyle=solid,opacity=0.1](4.,2.4242640687119286)(3.5757359312880714,2.4242640687119286)(3.5757359312880714,2.)(4.,2.)
\psline{->}(5,3)(5,4)
\psline{->}(5,3)(5,2)
\rput[tl](5.22,3.2){$2$}
\rput[tl](5.44,4.4){$3$}
\rput[tl](5.44,1.8){$3$}
\rput[tl](2.84,1.8){$2$}
\rput[bl](1.8,2.18){$A$}
\rput[bl](7.08,2.12){$B$}
\rput[bl](7.08,4.12){$C$}
\rput[bl](4.08,4.12){$D$}
\rput[bl](3.8,1.6){$H$}
\end{pspicture*}
\end{center}

Le trapèze est constitué~:

\begin{itemize}
\item[\textbullet] d'un rectangle $BHDC,$ d'aire $\ell\times L=3\times 2=6~;$
\item[\textbullet] d'un triangle $AHD,$ d'aire $\frac{B\times h}{2}=\frac{2\times 2}{2}=2.$
\end{itemize}
Donc l'aire du trapèze est $6+2=8.$

\medskip

\textbf{Remarque~:} On peut aussi utiliser la formule (hors-programme)~:
\[\mathcal{A}_{\text{trapèze}}=\frac{(B+b)\times h}{2}=\frac{(5+3)\times 2}{2}=8.\]
\end{exo}

\begin{exo}

Le losange est \og la moitié \fg~{} d'un rectangle de côtés $\ell$ et $L,$ donc son aire est $\frac{\ell\times L}{2}.$



\begin{center}
\psset{xunit=1.0cm,yunit=1.0cm,algebraic=true,dimen=middle,dotstyle=o,dotsize=5pt 0,linewidth=1.6pt,arrowsize=3pt 2,arrowinset=0.25}
\begin{pspicture*}(0.0,2.0)(5.24,5.24)
\psline[linewidth=2.pt](1.,4.)(3.,5.)
\psline[linewidth=2.pt](3.,5.)(5.,4.)
\psline[linewidth=2.pt](5.,4.)(3.,3.)
\psline[linewidth=2.pt](3.,3.)(1.,4.)
\psline[linewidth=2.pt,linestyle=dotted](1.,4.)(5.,4.)
\psline[linewidth=2.pt,linestyle=dotted](3.,5.)(3.,3.)
\psline[linewidth=2.pt,linestyle=dotted,linecolor=red](1.,3.)(1.,5.)
\psline[linewidth=2.pt,linestyle=dotted,linecolor=red](1.,3.)(5.,3.)
\psline[linewidth=2.pt,linestyle=dotted,linecolor=red](1.,5.)(5.,5.)
\psline[linewidth=2.pt,linestyle=dotted,linecolor=red](5.,5.)(5.,3.)

\rput[tl](0.65,4.2){\red{$\ell$}}
\rput[tl](2.75,2.86){\red{$L$}}
\end{pspicture*}
\end{center}
\end{exo}



\begin{exo}%29

\textbf{Rappels~:}
\begin{itemize}
\item[\textbullet] une hauteur est une droite qui passe par un sommet et qui est perpendiculaire au côté opposé (les hauteurs sont tracées en pointillés bleus)~;
\item[\textbullet] le fait que les hauteurs soient \og concourantes \fg~{} signifie qu'elles passent toutes les trois par un même point -- qu'on appelle \og orthocentre du triangle \fg~{} (nommé $O$ sur la figure ci-dessous).
\end{itemize}


\begin{center}
\newrgbcolor{xfqqff}{0.4980392156862745 0. 1.}
\psset{xunit=1.0cm,yunit=1.0cm,algebraic=true,dimen=middle,dotstyle=o,dotsize=5pt 0,linewidth=2.pt,arrowsize=3pt 2,arrowinset=0.25}
\begin{pspicture*}(0.58,0.54)(7.34,5.4)
\pspolygon[linewidth=2.pt,linecolor=xfqqff,fillcolor=xfqqff!20!white,fillstyle=solid,opacity=0.1](1.9037508537434635,2.9121465431835385)(2.287329637311028,2.730853088263545)(2.4686230922310215,3.1144318718311097)(2.085044308663457,3.295725326751103)
\pspolygon[linewidth=2.pt,linecolor=xfqqff,fillcolor=xfqqff!20!white,fillstyle=solid,opacity=0.1](3.66,3.66)(3.96,3.36)(4.26,3.66)(3.96,3.96)
\pspolygon[linewidth=2.pt,linecolor=xfqqff,fillcolor=xfqqff!20!white,fillstyle=solid,opacity=0.1](3.3378667606356793,1.0079249720699515)(3.3364285856148856,1.4321866032041017)(2.9121669544807354,1.430748428183308)(2.913605129501529,1.0064867970491578)
\psline[linewidth=2.pt](1.,1.)(6.9,1.02)
\psline[linewidth=2.pt](6.9,1.02)(2.9,5.02)
\psline[linewidth=2.pt](2.9,5.02)(1.,1.)
\psline[linewidth=2.pt,linestyle=dotted,linecolor=blue](2.9,5.02)(2.913605129501529,1.0064867970491578)
\psline[linewidth=2.pt,linestyle=dotted,linecolor=blue](1.,1.)(3.96,3.96)
\psline[linewidth=2.pt,linestyle=dotted,linecolor=blue](6.9,1.02)(2.085044308663457,3.295725326751103)
\psdots[dotsize=4pt 0,dotstyle=*,linecolor=blue](2.907162162162163,2.907162162162163)
\rput[bl](3.06,2.42){\blue{$O$}}
\end{pspicture*}
\end{center}

\end{exo}







\begin{exo}%34

On note $H$ le pied de la hauteur issue de $A$ dans le triangle $ABC.$


\begin{center}
\psset{xunit=1.0cm,yunit=1.0cm,algebraic=true,dimen=middle,dotstyle=o,dotsize=5pt 0,linewidth=2.pt,arrowsize=3pt 2,arrowinset=0.25}
\begin{pspicture*}(-0.02,0.32)(5.98,4.56)
\pspolygon[linewidth=2.pt,linecolor=red,fillcolor=red!10!white,fillstyle=solid,opacity=0.1](2.4242640687119286,1.)(2.4242640687119286,1.4242640687119286)(2.,1.4242640687119286)(2.,1.)
\psline[linewidth=2.pt](2.,4.)(5.,1.)
\psline[linewidth=2.pt](2.,4.)(1.,1.)
\psline[linewidth=2.pt](1.,1.)(5.,1.)
\psline[linewidth=2.pt,linestyle=dotted,linecolor=red](2.,4.)(2.,1.)
\psline[linewidth=2.pt](3.,1.)(2.,4.)
\rput[bl](2.08,4.04){$A$}
\rput[bl](5.,0.6){$B$}
\rput[bl](0.7,0.6){$C$}
\rput[bl](1.96,0.6){$H$}
\rput[bl](3.04,0.6){$I$}
\end{pspicture*}
\end{center}


$\left[AH\right]$ est une hauteur dans les triangles $BIA$ et $CIA,$ donc

\[\mathcal{A}_{BIA}=\frac{\textcolor{red}{BI}\times AH}{2}\hspace{4cm} \mathcal{A}_{CIA}=\frac{\textcolor{red}{CI}\times AH}{2}.\]
Or $\textcolor{red}{BI}=\textcolor{red}{CI}$ puisque $I$ et le milieu de $\left[BC\right],$ donc $BIA$ et $CIA$ ont la même aire.

\end{exo}


\newpage


\begin{exo}%30


\begin{enumerate}
\item La négation de

\begin{center}

 \blue{\underline{Tous}} \black les hommes \blue{\underline{sont mortels}}\black .
 
 \end{center}
 
 est
 
 \begin{center}

 \red{\underline{Il existe}} \black un homme \red{\underline{immortel}}\black .
 
 \end{center}
 
\item La négation de

\begin{center}

 \blue{\underline{Il existe}} \black un dessert \blue{\underline{sans sucre}} \black à la cantine.
 
 \end{center}
 
 est
 
 \begin{center}

 \red{\underline{Tous}} \black les desserts \red{\underline{sont sucrés}} \black à la cantine.
 \end{center}
 
 \medskip

\textbf{Remarque~:} Dans les deux exemples que nous venons de traiter, pour écrire la négation d'une phrase, il suffit de remplacer les \og tous \fg~{} par \og il existe \fg~{}, et réciproquement~; et d'inverser les conclusions (exemple~: immortel/mortel). C'est une technique qui fonctionne toujours.
\item La négation de

\begin{center}

 \blue{\underline{Il existe}} \black un pays dans lequel \blue{\underline{tous}} \black les hommes \black \blue{\underline{savent lire}}\black .
 
 \end{center}
 
 est
 
 \begin{center}

Dans \red{\underline{tous}} \black les pays, \red{\underline{il existe}} \black un homme qui  \red{\underline{ne sait pas lire}}\black .
 \end{center}
\item Le contraire de \og être allé en Angleterre ou en Espagne \fg~{} est \og n'être allé ni en Angleterre, ni en Espagne \fg , donc la négation de

\begin{center}

 \blue{\underline{Tous}} \black les élèves de la classe  \blue{\underline{sont déjà allés en Angleterre ou en Espagne }}\black .
 
 \end{center}
 
 est
 
 \begin{center}

 \red{\underline{Il existe}} \black un élève de la classe qui \red{\underline{n'est jamais allé en Angleterre, ni en Espagne}}\black .
 \end{center}
\item Comme dans l'exemple précédent, le contraire de \og ni... ni... \fg~{} est \og ou \fg . Donc la négation de

\begin{center}

 Chloé \blue{\underline{n'aime ni les fraises, ni les framboises}}\black .
 
 \end{center}
 
 est
 
 \begin{center}

 Chloé \red{\underline{aime les fraises ou les framboises}}\black .
 \end{center}

\end{enumerate}



\end{exo}


\begin{exo}%19


\begin{enumerate}
\item \begin{enumerate}
\item On identifie A et B dans l'implication~:

\begin{center}

Si $\underbrace{\text{un nombre se termine par 5}}_{\text{A}}$, alors $\underbrace{\text{il est multiple de 5}}_{\text{B}}.$

\end{center}

Cette implication est vraie (cours du primaire).

\item

\begin{itemize}
\item[\textbullet] L'implication contraposée est

\begin{center}

Si $\underbrace{\text{un nombre n'est pas multiple de 5}}_{\text{non B}}$, alors $\underbrace{\text{il ne se termine pas par 5}}_{\text{non A}}.$

\end{center}

Cette contraposée est vraie, puisque l'implication originale l'est (cf l'énoncé~: quand une implication est vraie, sa contraposée l'est aussi).

\medskip

\item[\textbullet] L'implication réciproque est

\begin{center}

Si $\underbrace{\text{un nombre est multiple de 5}}_{\text{B}}$, alors $\underbrace{\text{il se termine par 5}}_{\text{A}}.$

\end{center}

Elle est fausse, comme le montre le contre-exemple suivant~: 10 est multiple de 5, mais il ne se termine pas par 5.
\end{itemize}

\end{enumerate}
\item L'implication

\begin{center}

Si $\underbrace{\text{un nombre se termine par 0}}_{\text{A}}$, alors $\underbrace{\text{il est multiple de 10}}_{\text{B}}.$

\end{center}

et sa réciproque

\begin{center}

Si $\underbrace{\text{un nombre est multiple de 10}}_{\text{B}}$, alors $\underbrace{\text{il se termine par 0}}_{\text{A}}.$

\end{center}

sont vraies toutes les deux.

\end{enumerate}
\end{exo}

\begin{exo}

Soit $ABC$ un triangle
\begin{enumerate}
\item \textbf{Théorème de Pythagore.}
\begin{center}


Si $ABC$ est rectangle en $A,$ alors $BC^2=AB^2+BC^2.$

\begin{center}
\newrgbcolor{xfqqff}{0.4980392156862745 0. 1.}
\psset{xunit=1.0cm,yunit=1cm,algebraic=true,dimen=middle,dotstyle=o,dotsize=5pt 0,linewidth=2.pt,arrowsize=3pt 2,arrowinset=0.25}
\begin{pspicture*}(2.12,0.46)(7.88,4.24)
\pspolygon[linewidth=2.pt,linecolor=xfqqff,fillcolor=xfqqff!20!white,fillstyle=solid,opacity=0.1](3.4242640687119286,1.)(3.4242640687119286,1.4242640687119286)(3.,1.4242640687119286)(3.,1.)
\psline[linewidth=2.pt](7.,1.)(3.,1.)
\psline[linewidth=2.pt](3.,1.)(3.,4.)
\psline[linewidth=2.pt](3.,4.)(7.,1.)
\rput[tl](2.6,4){$C$}
\rput[tl](2.6,1){$A$}
\rput[tl](7.2,1.1){$B$}
\end{pspicture*}
\end{center}

\end{center}
\item \textbf{Théorème contraposé de Pythagore.}

\begin{center}

Si $BC^2\not=AB^2+BC^2,$ alors $ABC$ n'est pas rectangle en $A.$
\end{center}
\item \textbf{Théorème réciproque de Pythagore.}

\begin{center}
Si $BC^2=AB^2+BC^2,$ alors $ABC$ est  rectangle en $A.$
\end{center}

Le théorème réciproque est bien sûr vrai, comme vous l'avez appris au collège.

\medskip

\danger En devoir, le correcteur sera très attentif au nom du théorème utilisé dans les démonstrations~: théorème, théorème contraposé ou théorème réciproque -- il ne faudra pas confondre~!
\end{enumerate}

\end{exo}

\begin{exo}

\begin{enumerate}
\item Pour construire la figure, on trace successivement~:

\begin{itemize}
\item[\textbullet] Le segment $\left[EF\right].$
\item[\textbullet] La perpendiculaire à $\left[EF\right]$ passant par $E.$
\item[\textbullet] Un arc de cercle de centre $F,$ de rayon 7~cm. Il coupe la perpendiculaire que nous venons de tracer en $G.$
\end{itemize}

\setlength{\columnseprule}{1pt}
\begin{multicols}{2}
\begin{center}
\newrgbcolor{ududff}{0.30196078431372547 0.30196078431372547 1.}
\newrgbcolor{xfqqff}{0.4980392156862745 0. 1.}
\psset{xunit=0.8cm,yunit=0.8cm,algebraic=true,dimen=middle,dotstyle=o,dotsize=5pt 0,linewidth=2.pt,arrowsize=3pt 2,arrowinset=0.25}
\begin{pspicture*}(-2.94,-0.7)(5.42,6.16)
\pspolygon[linewidth=2.pt,linecolor=xfqqff,fillcolor=xfqqff!20!white,fillstyle=solid,opacity=0.1](-0.5757359312880717,0.)(-0.5757359312880717,0.4242640687119283)(-1.,0.4242640687119283)(-1.,0.)
\psline[linewidth=2.pt](-1.,0.)(4.,0.)
\rput[tl](1.,-0.2){$5~\text{cm}$}
\psline[linewidth=2.pt](-1.,-0.7)(-1.,6.16)
\parametricplot[linewidth=2.pt]{2.1976990745518967}{2.5472509907369547}{1.*7.*cos(t)+0.*7.*sin(t)+4.|0.*7.*cos(t)+1.*7.*sin(t)+0.}
\psline[linewidth=2.pt](4.,0.)(-1.,4.898979485566355)
\rput[tl](1.5,2.82){$7~\text{cm}$}
\rput[bl](-1.5,-0.3){{$E$}}
\rput[bl](4.14,-0.3){{$F$}}
\rput[bl](-1.44,5.04){{$G$}}
\end{pspicture*}
\end{center}

D'après \textbf{le théorème de Pythagore} dans $EFG$ rectangle en $E~:$
\begin{align*}
FG^2&=EF^2+EG^2\\
7^2&=5^2+EG^2\\
49&=25+EG^2\\
49-25&=EG^2\\
\sqrt{24}&=EG
\end{align*}

Conclusion~: $EG=\sqrt{24}~\text{cm}.$

\medskip

\danger Sauf si l'énoncé le demande, ne donnez pas de valeur approchée.

\end{multicols}
\item Le plus grand côté est $\left[BC\right],$ donc le triangle ne pourrait être rectangle qu'en $A.$

On calcule~:
\[
\left.
    \begin{array}{ll}
        BC^2=6^2=36\\
        AB^2+AC^2=5^2+4^2=25+16=41
    \end{array}
\right \}BC^2\not=AB^2+AC^2.
\]
D'après \textbf{la contraposée du théorème de Pythagore}, $ABC$ n'est pas rectangle en $A.$

\end{enumerate}

\end{exo}


\begin{exo}


$ABCDEFGH$ est un parallélépipède rectangle tel que $AB=BC=6$ et $CG=3.$
\begin{center}
\newrgbcolor{ududff}{0.30196078431372547 0.30196078431372547 1.}
\newrgbcolor{xfqqff}{0.4980392156862745 0. 1.}
\psset{xunit=1.0cm,yunit=1.0cm,algebraic=true,dimen=middle,dotstyle=o,dotsize=5pt 0,linewidth=2.pt,arrowsize=3pt 2,arrowinset=0.25}
\begin{pspicture*}(-3.3774236991656137,-0.31214840896786467)(5.5552457552212475,4.59336208917407)
\pspolygon[linewidth=0.pt,linecolor=xfqqff,fillcolor=xfqqff!20!white,fillstyle=solid,opacity=0.1](-2.6,0.2)(-2.2,0.2)(-2.6,0.)(-3.,0.)
\pspolygon[linewidth=0.pt,linecolor=xfqqff,fillcolor=xfqqff!20!white,fillstyle=solid,opacity=0.1](2.6046541583780107,0.09883646040549732)(3.,0.)(3.,0.4)(2.6049879803510483,0.5150893759578578)
\psline[linewidth=2.pt,linestyle=dashed,dash=2pt 2pt](-1.,1.)(-3.,0.)
\psline[linewidth=2.pt](-3.,0.)(3.,0.)
\psline[linewidth=2.pt](3.,0.)(5.,1.)
\psline[linewidth=2.pt,linestyle=dashed,dash=2pt 2pt](5.,1.)(-1.,1.)
\psline[linewidth=2.pt](-1.,4.)(-3.,3.)
\psline[linewidth=2.pt](-3.,3.)(3.,3.)
\psline[linewidth=2.pt](3.,3.)(5.,4.)
\psline[linewidth=2.pt](5.,4.)(-1.,4.)
\psline[linewidth=2.pt](-3.,3.)(-3.,0.)
\psline[linewidth=2.pt,linestyle=dashed,dash=2pt 2pt](-1.,4.)(-1.,1.)
\psline[linewidth=2.pt](3.,3.)(3.,0.)
\psline[linewidth=2.pt](5.,4.)(5.,1.)
\psline[linewidth=2.pt,linestyle=dashed,dash=2pt 2pt](-1.,1.)(3.,3.)
\psline[linewidth=2.pt,linestyle=dashed,dash=2pt 2pt](-1.,1.)(3.,0.)
\psline[linewidth=2.pt,linecolor=ududff](-2.6,0.2)(-2.2,0.2)
\psline[linewidth=2.pt,linecolor=ududff](-2.2,0.2)(-2.6,0.)
\psline[linewidth=2.pt,linecolor=ududff](2.6046541583780107,0.09883646040549732)(2.6049879803510483,0.5150893759578578)
\psline[linewidth=2.pt,linecolor=ududff](2.6049879803510483,0.5150893759578578)(3.,0.4)
\rput[tl](-2.018464709815479,0.8976565449657882){6}
\rput[tl](0.01997377420972306,0.3){6}
\rput[tl](3.1356358473539347,1.6599993926500078){3}
\rput[bl](-1.35,1.1628192745950818){{$A$}}
\rput[bl](-2.9299615929161793,0.2){{$B$}}
\rput[bl](3.069345164946611,0.16845903848523028){{$C$}}
\rput[bl](5.058065637166321,1.1628192745950818){{$D$}}
\rput[bl](-0.9412411206964699,4.162472653526468){{$E$}}
\rput[bl](-2.9299615929161793,3.168112417416616){{$F$}}
\rput[bl](2.95,3.168112417416616){{$G$}}
\rput[bl](5.058065637166321,4.162472653526468){{$H$}}
\end{pspicture*}
\end{center}

On utilise deux fois de suite le théorème de Pythagore~:

\setlength{\columnseprule}{1pt}
\begin{multicols}{2}
 Dans $ABC$ rectangle en $B,$
 
 \begin{align*}
AC^2&=AB^2+BC^2\\
AC^2&=6^2+6^2\\
AC^2&=36+36\\
AC^2&=72\\
(\text{Inutile de} & \text{ donner} AC~!)
\end{align*}

\columnbreak

 Dans $ACG$ rectangle en $C,$
 
 \begin{align*}
AG^2&=AC^2+CG^2\\
AG^2&=72+3^2\\
AG^2&=72+9\\
AG^2&=81\\
AG&=\sqrt{81}=9
\end{align*}

\end{multicols}




Conclusion~: $AG=9.$

\end{exo}

\begin{exo}
Sur la figure ci-dessous (qui n'est pas à l'échelle), le segment $\left[MK\right]$ mesure 3~cm, le segment $\left[MN\right]$ mesure 5~cm et $h=1,2$~cm.%l'aire du triangle $MNP$ est égale à $2,5~\text{cm}^2.$

\begin{center}
\psset{xunit=1.cm,yunit=1.cm,algebraic=true,dimen=middle,dotstyle=o,dotsize=5pt 0,linewidth=2.pt,arrowsize=3pt 2,arrowinset=0.25}
\begin{pspicture*}(0.34,0.28)(6.18,4.64)
\pspolygon[linewidth=2.pt,fillcolor=black!20!white,fillstyle=solid,opacity=0.1](1.424264068711929,1.)(1.424264068711929,1.424264068711929)(1.,1.424264068711929)(1.,1.)
\pspolygon[linewidth=2.pt,fillcolor=black!20!white,fillstyle=solid,opacity=0.1](3.465441558772843,1.6205887450304568)(3.804852813742386,1.3660303038032993)(4.059411254969543,1.7054415587728424)(3.72,1.96)
\psline[linewidth=2.pt](1.,1.)(5.,1.)
\psline[linewidth=2.pt](1.,1.)(1.,4.)
\psline[linewidth=2.pt](1.,4.)(5.,1.)
\psline[linewidth=2.pt](1.,4.)(3.,1.)
\psline[linewidth=2.pt](3.,1.)(3.72,1.96)
\rput[tl](3.1,1.75){$h$}
\psdots[dotsize=1pt 0,dotstyle=*](1.,1.)
\rput[bl](0.62,0.96){$K$}
\psdots[dotsize=1pt 0,dotstyle=*](5.,1.)
\rput[bl](5.1,0.76){$N$}
\psdots[dotsize=1pt 0,dotstyle=*](1.,4.)
\rput[bl](1.12,4.08){$M$}
\psdots[dotsize=1pt 0,dotstyle=*](3.,1.)
\rput[bl](2.8,0.68){$P$}
\end{pspicture*}
\end{center}

\begin{enumerate}
\item $\mathcal{A}_{MNP}=\frac{MN\times h}{2}=\frac{5\times 1,2}{2}=3~\text{cm}^2.$
\item On a aussi $\mathcal{A}_{MNP}=\frac{PN\times MK}{2},$ donc $3=\frac{PN\times 3}{2},$ soit $\cancel{3}\times 2=PN\times \cancel{3}~;$ et donc $PN=2~\text{cm}.$
\item (Non détaillé.) Il faut calculer successivement $KN,$ puis $KP$ et $MP.$

\danger On ne sait pas, à ce stade, que $P$ est le milieu de $\left[KN\right].$
\begin{itemize}
\item[\textbullet] Pour $KN,$ on utilise le théorème de Pythagore dans le triangle $KMN.$ On obtient $KN=4~\text{cm}.$
\item[\textbullet] $KP=KN-PN=4-2=2~\text{cm}.~~~$ %(\danger N'utilisez pas le fait que $P$ est le milieu de $\left[KN\right]$ pour calculer $KP,$ car vous n'en savez rien.)
\item[\textbullet] Enfin, pour calculer $PM,$ on utilise le théorème de Pythagore dans le triangle $KMP.$ On obtient $MP=\sqrt{13}~\text{cm}.$
\end{itemize}
\end{enumerate}

\end{exo}

\begin{exo}
\begin{enumerate}
\item Les côtés de l'angle droit d'un triangle rectangle mesurent $a$ et $b,$ l'hypoténuse mesure $c.$


\begin{center}
\newrgbcolor{xfqqff}{0.4980392156862745 0. 1.}
\psset{xunit=1.0cm,yunit=1.0cm,algebraic=true,dimen=middle,dotstyle=o,dotsize=5pt 0,linewidth=2.pt,arrowsize=3pt 2,arrowinset=0.25}
\begin{pspicture*}(2.12,0.46)(7.88,4.24)
\pspolygon[linewidth=2.pt,linecolor=xfqqff,fillcolor=xfqqff!20!white,fillstyle=solid,opacity=0.1](3.4242640687119286,1.)(3.4242640687119286,1.4242640687119286)(3.,1.4242640687119286)(3.,1.)
\psline[linewidth=2.pt](7.,1.)(3.,1.)
\psline[linewidth=2.pt](3.,1.)(3.,4.)
\psline[linewidth=2.pt](3.,4.)(7.,1.)
\rput[tl](5.2,2.62){$c$}
\rput[tl](4.62,0.9){$a$}
\rput[tl](2.45,2.5){$b$}
\end{pspicture*}
\end{center}

D'après le théorème de Pythagore, $c^2=a^2+b^2,$ donc \[c=\sqrt{a^2+b^2}.\]

\item L'affirmation 
\begin{center}
 Pour tous nombres positifs $a$ et $b,$ $\sqrt{a^2+b^2}=a+b.$ 
\end{center}

est FAUSSE~! Voici deux justifications~:

\begin{itemize}
\item[\textbullet] \textbf{Par le calcul.} Il suffit de donner un contre-exemple~: on choisit $a=4$ et $b=3.$ Dans ce cas
\[\sqrt{a^2+b^2}=\sqrt{4^2+3^2}=\sqrt{16+9}=\sqrt{25}=5\qquad \text{est différent de}\qquad a+b=4+3=7.\]
\item[\textbullet] \textbf{Géométriquement.} $\sqrt{a^2+b^2}$ est la longueur de l'hypoténuse $c$ du triangle rectangle de la question 1~; tandis que $a+b$ est la somme des longueurs des côtés de l'angle droit. Or cette somme est strictement plus grande que celle de l'hypoténuse, puisque le chemin le plus court d'un point à un autre est la ligne droite.
\end{itemize}
\end{enumerate}
\end{exo}





\begin{exo}

Soit $A$ un point et $\Delta$ une droite du plan. Le projeté orthogonal de $A$ sur $\Delta$ est le point $H$ de $\Delta$ tel que $(AH)\perp\Delta.$

\begin{enumerate}
\item On trace la perpendiculaire à $\Delta$ passant par $A.$ Elle coupe $\Delta$ en $H.$


\begin{center}
\newrgbcolor{ududff}{0.30196078431372547 0.30196078431372547 1.}
\newrgbcolor{xfqqff}{0.4980392156862745 0. 1.}
\psset{xunit=1.0cm,yunit=1.0cm,algebraic=true,dimen=middle,dotstyle=o,dotsize=5pt 0,linewidth=2.pt,arrowsize=3pt 2,arrowinset=0.25}
\begin{pspicture*}(-2.6,-0.64)(3.82,3.76)
\pspolygon[linewidth=2.pt,linecolor=xfqqff,fillcolor=xfqqff!20!white,fillstyle=solid,opacity=0.1](1.3760676226484692,0.807765180139509)(1.2347599047685438,1.2078053392080301)(0.8347197457000228,1.0664976213281048)(0.9760274635799481,0.6664574622595838)
\psplot[linewidth=2.pt]{-2.6}{3.82}{(--1.2932--1.42*x)/4.02}
\psplot[linewidth=2.pt,linestyle=dashed,dash=2pt 2pt,linecolor=red]{-2.6}{3.82}{(-4.87--4.02*x)/-1.42}
\rput[tl](-1.52,0.3){$\Delta$}
\psdots[dotstyle=*,linecolor=ududff](0.18,2.92)
\rput[bl](0.26,3.12){\ududff{$A$}}
\psdots[dotstyle=*,linecolor=ududff](0.9760274635799481,0.6664574622595838)
\rput[bl](0.7,0.22){\ududff{$H$}}
\end{pspicture*}
\end{center}
\item Par construction, le triangle $AMH$ est rectangle en $H,$ donc son hypoténuse $AM$ est strictement plus grande que le côté de l'angle droit $AH$ (c'est le même raisonnement que celui de l'exercice précédent)~:
\[AM>AH.\]

\begin{center}
\newrgbcolor{ududff}{0.30196078431372547 0.30196078431372547 1.}
\newrgbcolor{xfqqff}{0.4980392156862745 0. 1.}
\psset{xunit=1.0cm,yunit=1.0cm,algebraic=true,dimen=middle,dotstyle=o,dotsize=5pt 0,linewidth=2.pt,arrowsize=3pt 2,arrowinset=0.25}
\begin{pspicture*}(-2.6,-0.64)(4.18,3.76)
\pspolygon[linewidth=2.pt,linecolor=xfqqff,fillcolor=xfqqff!20!white,fillstyle=solid,opacity=0.1](1.3760676226484692,0.8077651801395092)(1.2347599047685438,1.2078053392080301)(0.8347197457000228,1.0664976213281048)(0.9760274635799481,0.6664574622595838)
\psplot[linewidth=2.pt]{-2.6}{4.18}{(--1.2932--1.42*x)/4.02}
\psplot[linewidth=2.pt,linestyle=dashed,dash=2pt 2pt,linecolor=red]{-2.6}{4.18}{(-4.87--4.02*x)/-1.42}
\rput[tl](-1.52,0.3){$\Delta$}
\psline[linewidth=2.pt](2.862531138594252,1.332834382289512)(0.18,2.92)
\psdots[dotstyle=*,linecolor=ududff](0.18,2.92)
\rput[bl](0.26,3.12){\ududff{$A$}}
\psdots[dotstyle=*,linecolor=ududff](0.9760274635799481,0.6664574622595838)
\rput[bl](0.7,0.22){\ududff{$H$}}
\psdots[dotstyle=*,linecolor=ududff](2.862531138594252,1.332834382289512)
\rput[bl](2.94,1.54){\ududff{$M$}}
\end{pspicture*}
\end{center}


\item Le segment $\left[AH\right]$ est la hauteur\footnote{Le mot \textit{hauteur} est polysémique (il a plusieurs sens)~: le segment $\left[AH\right]$ peut être appelé \textit{hauteur}, la droite $\left(AH\right)$ peut également être appelée \textit{hauteur}~; enfin la longueur $AH$ peut elle aussi être appelée \textit{hauteur} -- c'est cette longueur, par exemple, que l'on retrouve dans la formule $\frac{B\times h}{2}$ pour l'aire du triangle.} issue de $A$ dans le triangle $ABC.$


\begin{center}
\newrgbcolor{ududff}{0.30196078431372547 0.30196078431372547 1.}
\newrgbcolor{xfqqff}{0.4980392156862745 0. 1.}
\newrgbcolor{xdxdff}{0.49019607843137253 0.49019607843137253 1.}
\psset{xunit=1.0cm,yunit=1.0cm,algebraic=true,dimen=middle,dotstyle=o,dotsize=5pt 0,linewidth=2.pt,arrowsize=3pt 2,arrowinset=0.25}
\begin{pspicture*}(-2.6,-0.64)(4.18,3.76)
\pspolygon[linewidth=2.pt,linecolor=xfqqff,fillcolor=xfqqff!20!white,fillstyle=solid,opacity=0.1](1.3760676226484692,0.8077651801395092)(1.2347599047685438,1.2078053392080301)(0.8347197457000228,1.0664976213281048)(0.9760274635799481,0.6664574622595838)
\psline[linewidth=2.pt](2.862531138594252,1.332834382289512)(0.18,2.92)
\psline[linewidth=2.pt](0.18,2.92)(-1.2704379208661591,-0.12707011135073243)
\psline[linewidth=2.pt](-1.2704379208661591,-0.12707011135073243)(2.862531138594252,1.332834382289512)
\psline[linewidth=2.pt,linestyle=dashed,dash=2pt 2pt,linecolor=red](0.9760274635799481,0.6664574622595838)(0.18,2.92)
\psdots[dotstyle=*,linecolor=ududff](0.18,2.92)
\rput[bl](0.26,3.12){\ududff{$A$}}
\psdots[dotstyle=*,linecolor=ududff](0.9760274635799481,0.6664574622595838)
\rput[bl](0.7,0.22){\ududff{$H$}}
\psdots[dotstyle=*,linecolor=ududff](2.862531138594252,1.332834382289512)
\rput[bl](2.94,1.54){\ududff{$C$}}
\psdots[dotstyle=*,linecolor=xdxdff](-1.2704379208661591,-0.12707011135073243)
\rput[bl](-1.65,0.04){\xdxdff{$B$}}
\end{pspicture*}
\end{center}



\end{enumerate}
\end{exo}


\newpage

\begin{exo}


On résout les équations~:

\setlength{\columnseprule}{1pt}
\begin{multicols}{5}
\begin{align*}
x+7&=18\\
x+\cancel{7}\textcolor{red}{-\cancel{7}}&=18\textcolor{red}{-7}\\
x&=11
\end{align*}

La solution est $x=11$

\columnbreak

\begin{align*}
3x+4&=19\\
3x+\cancel{4}\textcolor{red}{-\cancel{4}}&=19\textcolor{red}{-4}\\
3x&=15\\
\frac{\cancel{3}x}{\textcolor{blue}{\cancel{3}}}&=\frac{15}{\textcolor{blue}{3}}\\
x&=5
\end{align*}

La solution est $x=5.$

\columnbreak

\begin{align*}
3,5x-9&=5\\
3,5x-\cancel{9}\textcolor{red}{+\cancel{9}}&=5\textcolor{red}{+9}\\
3,5x&=14\\
\frac{\cancel{3,5}x}{\textcolor{blue}{\cancel{3,5}}}&=\frac{14}{\textcolor{blue}{3,5}}\\
x&=\frac{14}{3,5}
\end{align*}

Or $\frac{14}{3,5}=\frac{14\times 2}{3,5\times 2}=\frac{28}{7}=4,$ donc la solution est $x=4.$

 
 \columnbreak
 
 \begin{align*}
x+1&=-2x-5\\
x+1\textcolor{green}{+2x}&=\cancel{-2x}-5\textcolor{green}{+\cancel{2x}}\\
3x+1&=-5\\
3x+\cancel{1}\textcolor{red}{-\cancel{1}}&=-5\textcolor{red}{-{1}}\\
3x&=-6\\
\frac{\cancel{3}x}{\textcolor{blue}{\cancel{3}}}&=\frac{-6}{\textcolor{blue}{3}}\\
x&=-2
\end{align*}

La solution est $x=-2.$

 \columnbreak
 
 \begin{align*}
-2x+4&=3x-6\\
-2x+4\textcolor{green}{-3x}&=\cancel{3x}-6\textcolor{green}{-\cancel{3x}}\\
-5x+4&=-6\\
-5x+\cancel{4}\textcolor{red}{-\cancel{4}}&=-6\textcolor{red}{-4}\\
-5x&=-10\\
\frac{\cancel{-5}x}{\textcolor{blue}{\cancel{-5}}}&=\frac{-10}{\textcolor{blue}{-5}}\\
x&=2
\end{align*}

La solution est $x=2.$

\end{multicols}

\end{exo}



\begin{exo}

Les deux plateaux de la balance ci-dessous sont en équilibre. Les poids noirs ont tous la même masse $M$~kg.


\begin{center}
\psset{xunit=1cm,yunit=0.8cm,algebraic=true,dimen=middle,dotstyle=o,dotsize=3pt 0,linewidth=0.8pt,arrowsize=3pt 2,arrowinset=0.25}
\begin{pspicture*}(-0.12,0.82)(6.12,4.7)
\pspolygon[fillcolor=black,fillstyle=solid,opacity=1.0](5.29,3.17)(5.29,3.67)(5.79,3.67)(5.79,3.17)
\pspolygon[fillcolor=black,fillstyle=solid,opacity=1.0](0.09,3.13)(0.09,3.63)(0.59,3.63)(0.59,3.13)
\pspolygon[fillcolor=black,fillstyle=solid,opacity=1.0](0.8,3.13)(0.8,3.63)(1.3,3.63)(1.3,3.13)
\pspolygon[fillcolor=black,fillstyle=solid,opacity=1.0](1.48,3.14)(1.48,3.64)(1.98,3.64)(1.98,3.14)
\pspolygon(4.04,4.19)(4.84,4.2)(4.84,3.19)(4.04,3.19)
\pspolygon(0.5,4.5)(0.5,4)(1.5,4)(1.5,4.5)
\psline[linewidth=2pt](3,1)(1,3)
\psline[linewidth=2pt](3,1)(5,3)
\psline[linewidth=2pt](0,3)(2,3)
\psline[linewidth=2pt](4,3)(6,3)
\psline(5.29,3.17)(5.29,3.67)
\psline(5.29,3.67)(5.79,3.67)
\psline(5.79,3.67)(5.79,3.17)
\psline(5.79,3.17)(5.29,3.17)
\psline(0.09,3.13)(0.09,3.63)
\psline(0.09,3.63)(0.59,3.63)
\psline(0.59,3.63)(0.59,3.13)
\psline(0.59,3.13)(0.09,3.13)
\psline(0.8,3.13)(0.8,3.63)
\psline(0.8,3.63)(1.3,3.63)
\psline(1.3,3.63)(1.3,3.13)
\psline(1.3,3.13)(0.8,3.13)
\psline(1.48,3.14)(1.48,3.64)
\psline(1.48,3.64)(1.98,3.64)
\psline(1.98,3.64)(1.98,3.14)
\psline(1.98,3.14)(1.48,3.14)
\psline(4.04,4.19)(4.84,4.2)
\psline(4.84,4.2)(4.84,3.19)
\psline(4.84,3.19)(4.04,3.19)
\psline(4.04,3.19)(4.04,4.19)
\rput[tl](4.05,3.82){10 kg}
\psline(0.5,4.5)(0.5,4)
\psline(0.5,4)(1.5,4)
\psline(1.5,4)(1.5,4.5)
\psline(1.5,4.5)(0.5,4.5)
\rput[tl](0.75,4.42){7 kg}
\end{pspicture*}
\end{center}


Le fait que la balance soit en équilibre se traduit par l'équation
\[3M+7=10+M.\] On la résout~:

\begin{align*}
3M+7\textcolor{green}{-M}&=10+\cancel{M}\textcolor{green}{-\cancel{M}}\\
2M+7&=10\\
2M+\cancel{7}\textcolor{red}{-\cancel{7}}&=10\textcolor{red}{-7}\\
2M&=3\\
\frac{\cancel{2}M}{\textcolor{blue}{\cancel{2}}}&=\frac{3}{\textcolor{blue}{2}}\\
M&=1,5
\end{align*}

Conclusion~: la solution est $M=1,5.$



\end{exo}






\begin{exo}

Le stade des Gones compte \np{15000} places. Il y a $x$ places dans les virages et les autres dans les tribunes. Une place en virage coûte 15 € et une place dans les tribunes coûte 25 €.

\medskip

Aujourd'hui, le stade est plein et la recette est de \np{295000} €.

\begin{enumerate}
\item Il y a $x$ places dans les virages, donc $(\np{15000}-x)$ places dans les tribunes. La recette totale en € est donc
\[15\times x+25\times(\np{15000}-x).\] Comme cette recette est \np{295000} €, $x$ est solution de l'équation
\[15 x+25(\np{15000}-x)=\np{295000}.\]
\item On résout l'équation de la question précédente~:

\begin{align*}
15 x+25(\np{15000}-x)&=\np{295000}\\
15x+25\times\np{15000}+25\times (-x)&=\np{295000}\\
15x+\np{375000}-25x&=\np{295000}\\
-10x+\np{375000}&=\np{295000}\\
-10x+\cancel{\np{375000}}\textcolor{red}{-\cancel{\np{375000}}}&=\np{295000}\textcolor{red}{-\np{375000}}\\
-10x&=-\np{80000}\\
\frac{\cancel{\textcolor{blue}{-10}}x}{\textcolor{blue}{\cancel{-10}}}&=\frac{-\np{80000}}{\textcolor{blue}{{-10}}}\\
x&=8000.\end{align*}

\medskip

Conclusion~: il y a $x=\np{8000}$ places dans les virages (et donc \np{7000} dans les tribunes).
\end{enumerate}


\end{exo}





\end{document}
