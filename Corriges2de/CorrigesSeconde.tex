\documentclass[10pt]{article}
\usepackage[T1]{fontenc}
\usepackage[utf8]{inputenc}
\usepackage{fourier}
\usepackage[scaled=0.875]{helvet}
\renewcommand{\ttdefault}{lmtt}
\usepackage{amsmath,amssymb,makeidx}
\usepackage[normalem]{ulem}
\usepackage{fancybox}
\usepackage{cancel}
\usepackage{stmaryrd}
\usepackage{ulem}
\usepackage{tabularx}
\usepackage{geometry}
\usepackage{enumerate}
\geometry{hmargin=1.5cm,vmargin=1.5cm}
\usepackage{dcolumn}
\usepackage{textcomp}
\usepackage{lscape}
\usepackage{eurosym}
%\newcommand{\euro}{\eurologo{}}
\usepackage[dvips]{color}
\usepackage[all]{xy}
\usepackage{xlop}

\usepackage{tikz,tkz-tab}

\usepackage{systeme}


\usepackage{pstricks,pst-plot,pst-text,pst-tree,pstricks-add}
\usepackage{colortbl}
\usepackage{diagbox}
\usepackage{fontawesome5}
\usepackage{pifont}
\usepackage{wasysym}


\usepackage{theorem}
\theorembodyfont{\upshape}
\newtheorem{exo}{Exercice}
%\newtheorem{exo}{Exercice}%[section]
\usepackage{hyperref}
\hypersetup{
    colorlinks=true,       % false: liens encadrés; true: liens colorés
    linkcolor=blue,          % couleur des liens (ou bordures) internes
}

%\setlength{\voffset}{-1,5cm}
\usepackage{fancyhdr} 
\usepackage{graphicx}
\usepackage[frenchb]{babel}
\usepackage[np]{numprint}
\usepackage{multicol}
\usepackage{xlop}
\usepackage{soul}


\title{Mathématiques -- Seconde}

\date{Corrigés des exercices}
\begin{document}
\setlength\parindent{0mm}
\renewcommand \footrulewidth{.2pt}

\maketitle

\tableofcontents


\newpage


\section{Rappels de calcul et de géométrie}

\begin{exo}

Dans chaque question, on obtient la réponse à l'aide d'un tableau de proportionnalité.

\begin{enumerate}
\item ~{}
\begin{center}
\begin{tabular}{|c|c|c|}\hline
Nombre de personnes& 4&6 \\ \hline 
Farine (en g)&250& ? \\ \hline
Lait (en mL)&500& ? \\ \hline
Œufs&4& 6 \\ \hline
\end{tabular}
\end{center}

Pour 6 personnes, il faut $\frac{250\times 6}{4}=\frac{\np{1500}}{4}=375$~g de farine, $\frac{500\times 6}{4}=\frac{\np{3000}}{4}=750$~mL de lait et, bien sûr, 6 œufs.

\item Les 6 yaourts pèsent $6\times 125=750$~g.

\begin{center}
\begin{tabular}{|c|c|c|}\hline
masse (en g)& 1000&750 \\ \hline 
prix (en \euro)&2& ? \\ \hline
\end{tabular}
\end{center}

Je payerai $\frac{750\times 2}{\np{1000}}=\frac{\np{1500}}{\np{1000}}=1,5~\text{\euro}.$

\item Généralement, dans ce type de question, il vaut mieux convertir en minutes\footnote{Les calculs ne sont pas toujours plus faciles en minutes qu'en heures, mais c'est généralement le cas.}.

\begin{center}
\begin{tabular}{|c|c|c|}\hline
temps (en min)& 60&? \\ \hline 
distance (en km)&20& 45 \\ \hline
\end{tabular}
\end{center}

On mettra $\frac{60\times 45}{20}=\frac{\cancel{20}\times 3\times 45}{\cancel{20}}=135$~min, soit 2~h~15~min (puisque $135=120+15$).
\item L'énoncé donne les informations recensées dans le tableau ci-dessous et demande de compléter la case \textcircled{\small{1}}.

\begin{center}
\begin{tabular}{|c|c|c|c|}\hline
Florins& 7&?&\textcircled{\small{1}} \\ \hline 
Pistoles&6& \textcolor{red}{4}&\textcircled{\small{\textcolor{black}{2}}} \\ \hline
Deniers&?& \textcolor{red}{5}&\textcolor{red}{30} \\ \hline
\end{tabular}
\end{center}

On complète d'abord la case \textcircled{\small{2}}~: en échange de 30 deniers, on a $4\times 30\div 5=24$~pistoles~:

\begin{center}
\begin{tabular}{|c|c|c|c|}\hline
Florins& \textcolor{red}{7}&?&\textcircled{\small{\textcolor{black}{1}}} \\ \hline 
Pistoles&\textcolor{red}{6}& 4&\textcolor{red}{24} \\ \hline
Deniers&?& 5&30 \\ \hline
\end{tabular}
\end{center}

On peut alors compléter la case \textcircled{\small{1}}~: en échange de 30 deniers, on a $\frac{7\times 24}{6}=\frac{7\times 4\times \cancel{6}}{\cancel{6}}=28$~florins.

\end{enumerate}
\end{exo}

\begin{exo}

\begin{enumerate}
\item On complète deux tableaux de proportionnalité (on travaille en min et en km)~:

\begin{multicols}{2}

\begin{center}
\begin{tabular}{|c|c|c|}\hline
temps (en min)& 60&? \\ \hline 
distance (en km)&3& 0,5 \\ \hline
\end{tabular}


\begin{tabular}{|c|c|c|}\hline
temps (en min)& 60&? \\ \hline 
distance (en km)&15& 5 \\ \hline
\end{tabular}
\end{center}

\end{multicols}

Stéphane nage $\frac{60\times 0,5}{3}=\frac{30}{3}=10$~min, puis il court $\frac{60\times 5}{15}=\frac{300}{15}=20$~min.


\item Stéphane a parcouru un total de $5+0,5=5,5$~km, en $10+20=30$~min.

\begin{center}
\begin{tabular}{|c|c|c|}\hline
temps (en min)& 30&60 \\ \hline 
distance (en km)&5,5& ? \\ \hline
\end{tabular}
\end{center}

La vitesse moyenne de Stéphane sur l’ensemble de son parcours est donc $\frac{60\times 5,5}{30}=\frac{\cancel{30}\times 2\times 5,5}{\cancel{30}}=11$~km/h.
\end{enumerate}
\end{exo}

\begin{exo}%28
~{}

\begin{center}
\newrgbcolor{xfqqff}{0.4980392156862745 0. 1.}
\psset{xunit=1.0cm,yunit=1.0cm,algebraic=true,dimen=middle,dotstyle=o,dotsize=3pt 0,linewidth=0.8pt,arrowsize=3pt 2,arrowinset=0.25}
\begin{pspicture*}(1.64,0.96)(7.78,5)
\pspolygon(2,2)(7,2)(7,4)(4,4)
\pspolygon(7,2)(6.6,2)(6.6,2.4)(7,2.4)
\pspolygon(7,4)(7,3.6)(6.6,3.6)(6.6,4)
\psline(2,2)(7,2)
\psline(7,2)(7,4)
\psline(7,4)(4,4)
\psline(4,4)(2,2)
\psline(4,4)(2,2)
\psline(4,4)(4,2)
\pspolygon[linewidth=1.pt,linecolor=xfqqff,fillcolor=xfqqff!20!white,fillstyle=solid,opacity=0.1](4.,2.4242640687119286)(3.5757359312880714,2.4242640687119286)(3.5757359312880714,2.)(4.,2.)
\psline{->}(5,3)(5,4)
\psline{->}(5,3)(5,2)
\rput[tl](5.22,3.2){$2$}
\rput[tl](5.44,4.4){$3$}
\rput[tl](5.44,1.8){$3$}
\rput[tl](2.84,1.8){$2$}
\rput[bl](1.8,2.18){$A$}
\rput[bl](7.08,2.12){$B$}
\rput[bl](7.08,4.12){$C$}
\rput[bl](4.08,4.12){$D$}
\rput[bl](3.8,1.6){$H$}
\end{pspicture*}
\end{center}

Le trapèze est constitué~:

\begin{itemize}
\item[\textbullet] d'un rectangle $BHDC,$ d'aire $\ell\times L=3\times 2=6~;$
\item[\textbullet] d'un triangle $AHD,$ d'aire $\frac{B\times h}{2}=\frac{2\times 2}{2}=2.$
\end{itemize}
Donc l'aire du trapèze est $6+2=8.$

\medskip

\textbf{Remarque~:} On peut aussi utiliser la formule (hors-programme)~:
\[\mathcal{A}_{\text{trapèze}}=\frac{(B+b)\times h}{2}=\frac{(5+3)\times 2}{2}=8.\]
\end{exo}

\begin{exo}

Le losange est \og la moitié \fg~{} d'un rectangle de côtés $\ell$ et $L,$ donc son aire est $\frac{\ell\times L}{2}.$



\begin{center}
\psset{xunit=1.0cm,yunit=1.0cm,algebraic=true,dimen=middle,dotstyle=o,dotsize=5pt 0,linewidth=1.6pt,arrowsize=3pt 2,arrowinset=0.25}
\begin{pspicture*}(0.0,2.0)(5.24,5.24)
\psline[linewidth=2.pt](1.,4.)(3.,5.)
\psline[linewidth=2.pt](3.,5.)(5.,4.)
\psline[linewidth=2.pt](5.,4.)(3.,3.)
\psline[linewidth=2.pt](3.,3.)(1.,4.)
\psline[linewidth=2.pt,linestyle=dotted](1.,4.)(5.,4.)
\psline[linewidth=2.pt,linestyle=dotted](3.,5.)(3.,3.)
\psline[linewidth=2.pt,linestyle=dotted,linecolor=red](1.,3.)(1.,5.)
\psline[linewidth=2.pt,linestyle=dotted,linecolor=red](1.,3.)(5.,3.)
\psline[linewidth=2.pt,linestyle=dotted,linecolor=red](1.,5.)(5.,5.)
\psline[linewidth=2.pt,linestyle=dotted,linecolor=red](5.,5.)(5.,3.)

\rput[tl](0.65,4.2){\red{$\ell$}}
\rput[tl](2.75,2.86){\red{$L$}}
\end{pspicture*}
\end{center}
\end{exo}



\begin{exo}%29

\textbf{Rappels~:}
\begin{itemize}
\item[\textbullet] une hauteur est une droite qui passe par un sommet et qui est perpendiculaire au côté opposé (les hauteurs sont tracées en pointillés bleus)~;
\item[\textbullet] le fait que les hauteurs soient \og concourantes \fg~{} signifie qu'elles passent toutes les trois par un même point -- qu'on appelle \og orthocentre du triangle \fg~{} (nommé $O$ sur la figure ci-dessous).
\end{itemize}


\begin{center}
\newrgbcolor{xfqqff}{0.4980392156862745 0. 1.}
\psset{xunit=1.0cm,yunit=1.0cm,algebraic=true,dimen=middle,dotstyle=o,dotsize=5pt 0,linewidth=2.pt,arrowsize=3pt 2,arrowinset=0.25}
\begin{pspicture*}(0.58,0.54)(7.34,5.4)
\pspolygon[linewidth=2.pt,linecolor=xfqqff,fillcolor=xfqqff!20!white,fillstyle=solid,opacity=0.1](1.9037508537434635,2.9121465431835385)(2.287329637311028,2.730853088263545)(2.4686230922310215,3.1144318718311097)(2.085044308663457,3.295725326751103)
\pspolygon[linewidth=2.pt,linecolor=xfqqff,fillcolor=xfqqff!20!white,fillstyle=solid,opacity=0.1](3.66,3.66)(3.96,3.36)(4.26,3.66)(3.96,3.96)
\pspolygon[linewidth=2.pt,linecolor=xfqqff,fillcolor=xfqqff!20!white,fillstyle=solid,opacity=0.1](3.3378667606356793,1.0079249720699515)(3.3364285856148856,1.4321866032041017)(2.9121669544807354,1.430748428183308)(2.913605129501529,1.0064867970491578)
\psline[linewidth=2.pt](1.,1.)(6.9,1.02)
\psline[linewidth=2.pt](6.9,1.02)(2.9,5.02)
\psline[linewidth=2.pt](2.9,5.02)(1.,1.)
\psline[linewidth=2.pt,linestyle=dotted,linecolor=blue](2.9,5.02)(2.913605129501529,1.0064867970491578)
\psline[linewidth=2.pt,linestyle=dotted,linecolor=blue](1.,1.)(3.96,3.96)
\psline[linewidth=2.pt,linestyle=dotted,linecolor=blue](6.9,1.02)(2.085044308663457,3.295725326751103)
\psdots[dotsize=4pt 0,dotstyle=*,linecolor=blue](2.907162162162163,2.907162162162163)
\rput[bl](3.06,2.42){\blue{$O$}}
\end{pspicture*}
\end{center}

\end{exo}







\begin{exo}%34

On note $H$ le pied de la hauteur issue de $A$ dans le triangle $ABC.$


\begin{center}
\psset{xunit=1.0cm,yunit=1.0cm,algebraic=true,dimen=middle,dotstyle=o,dotsize=5pt 0,linewidth=2.pt,arrowsize=3pt 2,arrowinset=0.25}
\begin{pspicture*}(-0.02,0.32)(5.98,4.56)
\pspolygon[linewidth=2.pt,linecolor=red,fillcolor=red!10!white,fillstyle=solid,opacity=0.1](2.4242640687119286,1.)(2.4242640687119286,1.4242640687119286)(2.,1.4242640687119286)(2.,1.)
\psline[linewidth=2.pt](2.,4.)(5.,1.)
\psline[linewidth=2.pt](2.,4.)(1.,1.)
\psline[linewidth=2.pt](1.,1.)(5.,1.)
\psline[linewidth=2.pt,linestyle=dotted,linecolor=red](2.,4.)(2.,1.)
\psline[linewidth=2.pt](3.,1.)(2.,4.)
\rput[bl](2.08,4.04){$A$}
\rput[bl](5.,0.6){$B$}
\rput[bl](0.7,0.6){$C$}
\rput[bl](1.96,0.6){$H$}
\rput[bl](3.04,0.6){$I$}
\end{pspicture*}
\end{center}


$\left[AH\right]$ est une hauteur dans les triangles $BIA$ et $CIA,$ donc

\[\mathcal{A}_{BIA}=\frac{\textcolor{red}{BI}\times AH}{2}\hspace{4cm} \mathcal{A}_{CIA}=\frac{\textcolor{red}{CI}\times AH}{2}.\]
Or $\textcolor{red}{BI}=\textcolor{red}{CI}$ puisque $I$ et le milieu de $\left[BC\right],$ donc $BIA$ et $CIA$ ont la même aire.

\end{exo}


\newpage


\begin{exo}%30


\begin{enumerate}
\item La négation de

\begin{center}

 \blue{\underline{Tous}} \black les hommes \blue{\underline{sont mortels}}\black .
 
 \end{center}
 
 est
 
 \begin{center}

 \red{\underline{Il existe}} \black un homme \red{\underline{immortel}}\black .
 
 \end{center}
 
\item La négation de

\begin{center}

 \blue{\underline{Il existe}} \black un dessert \blue{\underline{sans sucre}} \black à la cantine.
 
 \end{center}
 
 est
 
 \begin{center}

 \red{\underline{Tous}} \black les desserts \red{\underline{sont sucrés}} \black à la cantine.
 \end{center}
 
 \medskip

\textbf{Remarque~:} Dans les deux exemples que nous venons de traiter, pour écrire la négation d'une phrase, il suffit de remplacer les \og tous \fg~{} par \og il existe \fg~{}, et réciproquement~; et d'inverser les conclusions (exemple~: immortel/mortel). C'est une technique qui fonctionne toujours.
\item La négation de

\begin{center}

 \blue{\underline{Il existe}} \black un pays dans lequel \blue{\underline{tous}} \black les hommes \black \blue{\underline{savent lire}}\black .
 
 \end{center}
 
 est
 
 \begin{center}

Dans \red{\underline{tous}} \black les pays, \red{\underline{il existe}} \black un homme qui  \red{\underline{ne sait pas lire}}\black .
 \end{center}
\item Le contraire de \og être allé en Angleterre ou en Espagne \fg~{} est \og n'être allé ni en Angleterre, ni en Espagne \fg , donc la négation de

\begin{center}

 \blue{\underline{Tous}} \black les élèves de la classe  \blue{\underline{sont déjà allés en Angleterre ou en Espagne }}\black .
 
 \end{center}
 
 est
 
 \begin{center}

 \red{\underline{Il existe}} \black un élève de la classe qui \red{\underline{n'est jamais allé en Angleterre, ni en Espagne}}\black .
 \end{center}
\item Comme dans l'exemple précédent, le contraire de \og ni... ni... \fg~{} est \og ou \fg . Donc la négation de

\begin{center}

 Chloé \blue{\underline{n'aime ni les fraises, ni les framboises}}\black .
 
 \end{center}
 
 est
 
 \begin{center}

 Chloé \red{\underline{aime les fraises ou les framboises}}\black .
 \end{center}

\end{enumerate}



\end{exo}


\begin{exo}%19


\begin{enumerate}
\item \begin{enumerate}
\item On identifie A et B dans l'implication~:

\begin{center}

Si $\underbrace{\text{un nombre se termine par 5}}_{\text{A}}$, alors $\underbrace{\text{il est multiple de 5}}_{\text{B}}.$

\end{center}

Cette implication est vraie (cours du primaire).

\item

\begin{itemize}
\item[\textbullet] L'implication contraposée est

\begin{center}

Si $\underbrace{\text{un nombre n'est pas multiple de 5}}_{\text{non B}}$, alors $\underbrace{\text{il ne se termine pas par 5}}_{\text{non A}}.$

\end{center}

Cette contraposée est vraie, puisque l'implication originale l'est (cf l'énoncé~: quand une implication est vraie, sa contraposée l'est aussi).

\medskip

\item[\textbullet] L'implication réciproque est

\begin{center}

Si $\underbrace{\text{un nombre est multiple de 5}}_{\text{B}}$, alors $\underbrace{\text{il se termine par 5}}_{\text{A}}.$

\end{center}

Elle est fausse, comme le montre le contre-exemple suivant~: 10 est multiple de 5, mais il ne se termine pas par 5.
\end{itemize}

\end{enumerate}
\item L'implication

\begin{center}

Si $\underbrace{\text{un nombre se termine par 0}}_{\text{A}}$, alors $\underbrace{\text{il est multiple de 10}}_{\text{B}}.$

\end{center}

et sa réciproque

\begin{center}

Si $\underbrace{\text{un nombre est multiple de 10}}_{\text{B}}$, alors $\underbrace{\text{il se termine par 0}}_{\text{A}}.$

\end{center}

sont vraies toutes les deux.

\end{enumerate}
\end{exo}

\begin{exo}

Soit $ABC$ un triangle
\begin{enumerate}
\item \textbf{Théorème de Pythagore.}
\begin{center}


Si $ABC$ est rectangle en $A,$ alors $BC^2=AB^2+BC^2.$

\begin{center}
\newrgbcolor{xfqqff}{0.4980392156862745 0. 1.}
\psset{xunit=1.0cm,yunit=1cm,algebraic=true,dimen=middle,dotstyle=o,dotsize=5pt 0,linewidth=2.pt,arrowsize=3pt 2,arrowinset=0.25}
\begin{pspicture*}(2.12,0.46)(7.88,4.24)
\pspolygon[linewidth=2.pt,linecolor=xfqqff,fillcolor=xfqqff!20!white,fillstyle=solid,opacity=0.1](3.4242640687119286,1.)(3.4242640687119286,1.4242640687119286)(3.,1.4242640687119286)(3.,1.)
\psline[linewidth=2.pt](7.,1.)(3.,1.)
\psline[linewidth=2.pt](3.,1.)(3.,4.)
\psline[linewidth=2.pt](3.,4.)(7.,1.)
\rput[tl](2.6,4){$C$}
\rput[tl](2.6,1){$A$}
\rput[tl](7.2,1.1){$B$}
\end{pspicture*}
\end{center}

\end{center}
\item \textbf{Théorème contraposé de Pythagore.}

\begin{center}

Si $BC^2\not=AB^2+BC^2,$ alors $ABC$ n'est pas rectangle en $A.$
\end{center}
\item \textbf{Théorème réciproque de Pythagore.}

\begin{center}
Si $BC^2=AB^2+BC^2,$ alors $ABC$ est  rectangle en $A.$
\end{center}

Le théorème réciproque est bien sûr vrai, comme vous l'avez appris au collège.

\medskip

\danger En devoir, le correcteur sera très attentif au nom du théorème utilisé dans les démonstrations~: théorème, théorème contraposé ou théorème réciproque -- il ne faudra pas confondre~!
\end{enumerate}

\end{exo}

\begin{exo}

\begin{enumerate}
\item Pour construire la figure, on trace successivement~:

\begin{itemize}
\item[\textbullet] Le segment $\left[EF\right].$
\item[\textbullet] La perpendiculaire à $\left[EF\right]$ passant par $E.$
\item[\textbullet] Un arc de cercle de centre $F,$ de rayon 7~cm. Il coupe la perpendiculaire que nous venons de tracer en $G.$
\end{itemize}

\setlength{\columnseprule}{1pt}
\begin{multicols}{2}
\begin{center}
\newrgbcolor{ududff}{0.30196078431372547 0.30196078431372547 1.}
\newrgbcolor{xfqqff}{0.4980392156862745 0. 1.}
\psset{xunit=0.8cm,yunit=0.8cm,algebraic=true,dimen=middle,dotstyle=o,dotsize=5pt 0,linewidth=2.pt,arrowsize=3pt 2,arrowinset=0.25}
\begin{pspicture*}(-2.94,-0.7)(5.42,6.16)
\pspolygon[linewidth=2.pt,linecolor=xfqqff,fillcolor=xfqqff!20!white,fillstyle=solid,opacity=0.1](-0.5757359312880717,0.)(-0.5757359312880717,0.4242640687119283)(-1.,0.4242640687119283)(-1.,0.)
\psline[linewidth=2.pt](-1.,0.)(4.,0.)
\rput[tl](1.,-0.2){$5~\text{cm}$}
\psline[linewidth=2.pt](-1.,-0.7)(-1.,6.16)
\parametricplot[linewidth=2.pt]{2.1976990745518967}{2.5472509907369547}{1.*7.*cos(t)+0.*7.*sin(t)+4.|0.*7.*cos(t)+1.*7.*sin(t)+0.}
\psline[linewidth=2.pt](4.,0.)(-1.,4.898979485566355)
\rput[tl](1.5,2.82){$7~\text{cm}$}
\rput[bl](-1.5,-0.3){{$E$}}
\rput[bl](4.14,-0.3){{$F$}}
\rput[bl](-1.44,5.04){{$G$}}
\end{pspicture*}
\end{center}

D'après \textbf{le théorème de Pythagore} dans $EFG$ rectangle en $E~:$
\begin{align*}
FG^2&=EF^2+EG^2\\
7^2&=5^2+EG^2\\
49&=25+EG^2\\
49-25&=EG^2\\
\sqrt{24}&=EG
\end{align*}

Conclusion~: $EG=\sqrt{24}~\text{cm}.$

\medskip

\danger Sauf si l'énoncé le demande, ne donnez pas de valeur approchée.

\end{multicols}
\item Le plus grand côté est $\left[BC\right],$ donc le triangle ne pourrait être rectangle qu'en $A.$

On calcule~:
\[
\left.
    \begin{array}{ll}
        BC^2=6^2=36\\
        AB^2+AC^2=5^2+4^2=25+16=41
    \end{array}
\right \}BC^2\not=AB^2+AC^2.
\]
D'après \textbf{la contraposée du théorème de Pythagore}, $ABC$ n'est pas rectangle en $A.$

\end{enumerate}

\end{exo}


\begin{exo}


$ABCDEFGH$ est un parallélépipède rectangle tel que $AB=BC=6$ et $CG=3.$
\begin{center}
\newrgbcolor{ududff}{0.30196078431372547 0.30196078431372547 1.}
\newrgbcolor{xfqqff}{0.4980392156862745 0. 1.}
\psset{xunit=1.0cm,yunit=1.0cm,algebraic=true,dimen=middle,dotstyle=o,dotsize=5pt 0,linewidth=2.pt,arrowsize=3pt 2,arrowinset=0.25}
\begin{pspicture*}(-3.3774236991656137,-0.31214840896786467)(5.5552457552212475,4.59336208917407)
\pspolygon[linewidth=0.pt,linecolor=xfqqff,fillcolor=xfqqff!20!white,fillstyle=solid,opacity=0.1](-2.6,0.2)(-2.2,0.2)(-2.6,0.)(-3.,0.)
\pspolygon[linewidth=0.pt,linecolor=xfqqff,fillcolor=xfqqff!20!white,fillstyle=solid,opacity=0.1](2.6046541583780107,0.09883646040549732)(3.,0.)(3.,0.4)(2.6049879803510483,0.5150893759578578)
\psline[linewidth=2.pt,linestyle=dashed,dash=2pt 2pt](-1.,1.)(-3.,0.)
\psline[linewidth=2.pt](-3.,0.)(3.,0.)
\psline[linewidth=2.pt](3.,0.)(5.,1.)
\psline[linewidth=2.pt,linestyle=dashed,dash=2pt 2pt](5.,1.)(-1.,1.)
\psline[linewidth=2.pt](-1.,4.)(-3.,3.)
\psline[linewidth=2.pt](-3.,3.)(3.,3.)
\psline[linewidth=2.pt](3.,3.)(5.,4.)
\psline[linewidth=2.pt](5.,4.)(-1.,4.)
\psline[linewidth=2.pt](-3.,3.)(-3.,0.)
\psline[linewidth=2.pt,linestyle=dashed,dash=2pt 2pt](-1.,4.)(-1.,1.)
\psline[linewidth=2.pt](3.,3.)(3.,0.)
\psline[linewidth=2.pt](5.,4.)(5.,1.)
\psline[linewidth=2.pt,linestyle=dashed,dash=2pt 2pt](-1.,1.)(3.,3.)
\psline[linewidth=2.pt,linestyle=dashed,dash=2pt 2pt](-1.,1.)(3.,0.)
\psline[linewidth=2.pt,linecolor=ududff](-2.6,0.2)(-2.2,0.2)
\psline[linewidth=2.pt,linecolor=ududff](-2.2,0.2)(-2.6,0.)
\psline[linewidth=2.pt,linecolor=ududff](2.6046541583780107,0.09883646040549732)(2.6049879803510483,0.5150893759578578)
\psline[linewidth=2.pt,linecolor=ududff](2.6049879803510483,0.5150893759578578)(3.,0.4)
\rput[tl](-2.018464709815479,0.8976565449657882){6}
\rput[tl](0.01997377420972306,0.3){6}
\rput[tl](3.1356358473539347,1.6599993926500078){3}
\rput[bl](-1.35,1.1628192745950818){{$A$}}
\rput[bl](-2.9299615929161793,0.2){{$B$}}
\rput[bl](3.069345164946611,0.16845903848523028){{$C$}}
\rput[bl](5.058065637166321,1.1628192745950818){{$D$}}
\rput[bl](-0.9412411206964699,4.162472653526468){{$E$}}
\rput[bl](-2.9299615929161793,3.168112417416616){{$F$}}
\rput[bl](2.95,3.168112417416616){{$G$}}
\rput[bl](5.058065637166321,4.162472653526468){{$H$}}
\end{pspicture*}
\end{center}

On utilise deux fois de suite le théorème de Pythagore~:

\setlength{\columnseprule}{1pt}
\begin{multicols}{2}
 Dans $ABC$ rectangle en $B,$
 
 \begin{align*}
AC^2&=AB^2+BC^2\\
AC^2&=6^2+6^2\\
AC^2&=36+36\\
AC^2&=72\\
(\text{Inutile de} & \text{ donner} AC~!)
\end{align*}

\columnbreak

 Dans $ACG$ rectangle en $C,$
 
 \begin{align*}
AG^2&=AC^2+CG^2\\
AG^2&=72+3^2\\
AG^2&=72+9\\
AG^2&=81\\
AG&=\sqrt{81}=9
\end{align*}

\end{multicols}




Conclusion~: $AG=9.$

\end{exo}

\begin{exo}
Sur la figure ci-dessous (qui n'est pas à l'échelle), le segment $\left[MK\right]$ mesure 3~cm, le segment $\left[MN\right]$ mesure 5~cm et $h=1,2$~cm.%l'aire du triangle $MNP$ est égale à $2,5~\text{cm}^2.$

\begin{center}
\psset{xunit=1.cm,yunit=1.cm,algebraic=true,dimen=middle,dotstyle=o,dotsize=5pt 0,linewidth=2.pt,arrowsize=3pt 2,arrowinset=0.25}
\begin{pspicture*}(0.34,0.28)(6.18,4.64)
\pspolygon[linewidth=2.pt,fillcolor=black!20!white,fillstyle=solid,opacity=0.1](1.424264068711929,1.)(1.424264068711929,1.424264068711929)(1.,1.424264068711929)(1.,1.)
\pspolygon[linewidth=2.pt,fillcolor=black!20!white,fillstyle=solid,opacity=0.1](3.465441558772843,1.6205887450304568)(3.804852813742386,1.3660303038032993)(4.059411254969543,1.7054415587728424)(3.72,1.96)
\psline[linewidth=2.pt](1.,1.)(5.,1.)
\psline[linewidth=2.pt](1.,1.)(1.,4.)
\psline[linewidth=2.pt](1.,4.)(5.,1.)
\psline[linewidth=2.pt](1.,4.)(3.,1.)
\psline[linewidth=2.pt](3.,1.)(3.72,1.96)
\rput[tl](3.1,1.75){$h$}
\psdots[dotsize=1pt 0,dotstyle=*](1.,1.)
\rput[bl](0.62,0.96){$K$}
\psdots[dotsize=1pt 0,dotstyle=*](5.,1.)
\rput[bl](5.1,0.76){$N$}
\psdots[dotsize=1pt 0,dotstyle=*](1.,4.)
\rput[bl](1.12,4.08){$M$}
\psdots[dotsize=1pt 0,dotstyle=*](3.,1.)
\rput[bl](2.8,0.68){$P$}
\end{pspicture*}
\end{center}

\begin{enumerate}
\item $\mathcal{A}_{MNP}=\frac{MN\times h}{2}=\frac{5\times 1,2}{2}=3~\text{cm}^2.$
\item On a aussi $\mathcal{A}_{MNP}=\frac{PN\times MK}{2},$ donc $3=\frac{PN\times 3}{2},$ soit $\cancel{3}\times 2=PN\times \cancel{3}~;$ et donc $PN=2~\text{cm}.$
\item (Non détaillé.) Il faut calculer successivement $KN,$ puis $KP$ et $MP.$

\danger On ne sait pas, à ce stade, que $P$ est le milieu de $\left[KN\right].$
\begin{itemize}
\item[\textbullet] Pour $KN,$ on utilise le théorème de Pythagore dans le triangle $KMN.$ On obtient $KN=4~\text{cm}.$
\item[\textbullet] $KP=KN-PN=4-2=2~\text{cm}.~~~$ %(\danger N'utilisez pas le fait que $P$ est le milieu de $\left[KN\right]$ pour calculer $KP,$ car vous n'en savez rien.)
\item[\textbullet] Enfin, pour calculer $PM,$ on utilise le théorème de Pythagore dans le triangle $KMP.$ On obtient $MP=\sqrt{13}~\text{cm}.$
\end{itemize}
\end{enumerate}

\end{exo}

\begin{exo}
\begin{enumerate}
\item Les côtés de l'angle droit d'un triangle rectangle mesurent $a$ et $b,$ l'hypoténuse mesure $c.$


\begin{center}
\newrgbcolor{xfqqff}{0.4980392156862745 0. 1.}
\psset{xunit=1.0cm,yunit=1.0cm,algebraic=true,dimen=middle,dotstyle=o,dotsize=5pt 0,linewidth=2.pt,arrowsize=3pt 2,arrowinset=0.25}
\begin{pspicture*}(2.12,0.46)(7.88,4.24)
\pspolygon[linewidth=2.pt,linecolor=xfqqff,fillcolor=xfqqff!20!white,fillstyle=solid,opacity=0.1](3.4242640687119286,1.)(3.4242640687119286,1.4242640687119286)(3.,1.4242640687119286)(3.,1.)
\psline[linewidth=2.pt](7.,1.)(3.,1.)
\psline[linewidth=2.pt](3.,1.)(3.,4.)
\psline[linewidth=2.pt](3.,4.)(7.,1.)
\rput[tl](5.2,2.62){$c$}
\rput[tl](4.62,0.9){$a$}
\rput[tl](2.45,2.5){$b$}
\end{pspicture*}
\end{center}

D'après le théorème de Pythagore, $c^2=a^2+b^2,$ donc \[c=\sqrt{a^2+b^2}.\]

\item L'affirmation 
\begin{center}
 Pour tous nombres positifs $a$ et $b,$ $\sqrt{a^2+b^2}=a+b.$ 
\end{center}

est FAUSSE~! Voici deux justifications~:

\begin{itemize}
\item[\textbullet] \textbf{Par le calcul.} Il suffit de donner un contre-exemple~: on choisit $a=4$ et $b=3.$ Dans ce cas
\[\sqrt{a^2+b^2}=\sqrt{4^2+3^2}=\sqrt{16+9}=\sqrt{25}=5\qquad \text{est différent de}\qquad a+b=4+3=7.\]
\item[\textbullet] \textbf{Géométriquement.} $\sqrt{a^2+b^2}$ est la longueur de l'hypoténuse $c$ du triangle rectangle de la question 1~; tandis que $a+b$ est la somme des longueurs des côtés de l'angle droit. Or cette somme est strictement plus grande que celle de l'hypoténuse, puisque le chemin le plus court d'un point à un autre est la ligne droite.
\end{itemize}
\end{enumerate}
\end{exo}





\begin{exo}

Soit $A$ un point et $\Delta$ une droite du plan. Le projeté orthogonal de $A$ sur $\Delta$ est le point $H$ de $\Delta$ tel que $(AH)\perp\Delta.$

\begin{enumerate}
\item On trace la perpendiculaire à $\Delta$ passant par $A.$ Elle coupe $\Delta$ en $H.$


\begin{center}
\newrgbcolor{ududff}{0.30196078431372547 0.30196078431372547 1.}
\newrgbcolor{xfqqff}{0.4980392156862745 0. 1.}
\psset{xunit=1.0cm,yunit=1.0cm,algebraic=true,dimen=middle,dotstyle=o,dotsize=5pt 0,linewidth=2.pt,arrowsize=3pt 2,arrowinset=0.25}
\begin{pspicture*}(-2.6,-0.64)(3.82,3.76)
\pspolygon[linewidth=2.pt,linecolor=xfqqff,fillcolor=xfqqff!20!white,fillstyle=solid,opacity=0.1](1.3760676226484692,0.807765180139509)(1.2347599047685438,1.2078053392080301)(0.8347197457000228,1.0664976213281048)(0.9760274635799481,0.6664574622595838)
\psplot[linewidth=2.pt]{-2.6}{3.82}{(--1.2932--1.42*x)/4.02}
\psplot[linewidth=2.pt,linestyle=dashed,dash=2pt 2pt,linecolor=red]{-2.6}{3.82}{(-4.87--4.02*x)/-1.42}
\rput[tl](-1.52,0.3){$\Delta$}
\psdots[dotstyle=*,linecolor=ududff](0.18,2.92)
\rput[bl](0.26,3.12){\ududff{$A$}}
\psdots[dotstyle=*,linecolor=ududff](0.9760274635799481,0.6664574622595838)
\rput[bl](0.7,0.22){\ududff{$H$}}
\end{pspicture*}
\end{center}
\item Par construction, le triangle $AMH$ est rectangle en $H,$ donc son hypoténuse $AM$ est strictement plus grande que le côté de l'angle droit $AH$ (c'est le même raisonnement que celui de l'exercice précédent)~:
\[AM>AH.\]

\begin{center}
\newrgbcolor{ududff}{0.30196078431372547 0.30196078431372547 1.}
\newrgbcolor{xfqqff}{0.4980392156862745 0. 1.}
\psset{xunit=1.0cm,yunit=1.0cm,algebraic=true,dimen=middle,dotstyle=o,dotsize=5pt 0,linewidth=2.pt,arrowsize=3pt 2,arrowinset=0.25}
\begin{pspicture*}(-2.6,-0.64)(4.18,3.76)
\pspolygon[linewidth=2.pt,linecolor=xfqqff,fillcolor=xfqqff!20!white,fillstyle=solid,opacity=0.1](1.3760676226484692,0.8077651801395092)(1.2347599047685438,1.2078053392080301)(0.8347197457000228,1.0664976213281048)(0.9760274635799481,0.6664574622595838)
\psplot[linewidth=2.pt]{-2.6}{4.18}{(--1.2932--1.42*x)/4.02}
\psplot[linewidth=2.pt,linestyle=dashed,dash=2pt 2pt,linecolor=red]{-2.6}{4.18}{(-4.87--4.02*x)/-1.42}
\rput[tl](-1.52,0.3){$\Delta$}
\psline[linewidth=2.pt](2.862531138594252,1.332834382289512)(0.18,2.92)
\psdots[dotstyle=*,linecolor=ududff](0.18,2.92)
\rput[bl](0.26,3.12){\ududff{$A$}}
\psdots[dotstyle=*,linecolor=ududff](0.9760274635799481,0.6664574622595838)
\rput[bl](0.7,0.22){\ududff{$H$}}
\psdots[dotstyle=*,linecolor=ududff](2.862531138594252,1.332834382289512)
\rput[bl](2.94,1.54){\ududff{$M$}}
\end{pspicture*}
\end{center}


\item Le segment $\left[AH\right]$ est la hauteur\footnote{Le mot \textit{hauteur} est polysémique (il a plusieurs sens)~: le segment $\left[AH\right]$ peut être appelé \textit{hauteur}, la droite $\left(AH\right)$ peut également être appelée \textit{hauteur}~; enfin la longueur $AH$ peut elle aussi être appelée \textit{hauteur} -- c'est cette longueur, par exemple, que l'on retrouve dans la formule $\frac{B\times h}{2}$ pour l'aire du triangle.} issue de $A$ dans le triangle $ABC.$


\begin{center}
\newrgbcolor{ududff}{0.30196078431372547 0.30196078431372547 1.}
\newrgbcolor{xfqqff}{0.4980392156862745 0. 1.}
\newrgbcolor{xdxdff}{0.49019607843137253 0.49019607843137253 1.}
\psset{xunit=1.0cm,yunit=1.0cm,algebraic=true,dimen=middle,dotstyle=o,dotsize=5pt 0,linewidth=2.pt,arrowsize=3pt 2,arrowinset=0.25}
\begin{pspicture*}(-2.6,-0.64)(4.18,3.76)
\pspolygon[linewidth=2.pt,linecolor=xfqqff,fillcolor=xfqqff!20!white,fillstyle=solid,opacity=0.1](1.3760676226484692,0.8077651801395092)(1.2347599047685438,1.2078053392080301)(0.8347197457000228,1.0664976213281048)(0.9760274635799481,0.6664574622595838)
\psline[linewidth=2.pt](2.862531138594252,1.332834382289512)(0.18,2.92)
\psline[linewidth=2.pt](0.18,2.92)(-1.2704379208661591,-0.12707011135073243)
\psline[linewidth=2.pt](-1.2704379208661591,-0.12707011135073243)(2.862531138594252,1.332834382289512)
\psline[linewidth=2.pt,linestyle=dashed,dash=2pt 2pt,linecolor=red](0.9760274635799481,0.6664574622595838)(0.18,2.92)
\psdots[dotstyle=*,linecolor=ududff](0.18,2.92)
\rput[bl](0.26,3.12){\ududff{$A$}}
\psdots[dotstyle=*,linecolor=ududff](0.9760274635799481,0.6664574622595838)
\rput[bl](0.7,0.22){\ududff{$H$}}
\psdots[dotstyle=*,linecolor=ududff](2.862531138594252,1.332834382289512)
\rput[bl](2.94,1.54){\ududff{$C$}}
\psdots[dotstyle=*,linecolor=xdxdff](-1.2704379208661591,-0.12707011135073243)
\rput[bl](-1.65,0.04){\xdxdff{$B$}}
\end{pspicture*}
\end{center}



\end{enumerate}
\end{exo}


\newpage

\begin{exo}


On résout les équations~:

\setlength{\columnseprule}{1pt}
\begin{multicols}{5}
\begin{align*}
x+7&=18\\
x+\cancel{7}\textcolor{red}{-\cancel{7}}&=18\textcolor{red}{-7}\\
x&=11
\end{align*}

La solution est $x=11$

\columnbreak

\begin{align*}
3x+4&=19\\
3x+\cancel{4}\textcolor{red}{-\cancel{4}}&=19\textcolor{red}{-4}\\
3x&=15\\
\frac{\cancel{3}x}{\textcolor{blue}{\cancel{3}}}&=\frac{15}{\textcolor{blue}{3}}\\
x&=5
\end{align*}

La solution est $x=5.$

\columnbreak

\begin{align*}
3,5x-9&=5\\
3,5x-\cancel{9}\textcolor{red}{+\cancel{9}}&=5\textcolor{red}{+9}\\
3,5x&=14\\
\frac{\cancel{3,5}x}{\textcolor{blue}{\cancel{3,5}}}&=\frac{14}{\textcolor{blue}{3,5}}\\
x&=\frac{14}{3,5}
\end{align*}

Or $\frac{14}{3,5}=\frac{14\times 2}{3,5\times 2}=\frac{28}{7}=4,$ donc la solution est $x=4.$

 
 \columnbreak
 
 \begin{align*}
x+1&=-2x-5\\
x+1\textcolor{green}{+2x}&=\cancel{-2x}-5\textcolor{green}{+\cancel{2x}}\\
3x+1&=-5\\
3x+\cancel{1}\textcolor{red}{-\cancel{1}}&=-5\textcolor{red}{-{1}}\\
3x&=-6\\
\frac{\cancel{3}x}{\textcolor{blue}{\cancel{3}}}&=\frac{-6}{\textcolor{blue}{3}}\\
x&=-2
\end{align*}

La solution est $x=-2.$

 \columnbreak
 
 \begin{align*}
-2x+4&=3x-6\\
-2x+4\textcolor{green}{-3x}&=\cancel{3x}-6\textcolor{green}{-\cancel{3x}}\\
-5x+4&=-6\\
-5x+\cancel{4}\textcolor{red}{-\cancel{4}}&=-6\textcolor{red}{-4}\\
-5x&=-10\\
\frac{\cancel{-5}x}{\textcolor{blue}{\cancel{-5}}}&=\frac{-10}{\textcolor{blue}{-5}}\\
x&=2
\end{align*}

La solution est $x=2.$

\end{multicols}

\end{exo}



\begin{exo}

Les deux plateaux de la balance ci-dessous sont en équilibre. Les poids noirs ont tous la même masse $M$~kg.


\begin{center}
\psset{xunit=1cm,yunit=0.8cm,algebraic=true,dimen=middle,dotstyle=o,dotsize=3pt 0,linewidth=0.8pt,arrowsize=3pt 2,arrowinset=0.25}
\begin{pspicture*}(-0.12,0.82)(6.12,4.7)
\pspolygon[fillcolor=black,fillstyle=solid,opacity=1.0](5.29,3.17)(5.29,3.67)(5.79,3.67)(5.79,3.17)
\pspolygon[fillcolor=black,fillstyle=solid,opacity=1.0](0.09,3.13)(0.09,3.63)(0.59,3.63)(0.59,3.13)
\pspolygon[fillcolor=black,fillstyle=solid,opacity=1.0](0.8,3.13)(0.8,3.63)(1.3,3.63)(1.3,3.13)
\pspolygon[fillcolor=black,fillstyle=solid,opacity=1.0](1.48,3.14)(1.48,3.64)(1.98,3.64)(1.98,3.14)
\pspolygon(4.04,4.19)(4.84,4.2)(4.84,3.19)(4.04,3.19)
\pspolygon(0.5,4.5)(0.5,4)(1.5,4)(1.5,4.5)
\psline[linewidth=2pt](3,1)(1,3)
\psline[linewidth=2pt](3,1)(5,3)
\psline[linewidth=2pt](0,3)(2,3)
\psline[linewidth=2pt](4,3)(6,3)
\psline(5.29,3.17)(5.29,3.67)
\psline(5.29,3.67)(5.79,3.67)
\psline(5.79,3.67)(5.79,3.17)
\psline(5.79,3.17)(5.29,3.17)
\psline(0.09,3.13)(0.09,3.63)
\psline(0.09,3.63)(0.59,3.63)
\psline(0.59,3.63)(0.59,3.13)
\psline(0.59,3.13)(0.09,3.13)
\psline(0.8,3.13)(0.8,3.63)
\psline(0.8,3.63)(1.3,3.63)
\psline(1.3,3.63)(1.3,3.13)
\psline(1.3,3.13)(0.8,3.13)
\psline(1.48,3.14)(1.48,3.64)
\psline(1.48,3.64)(1.98,3.64)
\psline(1.98,3.64)(1.98,3.14)
\psline(1.98,3.14)(1.48,3.14)
\psline(4.04,4.19)(4.84,4.2)
\psline(4.84,4.2)(4.84,3.19)
\psline(4.84,3.19)(4.04,3.19)
\psline(4.04,3.19)(4.04,4.19)
\rput[tl](4.05,3.82){10 kg}
\psline(0.5,4.5)(0.5,4)
\psline(0.5,4)(1.5,4)
\psline(1.5,4)(1.5,4.5)
\psline(1.5,4.5)(0.5,4.5)
\rput[tl](0.75,4.42){7 kg}
\end{pspicture*}
\end{center}


Le fait que la balance soit en équilibre se traduit par l'équation
\[3M+7=10+M.\] On la résout~:

\begin{align*}
3M+7\textcolor{green}{-M}&=10+\cancel{M}\textcolor{green}{-\cancel{M}}\\
2M+7&=10\\
2M+\cancel{7}\textcolor{red}{-\cancel{7}}&=10\textcolor{red}{-7}\\
2M&=3\\
\frac{\cancel{2}M}{\textcolor{blue}{\cancel{2}}}&=\frac{3}{\textcolor{blue}{2}}\\
M&=1,5
\end{align*}

Conclusion~: la solution est $M=1,5.$



\end{exo}






\begin{exo}

Le stade des Gones compte \np{15000} places. Il y a $x$ places dans les virages et les autres dans les tribunes. Une place en virage coûte 15 € et une place dans les tribunes coûte 25 €.

\medskip

Aujourd'hui, le stade est plein et la recette est de \np{295000} €.

\begin{enumerate}
\item Il y a $x$ places dans les virages, donc $(\np{15000}-x)$ places dans les tribunes. La recette totale en € est donc
\[15\times x+25\times(\np{15000}-x).\] Comme cette recette est \np{295000} €, $x$ est solution de l'équation
\[15 x+25(\np{15000}-x)=\np{295000}.\]
\item On résout l'équation de la question précédente~:

\begin{align*}
15 x+25(\np{15000}-x)&=\np{295000}\\
15x+25\times\np{15000}+25\times (-x)&=\np{295000}\\
15x+\np{375000}-25x&=\np{295000}\\
-10x+\np{375000}&=\np{295000}\\
-10x+\cancel{\np{375000}}\textcolor{red}{-\cancel{\np{375000}}}&=\np{295000}\textcolor{red}{-\np{375000}}\\
-10x&=-\np{80000}\\
\frac{\cancel{\textcolor{blue}{-10}}x}{\textcolor{blue}{\cancel{-10}}}&=\frac{-\np{80000}}{\textcolor{blue}{{-10}}}\\
x&=8000.\end{align*}

\medskip

Conclusion~: il y a $x=\np{8000}$ places dans les virages (et donc \np{7000} dans les tribunes).
\end{enumerate}


\end{exo}



\begin{exo}

\begin{align*}
A&=\frac{5}{6}+\frac{2}{3}=\frac{5}{6}+\frac{2\times 2}{3\times 2}=\frac{5}{6}+\frac{4}{6}=\frac{5+4}{6}=\frac{9}{6}=\frac{3\times \cancel{3}}{2\times \cancel{3}}=\frac{3}{2}\\
B&=\frac{3}{4}-\frac{1}{6}=\frac{3\times 3}{4\times 3}-\frac{1\times 2}{6\times 2}=\frac{9}{12}-\frac{2}{12}=\frac{9-2}{12}=\frac{7}{12}
\\
C&=2+\frac{1}{5}=\frac{2}{1}+\frac{1}{5}=\frac{2\times 5}{1\times 5}+\frac{1}{5}=\frac{10}{5}+\frac{1}{5}=\frac{11}{5}\\
D&=\frac{3}{10}\times \frac{5}{6}=\frac{3\times 5}{10\times 6}=\frac{15}{60}=\frac{\cancel{15}}{\cancel{15}\times 4}=\frac{1}{4}\\
E&=2\times\frac{5}{6}-\frac{4}{9}=\frac{2\times 5}{6}-\frac{4}{9}=\frac{10\times 3}{6\times 3}-\frac{4\times 2}{9\times 2}=\frac{30}{18}-\frac{8}{18}=\frac{30-8}{18}=\frac{22}{18}=\frac{11\times \cancel{2}}{9\times \cancel{2}}=\frac{11}{9}\\
F&=4-3\times\frac{5}{6}=\frac{4}{1}-\frac{3\times 5}{6}=\frac{4\times 6}{1\times 6}-\frac{15}{6}=\frac{24}{6}-\frac{15}{6}=\frac{24-15}{6}=\frac{9}{6}=\frac{3\times \cancel{3}}{2\times \cancel{3}}=\frac{3}{2}
\\
G&=\frac{6}{10}\times \frac{15}{8}=\frac{6\times 15}{10\times 8}=\frac{90}{80}=\frac{9\times \cancel{10}}{8\times \cancel{10}}=\frac{9}{8}\\
H&=\left(\frac{2}{3}\right)^2=\frac{2}{3}\times\frac{2}{3}=\frac{2\times 2}{3\times 3}=\frac{4}{9}
\end{align*}

\end{exo}

\begin{exo}

Le père donne le tiers de la somme nécessaire et le petit-frère donne le quart, donc à eux deux ils en donnent
\[\frac{1}{3}+\frac{1}{4}=\frac{1\times 4}{3\times 4}+\frac{1\times 3}{4\times 3}=\frac{4}{12}+\frac{3}{12}=\frac{7}{12}.\]

Ainsi il reste $\frac{5}{12}$ du prix à payer à la charge du grand-frère. Or on sait que le grand frère a donné 10~\euro, donc le prix du livre (soit $\frac{12}{12}$ du prix) est égal à
\[\frac{12}{5}\times 10=\frac{12\times 10}{5}=\frac{120}{5}=24~\text{\euro}.\]

\medskip

\textbf{Remarque~:} Il peut être agréable de présenter les choses avec le schéma ci-dessous~: chaque petite tranche représente $\frac{1}{12}$ du prix du livre et vaut $2$~\euro. Ainsi, les $\frac{5}{12}$ du prix payé (c'est-à-dire le prix payé par le grand-frère) valent $5\times 2=10$~\euro~; et la valeur totale du livre est $12\times 2=24$~\euro.


\begin{center}
\psset{xunit=1.0cm,yunit=1.0cm,algebraic=true,dimen=middle,dotstyle=o,dotsize=5pt 0,linewidth=2.pt,arrowsize=3pt 2,arrowinset=0.25}
\begin{pspicture*}(-2.3,1.22)(10.36,3.8)
\pspolygon[linewidth=2.pt,linecolor=blue,fillcolor=blue!20!white,fillstyle=solid,opacity=0.1](-2.,2.)(3.,2.)(3.,3.)(-2.,3.)
\psline[linewidth=2.pt](-2.,2.)(10.,2.)
\psline[linewidth=2.pt](10.,2.)(10.,3.)
\psline[linewidth=2.pt](10.,3.)(-2.,3.)
\psline[linewidth=2.pt](-2.,3.)(-2.,2.)
\psline[linewidth=2.pt](-1.,3.)(-1.,2.)
\psline[linewidth=2.pt](0.,3.)(0.,2.)
\psline[linewidth=2.pt](1.,3.)(1.,2.)
\psline[linewidth=2.pt](2.,3.)(2.,2.)
\psline[linewidth=2.pt](3.,3.)(3.,2.)
\psline[linewidth=2.pt](4.,3.)(4.,2.)
\psline[linewidth=2.pt](5.,3.)(5.,2.)
\psline[linewidth=2.pt](6.,3.)(6.,2.)
\psline[linewidth=2.pt](7.,3.)(7.,2.)
\psline[linewidth=2.pt](8.,3.)(8.,2.)
\psline[linewidth=2.pt](9.,3.)(9.,2.)
\psline[linewidth=2.pt,linecolor=blue](-2.,2.)(3.,2.)
\psline[linewidth=2.pt,linecolor=blue](3.,2.)(3.,3.)
\psline[linewidth=2.pt,linecolor=blue](3.,3.)(-2.,3.)
\psline[linewidth=2.pt,linecolor=blue](-2.,3.)(-2.,2.)
\rput[tl](-1.15,3.46){\textcolor{blue}{part du grand frère$=10$~\euro}}
\rput[tl](-1.76,2.64){2~\euro}
\rput[tl](1.5,1.82){valeur totale du livre$=24$~\euro}
\end{pspicture*}
\end{center}



\end{exo}



\begin{exo}


\begin{align*}
A&=\frac{2^{15}\times 3^{6}}{2^{12}\times 3^4}=\frac{2^{15}}{2^{12}}\times \frac{ 3^{6}}{ 3^4}=2^{15-12}\times 3^{6-4}=2^3\times 3^2=8\times 9=72\\
B&=\frac{5^{3}\times 5^{6}}{5^7}=\frac{5^{3+6}}{5^7}=\frac{5^{9}}{5^7}=5^{9-7}=5^2=25\\
C&=\frac{2^{18}}{8\times 2^{12}}=\frac{2^{18}}{2^3\times 2^{12}}=\frac{2^{18}}{2^{3+12}}=\frac{2^{18}}{2^{15}}=2^{18-15}=2^3=8\\
D&=\frac{6^6}{2^5\times 3^4}=\frac{(2\times 3)^6}{2^5\times 3^4}=\frac{2^6\times 3^6}{2^5\times 3^4}=\frac{2^6}{2^5}\times \frac{3^6}{3^4}=2^{6-5}\times 3^{6-4}=2^1\times 3^2=2\times 9=18\\
E&=\frac{\left(10^4\right)^3}{10^8}=\frac{10^{4\times 3}}{10^8}=\frac{10^{12}}{10^8}=10^{12-8}=10^4=\np{10000}\\
 F&=\frac{4^5}{8^3}=\frac{(2^2)^5}{(2^3)^3}=\frac{2^{2\times 5}}{2^{3\times 3}}=\frac{2^{10}}{2^9}=2^{10-9}=2\\ 
 G&=\frac{10^{10}+10^8}{10^7}=\frac{10^{10}}{10^7}+\frac{10^8}{10^7}=10^{10-7}+10^{8-7}=10^3+10^1=\np{1000}+1=\np{1001}
 \end{align*}

\end{exo}


\begin{exo}%27

Pour ranger les nombres par ordre croissant, on les écrit sous forme décimale, en écrivant à chaque fois quatre chiffres après la virgule pour simplifier les comparaisons.

On rappelle avant cela que $10^{-3}=\dfrac{1}{10^3}=\dfrac{1}{1000}=\underbrace{0,00}_{\text{3 zéros}}1,$ donc multiplier un nombre par $10^{-3}$ revient à décaler la virgule de 3 rangs vers la gauche (le raisonnement est le même pour $10^{-2}$).

\begin{align*}
A&=35,4\times 10^{-3}&&=0,0354\\
B&=0,034&&=0,0340\\
C&=3,6\times 10^{-2}=0,036&&=0,0360\\
D&=\frac{355}{10^4}=\frac{355}{\np{10000}}&&=0,0355\\
E&=\frac{7}{60}\times\frac{3}{10}=\frac{7\times 3}{60\times 10}=\frac{7\times \cancel{3}}{20\times\cancel{3}\times 10}=\frac{7}{200}&&=0,0350
\end{align*}

Conclusion~: $B<E<A<D<C.$
\end{exo}





\begin{exo}

Avant de commencer, il est utile de se rappeler que 10~cm=1~dm~; et que 1~$\ell=1~\text{dm}^3.$ Autrement dit, un litre est le volume d'un cube qui mesure 1~dm sur 1~dm sur 1~dm, ou encore 10~cm sur 10~cm sur 10~cm (la figure ci-dessous n'est bien sûr pas à l'échelle).


\begin{center}
\psset{xunit=1.0cm,yunit=1.0cm,algebraic=true,dimen=middle,dotstyle=o,dotsize=5pt 0,linewidth=2.pt,arrowsize=3pt 2,arrowinset=0.25}
\begin{pspicture*}(-1.42,-1.12)(4.42,3.5)
\psline[linewidth=2.pt](0.,0.)(2.,0.)
\psline[linewidth=2.pt](2.,0.)(2.,2.)
\psline[linewidth=2.pt](2.,2.)(0.,2.)
\psline[linewidth=2.pt](0.,2.)(0.,0.)
\psline[linewidth=2.pt](0.,2.)(1.,3.)
\psline[linewidth=2.pt](1.,3.)(3.,3.)
\psline[linewidth=2.pt](2.,2.)(3.,3.)
\psline[linewidth=2.pt](2.,0.)(3.,1.)
\psline[linewidth=2.pt](3.,3.)(3.,1.)
\rput[tl](0.58,-0.16){1~dm}
\rput[tl](-1.,1.14){1~dm}
\rput[tl](2.7,0.52){1~dm}
\rput[tl](0.92,1.24){$1~\ell$}
\end{pspicture*}
\end{center}


\medskip


On remplit d'eau un aquarium rectangulaire dont la largeur est 80~cm, la profondeur 30~cm et la hauteur 40~cm. On dispose d'un robinet dont le débit est de 6 litres par minute.

\begin{enumerate}
\item Les dimensions de l'aquarium sont~:
\[\text{largeur}=8~\text{dm},\qquad \text{profondeur}=3~\text{dm},\qquad \text{hauteur}=4~\text{dm},\] donc son volume est
\[8\times 3\times 4=96~\ell.\]

\begin{center}
\psset{xunit=0.75cm,yunit=0.75cm,algebraic=true,dimen=middle,dotstyle=o,dotsize=5pt 0,linewidth=2.pt,arrowsize=3pt 2,arrowinset=0.25}
\begin{pspicture*}(0.476052349791793,0.11148839976204462)(12.435257584770984,6.724523497917915)
\psline[linewidth=2.pt](2.,1.)(10.,1.)
\psline[linewidth=2.pt](10.,1.)(10.,5.)
\psline[linewidth=2.pt](2.,1.)(2.,5.)
\psline[linewidth=2.pt](2.,5.)(10.,5.)
\psline[linewidth=2.pt](2.,5.)(4.,6.)
\psline[linewidth=2.pt](4.,6.)(12.,6.)
\psline[linewidth=2.pt](10.,5.)(12.,6.)
\psline[linewidth=2.pt](10.,1.)(12.,2.)
\psline[linewidth=2.pt](12.,2.)(12.,6.)
\psline[linewidth=2.pt,linestyle=dashed,dash=3pt 3pt](4.,6.)(4.,2.)
\psline[linewidth=2.pt,linestyle=dashed,dash=3pt 3pt](2.,1.)(4.,2.)
\psline[linewidth=2.pt,linestyle=dashed,dash=3pt 3pt](4.,2.)(12.,2.)
\rput[tl](5.720873289708514,0.8209327781082683){8 dm}
\rput[tl](0.932123735871508,3.2026389054134476){4 dm}
\rput[tl](11.09238072575849,1.4797025580011902){3 dm}
\end{pspicture*}
\end{center}
\item On peut se passer d'un tableau de proportionnalité~: le débit du robinet est de 6~$\ell$/min, donc il faut $96\div 6=16$~min pour remplir les 96~$\ell$ de l'aquarium.
\end{enumerate}
\end{exo}




\begin{exo}

On utilise les identités remarquables pour calculer~:

\begin{align*}
99^2&=(100-1)^2=100^2-2\times 100\times 1+1^2=\np{10000}-200+1=\np{9801}&&\text{(IR n°2)}\\
103^2&=(100+3)^2=100^2+2\times 100\times 3+3^2=\np{10000}+600+9=\np{10609}&&\text{(IR n°1)}\\
71\times 69&=(70+1)(70-1)=70^2-1^2=\np{4900}-1=\np{4899}&&\text{(IR n°3)}\\
2,05^2&=(2+0,05)^2=2^2+2\times 2\times 0,05+0,05^2=4+0,2+0,0025=4,2025&&\text{(IR n°1)}\\
4,3\times 3,7&=(4+0,3)(4-0,3)=4^2-0,3^2=16-0,09=15,91&&\text{(IR n°3)}
\end{align*}

\textbf{Remarque~:} Comment calculer $0,05^2$ de tête~? Comme $0,05^2=0,05\times 0,05$ et que $0,05$ a deux chiffres après la virgule, $0,05^2$ en aura $2+2=4.$ Il ne reste alors plus qu'à calculer $5^2=25$ pour pouvoir conclure~: $0,05^2=0,0025.$

Attention cependant à cette méthode~: les derniers chiffres du résultat peuvent être des $0,$ comme dans l'exemple suivant~:
\[0,05\times 0,0006=0,000030,\] puisque $6\times 5=30$ et que le résultat doit avoir $2+4=6$ chiffres après la virgule (le dernier, ici, étant un 0).

\end{exo}

\begin{exo}%8

Le côté du grand carré mesure $a+b,$ donc son aire est $(a+b)^2.$

D'un autre côté, le grand carré peut être découpé en quatre parties~: un carré de côté $a,$ donc d'aire $a^2$ (hachuré en bleu), un carré de côté $b,$ donc d'aire $b^2$ (hachuré en vert) et deux rectangles de côtés $a$ et $b,$ donc d'aires $a\times b$ (hachurés en rouge). Ainsi l'aire du grand carré est-elle aussi égale à
\[a^2+b^2+2\times a\times b.\] En comparant avec la première méthode de calcul de l'aire, on obtient la relation attendue~:
\[\left(a+b\right)^2=a^2+2ab+b^2.\]

\begin{center}
\psset{xunit=1.0cm,yunit=1.0cm,algebraic=true,dimen=middle,dotstyle=o,dotsize=5pt 0,linewidth=2.pt,arrowsize=3pt 2,arrowinset=0.25}
\begin{pspicture*}(-1.28,-1.06)(4.82,3.62)
\pscustom[linewidth=0.8pt,linecolor=blue,hatchcolor=blue,fillstyle=hlines,hatchangle=45.0,hatchsep=0.2]{\psplot{0.}{2.}{2.0}\lineto(2.,0)\lineto(0.,0)\closepath}
\pscustom[linewidth=0.8pt,linecolor=red,hatchcolor=red,fillstyle=hlines,hatchangle=45.0,hatchsep=0.2]{\psplot{2.}{3.}{2.0}\lineto(3.,0)\lineto(2.,0)\closepath}
\pscustom[linewidth=0.8pt,linecolor=red,hatchcolor=red,fillstyle=hlines,hatchangle=45.0,hatchsep=0.2]{\psplot{0.}{2.}{2.0}\lineto(2.,3.)\psplot{2.}{0.}{3.0}\lineto(0.,2.)\closepath}
\pscustom[linewidth=0.8pt,linecolor=green,hatchcolor=green,fillstyle=hlines,hatchangle=45.0,hatchsep=0.2]{\psplot{2.}{3.}{2.0}\lineto(3.,3.)\psplot{3.}{2.}{3.0}\lineto(2.,2.)\closepath}
\rput[tl](0.85,-0.14){$a$}
\rput[tl](2.40,-0.14){$b$}
\rput[tl](-0.44,1.02){$a$}
\rput[tl](-0.44,2.62){$b$}
\end{pspicture*}
\end{center}
\end{exo}

\begin{exo}

\begin{enumerate}
\item Pour comparer les fractions $a=\frac{4}{5}$ et $b=\frac{5}{6},$ on les réduit au même dénominateur~:
\[a=\frac{4\times 6}{5\times 6}=\frac{24}{30}\qquad,\qquad b=\frac{5\times 5}{6\times 5}=\frac{25}{30}.\]
Comme $24<25,$ on obtient $a<b.$
\item On compare à présent $c=\frac{524}{525}$ et $ d=\frac{525}{526}.$ On réduit là aussi au même dénominateur, mais on n'effectue aucun calcul (comme nous allons le voir, ce n'est pas nécessaire)~:
\[c=\frac{524\times 526}{525\times 526}\qquad,\qquad d=\frac{525\times 525}{526\times 525}.\]
Les dénominateurs sont identiques, donc il suffit de comparer les numérateurs. D'après l'identité remarquable n°3,
\[524\times 526=(525-1)(525+1)=525^2-1^2=525^2-1.\] Ce nombre est strictement inférieur à  $525\times 525=525^2,$ donc $c<d.$
\end{enumerate}

\end{exo}



\begin{exo}

La partie hachurée de la figure de gauche est un rectangle de côtés $(a-b)$ et $(a+b),$ donc son aire est égale à $(a-b)(a+b).$

Quant à la partie hachurée de la figure de droite, c'est un carré de côté $a$ duquel on a retiré un carré de côté $b.$ Son aire est donc égale à $a^2-b^2.$

L'identité remarquable n°3 nous dit que $(a-b)(a+b)=a^2-b^2,$ donc les aires des deux zones hachurées sont les mêmes.


\begin{center}
\psset{xunit=0.6cm,yunit=0.6cm,algebraic=true,dimen=middle,dotstyle=o,dotsize=3pt 0,linewidth=0.8pt,arrowsize=3pt 2,arrowinset=0.25}
\begin{pspicture*}(0,-2)(12,6)
\pspolygon[linecolor=gray](1,-1)(5,-1)(5,3)(1,3)
\pspolygon[linecolor=gray,hatchcolor=gray,fillstyle=hlines,hatchangle=45.0,hatchsep=0.28](1,4)(4,4)(4,-1)(1,-1)
\pspolygon[linecolor=gray](4,3)(4,4)(5,4)(5,3)
\pspolygon[linecolor=gray,hatchcolor=gray,fillstyle=hlines,hatchangle=45.0,hatchsep=0.28](7,3)(7,-1)(11,-1)(11,3)
\pspolygon[linecolor=gray,fillcolor=gray,fillstyle=solid,opacity=1.0](11,3)(10,3)(10,2)(11,2)
\psline[linecolor=gray](1,-1)(5,-1)
\psline[linecolor=gray](5,-1)(5,3)
\psline[linecolor=gray](5,3)(1,3)
\psline[linecolor=gray](1,3)(1,-1)
\psline[linecolor=gray](1,4)(4,4)
\psline[linecolor=gray](4,4)(4,-1)
\psline[linecolor=gray](4,-1)(1,-1)
\psline[linecolor=gray](1,-1)(1,4)
\rput[tl](0.43,3.85){$b$}
\rput[tl](4.32,-0.2){$b$}
\rput[tl](0.34,1.2){$a$}
\psline{->}(2.76,-1.62)(4.94,-1.62)
\psline{->}(2.76,-1.62)(0.98,-1.62)
\rput[tl](2.77,-1.25){$a$}
\psline[linecolor=gray](4,3)(4,4)
\psline[linecolor=gray](4,4)(5,4)
\psline[linecolor=gray](5,4)(5,3)
\psline[linecolor=gray](5,3)(4,3)
\psline[linecolor=gray](7,3)(7,-1)
\psline[linecolor=gray](7,-1)(11,-1)
\psline[linecolor=gray](11,-1)(11,3)
\psline[linecolor=gray](11,3)(7,3)
\rput[tl](8.91,-1.3){$a$}
\rput[tl](6.4,1.4){$a$}
\psline[linecolor=gray](11,3)(10,3)
\psline[linecolor=gray](10,3)(10,2)
\psline[linecolor=gray](10,2)(11,2)
\psline[linecolor=gray](11,2)(11,3)
\rput[tl](10.26,3.9){$b$}
\rput[tl](11.12,3.05){$b$}
\end{pspicture*}
\end{center}
\end{exo}

\begin{exo}~{}

\begin{center}
\psset{xunit=0.6cm,yunit=0.6cm,algebraic=true,dimen=middle,dotstyle=o,dotsize=5pt 0,linewidth=2.pt,arrowsize=3pt 2,arrowinset=0.25}
\begin{pspicture*}(0.16,0)(11.94,5.62)
\psline[linewidth=2.pt](1.,1.)(11.,1.)
\psline[linewidth=2.pt](1.,1.)(1.,5.)
\psline[linewidth=2.pt](11.,1.)(11.,4.)
\psline[linewidth=2.pt](1.,5.)(5.65,1.)
\psline[linewidth=2.pt](3.3653507283020185,3.123579200213858)(3.2088388124638874,2.9416340980520306)
\psline[linewidth=2.pt](5.65,1.)(11.,4.)
\psline[linewidth=2.pt](8.222696638345388,2.5802123568670416)(8.340080575223986,2.370877669433542)
\rput[tl](0.5,3.){$4$}
\rput[tl](11.3,2.52){$3$}
%\begin{scriptsize}
\psdots[dotsize=2pt 0,dotstyle=*,linecolor=darkgray](1.,1.)
\rput[bl](0.3,1.1){{$M$}}
\psdots[dotsize=2pt 0,dotstyle=*,linecolor=darkgray](11.,1.)
\rput[bl](11.1,1.08){{$N$}}
\psdots[dotsize=2pt 0,dotstyle=*,linecolor=darkgray](1.,5.)
\rput[bl](0.54,5.){{$A$}}
\psdots[dotsize=2pt 0,dotstyle=*,linecolor=darkgray](11.,4.)
\rput[bl](11.08,4.08){{$B$}}
\psdots[dotsize=2pt 0,dotstyle=*,linecolor=darkgray](5.65,1.)
\rput[bl](5.66,1.22){{$P$}}
\rput[bl](3,0.3){{$x$}}
\rput[bl](8,0.3){{$10-x$}}

%\end{scriptsize}
\end{pspicture*}
\end{center}

On pose $MP=x,$ on a donc $PN=MN-MP=10-x.$

D'après le théorème de Pythagore dans chacun des deux triangles rectangles $AMP$ et $NBP~:$

\setlength{\columnseprule}{1pt}
\begin{multicols}{2}

\begin{align*}
AP^2&=MP^2+MA^2\\
AP^2&=x^2+4^2\\
AP^2&=x^2+16
\end{align*}
\columnbreak

\begin{align*}
BP^2&=PN^2+BN^2\\
BP^2&=(10-x)^2+3^2\\
BP^2&=10^2-2\times 10\times x+x^2+9 \qquad\text{(on développe grâce à l'IR n°2)}\\
BP^2&=100-20x+x^2+9\\
BP^2&=x^2-20x+109
\end{align*}
\end{multicols} 
On sait que $AP=BP,$ donc $AP^2=BP^2~;$ et d'après les deux calculs ci-dessus~:
\[\cancel{x^2}+16=\cancel{x^2}-20x+109.\] Il n'y a plus qu'à résoudre~:

\begin{align*}
16&=-20x+109\\
16\textcolor{red}{-109}&=-20x+\cancel{109}\textcolor{red}{-\cancel{109}}\\
\frac{-93}{\textcolor{blue}{-20}}&=\frac{\cancel{-20}x}{\textcolor{blue}{\cancel{-20}}}\\
4,65&=x
\end{align*}

Conclusion~: $MP=4,65.$
\end{exo}

\begin{exo}~{}


\begin{center}
\newrgbcolor{xfqqff}{0.4980392156862745 0. 1.}
\psset{xunit=1.0cm,yunit=1.0cm,algebraic=true,dimen=middle,dotstyle=o,dotsize=5pt 0,linewidth=2.pt,arrowsize=3pt 2,arrowinset=0.25}
\begin{pspicture*}(0.92,0.36)(7.04,4.54)
\pspolygon[linewidth=2.pt,linecolor=xfqqff,fillcolor=xfqqff!20!white,fillstyle=solid,opacity=0.1](2.4242640687119286,1.)(2.4242640687119286,1.4242640687119286)(2.,1.4242640687119288)(2.,1.)
\psline[linewidth=2.pt](2.,1.)(6.,1.)
\psline[linewidth=2.pt](2.,1.)(2.,4.)
\psline[linewidth=2.pt](2.,4.)(6.,1.)
\rput[tl](3.36,0.88){$x$}
\rput[tl](1.34,2.54){$y$}
\rput[tl](4.24,2.64){13}
\end{pspicture*}
\end{center}

\begin{enumerate}
\item D'après le théorème de Pythagore, \[x^2+y^2=13^2=169.\]

D'après l'IR n°1, $(x+y)^2=x^2+y^2+2xy.$ Or $x^2+y^2=169,$ et $\frac{x\times y}{2}=30,$ puisque c'est l'aire du triangle. On en déduit $x\times y=30\times 2=60,$ puis
\[(x+y)^2=\textcolor{blue}{x^2+y^2}+2\textcolor{red}{xy}=\textcolor{blue}{169}+2\times \textcolor{red}{60}=169+120=289.\]
Finalement, comme $(x+y)^2=289,$
\[x+y=\sqrt{289}=17.\]

\item On utilise cette fois l'IR n°2~:
\[(x-y)^2=\textcolor{blue}{x^2+y^2}-2\textcolor{red}{xy}=\textcolor{blue}{169}-2\times \textcolor{red}{60}=169-120=49.\]

Or $x-y\geq 0,$ puisque $x$ est plus grand que $y,$ donc
\[x-y=\sqrt{49}=7.\]

\danger Si on ne savait pas lequel des deux côtés est le plus grand, on pourrait avoir $x-y=-7~!!!$

\medskip

On sait à présent que $x+y=17$ et $x-y=7.$ On ajoute membre à membre ces égalités et on en déduit $x~:$

\begin{align*}
(x+y)+(x-y)&=17+7\\
x+\cancel{y}+x-\cancel{y}&=24\\
\frac{\cancel{2}x}{\cancel{2}}&=\frac{24}{2}\\
x&=12
\end{align*}

Enfin, comme $x+y=17,$ on trouve $y=17-x=17-12=5.$

Conclusion~: $x=12,$ $y=5.$


\end{enumerate}
\end{exo}





\section{Nombres réels}


\begin{exo}

\begin{enumerate}
\item $-7\in\mathbb{Q}.$ \textbf{VRAI.}

Justification~: $-7=\dfrac{-7}{1},$ donc $-7\in \mathbb{Q}$ (il s'écrit comme le quotient de deux entiers).

\item $-7\in\mathbb{N}.$ \textbf{FAUX.}

Justification~: $-7$ est strictement négatif, donc ce n'est pas un entier naturel.

\item $-\dfrac{13}{4}\in\mathbb{Z}.$  \textbf{FAUX.}

Justification~: $-\dfrac{13}{4}=-3,25$ a des chiffres après la virgule, donc il n'est pas entier.

\textbf{Remarque~:} Pour obtenir $\dfrac{13}{4}=3,25$ sans calculatrice, trois possibilités~: \textcircled{\small{1}} Diviser de tête 13 par 2 deux fois de suite -- \textcircled{\small{2}} Poser la division -- \textcircled{\small{3}} Remarquer que $\dfrac{13}{4}=\dfrac{12}{4}+\dfrac{1}{4}=3+0,25=3,25.$

\item $-\dfrac{13}{4}\in\mathbb{D}.$  \textbf{VRAI.}

Justification~: $-\dfrac{13}{4}=-3,25$ a deux chiffres après la virgule, donc il est décimal.
\item $5,824\in\mathbb{D}.$ \textbf{VRAI.}

Justification~: $5,824$ a trois chiffres après la virgule, donc il est décimal

\item $5,824\in\mathbb{Q}.$ \textbf{VRAI.}

Justification n°1~: $5,824$ est décimal (cf question précédente), donc il est rationnel d'après le cours ($\mathbb{D}\subset \mathbb{Q}$).

Justification n°2~: $5,824=\dfrac{\np{5824}}{\np{1000}},$ donc $5,824\in \mathbb{Q}$ (il s'écrit comme le quotient de deux entiers).
\item $\dfrac{10}{6}\in\mathbb{D}.$ \textbf{FAUX.}

Justification~: On pose la division~:

\begin{center}
\opdiv[decimalsepsymbol={,},displayintermediary=nonzero,voperator=bottom,shiftdecimalsep=none,maxdivstep=2]{10}{6}
\end{center}

Comme on obtient deux fois le même reste (4), ça va continuer indéfiniment. Conclusion~: $\dfrac{10}{6}=1,666\cdots$ n'est pas décimal, puisqu'il a une infinité de chiffres après la virgule.
\item $\dfrac{17}{11}\in\mathbb{D}.$ \textbf{FAUX.}

Justification~: On pose la division~:

\begin{center}
\opdiv[decimalsepsymbol={,},displayintermediary=nonzero,voperator=bottom,shiftdecimalsep=none,maxdivstep=3]{17}{11}
\end{center}

Comme on obtient deux fois le même reste (6), ça va continuer indéfiniment. Conclusion~: $\dfrac{17}{11}=1,5454\cdots$ n'est pas décimal, puisqu'il a une infinité de chiffres après la virgule.
\end{enumerate}

\end{exo}



\begin{exo}

\begin{enumerate}
\item \[I_1=\left[1;4\right]\qquad I_2=\left[5;+\infty\right[\qquad I_3=\left]-2;0\right[\]

\begin{center}
\psset{xunit=1.0cm,yunit=1.0cm,algebraic=true,dimen=middle,dotstyle=o,dotsize=5pt 0,linewidth=2.pt,arrowsize=3pt 2,arrowinset=0.25}
\begin{pspicture*}(-2.16,-0.86)(9.56,0.96)
\psaxes[labelFontSize=\scriptstyle,xAxis=true,yAxis=false,labels=x,Dx=1.,Dy=1.,ticksize=-2pt 0,subticks=2]{->}(0,0)(-2.16,-0.86)(9.56,0.96)
\psline[linewidth=3.pt,linecolor=red](1.,0.)(4.,0.)
\psline[linewidth=3.pt,linecolor=blue](5.,0.)(9.3,0.)
\psline[linewidth=3.pt,linecolor=green](-2.,0.)(0.,0.)
\rput[tl](2.2,0.5){\red{$I_1$}}
\rput[tl](7.18,0.5){\blue{$I_2$}}
\rput[tl](-1.24,0.5){\green{$I_3$}}
\rput[tl](-2.12,0.32){\green{\Huge ]}}
\rput[tl](-0.12,0.32){\green{\Huge [}}
\rput[tl](0.88,0.32){\red{\Huge [}}
\rput[tl](3.88,0.32){\red{\Huge ]}}
\rput[tl](4.88,0.32){\blue{\Huge [}}
\end{pspicture*}
\end{center}


\item \[I_1=\left[-1;1\right[\qquad I_2=\left]3;+\infty\right[\qquad I_3=\left]-\infty;-2\right]\]

\begin{center}
\psset{xunit=1.0cm,yunit=1.0cm,algebraic=true,dimen=middle,dotstyle=o,dotsize=5pt 0,linewidth=2.pt,arrowsize=3pt 2,arrowinset=0.25}
\begin{pspicture*}(-6.16,-0.86)(7.56,0.96)
\psaxes[labelFontSize=\scriptstyle,xAxis=true,yAxis=false,labels=x,Dx=1.,Dy=1.,ticksize=-2pt 0,subticks=2]{->}(0,0)(-6.16,-0.86)(7.56,0.96)
\psline[linewidth=3.pt,linecolor=red](-1.,0.)(1.,0.)
\psline[linewidth=3.pt,linecolor=blue](3.,0.)(9.3,0.)
\psline[linewidth=3.pt,linecolor=green](-6.16,0.)(-2.,0.)
\rput[tl](-0.3,0.5){\red{$I_1$}}
\rput[tl](5.18,0.5){\blue{$I_2$}}
\rput[tl](-4.5,0.5){\green{$I_3$}}
\rput[tl](-2.12,0.32){\green{\Huge ]}}
\rput[tl](-1.12,0.32){\red{\Huge [}}
\rput[tl](0.88,0.32){\red{\Huge [}}
\rput[tl](2.88,0.32){\blue{\Huge ]}}
\end{pspicture*}
\end{center}


\end{enumerate}

\end{exo}

\begin{exo}


\begin{enumerate}
\item $5\in\left[2;6\right[$
\item $-2\notin \left]-2;1\right]$
\item $\pi\in \left]3;4\right[$ (on rappelle que $\pi\approx 3,14$)
\end{enumerate}

\end{exo}

\begin{exo}

\begin{enumerate}
\item $5\times|-6|=5\times 6=30$
\item $|3|+|-3|=3+3=6$
\item $|5|-|-5|=5-5=0$
\item $|-4|\times|2|=4\times 2=8$
\item $|7-4|=|3|=3$
\item $|4-7|=|-3|=3$
\item $|4-3|+|5-6|=|1|+|-1|=1+1=2$
\item $|5-11|+2\times|7-8|=|-6|+2\times |-1|=6+2\times 1=6+2=8$
\item $|8-5|\times |7-10|=|3|\times |-3|=3\times 3=9$
\item $|15-6|-4\times|1-4|=|9|-4\times |-3|=9-4\times 3=9-12=-3$
\end{enumerate}
\end{exo}

\begin{exo}


\begin{enumerate}
\item On résout l'équation $|x-2|=3.$

\setlength{\columnseprule}{1pt}
\begin{multicols}{2}

\textbf{Méthode n°1~: avec la définition de la valeur absolue.}

\medskip

Les nombres dont la valeur absolue vaut 3 sont $3$ et $-3.$

Donc dire que \[|x-2|=3\] revient à dire que
\[x-2=3\qquad\text{ou que}\qquad x-2=-3\]

Donc

\begin{alignat*}{3}
&x-\cancel{2}+\cancel{2}=3+2&&\text{\qquad ou \qquad} x-\cancel{2}+\cancel{2}=-3+2\\
&x=5&&\text{\qquad ou \qquad} x=-1\\
\end{alignat*}

Conclusion~: l'équation a deux solutions~: $x=5$ et $x=-1.$

\columnbreak

\textbf{Méthode n°2~: avec la distance.}

\medskip

Dire que \[|x-2|=3\] revient à dire que la distance entre $x$ et $2$ est égale à $3.$


\begin{center}
\psset{xunit=1.0cm,yunit=1.0cm,algebraic=true,dimen=middle,dotstyle=o,dotsize=5pt 0,linewidth=1.6pt,arrowsize=3pt 2,arrowinset=0.25}
\begin{pspicture*}(-1.45,-0.41034356916185444)(5.768763307234229,1.1013400300795446)
\psaxes[labelFontSize=\scriptstyle,xAxis=true,yAxis=false,Dx=0.5,Dy=0.5,ticksize=-2pt 0,subticks=2]{->}(0,0)(-1.45,-0.41034356916185444)(5.768763307234229,1.1013400300795446)
\psline[linewidth=2.pt,linecolor=red]{->}(0.5,0.5)(-1.,0.5)
\psline[linewidth=2.pt,linecolor=red]{->}(0.5,0.5)(2.,0.5)
\rput[tl](-0.3,1){\red{distance $=$ 3}}
\psline[linewidth=2.pt,linecolor=red]{->}(3.5,0.5)(2.,0.5)
\psline[linewidth=2.pt,linecolor=red]{->}(3.5,0.5)(5.,0.5)
\rput[tl](2.7,1){\red{distance $=$ 3}}
\psdots[dotstyle=*,linecolor=blue](2.,0.)
\psdots[dotstyle=*,linecolor=green](5.,0.)
\psdots[dotstyle=*,linecolor=green](-1.,0.)
\end{pspicture*}
\end{center}


On voit qu'il y a deux solutions~: $x=5$ et $x=-1.$

\end{multicols}
\item On résout l'équation $|x-1|=4.$

\setlength{\columnseprule}{1pt}
\begin{multicols}{2}

\textbf{Méthode n°1~: avec la définition de la valeur absolue.}

\medskip

Les nombres dont la valeur absolue vaut 4 sont $4$ et $-4.$

Donc dire que \[|x-1|=4\] revient à dire que
\[x-1=4\qquad\text{ou que}\qquad x-1=-4\]

Donc

\begin{alignat*}{3}
&x-\cancel{1}+\cancel{1}=4+1&&\text{\qquad ou \qquad} x-\cancel{1}+\cancel{1}=-4+1\\
&x=5&&\text{\qquad ou \qquad} x=-3\\
\end{alignat*}

Conclusion~: l'équation a deux solutions~: $x=5$ et $x=-3.$

\columnbreak

\textbf{Méthode n°2~: avec la distance.}

\medskip

Dire que \[|x-1|=4\] revient à dire que la distance entre $x$ et $1$ est égale à $4.$


\begin{center}
\psset{xunit=1.0cm,yunit=1.0cm,algebraic=true,dimen=middle,dotstyle=o,dotsize=5pt 0,linewidth=1.6pt,arrowsize=3pt 2,arrowinset=0.25}
\begin{pspicture*}(-3.45,-0.41034356916185444)(5.768763307234229,1.1013400300795446)
\psaxes[labelFontSize=\scriptstyle,xAxis=true,yAxis=false,Dx=0.5,Dy=0.5,ticksize=-2pt 0,subticks=2]{->}(0,0)(-3.45,-0.41034356916185444)(5.768763307234229,1.1013400300795446)
\psline[linewidth=2.pt,linecolor=red]{->}(0.5,0.5)(1.,0.5)
\psline[linewidth=2.pt,linecolor=red]{->}(0.5,0.5)(-3.,0.5)
\rput[tl](-1.8,1){\red{distance $=$ 4}}
\psline[linewidth=2.pt,linecolor=red]{->}(3.5,0.5)(1.,0.5)
\psline[linewidth=2.pt,linecolor=red]{->}(3.5,0.5)(5.,0.5)
\rput[tl](1.8,1){\red{distance $=$ 4}}
\psdots[dotstyle=*,linecolor=blue](1.,0.)
\psdots[dotstyle=*,linecolor=green](5.,0.)
\psdots[dotstyle=*,linecolor=green](-3.,0.)
\end{pspicture*}
\end{center}


On voit qu'il y a deux solutions~: $x=5$ et $x=-3.$

\end{multicols}
\item On résout l'équation $|x+2|=2.$

\setlength{\columnseprule}{1pt}
\begin{multicols}{2}

\textbf{Méthode n°1~: avec la définition de la valeur absolue.}

\medskip

Les nombres dont la valeur absolue vaut 2 sont $2$ et $-2.$

Donc dire que \[|x+2|=2\] revient à dire que
\[x+2=2\qquad\text{ou que}\qquad x+2=-2\]

Donc

\begin{alignat*}{3}
&x+\cancel{2}-\cancel{2}=2-2&&\text{\qquad ou \qquad} x+\cancel{2}-\cancel{2}=-2-2\\
&x=0&&\text{\qquad ou \qquad} x=-4\\
\end{alignat*}

Conclusion~: l'équation a deux solutions~: $x=0$ et $x=-4.$

\columnbreak

\textbf{Méthode n°2~: avec la distance.}

\medskip

Il y a une vraie difficulté~: l'égalité $|x+2|=2$ se réécrit \[|x-(-2)|=2\] (il faut absolument faire apparaître un \og $-$ \fg~{} pour pouvoir interpréter en termes de distance). Donc dire que \[|x+2|=2\] revient à dire que la distance entre $x$ et $-2$ est égale à $2.$


\begin{center}
\psset{xunit=1.0cm,yunit=1.0cm,algebraic=true,dimen=middle,dotstyle=o,dotsize=5pt 0,linewidth=1.6pt,arrowsize=3pt 2,arrowinset=0.25}
\begin{pspicture*}(-4.45,-0.41034356916185444)(1.268763307234229,1.1013400300795446)
\psaxes[labelFontSize=\scriptstyle,xAxis=true,yAxis=false,Dx=0.5,Dy=0.5,ticksize=-2pt 0,subticks=2]{->}(0,0)(-4.45,-0.41034356916185444)(1.268763307234229,1.1013400300795446)
\psline[linewidth=2.pt,linecolor=red]{->}(-3,0.5)(-4.,0.5)
\psline[linewidth=2.pt,linecolor=red]{->}(-3,0.5)(-2.,0.5)
\rput[tl](-3.5,1){\red{dist. $=$ 2}}
\psline[linewidth=2.pt,linecolor=red]{->}(-1,0.5)(-2.,0.5)
\psline[linewidth=2.pt,linecolor=red]{->}(-1,0.5)(0.,0.5)
\rput[tl](-1.5,1){\red{dist. $=$ 2}}
\psdots[dotstyle=*,linecolor=blue](-2.,0.)
\psdots[dotstyle=*,linecolor=green](-4.,0.)
\psdots[dotstyle=*,linecolor=green](0.,0.)
\end{pspicture*}
\end{center}


On voit qu'il y a deux solutions~: $x=0$ et $x=-4.$

\end{multicols}
\item 

On résout l'équation $|x-2|=|x-6|.$

Conformément à l'indication, on travaille avec la distance~:  dire que $|x-2|=|x-6|,$ c'est dire que la distance entre $x$ et $2$ est la même que la distance entre $x$ et $6.$ Autrement dit, $x$ est à égale distance de $2$ et de $6.$ Il y a un seul nombre $x$ qui convienne~: le milieu de l'intervalle $\left[2;6\right],$ c'est-à-dire $x=4.$

\begin{center}
\psset{xunit=1.0cm,yunit=1.0cm,algebraic=true,dimen=middle,dotstyle=o,dotsize=5pt 0,linewidth=1.6pt,arrowsize=3pt 2,arrowinset=0.25}
\begin{pspicture*}(1,-0.41034356916185444)(7,1.1013400300795446)
\psaxes[labelFontSize=\scriptstyle,xAxis=true,yAxis=false,Dx=0.5,Dy=0.5,ticksize=-2pt 0,subticks=2]{->}(0,0)(1,-0.41034356916185444)(7,1.1013400300795446)
\psline[linewidth=2.pt,linecolor=red]{->}(4.5,0.5)(6.,0.5)
\psline[linewidth=2.pt,linecolor=red]{->}(4.5,0.5)(4.,0.5)
\rput[tl](1.8,1){\red{4 est à égale distance de 2 et 6}}
\psline[linewidth=2.pt,linecolor=red]{->}(2.5,0.5)(2.,0.5)
\psline[linewidth=2.pt,linecolor=red]{->}(2.5,0.5)(4.,0.5)
\psdots[dotstyle=*,linecolor=green](4.,0.)
\psdots[dotstyle=*,linecolor=blue](6.,0.)
\psdots[dotstyle=*,linecolor=blue](2.,0.)
\end{pspicture*}
\end{center}


Conclusion~: il y a une seule solution, $x=4.$ 

\end{enumerate}
\end{exo}

\begin{exo}

Commençons par deux exemples~:

\begin{itemize}
\item[\textbullet] si $x=3,$ alors $\sqrt{x^2}=\sqrt{3^2}=\sqrt{9}=3.$
\item[\textbullet] si $x=-3,$ alors $\sqrt{x^2}=\sqrt{(-3)^2}=\sqrt{9}=3.$
\end{itemize}

On comprend que quand $x$ est positif, on aura toujours $\sqrt{x^2}=x=|x|~;$ tandis que dans le cas où $x$ est négatif, le signe $-$ \og disparaît \fg~{} lorsqu'on élève au carré, ce qui donne finalement $\sqrt{x^2}=|x|.$

Autrement dit, quel que soit $x$ (y compris si $x=0$), on a l'égalité


\[\sqrt{x^2}=|x|.\]

%\medskip

%\textbf{Remarque~:} Si vous n'êtes pas convaincu calculez $\sqrt{x^2}$ en prenant $x=3,$ puis en prenant $x=-3.$ Dans les deux cas, vous obtenez.

\end{exo}


\begin{exo}

\begin{enumerate}
\item Dire que $|x-2|<3,$ c'est dire que la distance entre $x$ et $2$ est strictement inférieure à $3.$ On voit que les $x$ qui conviennent sont tous les nombres de l'intervalle $\left]-1;5\right[$ (extrémités exclues, puisque l'inégalité est stricte).

\begin{center}
\psset{xunit=1.0cm,yunit=1.0cm,algebraic=true,dimen=middle,dotstyle=o,dotsize=5pt 0,linewidth=1.6pt,arrowsize=3pt 2,arrowinset=0.25}
\begin{pspicture*}(-1.45,-0.41034356916185444)(5.768763307234229,1.1013400300795446)
\psaxes[labelFontSize=\scriptstyle,xAxis=true,yAxis=false,Dx=0.5,Dy=0.5,ticksize=-2pt 0,subticks=2]{->}(0,0)(-1.45,-0.41034356916185444)(5.768763307234229,1.1013400300795446)
\psline[linewidth=2.pt,linecolor=red]{->}(0.5,0.5)(-1.,0.5)
\psline[linewidth=2.pt,linecolor=red]{->}(0.5,0.5)(2.,0.5)
\rput[tl](-0.3,1){\red{distance $=$ 3}}
\psline[linewidth=2.pt,linecolor=red]{->}(3.5,0.5)(2.,0.5)
\psline[linewidth=2.pt,linecolor=red]{->}(3.5,0.5)(5.,0.5)
\rput[tl](2.7,1){\red{distance $=$ 3}}
\psline[linewidth=3.pt,linecolor=green](-1,0.)(5.,0.)
\rput[tl](4.88,0.32){\green{\Huge [}}
\rput[tl](-1.12,0.32){\green{\Huge ]}}
\psdots[dotstyle=*,linecolor=blue](2.,0.)
\end{pspicture*}
\end{center}

\item Les points de l'intervalle ci-dessous sont les nombres $x$ dont la distance à $8$ est inférieure ou égale à $2$ (donc extrémités incluses)~; autrement dit, ce sont les nombres $x$ tels que \[|x-8|\leq 2.\]

\begin{center}
\psset{xunit=1.0cm,yunit=1.0cm,algebraic=true,dimen=middle,dotstyle=o,dotsize=5pt 0,linewidth=1.6pt,arrowsize=3pt 2,arrowinset=0.25}
\begin{pspicture*}(4,-0.5)(12,1.10)
\psaxes[labelFontSize=\scriptstyle,xAxis=true,yAxis=false,Dx=1.,Dy=1.,ticksize=-2pt 0,subticks=2]{->}(0,0)(4,-0.5)(12,0.52)
\psline[linewidth=3.pt,linecolor=green](6.,0.)(10.,0.)
\rput[tl](5.88,0.28){\huge \green{$\mathbf{{[}}$}}
\rput[tl](9.88,0.28){\huge \green{$\mathbf{{]}}$}}
\psline[linewidth=2.pt,linecolor=red]{->}(8.5,0.5)(10.,0.5)
\psline[linewidth=2.pt,linecolor=red]{->}(8.5,0.5)(8.,0.5)
\rput[tl](6.3,1){\red{dist. $=$ 2}}
\psline[linewidth=2.pt,linecolor=red]{->}(6.5,0.5)(6.,0.5)
\psline[linewidth=2.pt,linecolor=red]{->}(6.5,0.5)(8.,0.5)
\rput[tl](8.2,1){\red{dist. $=$ 2}}
\psdots[dotstyle=*,linecolor=blue](8.,0.)
\end{pspicture*}
\end{center}

\end{enumerate}

\end{exo}

\begin{exo}

Le but de l'exercice est de prouver que  $\sqrt{2}$ n'est pas un nombre rationnel. Pour cela, on fait un raisonnement par l'absurde~: on suppose que $\sqrt{2}$ est rationnel, c'est-à-dire qu'on peut l'écrire sous forme de fraction irréductible $\sqrt{2}=\frac{p}{q},$ où $p$ et $q$ sont deux entiers strictement positifs. Il faut, partant de là, aboutir à une absurdité.

\begin{enumerate}
\item On part de l'égalité  $\sqrt{2}=\frac{p}{q},$ on élève au carré et on multiplie par $q^2~:$

\begin{align*}
\sqrt{2}^2&=\left(\frac{p}{q}\right)^2\\
2&=\frac{p}{q}\times \frac{p}{q}\\
2&=\frac{p^2}{q^2}\\
2\textcolor{red}{\times q^2}&=\frac{p^2}{\cancel{q^2}}\textcolor{red}{\times \cancel{q^2}}\\
2q^2&=p^2
\end{align*}
\item Commençons par un exemple~: prenons un nombre qui \og se termine par 4 \fg~{} (donc le chiffre des unités est 4). Le carré de ce nombre va \og se terminer par 6 \fg, puisque $4^2=16.$ Autrement dit, le chiffre des unités du carré est $6.$

Avec la même technique, on voit que si le chiffre des unités est 9, celui du carré est 1 (puisque $9^2=81$)~; etc. On remplit ainsi le tableau~:

\begin{center}
\begin{tabular}{|l|c|c|c|c|c|c|c|c|c|c|}
\hline
   Chiffre des unités de $p$ &0&1&2&3&4&5&6&7&8&9 \\
	\hline
	Chiffre des unités de $p^2$ &0&1&4&9&6&5&6&9&4&1 \\
	\hline
\end{tabular}
\end{center}
\item Pour avoir le chiffre des unités de $2q^2,$ il suffit de reprendre la deuxième ligne du tableau précédent et de multiplier par $2.$ Par exemple, si le chiffre des unités de $q$ est 7, alors celui de $q^2$ est $9~;$ et celui de $2q^2$ est 8 (puisque $2\times 9=18$). On remplit ainsi le nouveau tableau~:

\begin{center}
\begin{tabular}{|l|c|c|c|c|c|c|c|c|c|c|}
\hline
   Chiffre des unités de $q$ &0&1&2&3&4&5&6&7&8&9 \\
	\hline
	Chiffre des unités de $2q^2$ &0&2&8&8&2&0&2&8&8&2 \\
	\hline
\end{tabular}
\end{center}
\item D'après la question 1, $2q^2=p^2.$ Les nombres $2q^2$ et $p^2$ étant égaux, ils ont le même chiffre des unités. Or dans nos deux tableaux, le seul chiffre en commun  des deuxièmes lignes est le $0~;$ et on l'obtient lorsque le chiffre des unités de $p$ est $0,$ et lorsque le chiffre des unités de $q$ est 0 ou 5.
\item Supposons que $\sqrt{2}$ soit rationnel~: on peut donc l'écrire sous forme de fraction irréductible $\sqrt{2}=\frac{p}{q}.$ D'après la question précédente, $p$ se termine par $0$ et $q$ se termine par $0$ ou $5.$ Mais alors $p$ et $q$ sont tous deux multiples de $5,$ et donc la fraction $\frac{p}{q}$ n'est pas irréductible, en contradiction avec l'hypothèse que nous avons faite au départ.

\medskip

Conclusion~: supposant que $\sqrt{2}$ était rationnel, on aboutit à une absurdité~; c'est donc que $\sqrt{2}$ est irrationnel~: $\sqrt{2}\notin\mathbb{Q}.$
\end{enumerate}

\end{exo}



\section{Géométrie repérée}




\begin{exo}

\begin{enumerate}
\item ~{}


\begin{center}
\newrgbcolor{ududff}{0.30196078431372547 0.30196078431372547 1.}
\psset{xunit=1.0cm,yunit=1.0cm,algebraic=true,dimen=middle,dotstyle=o,dotsize=5pt 0,linewidth=2.pt,arrowsize=3pt 2,arrowinset=0.25}
\begin{pspicture*}(-4.3,-3.48)(5.18,2.74)
\multips(0,-3)(0,1.0){7}{\psline[linestyle=dashed,linecap=1,dash=1.5pt 1.5pt,linewidth=0.4pt,linecolor=lightgray]{c-c}(-4.3,0)(5.18,0)}
\multips(-4,0)(1.0,0){10}{\psline[linestyle=dashed,linecap=1,dash=1.5pt 1.5pt,linewidth=0.4pt,linecolor=lightgray]{c-c}(0,-3.48)(0,2.74)}
\psaxes[labelFontSize=\scriptstyle,xAxis=true,yAxis=true,Dx=1.,Dy=1.,ticksize=-2pt 0,subticks=2]{->}(0,0)(-4.3,-3.48)(5.18,2.74)
\psline[linewidth=2.pt,linecolor=red](1.,2.)(2.5,0.)
\psline[linewidth=2.pt,linecolor=red](1.816,1.112)(1.624,0.968)
\psline[linewidth=2.pt,linecolor=red](1.876,1.032)(1.684,0.888)
\psline[linewidth=2.pt,linecolor=red](2.5,0.)(4.,-2.)
\psline[linewidth=2.pt,linecolor=red](3.316,-0.888)(3.124,-1.032)
\psline[linewidth=2.pt,linecolor=red](3.376,-0.968)(3.184,-1.112)
\psline[linewidth=2.pt,linecolor=red](-4.,2.)(-1.,-0.5)
\psline[linewidth=2.pt,linecolor=red](-2.4231778720402626,0.8421865535516854)(-2.576822127959738,0.657813446448315)
\psline[linewidth=2.pt,linecolor=red](-1.,-0.5)(2.,-3.)
\psline[linewidth=2.pt,linecolor=red](0.5768221279597381,-1.6578134464483147)(0.4231778720402625,-1.8421865535516846)
\psdots[dotstyle=*,linecolor=ududff](1.,2.)
\rput[bl](1.08,2.2){\ududff{$A$}}
\psdots[dotstyle=*,linecolor=ududff](4.,-2.)
\rput[bl](4.08,-1.8){\ududff{$B$}}
\psdots[dotsize=4pt 0,dotstyle=*,linecolor=green](2.5,0.)
\rput[bl](2.58,0.16){\green{$I$}}
\psdots[dotstyle=*,linecolor=ududff](-4.,2.)
\rput[bl](-3.92,2.2){\ududff{$C$}}
\psdots[dotstyle=*,linecolor=ududff](2.,-3.)
\rput[bl](2.08,-2.8){\ududff{$D$}}
\psdots[dotsize=4pt 0,dotstyle=*,linecolor=green](-1.,-0.5)
\rput[bl](-1.3,-0.74){\green{$J$}}
\end{pspicture*}
\end{center}

\begin{enumerate}
\item On a $A(\underset{x_A}{1}~;~\underset{y_A}{2})$ et $B(\underset{x_B}{4}~;~\underset{y_B}{-2}).$ On calcule les coordonnées de $I~:$
\[I\left(\frac{x_A+x_B}{2};\frac{y_A+y_B}{2}\right)\qquad I\left(\frac{1+4}{2};\frac{2+(-2)}{2}\right)\qquad I\left(\frac{5}{2};\frac{0}{2}\right)\qquad I\left(2,5;0\right).\]
\item \[AB=\sqrt{\left(x_B-x_A\right)^2+\left(y_B-y_A\right)^2}=\sqrt{\left(4-1\right)^2+\left(-2-2\right)^2}=\sqrt{3^2+\left(-4\right)^2}=\sqrt{9+16}=\sqrt{25}=5.\]
\end{enumerate}
\item 
\begin{enumerate}
\item On a $C(\underset{x_C}{-4}~;~\underset{y_C}{2})$ et $D(\underset{x_D}{2}~;~\underset{y_D}{-3}).$ On calcule les coordonnées de $J~:$
\[J\left(\frac{x_C+x_D}{2};\frac{y_C+y_D}{2}\right)\qquad J\left(\frac{-4+2}{2};\frac{2+(-3)}{2}\right)\qquad J\left(\frac{-2}{2};\frac{-1}{2}\right)\qquad J\left(-1;-0,5\right).\]
\item \[CD=\sqrt{\left(x_D-x_C\right)^2+\left(y_D-y_C\right)^2}=\sqrt{\left(2-(-4)\right)^2+\left(-3-2\right)^2}=\sqrt{6^2+\left(-5\right)^2}=\sqrt{36+25}=\sqrt{61}.\]
\end{enumerate}

\end{enumerate}
\end{exo}

\begin{exo}


\begin{enumerate}
\item ~{}


\begin{center}
\newrgbcolor{ududff}{0.30196078431372547 0.30196078431372547 1.}
\psset{xunit=1.0cm,yunit=1.0cm,algebraic=true,dimen=middle,dotstyle=o,dotsize=5pt 0,linewidth=2.pt,arrowsize=3pt 2,arrowinset=0.25}
\begin{pspicture*}(-1.5,-2.64)(8.72,5.56)
\multips(0,-2)(0,1.0){9}{\psline[linestyle=dashed,linecap=1,dash=1.5pt 1.5pt,linewidth=0.4pt,linecolor=lightgray]{c-c}(-1.5,0)(8.72,0)}
\multips(-1,0)(1.0,0){11}{\psline[linestyle=dashed,linecap=1,dash=1.5pt 1.5pt,linewidth=0.4pt,linecolor=lightgray]{c-c}(0,-2.64)(0,5.56)}
\psaxes[labelFontSize=\scriptstyle,xAxis=true,yAxis=true,Dx=1.,Dy=1.,ticksize=-2pt 0,subticks=2]{->}(0,0)(-1.5,-2.64)(8.72,5.56)
\psline[linewidth=2.pt](0.,-2.)(5.,0.)
\psline[linewidth=2.pt](5.,0.)(7.,5.)
\psline[linewidth=2.pt](7.,5.)(2.,3.)
\psline[linewidth=2.pt](2.,3.)(0.,-2.)
\psline[linewidth=2.pt,linecolor=red](0.,-2.)(3.5,1.5)
\psline[linewidth=2.pt,linecolor=red](1.6297918471982866,-0.2005025253169425)(1.7994974746830583,-0.37020815280171304)
\psline[linewidth=2.pt,linecolor=red](1.7005025253169408,-0.12979184719828712)(1.8702081528017125,-0.2994974746830577)
\psline[linewidth=2.pt,linecolor=red](3.5,1.5)(7.,5.)
\psline[linewidth=2.pt,linecolor=red](5.129791847198287,3.2994974746830588)(5.299497474683057,3.129791847198287)
\psline[linewidth=2.pt,linecolor=red](5.200502525316942,3.3702081528017134)(5.3702081528017125,3.2005025253169417)
\psline[linewidth=2.pt,linecolor=red](2.,3.)(3.5,1.5)
\psline[linewidth=2.pt,linecolor=red](2.8348528137423856,2.3348528137423856)(2.665147186257614,2.165147186257614)
\psline[linewidth=2.pt,linecolor=red](3.5,1.5)(5.,0.)
\psline[linewidth=2.pt,linecolor=red](4.334852813742385,0.8348528137423862)(4.165147186257615,0.6651471862576144)
\psdots[dotstyle=*,linecolor=ududff](0.,-2.)
\rput[bl](-0.32,-1.74){\ududff{$A$}}
\psdots[dotstyle=*,linecolor=ududff](7.,5.)
\rput[bl](7.08,5.2){\ududff{$B$}}
\psdots[dotstyle=*,linecolor=ududff](2.,3.)
\rput[bl](2.08,3.2){\ududff{$C$}}
\psdots[dotstyle=*,linecolor=ududff](5.,0.)
\rput[bl](5.2,0.2){\ududff{$D$}}
\psdots[dotsize=4pt 0,dotstyle=*,linecolor=darkgray](3.5,1.5)
\end{pspicture*}
\end{center}
\item On calcule les coordonnées de $M~:$
\[M\left(\frac{x_A+x_B}{2};\frac{y_A+y_B}{2}\right)\qquad M\left(\frac{0+7}{2};\frac{-2+5}{2}\right)\qquad M\left(\frac{7}{2};\frac{3}{2}\right)\qquad M\left(3,5;1,5\right).\]
Puis celles de $M'~:$

\[M'\left(\frac{x_C+x_D}{2};\frac{y_C+y_D}{2}\right)\qquad M'\left(\frac{2+5}{2};\frac{3+0}{2}\right)\qquad M'\left(\frac{7}{2};\frac{3}{2}\right)\qquad M'\left(3,5;1,5\right).\]
\item On constate dans la question précédente que $M=M',$ les diagonales $\left[AB\right]$ et $\left[CD\right]$ du quadrilatère $ACBD$ se coupent donc en leur milieu. D'après une propriété du collège, cela entraîne que $ACBD$ est un parallélogramme, puis que ses côtés opposés sont de même longueur~: $BD=AC,$ $CB=AD.$
\item On calcule les longueurs $AC$ et $CB~:$

\begin{itemize}
\item[\textbullet] $AC=\sqrt{\left(x_C-x_A\right)^2+\left(y_C-y_A\right)^2}=\sqrt{\left(2-0\right)^2+\left(3-(-2)\right)^2}=\sqrt{2^2+5^2}=\sqrt{4+25}=\sqrt{29}.$
\item[\textbullet] $CB=\sqrt{\left(x_B-x_C\right)^2+\left(y_B-y_C\right)^2}=\sqrt{\left(7-2\right)^2+\left(5-3\right)^2}=\sqrt{5^2+2^2}=\sqrt{25+4}=\sqrt{29}.$
\end{itemize}

On constate que $AC=CB,$ donc d'après la question précédente~:
\[BD=AC=CB=AD.\]

Conclusion~: le quadrilatère $ACBD$ a quatre côtés de même longueur, donc c'est un losange.

\end{enumerate}

\end{exo}

\begin{exo}~{}


\begin{center}
\newrgbcolor{ududff}{0.30196078431372547 0.30196078431372547 1.}
\newrgbcolor{xfqqff}{0.4980392156862745 0. 1.}
\psset{xunit=1.0cm,yunit=1.0cm,algebraic=true,dimen=middle,dotstyle=o,dotsize=5pt 0,linewidth=2.pt,arrowsize=3pt 2,arrowinset=0.25}
\begin{pspicture*}(-3.14,-3.72)(3.64,4.48)
\multips(0,-3)(0,1.0){9}{\psline[linestyle=dashed,linecap=1,dash=1.5pt 1.5pt,linewidth=0.4pt,linecolor=lightgray]{c-c}(-3.14,0)(3.64,0)}
\multips(-3,0)(1.0,0){7}{\psline[linestyle=dashed,linecap=1,dash=1.5pt 1.5pt,linewidth=0.4pt,linecolor=lightgray]{c-c}(0,-3.72)(0,4.48)}
\psaxes[labelFontSize=\scriptstyle,xAxis=true,yAxis=true,Dx=1.,Dy=1.,ticksize=-2pt 0,subticks=2]{->}(0,0)(-3.14,-3.72)(3.64,4.48)
\pspolygon[linewidth=2.pt,linecolor=xfqqff,fillcolor=xfqqff!20!white,fillstyle=solid,opacity=0.1](-1.7454415587728431,0.6605887450304574)(-1.4060303038033002,0.9151471862576144)(-1.6605887450304573,1.2545584412271569)(-2.,1.)
\psline[linewidth=2.pt](1.,-3.)(-2.,1.)
\psline[linewidth=2.pt](-0.566,-1.112)(-0.374,-0.968)
\psline[linewidth=2.pt](-0.626,-1.032)(-0.434,-0.888)
\psline[linewidth=2.pt](-2.,1.)(2.,4.)
\psline[linewidth=2.pt](-0.112,2.566)(0.032,2.374)
\psline[linewidth=2.pt](-0.032,2.626)(0.112,2.434)
\psline[linewidth=2.pt](2.,4.)(1.,-3.)
\psdots[dotstyle=*,linecolor=ududff](2.,4.)
\rput[bl](2.08,4.16){\ududff{$A$}}
\psdots[dotstyle=*,linecolor=ududff](-2.,1.)
\rput[bl](-2.34,1.14){\ududff{$B$}}
\psdots[dotstyle=*,linecolor=ududff](1.,-3.)
\rput[bl](1.2,-2.96){\ududff{$C$}}
\end{pspicture*}
\end{center}

On calcule la longueur des trois côtés~:

\begin{itemize}
\item[\textbullet] $AB=\sqrt{\left(x_B-x_A\right)^2+\left(y_B-y_A\right)^2}=\sqrt{\left(-2-2\right)^2+\left(1-4\right)^2}=\sqrt{(-4)^2+(-3)^2}=\sqrt{16+9}=\sqrt{25}=5.$

\item[\textbullet] $AC=\sqrt{\left(x_C-x_A\right)^2+\left(y_C-y_A\right)^2}=\sqrt{\left(1-2\right)^2+\left(-3-4\right)^2}=\sqrt{(-1)^2+(-7)^2}=\sqrt{1+49}=\sqrt{50}.$
\item[\textbullet] $BC=\sqrt{\left(x_C-x_B\right)^2+\left(y_C-y_B\right)^2}=\sqrt{\left(1-(-2)\right)^2+\left(-3-1\right)^2}=\sqrt{3^2+\left(-4\right)^2}=\sqrt{9+16}=\sqrt{25}=5.$
\end{itemize}


$AB=BC,$ donc $ABC$ est isocèle en $B.$ On utilise le théorème réciproque de Pythagore pour prouver qu'il est rectangle~:

\[
\left.
    \begin{array}{ll}
        AC^2=\sqrt{50}^2=50\\
        AB^2+BC^2=5^2+5^2=25+25=50
    \end{array}
\right \}AC^2=AB^2+BC^2.
\]
D'après le théorème réciproque de Pythagore, $ABC$ est rectangle en $B.$



\end{exo}





\begin{exo}


\begin{enumerate}
\item \begin{itemize}
\item[\textbullet] $AB=\sqrt{\left(x_B-x_A\right)^2+\left(y_B-y_A\right)^2}
= \sqrt{(1-6)^2+(5-0)^2}
=\sqrt{(-5)^2+5^2}
=\sqrt{25+25}
=\sqrt{50}.$

\item[\textbullet] $AC=\sqrt{\left(x_C-x_A\right)^2+\left(y_C-y_A\right)^2}
= \sqrt{(0-6)^2+(2-0)^2}
=\sqrt{(-6)^2+2^2}
=\sqrt{36+4}
=\sqrt{40}.$
\item[\textbullet] $BC=\sqrt{\left(x_C-x_B\right)^2+\left(y_C-y_B\right)^2}
= \sqrt{(0-1)^2+(2-5)^2}
=\sqrt{(-1)^2+(-3)^2}
=\sqrt{1+9}
=\sqrt{10}.$
\end{itemize}
On a donc~:
\[
\left.
    \begin{array}{ll}
        AB^2=\sqrt{50}^2=50\\
        AC^2+BC^2=\sqrt{40}^2+\sqrt{10}^2=40+10=50
    \end{array}
\right \}AB^2=AC^2+BC^2.
\]
D'après le théorème réciproque de Pythagore, $ABC$ est rectangle en $C.$ 
\item D'après la formule du cours~:
\[I\left(\frac{x_A+x_B}{2};\frac{y_A+y_B}{2}\right)\qquad I\left(\frac{6+1}{2};\frac{0+5}{2}\right)\qquad I\left(3,5~;~2,5\right).\]
\item Le triangle $ABC$ étant rectangle en $C,$ le milieu $I$ de l'hypoténuse $\left[AB\right]$ est le centre de $\Gamma$ (rappel de l'énoncé)~; et le rayon de $\Gamma$ est 
\[r=IA=\sqrt{\left(x_A-x_I\right)^2+\left(y_A-y_I\right)^2}
= \sqrt{(6-3,5)^2+(0-2,5)^2}
=\sqrt{2,5^2+(-2,5)^2}
=\sqrt{6,25+6,25}
=\sqrt{12,5}.\]
\item Savoir si $H(3,5~;~6)$  appartient à $\Gamma$ revient à savoir si la longueur $IH$ est égale à $r$ ou non. On calcule cette longueur avec la formule du  cours~:
\[IH=\sqrt{(x_H-x_I)^2+(y_H-y_I)^2}
= \sqrt{(3,5-3,5)^2+(6-2,5)^2}
=\sqrt{0^2+3,5^2}
=\sqrt{0+12,25}
=\sqrt{12,25}.\] Comme $\sqrt{12,25}\not=\sqrt{12,5},$ le point H n'appartient pas à $\Gamma.$

\medskip

\textbf{N.B.} La figure est trompeuse, puisqu'on a l'impression que $H$ est sur $\Gamma.$ En réalité, si vous avez pris 1~cm comme unité graphique, le point $H$ est à environ trois cheveux (au sens propre) du cercle.
\end{enumerate}


\begin{center}
\newrgbcolor{xdxdff}{0.49 0.49 1}
\newrgbcolor{xfqqff}{0.4980392156862745 0. 1.}
\psset{xunit=1cm,yunit=1cm,algebraic=true,dimen=middle,dotstyle=o,dotsize=3pt 0,linewidth=0.8pt,arrowsize=3pt 2,arrowinset=0.25}
\begin{pspicture*}(-1.4,-1.76)(7.5,6.5)
\multips(0,-1)(0,1.0){9}{\psline[linestyle=dashed,linecap=1,dash=1.5pt 1.5pt,linewidth=0.4pt,linecolor=lightgray]{c-c}(-1.46,0)(7.66,0)}
\multips(-1,0)(1.0,0){10}{\psline[linestyle=dashed,linecap=1,dash=1.5pt 1.5pt,linewidth=0.4pt,linecolor=lightgray]{c-c}(0,-1.76)(0,6.58)}
\psaxes[labelFontSize=\scriptstyle,xAxis=true,yAxis=true,Dx=1,Dy=1,ticksize=-2pt 0,subticks=2]{->}(0,0)(-1.4,-1.76)(27.04,12.76)
\pspolygon[linewidth=2.pt,linecolor=magenta,fillcolor=xfqqff!20!,fillstyle=solid,opacity=0.25](0.5,1.83)(0.66,2.33)(0.17,2.5)(0,2)
\psline[linewidth=2.pt](0,2)(1,5)
\psline[linewidth=2.pt](1,5)(6,0)
\psline[linewidth=2.pt](0,2)(6,0)
\pscircle[linecolor=green,linewidth=2.pt](3.5,2.5){3.54}
\rput[tl](6.39,2.1){\green{$\Gamma$}}
\psdots[dotstyle=*,linecolor=blue](6,0)
\rput[bl](5.92,0.34){\blue{$A$}}
\psdots[dotstyle=*,linecolor=blue](1,5)
\rput[bl](0.9,5.22){\blue{$B$}}
\psdots[dotstyle=*,linecolor=blue](0,2)
\rput[bl](-0.39,2.17){\blue{$C$}}
\psdots[dotstyle=*,linecolor=blue](3.5,2.5)
\rput[bl](3.59,2.64){\blue{$I$}}
\psdots[dotstyle=x,linecolor=red](3.5,6)
\rput[bl](3.74,5.54){\red{$H$}}
\end{pspicture*}
\end{center}

\end{exo}

\begin{exo}


~{}


\begin{center}
\newrgbcolor{ududff}{0.30196078431372547 0.30196078431372547 1.}
\psset{xunit=1.0cm,yunit=1.0cm,algebraic=true,dimen=middle,dotstyle=o,dotsize=5pt 0,linewidth=2.pt,arrowsize=3pt 2,arrowinset=0.25}
\begin{pspicture*}(-2.94,-3.5)(6.92,4.7)
\multips(0,-3)(0,1.0){9}{\psline[linestyle=dashed,linecap=1,dash=1.5pt 1.5pt,linewidth=0.4pt,linecolor=lightgray]{c-c}(-2.94,0)(6.92,0)}
\multips(-2,0)(1.0,0){10}{\psline[linestyle=dashed,linecap=1,dash=1.5pt 1.5pt,linewidth=0.4pt,linecolor=lightgray]{c-c}(0,-3.5)(0,4.7)}
\psaxes[labelFontSize=\scriptstyle,xAxis=true,yAxis=true,Dx=1.,Dy=1.,ticksize=-2pt 0,subticks=2]{->}(0,0)(-2.94,-3.5)(6.92,4.7)
\psline[linewidth=2.pt](-2.,0.)(0.,4.)
\psline[linewidth=2.pt](0.,4.)(6.,1.)
\psline[linewidth=2.pt](6.,1.)(4.,-3.)
\psline[linewidth=2.pt](4.,-3.)(-2.,0.)
\psline[linewidth=2.pt,linecolor=red](-2.,0.)(2.,0.5)
\psline[linewidth=2.pt,linecolor=red](-0.06449806198638836,0.36287160847617933)(-0.034729725684978785,0.12472491806489926)
\psline[linewidth=2.pt,linecolor=red](0.034729725684978785,0.3752750819351007)(0.06449806198638836,0.1371283915238206)
\psline[linewidth=2.pt,linecolor=red](2.,0.5)(6.,1.)
\psline[linewidth=2.pt,linecolor=red](3.935501938013612,0.8628716084761793)(3.9652702743150225,0.6247249180648992)
\psline[linewidth=2.pt,linecolor=red](4.034729725684978,0.8752750819351006)(4.064498061986389,0.6371283915238205)
\psline[linewidth=2.pt,linecolor=red](2.,0.5)(4.,-3.)
\psline[linewidth=2.pt,linecolor=red](3.0793822301370946,-1.1470511702909565)(2.871003876027224,-1.2661245154965972)
\psline[linewidth=2.pt,linecolor=red](3.1289961239727777,-1.2338754845034037)(2.920617769862907,-1.3529488297090442)
\psline[linewidth=2.pt,linecolor=red](2.,0.5)(0.,4.)
\psline[linewidth=2.pt,linecolor=red](0.9206177698629063,2.147051170290957)(1.1289961239727768,2.266124515496597)
\psline[linewidth=2.pt,linecolor=red](0.8710038760272234,2.233875484503403)(1.079382230137094,2.352948829709043)
\psdots[dotstyle=*,linecolor=ududff](0.,4.)
\rput[bl](0.08,4.2){\ududff{$A$}}
\psdots[dotstyle=*,linecolor=ududff](6.,1.)
\rput[bl](6.08,1.2){\ududff{$B$}}
\psdots[dotstyle=*,linecolor=ududff](4.,-3.)
\rput[bl](4.26,-2.92){\ududff{$C$}}
\psdots[dotstyle=*,linecolor=ududff](-2.,0.)
\rput[bl](-2.24,0.26){\ududff{$D$}}
\psdots[dotsize=4pt 0,dotstyle=*,linecolor=darkgray](2.,0.5)
\end{pspicture*}
\end{center}

\begin{enumerate}
\item \begin{itemize}
\item[\textbullet] Le milieu du segment $\left[AC\right]$ a pour coordonnées
\[\left(\frac{x_A+x_C}{2};\frac{y_A+y_C}{2}\right)\qquad \left(\frac{0+4}{2};\frac{4+(-3)}{2}\right)\qquad \left(2~;~0,5\right).\]
\item[\textbullet] Le milieu du segment $\left[BD\right]$ a pour coordonnées
\[\left(\frac{x_B+x_D}{2};\frac{y_B+y_D}{2}\right)\qquad \left(\frac{6+(-2)}{2};\frac{1+0}{2}\right)\qquad \left(2~;~0,5\right).\]
\end{itemize}
Les diagonales du quadrilatère $ABCD$ se coupent en leur milieu, donc c'est un parallélogramme (propriété du collège).

\medskip

\danger Si vous donnez un nom aux milieux des diagonales \textbf{avant de faire les calculs}, donnez-leur des noms différents~: avant de faire les calculs, on n'a pas encore prouvé que les milieux étaient les mêmes.
\item On calcule la longueur des diagonales~:

\begin{itemize}
\item[\textbullet] $AC=\sqrt{\left(x_C-x_A\right)^2+\left(y_C-y_A\right)^2}
= \sqrt{(4-0)^2+(-3-4)^2}
=\sqrt{4^2+(-7)^2}
=\sqrt{16+49}
=\sqrt{65}.$
\item[\textbullet] $BD=\sqrt{\left(x_D-x_B\right)^2+\left(y_D-y_B\right)^2}
= \sqrt{(-2-6)^2+(0-1)^2}
=\sqrt{(-8)^2+(-1)^2}
=\sqrt{64+1}
=\sqrt{65}.$
\end{itemize}
Les diagonales du parallélogramme $ABCD$ sont de même longueur, donc c'est un rectangle (propriété du collège).
\end{enumerate}

\end{exo}

\begin{exo}

\begin{enumerate}
\item Le symétrique de $2$ par rapport à $5,5$ est $9.$


\begin{center}
\newrgbcolor{ududff}{0.30196078431372547 0.30196078431372547 1.}
\psset{xunit=1.0cm,yunit=1.0cm,algebraic=true,dimen=middle,dotstyle=o,dotsize=5pt 0,linewidth=2.pt,arrowsize=3pt 2,arrowinset=0.25}
\begin{pspicture*}(-0.48,-1.16)(10.34,1.2)
\psaxes[labelFontSize=\scriptstyle,xAxis=true,yAxis=false,Dx=1.,Dy=1.,ticksize=-2pt 0,subticks=2]{->}(0,0)(-0.48,-1.16)(10.34,1.2)
\psline[linewidth=2.pt,linecolor=red](2.,0.)(5.5,0.)
\psline[linewidth=2.pt,linecolor=red](3.7,0.12)(3.7,-0.12)
\psline[linewidth=2.pt,linecolor=red](3.8,0.12)(3.8,-0.12)
\psline[linewidth=2.pt,linecolor=red](5.5,0.)(9.,0.)
\psline[linewidth=2.pt,linecolor=red](7.2,0.12)(7.2,-0.12)
\psline[linewidth=2.pt,linecolor=red](7.3,0.12)(7.3,-0.12)
\psdots[dotstyle=*,linecolor=ududff](2.,0.)
\psdots[dotstyle=*,linecolor=ududff](5.5,0.)
\psdots[dotstyle=*,linecolor=ududff](9.,0.)
\end{pspicture*}
\end{center}
\item On généralise le travail de la question précédente~: $c$ est le symétrique de $a$ par rapport à $b$ lorsque $b$ est le milieu du segment qui va de $a$ à $c.$


\begin{center}
\newrgbcolor{ududff}{0.30196078431372547 0.30196078431372547 1.}
\psset{xunit=1.0cm,yunit=1.0cm,algebraic=true,dimen=middle,dotstyle=o,dotsize=5pt 0,linewidth=2.pt,arrowsize=3pt 2,arrowinset=0.25}
\begin{pspicture*}(-0.48,-1.16)(10.34,1.2)
\psline[linewidth=2.pt,linecolor=red](2.,0.)(5.5,0.)
\psline[linewidth=2.pt,linecolor=red](3.7,0.12)(3.7,-0.12)
\psline[linewidth=2.pt,linecolor=red](3.8,0.12)(3.8,-0.12)
\psline[linewidth=2.pt,linecolor=red](5.5,0.)(9.,0.)
\psline[linewidth=2.pt,linecolor=red](7.2,0.12)(7.2,-0.12)
\psline[linewidth=2.pt,linecolor=red](7.3,0.12)(7.3,-0.12)
\rput[tl](1.7,-0.16){$a$}
\rput[tl](5.22,-0.16){$b$}
\rput[tl](8.74,-0.16){$c$}
\psdots[dotstyle=*,linecolor=ududff](2.,0.)
\psdots[dotstyle=*,linecolor=ududff](5.5,0.)
\psdots[dotstyle=*,linecolor=ududff](9.,0.)
\end{pspicture*}
\end{center}

Autrement dit $b=\frac{a+c}{2},$ ce qui donne $b\textcolor{red}{\times 2}=\frac{a+c}{\cancel{2}}\textcolor{red}{\times \cancel{2}},$ soit $2b=a+c~;$ et donc \[c=2b-a.\]
\item On place $C,$ symétrique du point $A$ par rapport au point $B.$


\begin{center}
\newrgbcolor{ududff}{0.30196078431372547 0.30196078431372547 1.}
\psset{xunit=1.0cm,yunit=1.0cm,algebraic=true,dimen=middle,dotstyle=o,dotsize=5pt 0,linewidth=2.pt,arrowsize=3pt 2,arrowinset=0.25}
\begin{pspicture*}(-2.06,-6.28)(7.,2.7)
\multips(0,-6)(0,1.0){9}{\psline[linestyle=dashed,linecap=1,dash=1.5pt 1.5pt,linewidth=0.4pt,linecolor=lightgray]{c-c}(-2.06,0)(7.,0)}
\multips(-2,0)(1.0,0){10}{\psline[linestyle=dashed,linecap=1,dash=1.5pt 1.5pt,linewidth=0.4pt,linecolor=lightgray]{c-c}(0,-6.28)(0,2.7)}
\psaxes[labelFontSize=\scriptstyle,xAxis=true,yAxis=true,Dx=1.,Dy=1.,ticksize=-2pt 0,subticks=2]{->}(0,0)(-2.06,-6.28)(7.,2.7)
\psline[linewidth=2.pt,linecolor=red](1.,2.)(3.25,-1.75)
\psline[linewidth=2.pt,linecolor=red](2.202174363314129,0.22961413693693053)(1.9963760611431185,0.10613515563432425)
\psline[linewidth=2.pt,linecolor=red](2.2536239388568817,0.14386484436567623)(2.047825636685871,0.02038586306306967)
\psline[linewidth=2.pt,linecolor=red](3.25,-1.75)(5.5,-5.5)
\psline[linewidth=2.pt,linecolor=red](4.452174363314129,-3.520385863063071)(4.246376061143119,-3.6438648443656763)
\psline[linewidth=2.pt,linecolor=red](4.503623938856881,-3.6061351556343246)(4.297825636685871,-3.7296141369369304)
\psdots[dotstyle=*,linecolor=ududff](1.,2.)
\rput[bl](1.08,2.2){\ududff{$A$}}
\psdots[dotstyle=*,linecolor=ududff](3.25,-1.75)
\rput[bl](3.34,-1.56){\ududff{$B$}}
\psdots[dotstyle=*,linecolor=ududff](5.5,-5.5)
\rput[bl](5.58,-5.3){\ududff{$C$}}
\end{pspicture*}
\end{center}

Par définition d'une symétrie centrale, $B$ est le milieu du segment $\left[AC\right],$ donc d'après la formule du cours pour le milieu d'un segment~:
\[x_B=\frac{x_A+x_C}{2}\qquad , \qquad y_B=\frac{y_A+y_C}{2}.\]

Autrement dit, en remplaçant avec les données de l'énoncé~:
\[3,25=\frac{1+x_C}{2}\qquad , \qquad -1,75=\frac{2+y_C}{2}.\]

En raisonnant comme dans la question précédente, on obtient
\[x_C=2\times 3,25-1=5,5\qquad , \qquad y_C=2\times (-1,75)-2=-5,5.\]

Conclusion~: $C(5,5~;~-5,5).$


\end{enumerate}

\end{exo}



\begin{exo}


Cet exercice d'introduction à la notion de vecteur appelle quelques commentaires~:

\begin{enumerate}
\item La télécabine $EFGH$ glisse pour aboutir à la position $IJKL.$ Ce déplacement est appelé \og translation de vecteur $\overrightarrow{AB}$ \fg.
\item Le vecteur $\overrightarrow{AB}$ a été représenté en violet sur la figure, il est égal à chacun des vecteurs $\overrightarrow{EI},$ $\overrightarrow{FJ},$ $\overrightarrow{GK}$ et $\overrightarrow{HL}.$ On peut donc écrire
\[\overrightarrow{AB}=\overrightarrow{EI}=\overrightarrow{FJ}=\overrightarrow{GK}=\overrightarrow{HL}.\]
\item Pour aller de $A$ à $B,$ on avance de 4 carreaux en abscisse et on descend de $1$ carreau en ordonnée~; on dit que $\overrightarrow{AB}$ a pour abscisse $4$ et pour ordonnée $-1.$ On note $\overrightarrow{AB}\begin{pmatrix} 4\\-1 \end{pmatrix}.$
\end{enumerate}

\begin{center}
\newrgbcolor{ududff}{0.30196078431372547 0.30196078431372547 1.}
\newrgbcolor{xdxdff}{0.49019607843137253 0.49019607843137253 1.}
\newrgbcolor{xfqqff}{0.4980392156862745 0. 1.}
\psset{xunit=1.0cm,yunit=1.0cm,algebraic=true,dimen=middle,dotstyle=o,dotsize=5pt 0,linewidth=2.pt,arrowsize=3pt 2,arrowinset=0.25}
\begin{pspicture*}(-3.4,-0.9)(3.86,4.64)
\multips(0,0)(0,1.0){6}{\psline[linestyle=dashed,linecap=1,dash=1.5pt 1.5pt,linewidth=0.4pt,linecolor=lightgray]{c-c}(-3.4,0)(3.86,0)}
\multips(-3,0)(1.0,0){8}{\psline[linestyle=dashed,linecap=1,dash=1.5pt 1.5pt,linewidth=0.4pt,linecolor=lightgray]{c-c}(0,-0.9)(0,4.64)}
\psaxes[labelFontSize=\scriptstyle,xAxis=true,yAxis=true,Dx=1.,Dy=1.,ticksize=-2pt 0,subticks=2]{->}(0,0)(-3.4,-0.9)(3.86,4.64)
\pspolygon[linewidth=2.pt,linecolor=red,fillcolor=red!20!white,fillstyle=solid,opacity=0.1](-3.,3.)(-1.,3.)(-1.,1.)(-3.,1.)
\pspolygon[linewidth=2.pt,linecolor=red,fillcolor=red!20!white,fillstyle=solid,opacity=0.1](1.,2.)(3.,2.)(3.,0.)(1.,0.)
\psline[linewidth=2.pt,linecolor=red](-3.,3.)(-1.,3.)
\psline[linewidth=2.pt,linecolor=red](-1.,3.)(-1.,1.)
\psline[linewidth=2.pt,linecolor=red](-1.,1.)(-3.,1.)
\psline[linewidth=2.pt,linecolor=red](-3.,1.)(-3.,3.)
\psline[linewidth=2.pt,linecolor=red](1.,2.)(3.,2.)
\psline[linewidth=2.pt,linecolor=red](3.,2.)(3.,0.)
\psline[linewidth=2.pt,linecolor=red](3.,0.)(1.,0.)
\psline[linewidth=2.pt,linecolor=red](1.,0.)(1.,2.)
\psline[linewidth=2.pt](-2.,4.)(-2.,3.)
\psline[linewidth=2.pt](2.,3.)(2.,2.)
\psline[linewidth=2.pt,linecolor=xfqqff]{->}(-2.,4.)(2.,3.)
\psline[linewidth=2.pt,linecolor=xfqqff]{->}(-1.,3.)(3.,2.)
\psline[linewidth=2.pt,linecolor=xfqqff]{->}(-3.,3.)(1.,2.)
\psline[linewidth=2.pt,linecolor=xfqqff]{->}(-3.,1.)(1.,0.)
\psline[linewidth=2.pt,linecolor=xfqqff]{->}(-1.,1.)(3.,0.)
\rput[bl](-1.92,4.2){\ududff{$A$}}
\rput[bl](2.08,3.2){\ududff{$B$}}
\rput[bl](-2.92,3.2){\ududff{$E$}}
\rput[bl](-0.92,3.2){\ududff{$F$}}
\rput[bl](-0.92,1.2){\ududff{$G$}}
\rput[bl](-2.92,1.2){\ududff{$H$}}
\rput[bl](1.04,2.1){\ududff{$I$}}
\rput[bl](3.08,2.2){\ududff{$J$}}
\rput[bl](3.08,0.2){\ududff{$K$}}
\rput[bl](1.08,0.1){\ududff{$L$}}
\end{pspicture*}
\end{center}


\end{exo}


\begin{exo}

\begin{enumerate}
\item ~{}


\begin{center}
\newrgbcolor{ududff}{0.30196078431372547 0.30196078431372547 1.}
\psset{xunit=1.0cm,yunit=1.0cm,algebraic=true,dimen=middle,dotstyle=o,dotsize=5pt 0,linewidth=2.pt,arrowsize=3pt 2,arrowinset=0.25}
\begin{pspicture*}(-3.7,-2.36)(5.74,2.84)
\multips(0,-2)(0,1.0){6}{\psline[linestyle=dashed,linecap=1,dash=1.5pt 1.5pt,linewidth=0.4pt,linecolor=lightgray]{c-c}(-3.7,0)(5.74,0)}
\multips(-3,0)(1.0,0){10}{\psline[linestyle=dashed,linecap=1,dash=1.5pt 1.5pt,linewidth=0.4pt,linecolor=lightgray]{c-c}(0,-2.36)(0,2.84)}
\psaxes[labelFontSize=\scriptstyle,xAxis=true,yAxis=true,Dx=1.,Dy=1.,ticksize=-2pt 0,subticks=2]{->}(0,0)(-3.7,-2.36)(5.74,2.84)
\psline[linewidth=2.pt,linecolor=red]{->}(-3.,0.)(1.,2.)
\psline[linewidth=2.pt,linecolor=red]{->}(1.,-2.)(5.,0.)
\psdots[dotstyle=*,linecolor=ududff](-3.,0.)
\rput[bl](-3.34,0.22){\ududff{$A$}}
\psdots[dotstyle=*,linecolor=ududff](1.,2.)
\rput[bl](1.08,2.2){\ududff{$B$}}
\psdots[dotstyle=*,linecolor=ududff](1.,-2.)
\rput[bl](0.7,-1.84){\ududff{$I$}}
\psdots[dotstyle=*,linecolor=ududff](5.,0.)
\rput[bl](5.08,0.2){\ududff{$J$}}
\end{pspicture*}
\end{center}
\item On calcule les coordonnées de $\overrightarrow{AB}~:$
\[\overrightarrow{AB}\begin{pmatrix} x_B-x_A\\y_B-y_A \end{pmatrix}\qquad \overrightarrow{AB}\begin{pmatrix} 1-(-3)\\2-0 \end{pmatrix}\qquad \overrightarrow{AB}\begin{pmatrix} 4\\2 \end{pmatrix}.\]
\item On calcule les coordonnées de $\overrightarrow{IJ}~:$ \[\overrightarrow{IJ}\begin{pmatrix} x_J-x_I\\y_J-y_I \end{pmatrix}\qquad \overrightarrow{IJ}\begin{pmatrix} 5-1\\0-(-2) \end{pmatrix}\qquad \overrightarrow{IJ}\begin{pmatrix} 4\\2 \end{pmatrix}.\]
Conclusion~: les vecteurs $\overrightarrow{AB}$ et $\overrightarrow{IJ}$ ont les mêmes coordonnées, donc ils sont égaux~: $\overrightarrow{AB}=\overrightarrow{IJ}.$
\end{enumerate}

\end{exo}


\newpage
\begin{exo}

On considère les points $A(-3;-2),$ $B(5;-2),$ $C(1;4),$ $D(-1;1),$ $E(3;1),$ $F(5;4).$

\vspace*{-1.2cm}
\begin{center}
\begin{pspicture*}(-3.46,-2.44)(5.42,4.52)
\psset{xunit=0.75cm,yunit=0.75cm,algebraic=true,dimen=middle,dotstyle=o,dotsize=3pt 0,linewidth=0.8pt,arrowsize=3pt 2,arrowinset=0.25}
\multips(0,-2)(0,1.0){7}{\psline[linestyle=dashed,linecap=1,dash=1.5pt 1.5pt,linewidth=0.4pt,linecolor=lightgray]{c-c}(-3.46,0)(5.42,0)}
\multips(-3,0)(1.0,0){9}{\psline[linestyle=dashed,linecap=1,dash=1.5pt 1.5pt,linewidth=0.4pt,linecolor=lightgray]{c-c}(0,-2.44)(0,4.52)}
\psaxes[labelFontSize=\scriptstyle,xAxis=true,yAxis=true,Dx=1,Dy=1,ticksize=-2pt 0,subticks=2]{->}(0,0)(-3.46,-2.44)(5.42,4.52)
\psline(-3,-2)(5,-2)
\psline(-3,-2)(1,4)
\psline(1,4)(5,-2)
\psline(-1,1)(3,1)
\psline(1,4)(5,4)
\psline(3,1)(5,4)
\psline(-3,-2)(3,1)
\psline(-1,1)(5,4)
\psdots[dotstyle=*](-3,-2)
\rput[bl](-3.26,-1.8){$A$}
\psdots[dotstyle=*](5,-2)
\rput[bl](5.08,-1.88){$B$}
\psdots[dotstyle=*](1,4)
\rput[bl](1.08,4.12){$C$}
\psdots[dotstyle=*](-1,1)
\rput[bl](-1.4,1.18){$D$}
\psdots[dotstyle=*](3,1)
\rput[bl](3.3,0.76){$E$}
\psdots[dotstyle=*](5,4)
\rput[bl](5.08,4.12){$F$}
\end{pspicture*}
\end{center}

\vspace*{-1cm}


Il y a trop de possibilités pour que les justifie toutes. Je vais me contenter de donner un couple de vecteurs égaux, avec la justification~; puis donner toutes les autres égalités possibles, mais sans les justifier~:
\begin{enumerate}
\item \textbf{Une égalité et sa justification.}

$\overrightarrow{DC}=\overrightarrow{EF}.$ En effet, ces vecteurs ont les mêmes coordonnées~:
\begin{itemize}
\item[\textbullet] $\overrightarrow{DC}\begin{pmatrix} x_C-x_D\\y_C-y_D \end{pmatrix}\qquad \overrightarrow{DC}\begin{pmatrix} 1-(-1)\\4-1 \end{pmatrix}\qquad \overrightarrow{DC}\begin{pmatrix} 2\\3 \end{pmatrix}.$
\item[\textbullet] $\overrightarrow{EF}\begin{pmatrix} x_F-x_E\\y_F-y_E \end{pmatrix}\qquad \overrightarrow{EF}\begin{pmatrix} 5-3\\4-1 \end{pmatrix}\qquad \overrightarrow{EF}\begin{pmatrix} 2\\3 \end{pmatrix}.$
\end{itemize}
\item \textbf{Toutes les autres égalités.}

$\overrightarrow{CF}=\overrightarrow{DE}\qquad \overrightarrow{FC}=\overrightarrow{ED}\qquad\overrightarrow{CD}=\overrightarrow{FE}\qquad\overrightarrow{DC}=\overrightarrow{AD}\qquad\overrightarrow{EF}=\overrightarrow{AD}\qquad\overrightarrow{DA}=\overrightarrow{CD}\qquad\overrightarrow{DA}=\overrightarrow{FE}
\qquad\overrightarrow{AE}=\overrightarrow{DF}
\qquad\overrightarrow{EA}=\overrightarrow{FD}
\qquad\overrightarrow{CE}=\overrightarrow{EB}
\qquad\overrightarrow{EC}=\overrightarrow{BE}
$

\medskip

\danger Attention à l'ordre des lettres~! Par exemple,  $\overrightarrow{DC}=\overrightarrow{EF},$ mais $\overrightarrow{DC}\not=\overrightarrow{FE}$ (il y a un problème de sens~: le vecteur $\overrightarrow{DC}$ \og monte vers le haut et la droite \fg~{}; tandis que $\overrightarrow{FE}$ \og descend vers le bas et la gauche \fg~{} -- l'erreur se détecte aussi bien sûr en calculant les coordonnées).
\end{enumerate}
\end{exo}

\begin{exo} En physique, un vecteur représente une force, et la longueur (ou norme) du vecteur correspond à l'intensité de la force. L'égalité $\left\|\overrightarrow{P_2}\right\|=2\left\|\overrightarrow{P_1}\right\|$ signifie que la masse 2 a un  poids deux fois plus important que celui de la masse 1.


\begin{center}
\psset{xunit=1.0cm,yunit=1.0cm,algebraic=true,dimen=middle,dotstyle=o,dotsize=5pt 0,linewidth=1.6pt,arrowsize=3pt 2,arrowinset=0.25}
\begin{pspicture*}(-3.86,1.2)(2.38,7.0)
\multips(0,2)(0,1.0){5}{\psline[linestyle=dashed,linecap=1,dash=1.5pt 1.5pt,linewidth=0.4pt,linecolor=lightgray]{c-c}(-3.86,0)(2.38,0)}
\multips(-3,0)(1.0,0){7}{\psline[linestyle=dashed,linecap=1,dash=1.5pt 1.5pt,linewidth=0.4pt,linecolor=lightgray]{c-c}(0,1.2)(0,7.0)}
\psline[linewidth=2.pt]{->}(-3.,5.)(-3.,3.)
\psline[linewidth=2.pt]{->}(-1.,6.)(-1.,2.)
\rput[tl](-2.6,4.5){$\overrightarrow{P_1}$}
\rput[tl](-0.6,4.5){$\overrightarrow{P_2}$}
\rput[tl](-3.46,5.64){masse 1}
\rput[tl](-1.6,6.7){masse 2}
\psdots[dotsize=8pt 0,dotstyle=*](-3.,5.)
\psdots[dotsize=11pt 0,dotstyle=*](-1.,6.)
\end{pspicture*}
\end{center}


\end{exo}

\newpage

\begin{exo}

L'image du triangle $ABC$ par la translation de vecteur $\overrightarrow{u}\begin{pmatrix}5 \\-3\end{pmatrix}$ est le triangle $DEF.$


\begin{center}
\newrgbcolor{ududff}{0.30196078431372547 0.30196078431372547 1.}
\newrgbcolor{xfqqff}{0.4980392156862745 0. 1.}
\psset{xunit=0.75cm,yunit=0.75cm,algebraic=true,dimen=middle,dotstyle=o,dotsize=5pt 0,linewidth=2.pt,arrowsize=3pt 2,arrowinset=0.25}
\begin{pspicture*}(-2.56,-7.42)(7.66,3.48)
\pspolygon[linewidth=2.pt,linecolor=red,fillcolor=red!20!white,fillstyle=solid,opacity=0.1](-2.,2.)(0.,-4.)(1.,3.)
\pspolygon[linewidth=2.pt,linecolor=green,fillcolor=green!20!white,fillstyle=solid,opacity=0.1](3.,-1.)(5.,-7.)(6.,0.)
\psline[linewidth=2.pt,linecolor=red](-2.,2.)(0.,-4.)
\psline[linewidth=2.pt,linecolor=red](0.,-4.)(1.,3.)
\psline[linewidth=2.pt,linecolor=red](1.,3.)(-2.,2.)
\psline[linewidth=2.pt,linecolor=xfqqff]{->}(-2.,2.)(3.,-1.)
\psline[linewidth=2.pt,linecolor=xfqqff]{->}(0.,-4.)(5.,-7.)
\psline[linewidth=2.pt,linecolor=xfqqff]{->}(1.,3.)(6.,0.)
\psline[linewidth=2.pt,linecolor=green](3.,-1.)(5.,-7.)
\psline[linewidth=2.pt,linecolor=green](5.,-7.)(6.,0.)
\psline[linewidth=2.pt,linecolor=green](6.,0.)(3.,-1.)
\rput[tl](3.4,2.5){\xfqqff{$\overrightarrow{u}\begin{pmatrix}5\\-3\end{pmatrix}$}}
\psdots[dotstyle=*,linecolor=ududff](-2.,2.)
\rput[bl](-1.92,2.2){\ududff{$A$}}
\psdots[dotstyle=*,linecolor=ududff](0.,-4.)
\rput[bl](-0.56,-3.8){\ududff{$B$}}
\psdots[dotstyle=*,linecolor=ududff](1.,3.)
\rput[bl](1.08,3.16){\ududff{$C$}}
\psdots[dotstyle=*,linecolor=ududff](3.,-1.)
\rput[bl](3.08,-0.8){\ududff{$D$}}
\psdots[dotstyle=*,linecolor=ududff](5.,-7.)
\rput[bl](5.18,-6.84){\ududff{$E$}}
\psdots[dotstyle=*,linecolor=ududff](6.,0.)
\rput[bl](6.08,0.2){\ududff{$F$}}
\multips(0,-7)(0,1.0){11}{\psline[linestyle=dashed,linecap=1,dash=1.5pt 1.5pt,linewidth=0.4pt,linecolor=gray]{c-c}(-2.56,0)(7.66,0)}
\multips(-2,0)(1.0,0){11}{\psline[linestyle=dashed,linecap=1,dash=1.5pt 1.5pt,linewidth=0.4pt,linecolor=gray]{c-c}(0,-7.42)(0,3.48)}
\psaxes[labelFontSize=\scriptstyle,xAxis=true,yAxis=true,Dx=1.,Dy=1.,ticksize=-2pt 0,subticks=2]{->}(0,0)(-2.56,-7.42)(7.66,3.48)
\end{pspicture*}
\end{center}

\end{exo}







\begin{exo}

Un voyageur de commerce ($=$ un représentant) fait une note de frais pour chaque jour de travail où il utilise sa voiture. Il reçoit une part fixe de 30~\euro, et une indemnité de 0,5~\euro/km.

\medskip

\textbf{Remarque~:} On peut penser que l'indemnité kilométrique sert à rembourser les frais de déplacement (par exemple si le représentant utilise sa propre voiture)~; et que la part fixe sert à payer les repas.

\begin{enumerate}
\item S'il fait 120~km dans la journée, le montant de la note de frais est de \[30+120\times 0,5=30+60=90~\text{\euro}.\]
\item On note $x$ le nombre de km parcourus par le voyageur de commerce, et $f(x)$ le montant de la note de frais. On a alors \[f(x)=30+x\times 0,5=0,5x+30.\]
\item La fonction $f$ est affine, puisque $f(x)=0,5x+30$ (c'est bien une fonction de la forme $f(x)=ax+b,$ avec $a=0,5$ et $b=30$). Sa courbe représentative est donc une droite, que l'on trace à partir d'un tableau de valeurs avec deux valeurs~; par exemple~:

\setlength{\columnseprule}{1pt}

\begin{multicols}{2}
\begin{center}
 \begin{tabular}{|c|c|c|}\hline
$x$& $0$ &$120$ \\ \hline 
$f(x)$&$30$ &$90$  \\ \hline
\end{tabular}
\end{center}

\begin{align*}f(0)&=0,5\times 0+30=30\\
f(120)&=0,5\times 120+30=90\end{align*}

On place les points de coordonnées $(0;30)$ et $(120;90),$ puis on trace la droite -- en réalité un segment, puisqu'on va de 0 à 200 en abscisses.

\end{multicols}

\medskip

\textbf{Remarque~:} On a choisi les valeurs  $0$ et $120,$ mais on peut prendre n'importe quelles valeurs -- l'avantage de $0,$ c'est que le calcul est facile~; et l'avantage de $120,$ c'est qu'on a déjà fait le calcul dans la question 1.

\begin{center}
\newrgbcolor{ududff}{0.30196078431372547 0.30196078431372547 1.}
\psset{xunit=0.025cm,yunit=0.05cm,algebraic=true,dimen=middle,dotstyle=o,dotsize=5pt 0,linewidth=2.pt,arrowsize=3pt 2,arrowinset=0.25}
\begin{pspicture*}(-25.571925933684398,-8.5)(219.39050611863857,137.42562531094046)
\multips(0,0)(0,10.0){15}{\psline[linestyle=dashed,linecap=1,dash=1.5pt 1.5pt,linewidth=0.4pt,linecolor=lightgray]{c-c}(0,0)(219.39050611863857,0)}
\multips(0,0)(20.0,0){13}{\psline[linestyle=dashed,linecap=1,dash=1.5pt 1.5pt,linewidth=0.4pt,linecolor=lightgray]{c-c}(0,0)(0,137.42562531094046)}
\psaxes[labelFontSize=\scriptstyle,xAxis=true,yAxis=true,Dx=20.,Dy=10.,ticksize=-2pt 0,subticks=2]{->}(0,0)(0.,0.)(219.39050611863857,137.42562531094046)
\rput[tl](130,8.092313283315638){km parcourus}
\rput[lt](5.833514073023679,126.71554530972082){\parbox{60.384832012684356 cm}{montant de la \\ note de frais}}
\psline[linewidth=2.pt,linecolor=ududff](0.,30.)(200.,130.)
\psline[linewidth=2.pt,linestyle=dashed,dash=2pt 2pt,linecolor=red](0.,75.)(90.,75.)
\psline[linewidth=2.pt,linestyle=dashed,dash=2pt 2pt,linecolor=red](90.,75.)(90.,0.)
\psdots[dotstyle=*,linecolor=ududff](0.,30.)
\psdots[dotstyle=*,linecolor=ududff](120.,90.)
\end{pspicture*}
\end{center}


\item Le voyageur de commerce a une note de frais de 75~\euro. Pour déterminer le nombre de km parcourus dans la journée, il y a deux méthodes~:

\begin{itemize}
\item[\textbullet] \textbf{Graphiquement.} On voit qu'il a parcouru 90~km (pointillés rouges)\footnote{La méthode graphique est simple, mais la réponse pourrait être imprécise.}.
\item[\textbullet] \textbf{Par le calcul.} On retire les frais fixes~: $90-30=60~\text{\euro}$ d'indemnité kilométrique. Puis, comme chaque km compte pour $0,5~\text{\euro},$ on divise~: $45\div 0,5=45\times 2=90~\text{km}.$\footnote{On peut aussi résoudre l'équation $0,5x+30=75.$}
\end{itemize}
\end{enumerate}

\end{exo}

\begin{exo}


\begin{enumerate}
\item \begin{itemize}
\item[\textbullet] Lorsqu'on télécharge 50 Mo, on paye 3~\euro.
\item[\textbullet] Lorsqu'on télécharge 150 Mo, les 100 premiers coûtent 3~\euro~; et les 50 suivants coûtent $50\times 0,04=2$~\euro. On paye donc au total $3+2=5$~\euro.
\end{itemize}

\item On complète le tableau de valeurs~:



\smallskip

\begin{center}
\begin{tabular}{|l|c|c|c|c|c|}
\hline
   Nombre de Mo &$0$ &$50$ &$100$ &$150$ &$200$ \\
	\hline
	Prix à payer &3&3&3&5&7 \\
	\hline
\end{tabular}
\end{center}

\textbf{Remarque~:} jusqu'à 100~Mo, on paye 3~\euro. Ensuite, chaque nouvelle tranche de 50~Mo est facturée 2~\euro.

\item On construit la courbe qui donne le prix payé en fonction du nombre de Mo téléchargés. Elle est constante sur l'intervalle $\left[0;100\right],$ puis affine sur l'intervalle $\left[100;200\right].$ Il faut donc utiliser une règle pour effectuer le tracé\footnote{On parle de fonction \og affine par morceaux \fg.}.

\begin{center}
\newrgbcolor{ududff}{0.30196078431372547 0.30196078431372547 1.}
\psset{xunit=0.015cm,yunit=0.75cm,algebraic=true,dimen=middle,dotstyle=o,dotsize=5pt 0,linewidth=2.pt,arrowsize=3pt 2,arrowinset=0.25}
\begin{pspicture*}(-47.57871396895785,-0.8165410199556531)(427.5432372505541,7.495654101995574)
\multips(0,0)(0,1.0){9}{\psline[linestyle=dashed,linecap=1,dash=1.5pt 1.5pt,linewidth=0.4pt,linecolor=lightgray]{c-c}(0,0)(427.5432372505541,0)}
\multips(0,0)(50.0,0){10}{\psline[linestyle=dashed,linecap=1,dash=1.5pt 1.5pt,linewidth=0.4pt,linecolor=lightgray]{c-c}(0,0)(0,7.495654101995574)}
\psaxes[labelFontSize=\scriptstyle,xAxis=true,yAxis=true,Dx=50.,Dy=1.,ticksize=-2pt 0,subticks=2]{->}(0,0)(0.,0.)(427.5432372505541,7.495654101995574)
\psline[linewidth=2.pt,linecolor=ududff](0.,3.)(100.,3.)
\psline[linewidth=2.pt,linecolor=ududff](100.,3.)(200.,7.)
\psline[linewidth=2.pt,linestyle=dashed,dash=2pt 2pt,linecolor=red](0.,4.6)(140.,4.6)
\psline[linewidth=2.pt,linestyle=dashed,dash=2pt 2pt,linecolor=red](140.,4.6)(140.,0.)
\rput[tl](205.1042128603103,0.5298004434589823){Nombre de Mo}
\rput[tl](3.152993348115293,6.871263858093134){Prix}
\psdots[dotstyle=*,linecolor=ududff](0.,3.)
\psdots[dotstyle=*,linecolor=ududff](50.,3.)
\psdots[dotstyle=*,linecolor=ududff](100.,3.)
\psdots[dotstyle=*,linecolor=ududff](150.,5.)
\psdots[dotstyle=*,linecolor=ududff](200.,7.)
\end{pspicture*}
\end{center}

\item Il y a deux méthodes~:

\begin{itemize}
\item[\textbullet] \textbf{Graphiquement.} On voit qu'on a téléchargé 140~Mo (pointillés rouges).
\item[\textbullet] \textbf{Par le calcul.} J'ai payé 4,60~\euro, donc $3+1,60$~\euro. J'ai donc téléchargé $1,60\div 0,04=40$~Mo au-delà du 100\up{e}. Autrement dit, j'ai téléchargé 140~Mo.
\end{itemize}
\end{enumerate}



\end{exo}

\begin{exo}

Les gares de Calais et de Boulogne-sur-Mer sont distantes de 30~km. Un train part à 12 h de Boulogne-sur-Mer en direction de Calais et roule à la vitesse de 40~km/h. Un train part de Calais à 12 h 15 et fait route en sens inverse à la vitesse de 60~km/h.

\begin{enumerate}
\item Le train qui part à 12 h de Boulogne-sur-Mer roule à la vitesse de 40~km/h, donc il parcourt 40~km en 60~min. Pour savoir quand il arrive à Calais, on complète un tableau de proportionnalité~:

\begin{center}
\begin{tabular}{|c|c|c|}\hline
temps (en min)& 60&? \\ \hline 
distance (en km)&40& 30 \\ \hline
\end{tabular}
\end{center}

Le train mettra $\frac{60\times 30}{40}=\frac{\np{1800}}{40}=45$~min pour arriver à Calais, donc il y sera à 12 h 45.

\medskip

Pour le train qui part de Calais, le calcul est plus facile~: il roule à 60~km/h, donc parcourt 60~km en 60~min~; et ainsi 30~km en 30~min. Comme il part à 12 h 15, il arrive à 12 h 45 lui aussi.

\medskip

On peut ainsi représenter la marche des deux trains~:

\begin{center}
\psset{xunit=0.75cm,yunit=0.75cm,algebraic=true,dimen=middle,dotstyle=o,dotsize=5pt 0,linewidth=2.pt,arrowsize=3pt 2,arrowinset=0.25}
\begin{pspicture*}(-3.5,-0.96)(11,6.5)
\multips(0,0)(0,1.0){7}{\psline[linestyle=dashed,linecap=1,dash=1.5pt 1.5pt,linewidth=0.4pt,linecolor=lightgray]{c-c}(0,0)(10,0)}
\multips(0,0)(1.0,0){11}{\psline[linestyle=dashed,linecap=1,dash=1.5pt 1.5pt,linewidth=0.4pt,linecolor=lightgray]{c-c}(0,0)(0,6)}
\psaxes[labelFontSize=\scriptstyle,xAxis=true,yAxis=true,labels=none,Dx=1.,Dy=1.,ticksize=-2pt 0,subticks=2]{->}(0,0)(0.,0.)(11,6.5)
\begin{scriptsize}
\rput[tl](-0.4,-0.34){12h}
\rput[tl](1.4,-0.34){12h10}
\rput[tl](3.4,-0.34){12h20}
\rput[tl](5.4,-0.34){12h30}
\rput[tl](7.4,-0.34){12h40}
\rput[tl](9.4,-0.34){12h50}
\rput[tl](-0.3,0.18){0}
\rput[tl](-0.6,2.12){10}
\rput[tl](-0.6,4.14){20}
\rput[tl](-2.1,0.4){\fbox{Boulogne}}
\rput[tl](-2.1,6.4){\fbox{Calais}}
\rput[tl](-0.6,6.18){30}
\rput[tl](4.5,-0.34){\green{12h27}}
\end{scriptsize}
\psline[linewidth=2.pt,linecolor=blue](0.,0.)(9.,6)
\psline[linewidth=2.pt,linecolor=red](0.,6.)(3,6)
\psline[linewidth=2.pt,linecolor=red](3.,6.)(9,0)
\psline[linewidth=2.pt,linestyle=dashed,dash=2pt 2pt,linecolor=green](5.4,3.6)(5.4,0)
\end{pspicture*}
\end{center}

\item Nous allons déterminer l'heure de croisement des trains par le calcul. Graphiquement, cela correspond à l'abscisse du point d'intersection des courbes.

\medskip

\`A 12h15, le train qui part de Boulogne-sur-Mer a parcouru 10~km (facile à vérifier), il est donc à 20~km de Calais. C'est l'heure à laquelle le deuxième train part. Comme l'un roule à 40~km/h et l'autre à 60~km/h, tout se passe comme si un seul train devait parcourir 20~km à la vitesse de $40+60=100$~km/h. On complète un tableau de proportionnalité~:

\begin{center}
\begin{tabular}{|c|c|c|}\hline
temps (en min)& 60&? \\ \hline 
distance (en km)&100& 20 \\ \hline
\end{tabular}
\end{center}

$\frac{60\times 20}{100}=\frac{\np{1200}}{100}=12,$ donc il faudrait 12~min à ce train pour parcourir 20~km. Ainsi, les deux trains se croiseront-ils à \[\text{12 h 15 min}+\text{12 min}=\text{12 h 27 min}.\]
 
\end{enumerate}

\end{exo}



\begin{exo}

Je me contenterai du graphique, donc je ne ferai pas les calculs pour avoir les heures exactes des deux rencontres -- elles s'obtiennent avec les mêmes techniques que dans l'exercice précédent.


\begin{center}
\newrgbcolor{ududff}{0.30196078431372547 0.30196078431372547 1.}
\psset{xunit=0.75cm,yunit=0.75cm,algebraic=true,dimen=middle,dotstyle=o,dotsize=5pt 0,linewidth=2.pt,arrowsize=3pt 2,arrowinset=0.25}
\begin{pspicture*}(-1.4,-0.76)(11.25,7.32)
\multips(0,0)(0,1.0){9}{\psline[linestyle=dashed,linecap=1,dash=1.5pt 1.5pt,linewidth=0.4pt,linecolor=lightgray]{c-c}(-0.0,0)(11.25,0)}
\multips(0,0)(1.0,0){12}{\psline[linestyle=dashed,linecap=1,dash=1.5pt 1.5pt,linewidth=0.4pt,linecolor=lightgray]{c-c}(0,0)(0,7.32)}
\psaxes[labelFontSize=\scriptstyle,xAxis=true,yAxis=true,labels=none,Dx=1.,Dy=1.,ticksize=-2pt 0,subticks=2]{->}(0,0)(0.,0.)(11.25,7.32)
\rput[tl](-0.2,-0.15){9h}
\rput[tl](2.66,-0.15){10h}
\rput[tl](5.72,-0.15){11h}
\rput[tl](8.68,-0.15){12h}
%\rput[tl](-0.4,0.2){A}
%\rput[tl](-0.4,6.14){B}
\rput[tl](-0.4,0.2){0}
\rput[tl](-0.6,1.2){10}
\rput[tl](-0.6,2.2){20}
\rput[tl](-0.6,3.2){30}
\rput[tl](-0.6,4.2){40}
\rput[tl](-0.6,5.2){50}
\rput[tl](-0.6,6.2){60}
\psline[linewidth=2.pt,linecolor=ududff](0.,0.)(4.5,3.)
\psline[linewidth=2.pt,linecolor=ududff](4.5,3.)(6.5,3.)
\psline[linewidth=2.pt,linecolor=ududff](6.5,3.)(11,6.)
\rput[tl](0.34,1.66){\ududff{Cycliste}}
\psline[linewidth=2.pt,linecolor=red](0.,6.)(2.,6.)
\psline[linewidth=2.pt,linecolor=red](2.,6.)(5.,0.)
\psline[linewidth=2.pt,linecolor=red](5.,0.)(6.,0.)
\psline[linewidth=2.pt,linecolor=red](6.,0.)(9.,6.)
\psline[linewidth=2.pt,linestyle=dashed,dash=2pt 2pt,linecolor=green](3.75,2.5)(3.75,0.)
\rput[tl](3.38,-0.15){\green{10h15}}
\psline[linewidth=2.pt,linestyle=dashed,dash=2pt 2pt,linecolor=green](8.,4.)(8.,0.)
\rput[tl](7.4,-0.15){\green{11h40}}
\rput[tl](0.3,6.56){\red{Automobile}}
\rput[tl](-1.3,0.4){\fbox{A}}
\rput[tl](-1.3,6.4){\fbox{B}}
\psdots[dotsize=4pt 0,dotstyle=*,linecolor=green](3.75,2.5)
\psdots[dotsize=4pt 0,dotstyle=*,linecolor=green](8.,4.)
\end{pspicture*}
\end{center}

\end{exo}






\end{document}
