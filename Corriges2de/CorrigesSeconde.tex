\documentclass[10pt]{article}
\usepackage[T1]{fontenc}
\usepackage[utf8]{inputenc}
\usepackage{fourier}
\usepackage[scaled=0.875]{helvet}
\renewcommand{\ttdefault}{lmtt}
\usepackage{amsmath,amssymb,makeidx}
\usepackage[normalem]{ulem}
\usepackage{fancybox}
\usepackage{cancel}
\usepackage{stmaryrd}
\usepackage{ulem}
\usepackage{tabularx}
\usepackage{geometry}
\usepackage{enumerate}
\geometry{hmargin=1.5cm,vmargin=1.5cm}
\usepackage{dcolumn}
\usepackage{textcomp}
\usepackage{lscape}
\usepackage{eurosym}
%\newcommand{\euro}{\eurologo{}}
\usepackage[dvips]{color}
\usepackage[all]{xy}
\usepackage{xlop}

\usepackage{tikz,tkz-tab}

\usepackage{systeme}


\usepackage{pstricks,pst-plot,pst-text,pst-tree,pstricks-add}
\usepackage{colortbl}
\usepackage{diagbox}
\usepackage{fontawesome5}
\usepackage{pifont}
\usepackage{wasysym}


\usepackage{theorem}
\theorembodyfont{\upshape}
\newtheorem{exo}{Exercice}
%\newtheorem{exo}{Exercice}%[section]
\usepackage{hyperref}
\hypersetup{
    colorlinks=true,       % false: liens encadrés; true: liens colorés
    linkcolor=blue,          % couleur des liens (ou bordures) internes
}

%\setlength{\voffset}{-1,5cm}
\usepackage{fancyhdr} 
\usepackage{graphicx}
\usepackage[frenchb]{babel}
\usepackage[np]{numprint}
\usepackage{multicol}
\usepackage{xlop}
\usepackage{soul}


\title{Mathématiques -- Seconde}

\date{Corrigés des exercices}
\begin{document}
\setlength\parindent{0mm}
\renewcommand \footrulewidth{.2pt}

\maketitle

\tableofcontents


\newpage


\section{Rappels de calcul et de géométrie}

\begin{exo}

Dans chaque question, on obtient la réponse à l'aide d'un tableau de proportionnalité.

\begin{enumerate}
\item ~{}
\begin{center}
\begin{tabular}{|c|c|c|}\hline
Nombre de personnes& 4&6 \\ \hline 
Farine (en g)&250& ? \\ \hline
Lait (en mL)&500& ? \\ \hline
Œufs&4& 6 \\ \hline
\end{tabular}
\end{center}

Pour 6 personnes, il faut $\frac{250\times 6}{4}=\frac{\np{1500}}{4}=375$~g de farine, $\frac{500\times 6}{4}=\frac{\np{3000}}{4}=750$~mL de lait et, bien sûr, 6 œufs.

\item Les 6 yaourts pèsent $6\times 125=750$~g.

\begin{center}
\begin{tabular}{|c|c|c|}\hline
masse (en g)& 1000&750 \\ \hline 
prix (en \euro)&2& ? \\ \hline
\end{tabular}
\end{center}

Je payerai $\frac{750\times 2}{\np{1000}}=\frac{\np{1500}}{\np{1000}}=1,5~\text{\euro}.$

\item Généralement, dans ce type de question, il vaut mieux convertir en minutes\footnote{Les calculs ne sont pas toujours plus faciles en minutes qu'en heures, mais c'est généralement le cas.}.

\begin{center}
\begin{tabular}{|c|c|c|}\hline
temps (en min)& 60&? \\ \hline 
distance (en km)&20& 45 \\ \hline
\end{tabular}
\end{center}

On mettra $\frac{60\times 45}{20}=\frac{\cancel{20}\times 3\times 45}{\cancel{20}}=135$~min, soit 2~h~15~min (puisque $135=120+15$).
\item L'énoncé donne les informations recensées dans le tableau ci-dessous et demande de compléter la case \textcircled{\small{1}}.

\begin{center}
\begin{tabular}{|c|c|c|c|}\hline
Florins& 7&?&\textcircled{\small{1}} \\ \hline 
Pistoles&6& \textcolor{red}{4}&\textcircled{\small{\textcolor{black}{2}}} \\ \hline
Deniers&?& \textcolor{red}{5}&\textcolor{red}{30} \\ \hline
\end{tabular}
\end{center}

On complète d'abord la case \textcircled{\small{2}}~: en échange de 30 deniers, on a $4\times 30\div 5=24$~pistoles~:

\begin{center}
\begin{tabular}{|c|c|c|c|}\hline
Florins& \textcolor{red}{7}&?&\textcircled{\small{\textcolor{black}{1}}} \\ \hline 
Pistoles&\textcolor{red}{6}& 4&\textcolor{red}{24} \\ \hline
Deniers&?& 5&30 \\ \hline
\end{tabular}
\end{center}

On peut alors compléter la case \textcircled{\small{1}}~: en échange de 30 deniers, on a $\frac{7\times 24}{6}=\frac{7\times 4\times \cancel{6}}{\cancel{6}}=28$~florins.

\end{enumerate}
\end{exo}

\begin{exo}

\begin{enumerate}
\item On complète deux tableaux de proportionnalité (on travaille en min et en km)~:

\begin{multicols}{2}

\begin{center}
\begin{tabular}{|c|c|c|}\hline
temps (en min)& 60&? \\ \hline 
distance (en km)&3& 0,5 \\ \hline
\end{tabular}


\begin{tabular}{|c|c|c|}\hline
temps (en min)& 60&? \\ \hline 
distance (en km)&15& 5 \\ \hline
\end{tabular}
\end{center}

\end{multicols}

Stéphane nage $\frac{60\times 0,5}{3}=\frac{30}{3}=10$~min, puis il court $\frac{60\times 5}{15}=\frac{300}{15}=20$~min.


\item Stéphane a parcouru un total de $5+0,5=5,5$~km, en $10+20=30$~min.

\begin{center}
\begin{tabular}{|c|c|c|}\hline
temps (en min)& 30&60 \\ \hline 
distance (en km)&5,5& ? \\ \hline
\end{tabular}
\end{center}

La vitesse moyenne de Stéphane sur l’ensemble de son parcours est donc $\frac{60\times 5,5}{30}=\frac{\cancel{30}\times 2\times 5,5}{\cancel{30}}=11$~km/h.
\end{enumerate}
\end{exo}

\begin{exo}%28
~{}

\begin{center}
\newrgbcolor{xfqqff}{0.4980392156862745 0. 1.}
\psset{xunit=1.0cm,yunit=1.0cm,algebraic=true,dimen=middle,dotstyle=o,dotsize=3pt 0,linewidth=0.8pt,arrowsize=3pt 2,arrowinset=0.25}
\begin{pspicture*}(1.64,0.96)(7.78,5)
\pspolygon(2,2)(7,2)(7,4)(4,4)
\pspolygon(7,2)(6.6,2)(6.6,2.4)(7,2.4)
\pspolygon(7,4)(7,3.6)(6.6,3.6)(6.6,4)
\psline(2,2)(7,2)
\psline(7,2)(7,4)
\psline(7,4)(4,4)
\psline(4,4)(2,2)
\psline(4,4)(2,2)
\psline(4,4)(4,2)
\pspolygon[linewidth=1.pt,linecolor=xfqqff,fillcolor=xfqqff!20!white,fillstyle=solid,opacity=0.1](4.,2.4242640687119286)(3.5757359312880714,2.4242640687119286)(3.5757359312880714,2.)(4.,2.)
\psline{->}(5,3)(5,4)
\psline{->}(5,3)(5,2)
\rput[tl](5.22,3.2){$2$}
\rput[tl](5.44,4.4){$3$}
\rput[tl](5.44,1.8){$3$}
\rput[tl](2.84,1.8){$2$}
\rput[bl](1.8,2.18){$A$}
\rput[bl](7.08,2.12){$B$}
\rput[bl](7.08,4.12){$C$}
\rput[bl](4.08,4.12){$D$}
\rput[bl](3.8,1.6){$H$}
\end{pspicture*}
\end{center}

Le trapèze est constitué~:

\begin{itemize}
\item[\textbullet] d'un rectangle $BHDC,$ d'aire $\ell\times L=3\times 2=6~;$
\item[\textbullet] d'un triangle $AHD,$ d'aire $\frac{B\times h}{2}=\frac{2\times 2}{2}=2.$
\end{itemize}
Donc l'aire du trapèze est $6+2=8.$

\medskip

\textbf{Remarque~:} On peut aussi utiliser la formule (hors-programme)~:
\[\mathcal{A}_{\text{trapèze}}=\frac{(B+b)\times h}{2}=\frac{(5+3)\times 2}{2}=8.\]
\end{exo}

\begin{exo}

Le losange est \og la moitié \fg~{} d'un rectangle de côtés $\ell$ et $L,$ donc son aire est $\frac{\ell\times L}{2}.$



\begin{center}
\psset{xunit=1.0cm,yunit=1.0cm,algebraic=true,dimen=middle,dotstyle=o,dotsize=5pt 0,linewidth=1.6pt,arrowsize=3pt 2,arrowinset=0.25}
\begin{pspicture*}(0.0,2.0)(5.24,5.24)
\psline[linewidth=2.pt](1.,4.)(3.,5.)
\psline[linewidth=2.pt](3.,5.)(5.,4.)
\psline[linewidth=2.pt](5.,4.)(3.,3.)
\psline[linewidth=2.pt](3.,3.)(1.,4.)
\psline[linewidth=2.pt,linestyle=dotted](1.,4.)(5.,4.)
\psline[linewidth=2.pt,linestyle=dotted](3.,5.)(3.,3.)
\psline[linewidth=2.pt,linestyle=dotted,linecolor=red](1.,3.)(1.,5.)
\psline[linewidth=2.pt,linestyle=dotted,linecolor=red](1.,3.)(5.,3.)
\psline[linewidth=2.pt,linestyle=dotted,linecolor=red](1.,5.)(5.,5.)
\psline[linewidth=2.pt,linestyle=dotted,linecolor=red](5.,5.)(5.,3.)

\rput[tl](0.65,4.2){\red{$\ell$}}
\rput[tl](2.75,2.86){\red{$L$}}
\end{pspicture*}
\end{center}
\end{exo}



\begin{exo}%29

\textbf{Rappels~:}
\begin{itemize}
\item[\textbullet] une hauteur est une droite qui passe par un sommet et qui est perpendiculaire au côté opposé (les hauteurs sont tracées en pointillés bleus)~;
\item[\textbullet] le fait que les hauteurs soient \og concourantes \fg~{} signifie qu'elles passent toutes les trois par un même point -- qu'on appelle \og orthocentre du triangle \fg~{} (nommé $O$ sur la figure ci-dessous).
\end{itemize}


\begin{center}
\newrgbcolor{xfqqff}{0.4980392156862745 0. 1.}
\psset{xunit=1.0cm,yunit=1.0cm,algebraic=true,dimen=middle,dotstyle=o,dotsize=5pt 0,linewidth=2.pt,arrowsize=3pt 2,arrowinset=0.25}
\begin{pspicture*}(0.58,0.54)(7.34,5.4)
\pspolygon[linewidth=2.pt,linecolor=xfqqff,fillcolor=xfqqff!20!white,fillstyle=solid,opacity=0.1](1.9037508537434635,2.9121465431835385)(2.287329637311028,2.730853088263545)(2.4686230922310215,3.1144318718311097)(2.085044308663457,3.295725326751103)
\pspolygon[linewidth=2.pt,linecolor=xfqqff,fillcolor=xfqqff!20!white,fillstyle=solid,opacity=0.1](3.66,3.66)(3.96,3.36)(4.26,3.66)(3.96,3.96)
\pspolygon[linewidth=2.pt,linecolor=xfqqff,fillcolor=xfqqff!20!white,fillstyle=solid,opacity=0.1](3.3378667606356793,1.0079249720699515)(3.3364285856148856,1.4321866032041017)(2.9121669544807354,1.430748428183308)(2.913605129501529,1.0064867970491578)
\psline[linewidth=2.pt](1.,1.)(6.9,1.02)
\psline[linewidth=2.pt](6.9,1.02)(2.9,5.02)
\psline[linewidth=2.pt](2.9,5.02)(1.,1.)
\psline[linewidth=2.pt,linestyle=dotted,linecolor=blue](2.9,5.02)(2.913605129501529,1.0064867970491578)
\psline[linewidth=2.pt,linestyle=dotted,linecolor=blue](1.,1.)(3.96,3.96)
\psline[linewidth=2.pt,linestyle=dotted,linecolor=blue](6.9,1.02)(2.085044308663457,3.295725326751103)
\psdots[dotsize=4pt 0,dotstyle=*,linecolor=blue](2.907162162162163,2.907162162162163)
\rput[bl](3.06,2.42){\blue{$O$}}
\end{pspicture*}
\end{center}

\end{exo}







\begin{exo}%34

On note $H$ le pied de la hauteur issue de $A$ dans le triangle $ABC.$


\begin{center}
\psset{xunit=1.0cm,yunit=1.0cm,algebraic=true,dimen=middle,dotstyle=o,dotsize=5pt 0,linewidth=2.pt,arrowsize=3pt 2,arrowinset=0.25}
\begin{pspicture*}(-0.02,0.32)(5.98,4.56)
\pspolygon[linewidth=2.pt,linecolor=red,fillcolor=red!10!white,fillstyle=solid,opacity=0.1](2.4242640687119286,1.)(2.4242640687119286,1.4242640687119286)(2.,1.4242640687119286)(2.,1.)
\psline[linewidth=2.pt](2.,4.)(5.,1.)
\psline[linewidth=2.pt](2.,4.)(1.,1.)
\psline[linewidth=2.pt](1.,1.)(5.,1.)
\psline[linewidth=2.pt,linestyle=dotted,linecolor=red](2.,4.)(2.,1.)
\psline[linewidth=2.pt](3.,1.)(2.,4.)
\rput[bl](2.08,4.04){$A$}
\rput[bl](5.,0.6){$B$}
\rput[bl](0.7,0.6){$C$}
\rput[bl](1.96,0.6){$H$}
\rput[bl](3.04,0.6){$I$}
\end{pspicture*}
\end{center}


$\left[AH\right]$ est une hauteur dans les triangles $BIA$ et $CIA,$ donc

\[\mathcal{A}_{BIA}=\frac{\textcolor{red}{BI}\times AH}{2}\hspace{4cm} \mathcal{A}_{CIA}=\frac{\textcolor{red}{CI}\times AH}{2}.\]
Or $\textcolor{red}{BI}=\textcolor{red}{CI}$ puisque $I$ et le milieu de $\left[BC\right],$ donc $BIA$ et $CIA$ ont la même aire.

\end{exo}


\newpage


\begin{exo}%30


\begin{enumerate}
\item La négation de

\begin{center}

 \blue{\underline{Tous}} \black les hommes \blue{\underline{sont mortels}}\black .
 
 \end{center}
 
 est
 
 \begin{center}

 \red{\underline{Il existe}} \black un homme \red{\underline{immortel}}\black .
 
 \end{center}
 
\item La négation de

\begin{center}

 \blue{\underline{Il existe}} \black un dessert \blue{\underline{sans sucre}} \black à la cantine.
 
 \end{center}
 
 est
 
 \begin{center}

 \red{\underline{Tous}} \black les desserts \red{\underline{sont sucrés}} \black à la cantine.
 \end{center}
 
 \medskip

\textbf{Remarque~:} Dans les deux exemples que nous venons de traiter, pour écrire la négation d'une phrase, il suffit de remplacer les \og tous \fg~{} par \og il existe \fg~{}, et réciproquement~; et d'inverser les conclusions (exemple~: immortel/mortel). C'est une technique qui fonctionne toujours.
\item La négation de

\begin{center}

 \blue{\underline{Il existe}} \black un pays dans lequel \blue{\underline{tous}} \black les hommes \black \blue{\underline{savent lire}}\black .
 
 \end{center}
 
 est
 
 \begin{center}

Dans \red{\underline{tous}} \black les pays, \red{\underline{il existe}} \black un homme qui  \red{\underline{ne sait pas lire}}\black .
 \end{center}
\item Le contraire de \og être allé en Angleterre ou en Espagne \fg~{} est \og n'être allé ni en Angleterre, ni en Espagne \fg , donc la négation de

\begin{center}

 \blue{\underline{Tous}} \black les élèves de la classe  \blue{\underline{sont déjà allés en Angleterre ou en Espagne }}\black .
 
 \end{center}
 
 est
 
 \begin{center}

 \red{\underline{Il existe}} \black un élève de la classe qui \red{\underline{n'est jamais allé en Angleterre, ni en Espagne}}\black .
 \end{center}
\item Comme dans l'exemple précédent, le contraire de \og ni... ni... \fg~{} est \og ou \fg . Donc la négation de

\begin{center}

 Chloé \blue{\underline{n'aime ni les fraises, ni les framboises}}\black .
 
 \end{center}
 
 est
 
 \begin{center}

 Chloé \red{\underline{aime les fraises ou les framboises}}\black .
 \end{center}

\end{enumerate}



\end{exo}


\begin{exo}%19


\begin{enumerate}
\item \begin{enumerate}
\item On identifie A et B dans l'implication~:

\begin{center}

Si $\underbrace{\text{un nombre se termine par 5}}_{\text{A}}$, alors $\underbrace{\text{il est multiple de 5}}_{\text{B}}.$

\end{center}

Cette implication est vraie (cours du primaire).

\item

\begin{itemize}
\item[\textbullet] L'implication contraposée est

\begin{center}

Si $\underbrace{\text{un nombre n'est pas multiple de 5}}_{\text{non B}}$, alors $\underbrace{\text{il ne se termine pas par 5}}_{\text{non A}}.$

\end{center}

Cette contraposée est vraie, puisque l'implication originale l'est (cf l'énoncé~: quand une implication est vraie, sa contraposée l'est aussi).

\medskip

\item[\textbullet] L'implication réciproque est

\begin{center}

Si $\underbrace{\text{un nombre est multiple de 5}}_{\text{B}}$, alors $\underbrace{\text{il se termine par 5}}_{\text{A}}.$

\end{center}

Elle est fausse, comme le montre le contre-exemple suivant~: 10 est multiple de 5, mais il ne se termine pas par 5.
\end{itemize}

\end{enumerate}
\item L'implication

\begin{center}

Si $\underbrace{\text{un nombre se termine par 0}}_{\text{A}}$, alors $\underbrace{\text{il est multiple de 10}}_{\text{B}}.$

\end{center}

et sa réciproque

\begin{center}

Si $\underbrace{\text{un nombre est multiple de 10}}_{\text{B}}$, alors $\underbrace{\text{il se termine par 0}}_{\text{A}}.$

\end{center}

sont vraies toutes les deux.

\end{enumerate}
\end{exo}




\end{document}
