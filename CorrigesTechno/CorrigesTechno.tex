\documentclass[10pt]{article}
\usepackage[T1]{fontenc}
\usepackage[utf8]{inputenc}
\usepackage{fourier}
\usepackage[scaled=0.875]{helvet}
\renewcommand{\ttdefault}{lmtt}
\usepackage{amsmath,amssymb,makeidx}
\usepackage[normalem]{ulem}
\usepackage{fancybox}
\usepackage{cancel}
\usepackage{stmaryrd}
\usepackage{ulem}
\usepackage{tabularx}
\usepackage{geometry}
\usepackage{enumerate}
\geometry{hmargin=1.5cm,vmargin=1.5cm}
\usepackage{dcolumn}
\usepackage{textcomp}
\usepackage{lscape}
\usepackage{eurosym}
%\newcommand{\euro}{\eurologo{}}
\usepackage[dvips]{color}
\usepackage[all]{xy}

\usepackage{tikz,tkz-tab}

\usepackage{systeme}
\usepackage{ upgreek }


\usepackage{pstricks,pst-plot,pst-text,pst-tree,pstricks-add}
\usepackage{colortbl}
\usepackage{diagbox}
\usepackage{fontawesome5}
\usepackage{pifont}
\usepackage{wasysym}


\usepackage{theorem}
\theorembodyfont{\upshape}
\newtheorem{exo}{Exercice}
%\newtheorem{exo}{Exercice}%[section]
\usepackage{hyperref}
\hypersetup{
    colorlinks=true,       % false: liens encadrés; true: liens colorés
    linkcolor=blue,          % couleur des liens (ou bordures) internes
}

%\setlength{\voffset}{-1,5cm}
\usepackage{fancyhdr} 
\usepackage{graphicx}
\usepackage[frenchb]{babel}
\usepackage[np]{numprint}
\usepackage{multicol}
\usepackage{xlop}
\usepackage{soul}

\usepackage{etoolbox}
\usepackage{multirow}
\usepackage{diagbox}


\title{Mathématiques -- Première technologique}

\date{Corrigés des exercices}
\begin{document}
\setlength\parindent{0mm}
\renewcommand \footrulewidth{.2pt}

\maketitle

\tableofcontents


\newpage

\section{Proportionnalité}



\begin{exo}



\begin{enumerate}
\item On complète un tableau de proportionnalité~:

\begin{center}
 \begin{tabular}{|m{2cm}|m{1cm}|m{1cm}|}\hline
Élèves& 40 & ? \\ \hline 
Pourcentage&100 & 70\\ \hline

\end{tabular}
\end{center}

Il y a $40\times 70\div 100=28$ garçons dans la classe.

\item On complète un tableau de proportionnalité~:

\begin{center}
 \begin{tabular}{|m{2cm}|m{1cm}|m{1cm}|}\hline
Marins& \np{1760} & \np{1046} \\ \hline 
Pourcentage&100 & ?\\ \hline

\end{tabular}
\end{center}

$\np{1046}\times 100\div \np{1760}\approx 59,43,$ donc environ 59,43~\% des marins sont tombés malades.

\medskip

\textbf{N.B.} On fait le calcul et, seulement après, on écrit la réponse avec le symbole \%. Rappelons à cette occasion la signification de 59,43~\%~:
\[59,43~\%=\dfrac{59,43}{100}=0,5943.\]
Donc dire que  59,43~\% des marins sont tombés malades, c'est dire que la proportion de malades est $\dfrac{59,43}{100}.$
\item Le fait que la bouteille soit titrée à 12~\% vol. signifie qu'elle contient 12~\% d'alcool pur. On complète donc un tableau de proportionnalité~:

\begin{center}
 \begin{tabular}{|m{2.5cm}|m{1cm}|m{1cm}|}\hline
Volume (en mL)& 500 & ? \\ \hline 
Pourcentage&100 & 12\\ \hline

\end{tabular}
\end{center} 
La bouteille contient $500\times 12\div 100=60$~mL d'alcool pur.
\item Sur 100 personnes de l'entreprise, il y a 56 hommes.

25~\% d'entre eux fument, ce qui représente
\[25\times 56\div 100=14~\text{personnes}\] (on peut bien sûr faire un tableau de proportionnalité pour obtenir cette réponse).

Conclusion~: les hommes fumeurs représentent 14~\% du personnel de l'entreprise.
\end{enumerate}

\end{exo}


\begin{exo}

\begin{enumerate}
\item ~{}
\begin{center}
\begin{tabular}{|c|c|c|}\hline
Nombre de personnes& 4&6 \\ \hline 
Farine (en g)&250& ? \\ \hline
Lait (en mL)&500& ? \\ \hline
Œufs&4& 6 \\ \hline
\end{tabular}
\end{center}

Pour 6 personnes, il faut $ 250\times 6\div 4=375$~g de farine, $500\times 6\div 4=750$~mL de lait et, bien sûr, 6 œufs.
\item Les 6 yaourts pèsent $6\times 125=750$~g.

\begin{center}
\begin{tabular}{|c|c|c|}\hline
masse (en g)& 1000&750 \\ \hline 
prix (en \euro)&2& ? \\ \hline
\end{tabular}
\end{center}

Je payerai $750\times 2\div \np{1000}=1,5~\text{\euro}.$
\end{enumerate}
\end{exo}

\begin{exo}

L'énoncé donne les informations recensées dans le tableau ci-dessous et demande de compléter la case \textcircled{\small{1}}.

\begin{center}
\begin{tabular}{|c|c|c|c|}\hline
Florins& 7&?&\textcircled{\small{1}} \\ \hline 
Pistoles&6& \textcolor{red}{4}&\textcircled{\small{\textcolor{black}{2}}} \\ \hline
Deniers&?& \textcolor{red}{5}&\textcolor{red}{30} \\ \hline
\end{tabular}
\end{center}

On complète d'abord la case \textcircled{\small{2}}~: en échange de 30 deniers, on a $4\times 30\div 5=24$~pistoles~:

\begin{center}
\begin{tabular}{|c|c|c|c|}\hline
Florins& \textcolor{red}{7}&?&\textcircled{\small{\textcolor{black}{1}}} \\ \hline 
Pistoles&\textcolor{red}{6}& 4&\textcolor{red}{24} \\ \hline
Deniers&?& 5&30 \\ \hline
\end{tabular}
\end{center}

On peut alors compléter la case \textcircled{\small{1}}~: en échange de 30 deniers, on a $7\times 24\div 6=28$~florins.

\end{exo}

\begin{exo}


\begin{enumerate}

\item Généralement, dans ce type de question, il vaut mieux convertir en minutes\footnote{Les calculs ne sont pas toujours plus faciles en minutes qu'en heures, mais c'est généralement le cas.}.

\begin{center}
\begin{tabular}{|c|c|c|}\hline
temps (en min)& 60&? \\ \hline 
distance (en km)&20& 45 \\ \hline
\end{tabular}
\end{center}

On mettra $60\times 45\div 20=135$~min, soit 2~h~15~min (puisque $135=120+15$).

\item On peut se passer d'un tableau de proportionnalité~: $1~\text{h}=60~\text{min},$ donc $0,6~\text{h}=0,6\times 60~\text{min}=36~\text{min}.$

\item \begin{enumerate}
\item On complète deux tableaux de proportionnalité (on travaille en min et en km)~:

\begin{multicols}{2}

\begin{center}
\begin{tabular}{|c|c|c|}\hline
temps (en min)& 60&? \\ \hline 
distance (en km)&3& 0,5 \\ \hline
\end{tabular}


\begin{tabular}{|c|c|c|}\hline
temps (en min)& 60&? \\ \hline 
distance (en km)&15& 5 \\ \hline
\end{tabular}
\end{center}

\end{multicols}

Stéphane nage $60\times 0,5 \div 3=10$~min, puis il court $60\times 5\div 15=20$~min.


\item Stéphane a parcouru un total de $5+0,5=5,5$~km, en $10+20=30$~min.

\begin{center}
\begin{tabular}{|c|c|c|}\hline
temps (en min)& 30&60 \\ \hline 
distance (en km)&5,5& ? \\ \hline
\end{tabular}
\end{center}

La vitesse moyenne de Stéphane sur l’ensemble de son parcours est donc $60\times 5,5\div 30=11$~km/h.
\end{enumerate}
\end{enumerate}

\end{exo}


\end{document}
