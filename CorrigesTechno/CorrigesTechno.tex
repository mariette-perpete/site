\documentclass[10pt]{article}
\usepackage[T1]{fontenc}
\usepackage[utf8]{inputenc}
\usepackage{fourier}
\usepackage[scaled=0.875]{helvet}
\renewcommand{\ttdefault}{lmtt}
\usepackage{amsmath,amssymb,makeidx}
\usepackage[normalem]{ulem}
\usepackage{fancybox}
\usepackage{cancel}
\usepackage{stmaryrd}
\usepackage{ulem}
\usepackage{tabularx}
\usepackage{geometry}
\usepackage{enumerate}
\geometry{hmargin=1.5cm,vmargin=1.5cm}
\usepackage{dcolumn}
\usepackage{textcomp}
\usepackage{lscape}
\usepackage{eurosym}
%\newcommand{\euro}{\eurologo{}}
\usepackage[dvips]{color}
\usepackage[all]{xy}

\usepackage{tikz,tkz-tab}

\usepackage{systeme}
\usepackage{ upgreek }


\usepackage{pstricks,pst-plot,pst-text,pst-tree,pstricks-add}
\usepackage{colortbl}
\usepackage{diagbox}
\usepackage{fontawesome5}
\usepackage{pifont}
\usepackage{wasysym}


\usepackage{theorem}
\theorembodyfont{\upshape}
\newtheorem{exo}{Exercice}
%\newtheorem{exo}{Exercice}%[section]
\usepackage{hyperref}
\hypersetup{
    colorlinks=true,       % false: liens encadrés; true: liens colorés
    linkcolor=blue,          % couleur des liens (ou bordures) internes
}

%\setlength{\voffset}{-1,5cm}
\usepackage{fancyhdr} 
\usepackage{graphicx}
\usepackage[frenchb]{babel}
\usepackage[np]{numprint}
\usepackage{multicol}
\usepackage{xlop}
\usepackage{soul}

\usepackage{etoolbox}
\usepackage{multirow}
\usepackage{diagbox}


\title{Mathématiques -- Première technologique}

\date{Corrigés des exercices}
\begin{document}
\setlength\parindent{0mm}
\renewcommand \footrulewidth{.2pt}

\maketitle

\tableofcontents


\newpage

\section{Proportionnalité}



\begin{exo}



\begin{enumerate}
\item On complète un tableau de proportionnalité~:

\begin{center}
 \begin{tabular}{|m{2cm}|m{1cm}|m{1cm}|}\hline
Élèves& 40 & ? \\ \hline 
Pourcentage&100 & 70\\ \hline

\end{tabular}
\end{center}

Il y a $40\times 70\div 100=28$ garçons dans la classe.

\item On complète un tableau de proportionnalité~:

\begin{center}
 \begin{tabular}{|m{2cm}|m{1cm}|m{1cm}|}\hline
Marins& \np{1760} & \np{1046} \\ \hline 
Pourcentage&100 & ?\\ \hline

\end{tabular}
\end{center}

$\np{1046}\times 100\div \np{1760}\approx 59,43,$ donc environ 59,43~\% des marins sont tombés malades.

\medskip

\textbf{N.B.} On fait le calcul et, seulement après, on écrit la réponse avec le symbole \%. Rappelons à cette occasion la signification de 59,43~\%~:
\[59,43~\%=\dfrac{59,43}{100}=0,5943.\]
Donc dire que  59,43~\% des marins sont tombés malades, c'est dire que la proportion de malades est $\dfrac{59,43}{100}.$
\item Le fait que la bouteille soit titrée à 12~\% vol. signifie qu'elle contient 12~\% d'alcool pur. On complète donc un tableau de proportionnalité~:

\begin{center}
 \begin{tabular}{|m{2.5cm}|m{1cm}|m{1cm}|}\hline
Volume (en mL)& 500 & ? \\ \hline 
Pourcentage&100 & 12\\ \hline

\end{tabular}
\end{center} 
La bouteille contient $500\times 12\div 100=60$~mL d'alcool pur.
\item Sur 100 personnes de l'entreprise, il y a 56 hommes.

25~\% d'entre eux fument, ce qui représente
\[25\times 56\div 100=14~\text{personnes}\] (on peut bien sûr faire un tableau de proportionnalité pour obtenir cette réponse).

Conclusion~: les hommes fumeurs représentent 14~\% du personnel de l'entreprise.
\end{enumerate}

\end{exo}


\begin{exo}

\begin{enumerate}
\item ~{}
\begin{center}
\begin{tabular}{|c|c|c|}\hline
Nombre de personnes& 4&6 \\ \hline 
Farine (en g)&250& ? \\ \hline
Lait (en mL)&500& ? \\ \hline
Œufs&4& 6 \\ \hline
\end{tabular}
\end{center}

Pour 6 personnes, il faut $ 250\times 6\div 4=375$~g de farine, $500\times 6\div 4=750$~mL de lait et, bien sûr, 6 œufs.
\item Les 6 yaourts pèsent $6\times 125=750$~g.

\begin{center}
\begin{tabular}{|c|c|c|}\hline
masse (en g)& 1000&750 \\ \hline 
prix (en \euro)&2& ? \\ \hline
\end{tabular}
\end{center}

Je payerai $750\times 2\div \np{1000}=1,5~\text{\euro}.$
\end{enumerate}
\end{exo}

\begin{exo}

L'énoncé donne les informations recensées dans le tableau ci-dessous et demande de compléter la case \textcircled{\small{1}}.

\begin{center}
\begin{tabular}{|c|c|c|c|}\hline
Florins& 7&?&\textcircled{\small{1}} \\ \hline 
Pistoles&6& \textcolor{red}{4}&\textcircled{\small{\textcolor{black}{2}}} \\ \hline
Deniers&?& \textcolor{red}{5}&\textcolor{red}{30} \\ \hline
\end{tabular}
\end{center}

On complète d'abord la case \textcircled{\small{2}}~: en échange de 30 deniers, on a $4\times 30\div 5=24$~pistoles~:

\begin{center}
\begin{tabular}{|c|c|c|c|}\hline
Florins& \textcolor{red}{7}&?&\textcircled{\small{\textcolor{black}{1}}} \\ \hline 
Pistoles&\textcolor{red}{6}& 4&\textcolor{red}{24} \\ \hline
Deniers&?& 5&30 \\ \hline
\end{tabular}
\end{center}

On peut alors compléter la case \textcircled{\small{1}}~: en échange de 30 deniers, on a $7\times 24\div 6=28$~florins.

\end{exo}

\begin{exo}


\begin{enumerate}

\item Généralement, dans ce type de question, il vaut mieux convertir en minutes\footnote{Les calculs ne sont pas toujours plus faciles en minutes qu'en heures, mais c'est généralement le cas.}.

\begin{center}
\begin{tabular}{|c|c|c|}\hline
temps (en min)& 60&? \\ \hline 
distance (en km)&20& 45 \\ \hline
\end{tabular}
\end{center}

On mettra $60\times 45\div 20=135$~min, soit 2~h~15~min (puisque $135=120+15$).

\item On peut se passer d'un tableau de proportionnalité~: $1~\text{h}=60~\text{min},$ donc $0,6~\text{h}=0,6\times 60~\text{min}=36~\text{min}.$

\item \begin{enumerate}
\item On complète deux tableaux de proportionnalité (on travaille en min et en km)~:

\begin{multicols}{2}

\begin{center}
\begin{tabular}{|c|c|c|}\hline
temps (en min)& 60&? \\ \hline 
distance (en km)&3& 0,5 \\ \hline
\end{tabular}


\begin{tabular}{|c|c|c|}\hline
temps (en min)& 60&? \\ \hline 
distance (en km)&15& 5 \\ \hline
\end{tabular}
\end{center}

\end{multicols}

Stéphane nage $60\times 0,5 \div 3=10$~min, puis il court $60\times 5\div 15=20$~min.


\item Stéphane a parcouru un total de $5+0,5=5,5$~km, en $10+20=30$~min.

\begin{center}
\begin{tabular}{|c|c|c|}\hline
temps (en min)& 30&60 \\ \hline 
distance (en km)&5,5& ? \\ \hline
\end{tabular}
\end{center}

La vitesse moyenne de Stéphane sur l’ensemble de son parcours est donc $60\times 5,5\div 30=11$~km/h.
\end{enumerate}
\end{enumerate}

\end{exo}

\begin{exo}

Avant de commencer, il est utile de se rappeler que 10~cm=1~dm~; et que 1~$\ell=1~\text{dm}^3.$ Autrement dit, un litre est le volume d'un cube qui mesure 1~dm sur 1~dm sur 1~dm, ou encore 10~cm sur 10~cm sur 10~cm (la figure ci-dessous n'est bien sûr pas à l'échelle).


\begin{center}
\psset{xunit=1.0cm,yunit=1.0cm,algebraic=true,dimen=middle,dotstyle=o,dotsize=5pt 0,linewidth=2.pt,arrowsize=3pt 2,arrowinset=0.25}
\begin{pspicture*}(-1.42,-1.12)(4.42,3.5)
\psline[linewidth=2.pt](0.,0.)(2.,0.)
\psline[linewidth=2.pt](2.,0.)(2.,2.)
\psline[linewidth=2.pt](2.,2.)(0.,2.)
\psline[linewidth=2.pt](0.,2.)(0.,0.)
\psline[linewidth=2.pt](0.,2.)(1.,3.)
\psline[linewidth=2.pt](1.,3.)(3.,3.)
\psline[linewidth=2.pt](2.,2.)(3.,3.)
\psline[linewidth=2.pt](2.,0.)(3.,1.)
\psline[linewidth=2.pt](3.,3.)(3.,1.)
\rput[tl](0.58,-0.16){1~dm}
\rput[tl](-1.,1.14){1~dm}
\rput[tl](2.7,0.52){1~dm}
\rput[tl](0.92,1.24){$1~\ell$}
\end{pspicture*}
\end{center}


\medskip


On remplit d'eau un aquarium rectangulaire dont la largeur est 80~cm, la profondeur 30~cm et la hauteur 40~cm. On dispose d'un robinet dont le débit est de 6 litres par minute.

\begin{enumerate}
\item ~{}

\begin{center}
\psset{xunit=0.75cm,yunit=0.75cm,algebraic=true,dimen=middle,dotstyle=o,dotsize=5pt 0,linewidth=2.pt,arrowsize=3pt 2,arrowinset=0.25}
\begin{pspicture*}(0.476052349791793,0.11148839976204462)(12.435257584770984,6.724523497917915)
\psline[linewidth=2.pt](2.,1.)(10.,1.)
\psline[linewidth=2.pt](10.,1.)(10.,5.)
\psline[linewidth=2.pt](2.,1.)(2.,5.)
\psline[linewidth=2.pt](2.,5.)(10.,5.)
\psline[linewidth=2.pt](2.,5.)(4.,6.)
\psline[linewidth=2.pt](4.,6.)(12.,6.)
\psline[linewidth=2.pt](10.,5.)(12.,6.)
\psline[linewidth=2.pt](10.,1.)(12.,2.)
\psline[linewidth=2.pt](12.,2.)(12.,6.)
\psline[linewidth=2.pt,linestyle=dashed,dash=3pt 3pt](4.,6.)(4.,2.)
\psline[linewidth=2.pt,linestyle=dashed,dash=3pt 3pt](2.,1.)(4.,2.)
\psline[linewidth=2.pt,linestyle=dashed,dash=3pt 3pt](4.,2.)(12.,2.)
\rput[tl](5.720873289708514,0.8209327781082683){8 dm}
\rput[tl](0.932123735871508,3.2026389054134476){4 dm}
\rput[tl](11.09238072575849,1.4797025580011902){3 dm}
\end{pspicture*}
\end{center}
\item Les dimensions de l'aquarium sont~:
\[\text{largeur}=8~\text{dm},\qquad \text{profondeur}=3~\text{dm},\qquad \text{hauteur}=4~\text{dm},\] donc son volume est
\[8\times 3\times 4=96~\ell.\]


\item On peut se passer d'un tableau de proportionnalité~: le débit du robinet est de 6~$\ell$/min, donc il faut $96\div 6=16$~min pour remplir les 96~$\ell$ de l'aquarium.
\end{enumerate}
\end{exo}

\section{Droites et suites de nombres}


\begin{exo}

Le tableau suivant donne l'évolution du tirage journalier (en millions d'exemplaires) de la presse quotidienne d'information générale et politique en France.

\begin{center}
\begin{tabularx}{\linewidth}{|m{2cm}|*{5}{>{\centering \arraybackslash}X|}}\hline
Année 							&2010 	&2011 	&2012 	&2013 	&2014\\ \hline
Numéro année~: $n$					&0 	&1 	&2 	&3 	&4\\ \hline
Tirage~: $u_n$ &1,80 	&1,73 	&1,60 	&1,47 	&1,36\\ \hline
\multicolumn{6}{r}{\emph{Source: INSEE}}
\end{tabularx}
\end{center}

On note $u_n$ le tirage journalier en millions d'exemplaires pour l'année numéro $n.$ On a donc~:


\medskip

\begin{itemize}
\item[\textbullet] $u_0=\text{tirage journalier l'année 0}=1,80~;$ 
\item[\textbullet] $u_1=\text{tirage journalier l'année 1}=1,73~;$ 
\item[\textbullet] $u_4=\text{tirage journalier l'année 4}=1,36.$ 
\end{itemize}

\end{exo}

\begin{exo}
$u$ est la suite des multiples de 4, en partant de $u_0=4\times 0=0.$

\begin{enumerate}
\item \begin{itemize}
\item[\textbullet] $u_1=4\times 1=4~;$ 
\item[\textbullet] $u_2=4\times 2=8~;$  
\item[\textbullet] $u_3=4\times 3=12.$ 
\end{itemize}
\item $u_{20}=4\times 20=80.$
\end{enumerate}

\end{exo}

\begin{exo}

$u$ est une suite telle que~:
\begin{itemize}
\item[\textbullet] $u_0=2,$
\item[\textbullet] tout terme de la suite se déduit du précédent en ajoutant $3.$
\end{itemize}

\begin{enumerate}
\item \begin{itemize}
\item[\textbullet] $u_1=3+2=5~;$ 
\item[\textbullet] $u_2=5+3=8~;$  
\item[\textbullet] $u_3=8+3=11~;$
\item[\textbullet] $u_4=11+3=14.$ 
\end{itemize}
\item Pour obtenir le tableau avec un tableur, on entre la formule \[\text{=B1+1}\] dans la cellule C1, et la formule \[\text{=B2+3}\] dans la cellule C2. Ensuite on étire vers la droite.

\medskip

\begin{center}
\begin{tabularx}{\linewidth}{|c|*{7}{>{\centering \arraybackslash}X|}}\hline
	&A   						&B   		&C   	&D   	&E   	&F\\ \hline   
1   &$n$  					& 0   	&=B1+1   & $\cdots$  & $\cdots$  &$\cdots$ \\ \hline   
2   &$u_n$   				&2   		&=B2+3   	&$\cdots$   	&$\cdots$   	&$\cdots$\\ \hline    
\end{tabularx}
\end{center}

\end{enumerate}


\end{exo}




\begin{exo}

Notre objet tombe de~:

\begin{itemize}
\item[\textbullet] 5~m pendant la 1\up{re} seconde~;
\item[\textbullet] 15~m pendant la 2\up{e} seconde~;
\item[\textbullet] 25~m pendant la 3\up{e} seconde~;
\item[\textbullet] 35~m pendant la 4\up{e} seconde~;
\item[\textbullet] 45~m pendant la 5\up{e} seconde.
\end{itemize}

\medskip

Conclusion~: pendant les 5 premières secondes, l'objet est tombé de
\[5+15+25+35+45=125~\text{m}.\]

\medskip

\textbf{Remarque~:} Les informations de l'énoncé sont imprécises~: si l'on néglige la résistance de l'air (frottements), un objet soumis à son propre poids tombe de 4,9~m pendant la 1\up{re} seconde, $4,9\times 3=14,7$~m pendant la 2\up{e}, $4,9\times 5=24,5$~m pendant la 3\up{e}, etc. Dans l'exercice, nous avons remplacé 4,9 par 5 pour simplifier les calculs.

\medskip

Notons par ailleurs que ces résultats doivent être fortement corrigés si l'on veut tenir compte de la résistance de l'air. Par exemple, un adulte en chute libre qui parvient à se mettre \og à plat \fg~{} devrait arrêter d'accélérer après une dizaine de secondes de chute environ, sans dépasser 60~m/s~; tandis qu'un chat ne dépassera pas les 20~m/s et pourra survivre à une chute d'une hauteur importante. La vidéo \href{https://www.youtube.com/watch?v=RFbmabdbBC0}{KEZAKO~: chute libre} explique ce problème en détail.
\end{exo}

\newpage

\begin{exo}

On trace les droites  $D_1:y=x-4,$ $D_2:y=2x,$ $D_3:y=-2x+3$ et $D_4:y=-2$ à partir de quatre tableaux de valeurs~:

\setlength{\columnseprule}{1pt}

\begin{multicols}{4}
\underline{Tracé de $D_1.$}
\begin{center}
\begin{tabular}{|c|c|c|}\hline
$x$&$0$&$2$\\ \hline
$y$&$-4$&$-2$\\ \hline
\end{tabular}
\end{center}

\begin{align*}
&0-4=-4\\
&2-4=-2\end{align*}

\columnbreak

\underline{Tracé de $D_2.$}
\begin{center}
\begin{tabular}{|c|c|c|}\hline
$x$&$0$&$2$\\ \hline
$y$&$0$&$4$\\ \hline
\end{tabular}
\end{center}

\begin{align*}
&2\times 0=2\\
&2\times 2=4\end{align*}

\columnbreak

\underline{Tracé de $D_3.$}
\begin{center}
\begin{tabular}{|c|c|c|}\hline
$x$&$0$&$2$\\ \hline
$y$&$3$&$-1$\\ \hline
\end{tabular}
\end{center}

\begin{align*}
&-2\times 0+3=3\\
&-2\times 2+3=-1\end{align*}

\columnbreak

\underline{Tracé de $D_4.$}
\begin{center}
\begin{tabular}{|c|c|c|}\hline
$x$&$0$&$2$\\ \hline
$y$&$-2$&$-2$\\ \hline
\end{tabular}
\end{center}

\end{multicols}

On place à chaque fois les deux points en gris, puis on trace les droites en couleur~:


\begin{center}
\newrgbcolor{ffxfqq}{1. 0.4980392156862745 0.}
\newrgbcolor{xfqqff}{0.4980392156862745 0. 1.}
\newrgbcolor{uququq}{0.25098039215686274 0.25098039215686274 0.25098039215686274}
\psset{xunit=0.8cm,yunit=0.8cm,algebraic=true,dimen=middle,dotstyle=o,dotsize=5pt 0,linewidth=2.pt,arrowsize=3pt 2,arrowinset=0.25}
\begin{pspicture*}(-3.135027624309382,-4.58)(7.206740331491703,4.3)
\multips(0,-4)(0,1.0){9}{\psline[linestyle=dashed,linecap=1,dash=1.5pt 1.5pt,linewidth=0.4pt,linecolor=lightgray]{c-c}(-3.135027624309382,0)(7.206740331491703,0)}
\multips(-3,0)(1.0,0){11}{\psline[linestyle=dashed,linecap=1,dash=1.5pt 1.5pt,linewidth=0.4pt,linecolor=lightgray]{c-c}(0,-4.58)(0,4.3)}
\psaxes[labelFontSize=\scriptstyle,xAxis=true,yAxis=true,Dx=1.,Dy=1.,ticksize=-2pt 0,subticks=2]{->}(0,0)(-3.135027624309382,-4.58)(7.206740331491703,4.3)
\psplot[linewidth=2.pt,linecolor=ffxfqq]{-3.135027624309382}{7.206740331491703}{(-4.--1.*x)/1.}
\rput[tl](4.454033149171266,1.66){\ffxfqq{$D_1$}}
\psplot[linewidth=2.pt,linecolor=xfqqff]{-3.135027624309382}{7.206740331491703}{(-0.--2.*x)/1.}
\rput[tl](1.3763535911602227,2.66){\xfqqff{$D_2$}}
\psplot[linewidth=2.pt,linecolor=green]{-3.135027624309382}{7.206740331491703}{(--3.-2.*x)/1.}
\rput[tl](3.4791160220994453,-3.5){\green{$D_3$}}
\psplot[linewidth=2.pt,linecolor=blue]{-3.135027624309382}{7.206740331491703}{(-2.-0.*x)/1.}
\rput[tl](4.,-1.5){\blue{$D_4$}}
\psdots[dotstyle=*,linecolor=gray](0.,0.)
\psdots[dotstyle=*,linecolor=gray](2.,4.)
\psdots[dotstyle=*,linecolor=gray](0.,3.)
\psdots[dotstyle=*,linecolor=gray](2.,-1.)
\psdots[dotstyle=*,linecolor=gray](0.,-4.)
\psdots[dotstyle=*,linecolor=gray](2.,-2.)
\psdots[dotstyle=*,linecolor=gray](0.,-2.)
\end{pspicture*}
\end{center}


\textbf{Remarque~:} La droite $D_4$ est horizontale. C'était prévisible, puisque la valeur de $y~$ ($-2$) est indépendante de $x.$


\end{exo}

\vspace*{-0.5cm}


\begin{exo}

On lit graphiquement les ordonnées à l'origine et les coefficients directeurs des droites~:


\setlength{\columnseprule}{1pt}

\begin{multicols}{2}

\begin{center}
\newrgbcolor{xfqqff}{0.4980392156862745 0. 1.}
\newrgbcolor{ffxfqq}{1. 0.4980392156862745 0.}
\psset{xunit=1.0cm,yunit=1.0cm,algebraic=true,dimen=middle,dotstyle=o,dotsize=5pt 0,linewidth=2.pt,arrowsize=3pt 2,arrowinset=0.25}
\begin{pspicture*}(-2.7,-2.56)(6.66,5.96)
\multips(0,-2)(0,1.0){9}{\psline[linestyle=dashed,linecap=1,dash=1.5pt 1.5pt,linewidth=0.4pt,linecolor=lightgray]{c-c}(-2.7,0)(6.66,0)}
\multips(-2,0)(1.0,0){10}{\psline[linestyle=dashed,linecap=1,dash=1.5pt 1.5pt,linewidth=0.4pt,linecolor=lightgray]{c-c}(0,-2.56)(0,5.96)}
\psaxes[labelFontSize=\scriptstyle,xAxis=true,yAxis=true,Dx=1.,Dy=1.,ticksize=-2pt 0,subticks=2]{->}(0,0)(-2.7,-2.56)(6.66,5.96)
\psplot[linewidth=2.pt,linecolor=green]{-1}{5}{(-1.--2.*x)/1.}
\psline[linewidth=2.pt]{->}(0.,-1.)(1.,-1.)
\psline[linewidth=2.pt,linecolor=xfqqff]{->}(1.,-1.)(1.,1.)
\psline[linewidth=2.pt]{->}(1.,1.)(2.,1.)
\psline[linewidth=2.pt,linecolor=xfqqff]{->}(2.,1.)(2.,3.)
\psline[linewidth=2.pt]{->}(2.,3.)(3.,3.)
\psline[linewidth=2.pt,linecolor=xfqqff]{->}(3.,3.)(3.,5.)
\rput[tl](2.5,5.58){\green{$D_1$}}
\rput[tl](1.06,-0.34){\xfqqff{$+2$}}
\rput[tl](2.06,2.04){\xfqqff{$+2$}}
\rput[tl](3.06,4.06){\xfqqff{$+2$}}
\rput[tl](-1.88,-0.85){$\ffxfqq{\fbox{$b=-1$}}$}
\psline[linewidth=2.pt]{->}(4.72,2.74)(3.84,3.82)
\rput[tl](4.22,2.7){\xfqqff{\fbox{$a=2$}}}
\psdots[dotstyle=*,linecolor=ffxfqq](0.,-1.)
\end{pspicture*}
\end{center}


\definecolor{VIOLET}{rgb}{0.4980392156862745,0.,1.}
\definecolor{ORANGE}{rgb}{1.,0.4980392156862745,0.}

\begin{align*}
&\textcolor{green}{D_1:y=\textcolor{VIOLET}{2}x\textcolor{ORANGE}{-1}}\\
&D_1:y=2x-1
\end{align*}

\columnbreak


\begin{center}
\newrgbcolor{xfqqff}{0.4980392156862745 0. 1.}
\newrgbcolor{ffxfqq}{1. 0.4980392156862745 0.}
\psset{xunit=1.0cm,yunit=1.0cm,algebraic=true,dimen=middle,dotstyle=o,dotsize=5pt 0,linewidth=2.pt,arrowsize=3pt 2,arrowinset=0.25}
\begin{pspicture*}(-2.7,-2.56)(6.66,5.96)
\multips(0,-2)(0,1.0){9}{\psline[linestyle=dashed,linecap=1,dash=1.5pt 1.5pt,linewidth=0.4pt,linecolor=lightgray]{c-c}(-2.7,0)(6.66,0)}
\multips(-2,0)(1.0,0){10}{\psline[linestyle=dashed,linecap=1,dash=1.5pt 1.5pt,linewidth=0.4pt,linecolor=lightgray]{c-c}(0,-2.56)(0,5.96)}
\psaxes[labelFontSize=\scriptstyle,xAxis=true,yAxis=true,Dx=1.,Dy=1.,ticksize=-2pt 0,subticks=2]{->}(0,0)(-2.7,-2.56)(6.66,5.96)
\psplot[linewidth=2.pt,linecolor=green]{-1}{5}{(--4.-1.*x)/1.}
\rput[tl](4.,0.56){\green{$D_2$}}
\rput[tl](1.06,3.56){\xfqqff{$-1$}}
\rput[tl](-1.78,4.48){$\ffxfqq{\fbox{$b=4$}}$}
\rput[tl](3.62,3.66){$\xfqqff{\fbox{$a=-1$}}$}
\psline[linewidth=2.pt]{->}(0.,4.)(1.,4.)
\psline[linewidth=2.pt,linecolor=xfqqff]{->}(1.,4.)(1.,3.)
\psline[linewidth=2.pt]{->}(1.,3.)(2.,3.)
\psline[linewidth=2.pt,linecolor=xfqqff]{->}(2.,3.)(2.,2.)
\psline[linewidth=2.pt]{->}(2.,2.)(3.,2.)
\psline[linewidth=2.pt,linecolor=xfqqff]{->}(3.,2.)(3,1.)
\rput[tl](2.06,2.56){\xfqqff{$-1$}}
\rput[tl](3.06,1.56){\xfqqff{$-1$}}
\psline[linewidth=2.pt]{->}(3.78,3.06)(2.76,2.48)
\psdots[dotstyle=*,linecolor=ffxfqq](0.,4.)
\end{pspicture*}
\end{center}

\definecolor{VIOLET}{rgb}{0.4980392156862745,0.,1.}
\definecolor{ORANGE}{rgb}{1.,0.4980392156862745,0.}

\begin{align*}
&\textcolor{green}{D_2:y=\textcolor{VIOLET}{-1}x\textcolor{ORANGE}{+4}}\\
&D_2:y=-x+4
\end{align*}

\end{multicols}

\setlength{\columnseprule}{1pt}

\begin{multicols}{2}

\begin{center}
\newrgbcolor{ffxfqq}{1. 0.4980392156862745 0.}
\newrgbcolor{xfqqff}{0.4980392156862745 0. 1.}
\psset{xunit=1.0cm,yunit=1.0cm,algebraic=true,dimen=middle,dotstyle=o,dotsize=5pt 0,linewidth=2.pt,arrowsize=3pt 2,arrowinset=0.25}
\begin{pspicture*}(-1.7098267872637365,-4.222754171484549)(7.291644850785794,1.951332144357107)
\multips(0,-4)(0,1.0){7}{\psline[linestyle=dashed,linecap=1,dash=1.5pt 1.5pt,linewidth=0.4pt,linecolor=lightgray]{c-c}(-1.7098267872637365,0)(7.291644850785794,0)}
\multips(-1,0)(1.0,0){10}{\psline[linestyle=dashed,linecap=1,dash=1.5pt 1.5pt,linewidth=0.4pt,linecolor=lightgray]{c-c}(0,-4.222754171484549)(0,1.951332144357107)}
\psaxes[labelFontSize=\scriptstyle,xAxis=true,yAxis=true,Dx=1.,Dy=1.,ticksize=-2pt 0,subticks=2]{->}(0,0)(-1.7098267872637365,-4.222754171484549)(7.291644850785794,1.951332144357107)
\psplot[linewidth=2.pt,linecolor=green]{-1}{5}{(-2.--0.5*x)/1.}
\rput[tl](4.2334525634997355,0.9127008015052398){\green{$D_3$}}
\rput[tl](-1.5,-1.43383667678972){$\ffxfqq{\fbox{$b=-2$}}$}
\rput[tl](3.86800820212593,-2.3){$\xfqqff{\fbox{$a=0,5$}}$}
\psline[linewidth=2.pt]{->}(0.,-2.)(1.,-2.)
\psline[linewidth=2.pt,linecolor=xfqqff]{->}(1.,-2.)(1.,-1.5)
\psline[linewidth=2.pt]{->}(1.,-1.5)(2.,-1.5)
\psline[linewidth=2.pt,linecolor=xfqqff]{->}(2.,-1.5)(2.,-1.)
\psline[linewidth=2.pt]{->}(2.,-1.)(3.,-1.)
\psline[linewidth=2.pt,linecolor=xfqqff]{->}(3.,-1.)(3.,-0.5)
\rput[tl](1.12,-1.7){\xfqqff{$+0,5$}}
\rput[tl](2.12,-1.2){\xfqqff{$+0,5$}}
\rput[tl](3.12,-0.7){\xfqqff{$+0,5$}}
\psline[linewidth=2.pt]{->}(4.214218649743219,-2.1)(3.6,-1.06)
\psdots[dotstyle=*,linecolor=ffxfqq](0.,-2.)
\end{pspicture*}

\end{center}

\definecolor{VIOLET}{rgb}{0.4980392156862745,0.,1.}
\definecolor{ORANGE}{rgb}{1.,0.4980392156862745,0.}

\begin{align*}
&\textcolor{green}{D_3:y=\textcolor{VIOLET}{0,5}x\textcolor{ORANGE}{-2}}\\
&D_3:y=0,5x-2
\end{align*}

\columnbreak


\begin{center}
\newrgbcolor{ffxfqq}{1. 0.4980392156862745 0.}
\newrgbcolor{xfqqff}{0.4980392156862745 0. 1.}
\psset{xunit=1.0cm,yunit=1.0cm,algebraic=true,dimen=middle,dotstyle=o,dotsize=5pt 0,linewidth=2.pt,arrowsize=3pt 2,arrowinset=0.25}
\begin{pspicture*}(-2.342916244511982,-4.076140688436127)(6.870029275496755,4.3690593715718675)
\multips(0,-4)(0,1.0){9}{\psline[linestyle=dashed,linecap=1,dash=1.5pt 1.5pt,linewidth=0.4pt,linecolor=lightgray]{c-c}(-2.342916244511982,0)(6.870029275496755,0)}
\multips(-2,0)(1.0,0){10}{\psline[linestyle=dashed,linecap=1,dash=1.5pt 1.5pt,linewidth=0.4pt,linecolor=lightgray]{c-c}(0,-4.076140688436127)(0,4.3690593715718675)}
\psaxes[labelFontSize=\scriptstyle,xAxis=true,yAxis=true,Dx=1.,Dy=1.,ticksize=-2pt 0,subticks=2]{->}(0,0)(-2.342916244511982,-4.076140688436127)(6.870029275496755,4.3690593715718675)
\psplot[linewidth=2.pt,linecolor=green]{-1}{5}{(--3.-1.5*x)/1.}
\rput[tl](3.15,-2.5012782063833545){\green{$D_4$}}
\rput[tl](-1.9295148429731284,3.6209996925968){$\ffxfqq{\fbox{$b=3$}}$}
\rput[tl](3.936847902673461,-1.9500763376648838){$\xfqqff{\fbox{$a=-1,5$}}$}
\rput[tl](1.0430380919014859,2.35){\xfqqff{$-1,5$}}
\rput[tl](2.06669870523579,0.85){\xfqqff{$-1,5$}}
\rput[tl](3.0706735375444345,-0.65){\xfqqff{$-1,5$}}
\psline[linewidth=2.pt]{->}(4.214218649743219,-1.8954506069461055)(3.791072547099865,-1.0683923154159147)
\psline[linewidth=2.pt]{->}(0.,3.)(1.,3.)
\psline[linewidth=2.pt,linecolor=xfqqff]{->}(1.,3.)(1.,1.5)
\psline[linewidth=2.pt]{->}(1.,1.5)(2.,1.5)
\psline[linewidth=2.pt,linecolor=xfqqff]{->}(2.,1.5)(2.,0.)
\psline[linewidth=2.pt]{->}(2.,0.)(3.,0.)
\psline[linewidth=2.pt,linecolor=xfqqff]{->}(3.,0.)(3.,-1.5)
\psdots[dotstyle=*,linecolor=ffxfqq](0.,3.)
\end{pspicture*}
\end{center}

\definecolor{VIOLET}{rgb}{0.4980392156862745,0.,1.}
\definecolor{ORANGE}{rgb}{1.,0.4980392156862745,0.}

\begin{align*}
&\textcolor{green}{D_4:y=\textcolor{VIOLET}{-1,5}x\textcolor{ORANGE}{+3}}\\
&D_4:y=-1,5x+3
\end{align*}

\end{multicols}

\end{exo}


\begin{exo}

Le graphique suivant donne le prix payé dans une pompe à essence en fonction de la quantité de gazole achetée.

\begin{center}
\newrgbcolor{xfqqff}{0.4980392156862745 0. 1.}

\psset{xunit=1cm,yunit=1cm,algebraic=true,dimen=middle,dotstyle=o,dotsize=5pt 0,linewidth=1.6pt,arrowsize=3pt 2,arrowinset=0.25}
\begin{pspicture*}(-0.7596559723553132,-0.6294697319234536)(6.484452736237389,6.424504889829535)
\multips(0,0)(0,1.0){8}{\psline[linestyle=dashed,linecap=1,dash=1.5pt 1.5pt,linewidth=0.4pt,linecolor=lightgray]{c-c}(-0.7596559723553132,0)(6.484452736237389,0)}
\multips(0,0)(1.0,0){8}{\psline[linestyle=dashed,linecap=1,dash=1.5pt 1.5pt,linewidth=0.4pt,linecolor=lightgray]{c-c}(0,-0.6294697319234536)(0,6.424504889829535)}
\psaxes[labelFontSize=\scriptstyle,xAxis=true,yAxis=true,Dx=1.,Dy=1.,ticksize=-2pt 0,subticks=2]{->}(0,0)(-0.7596559723553132,-0.6294697319234536)(6.484452736237389,6.424504889829535)
\rput[tl](5.15,0.8){\parbox{2.5210726947176276 cm}{gazole \\(en $\ell$)}}
\rput[lt](0.12,5.302713777475287){\parbox{2.5210726947176276 cm}{prix \\(en \euro)}}
\psplot[linewidth=2.pt]{0.}{6.484452736237389}{1.5*x}
\psline[linewidth=2.pt,linestyle=dashed,dash=2pt 2pt,linecolor=red](4.,0.)(4.,6.)
\psline[linewidth=2.pt,linestyle=dashed,dash=2pt 2pt,linecolor=red](4.,6.)(0.,6.)
\psline[linewidth=2.pt,linecolor=xfqqff]{->}(0.,0.)(1.,0.)
\psline[linewidth=2.pt,linecolor=xfqqff]{->}(1.,0.)(1.,1.5)
\rput[tl](1.06,0.86){\xfqqff{$+1,5$}}
\end{pspicture*}
\end{center}


Il y a deux méthodes possibles pour répondre à la question~:

\begin{itemize}
\item[\textbullet] \textbf{Pointillés rouges~:} 4 litres de gazole coûtent 6~\euro, donc le litre coûte $6\div 4=1,5$~\euro.
\item[\textbullet] \textbf{Flèches violettes~:} chaque litre coûte 1,5~\euro.
\end{itemize}


\end{exo}

\begin{exo}

Dans chaque question, $u$ est une suite arithmétique de raison $r.$

\begin{enumerate}
\item $u_0=2$ et $r=4.$

\begin{align*}
u_1&=2+4=6\\
u_2&=6+4=10\\
u_3&=10+4=14.\\
\end{align*}
\item $u_0=5$ et $r=-2.$

\begin{align*}
u_1&=5-2=3\\
u_2&=3-2=1\\
u_3&=1-2=-1.\\
\end{align*}

\item $u_0=10$ et $r=1,5.$

Pour obtenir $u_6,$ on part de $u_0=10$ et on rajoute 6 fois $1,5.$ Donc
\[u_6=10+6\times 1,5=10+9=19.\]

\item $u_0=4$ et $u_2=10.$

\begin{center}
    $\xymatrix@R=0.5pc@C=3pc{
    *+[F]+{4} \ar@/^0.5cm/[r]|{\red{+r}} & 
    *+[F]+{???} \ar@/^0.5cm/[r]|{\red{+r}} & *+[F]+{10} \\
    \txt{\blue{$u_0$}}&
    \txt{\blue{$u_1$}}&\txt{\blue{$u_2$}}
    }$
    \end{center}
    
    D'après le schéma ci-dessus~:
    \[r=(10-4)\div 2=6\div 2=3.\]  On obtient donc le schéma complété~:
    
    \begin{center}
    $\xymatrix@R=0.5pc@C=3pc{
    *+[F]+{4} \ar@/^0.5cm/[r]|{\red{+3}} & 
    *+[F]+{7} \ar@/^0.5cm/[r]|{\red{+3}} & 
    *+[F]+{10} \ar@/^0.5cm/[r]|{\red{+3}} & 
    *+[F]+{13} \ar@/^0.5cm/[r]|{\red{+3}}& 
    *+[F]+{16} \\
    \txt{\blue{$u_0$}}&
    \txt{\blue{$u_1$}}&\txt{\blue{$u_2$}}&
    \txt{\blue{$u_3$}}&
    \txt{\blue{$u_4$}}
    }$
    \end{center}
    
    (On peut aussi obtenir $u_4$ avec le calcul~: $u_4=4+4\times 3=4+12=16.$)

\item ~{}\begin{center}
    $\xymatrix@R=0.5pc@C=3pc{
    *+[F]+{5} \ar@/^0.5cm/[r]|{\red{+r}} & 
    *+[F]+{???} \ar@/^0.5cm/[r]|{\red{+r}} & *+[F]+{???} \ar@/^0.5cm/[r]|{\red{+r}} & 
    *+[F]+{12,5} \\
    \txt{\blue{$u_0$}}&
    \txt{\blue{$u_1$}}&\txt{\blue{$u_2$}}&
    \txt{\blue{$u_3$}}
    }$
    \end{center}
    
    D'après le schéma ci-dessus~:
    \[r=(12,5-5)\div 3=7,5\div 3=2,5.\]
\end{enumerate}

\end{exo}



\begin{exo}

\end{exo}

\begin{exo}

\end{exo}

\begin{exo}

On note $S$ la somme à calculer, que l'on écrit à l'endroit, puis à l'envers~:
\begin{alignat*}{8}
&S&&= \textcolor{red}{1}&&+\textcolor{blue}{2}&&+\textcolor{green}{3}&&+\cdots  &&+\textcolor{orange}{98} &&+ \textcolor{violet}{99}&&+ \textcolor{brown}{100}\\
&S&&= \textcolor{red}{100}&&+\textcolor{blue}{99}&&+\textcolor{green}{98}&&+\cdots &&+\textcolor{orange}{3} &&+ \textcolor{violet}{2}&&+ \textcolor{brown}{1}
\end{alignat*}


On ajoute membre à membre les deux lignes. On remarque que la somme de chaque couple d'une même couleur vaut toujours $101~:$


\[S+S=\underbrace{\textcolor{red}{101}+\textcolor{blue}{101}+\textcolor{green}{101}+\cdots+\textcolor{orange}{101}+\textcolor{violet}{101}+\textcolor{brown}{101}}_{100~\text{termes}}.\]
On a donc
\[2S=100\times 101\hspace{2cm} S=\frac{100\times 101}{2}=\np{5050}.\]
\end{exo}



\end{document}
