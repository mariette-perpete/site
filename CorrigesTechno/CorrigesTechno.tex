\documentclass[10pt]{article}
\usepackage[T1]{fontenc}
\usepackage[utf8]{inputenc}
\usepackage{fourier}
\usepackage[scaled=0.875]{helvet}
\renewcommand{\ttdefault}{lmtt}
\usepackage{amsmath,amssymb,makeidx}
\usepackage[normalem]{ulem}
\usepackage{fancybox}
\usepackage{cancel}
\usepackage{stmaryrd}
\usepackage{ulem}
\usepackage{tabularx}
\usepackage{geometry}
\usepackage{enumerate}
\geometry{hmargin=1.5cm,vmargin=1.5cm}
\usepackage{dcolumn}
\usepackage{textcomp}
\usepackage{lscape}
\usepackage{eurosym}
%\newcommand{\euro}{\eurologo{}}
\usepackage[dvips]{color}
\usepackage[all]{xy}

\usepackage{tikz,tkz-tab}

\usepackage{systeme}
\usepackage{ upgreek }


\usepackage{pstricks,pst-plot,pst-text,pst-tree,pstricks-add}
\usepackage{colortbl}
\usepackage{diagbox}
\usepackage{fontawesome5}
\usepackage{pifont}
\usepackage{wasysym}


\usepackage{theorem}
\theorembodyfont{\upshape}
\newtheorem{exo}{Exercice}
%\newtheorem{exo}{Exercice}%[section]
\usepackage{hyperref}
\hypersetup{
    colorlinks=true,       % false: liens encadrés; true: liens colorés
    linkcolor=blue,          % couleur des liens (ou bordures) internes
}

%\setlength{\voffset}{-1,5cm}
\usepackage{fancyhdr} 
\usepackage{graphicx}
\usepackage[frenchb]{babel}
\usepackage[np]{numprint}
\usepackage{multicol}
\usepackage{xlop}
\usepackage{soul}

\usepackage{etoolbox}
\usepackage{multirow}
\usepackage{diagbox}


\title{Mathématiques -- Première technologique}

\date{Corrigés des exercices}
\begin{document}
\setlength\parindent{0mm}
\renewcommand \footrulewidth{.2pt}

\maketitle

\tableofcontents


\newpage

\section{Proportionnalité}



\begin{exo}



\begin{enumerate}
\item On complète un tableau de proportionnalité~:

\begin{center}
 \begin{tabular}{|m{2cm}|m{1cm}|m{1cm}|}\hline
Élèves& 40 & ? \\ \hline 
Pourcentage&100 & 70\\ \hline

\end{tabular}
\end{center}

Il y a $40\times 70\div 100=28$ garçons dans la classe.

\item On complète un tableau de proportionnalité~:

\begin{center}
 \begin{tabular}{|m{2cm}|m{1cm}|m{1cm}|}\hline
Marins& \np{1760} & \np{1046} \\ \hline 
Pourcentage&100 & ?\\ \hline

\end{tabular}
\end{center}

$\np{1046}\times 100\div \np{1760}\approx 59,43,$ donc environ 59,43~\% des marins sont tombés malades.

\medskip

\textbf{N.B.} On fait le calcul et, seulement après, on écrit la réponse avec le symbole \%. Rappelons à cette occasion la signification de 59,43~\%~:
\[59,43~\%=\dfrac{59,43}{100}=0,5943.\]
Donc dire que  59,43~\% des marins sont tombés malades, c'est dire que la proportion de malades est $\dfrac{59,43}{100}.$
\item Le fait que la bouteille soit titrée à 12~\% vol. signifie qu'elle contient 12~\% d'alcool pur. On complète donc un tableau de proportionnalité~:

\begin{center}
 \begin{tabular}{|m{2.5cm}|m{1cm}|m{1cm}|}\hline
Volume (en mL)& 500 & ? \\ \hline 
Pourcentage&100 & 12\\ \hline

\end{tabular}
\end{center} 
La bouteille contient $500\times 12\div 100=60$~mL d'alcool pur.
\item Sur 100 personnes de l'entreprise, il y a 56 hommes.

25~\% d'entre eux fument, ce qui représente
\[25\times 56\div 100=14~\text{personnes}\] (on peut bien sûr faire un tableau de proportionnalité pour obtenir cette réponse).

Conclusion~: les hommes fumeurs représentent 14~\% du personnel de l'entreprise.
\end{enumerate}

\end{exo}


\end{document}
