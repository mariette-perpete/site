\documentclass[10pt]{article}
\usepackage[T1]{fontenc}
\usepackage[utf8]{inputenc}
\usepackage{fourier}
\usepackage[scaled=0.875]{helvet}
\renewcommand{\ttdefault}{lmtt}
\usepackage{amsmath,amssymb,makeidx}
\usepackage[normalem]{ulem}
\usepackage{fancybox}
\usepackage{cancel}
\usepackage{stmaryrd}
\usepackage{ulem}
\usepackage{tabularx}
\usepackage{geometry}
\usepackage{enumerate}
\geometry{hmargin=1.5cm,vmargin=1.5cm}
\usepackage{dcolumn}
\usepackage{textcomp}
\usepackage{lscape}
\usepackage{eurosym}
%\newcommand{\euro}{\eurologo{}}
\usepackage[dvips]{color}
\usepackage[all]{xy}

\usepackage{tikz,tkz-tab}

\usepackage{systeme}
\usepackage{ upgreek }


\usepackage{pstricks,pst-plot,pst-text,pst-tree,pstricks-add}
\usepackage{colortbl}
\usepackage{diagbox}
\usepackage{fontawesome5}
\usepackage{pifont}
\usepackage{wasysym}


\usepackage{theorem}
\theorembodyfont{\upshape}
\newtheorem{exo}{Exercice}
%\newtheorem{exo}{Exercice}%[section]
\usepackage{hyperref}
\hypersetup{
    colorlinks=true,       % false: liens encadrés; true: liens colorés
    linkcolor=blue,          % couleur des liens (ou bordures) internes
}

%\setlength{\voffset}{-1,5cm}
\usepackage{fancyhdr} 
\usepackage{graphicx}
\usepackage[frenchb]{babel}
\usepackage[np]{numprint}
\usepackage{multicol}
\usepackage{xlop}
\usepackage{soul}

\usepackage{etoolbox}
\usepackage{multirow}
\usepackage{diagbox}


\title{Mathématiques -- Première technologique}

\date{Corrigés des exercices}
\begin{document}
\setlength\parindent{0mm}
\renewcommand \footrulewidth{.2pt}

\maketitle

\tableofcontents


\newpage

\section{Proportionnalité}



\begin{exo}



\begin{enumerate}
\item On complète un tableau de proportionnalité~:

\begin{center}
 \begin{tabular}{|m{2cm}|m{1cm}|m{1cm}|}\hline
Élèves& 40 & ? \\ \hline 
Pourcentage&100 & 70\\ \hline

\end{tabular}
\end{center}

Il y a $40\times 70\div 100=28$ garçons dans la classe.

\item On complète un tableau de proportionnalité~:

\begin{center}
 \begin{tabular}{|m{2cm}|m{1cm}|m{1cm}|}\hline
Marins& \np{1760} & \np{1046} \\ \hline 
Pourcentage&100 & ?\\ \hline

\end{tabular}
\end{center}

$\np{1046}\times 100\div \np{1760}\approx 59,43,$ donc environ 59,43~\% des marins sont tombés malades.

\medskip

\textbf{N.B.} On fait le calcul et, seulement après, on écrit la réponse avec le symbole \%. Rappelons à cette occasion la signification de 59,43~\%~:
\[59,43~\%=\dfrac{59,43}{100}=0,5943.\]
Donc dire que  59,43~\% des marins sont tombés malades, c'est dire que la proportion de malades est $\dfrac{59,43}{100}.$
\item Le fait que la bouteille soit titrée à 12~\% vol. signifie qu'elle contient 12~\% d'alcool pur. On complète donc un tableau de proportionnalité~:

\begin{center}
 \begin{tabular}{|m{2.5cm}|m{1cm}|m{1cm}|}\hline
Volume (en mL)& 500 & ? \\ \hline 
Pourcentage&100 & 12\\ \hline

\end{tabular}
\end{center} 
La bouteille contient $500\times 12\div 100=60$~mL d'alcool pur.
\item Sur 100 personnes de l'entreprise, il y a 56 hommes.

25~\% d'entre eux fument, ce qui représente
\[25\times 56\div 100=14~\text{personnes}\] (on peut bien sûr faire un tableau de proportionnalité pour obtenir cette réponse).

Conclusion~: les hommes fumeurs représentent 14~\% du personnel de l'entreprise.
\end{enumerate}

\end{exo}


\begin{exo}

\begin{enumerate}
\item ~{}
\begin{center}
\begin{tabular}{|c|c|c|}\hline
Nombre de personnes& 4&6 \\ \hline 
Farine (en g)&250& ? \\ \hline
Lait (en mL)&500& ? \\ \hline
Œufs&4& 6 \\ \hline
\end{tabular}
\end{center}

Pour 6 personnes, il faut $ 250\times 6\div 4=375$~g de farine, $500\times 6\div 4=750$~mL de lait et, bien sûr, 6 œufs.
\item Les 6 yaourts pèsent $6\times 125=750$~g.

\begin{center}
\begin{tabular}{|c|c|c|}\hline
masse (en g)& 1000&750 \\ \hline 
prix (en \euro)&2& ? \\ \hline
\end{tabular}
\end{center}

Je payerai $750\times 2\div \np{1000}=1,5~\text{\euro}.$
\end{enumerate}
\end{exo}

\begin{exo}

L'énoncé donne les informations recensées dans le tableau ci-dessous et demande de compléter la case \textcircled{\small{1}}.

\begin{center}
\begin{tabular}{|c|c|c|c|}\hline
Florins& 7&?&\textcircled{\small{1}} \\ \hline 
Pistoles&6& \textcolor{red}{4}&\textcircled{\small{\textcolor{black}{2}}} \\ \hline
Deniers&?& \textcolor{red}{5}&\textcolor{red}{30} \\ \hline
\end{tabular}
\end{center}

On complète d'abord la case \textcircled{\small{2}}~: en échange de 30 deniers, on a $4\times 30\div 5=24$~pistoles~:

\begin{center}
\begin{tabular}{|c|c|c|c|}\hline
Florins& \textcolor{red}{7}&?&\textcircled{\small{\textcolor{black}{1}}} \\ \hline 
Pistoles&\textcolor{red}{6}& 4&\textcolor{red}{24} \\ \hline
Deniers&?& 5&30 \\ \hline
\end{tabular}
\end{center}

On peut alors compléter la case \textcircled{\small{1}}~: en échange de 30 deniers, on a $7\times 24\div 6=28$~florins.

\end{exo}

\begin{exo}


\begin{enumerate}

\item Généralement, dans ce type de question, il vaut mieux convertir en minutes\footnote{Les calculs ne sont pas toujours plus faciles en minutes qu'en heures, mais c'est généralement le cas.}.

\begin{center}
\begin{tabular}{|c|c|c|}\hline
temps (en min)& 60&? \\ \hline 
distance (en km)&20& 45 \\ \hline
\end{tabular}
\end{center}

On mettra $60\times 45\div 20=135$~min, soit 2~h~15~min (puisque $135=120+15$).

\item On peut se passer d'un tableau de proportionnalité~: $1~\text{h}=60~\text{min},$ donc $0,6~\text{h}=0,6\times 60~\text{min}=36~\text{min}.$

\item \begin{enumerate}
\item On complète deux tableaux de proportionnalité (on travaille en min et en km)~:

\begin{multicols}{2}

\begin{center}
\begin{tabular}{|c|c|c|}\hline
temps (en min)& 60&? \\ \hline 
distance (en km)&3& 0,5 \\ \hline
\end{tabular}


\begin{tabular}{|c|c|c|}\hline
temps (en min)& 60&? \\ \hline 
distance (en km)&15& 5 \\ \hline
\end{tabular}
\end{center}

\end{multicols}

Stéphane nage $60\times 0,5 \div 3=10$~min, puis il court $60\times 5\div 15=20$~min.


\item Stéphane a parcouru un total de $5+0,5=5,5$~km, en $10+20=30$~min.

\begin{center}
\begin{tabular}{|c|c|c|}\hline
temps (en min)& 30&60 \\ \hline 
distance (en km)&5,5& ? \\ \hline
\end{tabular}
\end{center}

La vitesse moyenne de Stéphane sur l’ensemble de son parcours est donc $60\times 5,5\div 30=11$~km/h.
\end{enumerate}
\end{enumerate}

\end{exo}

\begin{exo}

Avant de commencer, il est utile de se rappeler que 10~cm=1~dm~; et que 1~$\ell=1~\text{dm}^3.$ Autrement dit, un litre est le volume d'un cube qui mesure 1~dm sur 1~dm sur 1~dm, ou encore 10~cm sur 10~cm sur 10~cm (la figure ci-dessous n'est bien sûr pas à l'échelle).


\begin{center}
\psset{xunit=1.0cm,yunit=1.0cm,algebraic=true,dimen=middle,dotstyle=o,dotsize=5pt 0,linewidth=2.pt,arrowsize=3pt 2,arrowinset=0.25}
\begin{pspicture*}(-1.42,-1.12)(4.42,3.5)
\psline[linewidth=2.pt](0.,0.)(2.,0.)
\psline[linewidth=2.pt](2.,0.)(2.,2.)
\psline[linewidth=2.pt](2.,2.)(0.,2.)
\psline[linewidth=2.pt](0.,2.)(0.,0.)
\psline[linewidth=2.pt](0.,2.)(1.,3.)
\psline[linewidth=2.pt](1.,3.)(3.,3.)
\psline[linewidth=2.pt](2.,2.)(3.,3.)
\psline[linewidth=2.pt](2.,0.)(3.,1.)
\psline[linewidth=2.pt](3.,3.)(3.,1.)
\rput[tl](0.58,-0.16){1~dm}
\rput[tl](-1.,1.14){1~dm}
\rput[tl](2.7,0.52){1~dm}
\rput[tl](0.92,1.24){$1~\ell$}
\end{pspicture*}
\end{center}


\medskip


On remplit d'eau un aquarium rectangulaire dont la largeur est 80~cm, la profondeur 30~cm et la hauteur 40~cm. On dispose d'un robinet dont le débit est de 6 litres par minute.

\begin{enumerate}
\item ~{}

\begin{center}
\psset{xunit=0.75cm,yunit=0.75cm,algebraic=true,dimen=middle,dotstyle=o,dotsize=5pt 0,linewidth=2.pt,arrowsize=3pt 2,arrowinset=0.25}
\begin{pspicture*}(0.476052349791793,0.11148839976204462)(12.435257584770984,6.724523497917915)
\psline[linewidth=2.pt](2.,1.)(10.,1.)
\psline[linewidth=2.pt](10.,1.)(10.,5.)
\psline[linewidth=2.pt](2.,1.)(2.,5.)
\psline[linewidth=2.pt](2.,5.)(10.,5.)
\psline[linewidth=2.pt](2.,5.)(4.,6.)
\psline[linewidth=2.pt](4.,6.)(12.,6.)
\psline[linewidth=2.pt](10.,5.)(12.,6.)
\psline[linewidth=2.pt](10.,1.)(12.,2.)
\psline[linewidth=2.pt](12.,2.)(12.,6.)
\psline[linewidth=2.pt,linestyle=dashed,dash=3pt 3pt](4.,6.)(4.,2.)
\psline[linewidth=2.pt,linestyle=dashed,dash=3pt 3pt](2.,1.)(4.,2.)
\psline[linewidth=2.pt,linestyle=dashed,dash=3pt 3pt](4.,2.)(12.,2.)
\rput[tl](5.720873289708514,0.8209327781082683){8 dm}
\rput[tl](0.932123735871508,3.2026389054134476){4 dm}
\rput[tl](11.09238072575849,1.4797025580011902){3 dm}
\end{pspicture*}
\end{center}
\item Les dimensions de l'aquarium sont~:
\[\text{largeur}=8~\text{dm},\qquad \text{profondeur}=3~\text{dm},\qquad \text{hauteur}=4~\text{dm},\] donc son volume est
\[8\times 3\times 4=96~\ell.\]


\item On peut se passer d'un tableau de proportionnalité~: le débit du robinet est de 6~$\ell$/min, donc il faut $96\div 6=16$~min pour remplir les 96~$\ell$ de l'aquarium.
\end{enumerate}
\end{exo}

\section{Droites et suites de nombres}


\begin{exo}

Le tableau suivant donne l'évolution du tirage journalier (en millions d'exemplaires) de la presse quotidienne d'information générale et politique en France.

\begin{center}
\begin{tabularx}{\linewidth}{|m{2cm}|*{5}{>{\centering \arraybackslash}X|}}\hline
Année 							&2010 	&2011 	&2012 	&2013 	&2014\\ \hline
Numéro année~: $n$					&0 	&1 	&2 	&3 	&4\\ \hline
Tirage~: $u_n$ &1,80 	&1,73 	&1,60 	&1,47 	&1,36\\ \hline
\multicolumn{6}{r}{\emph{Source: INSEE}}
\end{tabularx}
\end{center}

On note $u_n$ le tirage journalier en millions d'exemplaires pour l'année numéro $n.$ On a donc~:


\medskip

\begin{itemize}
\item[\textbullet] $u_0=\text{tirage journalier l'année 0}=1,80~;$ 
\item[\textbullet] $u_1=\text{tirage journalier l'année 1}=1,73~;$ 
\item[\textbullet] $u_4=\text{tirage journalier l'année 4}=1,36.$ 
\end{itemize}

\end{exo}

\begin{exo}
$u$ est la suite des multiples de 4, en partant de $u_0=4\times 0=0.$

\begin{enumerate}
\item \begin{itemize}
\item[\textbullet] $u_1=4\times 1=4~;$ 
\item[\textbullet] $u_2=4\times 2=8~;$  
\item[\textbullet] $u_3=4\times 3=12.$ 
\end{itemize}
\item $u_{20}=4\times 20=80.$
\end{enumerate}

\end{exo}

\begin{exo}

$u$ est une suite telle que~:
\begin{itemize}
\item[\textbullet] $u_0=2,$
\item[\textbullet] tout terme de la suite se déduit du précédent en ajoutant $3.$
\end{itemize}

\begin{enumerate}
\item \begin{itemize}
\item[\textbullet] $u_1=3+2=5~;$ 
\item[\textbullet] $u_2=5+3=8~;$  
\item[\textbullet] $u_3=8+3=11~;$
\item[\textbullet] $u_4=11+3=14.$ 
\end{itemize}
\item Pour obtenir le tableau avec un tableur, on entre la formule \[\text{=B1+1}\] dans la cellule C1, et la formule \[\text{=B2+3}\] dans la cellule C2. Ensuite on étire vers la droite.

\medskip

\begin{center}
\begin{tabularx}{\linewidth}{|c|*{7}{>{\centering \arraybackslash}X|}}\hline
	&A   						&B   		&C   	&D   	&E   	&F\\ \hline   
1   &$n$  					& 0   	&=B1+1   & $\cdots$  & $\cdots$  &$\cdots$ \\ \hline   
2   &$u_n$   				&2   		&=B2+3   	&$\cdots$   	&$\cdots$   	&$\cdots$\\ \hline    
\end{tabularx}
\end{center}

\end{enumerate}


\end{exo}




\begin{exo}

Notre objet tombe de~:

\begin{itemize}
\item[\textbullet] 5~m pendant la 1\up{re} seconde~;
\item[\textbullet] 15~m pendant la 2\up{e} seconde~;
\item[\textbullet] 25~m pendant la 3\up{e} seconde~;
\item[\textbullet] 35~m pendant la 4\up{e} seconde~;
\item[\textbullet] 45~m pendant la 5\up{e} seconde.
\end{itemize}

\medskip

Conclusion~: pendant les 5 premières secondes, l'objet est tombé de
\[5+15+25+35+45=125~\text{m}.\]

\medskip

\textbf{Remarque~:} Les informations de l'énoncé sont imprécises~: si l'on néglige la résistance de l'air (frottements), un objet soumis à son propre poids tombe de 4,9~m pendant la 1\up{re} seconde, $4,9\times 3=14,7$~m pendant la 2\up{e}, $4,9\times 5=24,5$~m pendant la 3\up{e}, etc. Dans l'exercice, nous avons remplacé 4,9 par 5 pour simplifier les calculs.

\medskip

Notons par ailleurs que ces résultats doivent être fortement corrigés si l'on veut tenir compte de la résistance de l'air. Par exemple, un adulte en chute libre qui parvient à se mettre \og à plat \fg~{} devrait arrêter d'accélérer après une dizaine de secondes de chute environ, sans dépasser 60~m/s~; tandis qu'un chat ne dépassera pas les 20~m/s et pourra survivre à une chute d'une hauteur importante. La vidéo \href{https://www.youtube.com/watch?v=RFbmabdbBC0}{KEZAKO~: chute libre} explique ce problème en détail.
\end{exo}

\newpage

\begin{exo}

On trace les droites  $D_1:y=x-4,$ $D_2:y=2x,$ $D_3:y=-2x+3$ et $D_4:y=-2$ à partir de quatre tableaux de valeurs~:

\setlength{\columnseprule}{1pt}

\begin{multicols}{4}
\underline{Tracé de $D_1.$}
\begin{center}
\begin{tabular}{|c|c|c|}\hline
$x$&$0$&$2$\\ \hline
$y$&$-4$&$-2$\\ \hline
\end{tabular}
\end{center}

\begin{align*}
&0-4=-4\\
&2-4=-2\end{align*}

\columnbreak

\underline{Tracé de $D_2.$}
\begin{center}
\begin{tabular}{|c|c|c|}\hline
$x$&$0$&$2$\\ \hline
$y$&$0$&$4$\\ \hline
\end{tabular}
\end{center}

\begin{align*}
&2\times 0=2\\
&2\times 2=4\end{align*}

\columnbreak

\underline{Tracé de $D_3.$}
\begin{center}
\begin{tabular}{|c|c|c|}\hline
$x$&$0$&$2$\\ \hline
$y$&$3$&$-1$\\ \hline
\end{tabular}
\end{center}

\begin{align*}
&-2\times 0+3=3\\
&-2\times 2+3=-1\end{align*}

\columnbreak

\underline{Tracé de $D_4.$}
\begin{center}
\begin{tabular}{|c|c|c|}\hline
$x$&$0$&$2$\\ \hline
$y$&$-2$&$-2$\\ \hline
\end{tabular}
\end{center}

\end{multicols}

On place à chaque fois les deux points en gris, puis on trace les droites en couleur~:


\begin{center}
\newrgbcolor{ffxfqq}{1. 0.4980392156862745 0.}
\newrgbcolor{xfqqff}{0.4980392156862745 0. 1.}
\newrgbcolor{uququq}{0.25098039215686274 0.25098039215686274 0.25098039215686274}
\psset{xunit=0.8cm,yunit=0.8cm,algebraic=true,dimen=middle,dotstyle=o,dotsize=5pt 0,linewidth=2.pt,arrowsize=3pt 2,arrowinset=0.25}
\begin{pspicture*}(-3.135027624309382,-4.58)(7.206740331491703,4.3)
\multips(0,-4)(0,1.0){9}{\psline[linestyle=dashed,linecap=1,dash=1.5pt 1.5pt,linewidth=0.4pt,linecolor=lightgray]{c-c}(-3.135027624309382,0)(7.206740331491703,0)}
\multips(-3,0)(1.0,0){11}{\psline[linestyle=dashed,linecap=1,dash=1.5pt 1.5pt,linewidth=0.4pt,linecolor=lightgray]{c-c}(0,-4.58)(0,4.3)}
\psaxes[labelFontSize=\scriptstyle,xAxis=true,yAxis=true,Dx=1.,Dy=1.,ticksize=-2pt 0,subticks=2]{->}(0,0)(-3.135027624309382,-4.58)(7.206740331491703,4.3)
\psplot[linewidth=2.pt,linecolor=ffxfqq]{-3.135027624309382}{7.206740331491703}{(-4.--1.*x)/1.}
\rput[tl](4.454033149171266,1.66){\ffxfqq{$D_1$}}
\psplot[linewidth=2.pt,linecolor=xfqqff]{-3.135027624309382}{7.206740331491703}{(-0.--2.*x)/1.}
\rput[tl](1.3763535911602227,2.66){\xfqqff{$D_2$}}
\psplot[linewidth=2.pt,linecolor=green]{-3.135027624309382}{7.206740331491703}{(--3.-2.*x)/1.}
\rput[tl](3.4791160220994453,-3.5){\green{$D_3$}}
\psplot[linewidth=2.pt,linecolor=blue]{-3.135027624309382}{7.206740331491703}{(-2.-0.*x)/1.}
\rput[tl](4.,-1.5){\blue{$D_4$}}
\psdots[dotstyle=*,linecolor=gray](0.,0.)
\psdots[dotstyle=*,linecolor=gray](2.,4.)
\psdots[dotstyle=*,linecolor=gray](0.,3.)
\psdots[dotstyle=*,linecolor=gray](2.,-1.)
\psdots[dotstyle=*,linecolor=gray](0.,-4.)
\psdots[dotstyle=*,linecolor=gray](2.,-2.)
\psdots[dotstyle=*,linecolor=gray](0.,-2.)
\end{pspicture*}
\end{center}


\textbf{Remarque~:} La droite $D_4$ est horizontale. C'était prévisible, puisque la valeur de $y~$ ($-2$) est indépendante de $x.$


\end{exo}

\vspace*{-0.5cm}


\begin{exo}

On lit graphiquement les ordonnées à l'origine et les coefficients directeurs des droites~:


\setlength{\columnseprule}{1pt}

\begin{multicols}{2}

\begin{center}
\newrgbcolor{xfqqff}{0.4980392156862745 0. 1.}
\newrgbcolor{ffxfqq}{1. 0.4980392156862745 0.}
\psset{xunit=1.0cm,yunit=1.0cm,algebraic=true,dimen=middle,dotstyle=o,dotsize=5pt 0,linewidth=2.pt,arrowsize=3pt 2,arrowinset=0.25}
\begin{pspicture*}(-2.7,-2.56)(6.66,5.96)
\multips(0,-2)(0,1.0){9}{\psline[linestyle=dashed,linecap=1,dash=1.5pt 1.5pt,linewidth=0.4pt,linecolor=lightgray]{c-c}(-2.7,0)(6.66,0)}
\multips(-2,0)(1.0,0){10}{\psline[linestyle=dashed,linecap=1,dash=1.5pt 1.5pt,linewidth=0.4pt,linecolor=lightgray]{c-c}(0,-2.56)(0,5.96)}
\psaxes[labelFontSize=\scriptstyle,xAxis=true,yAxis=true,Dx=1.,Dy=1.,ticksize=-2pt 0,subticks=2]{->}(0,0)(-2.7,-2.56)(6.66,5.96)
\psplot[linewidth=2.pt,linecolor=green]{-1}{5}{(-1.--2.*x)/1.}
\psline[linewidth=2.pt]{->}(0.,-1.)(1.,-1.)
\psline[linewidth=2.pt,linecolor=xfqqff]{->}(1.,-1.)(1.,1.)
\psline[linewidth=2.pt]{->}(1.,1.)(2.,1.)
\psline[linewidth=2.pt,linecolor=xfqqff]{->}(2.,1.)(2.,3.)
\psline[linewidth=2.pt]{->}(2.,3.)(3.,3.)
\psline[linewidth=2.pt,linecolor=xfqqff]{->}(3.,3.)(3.,5.)
\rput[tl](2.5,5.58){\green{$D_1$}}
\rput[tl](1.06,-0.34){\xfqqff{$+2$}}
\rput[tl](2.06,2.04){\xfqqff{$+2$}}
\rput[tl](3.06,4.06){\xfqqff{$+2$}}
\rput[tl](-1.88,-0.85){$\ffxfqq{\fbox{$b=-1$}}$}
\psline[linewidth=2.pt]{->}(4.72,2.74)(3.84,3.82)
\rput[tl](4.22,2.7){\xfqqff{\fbox{$a=2$}}}
\psdots[dotstyle=*,linecolor=ffxfqq](0.,-1.)
\end{pspicture*}
\end{center}


\definecolor{VIOLET}{rgb}{0.4980392156862745,0.,1.}
\definecolor{ORANGE}{rgb}{1.,0.4980392156862745,0.}

\begin{align*}
&\textcolor{green}{D_1:y=\textcolor{VIOLET}{2}x\textcolor{ORANGE}{-1}}\\
&D_1:y=2x-1
\end{align*}

\columnbreak


\begin{center}
\newrgbcolor{xfqqff}{0.4980392156862745 0. 1.}
\newrgbcolor{ffxfqq}{1. 0.4980392156862745 0.}
\psset{xunit=1.0cm,yunit=1.0cm,algebraic=true,dimen=middle,dotstyle=o,dotsize=5pt 0,linewidth=2.pt,arrowsize=3pt 2,arrowinset=0.25}
\begin{pspicture*}(-2.7,-2.56)(6.66,5.96)
\multips(0,-2)(0,1.0){9}{\psline[linestyle=dashed,linecap=1,dash=1.5pt 1.5pt,linewidth=0.4pt,linecolor=lightgray]{c-c}(-2.7,0)(6.66,0)}
\multips(-2,0)(1.0,0){10}{\psline[linestyle=dashed,linecap=1,dash=1.5pt 1.5pt,linewidth=0.4pt,linecolor=lightgray]{c-c}(0,-2.56)(0,5.96)}
\psaxes[labelFontSize=\scriptstyle,xAxis=true,yAxis=true,Dx=1.,Dy=1.,ticksize=-2pt 0,subticks=2]{->}(0,0)(-2.7,-2.56)(6.66,5.96)
\psplot[linewidth=2.pt,linecolor=green]{-1}{5}{(--4.-1.*x)/1.}
\rput[tl](4.,0.56){\green{$D_2$}}
\rput[tl](1.06,3.56){\xfqqff{$-1$}}
\rput[tl](-1.78,4.48){$\ffxfqq{\fbox{$b=4$}}$}
\rput[tl](3.62,3.66){$\xfqqff{\fbox{$a=-1$}}$}
\psline[linewidth=2.pt]{->}(0.,4.)(1.,4.)
\psline[linewidth=2.pt,linecolor=xfqqff]{->}(1.,4.)(1.,3.)
\psline[linewidth=2.pt]{->}(1.,3.)(2.,3.)
\psline[linewidth=2.pt,linecolor=xfqqff]{->}(2.,3.)(2.,2.)
\psline[linewidth=2.pt]{->}(2.,2.)(3.,2.)
\psline[linewidth=2.pt,linecolor=xfqqff]{->}(3.,2.)(3,1.)
\rput[tl](2.06,2.56){\xfqqff{$-1$}}
\rput[tl](3.06,1.56){\xfqqff{$-1$}}
\psline[linewidth=2.pt]{->}(3.78,3.06)(2.76,2.48)
\psdots[dotstyle=*,linecolor=ffxfqq](0.,4.)
\end{pspicture*}
\end{center}

\definecolor{VIOLET}{rgb}{0.4980392156862745,0.,1.}
\definecolor{ORANGE}{rgb}{1.,0.4980392156862745,0.}

\begin{align*}
&\textcolor{green}{D_2:y=\textcolor{VIOLET}{-1}x\textcolor{ORANGE}{+4}}\\
&D_2:y=-x+4
\end{align*}

\end{multicols}

\setlength{\columnseprule}{1pt}

\begin{multicols}{2}

\begin{center}
\newrgbcolor{ffxfqq}{1. 0.4980392156862745 0.}
\newrgbcolor{xfqqff}{0.4980392156862745 0. 1.}
\psset{xunit=1.0cm,yunit=1.0cm,algebraic=true,dimen=middle,dotstyle=o,dotsize=5pt 0,linewidth=2.pt,arrowsize=3pt 2,arrowinset=0.25}
\begin{pspicture*}(-1.7098267872637365,-4.222754171484549)(7.291644850785794,1.951332144357107)
\multips(0,-4)(0,1.0){7}{\psline[linestyle=dashed,linecap=1,dash=1.5pt 1.5pt,linewidth=0.4pt,linecolor=lightgray]{c-c}(-1.7098267872637365,0)(7.291644850785794,0)}
\multips(-1,0)(1.0,0){10}{\psline[linestyle=dashed,linecap=1,dash=1.5pt 1.5pt,linewidth=0.4pt,linecolor=lightgray]{c-c}(0,-4.222754171484549)(0,1.951332144357107)}
\psaxes[labelFontSize=\scriptstyle,xAxis=true,yAxis=true,Dx=1.,Dy=1.,ticksize=-2pt 0,subticks=2]{->}(0,0)(-1.7098267872637365,-4.222754171484549)(7.291644850785794,1.951332144357107)
\psplot[linewidth=2.pt,linecolor=green]{-1}{5}{(-2.--0.5*x)/1.}
\rput[tl](4.2334525634997355,0.9127008015052398){\green{$D_3$}}
\rput[tl](-1.5,-1.43383667678972){$\ffxfqq{\fbox{$b=-2$}}$}
\rput[tl](3.86800820212593,-2.3){$\xfqqff{\fbox{$a=0,5$}}$}
\psline[linewidth=2.pt]{->}(0.,-2.)(1.,-2.)
\psline[linewidth=2.pt,linecolor=xfqqff]{->}(1.,-2.)(1.,-1.5)
\psline[linewidth=2.pt]{->}(1.,-1.5)(2.,-1.5)
\psline[linewidth=2.pt,linecolor=xfqqff]{->}(2.,-1.5)(2.,-1.)
\psline[linewidth=2.pt]{->}(2.,-1.)(3.,-1.)
\psline[linewidth=2.pt,linecolor=xfqqff]{->}(3.,-1.)(3.,-0.5)
\rput[tl](1.12,-1.7){\xfqqff{$+0,5$}}
\rput[tl](2.12,-1.2){\xfqqff{$+0,5$}}
\rput[tl](3.12,-0.7){\xfqqff{$+0,5$}}
\psline[linewidth=2.pt]{->}(4.214218649743219,-2.1)(3.6,-1.06)
\psdots[dotstyle=*,linecolor=ffxfqq](0.,-2.)
\end{pspicture*}

\end{center}

\definecolor{VIOLET}{rgb}{0.4980392156862745,0.,1.}
\definecolor{ORANGE}{rgb}{1.,0.4980392156862745,0.}

\begin{align*}
&\textcolor{green}{D_3:y=\textcolor{VIOLET}{0,5}x\textcolor{ORANGE}{-2}}\\
&D_3:y=0,5x-2
\end{align*}

\columnbreak


\begin{center}
\newrgbcolor{ffxfqq}{1. 0.4980392156862745 0.}
\newrgbcolor{xfqqff}{0.4980392156862745 0. 1.}
\psset{xunit=1.0cm,yunit=1.0cm,algebraic=true,dimen=middle,dotstyle=o,dotsize=5pt 0,linewidth=2.pt,arrowsize=3pt 2,arrowinset=0.25}
\begin{pspicture*}(-2.342916244511982,-4.076140688436127)(6.870029275496755,4.3690593715718675)
\multips(0,-4)(0,1.0){9}{\psline[linestyle=dashed,linecap=1,dash=1.5pt 1.5pt,linewidth=0.4pt,linecolor=lightgray]{c-c}(-2.342916244511982,0)(6.870029275496755,0)}
\multips(-2,0)(1.0,0){10}{\psline[linestyle=dashed,linecap=1,dash=1.5pt 1.5pt,linewidth=0.4pt,linecolor=lightgray]{c-c}(0,-4.076140688436127)(0,4.3690593715718675)}
\psaxes[labelFontSize=\scriptstyle,xAxis=true,yAxis=true,Dx=1.,Dy=1.,ticksize=-2pt 0,subticks=2]{->}(0,0)(-2.342916244511982,-4.076140688436127)(6.870029275496755,4.3690593715718675)
\psplot[linewidth=2.pt,linecolor=green]{-1}{5}{(--3.-1.5*x)/1.}
\rput[tl](3.15,-2.5012782063833545){\green{$D_4$}}
\rput[tl](-1.9295148429731284,3.6209996925968){$\ffxfqq{\fbox{$b=3$}}$}
\rput[tl](3.936847902673461,-1.9500763376648838){$\xfqqff{\fbox{$a=-1,5$}}$}
\rput[tl](1.0430380919014859,2.35){\xfqqff{$-1,5$}}
\rput[tl](2.06669870523579,0.85){\xfqqff{$-1,5$}}
\rput[tl](3.0706735375444345,-0.65){\xfqqff{$-1,5$}}
\psline[linewidth=2.pt]{->}(4.214218649743219,-1.8954506069461055)(3.791072547099865,-1.0683923154159147)
\psline[linewidth=2.pt]{->}(0.,3.)(1.,3.)
\psline[linewidth=2.pt,linecolor=xfqqff]{->}(1.,3.)(1.,1.5)
\psline[linewidth=2.pt]{->}(1.,1.5)(2.,1.5)
\psline[linewidth=2.pt,linecolor=xfqqff]{->}(2.,1.5)(2.,0.)
\psline[linewidth=2.pt]{->}(2.,0.)(3.,0.)
\psline[linewidth=2.pt,linecolor=xfqqff]{->}(3.,0.)(3.,-1.5)
\psdots[dotstyle=*,linecolor=ffxfqq](0.,3.)
\end{pspicture*}
\end{center}

\definecolor{VIOLET}{rgb}{0.4980392156862745,0.,1.}
\definecolor{ORANGE}{rgb}{1.,0.4980392156862745,0.}

\begin{align*}
&\textcolor{green}{D_4:y=\textcolor{VIOLET}{-1,5}x\textcolor{ORANGE}{+3}}\\
&D_4:y=-1,5x+3
\end{align*}

\end{multicols}

\end{exo}


\begin{exo}

Le graphique suivant donne le prix payé dans une pompe à essence en fonction de la quantité de gazole achetée.

\begin{center}
\newrgbcolor{xfqqff}{0.4980392156862745 0. 1.}

\psset{xunit=1cm,yunit=1cm,algebraic=true,dimen=middle,dotstyle=o,dotsize=5pt 0,linewidth=1.6pt,arrowsize=3pt 2,arrowinset=0.25}
\begin{pspicture*}(-0.7596559723553132,-0.6294697319234536)(6.484452736237389,6.424504889829535)
\multips(0,0)(0,1.0){8}{\psline[linestyle=dashed,linecap=1,dash=1.5pt 1.5pt,linewidth=0.4pt,linecolor=lightgray]{c-c}(-0.7596559723553132,0)(6.484452736237389,0)}
\multips(0,0)(1.0,0){8}{\psline[linestyle=dashed,linecap=1,dash=1.5pt 1.5pt,linewidth=0.4pt,linecolor=lightgray]{c-c}(0,-0.6294697319234536)(0,6.424504889829535)}
\psaxes[labelFontSize=\scriptstyle,xAxis=true,yAxis=true,Dx=1.,Dy=1.,ticksize=-2pt 0,subticks=2]{->}(0,0)(-0.7596559723553132,-0.6294697319234536)(6.484452736237389,6.424504889829535)
\rput[tl](5.15,0.8){\parbox{2.5210726947176276 cm}{gazole \\(en $\ell$)}}
\rput[lt](0.12,5.302713777475287){\parbox{2.5210726947176276 cm}{prix \\(en \euro)}}
\psplot[linewidth=2.pt]{0.}{6.484452736237389}{1.5*x}
\psline[linewidth=2.pt,linestyle=dashed,dash=2pt 2pt,linecolor=red](4.,0.)(4.,6.)
\psline[linewidth=2.pt,linestyle=dashed,dash=2pt 2pt,linecolor=red](4.,6.)(0.,6.)
\psline[linewidth=2.pt,linecolor=xfqqff]{->}(0.,0.)(1.,0.)
\psline[linewidth=2.pt,linecolor=xfqqff]{->}(1.,0.)(1.,1.5)
\rput[tl](1.06,0.86){\xfqqff{$+1,5$}}
\end{pspicture*}
\end{center}


Il y a deux méthodes possibles pour répondre à la question~:

\begin{itemize}
\item[\textbullet] \textbf{Pointillés rouges~:} 4 litres de gazole coûtent 6~\euro, donc le litre coûte $6\div 4=1,5$~\euro.
\item[\textbullet] \textbf{Flèches violettes~:} chaque litre coûte 1,5~\euro.
\end{itemize}


\end{exo}

\begin{exo}

Dans chaque question, $u$ est une suite arithmétique de raison $r.$

\begin{enumerate}
\item $u_0=2$ et $r=4.$

\begin{align*}
u_1&=2+4=6\\
u_2&=6+4=10\\
u_3&=10+4=14.\\
\end{align*}
\item $u_0=5$ et $r=-2.$

\begin{align*}
u_1&=5-2=3\\
u_2&=3-2=1\\
u_3&=1-2=-1.\\
\end{align*}

\item $u_0=10$ et $r=1,5.$

Pour obtenir $u_6,$ on part de $u_0=10$ et on rajoute 6 fois $1,5.$ Donc
\[u_6=10+6\times 1,5=10+9=19.\]

\item $u_0=4$ et $u_2=10.$

\begin{center}
    $\xymatrix@R=0.5pc@C=3pc{
    *+[F]+{4} \ar@/^0.5cm/[r]|{\red{+r}} & 
    *+[F]+{???} \ar@/^0.5cm/[r]|{\red{+r}} & *+[F]+{10} \\
    \txt{\blue{$u_0$}}&
    \txt{\blue{$u_1$}}&\txt{\blue{$u_2$}}
    }$
    \end{center}
    
    D'après le schéma ci-dessus~:
    \[r=(10-4)\div 2=6\div 2=3.\]  On obtient donc le schéma complété~:
    
    \begin{center}
    $\xymatrix@R=0.5pc@C=3pc{
    *+[F]+{4} \ar@/^0.5cm/[r]|{\red{+3}} & 
    *+[F]+{7} \ar@/^0.5cm/[r]|{\red{+3}} & 
    *+[F]+{10} \ar@/^0.5cm/[r]|{\red{+3}} & 
    *+[F]+{13} \ar@/^0.5cm/[r]|{\red{+3}}& 
    *+[F]+{16} \\
    \txt{\blue{$u_0$}}&
    \txt{\blue{$u_1$}}&\txt{\blue{$u_2$}}&
    \txt{\blue{$u_3$}}&
    \txt{\blue{$u_4$}}
    }$
    \end{center}
    
    (On peut aussi obtenir $u_4$ avec le calcul~: $u_4=4+4\times 3=4+12=16.$)

\item ~{}\begin{center}
    $\xymatrix@R=0.5pc@C=3pc{
    *+[F]+{5} \ar@/^0.5cm/[r]|{\red{+r}} & 
    *+[F]+{???} \ar@/^0.5cm/[r]|{\red{+r}} & *+[F]+{???} \ar@/^0.5cm/[r]|{\red{+r}} & 
    *+[F]+{12,5} \\
    \txt{\blue{$u_0$}}&
    \txt{\blue{$u_1$}}&\txt{\blue{$u_2$}}&
    \txt{\blue{$u_3$}}
    }$
    \end{center}
    
    D'après le schéma ci-dessus~:
    \[r=(12,5-5)\div 3=7,5\div 3=2,5.\]
\end{enumerate}

\end{exo}





\begin{exo}

Le 01/01/2019, on dépose $300$~\euro~{} sur un compte en banque. Tous les mois à partir de cette date, on déposera $75$~\euro~{} sur ce compte.

On note $u_n$ la somme sur le compte après $n$ mois -- on a donc en particulier $u_0=300.$

\begin{enumerate}
\item $u_1=300+75=375,$ $u_2=375+75=450.$

On aura 375~\euro~{} le 1\up{er} février et 450~\euro~{} le 1\up{er} mars.
\item La suite $u$ est arithmétique de raison $r=75.$

\item La formule à entrer dans la cellule C2 est \[\text{=B2+75}\]
\item ~{}


\begin{center}
\newrgbcolor{xfqqff}{0.4980392156862745 0. 1.}

\newrgbcolor{ududff}{0.30196078431372547 0.30196078431372547 1.}
\psset{xunit=0.5cm,yunit=0.005cm,algebraic=true,dimen=middle,dotstyle=o,dotsize=5pt 0,linewidth=1.6pt,arrowsize=3pt 2,arrowinset=0.25}
\begin{pspicture*}(-4.582075695581853,-212.02670121334532)(13.5,1300)
\multips(0,0)(0,100.0){13}{\psline[linestyle=dashed,linecap=1,dash=1.5pt 1.5pt,linewidth=0.4pt,linecolor=lightgray]{c-c}(0,0)(13.5,0)}
\multips(0,0)(1.0,0){14}{\psline[linestyle=dashed,linecap=1,dash=1.5pt 1.5pt,linewidth=0.4pt,linecolor=lightgray]{c-c}(0,0)(0,1300)}
\psaxes[labelFontSize=\scriptstyle,xAxis=true,yAxis=true,Dx=1.,Dy=100.,ticksize=-2pt 0,subticks=2]{->}(0,0)(0.,0.)(13.5,1300)
\rput[tl](7.439919702003364,-100){$n$ (nombre de mois)}
\rput[tl](-4.5,1240){$u_n$}
\rput[tl](-4.5,1170){(somme}
\rput[tl](-4.5,1100){en \euro)}
\rput[tl](1.3,350){\textcolor{xfqqff}{$+75$}}
\psline[linewidth=1.2pt]{->}(0.,300.)(1.,300)
\psline[linewidth=1.2pt,linecolor=xfqqff]{->}(1.,300.)(1.,375)
\psline[linewidth=2.pt,linecolor=red](0.,300.)(12.,1200.)
\psline[linewidth=2.pt,linecolor=red,linestyle=dotted](12.,0.)(12.,1200.)
\psline[linewidth=2.pt,linecolor=red,linestyle=dotted](0.,1200.)(12.,1200.)
\psdots[dotstyle=*,linecolor=ududff](0.,300.)
\psdots[dotstyle=*,linecolor=ududff](1.,375.)
\psdots[dotstyle=*,linecolor=ududff](2.,450.)
\psdots[dotstyle=*,linecolor=ududff](3.,525.)
\end{pspicture*}
\end{center}
\item  L'équation de la droite qui passe par tous les points est \[y=75x+300\]
($75$ correspond à $r,$ et $300$ à $u_0$).

\item Le 01/01/2020 (donc au bout de 12 mois), on aura
\[75\times 12+300=\np{1200}~\text{\euro}.\]

La réponse est confirmée par la construction en pointillés rouges du graphique.

\end{enumerate}
\end{exo}

\begin{exo}

\begin{enumerate}
\item $u_1=600-50=550,$ $u_2=550-50=500.$
\item La suite $u$ est arithmétique de raison $r=-50.$
\item ~{}


\begin{center}
\newrgbcolor{xfqqff}{0.4980392156862745 0. 1.}

\newrgbcolor{ududff}{0.30196078431372547 0.30196078431372547 1.}
\psset{xunit=0.8cm,yunit=0.008cm,algebraic=true,dimen=middle,dotstyle=o,dotsize=5pt 0,linewidth=1.6pt,arrowsize=3pt 2,arrowinset=0.25}
\begin{pspicture*}(-2.082075695581853,-102.02670121334532)(10.340667257696177,639.4216885220152)
\multips(0,0)(0,100.0){7}{\psline[linestyle=dashed,linecap=1,dash=1.5pt 1.5pt,linewidth=0.4pt,linecolor=lightgray]{c-c}(0,0)(10.340667257696177,0)}
\multips(0,0)(1.0,0){11}{\psline[linestyle=dashed,linecap=1,dash=1.5pt 1.5pt,linewidth=0.4pt,linecolor=lightgray]{c-c}(0,0)(0,639.4216885220152)}
\psaxes[labelFontSize=\scriptstyle,xAxis=true,yAxis=true,Dx=1.,Dy=100.,ticksize=-2pt 0,subticks=2]{->}(0,0)(0.,0.)(10.340667257696177,639.4216885220152)
\rput[tl](6.439919702003364,-70){$n~\text{(nombre d'années)}$}
\rput[tl](-1.9802499336697383,540){$u_n$}
\rput[tl](-1.9802499336697383,500){(quota}
\rput[tl](-1.9802499336697383,460){de pêche)}
\rput[tl](1.3,580){\textcolor{xfqqff}{$-50$}}
\psline[linewidth=1.2pt]{->}(0.,600.)(1.,600)
\psline[linewidth=1.2pt,linecolor=xfqqff]{->}(1.,600.)(1.,550)
\psline[linewidth=2.pt,linecolor=red](0.,600.)(10.,100.)
\psline[linewidth=2.pt,linecolor=red,linestyle=dotted](10.,0.)(10.,100.)
\psline[linewidth=2.pt,linecolor=red,linestyle=dotted](0.,100.)(10.,100.)
\psdots[dotstyle=*,linecolor=ududff](0.,600.)
\psdots[dotstyle=*,linecolor=ududff](1.,550.)
\psdots[dotstyle=*,linecolor=ududff](2.,500.)
\psdots[dotstyle=*,linecolor=ududff](3.,450.)
\end{pspicture*}
\end{center}

\item L'équation de la droite qui passe par tous les points est \[y=-50x+600\]
($-50$ correspond à $r,$ et $600$ à $u_0$).

\item Le quota de pêche en 2025 (donc au bout de 10 ans) est 
\[-50\times 10+600=100~\text{Tonnes}.\]

La réponse est confirmée par la construction en pointillés rouges du graphique.


\end{enumerate}



\end{exo}

\begin{exo}

On note $S$ la somme à calculer, que l'on écrit à l'endroit, puis à l'envers~:
\begin{alignat*}{8}
&S&&= \textcolor{red}{1}&&+\textcolor{blue}{2}&&+\textcolor{green}{3}&&+\cdots  &&+\textcolor{orange}{98} &&+ \textcolor{violet}{99}&&+ \textcolor{brown}{100}\\
&S&&= \textcolor{red}{100}&&+\textcolor{blue}{99}&&+\textcolor{green}{98}&&+\cdots &&+\textcolor{orange}{3} &&+ \textcolor{violet}{2}&&+ \textcolor{brown}{1}
\end{alignat*}


On ajoute membre à membre les deux lignes. On remarque que la somme de chaque couple d'une même couleur vaut toujours $101~:$


\[S+S=\underbrace{\textcolor{red}{101}+\textcolor{blue}{101}+\textcolor{green}{101}+\cdots+\textcolor{orange}{101}+\textcolor{violet}{101}+\textcolor{brown}{101}}_{100~\text{termes}}.\]
On a donc
\[2S=100\times 101\hspace{2cm} S=\frac{100\times 101}{2}=\np{5050}.\]
\end{exo}

\begin{exo}

On construit une pyramide en superposant des carrés~: tout en haut, on a $u_0=1$ carré, en dessous $u_1=3$ carrés, etc.


\begin{center}
\psset{xunit=1.0cm,yunit=1.0cm,algebraic=true,dimen=middle,dotstyle=o,dotsize=5pt 0,linewidth=1.6pt,arrowsize=3pt 2,arrowinset=0.25}
\begin{pspicture*}(-3.34,1.36)(2.42,5.64)
\pspolygon[linewidth=2.pt,fillcolor=pink,fillstyle=solid,opacity=0.1](-1.,5.)(0.,5.)(0.,4.)(-1.,4.)
\pspolygon[linewidth=2.pt,fillcolor=blue,fillstyle=solid,opacity=0.1](-2.,4.)(-1.,4.)(-1.,3.)(-2.,3.)
\pspolygon[linewidth=2.pt,fillcolor=blue,fillstyle=solid,opacity=0.1](-1.,4.)(0.,4.)(0.,3.)(-1.,3.)
\pspolygon[linewidth=2.pt,fillcolor=blue,fillstyle=solid,opacity=0.1](0.,4.)(1.,4.)(1.,3.)(0.,3.)
\pspolygon[linewidth=2.pt,fillcolor=green,fillstyle=solid,opacity=0.1](-3.,3.)(-2.,3.)(-2.,2.)(-3.,2.)
\pspolygon[linewidth=2.pt,fillcolor=green,fillstyle=solid,opacity=0.1](-2.,3.)(-1.,3.)(-1.,2.)(-2.,2.)
\pspolygon[linewidth=2.pt,fillcolor=green,fillstyle=solid,opacity=0.1](-1.,3.)(0.,3.)(0.,2.)(-1.,2.)
\pspolygon[linewidth=2.pt,fillcolor=green,fillstyle=solid,opacity=0.1](0.,3.)(1.,3.)(1.,2.)(0.,2.)
\pspolygon[linewidth=2.pt,fillcolor=green,fillstyle=solid,opacity=0.1](1.,3.)(2.,3.)(2.,2.)(1.,2.)
\end{pspicture*}
\end{center}

\begin{enumerate}
\item \`A chaque étage de la pyramide, on ajoute deux carrés, donc $u$ est arithmétique de raison $r=2.$
\item \begin{itemize}
\item[\textbullet] Le nombre de carrés de la 1\up{re} rangée est $u_0=1.$
\item[\textbullet] Le nombre de carrés de la 2\up{e} rangée est $u_1=3.$
\item[\textbullet] Le nombre de carrés de la 3\up{e} rangée est $u_2=5.$
\item[\textbullet] $\cdots$
\item[\textbullet] Le nombre de carrés de la 100\up{e} rangée est $u_{99}=1+99\times 2=199.$
\end{itemize}

\medskip

\danger Il y a un décalage~: le nombre de carrés de la 100\up{e} rangée est $u_{99}.$

\item Le nombre total de carrés de la 1\up{re} à la 100\up{e} rangée est 
\[1+3+5+\cdots+199.\]

On calcule cette somme comme dans l'exercice précédent~: on note \[S=1+3+5+\cdots+195+197+199\] et on écrit $S$ à l'endroit et à l'envers~:

\begin{alignat*}{8}
&S&&= \textcolor{red}{1}&&+\textcolor{blue}{3}&&+\textcolor{green}{5}&&+\cdots  &&+\textcolor{orange}{195} &&+ \textcolor{violet}{197}&&+ \textcolor{brown}{199}\\
&S&&= \textcolor{red}{199}&&+\textcolor{blue}{197}&&+\textcolor{green}{195}&&+\cdots &&+\textcolor{orange}{5} &&+ \textcolor{violet}{3}&&+ \textcolor{brown}{1}
\end{alignat*}

La somme des termes d'une même couleur est toujours égale à 200 et il y a 100 termes (autant que le nombre de rangées). On a donc~:

\[2S=100\times 200\hspace{2cm} S=\frac{100\times 200}{2}=\np{10000}.\]

\end{enumerate}
\end{exo}


\section{Études de fonctions}





\begin{exo}

Un voyageur de commerce ($=$ un représentant) fait une note de frais pour chaque jour de travail où il utilise sa voiture. Il reçoit une part fixe de 30~\euro, et une indemnité de 0,5~\euro/km.

\medskip

\textbf{Remarque~:} On peut penser que l'indemnité kilométrique sert à rembourser les frais de déplacement (par exemple si le représentant utilise sa propre voiture)~; et que la part fixe sert à payer les repas.

\begin{enumerate}
\item S'il fait 120~km dans la journée, le montant de la note de frais est de \[30+120\times 0,5=30+60=90~\text{\euro}.\]
\item On note $x$ le nombre de km parcourus par le voyageur de commerce, et $f(x)$ le montant de la note de frais. On a alors \[f(x)=30+x\times 0,5=0,5x+30.\]
\item La fonction $f$ est affine, puisque $f(x)=0,5x+30$ (c'est bien une fonction de la forme $f(x)=ax+b,$ avec $a=0,5$ et $b=30$). Sa courbe représentative est donc une droite, que l'on trace à partir d'un tableau de valeurs avec deux valeurs~; par exemple~:

\setlength{\columnseprule}{1pt}

\begin{multicols}{2}
\begin{center}
 \begin{tabular}{|c|c|c|}\hline
$x$& $0$ &$120$ \\ \hline 
$f(x)$&$30$ &$90$  \\ \hline
\end{tabular}
\end{center}

\begin{align*}f(0)&=0,5\times 0+30=30\\
f(120)&=0,5\times 120+30=90\end{align*}

On place les points de coordonnées $(0;30)$ et $(120;90),$ puis on trace la droite -- en réalité un segment, puisqu'on va de 0 à 200 en abscisse.

\end{multicols}

\medskip

\textbf{Remarque~:} On a choisi les valeurs  $0$ et $120,$ mais on peut prendre n'importe quelles valeurs -- l'avantage de $0,$ c'est que le calcul est facile~; et l'avantage de $120,$ c'est qu'on a déjà fait le calcul dans la question 1.

\begin{center}
\newrgbcolor{ududff}{0.30196078431372547 0.30196078431372547 1.}
\psset{xunit=0.025cm,yunit=0.05cm,algebraic=true,dimen=middle,dotstyle=o,dotsize=5pt 0,linewidth=2.pt,arrowsize=3pt 2,arrowinset=0.25}
\begin{pspicture*}(-25.571925933684398,-8.5)(219.39050611863857,137.42562531094046)
\multips(0,0)(0,10.0){15}{\psline[linestyle=dashed,linecap=1,dash=1.5pt 1.5pt,linewidth=0.4pt,linecolor=lightgray]{c-c}(0,0)(219.39050611863857,0)}
\multips(0,0)(20.0,0){13}{\psline[linestyle=dashed,linecap=1,dash=1.5pt 1.5pt,linewidth=0.4pt,linecolor=lightgray]{c-c}(0,0)(0,137.42562531094046)}
\psaxes[labelFontSize=\scriptstyle,xAxis=true,yAxis=true,Dx=20.,Dy=10.,ticksize=-2pt 0,subticks=2]{->}(0,0)(0.,0.)(219.39050611863857,137.42562531094046)
\rput[tl](130,8.092313283315638){km parcourus}
\rput[lt](5.833514073023679,126.71554530972082){\parbox{60.384832012684356 cm}{montant de la \\ note de frais}}
\psline[linewidth=2.pt,linecolor=ududff](0.,30.)(200.,130.)
\psline[linewidth=2.pt,linestyle=dashed,dash=2pt 2pt,linecolor=red](0.,75.)(90.,75.)
\psline[linewidth=2.pt,linestyle=dashed,dash=2pt 2pt,linecolor=red](90.,75.)(90.,0.)
\psdots[dotstyle=*,linecolor=ududff](0.,30.)
\psdots[dotstyle=*,linecolor=ududff](120.,90.)
\end{pspicture*}
\end{center}


\item Le voyageur de commerce a une note de frais de 75~\euro. Pour déterminer le nombre de km parcourus dans la journée, il y a deux méthodes~:

\begin{itemize}
\item[\textbullet] \textbf{Graphiquement.} On voit qu'il a parcouru 90~km (pointillés rouges)\footnote{La méthode graphique est simple, mais la réponse pourrait être imprécise.}.
\item[\textbullet] \textbf{Par le calcul.} On retire les frais fixes~: $75-30=45~\text{\euro}$ d'indemnité kilométrique. Puis, comme chaque km compte pour $0,5~\text{\euro},$ on divise~: $45\div 0,5=45\times 2=90~\text{km}.$\footnote{On peut aussi résoudre l'équation $0,5x+30=75.$}
\end{itemize}
\end{enumerate}

\end{exo}

\begin{exo}


\begin{enumerate}
\item \begin{itemize}
\item[\textbullet] Lorsqu'on télécharge 50 Mo, on paye 3~\euro.
\item[\textbullet] Lorsqu'on télécharge 150 Mo, les 100 premiers coûtent 3~\euro~; et les 50 suivants coûtent $50\times 0,04=2$~\euro. On paye donc au total $3+2=5$~\euro.
\end{itemize}

\item On complète le tableau de valeurs~:



\smallskip

\begin{center}
\begin{tabular}{|l|c|c|c|c|c|}
\hline
   Nombre de Mo &$0$ &$50$ &$100$ &$150$ &$200$ \\
	\hline
	Prix à payer &3&3&3&5&7 \\
	\hline
\end{tabular}
\end{center}

\textbf{Remarque~:} jusqu'à 100~Mo, on paye 3~\euro. Ensuite, chaque nouvelle tranche de 50~Mo est facturée 2~\euro.

\item On construit la courbe qui donne le prix payé en fonction du nombre de Mo téléchargés. Elle est constante sur l'intervalle $\left[0;100\right],$ puis affine sur l'intervalle $\left[100;200\right].$ Il faut donc utiliser une règle pour effectuer le tracé\footnote{On parle de fonction \og affine par morceaux \fg.}.

\begin{center}
\newrgbcolor{ududff}{0.30196078431372547 0.30196078431372547 1.}
\psset{xunit=0.015cm,yunit=0.75cm,algebraic=true,dimen=middle,dotstyle=o,dotsize=5pt 0,linewidth=2.pt,arrowsize=3pt 2,arrowinset=0.25}
\begin{pspicture*}(-47.57871396895785,-0.8165410199556531)(427.5432372505541,7.495654101995574)
\multips(0,0)(0,1.0){9}{\psline[linestyle=dashed,linecap=1,dash=1.5pt 1.5pt,linewidth=0.4pt,linecolor=lightgray]{c-c}(0,0)(427.5432372505541,0)}
\multips(0,0)(50.0,0){10}{\psline[linestyle=dashed,linecap=1,dash=1.5pt 1.5pt,linewidth=0.4pt,linecolor=lightgray]{c-c}(0,0)(0,7.495654101995574)}
\psaxes[labelFontSize=\scriptstyle,xAxis=true,yAxis=true,Dx=50.,Dy=1.,ticksize=-2pt 0,subticks=2]{->}(0,0)(0.,0.)(427.5432372505541,7.495654101995574)
\psline[linewidth=2.pt,linecolor=ududff](0.,3.)(100.,3.)
\psline[linewidth=2.pt,linecolor=ududff](100.,3.)(200.,7.)
\psline[linewidth=2.pt,linestyle=dashed,dash=2pt 2pt,linecolor=red](0.,4.6)(140.,4.6)
\psline[linewidth=2.pt,linestyle=dashed,dash=2pt 2pt,linecolor=red](140.,4.6)(140.,0.)
\rput[tl](205.1042128603103,0.5298004434589823){Nombre de Mo}
\rput[tl](3.152993348115293,6.871263858093134){Prix}
\psdots[dotstyle=*,linecolor=ududff](0.,3.)
\psdots[dotstyle=*,linecolor=ududff](50.,3.)
\psdots[dotstyle=*,linecolor=ududff](100.,3.)
\psdots[dotstyle=*,linecolor=ududff](150.,5.)
\psdots[dotstyle=*,linecolor=ududff](200.,7.)
\end{pspicture*}
\end{center}

\item Il y a deux méthodes~:

\begin{itemize}
\item[\textbullet] \textbf{Graphiquement.} On voit qu'on a téléchargé 140~Mo (pointillés rouges).
\item[\textbullet] \textbf{Par le calcul.} J'ai payé 4,60~\euro, donc $3+1,60$~\euro. J'ai donc téléchargé $1,60\div 0,04=40$~Mo au-delà du 100\up{e}. Autrement dit, j'ai téléchargé 140~Mo.
\end{itemize}
\end{enumerate}



\end{exo}



\begin{exo}

Pour louer une voiture je dois payer~:

\begin{itemize}
\item[\textbullet] une part fixe de 20~\euro.
\item[\textbullet] 0,6~\euro~{} par km parcouru.
\end{itemize}

\medskip

\begin{enumerate}
\item Pour 100~km, je payerai \[P(100)=20+100\times 0,6=80~\text{\euro}~;\] et pour 50~km, je payerai \[P(50)=20+50\times 0,6=50~\text{\euro}.\]

\item D'une manière générale, pour $x$~km parcourus je payerai
\[20+x\times 0,6~\text{\euro}.\] Avec les notations de l'énoncé, cela donne
\[P(x)=0,6x+20.\]
\end{enumerate}

\end{exo}



\begin{exo}

\begin{enumerate}
\item Comme $120=60+60=60+6\times 10,$ le coût pour 120 minutes de location est 
\[15+6\times 5=45~\text{\euro}.\]

\item On complète le tableau de valeurs~:

\smallskip

\begin{center}
\begin{tabular}{|l|c|c|c|c|c|c|c|}
\hline
   Durée&0 &20 &40 &60 &80&100&120 \\
	\hline
	Prix&15&15&15&15&25&35&45 \\
	\hline
\end{tabular}
\end{center}

\smallskip

\item On construit le graphique~:


\begin{center}
\newrgbcolor{rvwvcq}{0.08235294117647059 0.396078431372549 0.7529411764705882}
\psset{xunit=0.05cm,yunit=0.1cm,algebraic=true,dimen=middle,dotstyle=o,dotsize=5pt 0,linewidth=1.6pt,arrowsize=3pt 2,arrowinset=0.25}
\begin{pspicture*}(-15,-4.5)(127.83114817349465,50.324954219802905)
\multips(0,0)(0,5.0){11}{\psline[linestyle=dashed,linecap=1,dash=1.5pt 1.5pt,linewidth=0.4pt,linecolor=lightgray]{c-c}(0,0)(127.83114817349465,0)}
\multips(0,0)(10.0,0){14}{\psline[linestyle=dashed,linecap=1,dash=1.5pt 1.5pt,linewidth=0.4pt,linecolor=lightgray]{c-c}(0,0)(0,50.324954219802905)}
\psaxes[labelFontSize=\scriptstyle,xAxis=true,yAxis=true,Dx=10.,Dy=5.,ticksize=-2pt 0,subticks=2]{->}(0,0)(0.,0.)(127.83114817349465,50.324954219802905)
\psline[linewidth=2.pt,linecolor=rvwvcq](0.,15.)(60.,15.)
\psline[linewidth=2.pt,linecolor=rvwvcq](60.,15.)(120.,45.)
\rput[tl](70,5.091035826702215){temps (en min)}
\rput[tl](4,46.25846249170123){prix (en euros)}
\psdots[dotstyle=*,linecolor=rvwvcq](0.,15.)
\psdots[dotstyle=*,linecolor=rvwvcq](20.,15.)
\psdots[dotstyle=*,linecolor=rvwvcq](40.,15.)
\psdots[dotstyle=*,linecolor=rvwvcq](60.,15.)
\psdots[dotstyle=*,linecolor=rvwvcq](80.,25.)
\psdots[dotstyle=*,linecolor=rvwvcq](100.,35.)
\psdots[dotstyle=*,linecolor=rvwvcq](120.,45.)
\end{pspicture*}
\end{center}


\end{enumerate}

\end{exo}

\begin{exo}

Les gares de Calais et de Boulogne-sur-mer sont distantes de 30~km. Un train part à 12 h de Boulogne-sur-mer en direction de Calais et roule à la vitesse de 40~km/h. Un train part de Calais à 12 h 15 et fait route en sens inverse à la vitesse de 60~km/h.

\begin{enumerate}
\item Le train qui part à 12 h de Boulogne-sur-mer roule à la vitesse de 40~km/h, donc il parcourt 40~km en 60~min. Pour savoir quand il arrive à Calais, on complète un tableau de proportionnalité~:

\begin{center}
\begin{tabular}{|c|c|c|}\hline
temps (en min)& 60&? \\ \hline 
distance (en km)&40& 30 \\ \hline
\end{tabular}
\end{center}

Le train mettra $\frac{60\times 30}{40}=\frac{\np{1800}}{40}=45$~min pour arriver à Calais, donc il y sera à 12 h 45.

\medskip

Pour le train qui part de Calais, le calcul est plus facile~: il roule à 60~km/h, donc parcourt 60~km en 60~min~; et ainsi 30~km en 30~min. Comme il part à 12 h 15, il arrive à 12 h 45 lui aussi.

\medskip

On peut ainsi représenter la marche des deux trains~:

\begin{center}
\psset{xunit=0.75cm,yunit=0.75cm,algebraic=true,dimen=middle,dotstyle=o,dotsize=5pt 0,linewidth=2.pt,arrowsize=3pt 2,arrowinset=0.25}
\begin{pspicture*}(-3.5,-0.96)(11,6.5)
\multips(0,0)(0,1.0){7}{\psline[linestyle=dashed,linecap=1,dash=1.5pt 1.5pt,linewidth=0.4pt,linecolor=gray]{c-c}(0,0)(10,0)}
\multips(0,0)(1.0,0){11}{\psline[linestyle=dashed,linecap=1,dash=1.5pt 1.5pt,linewidth=0.4pt,linecolor=gray]{c-c}(0,0)(0,6)}
\psaxes[labelFontSize=\scriptstyle,xAxis=true,yAxis=true,labels=none,Dx=1.,Dy=1.,ticksize=-2pt 0,subticks=2]{->}(0,0)(0.,0.)(11,6.5)
\begin{scriptsize}
\rput[tl](-0.4,-0.34){12h}
\rput[tl](1.4,-0.34){12h10}
\rput[tl](3.4,-0.34){12h20}
\rput[tl](5.4,-0.34){12h30}
\rput[tl](7.4,-0.34){12h40}
\rput[tl](9.4,-0.34){12h50}
\rput[tl](-0.3,0.18){0}
\rput[tl](-0.6,2.12){10}
\rput[tl](-0.6,4.14){20}
\rput[tl](-2.1,0.4){\fbox{Boulogne}}
\rput[tl](-2.1,6.4){\fbox{Calais}}
\rput[tl](-0.6,6.18){30}
\rput[tl](4.5,-0.34){\green{12h27}}
\end{scriptsize}
\psline[linewidth=2.pt,linecolor=red](0.,0.)(9.,6)
\psline[linewidth=2.pt,linecolor=blue](0.,6.)(3,6)
\psline[linewidth=2.pt,linecolor=blue](3.,6.)(9,0)
\psline[linewidth=2.pt,linestyle=dashed,dash=2pt 2pt,linecolor=green](5.4,3.6)(5.4,0)
\end{pspicture*}
\end{center}

\item Nous allons déterminer l'heure de croisement des trains par le calcul. Graphiquement, cela correspond à l'abscisse du point d'intersection des courbes.

\medskip

\`A 12h15, le train qui part de Boulogne-sur-mer a parcouru 10~km (facile à vérifier), il est donc à 20~km de Calais. C'est l'heure à laquelle le deuxième train part. Comme l'un roule à 40~km/h et l'autre à 60~km/h, tout se passe comme si un seul train devait parcourir 20~km à la vitesse de $40+60=100$~km/h. On complète un tableau de proportionnalité~:

\begin{center}
\begin{tabular}{|c|c|c|}\hline
temps (en min)& 60&? \\ \hline 
distance (en km)&100& 20 \\ \hline
\end{tabular}
\end{center}

$\frac{60\times 20}{100}=\frac{\np{1200}}{100}=12,$ donc il faudrait 12~min à ce train pour parcourir 20~km. Ainsi, les deux trains se croiseront-ils à \[\text{12 h 15 min}+\text{12 min}=\text{12 h 27 min}.\]
 
\end{enumerate}

\end{exo}



\begin{exo}

~{}





\begin{center}
\psset{xunit=1.0cm,yunit=1.0cm,algebraic=true,dimen=middle,dotstyle=o,dotsize=5pt 0,linewidth=2.pt,arrowsize=3pt 2,arrowinset=0.25}
\begin{pspicture*}(-2.36,-1.92)(6.64,3.58)
\multips(0,-1)(0,1.0){6}{\psline[linestyle=dashed,linecap=1,dash=1.5pt 1.5pt,linewidth=0.4pt,linecolor=lightgray]{c-c}(-2.36,0)(6.64,0)}
\multips(-2,0)(1.0,0){10}{\psline[linestyle=dashed,linecap=1,dash=1.5pt 1.5pt,linewidth=0.4pt,linecolor=lightgray]{c-c}(0,-1.92)(0,3.58)}
\psaxes[labelFontSize=\scriptstyle,xAxis=true,yAxis=true,Dx=1.,Dy=1.,ticksize=-2pt 0,subticks=2]{->}(0,0)(-2.36,-1.92)(6.64,3.58)
\psline[linewidth=2.pt](-2.,1.)(0.,-1.)
\psline[linewidth=2.pt](0.,-1.)(4.,1.)
\psline[linewidth=2.pt](4.,1.)(5.,3.)
\psline[linewidth=2.pt](5.,3.)(6.,1.)
\psline[linewidth=2.pt,linestyle=dashed,dash=2pt 2pt,linecolor=green](0.,0.5)(3.,0.5)
\psline[linewidth=2.pt,linestyle=dashed,dash=2pt 2pt,linecolor=green](3.,0.5)(3.,0.)
\psline[linewidth=2.pt,linestyle=dashed,dash=2pt 2pt,linecolor=red](-2.,1)(6,1)
\psline[linewidth=2.pt,linestyle=dashed,dash=2pt 2pt,linecolor=red](6,0)(6,1)
\psline[linewidth=2.pt,linestyle=dashed,dash=2pt 2pt,linecolor=red](4,0)(4,1)
\psline[linewidth=2.pt,linestyle=dashed,dash=2pt 2pt,linecolor=red](-2,0)(-2,1)


\rput[tl](-0.65,0.62){\green{$0,5$}}
\rput[tl](4.7,-0.3){\red{$5$}}
\rput[tl](5.7,-0.3){\red{$6$}}

\psdots[dotstyle=*,linecolor=red](4.,0.)
\psdots[dotstyle=*,linecolor=red](-2.,0.)
\psdots[dotstyle=*,linecolor=red](6.,0.)
\psdots[dotstyle=*,linecolor=blue](0.,-1.)
\psdots[dotstyle=*,linecolor=blue](5.,3.)
\end{pspicture*}
\end{center}

\begin{enumerate}
\item L’image de 3 par $f$ est $0,5$ (pointillés verts).
\item Les solutions de l’équation $f(x)=1$ sont $-2~;$ $4$ et $6$ (pointillés rouges).
\item Tableau de signe de $f~:$

\begin{center}
\begin{tikzpicture}[scale=0.7]
\tkzTabInit{$x$/1,$f\left(x\right)$/2}{$-2$,$-1$,$2$,$6$}
\tkzTabLine{,+,z,-,z,+}
\end{tikzpicture}
\end{center}

\item Le maximum de $f$ est $3,$ son minimum est $-1$ (points bleus).
\item Tableau de variations de $f~:$

\begin{center}
\begin{tikzpicture}[scale=1]
\tkzTabInit{$x$/1,$f\left(x\right)$/2}{$-2$,$0$,$5$,$6$}
\tkzTabVar{+/$1$,-/$-1$,+/$3$,-/$1$}
\end{tikzpicture}
\end{center}
\end{enumerate}

\end{exo}



\begin{exo}

La fonction $f$ est définie sur l'intervalle $\left[1;5\right]$ par $f\left(x\right)=2x+\frac{8}{x}-10.$

\begin{enumerate}
\item ~{}

\begin{center}

\begin{tabular}{|c|c|c|c|c|c|c|c|c|c|c|}\hline
$x$& $1$ &$1,5$ &$2$ &$2,5$ &$3$ &$3,5$ &$4$ &$4,5$ &$5$ \\ \hline 
$f(x)$&$0$ &$-1,67$ &$-2$   &$-1,8$  &$-1,33$  &$-0,71$  &$0$  &$0,78$  &$1,6$    \\ \hline
\end{tabular}

\end{center}

\medskip

Détail de deux calculs~:
\begin{align*}
f(1)&=2\times 1+\frac{8}{1}-10=2+8-10=0\\
f(4)&=2\times 4+\frac{8}{4}-10=8+2-10=0.
\end{align*}


\item Courbe représentative~:

\begin{center}
\psset{xunit=1.0cm,yunit=1.0cm,algebraic=true,dimen=middle,dotstyle=o,dotsize=5pt 0,linewidth=2.pt,arrowsize=3pt 2,arrowinset=0.25}
\begin{pspicture*}(-0.82,-2.58)(5.74,2.72)
\multips(0,-2)(0,1.0){6}{\psline[linestyle=dashed,linecap=1,dash=1.5pt 1.5pt,linewidth=0.4pt,linecolor=lightgray]{c-c}(-0.82,0)(5.74,0)}
\multips(0,0)(1.0,0){7}{\psline[linestyle=dashed,linecap=1,dash=1.5pt 1.5pt,linewidth=0.4pt,linecolor=lightgray]{c-c}(0,-2.58)(0,2.72)}
\psaxes[labelFontSize=\scriptstyle,xAxis=true,yAxis=true,Dx=1.,Dy=1.,ticksize=-2pt 0,subticks=2]{->}(0,0)(-0.82,-2.58)(5.74,2.72)
\psplot[linewidth=2.pt,linecolor=blue,plotpoints=200]{1}{5}{(x+4.0/x-5.0)*2.0}
\psline[linewidth=2.pt,linestyle=dashed,dash=2pt 2pt,linecolor=red](0.,-1.)(3.280776401789959,-1.000000005799203)
\psline[linewidth=2.pt,linestyle=dashed,dash=2pt 2pt,linecolor=red](1.2192235899791208,-0.9999999877700336)(1.2192235899791208,0.)
\psline[linewidth=2.pt,linestyle=dashed,dash=2pt 2pt,linecolor=red](3.280776401789959,-1.000000005799203)(3.280776401789959,0.)
\end{pspicture*}
\end{center}

\item Les antécédents de $-1$ par $f$ sont $1,25$ et $3,25$ environ (pointillés rouges).
%\item Les solutions de l'inéquation $f(x)\leq -1$ sont les nombres de l'intervalle $\left[1,25;3,25\right]$ environ.
\item Tableau de variations~:
\begin{center}
\begin{tikzpicture}[scale=1]
\tkzTabInit{$x$/1,$f\left(x\right)$/2}{$1$,$2$,$5$}
\tkzTabVar{+/$0$,-/$-2$,+/$1.6$}
\end{tikzpicture}
\end{center}
\item Tableau de signe~:
\begin{center}
\begin{tikzpicture}[scale=1]
\tkzTabInit{$x$/1,$f\left(x\right)$/1}{$1$,$4$,$5$}
\tkzTabLine{z,-,z,+}
\end{tikzpicture}
\end{center}
\end{enumerate}

\end{exo}


\begin{exo}

On suppose que le pourcentage de femmes fumant du tabac quotidiennement en fonction de l’âge $x$ (en années), depuis 15 ans jusqu’à 40 ans, est le nombre $f(x)$ donné par la formule suivante~:

\[f(x)=-0,05 x^2+3 x-10.\]

\begin{enumerate}
\item ~{}

\begin{center}

\begin{tabular}{|c|c|c|c|c|c|c|}\hline
$x$& $15$ &$20$ &$25$ &$30$ &$35$ &$40$ \\ \hline 
$f(x)$&$23,75$&$30$&$33,75$&$35$&$33,75$&$30$    \\ \hline
\end{tabular}

\end{center}

\medskip

Détail de deux calculs~:
\begin{align*}
f(15)&=-0,05 \times 15^2+3 \times 15-10=23,75\\
f(40)&=-0,05 \times 40^2+3 \times 40-10=30.
\end{align*}


\item ~{}


\begin{center}
\psset{xunit=0.2cm,yunit=0.2cm,algebraic=true,dimen=middle,dotstyle=o,dotsize=5pt 0,linewidth=2.pt,arrowsize=3pt 2,arrowinset=0.25}
\begin{pspicture*}(-14.139505008692659,-4.909112669613822)(43.94312748497488,39.624226480762424)
\multips(0,0)(0,5.0){9}{\psline[linestyle=dashed,linecap=1,dash=1.5pt 1.5pt,linewidth=0.4pt,linecolor=lightgray]{c-c}(0,0)(43.94312748497488,0)}
\multips(0,0)(5.0,0){12}{\psline[linestyle=dashed,linecap=1,dash=1.5pt 1.5pt,linewidth=0.4pt,linecolor=lightgray]{c-c}(0,0)(0,39.624226480762424)}
\psaxes[labelFontSize=\scriptstyle,xAxis=true,yAxis=true,Dx=5.,Dy=5.,ticksize=-2pt 0,subticks=2]{->}(0,0)(0.,0.)(43.94312748497488,39.624226480762424)
\rput{-180.}(30.,35.){\psplot[linewidth=2.pt,linecolor=blue]{-10.}{15.}{x^2/2/10.}}\psline[linewidth=2.pt,linestyle=dashed,dash=12pt 12pt,linecolor=green](30.,35.)(30.,0.)
\psline[linewidth=2.pt,linestyle=dashed,dash=12pt 12pt,linecolor=red](0.,30.)(40.,30.)
\psline[linewidth=2.pt,linestyle=dashed,dash=12pt 12pt,linecolor=red](20.,30.)(20.,0.)
\rput[tl](36.76997218558541,-3.2266362936234776){âge}
\rput[tl](-12.54547049771722,37.7313104989791){\% de fumeuses}
\psdots[dotstyle=*,linecolor=green](30.,35.)
\end{pspicture*}
\end{center}

\item Tableau de variations~:
\begin{center}
\begin{tikzpicture}[scale=1]
\tkzTabInit{$x$/1,$f\left(x\right)$/2}{$15$,$30$,$40$}
\tkzTabVar{-/$23.75$,+/$35$,-/$30$}
\end{tikzpicture}
\end{center}
\item Le pourcentage de fumeuses est maximal à 30 ans (pointillés verts).
\item C'est à partir de 20 ans que plus de 30~\% des femmes fument quotidiennement (pointillés rouges).

\end{enumerate}

\end{exo}


\begin{exo}

Sur route sèche, la distance d’arrêt en mètres d’un véhicule roulant à $x$ km/h est modélisée par la
fonction $f$ définie sur $\left[0;120\right]$ par \[f (x) = 0,005x(x + 56).\]



\begin{enumerate}
\item $f(100)=0,005\times 100(100+56)=78.$ Cela signifie que la distance d’arrêt d’un véhicule roulant à $100$~km/h est 78~m.
\item ~{}

\begin{center}

\begin{tabular}{|c|c|c|c|c|c|c|c|}\hline
$x$& $0$ &$20$ &$40$ &$60$ &$80$ &$100$&$120$ \\ \hline 
$f(x)$&$0$&$7,6$&$19,2$&$34,8$&$54,4$&$78$&$105,6$    \\ \hline
\end{tabular}

\end{center}


\item ~{}


\begin{center}
\newrgbcolor{ududff}{0.30196078431372547 0.30196078431372547 1.}
\psset{xunit=0.06cm,yunit=0.06cm,algebraic=true,dimen=middle,dotstyle=o,dotsize=5pt 0,linewidth=2.pt,arrowsize=3pt 2,arrowinset=0.25}
\begin{pspicture*}(-11.155294908729049,-8.567713058731323)(127.12746579799764,117.54409305395806)
\multips(0,0)(0,10.0){13}{\psline[linestyle=dashed,linecap=1,dash=1.5pt 1.5pt,linewidth=0.4pt,linecolor=lightgray]{c-c}(0,0)(127.12746579799764,0)}
\multips(0,0)(10.0,0){14}{\psline[linestyle=dashed,linecap=1,dash=1.5pt 1.5pt,linewidth=0.4pt,linecolor=lightgray]{c-c}(0,0)(0,117.54409305395806)}
\psaxes[labelFontSize=\scriptstyle,xAxis=true,yAxis=true,Dx=10.,Dy=10.,ticksize=-2pt 0,subticks=2]{->}(0,0)(0.,0.)(127.12746579799764,117.54409305395806)
\rput{0.}(-28.,-3.92){\psplot[linewidth=2.pt,linecolor=ududff]{28.}{148.}{x^2/2/100.}}
\rput[lt](101.23181772557692,15){\parbox{19.644866612142927 cm}{vitesse \\  (en km/h)}}
\rput[lt](2.828355050378144,111.32913751657706){\parbox{34.146429532698534 cm}{distance d'arrêt \\  (en m)}}

\end{pspicture*}
\end{center}
\item $f(90)=65,7$ et $f(80)=54,4$ donc le fait de baisser la vitesse sur les routes de 90 km/h à 80 km/h permet de diminuer la distance d'arrêt de \[65,7-54=11,7~\text{m}.\]

L'information de la sécurité routière est donc imprécise selon les données de l'exercice\footnote{Il est illusoire de penser que tous les conducteurs ont le même temps de réaction et toutes les voitures le même comportement en termes de freinage. Les formules concernant les distances d'arrêt que l'on doit apprendre par cœur au moment de passer le code de la route ne peuvent donc donner que des ordres de grandeur~; et la réponse attendue \og 13 mètres \fg~{} est en réalité très proche de la réponse \og 11,7 mètres \fg~{} obtenue avec notre calcul.} .
%Ainsi l'affirmation de la sécurité routière semble-t-elle fausse a première vue. Cependant, la distance d'arrêt, que l'on calcule à l'aide de la formule
%\[\text{distance d'arrêt}=\text{temps de réaction}+\text{distance de freinage}\] aura diminué davantage~: l'usage est de considérer un temps de réaction de 1~s. Dans ce laps de temps, on parcourt 25~m à 90~km/h, et environ 22,2~m à 80~km/h. Comme $25-22,2=2,8,$ la distance d'arrêt diminue en réalité de \[11,7+2,2=13,9~\text{m}.\]

%Finalement, l'affirmation est vraie~!\footnote{La question reste tout de même ambiguë, car l'expression \og au moment du freinage \fg~{} peut être interprétée aussi bien comme la distance de freinage que comme la distance d'arrêt. De plus, il est bien illusoire de croire que tous les conducteurs ont le même temps de réaction et toutes les voitures le même comportement en termes de freinage. Les formules qu'on doit apprendre par cœur au moment de passer le code de la route ne peuvent donner que des ordres de grandeur.}
\end{enumerate}
\end{exo}

\begin{exo}

Le taux d’anticorps (en g/l) présents dans le sang d’un nourrisson en fonction de l’âge (en mois), depuis la naissance jusqu’à l’âge de 12 mois, est donné par la formule suivante~:
\[f(x)=0,1 x^2-1,6 x+12.\]

\begin{enumerate}
\item On fait un tableau de valeurs pour $f$ sur $\left[0;12\right]$ avec un pas de $2~:$

\smallskip

\begin{tabularx}{\linewidth}{|c|*{8}{>{\centering \arraybackslash}X|}}\hline
$x$& $0$ &$2$ &$4$ &$6$ &$8$ &$10$&$12$ \\ \hline 
$f(x)$&$12$ &$9,2$ &$7,2$   & $6$ &$5,6$  &$6$&$7,2$ \\ \hline
\end{tabularx}

\smallskip


Détail de deux calculs~:
\begin{itemize}
\item[\textbullet] $f(0)=0,1 \times 0^2-1,6\times 0+12=12.$
\item[\textbullet] $f(12)=0,1 \times 12^2-1,6\times 12+12=7,2.$
\end{itemize}

\item ~{}


\begin{center}
\newrgbcolor{ududff}{0.30196078431372547 0.30196078431372547 1.}
\psset{xunit=0.75cm,yunit=0.75cm,algebraic=true,dimen=middle,dotstyle=o,dotsize=5pt 0,linewidth=2.pt,arrowsize=3pt 2,arrowinset=0.25}
\begin{pspicture*}(-0.9838150193248827,-1.3295079159466527)(12.99324231159889,12.810705726233044)
\multips(0,0)(0,1.0){13}{\psline[linestyle=dashed,linecap=1,dash=1.5pt 1.5pt,linewidth=0.4pt,linecolor=gray]{c-c}(0,0)(12.2,0)}
\multips(0,0)(1.0,0){13}{\psline[linestyle=dashed,linecap=1,dash=1.5pt 1.5pt,linewidth=0.4pt,linecolor=gray]{c-c}(0,0)(0,12.2)}
\psaxes[labelFontSize=\scriptstyle,xAxis=true,yAxis=true,Dx=1.,Dy=1.,ticksize=-2pt 0,subticks=2]{->}(0,0)(0.,0.)(12.99324231159889,12.810705726233044)
\rput[tl](9.811694242108459,0.7643314118376485){âge (en mois)}
\rput[tl](1.0556388713741114,12.756320289147737){\parbox{60.384832012684356 cm}{Taux d'anticorps  (en g/$\ell$)}}
\rput{0.}(8.,5.6){\psplot[linewidth=2.pt,linecolor=ududff]{-8.}{4.}{x^2/2/5.}}
\psline[linewidth=2.pt,linecolor=red]{->}(8.,6.5)(5.,6.5)
\psline[linewidth=2.pt,linecolor=red]{->}(8.,6.5)(11.,6.5)
\rput[tl](7.2283859805564,7.1274275508185125){\red{\Large 6 mois}}
\psline[linewidth=2.pt,linestyle=dashed,dash=3pt 3pt,linecolor=red](0.,6.5)(5.,6.5)
\rput[tl](-0.7662732709836567,6.69234405413606){\red{6,5}}
\psdots[dotstyle=*,linecolor=ududff](0.,12.)
\psdots[dotstyle=*,linecolor=ududff](2.,9.2)
\psdots[dotstyle=*,linecolor=ududff](4.,7.2)
\psdots[dotstyle=*,linecolor=ududff](6.,6.)
\psdots[dotstyle=*,linecolor=ududff](8.,5.6)
\psdots[dotstyle=*,linecolor=ududff](10.,6.)
\psdots[dotstyle=*,linecolor=ududff](12.,7.2)
\psdots[dotstyle=*,linecolor=ududff](11.,6.5)
\end{pspicture*}
\end{center}

\item Le taux d’anticorps à la naissance est de 12 g/$\ell.$
\item Tableau de variations~:

\begin{center}
\begin{tikzpicture}[scale=1]
\tkzTabInit{$x$/1,$f\left(x\right)$/2}{$0$,$8$,$12$}
\tkzTabVar{+/$12$,-/$5.6$,+/$7.2$}
\end{tikzpicture}
\end{center}

Le taux d’anticorps est minimal à l'âge de 8 mois.
\item D'après le graphique, le taux d’anticorps est inférieur à 6,5 g/$\ell$ pendant 6 mois (du 5\up{e} au 11\up{e} mois).
\end{enumerate}

\end{exo}


\section{Tableaux d'effectifs et probabilités conditionnelles}





\begin{exo}

\begin{enumerate}
\item

On traduit les données de l'énoncé par un tableau d'effectifs~:

\begin{center}
 \begin{tabular}{|m{2cm}|m{2cm}|m{2cm}|m{2cm}|}\hline
& VTT &\st{VTT}& Total \\ \hline 
Escalade& 3& 7&10 \\ \hline
\st{Escalade}& 13& 9& 22\\ \hline
Total& 16& 16& 32\\ \hline
\end{tabular}
\end{center}

\medskip

\textbf{Remarque~:} on sait qu'il y a autant  d'élèves qui pratiquent le VTT que d'élèves qui ne le pratiquent pas, donc 16 élèves le pratiquent et 16 ne le pratiquent pas.

\item \begin{enumerate}
\item Par lecture du tableau~: $P(E)=\frac{10}{32}$ et $P(V)=\frac{16}{32}.$
\item On s'intéresse aux trois événements $\overline{V},$ $V\cap E$ et $V\cup E~:$
\begin{itemize}
\item[\textbullet] $\overline{V}$~:~\og l'élève ne pratique pas le VTT \fg. \[P\left(\overline{V}\right)=\frac{16}{32}.\]
\item[\textbullet] $V\cap E$~:~\og l'élève pratique l'escalade \textbf{et} le VTT \fg.
\[P\left(V\cap E\right)=\frac{3}{32}.\]

\item[\textbullet] $V\cup E$~:~\og l'élève pratique l'escalade \textbf{ou} le VTT \fg.

Le calcul de $P\left(V\cup E\right)$ est moins évident et peut être mené de plusieurs façons différentes. Par exemple~:
\begin{itemize}
\item[$\blacktriangleright$] ajouter ceux qui font du VTT à ceux qui font de l'escalade, puis retrancher les élèves qui pratiquent les deux sports (sinon ils sont comptés deux fois)~: $16+10-3=23.$
\item[$\blacktriangleright$] ajouter ceux qui ne font que du VTT, ceux qui ne font que de l'escalade, et ceux qui pratiquent les deux sports~: $13+7+3=23.$
\item[$\blacktriangleright$] remarquer que $C$ est le contraire de $B$ et donc faire le calcul~: $32-9=23.$
\end{itemize}

\end{itemize}

Dans tous les cas on obtient $P(V\cup E)=\frac{23}{32}.$
\item Parmi les 16 élèves qui pratiquent le VTT, 3 pratiquent également l'escalade, donc la probabilité qu'un élève qui pratique le VTT pratique également l'escalade est $\frac{3}{16}.$ Avec la notation du cours~:
\[P_V(E)=\frac{3}{16}.\]
\end{enumerate}
\end{enumerate}

\end{exo}


\begin{exo}

\begin{enumerate}
\item On représente la situation par un tableau d'effectifs.

\begin{center}
 \begin{tabular}{|m{1,3cm}|m{1,3cm}|m{1,3cm}|m{1,3cm}|}\hline
& Petit format &Grand format& Total \\ \hline 
Couleur&7 & 18&25 \\ \hline
N\&B& 0 & 5& 5\\ \hline
Total& 7& 23& 30\\ \hline
\end{tabular}
\end{center}



\begin{enumerate}
\item \begin{itemize}
\item[\textbullet] $P(G)=\frac{23}{30}.$
\item[\textbullet] $\overline{C}$~:~\og la BD est en N\&B \fg.
\[P\left(\overline{C}\right)=\frac{5}{30}=\frac{1}{6}.\]
\end{itemize}
\item \begin{itemize}
\item[\textbullet] $C\cap G$~:~\og la BD est en couleur \textbf{et} en grand format \fg.
\[P(C\cap G)=\frac{18}{30}.\]
\item[\textbullet] $C\cup G$~:~\og la BD est en couleur \textbf{ou} en grand format \fg.

C'est le cas de toutes les BD (!), car il n'y en a aucune en N\&B et en petit format. Conclusion~: 
\[P(C\cup G)=\frac{30}{30}=1.\]
\end{itemize}
\item Pierre a choisi une BD en couleur. La probabilité qu'elle soit en grand format est \[P_C(G)=\frac{18}{25}.\]
\item \begin{itemize}
\item[\textbullet] La BD est en grand format. La probabilité qu'elle soit en couleur est
\[P_G(C)=\frac{18}{23}.\]
\item[\textbullet] La BD est en grand format. La probabilité qu'elle \textbf{ne} soit \textbf{pas} en couleur est
\[P_G\left(\overline{C}\right)=\frac{5}{23}.\]
\end{itemize}
\end{enumerate}
\end{enumerate}

\end{exo}

\begin{exo}

\begin{enumerate}
\item Pour compléter le tableau, on calcule~:
\begin{itemize}
\item[\textbullet] 5\% de \np{10000} valent 500~;
\item[\textbullet] $\np{10000}-500=\np{9500}$~;
\item[\textbullet] 4\% de \np{9500} valent 380~;
\item[\textbullet] on complète le reste du tableau avec des additions/soustractions.
\end{itemize}

\begin{center}
\begin{tabular}{|c|c|c|c|}
   \hline
	& Animaux sains & Animaux malades & Total  \\
	 \hline
  Test positif  &380 &500 &880  \\
	 \hline
  Test négatif & \np{9120}& 0 & \np{9120} \\
	 \hline
    Total &\np{9500} &500 & \np{10000}  \\
    \hline
\end{tabular}
\end{center}
\item \begin{enumerate}
\item \begin{itemize}
\item[\textbullet] $P(M)=\dfrac{500}{\np{10000}}=0,05$ (c'est le 5~\% déjà donné par l'énoncé).
\item[\textbullet]  $P\left(\overline{T}\right)=\dfrac{9120}{\np{10000}}=0,912.$
\end{itemize}
\item $\overline{M}\cap T$~:~\og l'animal n'est pas malade et son test est positif. \fg

\[P\left(\overline{M}\cap T\right)=\dfrac{380}{\np{10000}}=0,038.\]
\item 500 des 880 animaux ayant un test positif sont malades, donc la probabilité qu'un animal ayant un test positif soit malade est \[P_T(M)=\frac{500}{880}\approx 57\%.\]
\item Tous les animaux malades ont un test positif, donc la probabilité qu'un animal malade ait un test positif est \[P_M(T)=1.\]
\end{enumerate}
\end{enumerate}

\end{exo}



\begin{exo}

\begin{enumerate}
\item On traduit les données de l'énoncé par un tableau d'effectifs~:

\begin{center}
 \begin{tabular}{|m{3.5cm}|m{3cm}|m{3cm}|m{3cm}|}\hline
&Abonnés au soir &Pas abonnés au soir& Total \\ \hline 
Abonnés au matin& 50& 20&70 \\ \hline
Pas abonnés au matin& 50&160 &210 \\ \hline
Total& 100& 180& 280\\ \hline
\end{tabular}
\end{center}
\item 

\begin{enumerate}
\item $P(S)=\frac{100}{280}$ et $ P\left(\overline{M}\right)=\frac{210}{280}.$
\item $S\cap M$~:~\og le pensionnaire est abonné aux deux journaux. \fg

\[P\left(S\cap M\right)=\frac{50}{280}.\]
\item On choisit au hasard un pensionnaire abonné au \textit{Matin}. La probabilité qu'il soit aussi abonné au \textit{Soir} est \[P_M(S)=\frac{50}{70}.\]
\end{enumerate}
\end{enumerate}
\end{exo}


\section{Taux d'évolution, suites géométriques}









\begin{exo}

\begin{enumerate}
\item $100~\%+12,5~\%=112,5~\%=\dfrac{112,5}{100}=\np{1,125},$ donc pour augmenter un nombre de $12,5~\%,$ il faut le multiplier par $\np{1,125}.$ On complète donc le schéma~:

\begin{center}
$\xymatrix@R=0.5pc@C=4pc{
    *+[F]+\txt{120} \ar@/^0.5cm/[r]|{\red{\times ~\np{1,125}}} & 
    *+[F]+\txt{?~?~?}  \\
    \txt{\blue{inscrits décembre}}&
    \txt{\blue{inscrits janvier}}}$
    
    \end{center}
    
    \medskip
    
 $???=120\times 1,125=135,$ donc il y aura 135 inscrits en janvier. 


\item Pour connaître le pourcentage d'augmentation, on complète le schéma~:

\begin{center}
$\xymatrix@R=0.5pc@C=4pc{
    *+[F]+\txt{\np{5812}} \ar@/^0.5cm/[r]|{\red{\times ~?~?~?}} & 
    *+[F]+\txt{\np{6065}}  \\
    \txt{\blue{logements 2007}}&
    \txt{\blue{logements 2015}}}$
    
    \end{center}
    
    \medskip
$\red{???}\black =\np{6065}\div\np{5812}\approx 1,0435.$

Or $1,0435=\dfrac{104,35}{100}=104,35~\%=100~\%+4,35~\%,$ donc le nombre de logements a augmenté de 4,35~\% environ. Autrement dit, le taux d'évolution du nombre de logements est $+4,35~\%.$

\medskip

\textbf{N.B.} Vous n'êtes pas obligés d'écrire le calcul final~: vous pouvez passer directement de \og 1,0435 \fg~{} à la réponse \og augmentation de 4,35~\% \fg. Notez également qu'il faut prendre 4 chiffres après la virgule pour avoir une réponse finale arrondie à 0,01~\% près.
\item On rappelle que la TVA (taxe sur la valeur ajoutée) est une taxe sur les produits dont le montant revient à l’État.

Dire qu' \og il y a 20~\% de TVA \fg~{} signifie qu'un commerçant qui veut gagner 100~\euro~{} avec la vente d'un article doit le mettre en vente à 120~\euro~{} (il ajoute 20~\% de la valeur, donc il multiplie par 1,20).

Sur l'article vendu 120~\euro~{} en magasin, le commerçant gardera 100~\euro~{} et devra donner 20~\euro~{} à L’État.

Le prix TTC (toutes taxes comprises) est de 120~\euro, le prix HT (hors taxe) est de 100~\euro~{} et le montant de la TVA est de 20~\euro.

\medskip

Pour avoir le prix TTC dans notre exemple, il faut compléter le schéma~:



\begin{center}
$\xymatrix@R=0.5pc@C=4pc{
    *+[F]+\txt{80} \ar@/^0.5cm/[r]|{\red{\times ~\np{1,20}}} & 
    *+[F]+\txt{?~?~?}  \\
    \txt{\blue{prix HT}}&
    \txt{\blue{prix TTC}}}$
    
    \end{center}
    
    \medskip
    
 $???=80\times 1,20=96$, donc le montant TTC est de 96~\euro.
 
 \medskip
 
 \textbf{Remarque~:} Le montant de la TVA est $96-80=16$~\euro.
 
 Il ne faut pas confondre ce montant (16~\euro) et le taux de TVA (ici 20~\%). Il y a une ambiguïté lorsqu'on parle de \og TVA \fg~{} sans préciser s'il s'agit d'un montant ou d'un taux.
\end{enumerate}

\end{exo}

\begin{exo}

\begin{enumerate}
\item $100~\%-15~\%=85~\%=\dfrac{85}{100}=0,85,$ donc pour diminuer un nombre de $15~\%,$ il faut le multiplier par $0,85.$ On complète donc le schéma~:

\begin{center}
$\xymatrix@R=0.5pc@C=4pc{
    *+[F]+\txt{\np{12000}} \ar@/^0.5cm/[r]|{\red{\times ~0,85}} & 
    *+[F]+\txt{?~?~?}  \\
    \txt{\blue{prix de départ}}&
    \txt{\blue{prix après remise}}}$
    
    \end{center}
    
    \medskip
    
$???=\np{12000}\times 0,85=\np{10200},$ donc le prix après remise est de \np{10200}~\euro. 

\item Pour connaître le pourcentage de baisse, on complète le schéma~:

\begin{center}
$\xymatrix@R=0.5pc@C=4pc{
    *+[F]+\txt{240} \ar@/^0.5cm/[r]|{\red{\times ~?~?~?}} & 
    *+[F]+\txt{228}  \\
    \txt{\blue{émissions année 0}}&
    \txt{\blue{émissions année 1}}}$
    
    \end{center}
    
    \medskip
    
$\textcolor{red}{\text{?~?~?}}=228\div 240=0,95.$

Or $0,95=\frac{95}{100}=95~\%,$ donc les émissions ont baissé de 5~\%. Autrement dit, le taux d'évolution est $-5~\%.$

\medskip

\danger Ne pas oublier le \og $-$ \fg~{} dans $-5~\%.$



\item $100~\%-25~\%=75~\%=\frac{75}{100}=0,75,$ donc la baisse de 25~\% équivaut à une multiplication par 0,75 et on peut utiliser le schéma~:

\begin{center}
$\xymatrix@R=0.5pc@C=4pc{
    *+[F]+\txt{?~?~?} \ar@/^0.5cm/[r]|{\red{\times 0,75}} & 
    *+[F]+\txt{63}  \\
    \txt{\blue{prix initial}}&
    \txt{\blue{prix après remise}}}$
    
    \end{center}
    
    \medskip

$???=63\div 0,75=84,$ donc le prix initial était de 84~\euro.

\end{enumerate}

\end{exo}

\begin{exo}

Pour obtenir le prix TTC, on augmente le prix HT de 10~\%~; autrement dit, on le multiplie par 1,10. On peut ainsi compléter le schéma~:

\begin{center}
$\xymatrix@R=0.5pc@C=4pc{
    *+[F]+\txt{?~?~?} \ar@/^0.5cm/[r]|{\red{\times ~\np{1,10}}} & 
    *+[F]+\txt{4,95}  \\
    \txt{\blue{prix HT}}&
    \txt{\blue{prix TTC}}}$
    
    \end{center}
    
    \medskip
    
 Conclusion~: $???=4,95\div 1,10=4,50,$ donc le prix HT est de 4,50~\euro ~; et le montant de la TVA (la somme qui revient à l’État) est
 \[\text{montant TVA}=\text{montant TTC}-\text{montant HT}=4,95-4,50=0,45~\text{\euro}.\]


\medskip

\textbf{N.B.} Il est totalement faux de prendre 10~\% de $4,95$ pour obtenir le montant de la TVA. Il faut d'abord se ramener au prix HT, puis prendre 10~\% de ce prix HT.

\end{exo}



\begin{exo}

\begin{enumerate}
\item $v_0=4$ et $q=5.$

\begin{align*}
v_1&=4\times 5=20\\
v_2&=20\times 5=100\\
v_3&=100\times 5=500.
\end{align*}

\item $v_0=12$ et $q=0,5.$

\begin{align*}
v_1&=12\times 0,5=6\\
v_2&=6\times 0,5=3\\
v_3&=3\times 0,5=1,5.
\end{align*}

\item $v_1=5$ et $v_2=20.$

~{}\begin{center}
    $\xymatrix@R=0.5pc@C=3pc{
    *+[F]+{???} \ar@/^0.5cm/[r]|{\red{\times q}} & 
    *+[F]+{5} \ar@/^0.5cm/[r]|{\red{\times q}} & *+[F]+{20} \ar@/^0.5cm/[r]|{\red{\times q}} & 
    *+[F]+{???} \\
    \txt{\blue{$v_0$}}&
    \txt{\blue{$v_1$}}&\txt{\blue{$v_2$}}&
    \txt{\blue{$v_3$}}
    }$
    \end{center}
    
    D'après le schéma ci-dessus~:
    \[q=20\div 5=4.\]
    
    On en déduit~:
    
    \begin{align*}
v_3&=20\times  4=80\\
v_0&=5\div 4=1,25.
\end{align*}

\item $v_0=10$ et $v_1=6.$

~{}\begin{center}
    $\xymatrix@R=0.5pc@C=3pc{
    *+[F]+{10} \ar@/^0.5cm/[r]|{\red{\times q}} & 
    *+[F]+{6} \ar@/^0.5cm/[r]|{\red{\times q}} & *+[F]+{???} \\
    \txt{\blue{$v_0$}}&
    \txt{\blue{$v_1$}}&\txt{\blue{$v_2$}}
    }$
    \end{center}
    
    D'après le schéma ci-dessus~:
    \[q=6\div 10=0,6,\] puis
    \[v_2=6\times 0,6=3,6.\]
    
    
\end{enumerate}
\end{exo}

\begin{exo}

On place $\np{1000}$~\euro~{} sur un compte au taux d'intérêt annuel de 5~\%. On note $v_n$ la somme sur le compte après $n$ années -- on a donc en particulier $v_0=\np{1000}.$
\begin{enumerate}
\item Augmenter un nombre de 5~\% revient à le multiplier par 1,05. On peut donc compléter le schéma~:

\begin{center}
$\xymatrix@R=0.5pc@C=4pc{
    *+[F]+\txt{\np{1000}} \ar@/^0.5cm/[r]|{\red{\times 1,05}} & 
    *+[F]+\txt{\np{1050}} \ar@/^0.5cm/[r]|{\red{\times 1,05}} & *+[F]+\txt{\np{1102,50}} \\
    \txt{\blue{\text{Placement initial}}}&
    \txt{\blue{\text{Après 1 an}}}&\txt{\blue{\text{Après 2 ans}}}
    }$
    
    \end{center}
    
\medskip
    
Conclusion~:

\begin{itemize}
\item[\textbullet] après 1 an, on a \np{1050}~\euro~{} sur le compte~;
\item[\textbullet] après 2 ans, on a \np{1102,50}~\euro~{} sur le compte.
\end{itemize}

Avec les notations de l'énoncé~:
\begin{align*}
v_0&=\np{1000}\\
v_1&=\np{1050}\\
v_2&=\np{1102,50}.
\end{align*}

\item La suite $v$ est une suite géométrique de raison $q=1,05.$

\medskip

\textbf{Remarque~:} \`A la question \og quelle est la nature de la suite~? \fg, il faut répondre \og arithmétique \fg~{} ou \og géométrique \fg~{} selon le cas, mais aussi donner sa raison.
\item Dans la cellule C2, il faut rentrer la formule \[\text{=B2*1,05}\]


\item On détermine la somme sur le compte après 10 ans de deux façons différentes.

\begin{itemize}
\item[\textbullet] Avec le tableur~: on étire les cellules C1 et C2 vers la droite jusqu'à L1 et L2, qui correspondent à l'année $n=10.$ On obtient \np{1628,89}~\euro.

\medskip

\begin{center}
\begin{tabularx}{\linewidth}{|c|*{7}{>{\centering \arraybackslash}X|}}\hline
	&A   	&B		&C			&$\cdots$   		&K   	&L\\ \hline   
1   &$n$  &0			&1		&$\cdots$&9&10 \\ \hline   
2   &$v_n$   &1000	&1050			&$\cdots$ &1551,33&1628,89  \\ \hline    
\end{tabularx}
\end{center}


\medskip

\item[\textbullet] Par le calcul~: la somme sur le compte après 10 ans est
\[v_{10}=\np{1000}\underbrace{\times 1,05\times 1,05\times \cdots\times 1,05}_{\text{10 fois}}=\np{1000}\times 1,05^{10}\approx \np{1628,89}~\text{\euro}.\]
\end{itemize}
\end{enumerate}
\end{exo}

\begin{exo}

\begin{enumerate}
\item $100~\%-12~\%=88~\%=0,88,$ donc pour diminuer un nombre de 12~\%, il faut le multiplier par 0,88. 


\medskip

Pour déterminer le prix du livre en 2010 (donc après 5 ans), on peut utiliser un diagramme~:


\begin{center}
$\xymatrix@R=0.5pc@C=3pc{
    *+[F]+\txt{50} \ar@/^0.5cm/[r]|{\red{\times 0,88}} & 
    *+[F]+\txt{44} \ar@/^0.5cm/[r]|{\red{\times 0,88}} & *+[F]+\txt{38,72} \ar@/^0.5cm/[r]|{\red{\times 0,88}} & *+[F]+\txt{34,07}\ar@/^0.5cm/[r]|{\red{\times 0,88}} & *+[F]+\txt{29,98}\ar@/^0.5cm/[r]|{\red{\times 0,88}} & *+[F]+\txt{26,39}\\
    \txt{\blue{\text{Année 0}}}&
    \txt{\blue{\text{Année 1}}}&\txt{\blue{\text{Année 2}}}&\txt{\blue{\text{Année 3}}}&\txt{\blue{\text{Année 4}}}&\txt{\blue{\text{Année 5}}}
    }$
    
    \end{center}
    

\medskip

Le plus rapide cependant, c'est de faire directement le calcul~:


\[50\underbrace{\times 0,88\times 0,88\times \cdots\times 0,88}_{\text{5 fois}}=50\times 0,88^{5}\approx 26,39~\text{\euro}.\]

Conclusion~: le prix du livre en 2010 est 26,39~\euro.

\item On rentre les valeurs initiales $2005$ et $50$ dans la colonne B, puis on rentre les formules ci-dessous dans la colonne C, que l'on étire vers la droite.

\medskip

\begin{center}
\begin{tabularx}{\linewidth}{|c|*{5}{>{\centering \arraybackslash}X|}}\hline
	&A   	&B		&C			&$\cdots$   	\\ \hline   
1 & Année &$2005$  	&=B1+1				&$\cdots$ \\ \hline   
2 & Prix &$50$   		&=B2*0,88		&$\cdots$  \\ \hline    
\end{tabularx}
\end{center}

\medskip

On étire jusqu'à obtenir un prix inférieur à 10~\euro ~:

\medskip

\begin{center}
\begin{tabularx}{\linewidth}{|c|*{7}{>{\centering \arraybackslash}X|}}\hline
	&A   	&B&C			&$\cdots$&N&O   	\\ \hline   
1 & Année&2005&2006 &$\cdots$&2017&2018 \\ \hline   
2 &  Prix&50&44 &$\cdots$ &10,78&9,49 \\ \hline    
\end{tabularx}
\end{center}

\medskip

Conclusion~: c'est en 2018 que le prix du livre descendra pour la première fois en-dessous de 10~\euro.


\end{enumerate}
\end{exo}



\begin{exo}

\begin{enumerate}
\item  $100~\%-15~\%=85~\%=0,85,$ donc pour diminuer un nombre de 15~\%, il faut le multiplier par 0,85. 

Ainsi, dans le schéma ci-dessous, l'intensité lumineuse est-elle multipliée par 0,85 à chaque nouvelle plaque~:

\begin{center}
\newrgbcolor{ududff}{0.30196078431372547 0.30196078431372547 1.}
\newrgbcolor{zzttqq}{0.2 0.2 0.6}
\psset{xunit=0.75cm,yunit=0.75cm,algebraic=true,dimen=middle,dotstyle=o,dotsize=5pt 0,linewidth=2.pt,arrowsize=3pt 2,arrowinset=0.25}
\begin{pspicture*}(-1.2095636134454244,-0.8024760882800114)(15,7)
\pspolygon[linewidth=2.pt,linecolor=zzttqq,fillcolor=zzttqq!20!white,fillstyle=solid,opacity=0.1](0.,4.)(5.,4.)(7.,5.)(2.,5.)
\pspolygon[linewidth=2.pt,linecolor=zzttqq,fillcolor=zzttqq!20!white,fillstyle=solid,opacity=0.1](0.,2.)(5.,2.)(7.,3.)(2.,3.)
\pspolygon[linewidth=2.pt,linecolor=zzttqq,fillcolor=zzttqq!20!white,fillstyle=solid,opacity=0.1](0.,0.)(5.,0.)(7.,1.)(2.,1.)
\psline[linewidth=2.pt]{->}(9.006307576513136,3.604370307388384)(6.96981037851486,3.604370307388384)
\psline[linewidth=2.pt]{->}(8.989614976529543,5.607482305419475)(6.986502978498453,5.607482305419475)
\psline[linewidth=2.pt]{->}(8.989614976529543,1.6012583093572934)(6.986502978498453,1.584565709373701)
\psline[linewidth=2.pt]{->}(9.006307576513136,-0.3851610886902048)(6.986502978498453,-0.4018536886737972)
\rput[tl](-0.95,5.15){\textcolor{zzttqq}{1\up{re} plaque}}
\rput[tl](-0.95,3.15){\textcolor{zzttqq}{2\up{e} plaque}}
\rput[tl](-0.95,1.15){\textcolor{zzttqq}{3\up{e} plaque}}
\rput[tl](9.256696576267021,5.8){$v_0=12~\text{lm}$}
\rput[tl](9.25,3.8){$v_1=12\times 0,85=10,2~\text{lm}$}
\rput[tl](9.25,1.8){$v_2=10,2\times 0,85=8,67~\text{lm}$}
\rput[tl](9.25,-0.2){$v_3=8,67\times 0,85=7,3695~\text{lm}$}
\psline[linewidth=2.pt,linecolor=yellow](4.5,7.)(4.5,5.5)
\psline[linewidth=2.pt,linecolor=yellow](4.,7.)(3.5,5.5)
\psline[linewidth=2.pt,linecolor=yellow](5.,7.)(5.5,5.5)
\rput[tl](1.5,6.328984750608794){\yellow{lumière}}
\end{pspicture*}
\end{center}

\medskip

\textbf{Remarque~:} Le lumen est une unité de mesure du flux lumineux, utilisée notamment pour indiquer la capacité d'éclairement des ampoules électriques.
\item La suite $v$ est une suite géométrique de raison $q=0,85.$
\item L'intensité lumineuse est divisée par 10 lorsqu'on descend en dessous de $12\div 10=1,2$~lm. Pour savoir le nombre minimal de plaques à superposer pour qu'il en soit ainsi, on rentre les valeurs initiales $0$ et $12$ dans la colonne B, puis on rentre les formules ci-dessous dans la colonne C, que l'on étire vers la droite jusqu'à obtenir une intensité lumineuse inférieure à 1,2.

\medskip

\begin{center}
\begin{tabularx}{\linewidth}{|c|*{7}{>{\centering \arraybackslash}X|}}\hline
	&A   	&B		&C			&$\cdots$ &P&Q  	\\ \hline   
1 & Nb de plaques &$0$  	&=B1+1				&$\cdots$&14&15 \\ \hline   
2 & Intensité (lm) &$12$   		&=B2*0,85		&$\cdots$&$1,23$&$1,05$  \\ \hline    
\end{tabularx}
\end{center}

\medskip

Conclusion~: il faut superposer au moins 15 plaques pour que l'intensité lumineuse soit divisée par 10.
\end{enumerate}

\end{exo}



\section{Dérivation et variations des fonctions du 2\up{nd} degré}





\begin{exo}

La fonction $f$ est définie sur l'intervalle $\left[-1;4\right]$ par

\[ f(x) = 0,5x^2-2x+1.\]

\begin{enumerate}
\item $f'(x)=0,5\times 2x-2\times 1+0=x-2.$
\item La dérivée est du premier degré, donc il faut résoudre une équation, puis regarder le signe de $a~:$
\begin{align*}x-2&=0\\
 x-\cancel{2}+\cancel{2}&=0+2\\
 x&=2.
 \end{align*}

$a=1$ (puisque $x-2$ signifie $\textcolor{red}{1}x-2$), $a$ est $\bigoplus$ donc le signe est de la forme \fbox{$-~\upphi~+$}

\medskip


On en déduit le tableau de signe de $f'$ et le tableau de variations de $f~:$


\medskip

\begin{center}
\begin{tikzpicture}[scale=0.8]
\tkzTabInit{$x$/1,$f'(x)$/1,$f(x)$/2}{$-1$,$2$,$4$}
\tkzTabLine{,-,z,+,}
\tkzTabVar{+/$3.5$,-/$-1$,+/$1$}
\end{tikzpicture}
\end{center}

\medskip

\textbf{Remarque~:} Pour les trois valeurs aux extrémités des flèches, on utilise le tableau de valeurs de la question suivante.
\item 

Rappelons la méthode pour obtenir le tableau de valeurs avec une calculatrice~:

\setlength{\columnseprule}{1pt}
\begin{multicols}{3}

\begin{center}
\textbf{Calculatrices collège}
\end{center}

\begin{itemize}
\item[\textbullet] \fbox{MODE} ou \fbox{MENU}
\item[\textbullet] 4 : TABLE ou 4 : Tableau
\item[\textbullet] f(X)=$0,5\text{X}^2-2\text{X}+1$ \fbox{EXE}

(si on demande g(X)=, ne rien rentrer)
\item[\textbullet] Début? $-1$ \fbox{EXE}
\item[\textbullet] Fin? $4$ \fbox{EXE}
\item[\textbullet] Pas? $1$ \fbox{EXE}
\end{itemize}





\columnbreak

\begin{center}
\textbf{TI graphiques}
\end{center}

X s'obtient avec la touche \fbox{$x,t,\theta,n$}
\begin{itemize}
\item[\textbullet] \fbox{$f(x)$}
\item[\textbullet] $\text{Y}_1=0.5\text{X}^2-2\text{X}+1$ \fbox{EXE}
\item[\textbullet] \fbox{2nde} \fbox{déf table}
\item[\textbullet] DébTable=$(-)1$ \fbox{EXE}
\item[\textbullet] PasTable=$1$ \fbox{EXE}

ou

\tiny{$\Delta$}\normalsize Tbl=$1$ \fbox{EXE}
\item[\textbullet] \fbox{2nde} \fbox{table}
\end{itemize}


\columnbreak

\begin{center}
\textbf{CASIO graphiques}
\end{center}

X s'obtient avec la touche \fbox{$\text{X},\theta,\text{T}$}
\begin{itemize}
\item[\textbullet] \fbox{MENU} puis choisir TABLE \fbox{EXE}
\item[\textbullet] $\text{Y}_1:0.5\text{X}^2-2\text{X}+1$ \fbox{EXE}
\item[\textbullet] \fbox{F5} (on choisit donc SET)
\item[\textbullet] Start:$(-)1$ \fbox{EXE}
\item[\textbullet] End:$4$ \fbox{EXE}
\item[\textbullet] Step:$1$ \fbox{EXE}
\item[\textbullet]\fbox{EXIT}
\item[\textbullet] \fbox{F6} (on choisit donc TABLE)
\end{itemize}


\end{multicols}

\medskip

\textbf{Remarque~:} Les calculatrices \textbf{NUMWORKS} permettent également de faire ces calculs et sont d'un usage beaucoup plus intuitif.

\medskip

On obtient finalement le tableau~:

\begin{center}

\begin{tabular}{|c|c|c|c|c|c|c|}\hline
$x$& $-1$ &$0$ &$1$ &$2$ &$3$ &$4$ \\ \hline 
$f(x)$&$3,5$ &$1$ &$-0,5$   & $-1$ &$-0,5$  &$1$  \\ \hline
\end{tabular}

\end{center}


\item ~{}


\begin{center}
\newrgbcolor{ududff}{0.30196078431372547 0.30196078431372547 1.}
\psset{xunit=1.0cm,yunit=1.0cm,algebraic=true,dimen=middle,dotstyle=o,dotsize=5pt 0,linewidth=2.pt,arrowsize=3pt 2,arrowinset=0.25}
\begin{pspicture*}(-2.04,-1.68)(5.6,4.64)
\multips(0,-1)(0,1.0){7}{\psline[linestyle=dashed,linecap=1,dash=1.5pt 1.5pt,linewidth=0.4pt,linecolor=lightgray]{c-c}(-2.04,0)(5.6,0)}
\multips(-2,0)(1.0,0){8}{\psline[linestyle=dashed,linecap=1,dash=1.5pt 1.5pt,linewidth=0.4pt,linecolor=lightgray]{c-c}(0,-1.68)(0,4.64)}
\psaxes[labelFontSize=\scriptstyle,xAxis=true,yAxis=true,Dx=1.,Dy=1.,ticksize=-2pt 0,subticks=2]{->}(0,0)(-2.04,-1.68)(5.6,4.64)
\rput{0.}(2.,-1.){\psplot[linewidth=2.pt,linecolor=ududff]{-3.}{2.}{x^2/2/1.}}
\psdots[dotstyle=*,linecolor=ududff](-1.,3.5)
\psdots[dotstyle=*,linecolor=ududff](0.,1.)
\psdots[dotstyle=*,linecolor=ududff](1.,-0.5)
\psdots[dotstyle=*,linecolor=ududff](2.,-1.)
\psdots[dotstyle=*,linecolor=ududff](3.,-0.5)
\psdots[dotstyle=*,linecolor=ududff](4.,1.)
\end{pspicture*}
\end{center}

\medskip

\textbf{Remarque~:} La courbe représentative est une parabole, puisque la fonction $f$ est du 2\up{nd} degré (de la forme $f(x)=ax^2+bx+c$). Ce sera le cas dans tous les exercices du chapitre.

\end{enumerate}

\end{exo}


\begin{exo}

La fonction $g$ est définie sur l'intervalle $\left[0;5\right]$ par

\[ g(x) = -x^2+5x-2.\]

\begin{enumerate}
\item $g'(x)=-2x+5\times 1-0=-2x+5.$
\item On résout l'équation~:
\begin{align*}-2x+5&=0\\
 -2x+\cancel{5}-\cancel{5}&=0-5\\
 \frac{\cancel{-2}x}{\cancel{-2}}&=\frac{-5}{-2}\\
 x&=2,5.
 \end{align*}

$a=-2,$ $a$ est \Large $\ominus$\normalsize donc le signe est de la forme \fbox{$+~\upphi~-$}

\medskip


On en déduit le tableau de signe de $g'$ et le tableau de variations de $g~:$


\medskip

\begin{center}
\begin{tikzpicture}[scale=0.8]
\tkzTabInit{$x$/1,$g'(x)$/1,$g(x)$/2}{$0$,$2.5$,$5$}
\tkzTabLine{,+,z,-,}
\tkzTabVar{-/$-2$,+/$4.25$,-/$-2$}
\end{tikzpicture}
\end{center}

\medskip

\textbf{Remarque~:} Pour compléter les valeurs aux extrémités des flèches, on fait un tableau de valeurs sur $\left[0;5\right]$ avec un pas de $0,5.$ Ce tableau a deux utilités~:

\begin{itemize}
\item[\textbullet] en prenant les valeurs $x=0,$ $x=2,5$ et $x=5,$ on peut compléter notre tableau de variations~;
\item[\textbullet] en prenant les valeurs $x=0,$ $x=1,$ $x=2,$ $x=3,$ $x=4$ et $x=5,$ on obtient le tableau de valeurs de la question suivante.
\end{itemize}

Les autres valeurs ($x=0,5,$ $x=1,5,$ $x=3,5$ et $x=4,5$) ne nous servent à rien.
\item ~{}

\begin{center}

\begin{tabular}{|c|c|c|c|c|c|c|}\hline
$x$ &$0$ &$1$ &$2$ &$3$ &$4$&$5$\\ \hline 
$g(x)$&$-2$ &$2$ &$4$   & $4$ &$2$  &$-2$  \\ \hline
\end{tabular}

\end{center}
\item Pour construire la courbe, on place les points du tableau de valeurs, mais aussi le point obtenu pour $x=2,5$ (maximum du tableau de variations).


\begin{center}
\newrgbcolor{ududff}{0.30196078431372547 0.30196078431372547 1.}
\psset{xunit=1.0cm,yunit=1.0cm,algebraic=true,dimen=middle,dotstyle=o,dotsize=5pt 0,linewidth=2.pt,arrowsize=3pt 2,arrowinset=0.25}
\begin{pspicture*}(-1.1,-2.76)(6.7,4.68)
\multips(0,-2)(0,1.0){8}{\psline[linestyle=dashed,linecap=1,dash=1.5pt 1.5pt,linewidth=0.4pt,linecolor=lightgray]{c-c}(-1.1,0)(6.7,0)}
\multips(-1,0)(1.0,0){8}{\psline[linestyle=dashed,linecap=1,dash=1.5pt 1.5pt,linewidth=0.4pt,linecolor=lightgray]{c-c}(0,-2.76)(0,4.68)}
\psaxes[labelFontSize=\scriptstyle,xAxis=true,yAxis=true,Dx=1.,Dy=1.,ticksize=-2pt 0,subticks=2]{->}(0,0)(-1.1,-2.76)(6.7,4.68)
\rput{-180.}(2.5,4.25){\psplot[linewidth=2.pt,linecolor=ududff]{-2.5}{2.5}{x^2/2/0.5}}
\psdots[dotstyle=*,linecolor=ududff](0.,-2.)
\psdots[dotstyle=*,linecolor=ududff](2.,4.)
\psdots[dotstyle=*,linecolor=ududff](3.,4.)
\psdots[dotstyle=*,linecolor=ududff](1.,2.)
\psdots[dotstyle=*,linecolor=ududff](2.5,4.25)
\psdots[dotstyle=*,linecolor=ududff](4.,2.)
\psdots[dotstyle=*,linecolor=ududff](5.,-2.)
\end{pspicture*}
\end{center}
\end{enumerate}

\end{exo}


\begin{exo}

On modélise le taux d’anticorps (en g/l) présents dans le sang d’un nourrisson en fonction de l’âge $t$ (en mois), depuis la naissance jusqu’à l’âge de 12 mois, par la formule suivante~:
\[f(t)=0,1 t^2-1,6 t+12.\]

\begin{enumerate}
\item $f'(t)=0,1\times 2t-1,6\times 1+0=0,2t-1,6.$

\medskip

\textbf{Remarque~:} On traite les \og $t$ \fg~{} comme s'il s'agissait de \og $x$ \fg.
\item On résout l'équation~:
\begin{align*}0,2t-1,6&=0\\
 0,2t-\cancel{1,6}+\cancel{1,6}&=0+1,6\\
 \frac{\cancel{0,2}t}{\cancel{0,2}}&=\frac{1,6}{0,2}\\
 t&=8.
 \end{align*}

$a=0,2,$ $a$ est $\bigoplus$ donc le signe est de la forme \fbox{$-~\upphi~+$}

\medskip


On en déduit le tableau de signe de $f'$ et le tableau de variations de $f~:$


\medskip

\begin{center}
\begin{tikzpicture}[scale=0.8]
\tkzTabInit{$t$/1,$f'(t)$/1,$f(t)$/2}{$0$,$8$,$12$}
\tkzTabLine{,-,z,+,}
\tkzTabVar{+/$12$,-/$5.6$,+/$7.2$}
\end{tikzpicture}
\end{center}

\medskip

\textbf{Remarque~:} Pour compléter les valeurs aux extrémités des flèches, on fait un tableau de valeurs sur $\left[0;12\right]$ avec un pas de $1,$ en prenant uniquement les valeurs qui nous intéressent.
\item D'après le tableau de variations, le taux d'anticorps est minimal à l'âge de 8 mois.
\end{enumerate}
\end{exo}


\begin{exo}

On suppose que le pourcentage de femmes fumant du tabac quotidiennement en fonction de l’âge $x$ (en années), depuis 15 ans jusqu’à 40 ans, est le nombre $f(x)$ donné par la formule suivante~:

\[f(x)=-0,05 x^2+3 x-10.\]

\begin{enumerate}
\item $f'(x)=-0,05\times 2x+3\times 1-0=-0,1x+3.$
\item On résout l'équation~:
\begin{align*}-0,1x+3&=0\\
 -0,1x+\cancel{3}-\cancel{3}&=0-3\\
 \frac{\cancel{-0,1}x}{\cancel{-0,1}}&=\frac{-3}{-0,1}\\
 x&=30.
 \end{align*}

$a=-0,1,$ $a$ est \Large $\ominus$\normalsize donc le signe est de la forme \fbox{$+~\upphi~-$}

\medskip


On en déduit le tableau de signe de $f'$ et le tableau de variations de $f~:$


\medskip

\begin{center}
\begin{tikzpicture}[scale=0.8]
\tkzTabInit{$x$/1,$f'(x)$/1,$f(x)$/2}{$15$,$30$,$40$}
\tkzTabLine{,+,z,-,}
\tkzTabVar{-/$23.75$,+/$35$,-/$30$}
\end{tikzpicture}
\end{center}

\medskip

\textbf{Remarque~:} Pour compléter les valeurs aux extrémités des flèches, on fait un tableau de valeurs sur $\left[15;40\right]$ avec un pas de $5,$ en prenant uniquement les valeurs qui nous intéressent.
\item D'après le tableau de variations, c'est chez les femmes de 30 ans que le pourcentage de fumeuses est le plus important.
\end{enumerate}
\end{exo}


\begin{exo}

On dispose d’une clôture de 100 mètres de long pour délimiter un terrain rectangulaire le long d’une rivière (la clôture est en pointillés –- on ne met pas de clôture le long de la rivière). On note $x$ et $x’$ les longueurs des côtés du terrain. 

\begin{center}
\newrgbcolor{wqwqwq}{0.3764705882352941 0.3764705882352941 0.3764705882352941}
\psset{xunit=1.0cm,yunit=1.0cm,algebraic=true,dimen=middle,dotstyle=o,dotsize=3pt 0,linewidth=0.8pt,arrowsize=3pt 2,arrowinset=0.25}
\begin{pspicture*}(0.,0.)(7,5)
\pspolygon[linecolor=blue,fillcolor=blue!30!white,fillstyle=solid,opacity=0.2](1.,4.)(6.,4.)(6.,3.)(1.,3.)
\pspolygon[linewidth=0.pt,linecolor=green,fillcolor=green!30!white,fillstyle=solid,opacity=0.1](2,3)(2,1)(5,1)(5,3)
\psline[linewidth=3pt,linestyle=dashed,dash=5pt 5pt](2.,3.)(2.,1.)
\psline[linewidth=3pt,linestyle=dashed,dash=5pt 5pt](2.,1.)(5.,1.)
\psline[linewidth=3pt,linestyle=dashed,dash=5pt 5pt](5.,1.)(5.,3.)
\rput[tl](1.54,2.0){$x$}
\rput[tl](5.14,2.0){$x$}
\rput[tl](3.36,0.8){$x'$}
%\psline[linecolor=gray](2.,3.)(2.,1.)
%\psline[linecolor=gray](2.,1.)(5.,1.)
%\psline[linecolor=gray](5.,1.)(5.,3.)
%\psline[linecolor=gray](5.,3.)(2.,3.)
%\rput[tl](3.0,2.0){terrain}
\end{pspicture*}
\end{center}


On voudrait délimiter le terrain le plus grand possible.

\begin{enumerate}
\item\begin{enumerate}[(a)]
\item $x$ est une longueur, donc $x\geq 0.$ Par ailleurs, la longueur $x$ apparaît deux fois sur la figure, donc $x$ ne peut pas dépasser 50 (car $2\times 50=100$).

Conclusion~: on a l'encadrement \[0\leq x\leq 50.\]
\item Le périmètre, 100~m, s'obtient en faisant le calcul
\[x+x+x',\]
donc \[2x+x'=100~;\]
et donc
\[x'=100-2x.\]
\item L’aire du terrain est \begin{align*}x\times x'&=x\times \left(100-2x\right)&& \text{(car } x'=100-2x)\\
&=x\times 100+x\times (-2x)&&\text{(on développe)}\\
&=100x-2x^2.&&\end{align*}
\end{enumerate}
\item On définit à présent la fonction $f$  sur $\left[0;50\right]$ par 
\[f(x)=100x-2x^2.\]
Autrement dit, $f(x)$ donne l'aire du terrain pour une valeur donnée de $x.$
\begin{enumerate}[(a)] \item $f'(x)=100\times 1-2\times 2x=100-4x.$
\item On résout l'équation~:
\begin{align*}100-4x&=0\\
 \cancel{100}-4x-\cancel{100}&=0-100\\
 \frac{\cancel{-4}x}{\cancel{-4}}&=\frac{-100}{-4}\\
 x&=25.
 \end{align*}

$a=-4$ (\danger on prend le coefficient devant $x$), $a$ est \Large $\ominus$\normalsize donc le signe est de la forme \fbox{$+~\upphi~-$}

\medskip


On en déduit le tableau de signe de $f'$ et le tableau de variations de $f~:$


\medskip

\begin{center}
\begin{tikzpicture}[scale=0.8]
\tkzTabInit{$x$/1,$f'(x)$/1,$f(x)$/2}{$0$,$25$,$50$}
\tkzTabLine{,+,z,-,}
\tkzTabVar{-/$0$,+/$\np{1250}$,-/$0$}
\end{tikzpicture}
\end{center}

\medskip

\textbf{Remarque~:} Pour compléter les valeurs aux extrémités des flèches, on fait un tableau de valeurs sur $\left[0;50\right]$ avec un pas de $25.$
\item D'après le tableau de variations, l'aire du terrain est maximale lorsque $x=25.$ 

\medskip

\textbf{Remarque~:} Notons que, dans ce cas, $x'=100-2\times 25=50.$ Le terrain d'aire maximale a donc un côté adjacent à la rivière de 25~m de long et un côté parallèle à la rivière de 50~m de long.
\end{enumerate}
\end{enumerate}

\end{exo}

\begin{exo}


Pour vérifier le fonctionnement de la régulation de la glycémie chez un individu, on lui injecte une quantité importante de glucose~: on mesure ensuite la concentration d’insuline plasmatique pendant 70 minutes.

La concentration d’insuline plasmatique (unité non précisée) en fonction du temps $x$ (exprimé en minutes), est donnée par la fonction $f$ définie pour $x\in \left[0;70\right]$ par \[f(x)=-0,1x^2+10x+35.\]

\begin{enumerate}
\item ~{}

\begin{center}

\begin{tabular}{|c|c|c|c|c|c|c|c|c|}\hline
$x$ &$0$ &$10$ &$20$ &$30$ &$40$&$50$&$60$&$70$\\ \hline 
$f(x)$&$35$ &$125$ &$195$   & $245$ &$275$  &$285$ &$275$&$245$ \\ \hline
\end{tabular}

\end{center}
\item ~{}


\begin{center}
\psset{xunit=0.1cm,yunit=0.01cm,algebraic=true,dimen=middle,dotstyle=o,dotsize=5pt 0,linewidth=2.pt,arrowsize=3pt 2,arrowinset=0.25}
\begin{pspicture*}(-7.003370746849452,-63.16883844530834)(94,348.23518900913695)
\multips(0,0)(0,100.0){5}{\psline[linestyle=dashed,linecap=1,dash=1.5pt 1.5pt,linewidth=0.4pt,linecolor=lightgray]{c-c}(0,0)(94,0)}
\multips(0,0)(10.0,0){9}{\psline[linestyle=dashed,linecap=1,dash=1.5pt 1.5pt,linewidth=0.4pt,linecolor=lightgray]{c-c}(0,0)(0,348.23518900913695)}
\psaxes[labelFontSize=\scriptstyle,xAxis=true,yAxis=true,Dx=10.,Dy=100.,ticksize=-2pt 0,subticks=2]{->}(0,0)(0.,0.)(94,348.23518900913695)
\rput{-180.}(50.,285.){\psplot[linewidth=2.pt,linecolor=blue]{-20.}{50.}{x^2/2/5.}}
\psline[linewidth=2.pt,linestyle=dashed,dash=5pt 5pt,linecolor=red](0.,250.)(68.7082869338697,250.)
\psline[linewidth=2.pt,linestyle=dashed,dash=5pt 5pt,linecolor=red](31.291713066130296,250.)(31.291713066130296,0.)
\psline[linewidth=2.pt,linestyle=dashed,dash=5pt 5pt,linecolor=red](68.7082869338697,250.)(68.7082869338697,0.)
\rput[tl](2,340){Concentration}
\rput[lt](81.53046519648794,85){\parbox{11.126868368109374 cm}{temps \\(en min)}}
\end{pspicture*}
\end{center}

\item ~{}

\begin{center}
\begin{tikzpicture}[scale=0.8]
\tkzTabInit{$x$/1,$f(x)$/2}{$0$,$50$,$70$}
\tkzTabVar{-/$35$,+/$285$,-/$245$}
\end{tikzpicture}
\end{center}
\item La concentration d'insuline est maximale au bout de 50~min.
\item La concentration d'insuline est supérieure à $250$ de 32 à 68~min environ (voir pointillés rouges), donc pendant 36~min.
\item On reprend les questions 3 et 4 à l'aide de la dérivation~:

\[f'(x)=-0,1\times 2x+10\times 1+0=-0,2x+10.\]

On résout l'équation~:

\begin{align*}-0,2x+10&=0\\
 -0,2x+\cancel{10}-\cancel{10}&=0-10\\
 \frac{\cancel{-0,2}x}{\cancel{-0,2}}&=\frac{-10}{-0,2}\\
 x&=50.
 \end{align*}

$a=-0,2,$ $a$ est \Large $\ominus$\normalsize donc le signe est de la forme \fbox{$+~\upphi~-$}

\medskip


On en déduit le tableau de signe de $f'$ et le tableau de variations de $f~:$


\medskip

\begin{center}
\begin{tikzpicture}[scale=0.8]
\tkzTabInit{$x$/1,$f'(x)$/1,$f(x)$/2}{$0$,$50$,$70$}
\tkzTabLine{,+,z,-,}
\tkzTabVar{-/$35$,+/$285$,-/$245$}
\end{tikzpicture}
\end{center}

\medskip

La concentration est bien maximale après 50~min.

\end{enumerate}

\end{exo}


\section{Arbres de probabilités}




\begin{exo}


\begin{enumerate}
\item On part de 100 coyotes. Comme 60~\% des coyotes sont malades, on en a 60 malades et 40 en bonne santé. Ensuite on calcule~:

\begin{itemize}
\item[\textbullet] 95~\% de 60 valent $\frac{60\times 95}{100}=57~;$
\item[\textbullet] 90~\% de 40 valent $\frac{40\times 90}{100}=36.$
\end{itemize}

On peut alors compléter le tableau~:



\begin{center}
 \begin{tabular}{|m{2cm}|m{2cm}|m{2cm}|m{2cm}|}\hline
& Test $\bigoplus$ &Test \Large$\ominus$\normalsize & Total \\ \hline 
Malades& 57& 3&60 \\ \hline
Sains& 4& 36& 40\\ \hline
Total& 61& 39& 100\\ \hline
\end{tabular}
\end{center}

\medskip

\item $P(M\cap T)=\frac{57}{100}=0,57$ et $P(T)=\frac{61}{100}=0,61.$
\item On représente la situation par un arbre pondéré~:

\begin{center}
\pstree[treemode=R,treesep=1,levelsep=3]{\TR{}}%
{
\pstree{\Tr{$M$}\taput{$0,60$}}
	{
	\Tr{$T$}\taput{$0,95$} 
	\Tr{$\overline{T}$}\tbput{$0,05$}
		}	
\pstree{\Tr{$\overline{M}$}\tbput{$0,40$}}
	{
	\Tr{$T$}\taput{$0,10$} 
	\Tr{$\overline{T}$}\tbput{$0,90$}
		}
}
\end{center}

On en déduit~:

\begin{itemize}
\item[\textbullet] $P(M\cap T)=0,60\times 0,95=0,57~;$
\item[\textbullet] D'après la formule des probabilités totales, la probabilité que le test soit positif est~:
\begin{align*}P(T)&=P\left(M\cap T\right)+P\left(\overline{M}\cap T\right)\\
&=0,60\times 0,95+0,40\times 0,10=0,61.\end{align*}
\end{itemize}

\medskip

On retrouve bien les résultats de la question 2.

\end{enumerate}


\end{exo}

\begin{exo}




\begin{enumerate} 
\item On représente la situation par un arbre pondéré~:

\begin{center}
\pstree[treemode=R,treesep=1,levelsep=3]{\TR{}}%
{
\pstree{\Tr{$A$}\taput{$0,70$}}
	{
	\Tr{$D$}\taput{$0,02$} 
	\Tr{$\overline{D}$}\tbput{$0,98$}
		}	
\pstree{\Tr{$\overline{A}$}\tbput{$0,30$}}
	{
	\Tr{$D$}\taput{$0,03$} 
	\Tr{$\overline{D}$}\tbput{$0,97$}
		}
}
\end{center}




\item

\begin{itemize}
\item[\textbullet] $P\left(A\cap \overline{D}\right)=0,70\times 0,98=0,686~;$
\item[\textbullet] d'après la formule des probabilités totales, la probabilité que la pièce ait un défaut est~:
\begin{align*}P(D)&=P\left(A\cap D\right)+P\left(\overline{A}\cap D\right)\\
&=0,70\times 0,02+0,30\times 0,03=0,023.\end{align*}
\end{itemize}
\end{enumerate}

\end{exo}

\begin{exo}




\begin{enumerate}

\setlength{\columnseprule}{1pt}

\begin{multicols}{2}
\item On rappelle les zones de tir à 2 et 3 points~:


\begin{center}
\psset{xunit=1.0cm,yunit=1.0cm,algebraic=true,dimen=middle,dotstyle=o,dotsize=5pt 0,linewidth=2.pt,arrowsize=3pt 2,arrowinset=0.25}
\begin{pspicture*}(-0.5006126521330643,-3.458863355936357)(4.527406792181142,3.3532920202312764)
\pspolygon[linewidth=2.pt,linecolor=red,fillcolor=red!30!white,fillstyle=solid,opacity=0.1](0.,3.)(4.,3.)(4.,-3.)(0.,-3.)
\pscustom[linewidth=0.8pt,linecolor=blue,fillcolor=blue!30!white,fillstyle=solid,opacity=0.1]{\psplot{0.}{2.}{-sqrt(4.0-x^(2.0))}\lineto(2.,0.)\psplot{2.}{0.}{sqrt(4.0-x^(2.0))}\lineto(0.,-2.)\closepath}
\parametricplot[linewidth=2.pt,linecolor=blue]{-1.5707963267948966}{1.5707963267948966}{1.*2.*cos(t)+0.*2.*sin(t)+0.|0.*2.*cos(t)+1.*2.*sin(t)+0.}
\psline[linewidth=2.pt](0.,3.)(4.,3.)
\psline[linewidth=2.pt](4.,3.)(4.,-3.)
\psline[linewidth=2.pt](0.,-3.)(4.,-3.)
\psline[linewidth=2.pt,linecolor=red](0.,3.)(4.,3.)
\psline[linewidth=2.pt,linecolor=red](4.,3.)(4.,-3.)
\psline[linewidth=2.pt,linecolor=red](4.,-3.)(0.,-3.)
\psline[linewidth=2.pt,linecolor=red](0.,-3.)(0.,3.)
\psline[linewidth=2.pt](0.,0.)(0.31047517904295696,0.)
\pscircle[linewidth=2.pt](0.31047517904295696,0.){0.1515996751199221}
\rput[tl](0.2382719257052241,-0.8817781210369613){\blue{tir à 2 pts}}
\rput[tl](1.9322999822125193,-2.1252668008135926){\red{tir à 3 pts}}
\end{pspicture*}
\end{center}
\columnbreak

On représente la situation par un arbre pondéré~:

\medskip

\begin{center}
\pstree[treemode=R,treesep=1,levelsep=3]{\TR{}}%
{
\pstree{\Tr{$D$}\taput{$0,53$}}
	{
	\Tr{$M$}\taput{$0,52$} 
	\Tr{$\overline{M}$}\tbput{$0,48$}
		}	
\pstree{\Tr{$\overline{D}$}\tbput{$0,47$}}
	{
	\Tr{$M$}\taput{$0,44$} 
	\Tr{$\overline{M}$}\tbput{$0,56$}
		}
}
\end{center}

\medskip

\textbf{Remarque~:} $\overline{D}$ signifie \og Stephen Curry tire à 3 points \fg.
\end{multicols}
\item La probabilité que Stephen Curry tire à 2 points et marque est
\[P(D\cap M)=0,53\times 0,52=0,2756.\]
\item D'après la formule des probabilités totales, la probabilité que Stephen Curry marque est~:

\begin{align*}P(M)&=P\left(D\cap M\right)+P\left(\overline{D}\cap M\right)\\
&=0,53\times 0,52+0,47\times 0,44=0,4824.\end{align*}
\item On sait que Stephen Curry a 52~\% de réussite lorsqu'il tire à 2 points, donc \[P_D(M)=0,52.\]
Ce nombre est celui qui apparaît sur la branche en haut à droite de l'arbre. Par conséquent, le calcul \[P(D\cap M)=0,53\times 0,52\] peut se réécrire
\[P(D\cap M)=P(D)\times P_D(M).\] \medskip

D'une manière générale, un arbre pondéré est toujours de la forme

\medskip

\begin{center}
\pstree[treemode=R,treesep=1,levelsep=3]{\TR{}}%
{
\pstree{\Tr{$~A~$}\taput{$\textcolor{red}{P(A)}$}}
	{
	\Tr{$~B~$}~[tnpos=r]{~~~~~~\textcolor{red}{$P(A\cap B)=P(A)\times P_A(B)$}}\taput{$\textcolor{red}{P_A(B)}$} 
	\Tr{$~\overline{B}~$}~[tnpos=r]{~~~~~~\textcolor{red}{$~$}}\tbput{$~$}
		}	
\pstree{\Tr{$~\overline{A}~$}\tbput{$~$}}
	{
	\Tr{$~B~$}~[tnpos=r]{~~~~~~\textcolor{red}{$~$}}\taput{$~$} 
	\Tr{$~\overline{B}~$}~[tnpos=r]{~~~~~~\textcolor{red}{$~$}} \tbput{$~$}
		}
}
\end{center}

et l'on a donc toujours $P(A\cap B)=P(A)\times P_A(B).$ On en déduit la relation fondamentale (hors-programme)~:
\[\boxed{P_A(B)=\frac{P(A\cap B)}{P(A)}.}\]

 
\item Pour calculer $P_M(D),$ on utilise la formule encadrée ci-dessus, avec $A=M$ et $B=D~:$\footnote{On a déjà calculé $P(M\cap D)$ et $P(M)$ dans les questions précédentes.}
\[P_M(D)=\frac{P(M\cap D)}{P(M)}=\frac{0,2756}{0,4824}\approx 0,57.\]

Conclusion~: sachant qu'il a marqué, il y a environ 57~\% de chances que Stephen Curry ait tiré à 2 points.

\end{enumerate}

\end{exo}




\begin{exo}



\begin{enumerate}
\item On représente la situation par un arbre pondéré~:


\begin{center}
\pstree[treemode=R,treesep=1,levelsep=3]{\TR{}}%
{
\pstree{\Tr{$D$}\taput{$0,08$}}
	{
	\Tr{$T$}\taput{$0,98$} 
	\Tr{$\overline{T}$}\tbput{$0,02$}
		}	
\pstree{\Tr{$\overline{D}$}\tbput{$0,92$}}
	{
	\Tr{$T$}\taput{$0,005$} 
	\Tr{$\overline{T}$}\tbput{$0,995$}
		}
}
\end{center}

\item $P(D\cap T)=0,08\times 0,98=0,0784.$
\item D'après la formule des probabilités totales, la probabilité que le test soit positif est~:

\begin{align*}P(T)&=P\left(D\cap T\right)+P\left(\overline{D}\cap T\right)\\
&=0,08\times 0,98+0,92\times 0,005=0,083.\end{align*}
\item On utilise à nouveau la formule encadrée de l'exercice précédent~: sachant qu'un athlète présente un test positif, la probabilité qu'il soit dopé est
\[P_T(D)=\frac{P(T\cap D)}{P(T)}=\frac{0,0784}{0,083}\approx 0,94.\]
\end{enumerate}


\end{exo}

\begin{exo}



\begin{enumerate}
\item On commence par faire un arbre pondéré. Comme un appareil en parfait état de fonctionnement est toujours accepté à l’issue du test, il y a un 1 et un 0 sur les branches en haut à droite. Notons par ailleurs que nous sommes obligés de conserver les fractions, parce que les résultats \og ne tombent pas juste \fg.

\medskip


\begin{center}
\pstree[treemode=R,treesep=1,levelsep=3]{\TR{}}%
{
\pstree{\Tr{$F$}\taput{$\frac{9}{10}$}}
	{
	\Tr{$T$}\taput{$1$} 
	\Tr{$\overline{T}$}\tbput{$0$}
		}	
\pstree{\Tr{$\overline{F}$}\tbput{$\frac{1}{10}$}}
	{
	\Tr{$T$}\taput{$\frac{1}{11}$} 
	\Tr{$\overline{T}$}\tbput{$\frac{10}{11}$}
		}
}
\end{center}


\medskip

On en vient au calcul des probabilités demandé par l'énoncé~:

\begin{itemize}
\item[\textbullet] $P\left(\overline{F}\cap T\right)=\frac{1}{10}\times \frac{1}{11}=\frac{1}{110}~;$
\item[\textbullet] d'après la formule des probabilités totales~:
\[P(T)=P\left(F\cap T\right)+P\left(\overline{F}\cap T\right)=\frac{9}{10}\times 1+\frac{1}{10}\times \frac{1}{11}=\frac{9}{10}+\frac{1}{110}=\frac{99}{110}+\frac{1}{110}=\frac{100}{110}=\frac{10}{11}.
\]
\end{itemize}

\item Sachant qu’un appareil a été accepté à l’issue du test, la probabilité qu’il ne fonctionne pas parfaitement est
\[P_T\left(\overline{F}\right)=\frac{P\left(T\cap\overline{F}\right)}{P(T)}=\frac{\frac{1}{110}}{\frac{10}{11}}=\frac{1}{110}\times \frac{11}{10}=\frac{11}{1100}=\frac{1}{100}.\]
\end{enumerate}

\end{exo}

\begin{exo}



\begin{enumerate}
\item On représente la situation par un arbre pondéré. Les probabilités sont les mêmes sur toutes les branches (0,7/0,3), car chaque jour Justin a 70~\% de chances d'avoir des spams  -- et ce indépendamment de ce qu'il s'est passé l'autre jour.

\medskip


\begin{center}
\pstree[treemode=R,treesep=1,levelsep=3]{\TR{}}%
{
\pstree{\Tr{$L$}\taput{$0,7$}}
	{
	\Tr{$M$}\taput{$0,7$} 
	\Tr{$\overline{M}$}\tbput{$0,3$}
		}	
\pstree{\Tr{$\overline{L}$}\tbput{$0,3$}}
	{
	\Tr{$M$}\taput{$0,7$} 
	\Tr{$\overline{M}$}\tbput{$0,3$}
		}
}
\end{center}

\item 

\begin{itemize}
\item[\textbullet] La probabilité que Justin ait des spams lundi et mardi est
\[P(A)=P(L\cap M)=0,7\times 0,7=0,49.\]
\item[\textbullet] La probabilité que Justin n'ait des spams qu'une seule fois\footnote{Donc il en a le lundi, mais pas le mardi~; ou bien le mardi, mais pas le lundi.} est
\[P(B)=P\left(L\cap\overline{M}\right)+P\left(\overline{L}\cap M\right)=0,7\times 0,3+0,3\times 0,7=0,42.\]

\end{itemize}


\end{enumerate}

\end{exo}

\begin{exo}


\begin{enumerate}
\item On représente la situation par un arbre pondéré, avec de gauche à droite le résultat du 1\up{er} tir, du 2\up{e} et du 3\up{e}. On note $R$ pour \og réussi \fg, $M$ pour \og manqué \fg. Comme dans l'exercice précédent, les tirs sont indépendants et Rémy a toujours 80~\% de chances de toucher le centre de la cible, donc les probabilités sont les mêmes sur toutes les branches (0,8/0,2).

\medskip

\begin{center}
\pstree[treemode=R,treesep=1,levelsep=3]{\TR{}}%
{
\pstree{\Tr{$R$}\taput{$0,8$}}
	{
	\pstree{\Tr{$R$}\taput{$0,8$}}
	{
	\Tr{$R$}~[tnpos=r]{~~\textcolor{red}{3 tirs réussis}}\taput{$0,8$}
	\Tr{$M$}~[tnpos=r]{~~\textcolor{blue}{2 tirs réussis}}\tbput{$0,2$}
		}
	\pstree{\Tr{$M$}\tbput{$0,2$}}
	{
	\Tr{$R$}~[tnpos=r]{~~\textcolor{blue}{2 tirs réussis}}\taput{$0,8$}
	\Tr{$M$}~[tnpos=r]{~~\textcolor{green}{1 tir réussi}}\tbput{$0,2$}
		}
		}	
\pstree{\Tr{$M$}\tbput{$0,2$}}
	{
	\pstree{\Tr{$R$}\taput{$0,8$}}
	{
	\Tr{$R$}~[tnpos=r]{~~\textcolor{blue}{2 tirs réussis}}\taput{$0,8$}
	\Tr{$M$}~[tnpos=r]{~~\textcolor{green}{1 tir réussi}}\tbput{$0,2$}
		}
	\pstree{\Tr{$M$}\tbput{$0,2$}}
	{
	\Tr{$R$}~[tnpos=r]{~~\textcolor{green}{1 tir réussi}}\taput{$0,8$}
	\Tr{$M$}~[tnpos=r]{~~\textcolor{gray}{0 tir réussi}}\tbput{$0,2$}
		}
		}
}
\end{center}

\medskip

On a indiqué le nombre de tirs réussis à l'extrémité droite de l'arbre, avec une couleur différente en fonction du résultat. Par exemple, lorsqu'on suit le chemin tout en haut, Rémy réussit le 1\up{er}, le 2\up{e} et le 3\up{e} tirs, donc on a indiqué \textcolor{red}{3 tirs réussis}.

\item Pour gagner un panda en peluche, Rémy doit toucher 2 fois le centre de la cible. Cela correspond aux trois chemins qui se terminent par \textcolor{blue}{2 tirs réussis}. On a donc
\[P(A)=0,8\times 0,8\times 0,2+0,8\times 0,2\times 0,8+0,2\times 0,8\times 0,8=0,384.\]


\end{enumerate}


\end{exo}



\section{Suites définies par récurrence}







\begin{exo}

L'énoncé demande de \og calculer les premiers termes \fg, ce qui est imprécis. Dans chaque cas, nous irons jusqu'à $u_3.$

\begin{enumerate}
\item $u_0=1$ et \[u_{n+1}=u_n+2\] pour tout entier naturel $n.$

\begin{align*}
u_0&=1\\
u_1&=1+2=3\\
u_2&=3+2=5\\
u_3&=5+2=7.\end{align*}

\medskip

La suite $u$ est arithmétique de raison $r=2.$

\item $v_0=2$ et \[v_{n+1}=3\times v_n\] pour tout entier naturel $n.$

\begin{align*}
v_0&=2\\
v_1&=3\times 2=6\\
v_2&=3\times 6=18\\
v_3&=3\times 18=54.\end{align*}

\medskip

La suite $v$ est géométrique de raison $q=3.$

\item $\begin{cases} w_0&=2\\
w_{n+1}&=5-w_n~\text{ pour tout } n\in\mathbb{N}. \end{cases}$

\begin{align*}
w_0&=2\\
w_1&=5-2=3\\
w_2&=5-3=2\\
w_3&=5-2=3.\end{align*}

\medskip

Cela va continuer ainsi indéfiniment~: 2, 3, 2, 3, 2, 3, etc. On dit que la suite $w$ est périodique.


\item $x_0=3$ et \[x_{n+1}=2\times x_n-1\] pour tout entier naturel $n.$

\begin{align*}
x_0&=3\\
x_1&=2\times 3-1=5\\
x_2&=2\times 5-1=9\\
x_3&=2\times 9-1=17.\end{align*}


\item $y_0=4$ et \[y_{n+1}=10-y_n\] pour tout $n\in\mathbb{N}.$

\begin{align*}
y_0&=4\\
y_1&=10-4=6\\
y_2&=10-6=4\\
y_3&=10-4=6.\end{align*}

\medskip

La suite $y$ est périodique.

\item $z_0=4$ et \[z_{n+1}=1,5\times z_n-2\] pour tout entier naturel $n.$

\begin{align*}
z_0&=4\\
z_1&=1,5\times 4-2=4\\
z_2&=1,5\times 4-2=4\\
z_3&=1,5\times 4-2=4.\end{align*}

On dit que la suite $z$ est constante.
\end{enumerate}
\end{exo}



\begin{exo}


\begin{enumerate}
\item On ajoute 40~\euro{} tous les mois sur un compte en banque.

\medskip

Cela correspond à une suite \fbox{arithmétique de raison $r=40.$}

\medskip

Si l'on note $u_n$ la somme sur le compte après $n$ mois, la relation de récurrence est \[u_{n+1}=u_n+40.\]
\item Une population de bactéries augmente de 20~\% toutes les minutes.

\medskip

Pour augmenter un nombre de 20~\%, il faut le multiplier par 1,20. Cela correspond donc à une suite \fbox{géométrique de raison $q=1,20.$}

\medskip

Si l'on note $v_n$ la population de bactéries après $n$ minutes, la relation de récurrence est \[v_{n+1}=v_n\times 1,20.\]
\item La suite est représentée par le nuage de points~:
\begin{center}
\psset{xunit=1cm,yunit=0.8cm,algebraic=true,dimen=middle,dotstyle=o,dotsize=5pt 0,linewidth=1.6pt,arrowsize=3pt 2,arrowinset=0.25}
\begin{pspicture*}(-0.94,-1.54)(4.7,5.52)
\multips(0,-1)(0,1.0){8}{\psline[linestyle=dashed,linecap=1,dash=1.5pt 1.5pt,linewidth=0.4pt,linecolor=
gray]{c-c}(-0.94,0)(4.7,0)}
\multips(0,0)(1.0,0){6}{\psline[linestyle=dashed,linecap=1,dash=1.5pt 1.5pt,linewidth=0.4pt,linecolor=
gray]{c-c}(0,-1.54)(0,5.52)}
\psaxes[labelFontSize=\scriptstyle,xAxis=true,yAxis=true,Dx=1.,Dy=1.,ticksize=-2pt 0,subticks=2]{->}(0,0)(-0.94,-1.54)(4.7,5.52)
\psline[linewidth=2.pt]{->}(0.,-1.)(1.,-1.)
\psline[linewidth=2.pt,linecolor=xfqqff]{->}(1.,-1.)(1.,1.)
\psline[linewidth=2.pt]{->}(1.,1.)(2.,1.)
\psline[linewidth=2.pt,linecolor=xfqqff]{->}(2.,1.)(2.,3.)
\psline[linewidth=2.pt]{->}(2.,3.)(3.,3.)
\psline[linewidth=2.pt,linecolor=xfqqff]{->}(3.,3.)(3.,5.)
\rput[tl](1.06,-0.34){\xfqqff{$+2$}}
\rput[tl](2.06,2.04){\xfqqff{$+2$}}
\rput[tl](3.06,4.06){\xfqqff{$+2$}}
\psdots[dotstyle=*,linecolor=red](0.,-1.)
\psdots[dotstyle=*,linecolor=red](1.,1.)
\psdots[dotstyle=*,linecolor=red](2.,3.)
\psdots[dotstyle=*,linecolor=red](3.,5.)
\end{pspicture*}
\end{center}

\medskip

Quand on avance de 1 en abscisse, on monte de 2 en ordonnée, donc cela correspond à une suite \fbox{arithmétique de raison $r=2.$}

\medskip

Si l'on note $u$ la suite des termes, la relation de récurrence est \[u_{n+1}=u_n+2.\]
\item Pour soigner son cancer de la thyroïde, un patient doit ingérer une petite quantité d'iode 131, dont la masse diminue ensuite de 8~\% par jour.

\medskip

Pour diminuer un nombre de 8~\%, il faut le multiplier par 0,92 (car $100~\%-8~\%=92~\%=0,92$). Cela correspond donc à une suite \fbox{géométrique de raison $q=0,92.$}

\medskip

Si l'on note $v_n$ la masse d'iode après $n$ minutes, la relation de récurrence est \[v_{n+1}=v_n\times 0,92.\]
\end{enumerate}
\end{exo}

\begin{exo}



\begin{enumerate}
\item Pour diminuer un nombre de 10~\%, il faut le multiplier par $0,90.$ Sachant cela, il est agréable de présenter la solution avec le schéma~:

\setlength{\columnseprule}{1pt}

\begin{multicols}{2}
~{}\begin{center}
    $\xymatrix@R=0.5pc@C=3pc{
    *+[F]+{270} \ar@/^0.5cm/[r]|{\red{\times 0,90}} & 
    *+[F]+{243} \ar@/^0.5cm/[r]|{\red{\times 0,90}} & *+[F]+{219} \\
    \txt{\blue{$p_0$}}&
    \txt{\blue{$p_1$}}&\txt{\blue{$p_2$}}\\
    \txt{\blue{pies en 2001}}&
    \txt{\blue{pies en 2002}}&\txt{\blue{pies en 2003}}
    }$
    \end{center}
    
  \columnbreak
  
  Calculs utiles~:
  \begin{align*}
  270\times 0,90&=243\\
  243\times 0,90&\approx 219
  \end{align*}
  
  \end{multicols}
    
    
\item La suite $p$ est géométrique de raison $q=0,90.$
\item La relation de récurrence est \[p_{n+1}=p_n\times 0,90.\]
\item La formule à entrer dans la cellule C2 est \[\text{=B2*0,90}\]
\item L'année 2021 est l'année n°20. On étire donc vers la droite jusqu'à la colonne V~:

\begin{center}
\begin{tabular}{|c|c|c|c|c|c|c|c|}
\hline
	& A	&B		&C		&D 		&$\cdots$&U	&V	\\\hline
1&$n$&0&1&2&$\cdots$&19&20\\\hline
2	& $p_n$&$270$&243&219&$\cdots$&36&33\\\hline
\end{tabular}
\end{center}


\medskip

Conclusion~: suivant ce modèle, on prévoit 33 pies en 2021.

\medskip

\textbf{Remarque~:} On peut s'en sortir sans le tableur, avec le calcul
\[270\underbrace{\times 0,90\times 0,90\times \cdots\times 0,90}_{\text{20 fois}}=270\times 0,90^{20}\approx 33.\]
\end{enumerate}
\end{exo}


\begin{exo}



\begin{enumerate}
\item Pour davantage de clarté, on laissera apparentes les unités de volume (mL).

\begin{itemize}
\item[\textbullet] $w_0$ est la quantité initiale de médicament~: \[w_0=10~\text{mL}.\] 
\item[\textbullet] \`A la fin de la 1\up{re} minute, on a perdu 20~\% du médicament (multiplication par $0,80$), il en reste donc
\[10\times 0,80=8~\text{mL}.\] On en injecte alors 1~mL, on obtient ainsi
\[w_1=8+1=9~\text{mL}.\] 
\item[\textbullet] \`A la fin de la 2\up{e} minute, on a de nouveau perdu 20~\% du médicament, donc il en reste
\[9\times 0,80=7,2~\text{mL}.\] On en injecte alors 1~mL~; il vient donc 
\[w_2=7,2+1=8,2~\text{mL}.\] 
\item[\textbullet] Cela continue ainsi de suite. Pour calculer $w_3,$ on peut faire les deux opérations en une seule étape~:
\[w_3=7,2\times 0,80+1=6,76~\text{mL}.\]
\end{itemize}

\medskip

Comme d'habitude, il est agréable d'utiliser un schéma~:

\medskip

\scriptsize
\begin{center}
    $\xymatrix@R=0.5pc@C=3pc{
    *+[F]+{10} \ar@/^0.5cm/[r]|{\red{\times 0,80}} & 
    *+[F]+{8} \ar@/^0.5cm/[r]|{\red{+1}} & *+[F]+{9}\ar@/^0.5cm/[r]|{\red{\times 0,80}} & *+[F]+{7,2}\ar@/^0.5cm/[r]|{\red{+1}} & *+[F]+{8,2}\ar@/^0.5cm/[r]|{\red{\times 0,80}} & *+[F]+{5,76}\ar@/^0.5cm/[r]|{\red{+1}} & *+[F]+{6,76} \\
    \txt{\blue{$w_0$}}&\txt{\blue{$~$}}&
    \txt{\blue{$w_1$}}&\txt{\blue{$~$}}&\txt{\blue{$w_2$}}&\txt{\blue{$~$}}&\txt{\blue{$w_3$}}\\
    \txt{\blue{au départ}}&
    \txt{\blue{fin 1\up{re} min}}&\txt{\blue{après 1 min}}&
    \txt{\blue{fin 2\up{e} min}}&\txt{\blue{après 2 min}}&
    \txt{\blue{fin 3\up{e} min}}&\txt{\blue{après 3 min}}
    }$
    \end{center}
\normalsize
\item La relation de récurrence est
\[w_{n+1}=w_n\times 0,80+1.\]

\item On entre les formules \[\text{=B1+1}\] et \[\text{=B2*0,80+1}\] dans les cellules C1 et C2, puis on étire vers la droite~:

\medskip

\begin{center}
\begin{tabularx}{\linewidth}{|c|*{5}{>{\centering \arraybackslash}X|}}\hline
	&A   	&B		&C			&$\cdots$   	\\ \hline   
1 & Minutes &$0$  	&=B1+1				&$\cdots$ \\ \hline   
2 & Volume (en mL) &$10$   		&=B2*0,80+1		&$\cdots$  \\ \hline    
\end{tabularx}
\end{center}

\medskip

On constate que le volume se stabilise à 5~mL sur le long terme (il est toujours légèrement supérieur à 5, mais semble s'en rapprocher). Voici ce que l'on obtient si l'on affiche trois chiffres après la virgule~:

\medskip

\begin{center}
\begin{tabularx}{\linewidth}{|c|*{8}{>{\centering \arraybackslash}X|}}\hline
	&A&$\cdots$ &AF  &$\cdots$ &AP&$\cdots$ &AZ	\\ \hline   
1&Minutes&$\cdots$ &$30$  &$\cdots$ &$40$&$\cdots$ &$50$\\ \hline   
2&Volume (en mL)&$\cdots$ &$5,006$  &$\cdots$ &$5,001$&$\cdots$ &$5,000$  \\ \hline    
\end{tabularx}
\end{center}



\end{enumerate}
\end{exo}


\begin{exo}

Une suite $v$ est définie par $v_0=4$ et la relation de récurrence \[v_{n+1}=2v_n+2\] pour tout entier naturel $n.$

\begin{enumerate}
\item \begin{align*}
v_0&=4\\
v_1&=2\times 4+2=10\\
v_2&=2\times 10+2=22.\end{align*}


\item Avec un schéma~:

\setlength{\columnseprule}{1pt}

\begin{multicols}{2}
~{}\begin{center}
    $\xymatrix@R=0.5pc@C=3pc{
    *+[F]+{4} \ar@/^0.5cm/[r]|{\red{+6}} \ar@/_0.5cm/[r]|{\green{\times 2,5}} & 
    *+[F]+{10} \ar@/^0.5cm/[r]|{\red{+12}} \ar@/_0.5cm/[r]|{\green{\times 2,2}} & *+[F]+{22} \\
    \txt{\blue{$v_0$}}&
    \txt{\blue{$v_1$}}&\txt{\blue{$v_2$}}    
    }$
    \end{center}
    
    
    \medskip
    
    Les résultats en rouge (\textcolor{red}{6} et \textcolor{red}{12}) sont différents, donc $u$ \textbf{n'est pas arithmétique.}
    
    \medskip
    
    Les résultats en vert (\textcolor{green}{2,5} et \textcolor{green}{2,2}) sont différents, donc $u$ \textbf{n'est pas géométrique.}
    
  \columnbreak
  
  Calculs utiles~:
  
  \medskip
  
  \begin{align*}
  10-4&=\textcolor{red}{6},\\
  22-10&= \textcolor{red}{12}.
  \end{align*}
  
  \medskip
  
  \begin{align*}
  10\div 4&=\textcolor{green}{2,5},\\
  22\div 10&= \textcolor{green}{2,2}.
  \end{align*}
  
  \end{multicols}


\end{enumerate}
\end{exo}

\begin{exo}

Une suite $u$ est définie par $u_0=2$ et la relation de récurrence \[u_{n+1}=3u_n-1\] pour tout entier naturel $n.$

\begin{enumerate}
\item \begin{align*}
u_0&=2\\
u_1&=3\times 2-1=5\\
u_2&=3\times 5-1=14.\end{align*}


\item Avec un schéma~:

\setlength{\columnseprule}{1pt}

\begin{multicols}{2}
~{}\begin{center}
    $\xymatrix@R=0.5pc@C=3pc{
    *+[F]+{2} \ar@/^0.5cm/[r]|{\red{+3}} \ar@/_0.5cm/[r]|{\green{\times 2,5}} & 
    *+[F]+{5} \ar@/^0.5cm/[r]|{\red{+9}} \ar@/_0.5cm/[r]|{\green{\times 2,8}} & *+[F]+{14} \\
    \txt{\blue{$u_0$}}&
    \txt{\blue{$u_1$}}&\txt{\blue{$u_2$}}    
    }$
    \end{center}
    
    
    \medskip
    
    Les résultats en rouge (\textcolor{red}{3} et \textcolor{red}{9}) sont différents, donc $u$ \textbf{n'est pas arithmétique.}
    
    \medskip
    
    Les résultats en vert (\textcolor{green}{2,5} et \textcolor{green}{2,8}) sont différents, donc $u$ \textbf{n'est pas géométrique.}
    
  \columnbreak
  
  Calculs utiles~:
  
  \medskip
  
  \begin{align*}
  5-2&=\textcolor{red}{3},\\
  14-5&= \textcolor{red}{9}.
  \end{align*}
  
  \medskip
  
  \begin{align*}
  5\div 2&=\textcolor{green}{2,5},\\
  14\div 5&= \textcolor{green}{2,8}.
  \end{align*}
  
  \end{multicols}


\end{enumerate}
\end{exo}



\begin{exo}


Le 01/01/2019, on emprunte \np{10000}~\euro~{} à la banque au taux d’intérêt mensuel de 2~\%. A chaque fin de mois on rembourse 300~\euro. On voudrait savoir en combien de temps on remboursera le crédit et calculer la somme totale remboursée à la banque.
\medskip

Comment ça marche ?...

Le 01/01/2019 on emprunte \np{10000}~\euro~au taux d’intérêt de 2~\%, donc à la fin du mois de janvier 2019 la somme à rembourser est passée à
\[\np{10000}\times 1,02=\np{10200}~\text{\euro}.\]
A ce moment on rembourse 300~\euro, donc le 01/02/2019 il reste à rembourser
\[\np{10200}-300=\np{9900}~\text{\euro}.\]
On note $u_n$  la somme restant à rembourser le 1\up{er} jour du  $n$\up{e} mois (en convenant que janvier 2019 est le mois 0, février 2019 le mois 1, etc.).
On a donc $u_0=\np{10000}$ et $u_1=\np{9900}.$

\begin{enumerate}
\item On complète le schéma ci-dessous pour calculer les termes $u_1$ et $u_2.$ Les sommes écrites dans chaque case sont les sommes restant à rembourser aux dates indiquées.

\medskip

\begin{center}
    $\xymatrix@R=0.5pc@C=3pc{
    *+[F]+{\np{10000}} \ar@/^0.5cm/[r]|{\red{\times 1,02}} & 
    *+[F]+{\np{10200}} \ar@/^0.5cm/[r]|{\red{-~300}} & *+[F]+{\np{9900}}\ar@/^0.5cm/[r]|{\red{\times 1,02}} & *+[F]+{\np{10098}}\ar@/^0.5cm/[r]|{\red{-~300}} & *+[F]+{\np{9798}} \\
    \txt{\blue{$u_0$}}&\txt{\blue{$~$}}&
    \txt{\blue{$u_1$}}&\txt{\blue{$~$}}&\txt{\blue{$u_2$}}\\
    \txt{\blue{01/01/19}}&
    \txt{\blue{31/01/19}}&\txt{\blue{01/02/19}}&
    \txt{\blue{28/02/19}}&\txt{\blue{01/03/19}}
    }$
    \end{center}

Pour passer d'un terme de la suite au terme suivant, on multiplie par $1,02$ (ajout des intérêts) puis on retranche 300 (remboursement mensuel). On peut donc continuer plus rapidement~:

\begin{align*}
u_3&=\np{9798}\times 1,02-300=\np{9693,96}&&\text{(somme à rembourser le 01/04/19)},\\
u_4&=\np{9693,96}\times 1,02-300=\np{9587,84}&&\text{(somme à rembourser le 01/05/19)}
.\end{align*}


\item La formule de récurrence est \[u_{n+1}=u_n\times 1,02-300.\]
\item On entre les formules \[\text{=B1+1}\] et \[\text{=B2*1,02-300}\] dans les cellules C1 et C2, puis on étire vers la droite~:

\medskip

\begin{center}
\begin{tabularx}{\linewidth}{|c|*{5}{>{\centering \arraybackslash}X|}}\hline
	&A   	&B		&C			&$\cdots$   	\\ \hline   
1 & Nombre de mois &$0$  	&=B1+1				&$\cdots$ \\ \hline   
2 & Reste à rembourser &$10000$   		&=B2*1,02-300		&$\cdots$  \\ \hline    
\end{tabularx}
\end{center}

\medskip

On continue jusqu'à ce que la somme à rembourser soit nulle. En réalité, au bout d'un moment, elle est négative~:

\medskip

\begin{center}
\begin{tabularx}{\linewidth}{|c|*{5}{>{\centering \arraybackslash}X|}}\hline
	&A&$\cdots$ &BD&BE &BF	\\ \hline   
1& Nombre de mois&$\cdots$ &54&55&56\\ \hline   
2& Reste à rembourser&$\cdots$ &$432,69$&$141,35$& $-155,83$ \\ \hline    
\end{tabularx}
\end{center}

\medskip

\`A la fin du 55\up{e} fois, il reste 141,35~\euro~{} à rembourser~; et si on rembourse 300~\euro~{} au début du 56\up{e} mois, la banque nous devra 155,83~\euro.


Conclusion~:

\begin{itemize}
\item[\textbullet] le crédit dure 56 mois~;
\item[\textbullet] on rembourse 56 fois 300~\euro, mais à la fin on a dépassé de 155,83~\euro~{} ce que l'on devait à la banque~;
\item[\textbullet] la somme totale remboursée est donc
\[56\times 300-155,83=\np{16664,17}~\text{\euro}~;\]
\item[\textbullet] le \og coût du crédit \fg~{} est la différence entre ce que l'on a remboursé et ce que la banque nous a prêté~:
\[\text{Coût du crédit}=\text{Somme remboursée}-\text{Somme empruntée}=
\np{16664,17}-\np{10000}=\np{6664,17}~\text{\euro}.\]
\end{itemize}

\end{enumerate}

\end{exo}


\begin{exo}



\begin{enumerate}
\item \begin{enumerate}
\item Chaque année, le nombre de poissons diminue de 5~\% (multiplication par 0,95), donc

\begin{align*}
v_0&=\np{12840}\\
v_1&=\np{12840}\times 0,95=\np{12198}\\
v_2&=\np{12198}\times 0,95\approx\np{11588}.
\end{align*} La suite $v$ est géométrique de raison $q=0,95.$
\item En 2025 (donc au bout de 7 ans), l’aquarium contiendra
\[\np{12840}\underbrace{\times 0,95\times 0,95\times \cdots\times 0,95}_{\text{7 fois}}=\np{12840}\times 0,95^{7}\approx 8967~\text{poissons}.\] Il ne sera donc plus attractif (moins de \np{9000} poissons).
\end{enumerate}
\item 
\begin{enumerate}
\item Chaque année, le nombre de poissons diminue de 5~\%, puis 400 nouveaux poissons sont ajoutés pour compenser. On utilise un schéma pour calculer le nombre de poissons début 2018 et début 2019.


\medskip

\scriptsize
\begin{center}
    $\xymatrix@R=0.5pc@C=3pc{
    *+[F]+{\np{12840}} \ar@/^0.5cm/[r]|{\red{\times 0,95}} & 
    *+[F]+{\np{12198}} \ar@/^0.5cm/[r]|{\red{+400}} & *+[F]+{\np{12598}}\ar@/^0.5cm/[r]|{\red{\times 0,95}} & *+[F]+{\np{11968}}\ar@/^0.5cm/[r]|{\red{+400}} & *+[F]+{\np{12368}} \\
    \txt{\blue{$u_0$}}&\txt{\blue{$~$}}&
    \txt{\blue{$u_1$}}&\txt{\blue{$~$}}&\txt{\blue{$u_2$}}\\
    \txt{\blue{début 2017}}&
    \txt{\blue{fin 2017}}&\txt{\blue{début 2018}}&
    \txt{\blue{fin 2018}}&\txt{\blue{début 2019}}
    }$
    \end{center}
\normalsize

\medskip

Conclusion~: $u_1=\np{12598}$ et  $u_2=\np{12368}.$
\item La formule de récurrence pour la suite $u$ est \[u_{n+1}=u_n\times 0,95+400.\]
\item On entre les formules \[\text{=B1+1}\] et \[\text{=B2*0,95+400}\] dans les cellules C1 et C2, puis on étire vers la droite~:

\medskip

\begin{center}
\begin{tabularx}{\linewidth}{|c|*{5}{>{\centering \arraybackslash}X|}}\hline
	&A   	&B		&C			&$\cdots$   	\\ \hline   
1 & Années &$2018$  	&=B1+1				&$\cdots$ \\ \hline   
2 & Nombre de poissons &$12840$   		&=B2*0,95+400		&$\cdots$  \\ \hline    
\end{tabularx}
\end{center}

\medskip

On obtient~:

\medskip

\begin{center}
\begin{tabularx}{\linewidth}{|c|*{6}{>{\centering \arraybackslash}X|}}\hline
	&A   	&B		&C			&$\cdots$   	&I\\ \hline   
1 & Années &$2018$  	&2019				&$\cdots$&2025 \\ \hline   
2 & Nombre de poissons &$12840$   		&$12198$		&$\cdots$  &11380\\ \hline    
\end{tabularx}
\end{center}

\medskip


Conclusion~: il y aura \np{11380} poissons dans l'aquarium en 2025.
\end{enumerate}
\end{enumerate}

\end{exo}


\section{Dérivation et variations des fonctions du 3\up{e} degré}






\begin{exo}

On considère la fonction $f$ définie sur $\left[-2;3\right]$ par \[f(x)=x^3-1,5x^2-6x+2.\]
 
\begin{enumerate}
\item \[f'(x)=3x^2-1,5\times 2x-6\times 1+0=3x^2-3x-6.\] Pour prouver que
\[f'(x)=(3x+3)(x-2),\] on développe et on réduit le membre de droite~:

\begin{align*}
(3x+3)(x-2)&=3x\times x+3x\times (-2)+3\times x+3\times (-2)\\
&=3x^2-6x+3x-6\\
&=3x^2-3x-6.\end{align*} On retombe bien sur l'expression de $f'(x)$ obtenue ci-dessus.
\item On étudie le signe de $f'(x)=(3x+3)(x-2).$

\medskip

{\setlength{\arrayrulewidth}{2pt}
\begin{center}
\begin{tabular}{l|l}
$3x+3=0$&$x-2=0$\\
$3x+\cancel{3}-\cancel{3}=0-3$&$x-\cancel{2}+\cancel{2}=0+2$\\
$\frac{\cancel{3}x}{\cancel{3}}=\frac{-3}{3}$&$x=2$\\
$x=-1$& \\
$a=3$ donc \fbox{$-~\upphi~+$}&$a=1$ donc \fbox{$-~\upphi~+$}
\end{tabular}
\end{center}}

\medskip


\begin{center}
\begin{tikzpicture}[scale=1]
\tkzTabInit{$x$/1,$3x+3$/1,$x-2$/1,$f'(x)$/1,$f(x)$/2}{$-2$,$-1$,$2$,$3$}
\tkzTabLine{,-,z,+,,+,}
\tkzTabLine{,-,,-,z,+,}
\tkzTabLine{,+,z,-,z,+,}
\tkzTabVar{-/$0$,+/$5.5$,-/$-8$,+/$-2.5$}
\end{tikzpicture}
\end{center}

\medskip

\textbf{Remarque~:} Pour compléter l'extrémité des flèches, on demande un tableau de valeurs pour $f$ sur $\left[-2;3\right]$ avec un pas de 1 et on ne prend en compte que les valeurs $x=-2,$ $x=-1,$ $x=2$ et $x=3.$

\end{enumerate}

\end{exo}



\begin{exo}

On considère la fonction $g$ définie sur $\left[0;5\right]$ par \[g(x)=-0,5x^3+3,75x^2-6x+1.\]

\begin{enumerate}
\item \[g'(x)=-0,5\times 3x^2+3,75\times 2x-6\times 1+0=-1,5x^2+7,5x-6.\] Pour prouver que
\[g'(x)=(-3x+3)(0,5x-2),\] on développe et on réduit le membre de droite~:

\begin{align*}
(-3x+3)(0,5x-2)&=(-3x)\times 0,5x+(-3x)\times (-2)+3\times 0,5 x+3\times (-2)\\
&=-1,5x^2+6x+1,5x-6\\
&=-1,5x^2+7,5x-6.\end{align*} On retombe bien sur l'expression de $g'(x)$ obtenue ci-dessus.
\item On étudie le signe de $g'(x)=(-3x+3)(0,5x-2).$

\medskip

{\setlength{\arrayrulewidth}{2pt}
\begin{center}
\begin{tabular}{l|l}
$-3x+3=0$&$0,5x-2=0$\\
$-3x+\cancel{3}-\cancel{3}=0-3$&$0,5x-\cancel{2}+\cancel{2}=0+2$\\
$\frac{\cancel{-3}x}{\cancel{-3}}=\frac{-3}{-3}$&$\frac{\cancel{0,5}x}{\cancel{0,5}}=\frac{2}{0,5}$\\
$x=1$&$x=4$ \\
$a=-3$ donc \fbox{$+~\upphi~-$}&$a=0,5$ donc \fbox{$-~\upphi~+$}
\end{tabular}
\end{center}}

\medskip


\begin{center}
\begin{tikzpicture}[scale=1]
\tkzTabInit{$x$/1,$-3x+3$/1,$0.5x-2$/1,$g'(x)$/1,$g(x)$/2}{$0$,$1$,$4$,$5$}
\tkzTabLine{,+,z,-,,-,}
\tkzTabLine{,-,,-,z,+,}
\tkzTabLine{,-,z,+,z,-,}
\tkzTabVar{+/$1$,-/$-1.75$,+/$5$,-/$2.25$}
\end{tikzpicture}
\end{center}



\end{enumerate}

\end{exo}

\begin{exo}


On considère la fonction $f$ définie sur $\left[5;20\right]$ par \[f(x)=x^3-24x^2+180x+250.\]

\begin{enumerate}
\item \[f'(x)= 3x^2-24\times 2x+180\times 1+0=3x^2-48x+180.\] Pour prouver que
\[f'(x)=(x-10)(3x-18),\] on développe et on réduit le membre de droite~:

\begin{align*}
(x-10)(3x-18)&=x\times 3x+x\times (-18)+(-10)\times 3x+(-10)\times (-18)\\
&=3x^2-18x-30x+180\\
&=3x^2-48x+180.\end{align*} On retombe bien sur l'expression de $f'(x)$ obtenue ci-dessus.
\item On étudie le signe de $f'(x)=(x-10)(3x-18).$

\medskip

{\setlength{\arrayrulewidth}{2pt}
\begin{center}
\begin{tabular}{l|l}
$x-10=0$&$3x-18=0$\\
$x-\cancel{10}+\cancel{10}=0+10$&$3x-\cancel{18}+\cancel{18}=0+18$\\
$x=10$&$\frac{\cancel{3}x}{\cancel{3}}=\frac{18}{3}$\\
&$x=6$ \\
$a=1$ donc \fbox{$-~\upphi~+$}&$a=3$ donc \fbox{$-~\upphi~+$}
\end{tabular}
\end{center}}

\medskip


\begin{center}
\begin{tikzpicture}[scale=1]
\tkzTabInit{$x$/1,$x-10$/1,$3x-18$/1,$f'(x)$/1,$f(x)$/2}{$5$,$6$,$10$,$20$}
\tkzTabLine{,-,,-,z,+,}
\tkzTabLine{,-,z,+,,+,}
\tkzTabLine{,+,z,-,z,+,}
\tkzTabVar{-/$675$,+/$682$,-/$650$,+/$\np{2250}$}
\end{tikzpicture}
\end{center}

%\medskip

%\textbf{Remarque~:} Comme le tableau \og démarre à $x=-1$ \fg , il n'y a pas de \og $-$ \fg~{} sur la ligne de $x+1.$
\item On voit dans le tableau de variations que la fonction $f$ atteint son minimum pour $x=10,$ donc il faut fabriquer $\np{10000}$ emballages pour minimiser le coût -- le coût minimal est alors de 650~\euro.
\end{enumerate}

\end{exo}







\end{document}